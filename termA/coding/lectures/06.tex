На прошлом занятии рассмотрели циклические коды.
Это подвид линейных, которые подвид нелинейных.
Мы так пытаемся сильно уменьшить сложность кодирования и декодирования.

Мы уже доказали единственность порождающего многочлена, построили
проверочную матрицу, проверочный многочлен.

Займёмся кодированием.

Взяли порождающий многочлен $g(x)$ степени $r$.
Говорим, что кодовое слово $C(x)=q(x) \cdot g(x)$.
Тогда у $q(x)$ должна быть степень строго меньше $k$.
Это метод \textit{несистематического кодирования}.

Второй способ, \textit{систематическое кодирование}.
Всего $n$ бит, сначала записали $r$ проверочных бит (сейчас придумаем, как построить),
а потом $k$ информационных:
\[
C(x) = V(X) + x^r \cdot a(x)
\]
\begin{Rem}
Тут важно договориться, в каком порядке биты идут (с технической точки зрения).
\end{Rem}
\begin{Rem}
Философский вопрос, что является результатом декордирования:
кодовое слово и информационные символы.
\end{Rem}
При этом у нас и в систематическом, и в несистематическом кодировании
одинаково сложное восстановление кодовых слов.

Давайте закодируем систематически (несистематически уже умеем, формула есть).
Нам надо взять $x^r \cdot a(x)$ и дописать ему что-то, чтобы получить кодовое слово
(т.е. делящееся на $g(x))$.
Попробуем, скажем, поделить с остатком на $g(x)$:
\begin{gather*}
	% Степени справа: n, k-1, r-1
	x^r \cdot a(x) &= g(x) \cdot b(x) + s(x) \\
	x^r \cdot a(x) - s(x) &= g(x) \cdot b(x) \\
\end{gather*}
Вот и проверочные символы: $v(x)=-s(x)$.

Декодирование.

Синдромы мы уже знаем, определим \textit{синдромный многочлен}.
Пусть $b(x)=c(x)+e(x)$ (принятый равен кодовому плюс ошибке).
Дальше поделили $c(x)$ и $e(x)$ на $g(x)$ с остатком, в первом случае остаток ноль,
во втором как раз синдром.

Теорема: каждому вектору ошибки взаимнооднозначно соответствует синдром.
\begin{theorem}
	Если есть две ошибки $w(e_1)\le t$ и $w(e_2) \le t$,
	то у них разные синдромы.
\end{theorem}
\begin{proof}
	Поделили, получили $g(x)(b_1(x)-b_2(x))=e_1(x)-e_2(x)$, т.е.
	разность векторов ошибки "--- кодовый вектор.
	Но мы знаем, что $w(e_1-e_2)\le 2t$ (потому что ошибки небольшие),
	но $2t < d$, следовательно противоречие (т.к. кодовый вектор веса
	меньше $d$, т.е. нулевой).
\end{proof}

\begin{Rem}
Теперь нам, конечно, всё ещё нужна табличка, но уже в $n$ раз меньше.
Потому что у нас коды циклические.
Можно для каждого синдрома запомнить не $n$ циклических сдвигов,
а поменьше.
Времени на декодирование, конечно, станет в $n$ раз больше
(попробовать все сдвиги).
\end{Rem}

Что-то про поля.

Типа вводим поле $GF(q^m)$.

Возьмём $GF(q)$.
Пусть $(n,q)=1$.
Тогда можно определить мультипликативный порядок $q$ по модулю $n$:
минимальное $m$ такое, что $q^m \equiv 1 \mod n$.
То есть $n$ делит $q^m-1$.

И можно в любом поле написать, что $x^n-1$ делит $x^{q^m-1}-1$
(это отдельная теорема: если натуральное число $a$
делит $b$, то $x^a-1$ делит $x^b-1$ в произвольном поле).

Теорема: если многочлен $f(x)$ взаимно прост со своей производной
$f'(x)$, то у них нет общих корней.
Это почти во всех полях верно, за одним исключением.

Тогда применим к многочлену $x^n-1$.
Он в любом поле не делится на $nx^{n-1}$.
Таким образом, у него нет нулевых корней.

% Выключили запись
Дальше вспомнили понятие "алгебраические коды" (такого понятия нет, оно неточное и рукомахательное).


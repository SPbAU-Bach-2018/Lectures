\begin{Def}
	\textit{Порождающий многочлен} $g(x)$ циклического кода $\cal C$ "--- это
	ненулевой приведённый (со старшим коэффициентом 1) многочлен минимальной степени,
	который является кодовым словом.
\end{Def}
\begin{Rem}
	Будем обозначать $r \coloneqq \deg g(x)$, а $k \coloneqq n - r$.
\end{Rem}
\begin{lemma}
	Порождающий многочлен существует.
\end{lemma}
\begin{proof}
	Для начала просто взяли многочлен минимальной степени.
	Если у него есть старший коэффициент $a$,
	домножим многочлен на $a^{-1}$, получим приведённый многочлен.
\end{proof}
\begin{lemma}
	Порождающий многочлен единственен.
\end{lemma}
\begin{proof}
	Пусть есть $g_1(x)$ и $g_2(x)$.
	Тогда их степени равны, иначе другой точно не является порождающим.
	Также их старшая степень равна единице.
	Значит, $\deg (g_1(x)-g_2(x)) < \deg g_1(x)$.
	А тогда многочлен $g_1(x)-g_2(x)$ должен быть либо нулевым,
	иначе мы бы выбрали его как порождающий.
\end{proof}

\begin{theorem}
	Для любого многочлена $a(x)$ многочлен $a(x) \cdot g(x)$ (всё по модулю)
	является кодовым словом.
\end{theorem}
\begin{proof}
	$g(x)$ "--- кодовое слово $\Rightarrow$ $x^k\cdot g(x)$ "--- тоже кодовые слова (код циклический).
	Значит, их линейная комбинация тоже является кодовым словом.
	А $a(x) \cdot g(x)$ "--- это как раз такая линейная комбинация.
\end{proof}

\begin{theorem}\label{cycle_code_all_words}
	Других кодовых слов не бывает, они все представимы в виде:
	\[
		\cal G = \{ a(x) \cdot g(x) \mid a(x) \le n - r \}
	\]
\end{theorem}
\begin{proof}
	Рассмотрим кодовое слово $c(x)$, поделим с остатком на $g(x)$ (просто как многочлены, не по модулю):
	\[
		c(x) = g(x) \cdot b(x) + s(x), \deg s(x) < \deg g(x) = r
	\]
	Теперь переносим:
	\[
		c(x) - \underbrace{g(x) \cdot b(x)}_\text{кодовое слово} = s(x)
	\]
	Получили кодовое слово степени меньше $r$, т.е. оно должно быть нулём.
\end{proof}

\begin{theorem}
	Порождающий многочлен делит $x^n-1$:
	\[
	g(x) \mid x^n - 1
	\]
\end{theorem}
\begin{proof}
	Разделили с остатком:
	\[
		x^n - 1 = g(x) \cdot b(x) + s(x), \deg s(x) < \deg g(x) = r
	\]
	Переносим:
	\[
		s(x) = (x^n - 1) - \underbrace{g(x) \cdot b(x)}_\text{кодовое слово}
	\]
	Теперь посмотрим на это по модулю $x^n - 1$, тогда $g(x) \cdot b(x) \bmod (x^n - 1)$
	есть кодовое слово, то есть $s(x)$ сравнимо с кодовым словом по модулю $x^n - 1$.
	Но степень $s(x)$ меньше $n$, то есть $s(x)$ является кодовым словом степени меньше $r$.
	Но тогда это нулевое кодовое слово, что и требовалось.
\end{proof}

Теперь определим порождающую матрицу:
\[
G=\begin{pmatrix}
g(x) \\
x \cdot g(x) \\
x^2 \cdot g(x) \\
\vdots \\
x^{k-1} \cdot g(x)
\end{pmatrix}
\]
Это просто порождающая матрица нашего кода.
Тут степени иксов идут только до $k-1$, потому что мы в теореме \label{cycle_code_all_words} говорили,
что у нас весь код вида $g(x) \cdot a(x)$, где $\deg a(x) < k$.

Теперь введём проверочный многочлен:
\[
	h(x) = \frac{x^n-1}{g(x)}
\]
\begin{theorem}
	$c(x)$ "--- кодовое слово $\iff$ $h(x) \cdot c(x) \equiv 0 \bmod x^n - 1$
\end{theorem}
\begin{proof}
	$\Rightarrow$: очевидно, подставили.

	$\Leftarrow$: очевидно (сказали, что такое <<сравнимо с нулём по модулю>>).
\end{proof}

Можно ещё сделать проверочную матрицу из коэффициентов: \TODO

\begin{Exercise}
	Пусть $n=1$, тогда мы знаем, что $g(x) \mid x^n-1$.
	Разложим $x^n-1$ на множители в поле $GF(2)$:
	\[
		x^7 = (x-1)(x^3+x+1)(x^3+x^2+1)
	\]
	Теперь у нас можно выбрать какой-нибудь $g(x)$, являющийся произведением какого-то подмножества скобок.
	Например, $g(x) = x^3+x+1$, $r=3$.
	Так мы получили $(7, 4, 3)$-код: $n=7$, $k=n-r=7-3$, а минимальное расстояние 3 до нуля мы
	посчитали, честно построив матрицу $G$: \TODO

	Из многочлена мы так сразу $d$ выяснить не можем.
\end{Exercise}

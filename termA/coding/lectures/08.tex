\begin{lemma}
	Числовое тождество в натуральных числах:
	\[
	a^l - 1 \mid a^n - 1 \iff l \mid n
	\]
\end{lemma}
Доказывается для чисел методом <<поделить $n$ на $l$ с остатком>>.
А потом получим то же самое для многочленов:
\begin{lemma}
	\[
	x^l - 1 \mid x^n - 1 \iff l \mid n
	\]
\end{lemma}
\begin{Rem}
	Это чем-то похоже на теорему Безу с подстановкой $y=x^l$
\end{Rem}

\begin{theorem}
	Теорема о подполе:
	\[
	GF(p^l) \subseteq GF(p^n) \iff l \mid n
	\]
\end{theorem}
\begin{proof}
В поле $GF(p^n)$ есть фундаментальный многочлен $x^{p^n}-x$,
всего его корни "--- элементы поля.
Аналогично с $GF(p^l)$.
Заметим, что $x^{p^l}-x \mid x^{p^n}-x$, т.е. все корни
одного являются корнями другого.
Отсюда можно сделать рукомахательный вывод про подполе.
В обратную сторону тоже работает.
\end{proof}
\begin{Rem}
Конечно, есть тонкости с выписыванием элементов полей.
Там есть толпа упражнений, но мы их выкидываем :(
\end{Rem}

\begin{lemma}
	Любое конечное поле имеет вид $GF(p^l)$.
	Например, не может быть поля из шести элементов.
\end{lemma}

Пусть есть многочлен $f(x)$ с корнем $\alpha$ кратности $i$:
\[
	f(x) = (x - \alpha)^i \cdot g(x)
\]
Запишем производную:
\[
	f'(x) = i(x - \alpha)^{i-1} \cdot g(x) + (x - \alpha)^i \cdot g'(x)
\]
Вопрос: чему равен их НОД?
В полях характеристики ноль у нас НОД точно делится на $(x-\alpha)^{i-1}$ и не делится на $(x-\alpha)^i$.
А вот в полях характеристики $p$ у нас при $i \nequiv 0 \mod p$
то же самое, а вот при $i \equiv 0 \mod p$ он делится и на $(x-\alpha)^i$
(но не большую степень).
\begin{conseq}
	Если НОД многочлена и производной "--- константа, то многочлен не имеет кратных корней.
\end{conseq}
\begin{proof}
	\TODO
\end{proof}
\begin{Rem}
	Кратность корня не обязательно уменьшается, если перейти от многочлна к НОДу многочлена и производной.
\end{Rem}
\begin{theorem}
	Для любого простого $p$ и натурального $m$
	существует конечное поле $p$.
\end{theorem}

\begin{theorem}
	Для любого многочлена над конечным полем имеется расширение конечного поля,
	в котором этот многочлен распадается на произведение линейных множителей.
\end{theorem}

\begin{theorem}
	Все конечные поля размера $p^m$ изоморфны друг другу.
\end{theorem}

\begin{theorem}
	Можно разложить фундаментальный многочлен в произведение
	всех нормированных и неприводимых над полем $GF(p)$ многочленов,
	степени которых делят $p^m$.
\end{theorem}

\section{Циклические коды}
Немного вернулись назад и повторим.

Код $\cal G$: есть константы $(n, k)$ и поле $GF(q)$.
У него есть $g(x) \in GF(q)[x])$,
причём $\deg g(x) = r = n - k$ и $g(x) \mid x^n - 1$.
Тут $n$ "--- длина кода, $k$ "--- количество информационных символов,
$r$ "--- проверочных символов.

Следствие: все корни $g(x)$ являются корнями $x^n-1$.

Есть требование: $n \perp q$ (взаимно просты).
Это важно, потому что мы хотим найти мультипликативный
порядок $q$ по модулю $n$: $q^m \equiv 1 \mod n$.
Нашли, отсюда следует: $x^n-1 \mid x^{q^m-1}-1$ (т.к. степени делят).

Следствие: все корни $x^n-1$ являются корнями $x^{q^m-1}-1$.
Отсюда следует, что все корни $g(x)$ являются корнями  $x^{q^m-1}-1$ (кто есть фундаментальный многочлен).
Итого все корни $g(x)$ лежат в поле $GF(q^m)$
\begin{lemma}
$GF(q^m)$ "--- минимальное поле, которое содержит корни всех возможных $g(x)$.
\end{lemma}
\begin{Rem}
$q$ не обязательно простое, может быть равно $p^a$.
\end{Rem}

\begin{Rem}
	Теперь у нас два поля: одно для кода $GF(q)$, одно его надполе
	корней многочлена $g(x)$ "--- $GF(q^m)$.
	Впрочем, $q$ может быть равно 1 ($2^4\equiv 1 \mod 15)$.
\end{Rem}

\begin{lemma}
	У $x^n-1$ кратных корней нет.
\end{lemma}
\begin{proof}
	Так как $q=p^a$ и $q \perp n$, то $p \perp n$.
	Отсюда производная $x^n-1$ равна $nx^{n-1}$ и не равна нулю.
	Отсюда НОД многочлена и его производной равен единице.
	Потому что иначе у нас у НОДа есть какой-то нетривиальный корень
	и есть нетривиальный корень у $nx^{n-1}$, а так не бывает.
\end{proof}

Вернёмся к корням $g(x)$ в расширении поля, т.е. в $GF(q^m)$:
\[
g(x) = \prod\limits_{i=1}^r (x-\gamma_i)
\]
Обычно $\gamma_i$ называют \textit{нулями кода} (как нули многочлена).

Теперь возьмём общий вид многочлена (по коэффициентам) и подставим туда какой-нибудь корень:
\begin{gather*}
g(\gamma_i) = \sum\limits_{j=0}^r g_j \gamma_i^j = 0
\\
\begin{pmatrix}
g_0 & g_1 & \dots & g_r
\end{pmatrix}
\times
\begin{pmatrix}
\gamma_i^0 \\ \gamma_i^1 \\ \dots \\ \gamma_i^r
\end{pmatrix}
\\
\text{Можно нулей добавить до длины $n$:}
\begin{pmatrix}
g_0 & g_1 & \dots & g_r & 0 & \dots & 0
\end{pmatrix}
\times
\begin{pmatrix}
\gamma_i^0 \\ \gamma_i^1 \\ \dots \\ \gamma_i^r \\ \gamma_i^{r+1} \\ \dots \\ \gamma_i^{n-1}
\end{pmatrix}
\end{gather*}
Теперь можно записать интересную матрицу над расширенным полем $GF(q^m)$:
\[
H_c = \begin{pmatrix}
\gamma_1^0 & \gamma_1^1 & \gamma_1^2 & \dots & \gamma_1^{n-1} \\
\gamma_2^0 & \gamma_2^1 & \gamma_2^2 & \dots & \gamma_2^{n-1} \\
\vdots & \vdots & \vdots & \ddots & \vdots \\
\gamma_r^0 & \gamma_r^1 & \gamma_r^2 & \dots & \gamma_r^{n-1} \\
\end{pmatrix}
\]
Теперь мы докажем, что это проверочная матрица, но в таком смысле:
это не в чистом виде ортогональное пространство к коду,
а <<ограничение кода на подполе>>.
Берём её двойственное пространство в $GF(q^m)$ (вектора, которые при умножении
на $H_c^\top$ дают ноль), оставляем только вектора с коэффициентами из $GF(q)$ "--- получили код.

\begin{lemma}
	Для любого кодового слова $c$:
	\[
	c H_c^\top = 0
	\]
\end{lemma}
\begin{proof}
	Любое кодовое слово:
	\[
	c(x) = q(x) \cdot g(x)
	\]
	Подставили:
	\[
	q(x) \cdot g(x) \cdot H_c^\top = q(x) \cdot (g(x) \cdot H_c^\top) = q(x) \cdot 0 = 0
	\]
\end{proof}
\begin{lemma}
	\[ c \in \cal G \iff (\forall i \colon c(\gamma_i)) = 0 \]
\end{lemma}
\begin{proof}
	Если $c$ "--- кодовое слово, оно вида $q(x) \cdot g(x)$.
	А $\gamma_i$ "--- корень $g(x)$, ура.

	В обратную сторону:
	пусть есть кодовое слово $c(x)$.
	У каждого корня $\gamma_j$ есть минимальный многочлен $M_j$ (мы не хотим дубликатов среди $M_j$,
	поэтому для них отдельная нумерация).
	Докажем, что такой минимальный многочлен делит наш кодовый вектор.
	Поделим:
	\[
	c(x) = M_j(x) \cdot Q_j(x) + S_j(x)
	\]
	Подставим $x=\gamma_1, \gamma_2, \dots, \gamma_r$.
	В $r$ точках кодовый вектор занулился,
	$M_j$ занулился (потому что $\gamma$ "--- корень $g(x)$, а он делится на $M_j$).
	То есть $S_j$ занулилось в $r$ точках, а его степень меньше $r$.
	(потому не больше степени $M_j$, а у него степень не больше $g$).

	Отсюда получаем, что кодовый вектор делится на все минимальные многочлены,
	а мы как раз знали $g(x)=\prod M_j(x)$.
\end{proof}

\begin{Rem}
	Получили матричное представление циклических кодов.
	Ну, уже над расширением поля $GF(q^m)$ вместо $GF(q)$,
	но зато снова можем привлечь алгебру.
\end{Rem}

\begin{Rem}
	Запишем код Хэмминга как циклический.
	Параметры: $(2^r-1,2^n-r-1,3)$ ($r \ge 2$).
	Мы его строили так: в проверочной матрице
	ширины $2^r-1$ записать все элементы из $r$ бит.
	Расстояние у этого кода "--- 3.

	Теперь строим циклический.
	Взяли $GF(q=2)$.
	Ищем мультипликативный порядок $q=2$ по модулю $n=2^r-1$.
	Это $m=r$.
	Расширились до $GF(q^r=2^r)$, взяли примитивный элемент $\alpha$.
	Строим проверочную матрицу $H_c$:
	\[
	H_c = \begin{pmatrix}
	\alpha^0 & \alpha^1 & \dots & \alpha^{2^r-1}
	\end{pmatrix}
	\]
	Как раз получили в качестве столбиков все возможные набор бит.

	У $\alpha$ есть минимальный многочлен, он также является примитивным,
	его и возьмём за g(x):
	\[
		M_\alpha(x) = \pi(x) = g(x)
	\]
\end{Rem}

\subsection{Пример кода Хэмминга как циклического}
Пусть $r=3$.
Будем строить $(7, 4, 3)$-код.
Проверочная матрица для обычного кода Хэмминга:
\[
H = {\begin{pmatrix}
0 & \TODO \\
0 & \\
1 & \\
\end{pmatrix}}
\]

Теперь берём $GF(2^3)$.
Возьмём порождающий (и примитивный)cмногочлен $x^3+x+1$ и его примитивный корень $\alpha$.
Получаем табличку:
\begin{align*}
0 = (0, 0, 0) \\
\alpha^0 = (1, 0, 0) \\
\alpha^1 = (0, 1, 0) \\
\alpha^2 = (0, 0, 1) \\
\alpha^3 = (1, 1, 0) \\
\alpha^4 = (0, 1, 1) \\
\alpha^5 = (1, 1, 1) \\
\alpha^6 = (1, 0, 1) \\
\alpha^7 = (1, 0, 0) \\
\end{align*}
Отсюда легко записывается табличка $H_c$.
У неё все столбцы разные.

А можно и порождающую матрицу записать:
\[
G=\begin{pmatrix}
1 & 1 & 0 & 1 & 0 & 0 & 0 \\
0 & 1 & 1 & 0 & 1 & 0 & 0 \\
0 & 0 & 1 & 1 & 0 & 1 & 0 \\
0 & 0 & 0 & 1 & 1 & 0 & 1 \\
\end{pmatrix}
\]

\begin{Exercise}
Записать $H_c$ как матрицу из элементов вида $\alpha_0$.
Потом раскроем элементы на строчки высоты три, получим матрицу высоты 9.
Потом докажем, что у неё только первые три строчки линейно независимы, дальше упс.
По этому поводу нам первых трёх строчек хватает (как раз где было $\alpha, \alpha^1, \alpha^2, \dots$).
\end{Exercise}

\begin{Exercise}
В кажестве упражнения: $r=4$, построить всё то же самое для длины $n=15$,
показав, что код Хэмминга можно рассмотреть, как циклический код.
Поле $2^4$, многочлен $x^4+x+1$.
\end{Exercise}

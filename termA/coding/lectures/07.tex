\section{Конечные поля}
Начнём с простого конечного поля $GF(p)$ (арифметика по модулю $p$)
и неприводимого многочлена $\pi(x) \in GF(p)[x]$ степени $m$.
Случаи будем рассматривать общие (для всех конечных полей), доказательства опускаем,
а примеры будут для $p=2$.

\begin{Rem}
Неприводимость нужна конкретно над нашим полем.
\end{Rem}

\textit{Расширим} поле так, чтобы многочлен стало можно разложить.
Или \textit{присоединяем} один корень полю.
По определению полагаем $\pi(\alpha)=0$.
После этого внезапно получим поле размера $p^m$ ($GF(p^m)$).

\begin{exmp}
Взяли $GF(2)$ и неприводимый $\pi(x)=x^2+x+1$.
Присоединили корень $\alpha$, получили $GF(2^2)=\{0, 1, \alpha, \alpha^2\}$.

Отсюда получаем $\alpha^2=\alpha+1$.
\end{exmp}

\begin{Rem}
Обычно используются не просто неприводимые, а примитивные многочлены,
чтобы можно было легко увидеть в конечном поле циклическую группу
из ненулевых элементов.
\end{Rem}

Как вообще можно смотреть на элементы конечного поля? (табличка)
\begin{itemize}
\item Многочлены по модулю $\pi(x)$: $0$, $1$, $x$, $x+1$
\item Вектора (как многочлен; важно, где старший разряд, а где младший)
\item Степени $\alpha$: $0$, $\alpha^0$, $\alpha^1$, $\alpha^2$, $\dots$, $\alpha^{p^m-1}$
\item Индексное представление (показатель степени $\alpha$): $-$ (потому что у нуля индекcа нет), $0$, $1$, $2$
\end{itemize}
Чтобы складывать/вычитать, удобны первые два представления.
Чтобы умножать, удобны два последних.
При этом переводить одно в другое сложно: тут просто для $GF(2^2)$ всё отсортировано получилось.

От представления элементов в поле у нас может очень сильно зависеть сложность вычислений.

\subsection{Примитивные элементы конечного поля}
\begin{Def}
	\textit{Мультипликативный порядок} (ord) $\beta \neq 0$ "--- это
	такое минимальное $l>0$, что $\beta^l=1$.
\end{Def}
\begin{Rem}
	Слово <<мультипликативный>> обычно опускают, элемент
	обычно всегда берётся ненулевой, это оговорку опускают.
\end{Rem}

\begin{theorem}
	Порядок любого элемента "--- делитель $p^m-1$.
\end{theorem}

\begin{theorem}
	Возьмём известный многочлен (сами будем называть <<фундаментальным>>,
	но это не общепринятое название):
	\[
		x^{p^m} - x = \prod_{\beta \in GF(p^m)} (x - \beta)
	\]
	Его корни "--- это в точности все элементы конечного поля.
\end{theorem}

\begin{theorem}
	В любом конечном поле существует элемент порядка $p^m-1$, он называется \textit{примитивным}.
\end{theorem}
\begin{Rem}
	Их бывает много.
	В теории чисел такая штука называется <<первообразный корень>>.
\end{Rem}

\begin{Def}
	Если у нас присоединённый корень в расширенном поле является примитивным,
	то наш неприводимый многочлен называется \textit{примитивным}.
\end{Def}
\begin{Rem}
	В алгебре тоже есть примитивные многочлены, но другие.
\end{Rem}

\begin{Exercise}
	Взять непримитивный многочлен и расписать его корни, проверить
	на примитивность.
\end{Exercise}

\begin{Rem}
	Можно ещё выделить в поле \textit{полиномиальный/стандартный базис}:
	$\alpha^0, \alpha^1, \dots, \alpha^{m-1}$
	(чтобы было удобно складывать).
\end{Rem}

\begin{Def}
	Характеристика элемента $x$ "--- минимальное число
	$n$ такое, что $\underbrace{\alpha+\alpha+\dots+\alpha}_{n}=0$.
\end{Def}
\begin{theorem}
	Характеристика поля (char) "--- НОК характеристик элементов.
\end{theorem}
\begin{Rem}
	У поля $\R$ и похожих считается характеристика, равная нулю.
\end{Rem}
\begin{lemma}
	$\Char GF(p^m) = p$
\end{lemma}

Ещё бывают термины: подполе, надполе (расширение поля).

\begin{theorem}
Любое конечное поле имеет единственное простое подполе (поле $GF(p)$).
\end{theorem}

\begin{Def}
	\textit{Минимальный многочлен} для элемента $\beta \in GF(p^m)$:
	\begin{enumerate}
	\item Степень минимальна
	\item Нормирован
	\item Все коэффициенты "--- \textit{из какого-то подполя}
	\end{enumerate}
\end{Def}
\begin{exmp}
	Минимальный многочлен для $i \in \C$ из подполя $\C$ "--- это $x-i$.
	А вот из подполя $\R$ "--- это уже $x^2+1$.
\end{exmp}
\begin{theorem}
	Для любого подполя и любого элемента мы минимальный многочлен найдём.
	Ну, для простого точно найдём.
\end{theorem}
\begin{Rem}
	Свойства минимальных многочленов:
	\begin{enumerate}
	\item Неприводим над полем своих коэффициентов
	\setcounter{enumi}{2}
	\item Всегда является делителем <<фундаментального>> многочлена $x^{p^m}-x$
	\item Степень всегда $\le m$
	\item Если элемент примтивный, то степень его минимального многочлена "--- ровно $m$
	\end{enumerate}
\end{Rem}
\begin{theorem}
	Для любого элемента существует \textit{единственный} минимальный многочлен над простым подполем.
\end{theorem}

\begin{lemma}
	Для любых элементов конечного поля характеристики $p$:
	\[ (\alpha + \beta)^p = \alpha^p+\beta^p \]
\end{lemma}
\begin{Def}
	Элементы конечного поля называются \textit{сопряжёнными по конкретному подполю},
	если их минимальные многочлены по этому же подполю совпадают.
\end{Def}
\begin{Rem}
	Здесь и далее мы будем рассматривать только простое подполе.
	Для простоты.
\end{Rem}
\begin{exmp}
	Пример для $\pi(x)=x^2+x+1$: можно взять его
	присоединённый корень $\alpha$ и $\alpha^2$.
	Это его сопряжённые корни.
\end{exmp}

\begin{Def}
Циклотомический класс с образующим $s$ по модулю $p^m-1$ "--- это
класс, в котором лежат числа по модулю $p^m-1$:
\[
C_s = \{s, sp, sp^2, \dots, sp^{l-1}\}
\]
\end{Def}
\begin{exmp}
Например, $p^m=2^4=16$, берём по модулю $15$:
\begin{itemize}
\item $\{0\}$ (тривиальный циклотомический класс)
\item $\{1,2,4,8\}$
\item $\{3,6,12,9\}$
\item $\{5,10\}$
\item $\{7,14,13,11\}$
\end{itemize}
\end{exmp}

\begin{Rem}
	Свойства минимальных многочленов:
	\begin{enumerate}
	\setcounter{enumi}{5}
	\item Любые элементы $\alpha$ и $\alpha^p$ сопряжены в простом подполе
		(см. $a_i=a_i^p$ и $(a+b+c)^p=a^p+b^p+c^p$)
	\item Формула для построения минимального многочлена любого элемента
		вида $\alpha^i$, где $i \in C_s$:
		$M_{\alpha^i}(x) = \prod_{j\in C_s} (x-\alpha^j)$.
		А вот для элемента ноль считаем минимальным многочленом $x$.
		Ещё можно строить не снизу-вверх, а сверху-вниз, разложив
		фундаментальный многочлен 
	\item Фундаментальный многочлен $x^{p^m}-x$ раскладывается
	    на множители над соответствующим подполем и получаем в точности
	    минимальные многочлены.
	\setcounter{enumi}{10}
	\item Все корни минимального многочлена имеют одинаковый порядок
	\end{enumerate}
\end{Rem}
\begin{conseq}
	Циклотомические классы дают способ генерировать минимальные многочлены, не разлагая фундаментальный многочлен:
	\begin{itemize}
	\item Взяли числа по модулю $p^m-1$ и разбили на циклотомические классы
	\item Взяли циклотомический класс, все элементы $i_1, i_2, \dots$ из него, перемножили
		многочлены вида $(x-i_1)$ (у них коэффициенты ещё из надполя)
	\item
		Обнаружили, что результат имеет коэффициенты из подполя "--- это как раз минимальный многочлен
	\end{itemize}
\end{conseq}

Где смотреть конечные поля: Нидл-Райтер "Конечные поля" (переведены на русский)
и хороший учебник по теории кодирования.

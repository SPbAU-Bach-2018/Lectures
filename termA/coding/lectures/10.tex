\subsection{Двоичные коды БЧХ}

Используем $q=2$ (алфавит).
\begin{lemma}
	В поле характеристики 2 минимальный многочлен элемента
	и его квадрата совпадают.
\end{lemma}
\begin{proof}
	\TODO
\end{proof}

Возьмём для простоты $b=1$ и нечётное $\delta=2t+1$.
Итого получаем порождающий многочлен:
\[
g(x) = НОК(M_1(x), M_3(x), M_5(x), M_7(x), \dots, M_{2t-1}(x))
\]
Чётные $M_{2k}(x)$ мы не написали, так $M_{2k}(x)=M_k(x)$ (так как
корни возвели в квадрат, а в поле характеристики два корень остался).

Теперь оценим степень:
\begin{gather*}
n - k = \deg g(x) \le t \cdot m \\
k \ge n - tm
\end{gather*}
Это более точная оценка, чем в общем случае.

Запишем проверочную матрицу:
\[
H = \begin{pmatrix}
1 & \alpha & \alpha^2 & \dots & \alpha^{n-1} \\
1 & \alpha^3 & \alpha^6 & \dots & \alpha^{3(n-1)} \\
\vdots & \vdots & \vdots & \ddots & \vdots \\
1 & \alpha^{2t-1} & \alpha^{(2t-1)\cdot2} & \dots & \alpha^{(2t-1)\cdot(n-1)}
\end{pmatrix}
\]

\begin{lemma}
Снова общий случай для $q$.
Взяли произвольный не-ноль $\gamma \in GF(q^m)$.
Составили из него две строчки:
\[
(\gamma^0, \gamma^1, \dots, \gamma^{n-1}) \\
((\gamma^0)^q, (\gamma^1)^q, \dots, (\gamma^{n-1})^q)
\]
Над $GF(q^m)$ они линейно независимы.
А вот если каждый элемент записать как столбец,
то последние $q$ строчек будут линейно зависимы от первых $q$.
Доказывать не будем, там техника.
\end{lemma}

\subsection{Примеры}
Возьмём $q=2$, $m=4$.
Максимально возможное $n$ тут равно 15, возьмём его.
Для начала надо разложить над $GF(2)$:
\[
x^{16}-x=x(x-1)(x^2+x+1)(x^4+x+1)(x^4+x^3+1)(x^4+x^3+x^2+x+1)
\]
\begin{Rem}
Сюда входят все неприводимые многочлены степеней 1, 2, 4 "--- делители 16.
Так для любых $q$, $m$.
\end{Rem}
Выбрали в качестве примитивного многочлена $x^4+x+1$.
Назовём его корнем $\alpha$.
Напишем циклотомические классы многочленов:
\[
\begin{matrix}
 0 			\\ 	1	\\		     \alpha^5   & \alpha^{10} \\ 
 \alpha^1 		& \alpha^2   & \alpha^4   & \alpha^8 \\ 
 \alpha^7  & \alpha^{14} & \alpha^{13} & \alpha^{11} \\ 
 \alpha^3  & \alpha^6   & \alpha^{12} & \alpha^9
\end{matrix}
\]
Просто берём и подбираем, пользуясь
свойствами $\alpha^4+\alpha+1=0$ и $\alpha^{15}=1$ (мы надеемся, что $\alpha$ примитивный).

Давайте построим коды БЧХ и их порождающие многочлены для разных $\delta$.
Для начала смотрим корни, потом смотрим на минимальные многочлены.

\[
\begin{matrix}
\delta & корни & g(x) & k \\
1&\varnothing&1&15\\
3&\{\alpha^1,\alpha^2\}&x^4+x+1&11\\
5&\{\alpha^1,\alpha^2,\alpha^3,\alpha^4\}&(x^4+x+1)(x^4+x^3+x^2+x+1)=x^8+x^7+x^6+x^4+1&7\\
7&\dots&(\dots)\cdot(x^2+x+1)=x^{10}+x^8+x^5+x^4+x^2+x+1&5\\
9&\dots&(\dots)\cdot(x^2+x+1)=x^{14}+x^{13}+\dots+x^1+x^0 &1\\
11&\\
13&\\
\end{matrix}
\]
$k$ вычисляется методом <<посмотреть на степень многочлена, это количество
проверочных символов, вычли из $n$>>.

Посмотрим на $\delta=3$.
Запишем для него $G$ прямым вычислением, посдвигав первую строчку:
\[
G=\begin{pmatrix}
1 & 1 & 0 & 0 & 1 & 0 & \dots & 0 \\
0 & 1 & 1 & 0 & 0 & 1 & \dots & 0 \\
\vdots
\end{pmatrix}
\]
А можно записать как код БЧХ:
\[
H=\begin{pmatrix}
1 & \alpha & \alpha^2 & \dots & \alpha^{14} \\
1 & \alpha^2 & \alpha^4 & \dots & \alpha^{28}
\end{pmatrix}
\]
И раскрыть элементы большого поля в элементы подполя:
\[
\hat H = \begin{pmatrix}
1 & 0 & 0 & 0 & 1 & 0 & 0 & 1 & 1 & 0 & 1 & 0 & 1 & 1 & 1 \\
0 & 1 & 0 & 0 & 1 & 1 & 0 & 1 & 0 & 1 & 1 & 1 & 1 & 0 & 0 \\
0 & 0 & 1 & 0 & 0 & 1 & 1 & 0 & 1 & 0 & 1 & 1 & 1 & 1 & 0 \\
0 & 0 & 0 & 1 & 0 & 0 & 1 & 1 & 0 & 1 & 0 & 1 & 1 & 1 & 1 \\

1 & 0 & 1 & 0 & 1 & 1 & 1 & 1 & 0 & 0 & 0 & 1 & 0 & 0 & 1 \\
0 & 0 & 1 & 0 & 0 & 1 & 1 & 0 & 1 & 0 & 1 & 1 & 1 & 1 & 0 \\
0 & 1 & 0 & 1 & 1 & 1 & 1 & 0 & 0 & 0 & 1 & 0 & 0 & 1 & 1 \\
0 & 0 & 0 & 1 & 0 & 0 & 1 & 1 & 0 & 1 & 0 & 1 & 1 & 1 & 1 \\
\end{pmatrix}
\]
\begin{Exercise}
Показать, что последние четыре строчки действительно выражаются через первые.
\end{Exercise}
Итого берём в качестве проверочной матрицы первые четыре строчки из нулей и единиц.
Заметим, что это как раз код Хэмминга "--- у него в проверочной матрице все столбцы различны.

Теперь можно записать для $\delta=5$:
\[
H=\begin{pmatrix}
1 & \alpha & \alpha^2 & \dots & \alpha^{14} \\
1 & \alpha^3 & \alpha^6 & \dots & \alpha^{42}
\end{pmatrix}
\]

Давайте проверим наши оценки.
Общий случай: $k \ge n - m(\delta-1)$, для двоичного: $k \ge n - tm$.
В случае $\delta=1$ общая оценка даёт $5$, а двоичная "--- $11$,
что совпадает с точной оценкой ($k$), лучше их использовать.

Теперь $\delta=7$.                          
Считаем оценки: $k \ge 15 - 6 \cdot (5-1) = -9$, что-то так себе.
А с другой уже неточно, но лучше: $15 - 3 \cdot 4=3$.
Неидеально, сбоит, но уже лучше.

Теперь $\delta=9$.
У нас истинное расстояние стало 14, что сильно больше конструктивного 9.
И мы въехали в крайний случай: остался всего один информационный символ.

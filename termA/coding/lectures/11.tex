\section{Коды Рида-Соломона}
Сейчас их введём как подкласс кодов БЧХ.
А потом покажем, что БЧХ тоже можно определить через Рида-Соломона.
Так что вопрос <<что первично>> "--- философский.

Определяем код Рида-Соломона как подкласс кодов БЧХ,
в котором $m=1$, т.е. $n=q-1$, т.е. $n=q-1$.

Здесь получается просто определить минимальные
многочлены: $M_i(x)=x - \alpha^i$.
Порождающий многочлен тоже получается (степени $n-k=\delta-1$):
\[
g(x) = (x-\alpha^b)(x-\alpha^{b+1})\dots(x-\alpha^{b+\delta-2})
\]

Отсюда видим, что коды лежат на границе Синглтона
(оптимальны в смысле границы Синглтона):
\[
\delta = d = n - k + 1
\]
Так что построить код лучше Рида-Соломона по расстоянию нельзя.
\begin{Rem}
	Не во всех задачах нам чем больше расстояние, тем лучше код.
	Например, иногда полезно находить очень много ошибок, а
	исправлять одну (например, в RAM).
\end{Rem}
\begin{Rem}
	В кодах Рида-Соломона любые $k$ символов можно объявить
	информационными (т.е. по ним можно получить остальные).
	Другими словами, он \textit{разделимый}.
	В общем случае это неверно, например, вот для такой
	матрицы у нас выбрать первые два столбца информационными нелья
	(информация потеряется):
	\[
	\begin{pmatrix}
	1 & 1 & 0 \\
	0 & 0 & 1
	\]
\end{Rem}

\begin{exmp}
	Возьмём $q=5$ и $GF(5)$.
	Длина $n=q-1=4$.
	Построим код с расстоянием $\delta=3$, и возьмём $b=1$:
	\[
	g(x) = (x - \alpha)(x - \alpha^2)
	\]
	$\alpha$ "--- это примитивный элемент поля.
	Например, двойку (ещё бывает тройка, выбор неважен):
	\[
	g(x) = (x - 2)(x - 4) = x^2 + 4x + 3
	\]
	Строим порождающую матрицу для нашего кода $GF(5), (4, 2, 3)$:
	\[
	G =
	\begin{pmatrix}
	3 & 4 & 1 & 0 \\
	0 & 3 & 4 & 1
	\end{pmatrix}
	\]
	Итого формула для кодирования:
	\[
	(c_1, c_2, c_3, c_4) =
	\begin{pmatrix}
	v_1 & v_2
	\end{pmatrix}
	\begin{pmatrix}
	3 & 4 & 1 & 0 \\
	0 & 3 & 4 & 1
	\end{pmatrix}
	\]
	Всего у нас получается $q^k=25$ кодовых слов.

	Проверочная матрица от кодов БЧХ прекрасно работает и для кодов Рида-Соломона.
	Чтобы была согласована с нашей $G$, строим так (так как $b=1$):
	\begin{gather*}
	H = \begin{pmatrix}
	1 & \alpha & \alpha^2 & \alpha^3 \\
	1 & \alpha^2 & \alpha^4 & \alpha^6=\alpha^2
	\end{pmatrix} \\
	H = \begin{pmatrix}
	1 & 2 & 4 & 3 \\
	1 & 4 & 1 & 4
	\end{pmatrix}
	\end{gather*}
\end{exmp}

Сведение к кодам БЧХ: возьмём код Рида-Соломона над полем $GF(q=p^k)$.
Если записать каждый элемент в матрице как $k$ элементов над подполем $GF(q^l)$
(где $k \mid l$), то получим код БЧХ над подполем $GF(q^l)$.
Расстояния, правда, разные получаются, ну да ладно.

\subsection{Методы кодирования}
Как для циклических кодов и БЧХ (\textit{несистематический метод}):
\[
C(x) = a(x) \cdot g(x)
\]

Как для циклических кодов и БЧХ (\textit{систематический метод}):
\begin{gather*}
x^r a(x) = y(x) b(x) + v(x) \quad \text{поделили с остатком}
c(x) = x^r a(x) - v(x) \quad \text{закодировали}
\end{gather*}

Спектральное кодирование (термин неформальный), специфично для Рида-Соломона
(можно и для любых алгебраических, но мы рассматриваем только для этих):
основано на дискретном преобразовании Фурье.

Сначала вспомним его над конечным полем (вместо поля комплексных чисел):
\[
F_k = \sum_{i=0}^n f_i \alpha^{ik}, k \in [0, n - 1]
\]
Тут $f_i$ "--- входной вектор размера $n=p^m-1$, $F_k$ "--- выходной вектор и размера $k$,
а $\alpha$ "--- примитивный элемент нашего конечного поля $GF(p^m)$ (порядка $n$).

Обратное преобразование:
\[
f_i = \frac{1}{n \bmod p} \sum_{k=0}^{n-1} F_k \alpha^{-ik}
\]

Есть тонкость: не так просто написать FFT при $p=2$,
потому что там может быть надо делить на 2, а у нас поле характеристики 2, делить не можем.

Можно записать в матричном виде (при $n=k$):
\[
\begin{pmatrix}
F_0 \\ F_1 \\ \vdots \\ F_{n-1}
\end{pmatrix}
=
\begin{pmatrix}
1 & 1 & 1 & \dots & 1 \\
1 & \alpha & \alpha^2 & \dots & \alpha^{n-1} \\
\vdots & \vdots & \vdots & \ddots & \vdots \
1 & \alpha^{n-1} & (\alpha^2)^{n-1} & \dots & (\alpha^{n-1})^{n-1})
\end{pmatrix}
\begin{pmatrix}
f_0 \\ f_1 \\ \vdots \\ F_{f-1}
\end{pmatrix}
\]
Это весьма похоже на матрицу Вандермонда и проверочную матрицу кода.

Кстати, преобразование Фурье "--- это ещё и просто вычисление
многочлена в точках циклической группы $\alpha^0$, $\alpha^1$, \dots, $\alpha^{n-1}$.

А в нашем случае (когда мы что-то знаем про $n$ и $p$) у нас нормирующий множитель простой:
\[
n \bmod p = (q - 1) \bmod p = -1
\]

Теперь можно, наконец, вывести спектральное кодирование:
\[
C(x) = DFT(a(x))
\]
Применили дискретное преобразование Фурье к вектору, добитому нулями
(вычислили в точках циклической группы; это как умножение на матрицу Вандермонда),
получили кодовый вектор.

\begin{theorem}
	Коды $\cal G = \{ c \mid c = DFS(a), \deg (a) < k, a(x) \in GF(q)[x] \}$
	являются $(n,k)$-кодом Рида-Соломона.
\end{theorem}
\begin{proof}
	\TODO
\end{proof}

Красивая теорема, чтобы можно было напрямую доказать расстояние кода Рида-Соломона,
а не мучаться с Вандермондом (как в БЧХ):
\begin{theorem}
	Расстояние кода Рида-Соломона над $GF(q)$ равно $n-k+1$.
\end{theorem}
\begin{proof}
	Доказательство через спектр: возьмём какой-нибудь
	ненулевой информационный многочлен $a(x)$ степени не больше $k-1$.
	Правило получение очередного кодового символа: $c_i=a(\alpha^i)$.
	Чтобы он был ненулевым, $\alpha^i$ должно быть корнем.
	А корней в поле не больше, чем степень, т.е. $k-1$.
	Значит, ненулей из $n$ хотя бы $n-(k-1)=n-k+1$, что и требовалось.
\end{proof}

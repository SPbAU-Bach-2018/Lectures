\section{Коды БЧХ}
Сужаемся ещё больше: мы сидели в циклических кодах,
а теперь смотрим на БЧХ-кода (BCH codes).
Очень специфический класс кодов.

Ещё рядом посмотрим на коды Рида-Соломона (RS).
Кто из них подкласс другого "--- зависит как посмотреть.
Были придуманы в одно и то же время.

Эти коды "--- прорыв: можно не только конструктивно
построить, но и гарантировать что-то про параметры
(расстояние кода).

Вспомним матрицу Вандермонда:
\[
\begin{pmatrix}
1 & a_1 & a_1^2 & \dots & a_1^n \\
1 & a_2 & a_2^2 & \dots & a_2^n \\
\vdots & \vdots & \vdots & \ddots & \vdots \\
1 & a_n & a_n^2 & \dots & a_n^n
\end{pmatrix}
\]
Чем она хороша?
Невырождена $\iff$ все $a_i$ различны.

Ещё бывают \textit{вполне положительные матрицы} "--- те, у которых
все подматрицы невырождены.
Матрицы Вандермонда вполне положительны почти над любым полем
(над некоторыми конечными не вполне положительны), это нам будет полезно
для построения кодов Рида-Соломона и БЧХ.
А ещё бывают матрицы Коши, на них построены коды Гоппы.

\begin{theorem}[Граница БЧХ]
	Пусть есть циклический $(n, k, d)$-код $\cal G$ над $GF(q)$,
	У него есть порождающий многочлен $g(x)$, а у него есть корни.
	Например, корень $\alpha$ "--- корень из расширения поля $GF(q^m)$,
	причём такой, что $ord (\alpha) = n \mid q^m-1$ (в частности, $q \perp n$).
	Тогда если есть $b \ge 0$ и есть \textit{подряд идущие корни} (жаргон)
	$\alpha^b, \alpha^{b+1}, \dots, \alpha^{b+\delta-2}$,
	то расстояние кода $d \ge \delta$.
\end{theorem}
\begin{Rem}
	Обычно берут $b=0$.
\end{Rem}
\begin{Rem}
	Бывает тривиальный случай, когда $g(x)=1$ и не имеет корней,
	тогда расстояние кода "--- единица.
	А в остальных случаях обычно найдём $\delta \ge 2$,
	т.е. будет $d \ge 2$.
\end{Rem}
\begin{Rem}
	План совсем потом будет такой: зафиксировали $q$, $n$, $\delta$.
	Дальше как-то нашли $\alpha$ и $b$.
	Ещё получаем какое-то ограничение сверху на $k$: $\deg g(x)=n-k$.
\end{Rem}
\begin{Rem}
	Просто взять абы какое $\alpha$, большое $\delta$ не поможет построить $g(x)$
	как $(x-\alpha^0)(x-\alpha^1)\dots(x-\alpha^{\delta-2}$, потому что
	$g(x)$ должен быть порождающим, а тут может оказаться даже приводимым.
\end{Rem}
\begin{proof}
	Мы ищем кодовое слово минимального веса.
	Пусть есть кодовое слово $c(x) \in \cal C=\sum_{j=0}^{n-1} c_jx^j$.
	У него есть корни (потому что оно делится на $g(x)$):
	\[
		c(\alpha^i) = 0, i \in [b; b+\delta-2]
	\]
	Получаем систему из $\delta-1$ уравнений:
	\[	
		\begin{cases}
			\sum_{j=0}^{n-1} c_j\cdot(\alpha^i)^j = 0, i \in [b; b + \delta-2]
		\end{cases}
	\]
	Можно переписать в матричном виде ($c = (c_0, c_1, \dots, c_{n-1})$):
	\begin{gather*}
		H=
		\begin{pmatrix}
			(\alpha^b)^0 & (\alpha^b)^1 & \dots & (\alpha^b)^{n-1} \\
			\vdots & \vdots & \ddots & \vdots \\
			(\alpha^{b+\delta-2})^0 & (\alpha^{b+\delta-2})^1 & \dots & (\alpha^{b+\delta-2})^{n-1}
		\end{pmatrix} \\
		cH = \begin{pmatrix}
			0 & 0 & \dots & 0
		\end{pmatrix}
	\end{gather*}
	Это \textit{неполная проверочная матрица}: в ней не хватает условий
	на остальные корни $g(x)$ (коих $n-k \ge \delta - 1$; корни кода).
	Но у нас там была на прошлой лекции лемма/теорема про ограничение
	кодового слова на подполе, поэтому нам этой матрицы хватит и для проверки
	(\TODO).

	Возьмём матрицу и поменяем степени местами:
	\[
		H=
		\begin{pmatrix}
			(\alpha^0)^b & (\alpha^1)^b & \dots & (\alpha^{n-1})^b \\
			\vdots & \vdots & \ddots & \vdots \\
			(\alpha^0)^{b+\delta-2} & (\alpha^1)^{b+\delta-2} & \dots & (\alpha^{n-1})^{b+\delta-2}
		\end{pmatrix}
	\]
	Теперь достаточно показать, что любые $\delta-1$ столбцов линейно независимы
	(это не ранг, если что; мы уже это доказывали).
	Взяли столбцы с различными номерами $j_1, j_2, \dots, j_{\delta-1}$.
	Определили $\beta_1=\alpha^{j_1}, \beta_2=\alpha^{j_2}, \dots$.
	Выписали подматрицу из этих столбцов:
	\[
		\begin{pmatrix}
		(\alpha^{j_1})^b & (\alpha^{j_2})^b & \dots & (\alpha^{j_{\delta-1}})^b \\
		\vdots & \vdots & \ddots & \vdots \\
		(\alpha^{j_1})^{b+\delta-2} & (\alpha^{j_2})^{b+\delta-2} & \dots & (\alpha^{j_{\delta-1}})^{b+\delta-2}
		\end{pmatrix}
		=
		\begin{pmatrix}
		\beta_1^b & \beta_2^b & \dots & \beta_{\delta-1}^b \\
		\vdots & \vdots & \ddots & \vdots \\
		\beta_1^{b+\delta-2} & \beta_2^{b+\delta-2} & \dots & \beta_{\delta-1}^{b+\delta-2}
		\end{pmatrix}
	\]
	Получилась квадратная матрица.
	Считаем определитель: выносим $\beta_1^b \cdot \beta_2^b \cdots \dots \cdot \beta_{\delta-1}^b$,
	получился определитель Вандермонда.
	Итого матрица невырождена $\iff$ все $\beta_j$ различны и не нули.
	А они различны, потому что $ord(\alpha)=n$ и $0 \le j < n$.
\end{proof}

\begin{Def}[Код БЧХ]
	Циклический код БЧХ длины $n$ над $GF(q)$ (где $n \perp q$)
	задаётся порождающим многочленом:
	\[
	g(x) = НОК(M_b(x), M_{b+1}(x), M_{b+\delta-2}(x))
	\]
	где $M_i(x)$ "--- минимальный многочлен элемента $\alpha^i$
	(многочлен наименьшей степени, который имеет такой корень; коэффициенты лежат в подполе),
	где $\alpha \in GF(q^m)$, а $ord \alpha = n \mid q^m-1$
	и $b \ge 0$.
	У него конструктивное расстояние $\delta$ (т.е. реальное расстояние хотя бы $d$).
\end{Def}

Описание происходящего для кода БЧХ:
можно взять (почти?) произвольные $n \perp q$,
потом взять поле $GF(q^m)$ (где $m$ "--- мультипликативный порядок $q$ по модулю $m$).
Потом можно взять какой-нибудь элемент $\alpha$ порядка $n$,
его минимальный многочлен делит $x^n-1$.
Дальше можно взять минимальные многочлены для $\alpha^b$, $\alpha^{b+1}$,
у этих элементов порядок "--- делитель $n$, следовательно,
эти минимальные многочлены тоже делят $x^n-1$.

Теперь у нас $g(x)$ берётся как произведение различных $M(x)$
(так как все минимальные многочлены неприводимы $\Rightarrow$
они либо различны, либо совпадают).
Каждый из них был делителем $x^n-1$, значит, $g(x)$ тоже делитель
(а это важно, чтобы код был циклический).

Его степень "--- ранг полной проверочной матрицы
(была теорема).
Дальше будет теорема о том, что ранг полной совпадает
с рангом неполной матрицы (она техническая, возможно, будет потом:
там идея в том, что минимальные многочлены нам привносят только корни
вида $(b+i)^k$).
Итого есть ограничение $\deg g(x) \le m(\delta - 1)$
(потому что в неполной матрице над $GF(q^m)$ у нас
$\delta-1$ строчка, а в неполной над $GF(q)$ "--- уже $m(\delta-1)$).

\begin{theorem}
	Была старая теорема: если есть элемент $\beta$ порядка
	$n$, то $M_\beta(x) \mid x^n-1$.
\end{theorem}
\begin{Def}
	Если $b=1$, то у нас код БЧХ \textit{в узком смысле}.
\end{Def}
\begin{Def}
	Если $n=q^m-1$, то у нас \textit{примитивный код БЧХ}
	(потому что $\alpha$ получается примитивным элементом).
\end{Def}

\begin{Exercise}
	Построить коды БЧХ.
\end{Exercise}

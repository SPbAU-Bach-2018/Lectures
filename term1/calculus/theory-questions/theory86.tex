\section{Критерии выпуклости в терминах первой и второй производных}

\begin{theorem}{Критерий выпуклости}
$f\colon \left<a, b\right> \ra \R$, $f$ дифференцируема на $(a, b)$.
$$f\text{ (строго) выпукла} \Lra f' \text{ (строго) возрастает}$$
\end{theorem}
\begin{proof}
$\Ra$: $x_1 < x_2$
$$f(x) \geqslant f(x_1) + (x - x_1) f'(x_1)$$
$$f(x) \geqslant f(x_2) + (x - x_2) f'(x_2)$$
Подставим
$$f(x_2) \geqslant f(x_1) + (x_2 - x_1) f'(x_1)$$
$$f(x_1) \geqslant f(x_2) + (x_1 - x_2) f'(x_2)$$
$$f'(x_1) \leqslant \frac{f(x_2)-f(x_1)}{x_2-x_1} \leqslant f'(x_2)$$
$La$:
Нужно проверить, что 
$$\frac{f(u)-f(v)}{u-v} \leqslant \frac{f(v) - f(w)}{v-w}$$
По теороеме Лагранжа, есть точки $\xi < \eta$
$$\frac{f(u)-f(v)}{u-v} = f'(\xi) \leqslant f'(\eta) = \frac{f(v) - f(w)}{v-w}$$
\end{proof}

\begin{theorem}{Критерий выпуклости через вторую производную}
$f\colon \left<a, b\right> \ra \R$, $f$ дважды дифференцируема на $(a, b)$.
$$f\text{ выпукла} \Lra f'' > 0$$
\end{theorem}
\begin{proof}
Смотрим на теоремы о монотонности.
\end{proof}

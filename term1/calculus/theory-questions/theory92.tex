\section{Интегрирование по частям}

\begin{theorem}{Интегрирование по частям}
$f, g$~--- дифференцируемые, $f'g$~--- интегрируемая.
$$\int fg'\d x = fg - \int f'g \d x$$
\end{theorem}
\begin{proof}
$\Phi$~--- первообразная $f'g$.
$$(fg - \Phi + c)' = fg' + f'g - f'g = fg'$$
\end{proof}

Пример:
$$\int x^2 e^x \d x = x^2 e^x - \int 2x e^x \d x = x^2 e^x - 2 \int x e^x \d x = $$
$$ = x^2 e^x - 2\left(x e^x - \int e^x \d x\right) = x^2 e^x - 2x e^x + 2e^x + c$$

Есть термин <<берущеися>> интегралы. Это интегралы, выражаемые через элементарные функции. Их, вообще говоря, мало. К ним относятся рациональные функции (отношение многочленов), произведение тригинометрических функций, $x\sqrt{ax^2 + bx + c}$. Не берутся, например, 
$$\int e^{x^2} \d x$$
$$\int \frac{e^x}x \d x$$
$$\int \frac{\sin x}x \d x$$
$$\int \frac{\cos x}x\d x$$
$$\int \frac{\d x}{\ln x}$$
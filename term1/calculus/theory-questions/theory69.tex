\section{Теоремы Ферма и Ролля}
\begin{theorem}{Теорема Ферма}
$f\colon \left<a, b\right>$, $x_0 \in (a, b)$, $f$ дифференцируема в $x_0$, $x_0$~--- точка экстремума. Тогда
$$f'(x_0) = 0$$
\end{theorem}
\begin{proof}
Пусть $x > x_0$.
$$\lim_{x\ra x_0} \frac{f(x) - f(x_0)}{x - x_0} \geqslant 0$$
Пусть $x < x_0$.
$$\lim_{x\ra x_0} \frac{f(x) - f(x_0)}{x - x_0} \leqslant 0$$
Но тогда
$$f'(x_0) = 0$$
\end{proof}

\begin{theorem}{Теорема Ролля}
$f\colon [a, b] \in \R$, $f$ непрерывна, $f$ дифференцируема на $(a, b)$, $f(a) = f(b)$.
Тогда
$$\exists c\in(a, b)\colon f'(c) = 0$$
\end{theorem}
\begin{proof}
Если функция константна, то всё доказано. Иначе есть глобальный максимум и минимум, причём они не могут быть оба в концах.
\end{proof}

\begin{conseq}
Между корнями функции есть корень производной.
\end{conseq}

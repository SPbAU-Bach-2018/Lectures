\section{Верхняя и нижняя граница}

\begin{Def}
$A \subset \R$.\\
$x \in R $ --- верхняя граница $A$, если $$\forall a \in A: a \leqslant x$$
$x \in R $ --- нижняя граница $A$, если $$\forall a \in A: a \geqslant x$$
\end{Def}
\begin{Def}
$A$ ограничено сверху, если
$$\exists x \in R: x \text{~--- верхняя граница} A$$
$A$ ограничено снизу, если 
$$\exists x \in R: x \text{~--- нижняя граница} A$$
$A$ ограничено, если $A$ ограничено сверху и снизу.
\end{Def}
\begin{Rem}
Границ, если они есть, много.
\end{Rem}
\begin{Def} $A \subset \R$, $A$ ограничено сверху.
$x$~--- супремум $A$, если $x$ --- наименьшая из верхних границ.\\
\end{Def}
\begin{Def}
$A \subset \R$, $A$ ограничено снизу.
$x$~--- инфимум $A$, если $x$ --- наибольшая из нижних границ.
\end{Def}

Пример:
$$A = \left\{1, \frac12, \frac13, \frac14, \cdots\right\}$$
$$\sup A = 1, \inf A = 0$$

\begin{assertion}
$\N$ не ограничено сверху.
\end{assertion}
\begin{proof}
$x\text{~--- верхняя граница }\Ra \exists n \in \N: n > x$.
\end{proof}

\begin{theorem}{Существование точной границы}
$A \neq \emptyset$.
\begin{enumerate}
\item Если $A$ ограничено сверху, то $\exists x = \sup A$.
\item Если $A$ ограничено снизу, то $\exists x = \inf A$.
\end{enumerate}
\end{theorem}
Эта теорема равносильна аксиоме полноты.
\begin{proof}
\begin{enumerate}
\item $B$~--- множество всех верхних границ $A$.
$$\forall a \in A\: \forall b \in B\: a \leqslant b \Ra \exists c \in \R\colon \forall a \in A\: a \leqslant c \land \forall b \in B\: c \leqslant b \Ra \exists \sup A = c$$
\item Рассмотрим $ B = \{-a : a \in A\}$. Тогда $$\inf A = -\sup B$$
\end{enumerate}
\end{proof}

\begin{Rem}
Без аксиомы полноты это неверно. Рассмотрим $ A = \{x \in \Q : x^2 < 2\}, U = \Q$
\end{Rem}

\begin{theorem}{Свойство и признак точной границы}
\begin{enumerate}
\item $A$ ограничено сверху. Тогда $$b = \sup A \Lra (\forall a \in A\: a \leqslant b \land \forall \epsilon > 0\: \exists a \in A\colon a > b - \epsilon)$$
\item $A$ ограничено снизу. Тогда $$c = \inf A \Lra (\forall a \in A\: a \geqslant c \land \forall \epsilon > 0\: \exists a \in A\colon a < c + \epsilon)$$
\end{enumerate}
\end{theorem}
\begin{proof}
$$b = \sup A \Lra (b \text{~--- верхняя граница } A \land \forall \epsilon > 0\: b - \epsilon\text{~--- не верхняя граница}) \Lra $$ 
$$ \Lra (\forall a \in A\: a \leqslant b \land \forall \epsilon > 0\: \exists a \in A\colon a > b - \epsilon)$$
\end{proof}

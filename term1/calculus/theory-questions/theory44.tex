\section{Арифметические действия с пределами}


\begin{theorem}{Арифметические свойства предела функции.}
$f, g\colon E \subset \ra \R^d$, $\lambda\colon E \ra \R$, $a$ предельная точка $E$.
\begin{enumerate}
\item $\lim\limits{x\ra a} (f(x) + g(x)) = f_0 + g_0$
\item $\lim\limits{x\ra a} (\lambda(x)  g(x)) = \lambda_0 g_0$
\item $\lim\limits{x\ra a} (f(x) - g(x)) = f_0 - g_0$
\item $\lim\limits{x\ra a} \left\|f(x)\right\| = \left\|f_0\right\|$
\item $\lim\limits{x\ra a} \left<f(x), g(x)\right> = \left<f_0, g_0\right>$
\end{enumerate}
\end{theorem}
\begin{proof}
Возьмём любые сходящиеся к $a$ последовательности. Для них будет справедлива теорема об арифметических действиях с пределами последовательности.
\end{proof}

\begin{theorem}{Арифметические свойства предела функции.}
$f, g\colon E \subset \ra \R$, $a$ предельная точка $E$.
\begin{enumerate}
\item $\lim\limits{x\ra a} (f(x) \pm g(x)) = f_0 \pm g_0$
\item $\lim\limits{x\ra a} (f(x) g(x)) = f_0 g_0$
\item $\lim\limits{x\ra a} \left|f(x)\right| = \left|f_0\right|$
\item $\lim\limits{x\ra a} \cfrac{f(x)}{g(x)}=\cfrac{f_0}{g_0}$
\end{enumerate}
\end{theorem}
\begin{proof}
Аналогично.
\end{proof}

\begin{Rem}
Арифметические свойства расширяются на бесконечности.
\end{Rem}
\section{Диаметр множества}
\begin{Def}
Диаметр множеста:
$$\diam A = \sup \rho(x, y)$$
\end{Def}

\begin{theorem}{Свойства диаметра}
\begin{enumerate}
\item $\diam E = \diam \cl E$
\item $K_1 \supset K_2 \supset K_3 \cdots $~--- последовательность вложенных компактов, $\diam K_n \ra 0$. Тогда $\bigcap K_i$~--- одноточечное.
\end{enumerate}
\end{theorem}
\begin{proof}
\begin{enumerate}
\item
$$E \subset \cl E \Ra \diam E \leqslant \diam \cl E$$
$$d = \diam \cl E = \sup \rho(x, y)$$
$$\forall \epsilon > 0\: \exists x_0, y_0\colon \rho(x_0, y_0) > d - \epsilon$$
$$x_0 \in \cl E \Ra \exists x_1 \in E\colon \rho (x_0, x_1) < \epsilon$$ 
$$y_0 \in \cl E \Ra \exists y_1 \in E\colon \rho (y_0, y_1) < \epsilon$$ 
Тогда
$$\rho(x_1, y_1) + 2\epsilon > \rho(x_0, x_1) + \rho(x_1, y_1) + \rho(y_1, y_0) \geqslant \rho(x_0, y_0) > d - \epsilon$$
$$\rho(x_1, y_1) > \rho(x_0, y_0) - 3\epsilon$$
Устремив $\epsilon \ra 0$, получим
$$\diam E \geqslant \diam \cl E$$
\item Пусть в пересечение лежат две точки, но тогда диаметр для любого n хотя бы $\rho(a, b)$. Противоречие.
\end{enumerate}
\end{proof}
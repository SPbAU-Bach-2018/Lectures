\section{Пределы функций}

\begin{Def}
$(X, \rho_x)$ и $(Y, \rho_y)$~--- метрические пространства. $E \subset X$, $a$~--- предельная точка $E$. $f\colon E \ra Y$.
Тогда говорят, что
$$\lim_{x \ra a} f(x) = b$$
если $b \in Y$ и
$$\forall \epsilon>0\: \exists \delta > 0\colon \forall x \in \dot B_\delta(a)\: \cap E \Ra f(x) \in B_\epsilon (b)$$
или, что то же самое
$$\forall \epsilon>0\: \exists \delta > 0\colon \forall x \in E \: (x \ne a \land \rho(x, a) < \delta) \Ra \rho(f(x), b) < \epsilon$$
\end{Def}

\begin{Rem}
Для бесконечности на $\R$ есть частные случаи.
\end{Rem}

\begin{Def}
По Гейне,
$$\lim_{x \ra a} f(x) = b \Lra \forall \{x_n\}\subset E\colon x_n \ne a\: \lim_{n\ra \infty} x_n = a \Ra \lim_{n \ra \infty} f(x_n) = b$$
\end{Def}


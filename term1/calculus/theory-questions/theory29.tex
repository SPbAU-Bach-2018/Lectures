\section{Секвенциальная компактность}

\begin{theorem}{Компактность в $\R^d$}
Следующее в $\R^d$ равносильно:
\begin{enumerate}
\item Компактно
\item Замкнуто и ограниченно
\item Для любой последовательности в множестве можно выбрать подпоследовательность, сходящуюсю к некоторой точке множества (\textit{секвенциально компактно})
\end{enumerate}
\end{theorem}
\begin{proof}
$2 \Ra 1$: $К$ ограниченно, значит можно его ограничить кубом, значит оно подмножество компактного и замкнуто, значит компактно.

$1 \Ra 3$:
Возьмём последовательность $\{x_n\}\lrh E$ элементов множества $F$. Если множество элементов $E$ конечно, то какой-то элемент повторился бесконечно. Возьмём новую стационарную последовательность ровно из этого элемента, имеющую предел. Если же оно бесконечно, докажем, что у него есть предельная точка.

Пусть ни одна точка не предельна. Значит 
$$\forall x \in X\: \exists r_x > 0\colon \dot B_{r_x}(x) \cap F = \emptyset$$
Но тогда возьмём покрытие
$$\bigcup_{x\in X} B_{r_x} (x)$$
В нём есть конечное подпокрытие. Возьмём его
$$\bigcup_{i=1}^k \dot B_{r_{y_i}} \supset K \supset E$$
Но также
$$\bigcup \dot B_{r_{y_i}} \cap E = \varnothing$$
Значит 
$$E \subset \bigcup_{i=1}^k \{y_i\}$$
Получили, что $E$ конечное. 

Таким образом предельная точка существует, а значит можно выбрать подпоследовательность можно.

$3 \Ra 2$:
Пусть $K$ не замкнуто. Возьмём предельную точку, которой нет в $K$. Значит есть последовательность, сходящаяся к ней. Из неё нельзя выбрать подпоследовательность, сходящуюся к элементу $K$.

Пусть $K$ не ограничено. Значит есть точка, не лежащая в данном шарике.
$$K \nsubset B_1(a) \Ra \exists x_1\colon \rho(x_1, a) > 1$$
$$K \nsubset B_{\rho(a, x_1) + 1}(a) \Ra \exists x_2\colon \rho(x_2, a) > \rho(x_1, a) + 1$$
$$ \vdots $$
Рассмотрим сходящуюся подпоследовательность. Она ограничена шариком радиуса $R$. Но
$$\rho(a, x_n) > \rho(a, x_{n-1}) + 1 > \cdots > n$$
$$R > \rho\left(b, x_{n_k}\right) > \rho\left(a, x_{n_k}\right) + \rho(a, b) > n_k + \rho(a, b) \ra \infty$$
Значит $K$ ограниченно.  
\end{proof}

\begin{Rem}
$1\Ra 3; 3\Ra 2; 1 \Ra 2$ справедливы для всех пространств. $2 \Ra 1$ ломается, например, на $\R$ с дискретной метрикой.
\end{Rem}

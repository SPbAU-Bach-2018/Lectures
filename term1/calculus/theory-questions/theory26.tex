\section{Теорема о вложенных параллелепипедах}

\begin{Def}
Параллелепипедом на $\R^d$ и $a, b \in \R^d$ назовём
$$[a, b] = \left\{x \in \R^d \mid \forall i=1..d\: a_i \leqslant x_i \leqslant b_i\right\} \text{ (закрытый)}$$
$$(a, b) = \left\{x \in \R^d \mid \forall i=1..d\: a_i < x_i < b_i\right\} \text{ (открытый)}$$
\end{Def}

\begin{theorem}{О вложенных параллелепипедах}
$P_1 \supset P_2 \supset P_3 \supset \ldots$ имеют непустое пересечение.
\end{theorem}
\begin{proof}
Применим теорему о вложенных отрезках по каждой координате.
\end{proof}

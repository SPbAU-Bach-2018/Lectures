\section{Замена переменной}
\begin{theorem}{Замена переменной в неопределённом интеграле}
$f\colon \left<a, b\right> \ra \R$ непрерывна, $\phi\colon \left<c, d\right> \ra \left<a, b\right>$ непрерывно дифференцируема.
$$\int f(\phi(t)) \phi'(t) \d t = F(\phi(t)) + c$$
\end{theorem}
\begin{proof}
$$\left(F(\phi(t)) + c\right)' = (F(\phi(t)))' = F'(\phi(t)) \phi'(t) = f(\phi(t)) \phi'(t)$$
\end{proof}
\begin{conseq}
$$\int f(\alpha x + \beta) \d x = \frac1{\alpha}F(\alpha x + \beta) + c$$
\end{conseq}

Примеры:
$$\int \frac{\ln^2 x}{x}\d x$$
$f = x^2, \phi = \ln x$
$$\int \frac{\ln^2 x}{x} \d x = \int (\ln x)^2 (\ln x)' \d x = \frac{(\ln x)^3}3 + c = \frac{\ln^3 x}3 + c$$
$a>0$
$$\int \frac{\d x}{x^2 + a^2} = \frac1{a^2} \int \frac{\d x}{\left(\frac{x}a\right)^2 + 1} = \frac1{a^2} \frac1{\frac1a} \arctg \frac{x}a + c=$$
$$= \frac1a \arctg \frac{x}a$$
$f=\frac1{x^2+1}$


\section{Предел монотонной последовательности}

\begin{Def}
$(x_n)$ нестрого монотонно возрастает, если $$x_1 \leqslant x_2 \leqslant x_3 \leqslant \cdots$$

$(x_n)$ строго монотонно возрастает, если $$x_1 < x_2 < x_3 < \cdots$$

$(x_n)$ нестрого монотонно убывает, если $$x_1 \geqslant x_2 \geqslant x_3 \geqslant \cdots$$

$(x_n)$ строго монотонно убывает, если $$x_1 > x_2 > x_3 > \cdots$$
\end{Def}

\begin{theorem}{Теорема Вейерштрасса}
Монотонная последовательность ограниченна тогда и только тогда, когда имеет предел.
\end{theorem}
\begin{proof}
$\La$: Очевидно.

$\Ra$: Пусть $(x_n)$ возрастает. Она ограниченна, значит есть супремум. Докажем, что это и есть предел. Возьмём $\epsilon > 0$.
$$a = \sup \{x_n\} \Ra \exists x_k\colon x_k > x - \epsilon \Ra a - \epsilon < x_k \leqslant x_{k+1} \leqslant \ldots \leqslant a$$
Тогда $$\forall n \geqslant k\: |x_n - a| < \epsilon$$
\end{proof}
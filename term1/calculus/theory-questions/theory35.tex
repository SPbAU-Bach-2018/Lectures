\section{Характеристика верхних и нижних пределов с помощью N и eps}
\begin{theorem}{Определение верхнего и нижнего предела через $N$ и $\epsilon$}
\begin{enumerate}
\item $$a = \varliminf x_n \Lra \begin{cases}\forall \epsilon > 0\: \exists N\colon \forall n>N\: x_n > a - \epsilon \\ \forall \epsilon > 0\: \forall N\colon \exists n>N\: x_n < a + \epsilon\end{cases}$$
\item $$a = \varlimsup x_n \Lra \begin{cases}\forall \epsilon > 0\: \forall N\colon \exists n>N\: x_n > a - \epsilon \\ \forall \epsilon > 0\: \exists N\colon \forall n>N\: x_n < a + \epsilon\end{cases}$$
\end{enumerate}
\end{theorem}
\begin{proof}
\begin{enumerate}
\item Запишем в терминах $y_n$:
$$\forall \epsilon > 0\: \exists N\colon \inf_{n>N} > a-\epsilon ; \forall \epsilon > 0\: \exists N\colon \inf_{n>N} < a+\epsilon$$
Уже видно, что эти условия и задают предел.
\item Аналогично.
\end{enumerate}
\end{proof}

\begin{theorem}{О предельном переходе в неравенстве}
$$a_n \leqslant b_n \Ra \begin{cases}\varliminf a_n \leqslant \varliminf b_n \\ \varlimsup a_n \leqslant \varlimsup b_n\end{cases}$$
\end{theorem}                                                                                                                       
\begin{proof}
Просто сводим к пределам инфимумов. 
\end{proof}

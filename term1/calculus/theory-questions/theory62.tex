\section{Геометрический смысл производной}

Если рассмотреть график непрерывной функции 
$$y = f(x)$$
то в каждой точке $x_0$, где функция непрерывна, можно рассмотреть касательную к её графику
$$y = kx + b$$

\begin{center}
\def\svgwidth{6.0cm}
\input{theory62-tangent.pdf_tex}
\end{center}

Давайте посчитаем угловой коэффициент касательной $k$.
$$k = \lim_{x\ra x_0} \frac{f(x) - f(x_0)}{x - x_0} = f'(x_0)$$

Таким образом, производная равна тангенсу угла наклона касательной к графику функции в соотвествующей точке.

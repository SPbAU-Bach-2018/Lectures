\section{Вещественные числа}
\begin{Def}
Множество вещественных чисел можно определить как множество, на котором есть операции $+$ и $\times$, причём:
\begin{enumerate}
\item Коммутативность $\forall a, b\: a + b = b + a; a \times b = b \times a$
\item Ассоциативность $\forall a, b, c\: a + (b + c) = (a + b) + c; a \times (b \times c) = (a \times b) \times c$
\item Нейтральный элемент
$\exists o\colon \forall a\: a + o = a; \exists e\colon \forall a\:a \times e = a; o \ne e$
\item Обратный элемент
$\forall a\: \exists{-a}\colon a + -a = o; \forall a \ne o\: \exists a^{-1}: a \times a^{-1} = a$
\item Дистрибутивность $\forall a, b, c\: a \times (b + c) = (a \times b) + (a \times c)$
\end{enumerate}
Кроме того, есть отношения $\leqslant$ (и аналогично $<$, также определены обратные):
\begin{enumerate}
\item Рефлексивно
\item Антисимметрично
\item Транзитивно
\item Любые два элемента сравнимы
\item $\forall a, b, c\: a \leqslant b => a + c \leqslant b + c$
\item $\forall a, b\: a > 0 \land b \geqslant 0 \Ra ab \geqslant 0$
\end{enumerate}
\end{Def}

Также выполнена аксиома полноты: $A, B \subset \R$, $A \cup B \ne \varnothing$, $ \forall a \in A\: \forall b \in B\: a \leqslant b $. Тогда 
$$\exists c \in \R\colon \forall a \in A\: a \leqslant c \land \forall b \in B\: c \leqslant b$$

\begin{Rem}
На $\Q$ аксиома не выполняется: 
$$A = \left\{r \in \Q \mid r^2 < 2\right\}; B = \left\{r \in \Q_{+}\mid r ^ 2 > 2\right\}; c = \sqrt{2} \notin \Q$$
\end{Rem}
\begin{theorem}{Принцип Архимеда}
Пусть $ x, y \in \R, y > 0 $.
Тогда $$ \exists n \in \N: x < ny$$
\end{theorem}
\begin{proof}
$$ A \lrh \left\{u \in \R \mid \exists n \in \N: u < ny\right\}; y \in A$$
Пусть $A \ne \R$. Тогда $B \lrh \R - A \ne \varnothing$.
Рассмотрим $a \in A; b \in B$.
$$b < a \Ra b < a < ny \Ra b \in A \text{ --- противоречие}$$
Таким образом 
$$\forall a \in A\: \forall b \in B\: a \leqslant b$$
Тогда 
$$\exists c \in \R\colon \forall a \in A\: a \leqslant c \land \forall b \in B\: c \leqslant b$$
$$c \in A \Ra c + y \in A \Ra c > c + y \Ra y < 0 \text{~--- противоречие}$$
Тогда $c \in B$.
Пусть $c - y \notin B$, тогда 
$$c - y \in A \Ra c - y < ny \Ra c < (n + 1)y \Ra c \in A \text{~--- противоречие}$$
Значит 
$$c - y \in B \Ra c - y \geqslant c \Ra y \leqslant 0 \text{~--- противоречие}$$
Таким образом $A = \R$
\end{proof}

\begin{conseq}
$$\forall \epsilon > 0\: \exists n \in \N\colon \frac{1}n < \epsilon$$
\end{conseq}
\begin{proof}
Рассмотрим $x=1, y=\epsilon$
\end{proof}

\begin{conseq}
$x, y \in \R, x < y$
$$\exists r \in \Q: x < r < y$$
\end{conseq}
\begin{proof} 
$$y - x > 0 \Ra \exists n \in \N\colon \frac1n < y - x$$
Покажем, что $ \exists m \in \Z\colon m \leqslant nx < m + 1$. Вообще говоря, $m \eqDef \lfloor nx \rfloor$.
$$M \lrh \{m \in \Z\mid m \leqslant nx\}$$
$$x \geqslant 0 \Ra M \neq \varnothing$$
$$x < 0 \Ra \exists \tilde m \in \N\colon \tilde m-1 > n(-x) \Ra -\tilde m \in M \Ra M \neq \varnothing$$
Рассмторим $ y = 1; x = nx; y > 0$. По принципу Архимеда 
$$ \exists k \in \N\colon k > nx $$
Тогда 
$$\forall m \in M\: m < k \Ra \exists m = \max M\colon m \leqslant nx < m + 1$$
$$m \leqslant nx < m + 1 \Ra \frac{m}n \leqslant x \leqslant \frac{m + 1}n$$
Осталось проверить $\frac{m+1}n < y$.
$$\frac{m}n \leqslant x \land \frac1n < y - x \Ra \frac{m+1}n < y$$
\end{proof}

\begin{conseq}
$x, y \in \R$, $x < y$. 
$$\exists z \in \R \setminus \Q: x < z < y$$
\end{conseq}
\begin{proof}
$$\sqrt{2} \in \R - \Q$$
$$x < y \Ra x - \sqrt{2} < y - \sqrt{2} \Ra \exists r \in \Q: x - \sqrt{2} < r < y - \sqrt{2} \Ra $$
$$\Ra \exists z = r + \sqrt{2}: z \in \R - \Q: x < z < y$$
\end{proof}

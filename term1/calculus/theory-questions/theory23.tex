\section{Компактность}

\begin{Def}
Множество $A$ имеет покрытие множествами $B_\alpha$, если $A \subset \bigcup_{\alpha \in A} B_\alpha$.
\end{Def}
\begin{Def}
Множество $A$ имеет открытое покрытие открытыми множествами $B_\alpha$, если $A \subset \bigcup_{\alpha \in A} B_\alpha$.
\end{Def}
\begin{Def}
Множество $A$ компактно, если из любого его открытого покрытия можно выбрать конечное подкокрытие.
$$\forall B_\alpha\colon K \subset \bigcup_{\alpha \in A} B_\alpha\; \exists \alpha_1, \alpha_2, \ldots, \alpha_n\colon K\subset \bigcup_{i=1}^{n} B_{\alpha_i}$$
\end{Def}

\begin{theorem}{Компактность и подпространства}
Пусть $(X, \rho)$~--- метрическое пространство, $K \subset Y \subset X$. Тогда 
$$K\text{ компактно в } (X, \rho) \Lra K\text{ компактно в } (Y, \rho)$$
\end{theorem}
\begin{proof}
$\Ra$: Пусть $B_\alpha$~--- открытое в $Y$, что 
$$K \subset \bigcup_{\alpha \in A} B_\alpha = \bigcup_{\alpha \in A} (G_\alpha \cap Y) \subset \bigcup_{\alpha \in A} G_\alpha$$
Тогда можно заменить покрытие в $Y$ покрытием соотвествующими множествами в $X$, выбрать конечное подпокрытие, а потом перейти обратно в $Y$.

$\La$: Пусть $K = \bigcup_{\alpha \in I} G_\alpha$. Тогда 
$$K = K \cap Y \subset \left(\bigcup_{\alpha \in I} G_\alpha\right) \cap Y = \bigcup_{\alpha \in I} \left(G_\alpha \cap Y\right)$$
Получим покрытие в пространстве $Y$, в нём есть конечное подпокрытие. Выберем соответствующие шарики из $X$.
\end{proof}

\begin{Rem}
Например, $(0, 1)$ не компактно. Например, из $$\bigcup_{i=2}^\infty \left(\frac1i, 1\right)$$ не выбрать.
\end{Rem}
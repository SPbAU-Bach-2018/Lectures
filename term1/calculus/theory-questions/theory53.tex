\section{Теоремы о непрерывности обратного отображения и о непрерывности монотонной функции}

\begin{theorem}{}
$f\colon K \ra Y$ непрерывно на компакте $K$ и биективно. Тогда $f^{-1}\colon Y \ra K$ непрерывно.
\end{theorem}
\begin{proof}
Надо проверить, что для $f^{-1}$ прообраз открытого множества~--- открытое. То есть надо проверить для $f$, что образ открытого~--- открытое.

Берем $G \subset K$~--- открытое. Тогда $K \setminus G$~--- замкнутое подмножество K. Тогда $K \setminus G$~--- компакт. Тогда $f(K \setminus G)$~--- компакт, в том числе замкнутое.
Тогда $f(G)$~--- открытое.
\end{proof}

\begin{conseq}
$f\colon [a, b] \ra \R$ строго монотонна и $f$ непрерывна на $[a, b]$. Тогда $f^{-1}$ непрерывна на множестве задания.
\end{conseq}
\begin{proof}
$[a, b] = K$~--- компакт.
Так как $f$ строго монотонна, то $f$~--- биекция между $[a, b]$ и $f([a,b])$.
\end{proof}

\begin{conseq}
$f\colon \left<a,b\right> \to \R$ строго монотона и непрерывна. Тогда $f^{-1}$ непрерывна на множестве задания.
\end{conseq}
\begin{proof}
$$ Y = f(\left<a,b\right>) $$
$$f^{-1}\colon Y \to \R$$
Надо доказать непрерывность. Берем $c \in Y$. Тогда 
$$\exists x_0\in\left<a, b\right>\colon c = f(x_0)$$
Возьмем $x_0 \in [\alpha, \beta] \subset \left<a, b\right>$
$$g = f\mid_{[\alpha, \beta]}\colon [\alpha, \beta] \to \R$$  
Применяем к $g$ следствие 1.
$$\forall \epsilon > 0\: \exists \delta > 0\colon \forall y \in B_{\delta}(c) \cap f([\alpha, \beta])\: g^{-1}(y) \in B_{\epsilon}(g^{-1}(c))$$
$f\colon X \to Y$ непрерывно на $X$.
$$\forall a \in X\: \forall \epsilon > 0\: \exists \delta > 0\colon \forall x \in B_{\delta}(a)\: f(x) \in B_{\epsilon}(f(a))$$
\end{proof}


\section{Фундамитальные последовательности}

\begin{Def}
Последовательность называется фундаментальной, если 
$$\forall \epsilon > 0\: \exists N\colon \forall n,m > N\: \rho(n, m) < \epsilon$$
\end{Def}
\begin{Rem}
$E \lrh \{x_i\}_{i=n}^{\infty}$
$$\{x_n\}\text{ фундаментальная} \Lra \diam E \ra 0$$
\end{Rem}

Свойства фундаментальных последовательностей:
\begin{enumerate}
\item Ограничена
\item Если есть сходящаяся подпоследовательность, то она сходится.
\begin{proof}
$$\forall \epsilon > 0\: \exists K\colon \forall k > K\: \rho(x_{n_k}, a) < \epsilon$$
$$\forall \epsilon > 0\: \exists N\colon \forall n,m > K\: \rho(x_n, x_m) < \epsilon$$

Т.о.
$$\exists n_k > M = \max\{N, K\}\colon \forall n > n_k \rho(x_n, a) \leqslant \rho(x_{n_k}, a) + \rho(x_{n_k}, x_k) < 2\epsilon$$
\end{proof}
\end{enumerate}

\begin{Def}
Пространство называют полным, если любая фундаментальная последовательность имеет предел.
\end{Def}   

\section{Бинарные отношения}

\begin{Def}
Отношение на множествах $A$ и $B$ --- произвольное подмножество их декартова произведения.
$$ a \mathop{R} b \LraDef (a, b) \in R $$
\end{Def}
\begin{Def}
Область определения отношения 
$$ \beta_R = dom_R = \{a \in A \mid \exists b \in B\colon (a,b) \in R\} $$
\end{Def}
\begin{Def}
Обсласть значения отношения 
$$ \rho_R = ran_R =\{b \in B \mid \exists a \in A\colon (a, b) \in R\}$$
\end{Def}
\begin{Def}
Обратное отношение
$$R^{-1} \colon \beta_{R^{-1}} = \rho_R; \rho_{R^{-1}} = \beta_R; b \mathop{R^{-1}} a \LraDef a \mathop{R} b$$
\end{Def}
\begin{Def}
Композиция отношений
$$ R_1\colon A \ra B; R_2\colon B \ra C $$
$$ R_1 \circ R_2 = \{(a, c) \mid a \mathop{R_1} b \land b \mathop{R_2} c\} $$
Про значок ~--- его использовать не будем
\end{Def}

Пример композиции: $<\colon \N \ra \N$. $$< \circ < = \{(a, b) \mid b - a \geqslant 2\}$$

\begin{Def}
Функция (отображение) ~--- такое отношение, что первый ключ уникален.
$$f\colon A \ra B$$
$$ a \mathop{f} b_1 \land a \mathop{f} b_2 \Ra b_1 = b_2 $$
$$ a \mathop{f} b \LraDef f(a) = b $$
$$ A = \beta_f \quad \text{($A$~--- область определения)}$$
\end{Def}

\begin{Def}Свойтва отображеий:
\begin{enumerate}
\item Рефлексивность $a \mathop{R} a$
\item Cимметричность $a \mathop{R} b \Lra b \mathop{R} a$
\item Транзитивность $a \mathop{R} b \land b \mathop{R} c \Ra a \mathop{R} c$
\item Иррефлексивность $\lnot a \mathop{R} a$
\item Антисимметричность $a \mathop{R} b \land b \mathop{R} a \Ra a = b$
\end{enumerate}
\end{Def}

Примеры:
\begin{itemize} 
\item $=$: 1, 2, 3, 5
\item $\emod{5}$: 1, 2, 3
\item $\leqslant$: 1, 3, 5
\item $<$: 3, 4, 5
\item $\subset$: 1, 3, 5
\end{itemize}
\section{Внутренние точки и внутренность множества}
\begin{Def}
$x \in A$ --- внутренняя точка $A$, если $\exists r > 0\colon B_r(x) \in A$
\end{Def}
\begin{Rem}
$x$ --- внутренняя точка $A$ эквивалентно тому, что в $A$ содержится некое открытое множество, содержащее x.
\end{Rem}
\begin{Def}
Внутренность множества $A$:
$$A^0 = \Int A \eqDef \bigcup_{\substack{G \text{ открыто} \\ G \subset A}} G$$
\end{Def}

Свойства:
\begin{enumerate}
\item $\Int A \subset A$
\item $\Int A$ --- множество всех внутренних точек.
\item $\Int A$ открыто.
\item $A \text{ открыто} \Lra A = \Int A$
\item $A \subset B \Ra \Int A \subset \Int B$
\item $\Int (A \cap B) = \Int A \cap \Int B$
\item $\Int \Int A = \Int A$
\end{enumerate}
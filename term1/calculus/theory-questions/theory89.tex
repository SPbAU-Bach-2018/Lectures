
\section{Неопределённый интеграл}

\begin{Def}
$f\colon \left<a, b\right> \ra \R$. Функция $F\colon \left<a, b\right> \ra \R$ называется первообразной $f$, если
$$F' = f$$
\end{Def}

Не для всех $f$ существует $F$. Например,
$$f(x) = \begin{cases}1 & x \geqslant 0 \\ 0 & x < 0\end{cases}$$
\begin{proof}
Пусть есть $F' = f$. Тогда по теореме Дарбу
$$\forall a, b \in (-1, 1), c \in (F'(a), F'(b))\; \exists c \in (a, b)\colon F'(c) = C$$
\end{proof}

\begin{theorem}{О существовании первообразной}
Для любой непрерывной $f\colon \left<a, b\right> \ra \R$ есть первообразная $F$.
\end{theorem}
Докажем в следующем семестре.

\begin{theorem}{}
$f, F\colon \left<a, b\right> \ra \R$, $F$~--- первоообразная. Тогда
\begin{enumerate}
\item $F + c$~$(c \in \R)$~--- также первообразная.
\item $\Phi$~--- первообразная $\iff$ $\Phi = F + c$.
\end{enumerate}
\end{theorem}
\begin{proof}
$$(F+c)' = F' + 0 = f$$
Рассмотрим $G = \Phi - F$. Она дифференцируема и
$$G' = (\Phi - F)' = \Phi' - F' = f - f = 0$$
Но тогда $$G = const$$
\end{proof}

\begin{Def}
Неопределённым интегралом функции $f$ называется множество её первообразных.
$$\int f(x)\d x$$
\end{Def}
Пока стоит воспринимать все символы интеграла как некоторые <<скобки>>.

Если есть некоторая первообразная $F$, то
$$\int f(x)\d x = \left\{F(x) + c \mid c \in \R\right\}$$
Тот же смысл имеют записи
$$\int f(x)\d x = F(x) + c$$
$$\int f\d x = F + c$$

Для того, чтобы найти неопределённый интеграл, достаточно найти какую-то первообразную, а для проверки первообразной достаточно взять от неё производную.

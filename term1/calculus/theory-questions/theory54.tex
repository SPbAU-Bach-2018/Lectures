\section{Равномерная непрерывность на функции. Теорема Кантора}

\begin{Def}
$f\colon X \to Y$ равномерно непрерывна, если
$$\forall \epsilon > 0\: \exists \delta > 0\colon \forall x, y \in X \land \rho(x, y) < \delta\: \rho(f(x), f(y)) < \epsilon$$
\end{Def}

\begin{theorem}{Теорема Кантора}
$f\colon K \to Y$, $K$~--- компакт, $f$ непрерывна на $K$. Тогда $f$ равномерно непрерывно.
\end{theorem}

\begin{proof}
От противного. Пусть

$$\exists \epsilon>0\colon \exists x_n, \tilde x_n \in K\colon \rho(x_n, \tilde x_n) < \frac{1}{n} \land \rho(f(x_n), f(\tilde x_n)) \ge \epsilon$$

$x_n$, $\tilde x_n$ последовательность точек из $K$. Возьмём из $x_n$ сходящуюся подпоследовательность $x_{n_{k}}$, $x_{n_k} \to a \in K$. Тогда
$\tilde x_{n_k} \to a$, так как 
$$\rho(x_{n_k}, \tilde x_{n_k}) < \frac{1}{n_k} \to 0$$
$f$ непрерывно в точке $a$
$$\exists \delta > 0\colon \forall x \in B_{\delta}(a)\: f(x) \in B_{\frac{\epsilon}{2}}(f(a))$$

Начиная с какого-то $N$ 
$$x_{n_k}, \tilde x_{n_k} \in B_{\delta}(a) \Ra f(x_{n_k}), f(\tilde x_{n_k}) \in B_{\frac{\epsilon}2}(f(a))$$

$$\Ra \rho(f(x_{n_k}), f(\tilde x_{n_k})) < \epsilon \quad \text{противоречие}$$
\end{proof}

\begin{conseq}
Непрерывная на $[a, b]$ функция равномерно непрерывна.
\end{conseq}

\section{Характеристика непрерывности в терминах прообразов}

\begin{theorem}{}
$f\colon X \ra Y$. $f$ непрерывна во всех точках $\Lra$ прообраз любого открытого множества открыт.
\end{theorem}

\begin{proof}
\begin{enumerate}
\item $\Ra$:

$G \subset Y$, $G$ открытое. Надо доказать, что  $f^{-1}(G)$~--- открытое. Возьмем $a \in f^{-1}(G)$. Надо доказать, что существует шар с центром в точке $a$, содержащийся в $f^{-1}(G)$.

$$f(a) \in G \Ra \exists B_{\epsilon}(f(a)) \subset G$$
Знаем,что $f$ непрерывна в точке $а$
$$\exists \delta > 0\colon \forall x \in B_{\delta}(a)\: f(x) \in B_{\epsilon}(f(a)) \subset G$$
То есть
$$\forall x \in B_{\delta}(a)\: f(x) \in G $$
То есть 
$$B_{\delta}(a) \subset f^{-1}(G) $$

\item $\La$
Зафиксируем $a \in x$. Надо доказать, что 
$$\forall \epsilon > 0\: \exists \delta > 0\colon \forall x \in B_{\delta}(a)\: f(x) \in B_{\epsilon}(f(a))$$

Возьмем $B_{\epsilon}(f(a))$ ~--- открытое множество, $a \in f^{-1}(B_{\epsilon}(f(a)))$ ~--- открытое, поэтому 

$$\exists B_{\delta}(a) \subset f^{-1}(B_{\epsilon}(f(a))) \Ra f(B_{\delta}(a)) \subset B_{\epsilon}(f(a))$$
\end{enumerate}
\end{proof}
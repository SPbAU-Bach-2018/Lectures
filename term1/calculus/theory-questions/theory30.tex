\section{Теорема Больцано-Вейерштрасса и другие следствия}
\begin{conseq}
В $\R^d$ компактность $K$ равносильна наличию предельной точки для любого подмножества.  
\end{conseq}
\begin{proof}
В одну сторону просто по теореме.
Обратно: возьмём часть доказательства, объясняющее взятие подпоследовательности.
\end{proof}

\begin{conseq}{Теорема Больцано-Вейерштрасса.}
Из любой ограниченной последовательности в $\R^d$ можно выбрать сходящуюся подпоследовательность.
\end{conseq}
\begin{proof}
Множество значений ограниченно, значит его замыкание компактно, значит в компактном есть сходящаяся подпоследовательность.
\end{proof}

\begin{conseq}
В любой последовательности в $\R$ есть сходящаяся в $\bar \R$ подпоследовательность.
\end{conseq}
\begin{proof}
Если ограничена, то см. предыдущее. Иначе она стремится к бесконечности. Тогда выберем бесконечную подпоследовательность, стремящуюся к бесконечности. В ней бесконечное число положительных или бесконечное число отрицательных.
\end{proof}
\section{Бесконечно малые и большие}

\begin{Def}
$$\lim x_n = +\infty \LraDef \forall E\; \exists N\colon \forall n > N\; x_n > E$$
$$\lim x_n = -\infty \LraDef \forall E\; \exists N\colon \forall n > N\; x_n < E$$
$$\lim x_n = \infty \LraDef \forall E\; \exists N\colon \forall n > N\; \left|x_n\right| > E$$
\end{Def}
\begin{Rem}
$$\left[\begin{array}{ll}\lim x_n = +\infty\\\lim x_n = -\infty\end{array}\right.\Ra \lim x_n = \infty$$
Также заметим, что обратное неверно ($x_n = (-1)^n n$).
\end{Rem}

\begin{Rem}
$\lim x_n = \infty \Ra x_n\text{ неограниченна}$
\end{Rem}
\begin{Rem}
Единтсвенность предела справедлива и расширенная на $\pm \infty$.
\end{Rem}
\begin{Rem}
Теорема о двух миллиционерах справедлива и для бесконечно больших.
\end{Rem}

\begin{Rem}
${\bar\R} = \R \cup \{+\infty, -\infty\}$
\begin{enumerate}
\item $\pm c+\pm\infty = \pm\infty$
\item $\pm c-\pm\infty = \mp\infty$
\item $c>0\colon \pm \infty \times c = \pm \infty$
\item $c<0\colon \pm \infty \times c = \mp \infty$
\item $c>0\colon \frac{\pm \infty}{c} = \pm \infty$
\item $c<0\colon \frac{\pm \infty}{c} = \mp \infty$
\item $\frac{c}{\pm \infty} = 0$
\item $(+\infty) + (+\infty) = +\infty$
\item $(+\infty) - (-\infty) = +\infty$
\item $(-\infty) + (-\infty) = -\infty$
\item $(-\infty) - (+\infty) = -\infty$
\item $\pm \infty \times (+ \infty) = \pm \infty$
\item $\pm \infty \times (- \infty) = \mp \infty$
\end{enumerate}
\end{Rem}

\begin{Def}
Последовательность называют бесконечно большой, если её предел бесконечнен.
\end{Def}
\begin{Def}
Последовательность называют бесконечно малой, если её предел равен нулю.
\end{Def}
\begin{theorem}{Достаточное условие экстремума}
$f\colon \left<a, b\right> \ra R$, $x_0 \in (a, b)$, $f$ непрерывна на $(x_0-\delta, x_0+\delta)$$f$ дифференцируема на $(x_0-\delta, x_0)\cup(x_0+\delta)$. Тогда
\begin{itemize}
\item $f'((x_0-\delta, x_0)) > 0 \land f'((x_0, x_0 + \delta)) < 0 \Ra x_0\text{~--- точка максимума}$
\item $f'((x_0-\delta, x_0)) < 0 \land f'((x_0, x_0 + \delta)) > 0 \Ra x_0\text{~--- точка минимума}$
\end{itemize}
\end{theorem}
\begin{proof}
$$f'((x_0-\delta, x_0)) > 0 \Ra f\text{ возрастает на } (x_0-\delta, x_0) \Ra f(x_0) > f((x_0-\delta, x_0))$$
$$f'((x_0, x_0+\delta)) < 0 \Ra f\text{ убывает на } (x_0, x_0+\delta) \Ra f(x_0) > f((x_0, x_0+\delta))$$
\end{proof}

\begin{theorem}{Достаточное условие экстремума через вторую производную}
$f\colon \left<a, b\right> \ra R$, $x_0 \in (a, b)$, $f$ дважды дифференцируема в $x_0$ и $f'(x_0) = 0$. Тогда
\begin{itemize}
\item $f''(x_0) < 0 \Ra x_0\text{~--- точка максимума}$
\item $f''(x_0) > 0 \Ra x_0\text{~--- точка минимума}$
\end{itemize}
\end{theorem}
\begin{theorem}{Достаточное условие экстремума через $n$-ую производную}
$f\colon \left<a, b\right> \ra R$, $x_0 \in (a, b)$, $f$ дифференцируема $n$ раз в $x_0$ и $f'(x_0) = f''(x_0) \cdots = f^{(n-1)}(x_0) = 0$. Тогда
\begin{itemize}
\item $2 \mid n \land f''(x_0) < 0 \Ra x_0\text{~--- точка максимума}$
\item $2 \mid n \land f''(x_0) > 0 \Ra x_0\text{~--- точка минимума}$
\item $2 \not \mid 2 \land f''(x_0) \ne 0 \Ra x_0\text{~--- не экстремум}$
\end{itemize}
\end{theorem}
\begin{proof}
$$f(x) = f(x_0) + \frac{f'(x_0)}{1!}(x-x_0) + \cdots + \frac{f^{(n-1)}}{(n-1)!}(x-x_0)^{n-1} + \frac{f^{(n)}}{n!}(x-x_0)^{n} + o((x-x_0)^n)$$
$$f(x) - f(x_0) = \frac{f^{(n)}}{n!}(x-x_0)^{n} + o((x-x_0)^n) = (x-x_0)^n \left(\frac{f^{(n)}(x_0)}{n!} + o(1)\right)$$
$2 \div n \land f^{(n)}(x_0) > 0 \Ra \exists \epsilon>0\colon \forall x\in(x_0-\epsilon,x_0)\cup(x_0, x_0+\epsilon)\; f(x) - f(x_0) > 0$
$2 \div n \land f^{(n)}(x_0) < 0 \Ra \exists \epsilon>0\colon \forall x\in(x_0-\epsilon,x_0)\cup(x_0, x_0+\epsilon)\; f(x) - f(x_0) < 0$
$2 \not\div n \land f^{(n)}(x_0) \ne 0 \Ra \exists \epsilon>0\colon \forall x\in(x_0-\epsilon,x_0)\cup(x_0, x_0+\epsilon)\; sign(f(x) - f(x_0)) = sign(x-x_0)$
\end{proof}

\section{Выпуклость}

\begin{Def}
$f\colon \left<a, b\right> \ra \R$. \\
$f$ выпукла вниз, если 
$$\forall x, y \in \left<a, b\right>\; \forall \lambda \in (0, 1) f(\lambda x + (1-\lambda)y) \leqslant \lambda f(x) + (1-\lambda) f(y)$$
$f$ строго выпукла вниз, если 
$$\forall x, y \in \left<a, b\right>\colon x \ne y\; \forall \lambda \in (0, 1) f(\lambda x + (1-\lambda)y) < \lambda f(x) + (1-\lambda) f(y)$$
$f$ выпукла вверх, если 
$$\forall x, y \in \left<a, b\right>\; \forall \lambda \in (0, 1) f(\lambda x + (1-\lambda)y) \geqslant \lambda f(x) + (1-\lambda) f(y)$$
$f$ строго выпукла вверх, если 
$$\forall x, y \in \left<a, b\right>\colon x \ne y\; \forall \lambda \in (0, 1) f(\lambda x + (1-\lambda)y) > \lambda f(x) + (1-\lambda) f(y)$$
\end{Def}
Абсолютно эквивалентная запись, геом. смысл...	0,0301
10.12	

\begin{Rem}
Сумма выпуклых и выпуклая, умноженная на положительную, выпуклы.
\end{Rem}

\begin{lemma}
О трёх хордах. $f\colon \left<a, b\right> \ra R$~--- выпуклая, $u < v < w$, $u,v,w \in \left<a, b\right>$. Тогда
$$\frac{f(v) - f(u)}{v - u} \leqslant \frac{f(w) - f(u)}{w-u} \leqslant \frac{f(w) - f(v)}{w-v}$$
\end{lemma}
\begin{proof}
$$\frac{f(v) - f(u)}{v - u} \leqslant \frac{f(w) - f(u)}{w-u} \Lra (w-u)(f(v) - f(u)) \leqslant (v-u)(f(w) - f(u)) \Lra$$
$$\Lra (w-u)f(v) - (w-u)f(u) \leqslant (v-u)f(w) - (v-u)f(u) \Lra (w-u)f(v) \leqslant (v-u)f(w) + (w-v)f(u)$$
\end{proof}

\begin{theorem}{}
$f\colon \left<a, b\right> \ra R$~--- выпуклая. Тогда 
$$\forall x \in (a, b)\; f'_-(x) \leqslant f'_+(x)$$
\end{theorem}
\begin{proof}
$u_1 < u_2 < x < v$
$$\frac{f(x)-f(u_1)}{x-u_1} \leqslant \frac{f(x)-f(u_2)}{x-u_2} \leqslant \frac{f(x)-f(v)}{x-v}$$
Тогда $\frac{f(x)-f(u)}{x-u}$ растёт и ограничено, т.е. предел $f'_-(x)$ существует.
Аналогично существует $f'_+(x)$, она убывает. Как видно, они в правильном порадке.
\end{proof}

\begin{theorem}{}
$f$~--- выпуклая на $\left<a, b\right>$ тогда и только тогда, когда
$$\forall x, x_0 \in \left<a, b\right>\; f(x) \geqslant f(x_0) + (x-x_0)f'(x_0)$$
\end{theorem}
\begin{proof}
$\Ra$:\\
$x > x_0$, $y \in (x_0, x)$
$$\frac{f(x_0) - f(y)}{x_0-y} \leqslant \frac{f(x_0)-f(x)}{x_0 - x}$$
$$f'(x_0) = \lim_{y \ra x_0} \frac{f(x_0) - f(y)}{x_0-y} \leqslant \frac{f(x_0) - f(x)}{x_0 - x}$$
$x_0 - x > 0$
$$f'(x_0)(x-x_0) \leqslant f(x_0) - f(x)$$
Аналогично $x < x_0$, $y \in (x, x_0)$
$$\frac{f(x) - f(x_0)}{x-x_0} \leqslant \frac{f(y)-f(x_0)}{y - x_0}$$
$\La$:\\
$u<v<w$
$$\forall x\; f(x) \geqslant f(v) + (x-v) f'(v)$$
$$f(u) \geqslant f(v) + (u-v) f'(v)$$
$$f(w) \geqslant f(v) + (w-v) f'(v)$$
Сложим с правильными коэффициентами:
$$(w-v)f(u) \geqslant (w-v) f(v) + (w-v)(u-v) f'(v)$$
$$(v-u)f(w) \geqslant (v-u) f(v) + (w-v)(v-u) f'(v)$$
$$(w-v)f(u) + (v-u) f(w) \geqslant (w-u) f(v)$$
\end{proof}

\begin{theorem}{Критерий выпуклости}
$f\colon \left<a, b\right> \ra \R$, $f$ дифференцируема на $(a, b)$.
$$f\text{ (строго) выпукла} \Lra f' \text{ (строго) возрастает}$$
\end{theorem}
\begin{proof}
$\Ra$: $x_1 < x_2$
$$f(x) \geqslant f(x_1) + (x - x_1) f'(x_1)$$
$$f(x) \geqslant f(x_2) + (x - x_2) f'(x_2)$$
Подставим
$$f(x_2) \geqslant f(x_1) + (x_2 - x_1) f'(x_1)$$
$$f(x_1) \geqslant f(x_2) + (x_1 - x_2) f'(x_2)$$
$$f'(x_1) \leqslant \frac{f(x_2)-f(x_1)}{x_2-x_1} \leqslant f'(x_2)$$
$La$:
Нужно проверить, что 
$$\frac{f(u)-f(v)}{u-v} \leqslant \frac{f(v) - f(w)}{v-w}$$
По теороеме Лагранжа, есть точки $\xi < \eta$
$$\frac{f(u)-f(v)}{u-v} = f'(\xi) \leqslant f'(\eta) = \frac{f(v) - f(w)}{v-w}$$
\end{proof}

\begin{theorem}{Критерий выпуклости через вторую производную}
$f\colon \left<a, b\right> \ra \R$, $f$ дважды дифференцируема на $(a, b)$.
$$f\text{ выпукла} \Lra f'' > 0$$
\end{theorem}
\begin{proof}
Смотрим на теоремы о монотонности.
\end{proof}

\begin{theorem}{Неравенство Денсена}
$f\colon \left<a, b\right> \ra \R$ выпукла.
$$\forall \{x_i\}_{i=1}^n \subset \left<a, b\right> \forall \{\lambda_i\}_{i=1}^n \subset [0, 1]\colon \sum_{i=1}^n \lambda_i = 1$$
$$f\left(\sum_{i=1}^n \lambda_i x_i\right) \leqslant \sum_{i=1}^n \lambda_i f(x_i)$$ 
\end{theorem}
\begin{proof}
Метод математической индукции. Теорема при $n = 2$ совпадает с определением выпуклости.
$$f\left(\underbrace{\sum_{i=1}^n \lambda_i x_i}_{\lrh y} + \lambda x_{n+1} x_{n+1}\right) = f((1 - \lambda_{n+1})y + \lambda_{n+1} x_{n+1}) \geqslant $$
$$ \geqslant (1-\lambda_{n+1}) f(y) + \lambda_{n+1} f(x_{n+1}) = (1 - \lambda_{n+1}) f\left(\sum_{i=1}^n \frac{\lambda_i}{1 - \lambda_{n+1}} x_i\right) \leqslant (1 - \lambda_{n+1}) \sum_{i=1}^n \frac{\lambda_i}{1-\lambda_{n+1}} f(x_i) + \lambda_{n+1} f(x_{n+1}) = $$
$$= \sum_{i=1}^n \lambda_i f(x_i) + \lambda_{n+1} f(x_{n+1})$$
\end{proof}

\begin{conseq}
Неравенство о средних~--- достаточно рассмотреть $$f(x) = -\ln x$$
\end{conseq}
\begin{conseq}
Неравенство Гельдера:
$$x_1, \ldots, x_n, y_1, \ldots, y_n \in \R \quad p,q > 1 \quad \frac1p + \frac1q = 1$$
$$\left|\sum_{i=1}^n x_iy_i\right| \leqslant \left(\sum_{i=1}^n |x_i|^p\right)^{\frac1p} \left(\sum_{i=1}^n |x_i|^q\right)^{\frac1q}$$
\end{conseq}
\begin{proof}
Если есть нули или отрицательные~--- перейдём к модулям.
$$f(x) = x^p$$
$$f\left(  \right) = $$
$$ \lambda_i a_i = \frac{x_iy_i}{(\sum_{i=1}^n y_i^p) ^ {\frac1q}}$$
\end{proof}

\begin{conseq}
Неравентсво Минковского
\end{conseq}
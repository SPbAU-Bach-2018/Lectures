\section{Формула Тейлора}

\begin{theorem}{Формула Тейлора}
$$T(x) = \sum{i=0}^n \frac{T^{(i)} (x_0)}{i!} (x-x_0)^i$$
\end{theorem}
\begin{proof}
$$T(x) = \sum_{i=0}^n a_k (x-x_0)^k$$
$$((x-x_o)^k)^{(m)} = \begin{cases}0 & k < m \\ m! & k = m \\ k(k-1)(k-2)\cdots(k-m+1)(x-x_0)^{k-m} & k > m\end{cases}$$
$$T(x)^{(m)} = \sum_{i=m}^n a_k k(k-1)(k-2)(k-3)\cdots(k-i+1)(x-x_0)^{k-m}$$
$$T(x_0)^{(m)} = a_m m! $$
$$a_m = \frac {T^{(m)}(x_0)}{m!} $$
\end{proof}

\begin{Def}
$f$ дифференцируема $n$ раз в точке $x_0$. Тогда многочленом Тейлора функции $f$ в точке $x_0$ есть
$$T_{n,x_0} f(x) = \sum_{i=0}^n \frac{f^{(i)} (x)}{k!} (x-x_0)^k$$
\end{Def}

\begin{Def}
Формула Тейлора:
$$f(x) = T_{n, x_0} f(x) + R_{n, x_0} f(x)$$
\end{Def}

\begin{lemma}
$g$ дифференцируема $n$ раз в $x_0$. $g(x_0) = g'(x_0) = g''(x_0) = \cdots = g^{(n)}(x_0) = 0$. Тогда
$$g(x) = o\left((x - x_0)^n\right) при x \ra x_0$$
\end{lemma}
\begin{proof}
$$\lim_{x \ra x_0} \frac{g(x)}{(x-x_0) ^ n} = \lim{x \ra x_0} \frac{g'(x)}{n (x-x_0)^{n-1}} = \cdots = \lim_{x\ra x_0} \frac{g^(n-1)}{n! (x-x_0)}$$
$g^{(n-1)}$ дифференцируема в $x_0$, а значит
$$g^{(n-1)}(x) = g^{(n-1)}(x_0) + g^{(n)}(x_0) (x-x_0) + o(x-x_0) = o(x-x_0)$$
Т.о.
$$\lim_{x\ra x_0} \frac{g^(n-1)}{n! (x-x_0)} = 0$$
Тогда 
$$g(x) = o\left((x-x_0)^n\right)$$
\end{proof}

\begin{theorem}{Формула Тейлора с остатком в форме Пеано}
$f$ дифференцируема $n$ раз в $x_0$.
$$f(x) = T_{n, k} f(x) + o((x - x_0)^n) \quad x \ra x_0$$
\end{theorem}
\begin{proof}
$$g(x) = f(x) - T_{n, k} f(x)$$
$$\forall k \leqslant n\; g^{(k)} (x_0) = f^{(k)} (x_0) - \left(T_{n, x_0} f\right)^{(k)} (x_0) = 0$$
Пользуемся леммой.
\end{proof}

\begin{conseq}
$$\exists! P \in \R[x]\colon f(x) = P(x) + o((x - x_n)^k) \quad x \ra x_0$$
\end{conseq}
\begin{proof}
$x \ra x_0$:
$$T_{n, x_0} f(x) + o\left((x-x_0)^n\right) = f(x) = P(x) + o\left((x-x_0)^n\right)$$
$$q(x) \lrh T_{n, x_0} f(x) - P(x) = o\left((x-x_0)^k\right)$$
$$q(x_0) = 0$$
$q \in \R[x]$
$$q(x) = (x - x_0) q_1(x)$$
$$q_1(x) = o\left((x - x_0)^{n-1}\right)$$
$$q_1(x_0) = 0$$
$$\vdots$$
$$q_n(x_0) = o(1)$$
$$q_n \equiv 0$$
$$q \equiv 0$$
$$P \equiv T_{n, x_0} f$$
\end{proof}

\begin{theorem}{Формула Тейлора с остатком в форме Лагранжа}
$f$ дифференцируема $n/ + 1/$ раз в $x_0$, $f^{(n)}$ непрерывна на $[x, x_0]$.
$$\exists c \in (x, x_0)\colon f(x) = T_{n, x_0} f(x) + \frac{f^{(n+1)}(c)}{(n+1)!} (x-x_0)^{n+1}$$
\end{theorem}
\begin{Rem}
Теорема Лагража~--- частный случай для $n = 0$.
$$\exists c\in(x,x_0)\colon f(x) = f(x_0) + f'(c)(x-x_0)$$
\end{Rem}
\begin{proof}
$$f(x) = T_{n, x_0} f(x) + M \frac{(x-x_0)^{n+1}}{(n+1)!}$$
Надо доказать, что в форме
$$\exists c\in(x,x_0)\colon M = \frac{f^{(n+1)}(c)}{(n+1)!}$$
$$g(t) \lrh f(t) - T_{n, x_0} f(t) - M(t-x_0)^{n+1}$$
$$g^{(k)} (t) = f^{(k)}(t) - (T_{n, x_0})^{(k)} (t) - M(n+1)(n+2)(n+3)\cdots(n-k+2)(t-x_0)^{n-k+1}$$
$$g^{(k)} (x_0) = 0$$
Тогда у функции $g$ первые $n$ производных равны нулю, а также $g(x) = 0$, значит
$$g(x_0) = g(x) = 0$$
По теореме Ролля
$$\exists x_1\in(x, x_0)\colon g'(x_1) = 0$$
$$g'(x_0) = g'(x_1) = 0$$
По теореме Ролля
$$\exists x_2\in(x, x_1)\colon g'(x_2) = 0$$
$$\vdots$$
$$\exists x_{n+1}\in(x, x_0)\colon g^{(n+1)}(x_{n+1}) = 0$$
$$g^{(n+1)}(t) = f{(n-1)}(t) - M(n+1)!$$
$$c = x_{n+1}$$
\end{proof}

\begin{conseq}
$f\colon [a, b] \ra \R$, $n+1$ раз дифференцируема на $[a, b]$, $x_0 \in (a, b)$, $\left|f^{(n+1)} (t)\right| \leqslant M$.
$$\left|f(x) - T_{n, x_0} f(x)\right| \leqslant \frac{M \left|x-x_0\right|^{n+1}}{(n+1)!} = O\left((x-x_0)^n\right)$$
\end{conseq}
\begin{proof}
$$\exists c \in (x,x_0)\colon \left|f(x) - T_{n, x_0} f(x)\right| = \left|\frac{f^{(n+1)}(v)}{(n+1)!}(x-x_0)^{n+1}\right|$$
\end{proof}

\begin{conseq}
$f\colon [a, b] \ra \R$, $n+1$ раз дифференцируема на $[a, b]$, $x_0 \in (a, b)$, $forall n\; \left|f^{(n+1)} (t)\right| \leqslant M$.
$$\lim_{n\ra\infty} T_{n, x_0} = f(x)$$
\end{conseq}
\begin{proof}
$$\left|f(x) - T_{n, x_0} f(x)\right| \leqslant \frac{M \left|x-x_0\right|^{n+1}}{(n+1)!} \ra 0$$
\end{proof}

$x_0 = 0$:
\begin{align*}
e^x &= 1 &+ x &+ \frac{x^2}{2!} &+ \frac{x^3}{3!} &+ \frac{x^4}{4!} &+ \cdots &+ o(x^n) \\
e^x &= 1 &+ x &+ \frac{x^2}{2!} &+ \frac{x^3}{3!} &+ \frac{x^4}{4!} &+ \cdots &+ \frac{e^cx^{n+1}}{(n+1)!}
\end{align*}
\begin{align*}
\sin x &= 0 &+ x &+ 0 &- \frac{x^3}{3!} &+ 0 &+ \cdots &+ o(x^{2n+1}) \\
\cos x &= 1 &+ 0 &+ \frac{x^2}{2!} &+ 0 &- \frac{x^4}{4!} &+ \cdots &+ o(x^{2n+1}) \\
\ln (x+1) &= 0 &+ x &- \frac{x^2}2 &+ \frac{x^3}3 &- \frac{x^4}4 &+ \cdots &+ o(x^n) \\
(x+1)^p &= 1 &+ px &+ \frac{p(p-1)}{2!}x^2 &+ \frac{p(p-1)(p-2)}{3!}x^3 &+ \frac{p(p-1)(p-2)(p-3)}{4!}x^4 &+ \cdots &+ o(x^n)
\end{align*}

\begin{Def}
$a_n \in \R$
$$\sum_{i=0}^\infty \eqDef \lim_{n\ra\infty} \sum{i=0}^n a_n$$
\end{Def}

\begin{conseq}
$\forall x\ in \R\;$
$$e^x = \sum_{n=0}^\infty \frac{x^n}{n!}$$
$$\sin x = \sum_{n=0}^\infty \frac{(-1)^n x^{2n+1}}{(2n+1)!}$$
$$\sin x = \sum_{n=0}^\infty \frac{(-1)^n x^{2n}}{(2n)!}$$
\end{conseq}

\begin{theorem}{Иррациональность $e$}
$$e \notin \Q$$
\end{theorem}
\begin{proof}
$$\left(1+\frac1n\right)^n \leqslant e \leqslant \left(1+\frac1{n+1}\right)^n$$
$$2 < e < 3$$
Пусть $e = \frac{m}{n}$
$$e^1 = 1 + 1 + \frac1{2!} + 1\frac{3!} + \cdots + \frac{e^c}{(n+1)!} = \frac{m}{n} \Ra$$
$$\Ra \underbrace{n!(1 + 1 + \frac1{2!} + 1\frac{3!} + \cdots)}_{\in \N} + \frac{e^c}{n+1} = \underbrace{m(n-1)!}_{\in N} \Ra$$
$$\Ra \frac{e^c}{n+1} \in \N$$
$$0<c<1 \Ra 1 < e^c < 3$$
$$0 < \frac{1}{n+1} < \frac{e^c}{n+1} < \frac{3}{n + 1} < 1$$
Т.о. $e \ne \frac{m}{n}$
\end{proof}

\section{Экстремумы функции}

\begin{Def}
$f\colon \left<a, b\right> \ra \R$, $x_0 \in (a, b)$. 
$x_0$~--- точка строгого локального минимума, если
$$\exists \delta>0\colon \forall x \in (x - \delta, x + \delta) \ \{x_0\} f(x) > f(x_0)$$
$x_0$~--- точка нестрогого локального минимума, если
$$\exists \delta>0\colon \forall x \in (x - \delta, x + \delta) f(x) \geqslant f(x_0)$$

$x_0$~--- точка строгого локального максимума, если
$$\exists \delta>0\colon \forall x \in (x - \delta, x + \delta) \ \{x_0\} f(x) < f(x_0)$$
$x_0$~--- точка нестрогого локального максимума, если
$$\exists \delta>0\colon \forall x \in (x - \delta, x + \delta) f(x) \leqslant f(x_0)$$

Точка локального максимума или минимума также называется точкой локального экстремума.
\end{Def}
\begin{theorem}{Необходимое условие экстремума}
$f\colon \left<a, b\right> \ra \R$, $x_0 \in (a, b)$, $f$ дифференцируема в $x_0$.
$$x_0\text{~--- экстремум} \Ra f'(x_0) = 0$$
\end{theorem}
\begin{proof}
Сузим до окрестности, там по теореме Ферма всё работает.
\end{proof}
\begin{Rem}
Обратное неверно, смотри $f(x) = x^3$.
\end{Rem}\
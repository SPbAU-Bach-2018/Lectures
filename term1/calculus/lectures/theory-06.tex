
$1 \Ra 3$:
Возьмём последовательность $\{x_n\}\lrh E$ элементов множества $F$. Если множество элементов $E$ конечно, то какой-то элемент повторился бесконечно. Возьмём новую стационарную последовательность ровно из этого элемента, имеющую предел. Если же оно бесконечно, докажем, что у него есть предельная точка.

Пусть ни одна точка не предельна. Значит 
$$\forall x \in X\: \exists r_x > 0\colon \dot B_{r_x}(x) \cap F = \emptyset$$
Но тогда возьмём покрытие
$$\bigcup_{x\in X} B_{r_x} (x)$$
В нём есть конечное подпокрытие. Возьмём его
$$\bigcup_{i=1}^k \dot B_{r_{y_i}} \supset K \supset E$$
Но также
$$\bigcup \dot B_{r_{y_i}} \cap E = \varnothing$$
Значит 
$$E \subset \bigcup_{i=1}^k \{y_i\}$$
Получили, что $E$ конечное. 

Таким образом предельная точка существует, а значит можно выбрать подпоследовательность можно.

$3 \Ra 2$:
Пусть $K$ не замкнуто. Возьмём предельную точку, которой нет в $K$. Значит есть последовательность, сходящаяся к ней. Из неё нельзя выбрать подпоследовательность, сходящуюся к элементу $K$.

Пусть $K$ не ограничено. Значит есть точка, не лежащая в данном шарике.
$$K \nsubset B_1(a) \Ra \exists x_1\colon \rho(x_1, a) > 1$$
$$K \nsubset B_{\rho(a, x_1) + 1}(a) \Ra \exists x_2\colon \rho(x_2, a) > \rho(x_1, a) + 1$$
$$ \vdots $$
Рассмотрим сходящуюся подпоследовательность. Она ограничена шариком радиуса $R$. Но
$$\rho(a, x_n) > \rho(a, x_{n-1}) + 1 > \cdots > n$$
$$R > \rho\left(b, x_{n_k}\right) > \rho\left(a, x_{n_k}\right) + \rho(a, b) > n_k + \rho(a, b) \ra \infty$$
Значит $K$ ограниченно.  
\end{proof}

\begin{Rem}
$1\Ra 3; 3\Ra 2; 1 \Ra 2$ справедливы для всех пространств. $2 \Ra 1$ ломается, например, на $\R$ с дискретной метрикой.
\end{Rem}

\begin{conseq}
В $\R^d$ компактность $K$ равносильна наличию предельной точки для любого подмножества.  
\end{conseq}
\begin{proof}
В одну сторону просто по теореме.
Обратно: возьмём часть доказательства, объясняющее взятие подпоследовательности.
\end{proof}

\begin{conseq}{Теорема Больцано-Вейерштрасса.}
Из любой ограниченной последовательности в $\R^d$ можно выбрать сходящуюся подпоследовательность.
\end{conseq}
\begin{proof}
Множество значений ограниченно, значит его замыкание компактно, значит в компактном есть сходящаяся подпоследовательность.
\end{proof}

\begin{conseq}
В любой последовательности в $\R$ есть сходящаяся в $\bar \R$ подпоследовательность.
\end{conseq}
\begin{proof}
Если ограничена, то см. предыдущее. Иначе она стремится к бесконечности. Тогда выберем бесконечную подпоследовательность, стремящуюся к бесконечности. В ней бесконечное число положительных или бесконечное число отрицательных.
\end{proof}

\begin{Def}
Диаметр множеста:
$$\diam A = \sup \rho(x, y)$$
\end{Def}

\begin{theorem}{Свойства диаметра}
\begin{enumerate}
\item $\diam E = \diam \cl E$
\item $K_1 \supset K_2 \supset K_3 \ldots(\text{последовательность вложенных компактов}); \diam K_n \ra 0 \Ra \bigcap K_i\text{~--- одноточечное}$
\end{enumerate}
\end{theorem}
\begin{proof}
\begin{enumerate}
\item
$$E \subset \cl E \Ra \diam E \leqslant \diam \cl E$$
$$d = \diam \cl E = \sup \rho(x, y)$$
$$\forall \epsilon > 0\: \exists x_0, y_0\colon \rho(x_0, y_0) > d - \epsilon$$
$$x_0 \in \cl E \Ra \exists x_1 \in E\colon \rho (x_0, x_1) < \epsilon$$ 
$$y_0 \in \cl E \Ra \exists y_1 \in E\colon \rho (y_0, y_1) < \epsilon$$ 
Тогда
$$\rho(x_1, y_1) + 2\epsilon > \rho(x_0, x_1) + \rho(x_1, y_1) + \rho(y_1, y_0) \geqslant \rho(x_0, y_0) > d - \epsilon$$
$$\rho(x_1, y_1) > \rho(x_0, y_0) - 3\epsilon$$
Устремив $\epsilon \ra 0$, получим
$$\diam E \geqslant \diam \cl E$$
\item Пусть в пересечение лежат две точки, но тогда диаметр для любого n хотя бы $\rho(a, b)$. Противоречие.
\end{enumerate}
\end{proof}

\begin{Def}
Последовательность называется фундаментальной, если 
$$\forall \epsilon > 0\: \exists N\colon \forall n,m > N\: \rho(n, m) < \epsilon$$
\end{Def}
\begin{Rem}
$E \lrh \{x_i\}_{i=n}^{\infty}$
$$\{x_n\}\text{ фундаментальная} \Lra \diam E \ra 0$$
\end{Rem}
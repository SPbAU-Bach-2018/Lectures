\chapter[Посл-ти в метрических пространствах]{Последовательности в метрических пространствах}

\section{Метрические пространства}

\begin{Def}
Пусть есть множество $X$ и отображение $\rho \colon X \times X \ra \left[0; +\infty\right) $. Тогда $\rho$ называется метрикой, если:
\begin{enumerate}
\item $\rho(x, y) = 0 \Lra x = y$
\item $\rho(x, y) = \rho(y, x)$
\item $\rho(x, y) + \rho(y, z) \geqslant \rho(x, z)$
\end{enumerate} 
Также пара $(X, \rho)$ называется метричесикм пространством.
\end{Def}

Примеры:
\begin{enumerate}
\item Дискретная метрика
$\rho(x, y) = \begin{cases}0 & x \ne y \\ 1 & x = y\end{cases}$
\item $\rho(x, y) = \left|x - y\right|$
\item Евклидовская метрика. $\rho$ --- длина отрезка на плоскости между точками
\item Манхеттанская метрика. $\rho\left((x_1, y_1), (x_2, y_2)\right) = |x_1 - x_2| + |y_1 - y_2|$
\item Расстояния на сфере.
\item Французская железнодорожная метрика. Есть центр --- точка $O$. Тогда для точек на одном луче из $O$ расстояние $\rho(A, B) = |AB|$, иначе $\rho(A, B) = |AO| + |BO|$/
\item Пространство $\R^n$, метрика $$\rho(x, y) = \sqrt{\sum_{i=1}^n \left(x_i-y_i\right)^2}$$
\end{enumerate}

\begin{Def}
Пусть $(X, \rho)$ --- метрическое пространство. Тогда $(Y, \rho|_{Y \times Y})$ --- подпространство X. $Y \subset X$.
\end{Def}


\begin{Def}
$B_r(a) = \left\{x \in X \mid \rho(x, a) < r\right\}$ --- открытый шар.
\end{Def}
\begin{Def}
$\bar B_r(a) = \left\{x \in X \mid \rho(x, a) \leqslant r\right\}$ --- замкнутый шар.
\end{Def}

Свойства:
\begin{enumerate}
\item $B_{r_1}(a) \cap B_{r_2}(a) = B_{\min\{r_1, r_2\}}(a)$
\item $x \ne y \Ra \exists r > 0\colon B_r(x) \cap B_r(y) = \varnothing$
\begin{proof}
Рассмотрим $r = \frac13 \rho(x,y) > 0$.
\end{proof}
\end{enumerate}
         

\begin{theorem}{Неравенство Коши-Буняковского}
$a_1, a_2, \ldots a_n, b_1, b_2, \ldots, b_n \in \R$
$$\left(\sum_{k=1}^n a_kb_k\right){}^2 \leqslant \sum_{k=1}^n a_k^2 \sum_{k=1}^n b_k^2 $$
\end{theorem}
\begin{proof}
$$f(t) \lrh \sum_{k=1}^n(a_kt-b_k)^2 = \left(\underbrace{a_1^2 + a_2^2 + \ldots + a_n^2}_{\lrh A}\right)t^2 - 
2\left(\underbrace{a_1b_1 + \ldots + a_nb_n}_{\lrh C}\right)t + \left(\underbrace{b_1^2 + \ldots + b_2^2}_{\lrh B}\right)$$
$f$ имеет не более 1 корня, следовательно
$$ (2C)^2 - 4AB \leqslant 0 \Ra 4\left(C^2 - AB\right) \leqslant 0 \Lra C^2 \leqslant AB$$
Можно считать, что все числа не равны 0~--- иначе всё тривиально.
\begin{Rem}
Равентсво в случае, если числа пропорциональны.
\end{Rem}
\begin{proof}
$$a_i = \alpha b_i$$

$\Lra$

$$C^2 = AB \Lra \text{есть корень} t_0 \Lra \forall a_k t_0 - b_k = 0$$
\end{proof}
\end{proof}

\begin{theorem}{Неравенство Минковского}
$$\sqrt{\sum_{i=1}^n (a_i+b_i)^2} \leqslant \sqrt{\sum_{i=1}^k a_i^2} + \sqrt{\sum_{i=1}^k b_i^2}$$
\end{theorem}
\begin{proof}
Возведём в квадрат
$$ \sqrt{\sum_{i=1}^n (a_i+b_i)^2} \leqslant \sqrt{\underbrace{\sum_{i=1}^k a_i^2}_{\lrh A}} + \sqrt{\underbrace{\sum_{i=1}^k b_i^2}_{\lrh B}} \Lra \sum_{i=1}^n (a_i + b_i)^2 \leqslant A + 2\sqrt{AB} + B \Lra$$
$$ \Lra A + B + 2\sum_{i=1}^n a_ib_i \Lra A + B + 2\sqrt{AB} \Lra \sum_{i=1}^n a_ib_i \leqslant \sqrt{AB} \La$$
$$ \La \text{Неравенство Коши-Буняковского}$$
\begin{Rem}
Равентсво в случае, если числа пропорциональны.
\end{Rem}
\end{proof}      

\begin{Def}
$(X, \rho)$ --- метрическое пространство. $G \subset X$ --- открытое множество, если $$\forall x \in G\: \exists r > 0\colon B_r(x) \subset G$$
\end{Def}

\begin{theorem}{О свойтсвах открытых множеств}
Пусть $(X, \rho)$ --- метрическое пространство.
\begin{enumerate}
\item $\varnothing$ и $X$ --- открыты.
\item Объединение открытых открыто.
\item Пересечение \textbf{конечного числа} открытых открыто.
\item $B_r(a)$ открыт.
\end{enumerate}
\end{theorem}
\begin{proof}
\begin{enumerate}
\item Очевидно.
\item $$x \in \bigcup G_\alpha \Ra \exists \alpha_0 \colon x \in G_{\alpha_0} \Ra \exists r > 0: B_r(x) \in \bigcup G_\alpha$$
\item $x \in \bigcap_{k=1}^n G_k$ 
$$ \forall k=1..n\: x \in G_k \Ra \forall k=1..n\: \exists r_k > 0\colon B_{r_k}(x) \in G_k \Ra \exists r = \min{r_k}\colon G_r \in \bigcap_{k=1}^n G_k$$
\item $$\forall x \in B_r(a)\: \exists r_x = \frac12 \left(r - \rho(a, x)\right)$$
$$y \in B_{r_x}(x) \Ra \rho(y, x) < r_x \Ra \rho(y, x) + \rho(a, x) < r_x + \rho(a, x) \Ra \rho(y, a) < r$$
\end{enumerate}
\end{proof}

\begin{Rem}
$$\bigcap_{n=1}^\infty \left(0; 1 + \frac1n\right) = \left(0;1\right] \text{ --- не открытое множество}$$
\end{Rem}

\begin{Def}
$x \in A$ --- внутренняя точка $A$, если $\exists r > 0\colon B_r(x) \in A$
\end{Def}
\begin{Rem}
$x$ --- внутренняя точка $A$ эквивалентно тому, что в $A$ содержится некое открытое множество, содержащее x.
\end{Rem}
\begin{Def}
Внутренность множества $A$:
$$A^0 = \Int A \eqDef \bigcup_{\substack{G \text{ открыто} \\ G \subset A}} G$$
\end{Def}

Свойства:
\begin{enumerate}
\item $\Int A \subset A$
\item $\Int A$ --- множество всех внутренних точек.
\item $\Int A$ открыто.
\item $A \text{ открыто} \Lra A = \Int A$
\item $A \subset B \Ra \Int A \subset \Int B$
\item $\Int (A \cap B) = \Int A \cap \Int B$
\item $\Int \Int A = \Int A$
\end{enumerate}
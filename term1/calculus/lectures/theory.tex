%% texify: XeLaTeX + MakeIndex + BibTeX, modificated
%% texify
%% --pdf
%% --engine=xelatex
%% --tex-option=$synctexoption // You may delete this, affords to skip from preview to code in one click, that i seldom do
%% --tex-option=-8bit // Else minted fails on tabs
%% --tex-option=--shell-escape // For minted to live
%% $fullname
%% I prefer to build with TeXworks for better view of errors and warnings. It's hard to read all log file. 

\documentclass[12pt,a4paper]{book}
\usepackage{polyglossia} %% Better than babel on XeLaTeX
\usepackage{amsmath, amssymb} %% Cool math!
\usepackage{color} %% Coloring almost anything
\usepackage[russian]{hyperref} %% Clickable links is pdf
\usepackage{indentfirst}
\usepackage[left=1cm,right=1cm,top=2cm,bottom=2cm]{geometry}
%% WARNING: latest minted is used. Download from github!
%% Works fine, though
%\usepackage{minted} %% Highlighting code. Installation is hard: requires python2 and script Pygments. Look at documentation for help!
\usepackage[math-style=ISO,vargreek-shape=unicode]{unicode-math} %% MAGIC! INCLUDE AS LAST!

\setdefaultlanguage[spelling=modern]{russian} %% Languages for polyglossia
\setotherlanguage{english}

\defaultfontfeatures{Ligatures={TeX}} %% Fonts and ligatures.
\setmainfont{CMU Serif} %% There are original Knuth's fonts in Unicode, called Computer Modern Unicode. Download anywhere, just install them
\setsansfont{CMU Sans Serif}
\setmonofont{CMU Typewriter Text}  
\setmathfont{Latin Modern Math} %% Download too. You may change it :)
\AtBeginDocument{\def\setminus{\mathbin{\backslash}}}
%\setmathfont{XITS}

%% Magic as black as my working table
%\DeclareSymbolFont{cyrletters}{\encodingdefault}{\familydefault}{m}{it}
%\newcommand{\makecyrmathletter}[1]{%
%  \begingroup\lccode`a=#1\lowercase{\endgroup
%  \Umathcode`a}="0 \csname symcyrletters\endcsname\space #1
%}
%\count255="409
%\loop\ifnum\count255<"44F
%  \advance\count255 by 1
%  \makecyrmathletter{\count255}
%\repeat
%% Simpy adds cyrillic to maths!

\frenchspacing %% One space before sentence, not two!

%% Shortcuts:
\def\la{\leftarrow}
\def\ra{\rightarrow}
\def\lra{\leftrightarrow}
\def\La{\Leftarrow}
\def\Ra{\Rightarrow}
\def\Lra{\Leftrightarrow}
\def\lrh{\leftrightharpoons}
\def\xra{\xrightarrow}
\def\btu{\bigtriangleup}

\def\N{\mathbb{N}}
\def\Z{\mathbb{Z}}
\def\Q{\mathbb{Q}}
\def\R{\mathbb{R}}
\def\C{\mathbb{C}}

\def\LraDef{\stackrel{\mathrm{Def}}{\Lra}}
\def\eqDef{\stackrel{\mathrm{Def}}{=}}
\def\d{\mathup{d}}

% ======================================

%% Change Chapter and Section numeration style
\renewcommand{\thechapter}{\Roman{chapter}}
\renewcommand{\thesection}{\thechapter.\arabic{section}}

%% Indent for first par in chapter
\makeatletter
\renewcommand{\chapter}{\clearpage %% no double page, only
\thispagestyle{empty}%% not plain, empty. wanna number of page!
\global\@topnum=0
\@afterindenttrue %% Set to true!
\secdef\@chapter\@schapter}
\makeatother

%% Environment for theorem body
\newcounter{theorem}[section]
\renewcommand{\thetheorem}{\thesection.\arabic{theorem}}
\newcommand*{\theoremheader}[1]{\par\refstepcounter{theorem}%
\textbf{Теорема \thetheorem. #1.}}
\newenvironment*{theorem}[1]{
\theoremheader{#1}%
}{%
}

%% Environment for consequence body
\newcounter{conseq}[theorem]
\renewcommand{\theconseq}{\thetheorem.\arabic{conseq}}
\newcommand*{\conseqheader}{\par\refstepcounter{conseq}%
\textit{Следствие \theconseq.} }
\newenvironment*{conseq}{
\conseqheader%
}{%
}

\newcounter{lemma}[section]
\renewcommand{\thelemma}{\thesection.\arabic{lemma}}
\newcommand*{\lemmaheader}{\par\refstepcounter{lemma}%
\textit{Лемма \thelemma.}}
\newenvironment*{lemma}{
	\lemmaheader%
}{%
}

\newenvironment{assertion}{%
\par\textbf{Утверждение. }%
}{%
%
}

%% Environment for proof body. I like this style, but you are free to change it.
\newenvironment{proof}{%
\par$\blacktriangleright$%
}{%
\hfill$\blacktriangleleft$%
}

%% Environment for definitions. Pretty raw one.
\newenvironment{Def}{%
\par$\mathfrak{Def\colon}$%
}{%
}

%% Environment for remarks.
\newenvironment{Rem}{%
\par\textit{REM: }%
}{%
}

\setcounter{MaxMatrixCols}{40}

% ==================================

%% In-line code highlighting. Using: \py|a = input()|
%\newmintinline[cinl]{c}{} %\c is defined :(
%\newmintinline[cpp]{cpp}{}
%\newmintinline[python]{python}{}
%\newmintinline[bash]{bash}{}
%\newmintinline[make]{make}{}

%% Escaped code highlighting. Using: \begin{cppcode} ... \end{cppcode}
%\setminted{obeytabs,tabsize=4,linenos,texcomments}
%\newminted{c}{}
%\newminted{cpp}{}
%\newminted{python}{}
%\newminted{bash}{}
%\newminted{make}{}

% ==================================
\DeclareMathOperator{\Int}{int}
\DeclareMathOperator{\cl}{cl}
\DeclareMathOperator{\diam}{diam}
\newcommand{\emod}[1]{\mathop{\equiv}\limits_{#1}}


\begin{document}
\chapter{Введение}

\section{Множества}

Не любая совокупность элементов --- множество. Про каждый объект можно сказать, принадлежит ли он множеству ($x \in A$) или нет ($x \notin A$).

\begin{Def}
Множество $A$ - подмножество $B$, если все элементы $A$ содержатся и в $B$. 
$$ A \subset B \LraDef \forall x \in A\; x \in B $$
\end{Def}
\begin{Def}
Множества называются равными, если они содержатся друг в друге.
$$ A = B \LraDef A \subset B \land B \subset A $$
\end{Def}
\begin{Def}
Пустое множество --- это множество без элементов.
$$ \forall x\: x \notin \emptyset $$
\end{Def}
\begin{Def}
$2^A$ --- множество всех подмножеств $A$.
$$ 2^A \eqDef \left\{B \mid B \subset A \right\} $$
\end{Def}

\begin{itemize}
\item $\N$ --- множество натуральных чисел. 
\item $\Z$ --- множество целых чисел.
\item $\Q$ --- множество рациональных чисел.
\item $\R$ --- множества вещественных чисел.
\item $\C$ --- множества комплексных чисел.
\end{itemize}

Задание множеств:
\begin{itemize} 
\item $\left\{a,b,c\right\}$
\item $\left\{a_1, a_2, \ldots, a_n\right\}$
\item $\left\{a_1, a_2, \ldots\right\}$
\item $\left\{x \in A \mid \Phi(x)\right\}, \Phi(x) - \text{условие}$.
\end{itemize} 
Например, $\left\{p \in \N \mid p \text{ имеет ровно 2 натуральных делителя}\right\}$.

Бывают некорректно заданные ,,множества``. Например, множество художественных произведений на русском языке --- плохо заданное множество. Рассмотрим 
$\Phi(n)$ --- истина, если n нельзя записать в не более чем тридцать слов русского языка. Тогда
$\left\{n \in \N \mid \Phi(n)\right\}$~--- не множество. Если бы это было множеством, то в нём есть наименьший элемент, 
который описывается как ,,Наименьший элемент множества...``

\begin{Def}
Пересечение двух множеств~--- множество, состоящие из всех элементов, находящихся одновременно в обоих множествах.
$$ A \cap B \eqDef \left\{x \in A \mid x \in B \right\} $$
\end{Def}
\begin{Def}
Объединение двух множеств~--- множество, состоящее из элементов обоих множеств.
$$ A \cup B \eqDef \left\{x \mid x \in A \lor x \in B \right\} $$
\end{Def}
\begin{Def}
Разность множеств~--- это множество тех элементов, которые лежат в первом, но не во втором.
$$ A \setminus B \eqDef \left\{ x \in A \mid x \notin B \right\}$$
\end{Def}
\begin{Def}
Симметрическя разность~--- объединение разностей.
$$ A \btu B \eqDef \left(A \setminus B\right) \cup \left(B \setminus A\right) $$
\end{Def}

Объединение и пересечение множно записать для многих множеств.
$$ \bigcup_{i \in I} A_i = \left\{x \mid \exists i \in I\colon x \in A_i\right\}; 
\bigcap_{i \in I} A_i = \left\{x \mid \forall i \in I\: x \in A_i \right\} $$

Свойства операций со множествами:
\begin{enumerate}
\item Ассоциативность
$$ A \cap B = B \cap A; A \cup B = B \cup A $$
\item Коммутативность
$$ \left(A \cap B \right) \cap C = A \cap \left(B \cap C \right); \left(A \cup B \right) \cup C = A \cup \left(B \cup C \right) $$
\item Рефлексивность
$$ A \cap A = A; A \cup A = A $$
\item Дистрибутивность
$$ A \cap \left(B \cup C \right) = \left(A \cap B\right) \cup \left(A \cap C \right) $$
$$ A \cup \left(B \cap C \right) = \left(A \cup B\right) \cap \left(A \cup C \right) $$
\item Нейтральный элемент
$$ A \cap \emptyset = \emptyset$$
$$ A \cup \emptyset = A$$
\end{enumerate}

\begin{theorem}{Правила де Моргана}
$ A, B_\alpha, \alpha \in I $.
Тогда 
$$ A \setminus \bigcup_{\alpha \in I} B_\alpha = \bigcap_{\alpha \in I} \left(A \setminus B_\alpha\right) ; 
A \setminus \bigcap_{\alpha \in I} B_\alpha = \bigcup_{\alpha \in I} \left(A \setminus B_\alpha\right) $$
\end{theorem} 
\begin{proof}
$$
x \in A \setminus \bigcup_{\alpha \in I} B_{\alpha} \Lra \left\{\begin{aligned}x &\in A \\ x &\notin \bigcup_{\alpha \in I} B_{\alpha}\end{aligned}\right. \Lra 
\left\{\begin{aligned} x &\in A \\ \forall \alpha \in I\: x &\notin B_\alpha \end{aligned}\right. \Lra
\forall \alpha \in I\: \left\{\begin{aligned} x &\in A \\ x &\notin B_\alpha \end{aligned}\right.  
\Lra x \in \bigcap_{\alpha \in I} \left(A \setminus B_\alpha\right) 
$$
$$
x \in A \setminus \bigcap_{\alpha \in I} B_{\alpha} \Lra \left\{\begin{aligned}x &\in A \\ x &\notin \bigcap_{\alpha \in I} B_{\alpha}\end{aligned}\right. \Lra 
\left\{\begin{aligned} x &\in A \\ \lnot \forall \alpha \in I\: x &\in B_\alpha \end{aligned}\right. \Lra
\exists \alpha \in I\colon \left\{\begin{aligned} x &\in A \\ x &\notin B_\alpha \end{aligned}\right.  
\Lra x \in \bigcup_{\alpha \in I} \left(A \setminus B_\alpha\right) 
$$
\end{proof}

\begin{theorem}{Обобщение дистрибутивности}
$ A, B_\alpha, \alpha \in I $.
Тогда 
$$ A \cap \bigcup_{\alpha \in I} B_\alpha = \bigcup_{\alpha \in I} (A \cap B_\alpha) $$
$$ A \cup \bigcap_{\alpha \in I} B_\alpha = \bigcap_{\alpha \in I} (A \cup B_\alpha) $$
\end{theorem}
\begin{proof}
$$
x \in A \cap \bigcup_{\alpha \in I} B_{\alpha} \Lra \left\{\begin{aligned}x &\in A \\ x &\in \bigcup_{\alpha \in I} B_{\alpha}\end{aligned}\right. \Lra 
\left\{\begin{aligned} x &\in A \\ \exists \alpha \in I\colon x &\in B_\alpha \end{aligned}\right. \Lra
\exists \alpha \in I\colon \left\{\begin{aligned} x &\in A \\ x &\in B_\alpha \end{aligned}\right.  
\Lra x \in \bigcup_{\alpha \in I} \left(A \cap B_\alpha\right) 
$$
$$
x \in A \cup \bigcap_{\alpha \in I} B_{\alpha} \Lra \left[\begin{aligned}x &\in A \\ x &\in \bigcap_{\alpha \in I} B_{\alpha}\end{aligned}\right. \Lra 
\left[\begin{aligned} x &\in A \\ \forall \alpha \in I\: x &\in B_\alpha \end{aligned}\right. \Lra
\forall \alpha \in I\: \left[\begin{aligned} x &\in A \\ x &\in B_\alpha \end{aligned}\right.  
\Lra x \in \bigcap_{\alpha \in I} \left(A \cup B_\alpha\right) 
$$
\end{proof}

\begin{Def}
Упорядоченная пара $\langle a, b \rangle$ или $(a, b)$ --- объект
$$ (a_1; b_1) = (a_2; b_2) \LraDef a_1 = a_2 \land b_1 = b_2 $$
\end{Def}
\begin{Def}
Упорядоченная $n$-ка, или кортеж --- объект
$$ (a_1, a_2, \ldots, a_n) = (b_1, b_2, \ldots, b_n) \LraDef \forall i=1..n\: a_i = b_i $$
\end{Def}

\section{Бинарные отношения}

\begin{Def}
Декартого произведение множеств --- множество кортежей, состоящих из элементов соответствующих множеств.
$$ \left(a_1, a_2, \ldots, a_n\right) \in A_1 \times A_2 \times \ldots \times A_n \LraDef \forall i=1..n\: a_i \in A_i $$
\end{Def}
\begin{Def}
Отношение на множествах $A$ и $B$ --- произвольное подмножество их декартова произведения.
$$ a \mathop{R} b \LraDef (a, b) \in R $$
\end{Def}
\begin{Def}
Область определения отношения 
$$ \beta_R = dom_R = \{a \in A \mid \exists b \in B\colon (a,b) \in R\} $$
\end{Def}
\begin{Def}
Обсласть значения отношения 
$$ \rho_R = ran_R =\{b \in B \mid \exists a \in A\colon (a, b) \in R\}$$
\end{Def}
\begin{Def}
Обратное отношение
$$R^{-1} \colon \beta_{R^{-1}} = \rho_R; \rho_{R^{-1}} = \beta_R; b \mathop{R^{-1}} a \LraDef a \mathop{R} b$$
\end{Def}
\begin{Def}
Композиция отношений
$$ R_1\colon A \ra B; R_2\colon B \ra C $$
$$ R_1 \circ R_2 = \{(a, c) \mid a \mathop{R_1} b \land b \mathop{R_2} c\} $$
Про значок ~--- его использовать не будем
\end{Def}

Пример композиции: $<\colon \N \ra \N$. $$< \circ < = \{(a, b) \mid b - a \geqslant 2\}$$

\begin{Def}
Функция (отображение) ~--- такое отношение, что первый ключ уникален.
$$f\colon A \ra B$$
$$ a \mathop{f} b_1 \land a \mathop{f} b_2 \Ra b_1 = b_2 $$
$$ a \mathop{f} b \LraDef f(a) = b $$
$$ A = \beta_f \quad \text{($A$~--- область определения)}$$
\end{Def}

\begin{Def}Свойтва отображеий:
\begin{enumerate}
\item Рефлексивность $a \mathop{R} a$
\item Cимметричность $a \mathop{R} b \Lra b \mathop{R} a$
\item Транзитивность $a \mathop{R} b \land b \mathop{R} c \Ra a \mathop{R} c$
\item Иррефлексивность $\lnot a \mathop{R} a$
\item Антисимметричность $a \mathop{R} b \land b \mathop{R} a \Ra a = b$
\end{enumerate}
\end{Def}

Примеры:
\begin{itemize} 
\item $=$: 1, 2, 3, 5
\item $\emod{5}$: 1, 2, 3
\item $\leqslant$: 1, 3, 5
\item $<$: 3, 4, 5
\item $\subset$: 1, 3, 5
\end{itemize}

\section{Вещественные числа}
\begin{Def}
Множество вещественных чисел можно определить как множество, на котором есть операции $+$ и $\times$, причём:
\begin{enumerate}
\item Коммутативность $\forall a, b\: a + b = b + a; a \times b = b \times a$
\item Ассоциативность $\forall a, b, c\: a + (b + c) = (a + b) + c; a \times (b \times c) = (a \times b) \times c$
\item Нейтральный элемент
$\exists o\colon \forall a\: a + o = a; \exists e\colon \forall a\:a \times e = a; o \ne e$
\item Обратный элемент
$\forall a\: \exists{-a}\colon a + -a = o; \forall a \ne o\: \exists a^{-1}: a \times a^{-1} = a$
\item Дистрибутивность $\forall a, b, c\: a \times (b + c) = (a \times b) + (a \times c)$
\end{enumerate}
Кроме того, есть отношения $\leqslant$ (и аналогично $<$, также определены обратные):
\begin{enumerate}
\item Рефлексивно
\item Антисимметрично
\item Транзитивно
\item Любые два элемента сравнимы
\item $\forall a, b, c\: a \leqslant b => a + c \leqslant b + c$
\item $\forall a, b\: a > 0 \land b \geqslant 0 \Ra ab \geqslant 0$
\end{enumerate}
\end{Def}

Также выполнена аксиома полноты: $A, B \subset \R$, $A \cup B \ne \varnothing$, $ \forall a \in A\: \forall b \in B\: a \leqslant b $. Тогда 
$$\exists c \in \R\colon \forall a \in A\: a \leqslant c \land \forall b \in B\: c \leqslant b$$

\begin{Rem}
На $\Q$ аксиома не выполняется: 
$$A = \left\{r \in \Q \mid r^2 < 2\right\}; B = \left\{r \in \Q_{+}\mid r ^ 2 > 2\right\}; c = \sqrt{2} \notin \Q$$
\end{Rem}
\begin{theorem}{Принцип Архимеда}
Пусть $ x, y \in \R, y > 0 $.
Тогда $$ \exists n \in \N: x < ny$$
\end{theorem}
\begin{proof}
$$ A \lrh \left\{u \in \R \mid \exists n \in \N: u < ny\right\}; y \in A$$
Пусть $A \ne \R$. Тогда $B \lrh \R - A \ne \varnothing$.
Рассмотрим $a \in A; b \in B$.
$$b < a \Ra b < a < ny \Ra b \in A \text{ --- противоречие}$$
Таким образом 
$$\forall a \in A\: \forall b \in B\: a \leqslant b$$
Тогда 
$$\exists c \in \R\colon \forall a \in A\: a \leqslant c \land \forall b \in B\: c \leqslant b$$
$$c \in A \Ra c + y \in A \Ra c > c + y \Ra y < 0 \text{~--- противоречие}$$
Тогда $c \in B$.
Пусть $c - y \notin B$, тогда 
$$c - y \in A \Ra c - y < ny \Ra c < (n + 1)y \Ra c \in A \text{~--- противоречие}$$
Значит 
$$c - y \in B \Ra c - y \geqslant c \Ra y \leqslant 0 \text{~--- противоречие}$$
Таким образом $A = \R$
\end{proof}

\begin{conseq}
$$\forall \epsilon > 0\: \exists n \in \N\colon \frac{1}n < \epsilon$$
\end{conseq}
\begin{proof}
Рассмотрим $x=1, y=\epsilon$
\end{proof}

\begin{conseq}
$x, y \in \R, x < y$
$$\exists r \in \Q: x < r < y$$
\end{conseq}
\begin{proof} 
$$y - x > 0 \Ra \exists n \in \N\colon \frac1n < y - x$$
Покажем, что $ \exists m \in \Z\colon m \leqslant nx < m + 1$. Вообще говоря, $m \eqDef \lfloor nx \rfloor$.
$$M \lrh \{m \in \Z\mid m \leqslant nx\}$$
$$x \geqslant 0 \Ra M \neq \varnothing$$
$$x < 0 \Ra \exists \tilde m \in \N\colon \tilde m-1 > n(-x) \Ra -\tilde m \in M \Ra M \neq \varnothing$$
Рассмторим $ y = 1; x = nx; y > 0$. По принципу Архимеда 
$$ \exists k \in \N\colon k > nx $$
Тогда 
$$\forall m \in M\: m < k \Ra \exists m = \max M\colon m \leqslant nx < m + 1$$
$$m \leqslant nx < m + 1 \Ra \frac{m}n \leqslant x \leqslant \frac{m + 1}n$$
Осталось проверить $\frac{m+1}n < y$.
$$\frac{m}n \leqslant x \land \frac1n < y - x \Ra \frac{m+1}n < y$$
\end{proof}

\begin{conseq}
$x, y \in \R$, $x < y$. 
$$\exists z \in \R \setminus \Q: x < z < y$$
\end{conseq}
\begin{proof}
$$\sqrt{2} \in \R \setminus \Q$$
$$x < y \Ra x - \sqrt{2} < y - \sqrt{2} \Ra \exists r \in \Q: x - \sqrt{2} < r < y - \sqrt{2} \Ra $$
$$\Ra \exists z = r + \sqrt{2}: z \in \R \setminus \Q: x < z < y$$
\end{proof}

\section{Верхняя и нижняя граница}

\begin{Def}
$A \subset \R$.\\
$x \in R $ --- верхняя граница $A$, если $$\forall a \in A: a \leqslant x$$
$x \in R $ --- нижняя граница $A$, если $$\forall a \in A: a \geqslant x$$
\end{Def}
\begin{Def}
$A$ ограничено сверху, если
$$\exists x \in R: x \text{~--- верхняя граница} A$$
$A$ ограничено снизу, если 
$$\exists x \in R: x \text{~--- нижняя граница} A$$
$A$ ограничено, если $A$ ограничено сверху и снизу.
\end{Def}
\begin{Rem}
Границ, если они есть, много.
\end{Rem}
\begin{Def} $A \subset \R$, $A$ ограничено сверху.
$x$~--- супремум $A$, если $x$ --- наименьшая из верхних границ.\\
\end{Def}
\begin{Def}
$A \subset \R$, $A$ ограничено снизу.
$x$~--- инфимум $A$, если $x$ --- наибольшая из нижних границ.
\end{Def}

Пример:
$$A = \left\{1, \frac12, \frac13, \frac14, \cdots\right\}$$
$$\sup A = 1, \inf A = 0$$

\begin{assertion}
$\N$ не ограничено сверху.
\end{assertion}
\begin{proof}
$x\text{~--- верхняя граница }\Ra \exists n \in \N: n > x$.
\end{proof}

\begin{theorem}{Существование точной границы}
$A \neq \emptyset$.
\begin{enumerate}
\item Если $A$ ограничено сверху, то $\exists x = \sup A$.
\item Если $A$ ограничено снизу, то $\exists x = \inf A$.
\end{enumerate}
\end{theorem}
Эта теорема равносильна аксиоме полноты.
\begin{proof}
\begin{enumerate}
\item $B$~--- множество всех верхних границ $A$.
$$\forall a \in A\: \forall b \in B\: a \leqslant b \Ra \exists c \in \R\colon \forall a \in A\: a \leqslant c \land \forall b \in B\: c \leqslant b \Ra \exists \sup A = c$$
\item Рассмотрим $ B = \{-a : a \in A\}$. Тогда $$\inf A = -\sup B$$
\end{enumerate}
\end{proof}

\begin{Rem}
Без аксиомы полноты это неверно. Рассмотрим $ A = \{x \in \Q : x^2 < 2\}, U = \Q$
\end{Rem}

\begin{theorem}{Свойство и признак точной границы}
\begin{enumerate}
\item $A$ ограничено сверху. Тогда $$b = \sup A \Lra (\forall a \in A\: a \leqslant b \land \forall \epsilon > 0\: \exists a \in A\colon a > b - \epsilon)$$
\item $A$ ограничено снизу. Тогда $$c = \inf A \Lra (\forall a \in A\: a \geqslant c \land \forall \epsilon > 0\: \exists a \in A\colon a < c + \epsilon)$$
\end{enumerate}
\end{theorem}
\begin{proof}
$$b = \sup A \Lra (b \text{~--- верхняя граница } A \land \forall \epsilon > 0\: b - \epsilon\text{~--- не верхняя граница}) \Lra $$ 
$$ \Lra (\forall a \in A\: a \leqslant b \land \forall \epsilon > 0\: \exists a \in A\colon a > b - \epsilon)$$
\end{proof}

\begin{theorem}{Теорема о вложенных отрезках}
Вместе с теоремой Архимеда выводят полноту.
$\left\{\left[a_n, b_n\right]\right\}_{i=1}^n: \forall i \in \N \left(a_i <= a_{i + 1} \land b_i >= b_{i + 1}\right) \land \forall i, j \in \N a_i < b_j$. 
Тогда 
$$\bigcap_{i=1}^\infty [a_i, b_i] \neq \emptyset$$
\end{theorem}
\begin{proof}
$A = \{a_i\}, B = \{b_i\}$. 
Тогда по аксиоме полноты 
$$\exists c \in \R\colon \forall i \in \N\: c \in \left[a_i, b_i\right] \Ra c \in \bigcap_{i=1}^\infty [a_i, b_i] \neq \emptyset$$
\end{proof}

\begin{Rem} 
Существенна замкнутость отрезков.
$$\bigcap_{n=1}^\infty \left(0, \frac1n\right] = \emptyset$$
\end{Rem}
\begin{Rem} 
Не лучи.
$$\bigcap_{n=1}^\infty \left[n, +\infty\right) = \emptyset $$
\end{Rem}
\begin{Rem} 
$\R$. Рассмотрим приблежения $\sqrt{2}$.
\end{Rem}

\chapter[Посл-ти в метрических пространствах]{Последовательности в метрических пространствах}

\section{Метрические пространства}

\begin{Def}
Пусть есть множество $X$ и отображение $\rho \colon X \times X \ra \left[0; +\infty\right) $. Тогда $\rho$ называется метрикой, если:
\begin{enumerate}
\item $\rho(x, y) = 0 \Lra x = y$
\item $\rho(x, y) = \rho(y, x)$
\item $\rho(x, y) + \rho(y, z) \geqslant \rho(x, z)$
\end{enumerate} 
Также пара $(X, \rho)$ называется метричесикм пространством.
\end{Def}

Примеры:
\begin{enumerate}
\item Дискретная метрика
$\rho(x, y) = \begin{cases}0 & x \ne y \\ 1 & x = y\end{cases}$
\item $\rho(x, y) = \left|x - y\right|$
\item Евклидовская метрика. $\rho$ --- длина отрезка на плоскости между точками
\item Манхеттанская метрика. $\rho\left((x_1, y_1), (x_2, y_2)\right) = |x_1 - x_2| + |y_1 - y_2|$
\item Расстояния на сфере.
\item Французская железнодорожная метрика. Есть центр --- точка $O$. Тогда для точек на одном луче из $O$ расстояние $\rho(A, B) = |AB|$, иначе $\rho(A, B) = |AO| + |BO|$/
\item Пространство $\R^n$, метрика $$\rho(x, y) = \sqrt{\sum_{i=1}^n \left(x_i-y_i\right)^2}$$
\end{enumerate}

\begin{Def}
Пусть $(X, \rho)$ --- метрическое пространство. Тогда $(Y, \rho|_{Y \times Y})$ --- подпространство X. $Y \subset X$.
\end{Def}


\begin{Def}
$B_r(a) = \left\{x \in X \mid \rho(x, a) < r\right\}$ --- открытый шар.
\end{Def}
\begin{Def}
$\bar B_r(a) = \left\{x \in X \mid \rho(x, a) \leqslant r\right\}$ --- замкнутый шар.
\end{Def}

Свойства:
\begin{enumerate}
\item $B_{r_1}(a) \cap B_{r_2}(a) = B_{\min\{r_1, r_2\}}(a)$
\item $x \ne y \Ra \exists r > 0\colon B_r(x) \cap B_r(y) = \varnothing$
\begin{proof}
Рассмотрим $r = \frac13 \rho(x,y) > 0$.
\end{proof}
\end{enumerate}
         

\begin{theorem}{Неравенство Коши-Буняковского}
$a_1, a_2, \ldots a_n, b_1, b_2, \ldots, b_n \in \R$
$$\left(\sum_{k=1}^n a_kb_k\right){}^2 \leqslant \sum_{k=1}^n a_k^2 \sum_{k=1}^n b_k^2 $$
\end{theorem}
\begin{proof}
$$f(t) \lrh \sum_{k=1}^n(a_kt-b_k)^2 = \left(\underbrace{a_1^2 + a_2^2 + \ldots + a_n^2}_{\lrh A}\right)t^2 - 
2\left(\underbrace{a_1b_1 + \ldots + a_nb_n}_{\lrh C}\right)t + \left(\underbrace{b_1^2 + \ldots + b_2^2}_{\lrh B}\right)$$
$f$ имеет не более 1 корня, следовательно
$$ (2C)^2 - 4AB \leqslant 0 \Ra 4\left(C^2 - AB\right) \leqslant 0 \Lra C^2 \leqslant AB$$
Можно считать, что все числа не равны 0~--- иначе всё тривиально.
\begin{Rem}
Равентсво в случае, если числа пропорциональны.
\end{Rem}
\begin{proof}
$$a_i = \alpha b_i$$

$\Lra$

$$C^2 = AB \Lra \text{есть корень} t_0 \Lra \forall a_k t_0 - b_k = 0$$
\end{proof}
\end{proof}

\begin{theorem}{Неравенство Минковского}
$$\sqrt{\sum_{i=1}^n (a_i+b_i)^2} \leqslant \sqrt{\sum_{i=1}^k a_i^2} + \sqrt{\sum_{i=1}^k b_i^2}$$
\end{theorem}
\begin{proof}
Возведём в квадрат
$$ \sqrt{\sum_{i=1}^n (a_i+b_i)^2} \leqslant \sqrt{\underbrace{\sum_{i=1}^k a_i^2}_{\lrh A}} + \sqrt{\underbrace{\sum_{i=1}^k b_i^2}_{\lrh B}} \Lra \sum_{i=1}^n (a_i + b_i)^2 \leqslant A + 2\sqrt{AB} + B \Lra$$
$$ \Lra A + B + 2\sum_{i=1}^n a_ib_i \Lra A + B + 2\sqrt{AB} \Lra \sum_{i=1}^n a_ib_i \leqslant \sqrt{AB} \La$$
$$ \La \text{Неравенство Коши-Буняковского}$$
\begin{Rem}
Равентсво в случае, если числа пропорциональны.
\end{Rem}
\end{proof}      

\begin{Def}
$(X, \rho)$ --- метрическое пространство. $G \subset X$ --- открытое множество, если $$\forall x \in G\: \exists r > 0\colon B_r(x) \subset G$$
\end{Def}

\begin{theorem}{О свойтсвах открытых множеств}
Пусть $(X, \rho)$ --- метрическое пространство.
\begin{enumerate}
\item $\varnothing$ и $X$ --- открыты.
\item Объединение открытых открыто.
\item Пересечение \textbf{конечного числа} открытых открыто.
\item $B_r(a)$ открыт.
\end{enumerate}
\end{theorem}
\begin{proof}
\begin{enumerate}
\item Очевидно.
\item $$x \in \bigcup G_\alpha \Ra \exists \alpha_0 \colon x \in G_{\alpha_0} \Ra \exists r > 0: B_r(x) \in \bigcup G_\alpha$$
\item $x \in \bigcap_{k=1}^n G_k$ 
$$ \forall k=1..n\: x \in G_k \Ra \forall k=1..n\: \exists r_k > 0\colon B_{r_k}(x) \in G_k \Ra \exists r = \min{r_k}\colon G_r \in \bigcap_{k=1}^n G_k$$
\item $$\forall x \in B_r(a)\: \exists r_x = \frac12 \left(r - \rho(a, x)\right)$$
$$y \in B_{r_x}(x) \Ra \rho(y, x) < r_x \Ra \rho(y, x) + \rho(a, x) < r_x + \rho(a, x) \Ra \rho(y, a) < r$$
\end{enumerate}
\end{proof}

\begin{Rem}
$$\bigcap_{n=1}^\infty \left(0; 1 + \frac1n\right) = \left(0;1\right] \text{ --- не открытое множество}$$
\end{Rem}

\begin{Def}
$x \in A$ --- внутренняя точка $A$, если $\exists r > 0\colon B_r(x) \in A$
\end{Def}
\begin{Rem}
$x$ --- внутренняя точка $A$ эквивалентно тому, что в $A$ содержится некое открытое множество, содержащее x.
\end{Rem}
\begin{Def}
Внутренность множества $A$:
$$A^0 = \Int A \eqDef \bigcup_{\substack{G \text{ открыто} \\ G \subset A}} G$$
\end{Def}

Свойства:
\begin{enumerate}
\item $\Int A \subset A$
\item $\Int A$ --- множество всех внутренних точек.
\item $\Int A$ открыто.
\item $A \text{ открыто} \Lra A = \Int A$
\item $A \subset B \Ra \Int A \subset \Int B$
\item $\Int (A \cap B) = \Int A \cap \Int B$
\item $\Int \Int A = \Int A$
\end{enumerate}
\begin{Def}
Закрытое множество --- множество, дополнение которого открыто.
\end{Def}

\begin{theorem}{О свойствах закмнутых множеств}
Пусть $(X, \rho)$ --- метрическое пространство.
\begin{enumerate}
\item $\varnothing$ и $X$ --- закмнуты.
\item Перечечение замкнутых --- замкнуто.
\item Объеднинение конечного числа замкнутых замкнуто.
\item Замкнутый шар замкнут.
\end{enumerate}
\end{theorem}
\begin{proof}
\begin{enumerate}
\item Очевидно
\item По формулам де Моргана
$$X \setminus \bigcap_{\alpha \in I} F_\alpha = \bigcup_{\alpha \in I} \left(X \setminus F_\alpha \right)$$
\item По формуле де Моргана
$$$$
\item Докажем, что $X \setminus \bar B_r(a)$ открыт. Рассмотрим $x \in X \setminus \bar B_r(a)$. Тогда по определению $$\rho(a, x) > r$$
Покажем, что $$B_{\rho(a, x) - r}(x) \cap \bar B_r(a) = \varnothing$$
Пусть $\exists y \in B_{\rho(a, x) - r}(x) \cap \bar B_r(a)$. Тогда
$$y \in \bar B_r(a) \Ra \rho(a, y) \leqslant r$$
$$y \in B_{\rho(a, x) - r}(x) \Ra \rho(x, y) < \rho(a, x) - r$$
$$\rho(a, x) \leqslant \rho(a, y) + \rho (x, y) < r + (\rho(a, x) - r) = \rho(a, x) \text{ --- противоречие}$$
\end{enumerate}
\end{proof}
\begin{Rem}
$$\bigcup_{n=1}^\infty \left[\frac1n;1\right] = \left(0; 1\right]$$
\end{Rem}

\begin{Def}
$A \subset X$, $(X, \rho)$. Тогда замыкание множества $A$ --- перечесение всех замкнутых множеств, содержащих A.
$$\cl A = \bigcap_{\substack{F \text{ замкнуто}\\F \supset A}}F$$
\end{Def}

\begin{theorem}{О связи замыкания и внутренности}
$$X \setminus \cl A = \Int (X \setminus A)$$
$$X \setminus \Int A = \cl (X \setminus A)$$
\end{theorem}
\begin{proof}
$$X \setminus \cl A = X \setminus \bigcap_{\substack{F \text{ замкнуто}\\F \supset A}} = \bigcup_{\substack{F \text{ замкнуто}\\F \supset A}} (X \setminus F)$$
$$X \setminus F \text{ открыто}$$
$$X \setminus F \subset X \setminus A$$
То
$$\bigcup_{\substack{F \text{ замкнуто}\\F \supset A}} (X \setminus F) = \bigcup_{\substack{G \text{ открыто}\\G \subset X \setminus A}} G = \Int (X \setminus A)$$
Аналогично
\end{proof}
\begin{conseq}
$$ \Int A = \cl (X \setminus A)$$
$$ \cl A = \Int (X \setminus A)$$
\end{conseq}

Свойства замыкания:
\begin{enumerate}
\item $A \subset \cl A$
\item $cl A$ замкнуто.
\item $A \text{ замкнуто} \Lra A = \cl A$
\item $A \subset B \Ra \cl A \subset \cl B$
\item $\cl (A \cup B) = \cl A \cup \cl B$
\item $\cl \cl A = \cl A$
\end{enumerate}

\begin{theorem}{Существование открытого/замкнутого надмножества в надпространстве}
$(X; \rho)$ --- пространство, $(Y; \rho)$ --- подпространство.
\begin{enumerate}
\item $A \text{ открыто в } Y \Lra \exists G \subset X \text{ --- открытое в } X\colon A = G \cap Y$ 
\item $A \text{ замкнутыо в } Y \Lra \exists F \subset X \text{ --- замкнутое в } X\colon A = F \cap Y$ 
\end{enumerate}
\end{theorem}
\begin{proof}
\begin{enumerate}
\item $\Ra$:
$$A \text{ открыто в } Y \Lra \forall x \in A\: \exists r_x > 0\colon B_{r_x}^Y(x) \subset A$$
$$G \lrh \bigcup_{x \in A} B_{r_x}^X(x) \text{ --- открыто в } X$$
$$G \cap Y = \bigcup_{x \in A} \left(B_{r_x}^X(x) \cap Y\right) = \bigcup_{x \in A} B_{r_x}^Y(x) = A$$
$\La$:
$$x \in A \subset G \Ra \exists r > 0\colon B_r^X(x) \subset G$$
$$B_r^Y(x) = B_r^X(x) \cap Y \subset G \cap Y = A$$
\item Перейдём к доплнениям
\end{enumerate}
\end{proof}

\begin{theorem}{О замыканиях}
$(X, \rho)$, $A \subset X$
$$x \in \cl A \Lra \forall r>0\: B_r(x) \cap A \ne \varnothing$$
\end{theorem}
\begin{proof}
$\Ra$: Пусть $\exists r > 0\colon B_r(x) \cap A = \varnothing$. Тогда
$$B_r(x) \subset X \setminus A$$
$$X \setminus B_r(x) \text{ замнкуто}$$
$$X \setminus B_r(x) \supset A$$
$$x \notin X \setminus B_r(x)$$
Тогда
$$ \cl A \subset X \setminus B_r(x)$$
Но тогда
$$x \notin \cl A$$
$\La$: Пусть $x \notin \cl A \Ra \exists F \supset A\colon x \notin F \land F \text{ закрыто}$. Тогда
$$x \in X \setminus F \text{ --- открытое} \Ra \exists r > 0\colon B_r(x) \subset X \setminus F \Ra \exists r>0\colon B_r(x) \cap A = \emptyset $$
\end{proof}
\begin{conseq}
$ U \text{ открытое} \land U \cap A = \varnothing \Ra U \cap \cl A = \varnothing$
\end{conseq}
\begin{proof}
Пусть $x \in U \cap \cl A$.
$$x \in \cl A \Ra \forall r > 0\: B_r(x) \cap A \ne \varnothing$$
$$x \in U \Ra \exists r_0 > 0\colon B_{r_0} \subset U$$
Но $B_{r_0}(x) \cap A \ne \varnothing \Ra U \cap A \ne \varnothing$
\end{proof}

\begin{Def}
Проколотая окрестность точки:
$$\dot B_r(x) = B_r(x) \setminus \{x\}$$
\end{Def}
\begin{Def}
Точка $x \in X$ предельная у множества $A$, если
$$\forall r > 0\: \dot B_r(x) \cap A \ne \varnothing$$
\end{Def}
\begin{Def}
$A'$~--- множество предельных точек.
\end{Def}

Свойства:
\begin{enumerate}
\item $\cl A = A \cup A'$
\item $A \subset B \Ra A' \subset B'$
\item $(A \cup B)' = A' \cup B'$
\begin{proof}
$\supset$:
$$A \cup B \supset A \Ra (A \cup B)' \supset A'$$
$$A \cup B \supset B \Ra (A \cup B)' \supset B'$$
Тогда $$(A \cup B)' \supset A' \cup B'$$
$\subset$: Пусть $x \in (A \cup B)' \land x \notin B'$.
$$x \in (A \cup B)' \Ra \forall r > 0\: B_r(x) \cap (A \cup B) \ne \varnothing$$
$$x \notin B' \Ra \exists r_0 > 0\colon \dot B_{r_0}(x) \cap B = \varnothing \Ra \forall r \leqslant r_0\: \dot B_r(x) = \varnothing$$
Тогда $$\forall r > 0\: \dot B_r(x) \cap A \ne \varnothing \Ra x \in A'$$
\end{proof} 
\end{enumerate}

\begin{theorem}{Об окрестности предельной точки}
$$x \in A' \Lra \forall r > 0 \left|B_r(x) \cap A\right| = \infty$$
\end{theorem}
\begin{proof}
$$x \in A' \Ra \dot B_r(x) \cap A \ne \varnothing \Ra \exists y_1 \in A\colon y_1 \ne x \land y \in B_r(x)$$
Тогда
$$\dot B_{\rho(x,y_1)} \cap A \ne \varnothing \Ra \exists y_2 \in A\colon y_2 \ne x \land y_2 \ne y_1 \land y \in B_{\rho(x,y_1)}$$
Тогда рассмотрим
$$\{y_i\}_{i=1}^\infty\colon y_i \ne y_j \land y_i \ne x \land y_i \in A$$
\end{proof}
\begin{conseq}
$|A| < \infty \Ra A' = \varnothing$
\end{conseq}

\begin{theorem}{О точной границе замкнутого множества}
$$A \text{ ограниченно сверху и замкнуто} \Ra \sup A \in A$$
$$A \text{ ограниченно снизу и замкнуто} \Ra \inf A \in A$$
\end{theorem}
\begin{proof}
$a = \sup A$. Тогда
$$\forall x \in A\: x \leqslant a \land \forall \epsilon > 0\: \exists x \in A\colon x > a - \epsilon$$
Пусть $a \notin A$. Рассмотрим $\dot B_r(a) = (a - r, a + r) \setminus \{a\}$.
$$ \dot B_r(a) \cap A \ne \varnothing \Ra x \in A' \Ra x \in A$$
\end{proof}

\section{Предел последовательности}

\begin{Def}
Пусть есть пространство $(X, \rho)$ и последовательность $(x_i)$. Тогда
$$x^* = \lim_{n\ra\infty} x_n \LraDef x^* \in X \land \forall \epsilon > 0\: \exists N\colon \forall n \geqslant N\: \rho(x^*;x_i) < \epsilon$$
\end{Def}
Примеры:
\begin{itemize}
\item $\lim_{n\ra\infty} x = x$
\item $\R\colon \lim_{n\ra\infty} \frac1n = 0$
\end{itemize}
\begin{Rem}
Определение зависит от метрического пространства, в котором мы находимся. Последнего предела на $(0; +\infty)$ нет. А на метрике
$$\rho(x; y) = \begin{cases}0 & x = y \\ 1 & x \ne y \end{cases}$$ предел есть только у стационарных последовательностей.
\end{Rem}

\begin{theorem}{Свойства предела}
\begin{enumerate}
\item $x^* = \lim_{n\ra\infty} x_n \Lra$ каждая окрестность $x^*$ содержит всю последовательность с некотрого элемента
\item $x^* = \lim_{n\ra\infty} x_n \land x^{**} = \lim_{n\ra\infty} x_n \Ra x^* = x^{**}$
\item $\exists x^* = \lim_{n\ra\infty} x_n \Ra (x_n) \text{ ограниченна}$
\item $x \in A' \Ra \exists (x_n) \subset A\colon \lim_{n\ra\infty} x_n = x$
\end{enumerate}
\end{theorem}
\begin{proof}
\begin{enumerate}
\item $\Ra$: Пусть $x^* \in U$ --- открытое множество. Тогда
$$\exists r > 0\colon B_r(x^*) \subset U$$
$$\forall \epsilon > 0\: \exists N\colon \forall n \geqslant N\: \rho(x^*;x_n) < \epsilon \Ra \exists N\colon \forall n \geqslant N\: x_n \in U$$
$\La$: $U \lrh B_\epsilon(x^*)$.
$$\forall \epsilon > 0\: \exists N\colon \forall n \geqslant N\: x_n \in U \Ra x_* = \lim_{n\ra\infty} x_n$$
\item Пусть $\epsilon \lrh \frac{\rho(x^*;x^{**})}2 > 0$
$$x^* = \lim_{n\ra\infty} x_n \Ra \exists N_1\colon \forall n \geqslant N_1\: \rho(x^*;x_n) < \epsilon$$
$$x^{**} = \lim_{n\ra\infty} x_n \Ra \exists N_2\colon \forall n \geqslant N_2\: \rho(x^{**};x_n) < \epsilon$$
Тогда
$$\forall n \geqslant \max\{N_1; N_2\} \left\{\begin{aligned}\rho(x^*;x_n) < \epsilon \\ \rho(x^{**};x_n) < \epsilon\end{aligned}\right. \Ra$$
$$\Ra 2\epsilon = \rho(x^*;x^{**}) \leqslant \rho(x^*;x_n) + \rho(x^{**}; x_n) < 2\epsilon$$
\item $x^* = \lim_{n\ra\infty} x_n \Ra \exists N\colon \forall n \geqslant N\: \rho(x^{*}; x_n) < 1$. Рассмотрим 
$$R = 1 + \max_{n < N}\{\rho(x^*;x_n)\}$$
Тогда $$\forall n\: x_n \in B_R(x^*)$$
\item $x \in A'$. Рассмотрим 
$$x_1 \in \dot B_1(x) \cap A \ne \varnothing$$
$$x_2 \in \dot B_{\min\left\{\frac12;\rho(x;x_1)\right\}}(x) \cap A \ne \varnothing$$
$$x_3 \in \dot B_{\min\left\{\frac13;\rho(x;x_2)\right\}}(x) \cap A \ne \varnothing$$
$$\vdots$$
$$x_n \in \dot B_{\min\left\{\frac1n;\rho(x;x_n)\right\}}(x) \cap A \ne \varnothing$$
Тогда $$\forall n \geqslant N\: \rho(x; x_n) < \frac1N \Ra x = \lim_{n\ra\infty} x_n$$
\end{enumerate}
\end{proof}
\begin{Rem}
В пункте 4 можно выбрать различные $x_n$.
\end{Rem}
\begin{Rem}
Если $x_n$ --- различные и $x^*$ --- их предел, то $x^* \in \{x_n\}'$
\end{Rem}
\begin{Rem}
$$x = \lim_{n\ra\infty} x_n \land x_n \in A \Ra x \in \cl A$$
\end{Rem}

Далее будем работать с $(\R; |x - y|)$.

\begin{theorem}{Предельный переход в неравенстве}
Пусть $x_n, y_n \in \R; x = \lim x_n; y = \lim y_n; x_n \leqslant y_n$ (или $x_n < y_n$). Тогда $x \leqslant y$.
\end{theorem}
\begin{proof}
Пусть $y < x$; $\epsilon \lrh \frac{x - y}2$. Тогда 
$$\exists N_1: \forall n \geqslant N_1\: |x - x_n| < \epsilon$$
$$\exists N_2: \forall n \geqslant N_2\: |y - y_n| < \epsilon$$
Тогда
$$\forall n \geqslant \max\{N_1, N_2\}\: x_n > x - \epsilon = y + \epsilon > y_n$$
\end{proof}
\begin{Rem}
Понятно, что можно потребовать отношение между последовательностями только с некоторого номера.
\end{Rem}                                                 
\begin{Rem}
Строгие неравенства не сохраняются.
\end{Rem}
\begin{conseq}
$x_n \leqslant b \Ra x \leqslant b$
\end{conseq}
\begin{conseq}
$x_n \geqslant a \Ra x \geqslant a$
\end{conseq}
\begin{conseq}
$x_n \in [a;b] \Ra x \in [a; b]$
\end{conseq}

\begin{theorem}{О двух миллиционерах}
Пусть $x_n \leqslant y_n \leqslant z_n$ и $\lim x_n = \lim z_n = l$. Тогда $\lim y_n = l$.
\end{theorem}
\begin{proof}
Выберем $\epsilon > 0$.
$$\exists N_1\colon\forall n \geqslant N_1 x_n > l - \epsilon$$
$$\exists N_2\colon\forall n \geqslant N_2 z_n < l + \epsilon$$
Тогда
$$\exists N = \max\{N_1,N_2\}\colon\forall n \geqslant N\:l - \epsilon < x_n \leqslant y_n \leqslant z_n < l + \epsilon$$
Тогда $\lim y_n = l$
\end{proof}
\begin{conseq}
$\lim z_n = 0 \land |y_n| \leqslant z_n \Ra \lim y_n = 0$
\end{conseq}
\begin{conseq}
Если $\lim x_n = 0$, а $y_n$ ограниченна, то $\lim x_ny_n = 0$.
\end{conseq}

\begin{Def}
$(x_n)$ нестрого монотонно возрастает, если $$x_1 \leqslant x_2 \leqslant x_3 \leqslant \cdots$$

$(x_n)$ строго монотонно возрастает, если $$x_1 < x_2 < x_3 < \cdots$$

$(x_n)$ нестрого монотонно убывает, если $$x_1 \geqslant x_2 \geqslant x_3 \geqslant \cdots$$

$(x_n)$ строго монотонно убывает, если $$x_1 > x_2 > x_3 > \cdots$$
\end{Def}

\begin{theorem}{Теорема Вейерштрасса}
Монотонная последовательность ограниченна тогда и только тогда, когда имеет предел.
\end{theorem}
\begin{proof}
$\La$: Очевидно.

$\Ra$: Пусть $(x_n)$ возрастает. Она ограниченна, значит есть супремум. Докажем, что это и есть предел. Возьмём $\epsilon > 0$.
$$a = \sup \{x_n\} \Ra \exists x_k\colon x_k > x - \epsilon \Ra a - \epsilon < x_k \leqslant x_{k+1} \leqslant \ldots \leqslant a$$
Тогда $$\forall n \geqslant k\: |x_n - a| < \epsilon$$
\end{proof}
\section{Конечное векторное пространство}

\begin{Def}
Вектор~--- кортеж $x = (x_1, x_2, \ldots, x_d) \in \R^d$. Операция сложения 
$$+\colon \R^d \times \R^d \ra \R^d;x+y = (x_1+y_1, x_2+y_2, \ldots, x_d + y_d)$$ 
и умножения 
$$\times\colon \R \times \R^d \ra \R^d; \lambda x = (\lambda x_1, \lambda x_2, \ldots, \lambda x_n)$$
\end{Def}
\begin{enumerate}
\item Сложение
\begin{enumerate}
\item Коммутативно
\item Ассоциативно
\item Существует ноль $\vec 0 = \underbrace{(0, 0, \ldots, 0)}_d$
\item Существует обратный элемент
\end{enumerate}
\item $\alpha (x + y) = \alpha x + \alpha y$
\item $(\alpha + \beta) x = \alpha x + \beta x$
\item $(\alpha\beta)x = \alpha(\beta x)$
\item $1x = x$
\end{enumerate}
\begin{Def}
Общее определение векторного пространства~--- 

$$"+": X + X \to X$$

$$"\times": \R \times X \to X$$

Обладает свойствами 1-4 и $1X = X$ 
\end{Def}

\begin{Def}
Скалярное произведение векторов (евклидово):
$$\langle x, y\rangle = \sum_{i=1}^d x_iy_i$$
\end{Def}
Свойства:~%
\begin{enumerate}
\item $\langle x, x\rangle \geqslant 0; \langle x, x\rangle = 0 \Lra x = \vec 0$
\item $\langle \lambda x, y\rangle = \lambda \langle x, y\rangle$
\item $\langle x, y\rangle = \langle y, x\rangle$
\item $\langle x + y, z\rangle = \langle x, z\rangle + \langle y, z\rangle$
\end{enumerate}

\begin{Def}
Общее определение скалярного произведения: $X$~--- веторное пространство. Задана операция $\langle x,y\rangle\colon X \times X \ra \R$ обладающая указынными свойствами.
\end{Def}
Например, если приписать в определение положительную константу~--- ничего не поменяется.

\begin{Def}
(Евклидова) норма:
$$\|x\| = \sqrt{\langle x, x\rangle}$$
\end{Def}
\begin{enumerate}
\item $\|x\| \geqslant 0; \|x\| = 0 \Lra x = \vec 0$
\item $\|\lambda x\| = |\lambda| \|x\|$
\item $|\langle x,y\rangle| \leqslant \|x\|\|y\|$ (нер-во Коши--Вуняковкского)
\item $\|x + y\| \leqslant \|x\| + \|y\|$ (нер-во треугольника)
\item $\|x - z\| \leqslant \|x - y\| + \|y - z\|$ (нер-во Минковского)
\item $\|x - y\| \geqslant \left|\|x\| - \|y\|\right|$
\begin{proof}
$\|x - y\| = \|y - x\|$. Таким образом достаточно показать, что 
$$\|x - y\| \geqslant \|x\| - \|y\| \La \|x - y\| + \|y\| \geqslant \|x\|$$
А это неравнство треугольника.
\end{proof}
\item $\rho(x, y) = \|x - y\|$~--- метрика. Это ровно евклидово пространтво на $\R^d$.
\end{enumerate}

\begin{Def}
Общее определение нормы: $\|x\|\colon X \Ra \R$, обладает свойствами 1, 2 и 4.
\end{Def}
Свойство 3 касается скаляроного произведения, которого может и не быть.

Примеры:~%
\begin{enumerate}
\item $\|x\|_1 = \sum\limits_{k=1}^d |x_k|$
\item $\|x\|_\infty = \max\limits_{k=1..d} |x_k|$
\begin{proof}
$$\|x + y\| = \max_{k=1..d} |x_k + y_k| \leqslant \max_{k=1..d} (|x_k| + |y_k|) = |x_{k_0}| + |y_{k_0}| \leqslant \|x\| + \|y\|$$
\end{proof}
\item $$\|x\|_d = \sqrt[p]{\sum\limits_{k=1}^d |x_k|^p}$$
\end{enumerate}

\section{Арифметические свойства предела}
Пусть есть $(\R^d, \rho)$ со стандартной метрикой и нормой.

\textbf{Утверждение.} $x_n \in \R^d$. $$\lim_{n\ra\infty} x_n = \vec 0 \Lra \lim_{n\ra\infty} \|x_n\| = 0$$
\begin{proof}
$$\lim x_n = 0 \Lra \forall \epsilon > 0\;\exists N\colon\forall n>N\; \|x_n\| < \epsilon \Lra \lim \|x_n\| = 0$$
\end{proof}
\begin{Rem}
$A \subset \R^d\text{ ограниченно} \Lra \exists M\colon \forall x \in A\: \|x\| \leqslant M$
\end{Rem}

\begin{theorem}{Арифметические свойства предела}
$x_n, y_n \in \R^d$, $\lambda \in \R$, $\lim x_n = x_0$, $\lim y_n = y_0$, $\lim \lambda = \lambda_0$.
\begin{enumerate}
\item $\lim (x_n + y_n) = x_0 + y_0$
\item $\lim (\lambda x_n) = \lambda_0x_0$
\item $\lim (x_n - y_n) = x_0 - y_0$
\item $\lim \langle x_n, y_n\rangle = \langle x_0, y_0\rangle$
\item $\lim \|x_n\| = \|x_0\|$
\end{enumerate}
\end{theorem}
\begin{proof}
$$\forall \epsilon > 0\; \exists N_1\colon \forall n > N_1\; \|x_n - x_0\| < \epsilon$$
$$\forall \epsilon > 0\; \exists N_2\colon \forall n > N_2\; \|y_n - y_0\| < \epsilon$$
$$\forall \epsilon > 0\; \exists N_3\colon \forall n > N_3\; |\lambda - \lambda_0| < \epsilon$$
\begin{enumerate}
\item $$\forall \epsilon > 0\;\begin{cases}\|x_n-x_0\| < \epsilon \\ \|y_n-y_0\| < \epsilon\end{cases} \Ra 
\|x_n + y_n - x_0 - y_0\| \leqslant \|x_n - x_0\| + \|y_n - y_0\| < \epsilon + \epsilon = 2\epsilon$$
\item $$\|\lambda_nx_n-\lambda_0x_0\| = \|\lambda_nx_n - \lambda_nx_0 + \lambda_nx_0 - \lambda_0x_0\| \leqslant
\|\lambda_nx_n - \lambda_nx_0\|+\|\lambda_nx_0-\lambda_0x_0\| = $$
$$ = |\lambda_n| \|x_n-x_0\| + |\lambda_n - \lambda_0| \|x_0\| \leqslant
M \|x_n-x_0\| + |\lambda_n - \lambda_0| \|x_0\|$$
Но тогда
$$\forall n > \max{N_1, N_3}\; \begin{cases}\|x_n-x_0\| < \frac{\epsilon}M \\ |\lambda_n - \lambda_0| < \frac{\epsilon}{\|x_0\|}\end{cases} \Ra
\|\lambda_nx_n-\lambda_0x_0\| < \epsilon$$
\item Следствие 1 и 2
\item $x_n = \left(x_n^{(1)}, x_n^{(2)}, \ldots, x_n^{(d)}\right); y_n = \left(y_n^{(1)}, y_n^{(2)}, \ldots, y_n^{(d)}\right)$
Это докажем позже
\item $$0 \leqslant \left|\|x_n\|-\|x_0\|\right| \leqslant \|x_n-x_0\| \longrightarrow 0 \Ra \|x_n\| - \|x_0\| \longrightarrow 0 \Ra \|x_n\| \longrightarrow \|x_0\|$$
\end{enumerate}
\end{proof}

\begin{theorem}{Свойства предела на вещественных}
$x_n, y_n \in \R; \lim x_n = x_0; \lim y_n = y_0$
\begin{enumerate}
\item $\lim (x_n + y_n) = x_0 + y_0$
\item $\lim x_ny_n = x_0y_0$
\item $\lim (x_n - y_n) = x_0 - y_0$
\item $\lim |x_n| = |x_0|$
\item Если $y_n, y_0 \ne 0$, то $\lim \frac{x_n}{y_n} = \frac{x_0}{y_0}$
\end{enumerate}
\end{theorem}
\begin{proof}
Докажем, что $\lim \frac1{y_n} = \frac1{y_0}$.
$$\left|\frac{1}{y_n} - \frac{1}{y_0}\right| = \frac{|y_n - y_0|}{|y_n||y_0|} \lrh A$$
$$\exists N_1\colon \forall n > N_1\; |y_n-y_0| < \frac{|y_0|}2 \Ra |y_n| \geqslant |y_0| - |y_0 - y_n| > |y_0| - \frac{|y_0|}2 = \frac{|y_0|}2$$
Тогда
$$A < \frac{|y_n - y_0|}{\frac{|y_0|}2 |y_0|} < \frac{\frac{\epsilon|y_0|^2}2}{\frac{|y_0|}2 |y_0|}$$
\end{proof}

\begin{Def}
$\{x_n\}$~--- последовательность в $\R^d$. Тогда $\{x_n\}$ сходится в $x_0$ покоординатно, если 
$$x_n=\{x_n^{(1)}, x_n^{(2)}, \ldots, x_n^{(d)}\}\colon \lim x_n^{(i)} = x_0^i$$
\end{Def}
\begin{theorem}{О сходимости покоординатно}
$\{x_n\}$ сходится тогда и только тогда, когда последовательность сходится покоординатно.
\end{theorem}
\begin{proof}
$$\left|x_n^{(i)} - x_0^{(i)}\right| \leqslant \sqrt{\sum_{i=1}^d \left(x_n^{(i)} - x_0^{(i)}\right)^2} \leqslant \sum_{i=1}^d \left(x_n^{(i)} - x_0^{(i)}\right)$$
\end{proof}
\begin{conseq}
$x_n \ra x_0, y_n \ra y_0$. Тогда $\langle x_n, y_n\rangle \ra \langle x_0, y_0\rangle$
\end{conseq}
\begin{proof}
$$\left.\begin{array}{rr} x_n \ra x_0 \Ra x_n^{(i)} \ra y_n^{(i)}\\y_n \ra y_0 \Ra y_n^{(i)} \ra y_0^{(i)}\end{array}\right\} \Ra x_n^{(i)}y_n^{(i)} \ra x_0^{(i)}y_0^{(i)}$$
Тогда $$\sum_{i=1}^d x_n^{(i)} y_n^{(i)} \ra \sum_{i=1}^d x_0^{(i)} y_0^{(i)} \Lra \langle x_n, y_n\rangle \ra \langle x_0, y_0\rangle$$
\end{proof}

\section{Бесконечно малые и большие}

\begin{Def}
$$\lim x_n = +\infty \LraDef \forall E\; \exists N\colon \forall n > N\; x_n > E$$
$$\lim x_n = -\infty \LraDef \forall E\; \exists N\colon \forall n > N\; x_n < E$$
$$\lim x_n = \infty \LraDef \forall E\; \exists N\colon \forall n > N\; \left|x_n\right| > E$$
\end{Def}
\begin{Rem}
$$\left[\begin{array}{ll}\lim x_n = +\infty\\\lim x_n = -\infty\end{array}\right.\Ra \lim x_n = \infty$$
Также заметим, что обратное неверно ($x_n = (-1)^n n$).
\end{Rem}

\begin{Rem}
$\lim x_n = \infty \Ra x_n\text{ неограниченна}$
\end{Rem}
\begin{Rem}
Единтсвенность предела справедлива и расширенная на $\pm \infty$.
\end{Rem}
\begin{Rem}
Теорема о двух миллиционерах справедлива и для бесконечно больших.
\end{Rem}

\begin{Rem}
${\bar\R} = \R \cup \{+\infty, -\infty\}$
\begin{enumerate}
\item $\pm c+\pm\infty = \pm\infty$
\item $\pm c-\pm\infty = \mp\infty$
\item $c>0\colon \pm \infty \times c = \pm \infty$
\item $c<0\colon \pm \infty \times c = \mp \infty$
\item $c>0\colon \frac{\pm \infty}{c} = \pm \infty$
\item $c<0\colon \frac{\pm \infty}{c} = \mp \infty$
\item $\frac{c}{\pm \infty} = 0$
\item $(+\infty) + (+\infty) = +\infty$
\item $(+\infty) - (-\infty) = +\infty$
\item $(-\infty) + (-\infty) = -\infty$
\item $(-\infty) - (+\infty) = -\infty$
\item $\pm \infty \times (+ \infty) = \pm \infty$
\item $\pm \infty \times (- \infty) = \mp \infty$
\end{enumerate}
\end{Rem}

\begin{Def}
Последовательность называют бесконечно большой, если её предел бесконечнен.
\end{Def}
\begin{Def}
Последовательность называют бесконечно малой, если её предел равен нулю.
\end{Def}

\begin{theorem}{О связи бесконечно больших и малых}
Пусть $x_n \ne 0$. Тогда
$$x_n \ra \infty \Lra \frac1{x_n} \ra 0$$
\end{theorem}
\begin{proof}
$$x_n \ra \infty \Lra \forall E > 0\; \exists N\colon \forall n > N\; |x_n| > E \Lra \forall \epsilon > 0\; \exists N\colon \forall n > N\; |\frac1{x_n}| < \epsilon \Lra 
\frac1{x_n} \ra 0$$
\end{proof}

\begin{theorem}{Об арифметических действиях с бесконечно малыми}
Пусть $\{x_n\}$, $\{y_n\}$~--- бесконечно малые, $\{z_n\}$ ограниченна. Тогда
\begin{enumerate}
\item $x_n \pm y_n$~--- бесконечно малая
\item $x_n z_n$~--- бесконечно малая
\end{enumerate}
\end{theorem}
\begin{theorem}{Об арифметических действиях с бесконечно большими}
\begin{enumerate}
\item $x_n \ra +\infty \land y_n\text{ ограниченна снизу} \Ra x_n + y_n \ra +\infty$
\item $x_n \ra -\infty \land y_n\text{ ограниченна сверху} \Ra x_n + y_n \ra -\infty$
\item $x_n \ra \infty \land y_n\text{ ограниченна} \Ra x_n + y_n \ra +\infty$
\item $x_n \ra \pm\infty \land y_n\geqslant a > 0 \Ra x_n y_n \ra +\infty$
\item $x_n \ra \pm\infty \land y_n\leqslant a < 0 \Ra x_n y_n \ra -\infty$
\item $x_n \ra \infty \land \left|y_n\right| \geqslant a > 0 \Ra x_ny_n \ra \infty$
\item $x_n \ra a \ne 0 \land y_n \ra 0 \land y_n \ne 0 \Ra \frac{x_n}{y_n} \ra \infty$
\item $x_n\text{ ограниченна} \land y_n \ra \infty \Ra \frac{x_n}{y_n} \ra 0$
\item $x_n \ra \infty \land y_n\text{ ограниченна} \land y_n \ne 0 \Ra \frac{x_n}{y_n} \ra \infty$
\end{enumerate}
\end{theorem}

\begin{Rem}
$$\lim x_n = l \in \bar \R \land l > 0 \Ra \exists a > 0\colon \exists N\colon \forall n > N\; x_n \geqslant a$$
$$\lim x_n = l \in \bar \R \land l < 0 \Ra \exists a < 0\colon \exists N\colon \forall n > N\; x_n \leqslant a$$
\end{Rem}

\section{Компактность}

\begin{Def}
Множество $A$ имеет покрытие множествами $B_\alpha$, если $A \subset \bigcup_{\alpha \in A} B_\alpha$.
\end{Def}
\begin{Def}
Множество $A$ имеет открытое покрытие открытыми множествами $B_\alpha$, если $A \subset \bigcup_{\alpha \in A} B_\alpha$.
\end{Def}
\begin{Def}
Множество $A$ компактно, если из любого его открытого покрытия можно выбрать конечное подкокрытие.
$$\forall B_\alpha\colon K \subset \bigcup_{\alpha \in A} B_\alpha\; \exists \alpha_1, \alpha_2, \ldots, \alpha_n\colon K\subset \bigcup_{i=1}^{n} B_{\alpha_i}$$
\end{Def}

\begin{theorem}{Компактность и подпространства}
Пусть $(X, \rho)$~--- метрическое пространство, $K \subset Y \subset X$. Тогда 
$$K\text{ компактно в } (X, \rho) \Lra K\text{ компактно в } (Y, \rho)$$
\end{theorem}
\begin{proof}
$\Ra$: Пусть $B_\alpha$~--- открытое в $Y$, что 
$$K \subset \bigcup_{\alpha \in A} B_\alpha = \bigcup_{\alpha \in A} (G_\alpha \cap Y) \subset \bigcup_{\alpha \in A} G_\alpha$$
Тогда можно заменить покрытие в $Y$ покрытием соотвествующими множествами в $X$, выбрать конечное подпокрытие, а потом перейти обратно в $Y$.

$\La$: Пусть $K = \bigcup_{\alpha \in I} G_\alpha$. Тогда 
$$K = K \cap Y \subset \left(\bigcup_{\alpha \in I} G_\alpha\right) \cap Y = \bigcup_{\alpha \in I} \left(G_\alpha \cap Y\right)$$
Получим покрытие в пространстве $Y$, в нём есть конечное подпокрытие. Выберем соответствующие шарики из $X$.
\end{proof}

\begin{Rem}
Например, $(0, 1)$ не компактно. Например, из $$\bigcup_{i=2}^\infty \left(\frac1i, 1\right)$$ не выбрать.
\end{Rem}

\begin{theorem}{Свойства компактного множества}
Если $K$ компактно, то $K$ замкнуто и ограниченно.
\end{theorem}
\begin{proof}
$$K \subset \bigcup_{n=1}^\infty B_n(x) \Ra K \subset \bigcup_{i=1}^k B_{r_i}(x) \Ra K \subset B_{R}(x) \Lra K\text{ ограниченно}$$
Возьмём произвольный $a \notin K$. Тогда                                                                
$$K \subset \bigcup_{x\in K} B_{\frac12\rho(a, x)}(x) \Ra K \subset \bigcup_{i=1}^k B_{\frac12 \rho(a, x_i)}(x_i)$$
Но ($r \lrh \min_{i=1}^k\left\{\frac12 \rho(a, x_i)\right\}$)
$$\forall i=1..k\; B_r(a) \cap B_{\frac12 \rho(a, x_i)}(x_i) = \varnothing \Ra B_r(a) \cap \bigcup_{i=1}^k B_{\frac12 \rho(a, x_i)}(x_i) = \varnothing$$
Но $K \subset \bigcup_{i=1}^k B_{\frac12 \rho(a, x_i)}(x_i)$. Т. о. $B_r(a) \cap K = \varnothing$.
\end{proof}

\begin{theorem}{Признак компактного множества}
Замкнутое подмножество компактного компактно.
\end{theorem}
\begin{proof}
Добавим к покрытию подмножества $X \setminus K_1$.
\end{proof}

\begin{theorem}{Пересечение компактных}
Дан набор компактных множеств, любое конечное пересечение которых не пусто. Тогда их пересечение не пусто.
\end{theorem}
\begin{proof}
$K_0$~--- любое их них. Пусть пересечение всех пусто. 
$$\bigcap_{\alpha\in I} K_\alpha = \emptyset$$
Тогда 
$$\bigcup_{\alpha\in I} \left(X \setminus K_\alpha\right) \supset K_0$$
Но тогда можно выбрать конечное покрытие. Тогда 
$$\bigcup_{i=1}^k \left(X \setminus K_{x_i}\right) \supset K_0$$
Но тогда 
$$\bigcap_{i=0}^k K_{x_i} = \emptyset \quad\text{противоречие}$$
\end{proof}

\begin{conseq}
Пусть есть цепочка вложенных непустых компактных. Тогда их пересечение не пусто.
\end{conseq}

\begin{Def}
Параллелепипедом на $\R^d$ и $a, b \in \R^d$ назовём
$$[a, b] = \left\{x \in \R^d \mid \forall i=1..d\: a_i \leqslant x_i \leqslant b_i\right\} \text{ (закрытый)}$$
$$(a, b) = \left\{x \in \R^d \mid \forall i=1..d\: a_i \leqslant x_i \leqslant b_i\right\} \text{ (открытый)}$$
\end{Def}

\begin{theorem}{О вложенных параллелепипедах}
$P_1 \supset P_2 \supset P_3 \supset \ldots$ имеют непустое пересечение.
\end{theorem}
\begin{proof}
Применим теорему о вложенных отрезках по каждой координате.
\end{proof}

\begin{theorem}{Теорема Гейне-Бореля}
Замкнутый куб компактен
\end{theorem}
\begin{proof}
$$I = \left\{x \in \R^d \mid \forall i=1..d\: 0 \leqslant x_i \leqslant a\right\}$$
Рассмотрим произвольное покрытие. Пусть из него нельзя выбрать конечное подпокрытие. Тогда разобъём куб по кажому измерению пополам. Хотя бы один из результирующих не покрываем. 
Повторим процесс до бесконечности. У них есть точка в пересечении. Но она тогда есть покрывающее её множество. Оно открыто, а значит оно покроет ещё и некоторый хвост подкубов.
Ну а тогда возьмём его и все вышестоящие покрытия. Результат конечен и покрыл куб.
\end{proof}

\begin{Def}
Подпоследовательность:
$$\left\{x_{n_i}\right\}_{i=1}^\infty; {n_i} \uparrow$$
\end{Def}

\begin{theorem}{Предел подпоследовательности}
Подпоследовательность имеет тот же предел.
Объединение 2 подпоследовательностей с общим пределом имеет тот же предел.
\end{theorem}

\begin{theorem}{Компактность в $\R^d$}
Следующее в $\R^d$ равносильно:
\begin{enumerate}
\item Компактно
\item Замкнуто и ограниченно
\item Для любой последовательности в множестве можно выбрать подпоследовательность, сходящуюсю к некоторой точке множества (\textit{секвенциально компактно})
\end{enumerate}
\end{theorem}
\begin{proof}
$2 \Ra 1$: $К$ ограниченно, значит можно его ограничить кубом, значит оно подмножество компактного и закрыто, значит компактно.

$1 \Ra 3$:
Возьмём последовательность $\{x_n\}\lrh E$ элементов множества $F$. Если множество элементов $E$ конечно, то какой-то элемент повторился бесконечно. Возьмём новую стационарную последовательность ровно из этого элемента, имеющую предел. Если же оно бесконечно, докажем, что у него есть предельная точка.

Пусть ни одна точка не предельна. Значит 
$$\forall x \in X\: \exists r_x > 0\colon \dot B_{r_x}(x) \cap F = \emptyset$$
Но тогда возьмём покрытие
$$\bigcup_{x\in X} B_{r_x} (x)$$
В нём есть конечное подпокрытие. Возьмём его
$$\bigcup_{i=1}^k \dot B_{r_{y_i}} \supset K \supset E$$
Но также
$$\bigcup \dot B_{r_{y_i}} \cap E = \varnothing$$
Значит 
$$E \subset \bigcup_{i=1}^k \{y_i\}$$
Получили, что $E$ конечное. 

Таким образом предельная точка существует, а значит можно выбрать подпоследовательность можно.

$3 \Ra 2$:
Пусть $K$ не замкнуто. Возьмём предельную точку, которой нет в $K$. Значит есть последовательность, сходящаяся к ней. Из неё нельзя выбрать подпоследовательность, сходящуюся к элементу $K$.

Пусть $K$ не ограничено. Значит есть точка, не лежащая в данном шарике.
$$K \nsubset B_1(a) \Ra \exists x_1\colon \rho(x_1, a) > 1$$
$$K \nsubset B_{\rho(a, x_1) + 1}(a) \Ra \exists x_2\colon \rho(x_2, a) > \rho(x_1, a) + 1$$
$$ \vdots $$
Рассмотрим сходящуюся подпоследовательность. Она ограничена шариком радиуса $R$. Но
$$\rho(a, x_n) > \rho(a, x_{n-1}) + 1 > \cdots > n$$
$$R > \rho\left(b, x_{n_k}\right) > \rho\left(a, x_{n_k}\right) + \rho(a, b) > n_k + \rho(a, b) \ra \infty$$
Значит $K$ ограниченно.  
\end{proof}

\begin{Rem}
$1\Ra 3; 3\Ra 2; 1 \Ra 2$ справедливы для всех пространств. $2 \Ra 1$ ломается, например, на $\R$ с дискретной метрикой.
\end{Rem}

\begin{conseq}
В $\R^d$ компактность $K$ равносильна наличию предельной точки для любого подмножества.  
\end{conseq}
\begin{proof}
В одну сторону просто по теореме.
Обратно: возьмём часть доказательства, объясняющее взятие подпоследовательности.
\end{proof}

\begin{conseq}{Теорема Больцано-Вейерштрасса.}
Из любой ограниченной последовательности в $\R^d$ можно выбрать сходящуюся подпоследовательность.
\end{conseq}
\begin{proof}
Множество значений ограниченно, значит его замыкание компактно, значит в компактном есть сходящаяся подпоследовательность.
\end{proof}

\begin{conseq}
В любой последовательности в $\R$ есть сходящаяся в $\bar \R$ подпоследовательность.
\end{conseq}
\begin{proof}
Если ограничена, то см. предыдущее. Иначе она стремится к бесконечности. Тогда выберем бесконечную подпоследовательность, стремящуюся к бесконечности. В ней бесконечное число положительных или бесконечное число отрицательных.
\end{proof}

\begin{Def}
Диаметр множеста:
$$\diam A = \sup \rho(x, y)$$
\end{Def}

\begin{theorem}{Свойства диаметра}
\begin{enumerate}
\item $\diam E = \diam \cl E$
\item $K_1 \supset K_2 \supset K_3 \ldots(\text{последовательность вложенных компактов}); \diam K_n \ra 0 \Ra \bigcap K_i\text{~--- одноточечное}$
\end{enumerate}
\end{theorem}
\begin{proof}
\begin{enumerate}
\item
$$E \subset \cl E \Ra \diam E \leqslant \diam \cl E$$
$$d = \diam \cl E = \sup \rho(x, y)$$
$$\forall \epsilon > 0\: \exists x_0, y_0\colon \rho(x_0, y_0) > d - \epsilon$$
$$x_0 \in \cl E \Ra \exists x_1 \in E\colon \rho (x_0, x_1) < \epsilon$$ 
$$y_0 \in \cl E \Ra \exists y_1 \in E\colon \rho (y_0, y_1) < \epsilon$$ 
Тогда
$$\rho(x_1, y_1) + 2\epsilon > \rho(x_0, x_1) + \rho(x_1, y_1) + \rho(y_1, y_0) \geqslant \rho(x_0, y_0) > d - \epsilon$$
$$\rho(x_1, y_1) > \rho(x_0, y_0) - 3\epsilon$$
Устремив $\epsilon \ra 0$, получим
$$\diam E \geqslant \diam \cl E$$
\item Пусть в пересечение лежат две точки, но тогда диаметр для любого n хотя бы $\rho(a, b)$. Противоречие.
\end{enumerate}
\end{proof}

\begin{Def}
Последовательность называется фундаментальной, если 
$$\forall \epsilon > 0\: \exists N\colon \forall n,m > N\: \rho(n, m) < \epsilon$$
\end{Def}
\begin{Rem}
$E \lrh \{x_i\}_{i=n}^{\infty}$
$$\{x_n\}\text{ фундаментальная} \Lra \diam E \ra 0$$
\end{Rem}
Свойства фундаментальных последовательностей:
\begin{enumerate}
\item Ограничена
\item Если есть сходящаяся подпоследовательность, то она сходится.
\begin{proof}
$$\forall \epsilon > 0\: \exists K\colon \forall k > K\: \rho(x_{n_k}, a) < \epsilon$$
$$\forall \epsilon > 0\: \exists N\colon \forall n,m > K\: \rho(x_n, x_m) < \epsilon$$

Т.о.
$$\exists n_k > M = \max\{N, K\}\colon \forall n > n_k \rho(x_n, a) \leqslant \rho(x_{n_k}, a) + \rho(x_{n_k}, x_k) < 2\epsilon$$
\end{proof}
\end{enumerate}

\begin{Def}
Пространство называют полным, если любая фундаментальная последовательность имеет предел.
\end{Def}   

\begin{theorem}{О сходимости фундаментальных последовательностей}
\begin{enumerate}
\item Любая сходящаяся последовательность фундаментальна.
\item В $\R^d$ фундаментальная последовательность всегда сходится.
\end{enumerate}
\end{theorem}
\begin{proof}
$\lim x_n = a$
$$\forall \epsilon > 0\: \exists N\colon \begin{aligned}\forall n>N \rho(x_n, a) &< \epsilon \\ \forall m>N \rho(x_m, a) &< \epsilon\end{aligned} \Ra \forall \epsilon>0\: \exists N\colon \forall n,m>N\: \rho(x_m, x_n) < 2\epsilon$$

$x_n$~--- фундаментальная последовательность в $\R^d$. $E_n \lrh \{x_n, x_{n+1}, \ldots\}$~--- ограниченно.
$\cl E_n$~--- ещё и замкнуто. Т.е. компактно.
$$\cl E_1 \supset \cl E_2 \supset \cl E_3 \supset \cdots$$
$$\diam \cl E_n = \diam E_n \ra 0$$

Т.о.
$$\exists! a\colon a \in \bigcap_{i=1}^{\infty} \cl E_n$$
$$a \in \cl E_n \Ra \forall i>n\:0\leqslant\rho(a, x_i) \leqslant \diam E_n \ra 0$$

Т.о $x_n \ra a$.
\end{proof}
                                                
\begin{Rem}
$\R^d$ полно. $\left<\Q, \rho\right>$ не полно. Пространство с дискретной метрикой полно.
\end{Rem}

\begin{theorem}{О полноте компактного пространства}
Компактное метрическое пространство полно.
\end{theorem}
\begin{proof}
В компакте у любой последовательности есть сходящаяся подпоследовательность. А значит любая фундаментальная последовательность имеет сходящуюся подпоследовательность.
А значит она сама сходится. А значит пространство полно.
\end{proof}

\section{Верхний и нижний предел}

\begin{Def}
Верхний и нижний предел
$$\liminf x_n = \varliminf x_n = \lim_{x\ra\infty}\inf_{k>n} x_k$$
$$\limsup x_n = \varlimsup x_n = \lim_{x\ra\infty}\sup_{k>n} x_k$$
\end{Def}
\begin{Rem}
$y_n \lrh \inf_{k>n} x_n$, $z_n \lrh \sup_{k>n} x_n$.
$$y_n<x_n<z_n$$
$$y_n \nearrow{}; z_n \searrow$$
\end{Rem}

\begin{Def}
$a$~--- частичный предел последовательности, если $a$ предел подпоследовательности.
\end{Def}

\begin{lemma}
Если $x_n$ монотонно возрастает и неограничена, то $\lim x_n = +\infty$
\end{lemma}

\begin{theorem}{Существование верхнего и нижнего пределов}
У любой последовательности есть верхний и нижний предел в $\bar\R$, при этом
$$\varliminf x_n \leqslant \varlimsup x_n$$
\end{theorem}
\begin{proof}
$y_n \lrh \inf_{k>n} x_n$, $z_n \lrh \sup_{k>n} x_n$.
Если $x_n$ ограниченно, то и $y_n$ ограниченно. Если $x_n$ не ограниченно снизу, то и $y_n$ не ограниченно снизу. Т.о. $\lim y_n = \varliminf x_n$. Аналогично 
существует верхний предел.
\end{proof}

\begin{theorem}{Верхний и нижний предел и частичные пределы}
\begin{enumerate}
\item $\limsup$~--- наибольший частичный предел.
\item $\liminf$~--- наименьший частичный предел.
\item $\lim\text{ существует}\Lra\varlimsup = \varliminf$
\end{enumerate}
\end{theorem}
\begin{proof}
\begin{enumerate}
\item 
$a = \limsup x_n$. Покажем, что $a$~--- частичный предел.
$$z_n \searrow{} \Ra \sup_{k>n} x_k \geqslant a$$

Выберем $$x_{k_m}\colon x_{k_m} > a - \frac1m; k_{m+1} > k_m$$
Оно стремится к $a$.

Пусть есть больший частичный предел. Но тогда с какого-то места последовательность, сходящаяся к $b$, уйдёт выше супремума, что плохо.
\item Аналогично.
\item Два миллиционера.
\end{enumerate}
\end{proof}

\begin{theorem}{Определение верхнего и нижнего предела через $N$ и $\epsilon$}
\begin{enumerate}
\item $$a = \varliminf x_n \Lra \begin{cases}\forall \epsilon > 0\: \exists N\colon \forall n>N\: x_n > a - \epsilon \\ \forall \epsilon > 0\: \forall N\colon \exists n>N\: x_n < a + \epsilon\end{cases}$$
\item $$a = \varlimsup x_n \Lra \begin{cases}\forall \epsilon > 0\: \forall N\colon \exists n>N\: x_n > a - \epsilon \\ \forall \epsilon > 0\: \exists N\colon \forall n>N\: x_n < a + \epsilon\end{cases}$$
\end{enumerate}
\end{theorem}
\begin{proof}
\begin{enumerate}
\item Запишем в терминах $y_n$:
$$\forall \epsilon > 0\: \exists N\colon \inf_{n>N} > a-\epsilon ; \forall \epsilon > 0\: \exists N\colon \inf_{n>N} < a+\epsilon$$
Уже видно, что эти условия и задают предел.
\item Аналогично.
\end{enumerate}
\end{proof}

\begin{theorem}{О предельном переходе в неравенстве}
$$a_n \leqslant b_n \Ra \begin{cases}\varliminf a_n \leqslant \varliminf b_n \\ \varlimsup a_n \leqslant \varlimsup b_n\end{cases}$$
\end{theorem}                                                                                                                       
\begin{proof}
Просто сводим к пределам инфимумов. 
\end{proof}

\begin{theorem}{Неравенство Бернулли}
$$\forall x>-1\: \forall n \in \N\: (1+x)^n \geqslant 1 + nx$$
\end{theorem}
\begin{proof}
Индукция: база очевидна. Пусть $(1+x)^k \geqslant 1 + nk$. Тогда
$$(1+x)^{k+1} = \underbrace{(1+x)^k}_{>0} (1+x) \geqslant (1+kx)(1+x) = 1 + kx + x + kx^2 \geqslant 1 + (k+1)x$$
\end{proof}
\begin{conseq}
Если $|t| > 1$, то $\lim t^n = +\infty$. Если $|t| < 1$, то $\lim t^n = 0$.
\end{conseq}

\begin{theorem}{Предел убывающей по отношению}
$x_n > 0$, $\lim \frac{x_{n+1}}{x_n} < 1$. Тогда $x_n \ra 0$.
\end{theorem}
\begin{proof}
С какого-то места отношение довольно мало (меньше 1).
\end{proof}
\begin{conseq}
$$\lim_{n\ra\infty} \frac{n^k}{a^n} = 0\quad a>1$$
\end{conseq}
\begin{proof}
$$x_n = \frac{n^k}{a^n}$$
$$\frac{x_{n+1}}{x_n} = \left(\frac{n+1}n\right)^k \frac1a < 1$$
\end{proof}
\begin{conseq}
$$\lim \frac{a^n}{n!} = 0$$
\end{conseq}

Определим число $e$:
$$x_n = \left(1+\frac1n\right)^n; y_n=\left(1+\frac1n\right)^{n+1}$$
Покажем, что $x_n \uparrow; y_n \downarrow$.
\begin{proof}
$$x_n < x_{n+1} \La \frac{(n+1)^n}{n^n} < \frac{(n+2)^{n+1}}{(n+1)^{n+1}} \La \frac{n+1}{n+2} < \frac{n^n(n+2)^n}{(n+1)^{2n}} \La $$
$$ \La \frac{n+1}{n+2} < \left(1-\frac1{n^2+2n+1}\right)^n \La 1 - \frac{1}{n+2} < 1 - \frac{n}{n^2+2n+1} \leqslant \left(1-\frac1{n^2+2n+1}\right)^n$$

$$y_n < y_{n-1} \La \frac{(n+1)^{n+1}}{n^{n+1}} < \frac{n^n}{(n-1)^n} \La \frac{n+1}{n} < \frac{n^{2n}}{(n-1)^n(n+1)^n} \La $$
$$\La \frac{n+1}{n} < \left(1+\frac1{n^2-1}\right)^n \La 1 + \frac1n < 1 - \frac{n}{n^2-1} \leqslant \left(1-\frac1{n^2-1}\right)^n$$
\end{proof}

Заметим, что при этом $x_n < y_n$. Собственно, тогда $\lim x_n$ существует.
$$\lim \left(1+\frac1n\right)^n \lrh e$$

Свойства:
\begin{enumerate}
\item $\lim y_n = e$
\item $x_n < e < y_n$
\end{enumerate}

\begin{conseq}
$$\lim \frac{n!}{n^n} = 0$$
\end{conseq}
\begin{proof}
$$x_n = \frac{n!}{n^n}$$
$$\frac{x_{n+1}}{x_n} = (1+\frac1n)^{-n} \ra \frac1e < 1$$
\end{proof}

\begin{theorem}{Теорема Штольца для бесконечно малых}
$0<y_n<y_{n-1}$, $\lim x_n = \lim y_n = 0$, $\lim \frac{x_n-x_{n-1}}{y_n-y_{n-1}} = a \in \bar\R$.
Тогда 
$$\lim \frac{x_n}{y_n} = a$$
\end{theorem}
\begin{proof}
\begin{enumerate}
\item $a = 0$.
$$\epsilon_n = \frac{x_n - x_{n - 1}}{y_n - y_{n - 1}} \to 0$$
$n > m$:
$$x_n - x_m = \sum_{k = m + 1}^{n}(x_k - x_{k - 1}) = \sum_{k = m + 1}^{n}\epsilon_k(y_k - y_{k - 1})$$

$$|x_n - x_m| = \left|\sum_{k = m + 1}^{n}\epsilon_k(y_k - y_{k - 1})\right| = \sum_{k = m + 1}^{n} |\epsilon_k|(y_{k-1} - y_k) < $$

$$\forall \epsilon < 0\: \exists N\colon \forall k > N\: |\epsilon_k| < \epsilon$$

Тогда при $n > m > N$
$$< \sum^{n}_{k = m + 1}\epsilon(y_{k - 1} - y_k) = \epsilon\sum_{k = m + 1}^{n}(y_{k - 1} - y_k) = \epsilon(y_m - y_n)$$

$$|x_n - x_m| < \epsilon|y_n - y_m|$$
$n \ra \infty$

$$|x_m| < \epsilon y_m$$
$$\frac{x_m}{y_m} < \epsilon$$

\item $a \in \R$: $\tilde x_n = x_n - a y_n$.
\item $a = +\infty$: Поменяем местами $x_n$ и $y_n$. Проверим свойство для $x$:
$$\lim \frac{x_n - x_{n-1}}{y_n - y_{n-1}} = +\infty \Ra \frac{x_n - x_{n-1}}{y_n - y_{n-1}} > 1$$
$$x_n - x_{n-1} < y_n - y_{n-1} < 0$$
\end{enumerate}
\end{proof}

\begin{theorem}{Теорема Штольца для бесконечно больших}
$0<y_n<y_{n+1}$, $\lim x_n = \lim y_n = +\infty$, $\lim \frac{x_n-x_{n-1}}{y_n-y_{n-1}} = a \in \bar\R$.
Тогда 
$$\lim \frac{x_n}{y_n} = a$$
\end{theorem}
\begin{proof}
\begin{enumerate}
\item $a = 0$:
$$\epsilon_n \lrh \frac{x_n - x_{n-1}}{y_n - y_{n-1}}$$
$$x_n = x_1 + \sum_{i=2}^n (x_i - x_{i-1}) = x_1 + \sum_{i=2}^n \epsilon_i(y_i - y_{i-1})$$
$$\frac{x_n}{y_n} = \frac{x_1}{y_n} + \sum_{i=2}^n \epsilon_i \frac{y_i - y_{i-1}}{y_n} = $$
$$\forall \epsilon > 0\: \exists N\colon \forall n > N\: |\epsilon_n| < \epsilon$$
$$= \frac{x_1}{y_n} + \sum_{i=2}^N \epsilon_i \frac{y_i - y_{i-1}}{y_n} + \sum_{i=N+1}^n \epsilon_i \frac{y_i - y_{i-1}}{y_n}$$

$$\left|\sum_{i=N+1}^n \epsilon_i \frac{y_i - y_{i-1}}{y_n}\right| = \sum_{i=N+1}^n |\epsilon_i| \frac{y_i - y_{i-1}}{y_n} < 
\sum_{i=N+1}^n \epsilon \frac{y_i - y_{i-1}}{y_n} <$$
$$< \frac{\epsilon}{y_n}\sum_{i=N+1}^n (y_i - y_{i-1}) = \epsilon \frac{y_n - y_N}{y_n} < \epsilon$$

$$\sum_{i=2}^N \epsilon_i \frac{y_i - y_{i-1}}{y_n} \leqslant \frac{1}{y_n}\sum_{i=2}^N \epsilon_i(y_i - y_{i-1}) < \epsilon$$

$$\frac{x_1}{y_n} < \epsilon$$

Т.о.
$$\left|\frac{x_n}{y_n}\right| < \epsilon \Ra \frac{x_n}{y_n} \ra 0$$

\item $a\in\R$:
$\tilde x_n = x_n - a y_n$. Фактом $x_n\ra\infty$ мы не пользовались.

$$\frac{\tilde x_n - \tilde x_{n - 1}}{y_n - y_{n - 1}} = \frac{(x_n - ay_n) - (x_{n - 1} - ay_{n - 1})}{y_n - y_{n - 1}} = \frac{x_n - x_{n - 1}}{y_n - y_{n - 1}} - a \to 0$$

\item $a=+\infty$: Поменяем местами $x_n$ и $y_n$. Проверим, что $x_n$ монотонно растёт и не ноль.
$$\frac{x_n - x_{n-1}}{y_n - y_{n-1}} = +\infty \Ra \frac{x_n - x_{n-1}}{y_n - y_{n-1}} > 1$$
$$x_n - x_{n-1} > y_n - y_{n-1} > 0$$

\item $a=-\infty$: Сменим знаки $x_n$.
\end{enumerate}
\end{proof}

\chapter{Пределы и непрерывность отображений}
\section{Пределы функций}

\begin{Def}
$(X, \rho_x)$ и $(Y, \rho_y)$~--- метрические пространства. $E \subset X$, $a$~--- предельная точка $E$. $f\colon E \ra Y$.
Тогда $b$ является пределом $f$ в $a$ по Коши
$$\lim_{x \ra a} f(x) = b$$
если $b \in Y$ и
$$\forall \epsilon>0\: \exists \delta > 0\colon \forall x\: x \in \dot B_\delta(a) \cap E \Ra f(x) \in B_\epsilon (b)$$
или, что то же самое
$$\forall \epsilon>0\: \exists \delta > 0\colon \forall x\in E \: (x \ne a \land \rho(x, a) < \delta) \Ra \rho(f(x), b) < \epsilon$$
\end{Def}

\begin{Rem}
Для бесконечности на $\R$ есть частные случаи.
\end{Rem}

\begin{Def}
По Гейне,
$$\lim_{x \ra a} f(x) = b \Lra \forall \{x_n\}\subset E\colon x_n \ne a\: \lim_{n\ra \infty} x_n = a \Ra \lim_{n \ra \infty} f(x_n) = b$$
\end{Def}

\begin{theorem}{Равносильность определений предела функции}
Определения равносильны.
\end{theorem}
\begin{proof}
Коши $\Ra$ Гейне:
$$\forall \epsilon>0\: \exists \delta > 0\colon \forall x\: x \in \dot B_\delta(a) \cap E \Ra f(x) \in B_\epsilon (b)$$
Рассмотрим произвольную $\{x_n\} \subset E \setminus \{a\}$, $\lim_{n\ra\infty} x_n = a$. Для неё выполнено указанное выше. Тогда $\{f(x_n)\}$ сходится к $b$.

Гейне $\Ra$ Коши: от противного. Пусть
$$\exists \epsilon>0\colon \forall \delta > 0\: \exists x\: x \in \dot B_\delta(a) \cap E \land f(x) \notin B_\epsilon(b)$$
Возьмём данный $\epsilon$ и выберем последовательность $\delta_n = \frac1n$. Тогда получим, что
$$\exists \{x_n\}\colon x_n /ne a \land x_n \ra a \land f(x_n) \nrightarrow b$$
что противоречит Гейне.
\end{proof}
\begin{Rem}
Если в определении по Гейне все пределы существуют, то они будут равны. То есть достаточно доказать,
что для любой сходящейся последовательности $\{x_n\}$ предел $f(x_n)$ существует, из этого
будет следовать по Гейне.
\end{Rem}
\begin{proof}
Возьмём две сходящиеся последовательности $x_n$ и $y_n$, после применения функций стремящиеся к каким-то разным значениям $b$ и $c$. Но тогда
у последовательности 
$$x_1, y_1, x_2, y_2, x_3, y_3, \ldots$$
сходящейся к той же точке, будет предел. Но тогда у подпоследовательностей одинаковые пределы.
\end{proof}

\begin{assertion}{Единственность предела}
$f\colon E \subset X \ra Y$, $a$~--- предельная точка. Тогда предел $\lim\limits_{x\ra a} f(x)$ единственнен.
\end{assertion}
\begin{proof}
Пусть есть два различных предела. Тогда из определения по Коши с какого-то расстояния весь хвост должен быть ближе к одному пределу, чем к другому.
\end{proof}

\begin{theorem}{Ограниченность}
$f\colon E \subset X \ra Y$, $\lim\limits_{x\ra a} = b$. Тогда 
$$\exists r>0\colon f \mid_{E \cap B_r(x)}\text{ограничена}$$
\end{theorem}

\begin{theorem}{Уход от нуля}
$f\colon E \ra \R^d$, $\lim\limits_{x\ra a} = b \ne \vec 0$. Тогда
$$\exists r>0\colon \forall x \in \dot B_r(a) \cap E\: f(x) \ne 0$$
\end{theorem}
\begin{proof}
$\epsilon \lrh \rho(x, \vec 0)$
\end{proof}

\begin{theorem}{Арифметические свойства предела функции}
$f, g\colon E \subset \ra \R^d$, $\lambda\colon E \ra \R$, $a$ предельная точка $E$.
\begin{enumerate}
\item $\lim\limits_{x\ra a} (f(x) + g(x)) = f_0 + g_0$
\item $\lim\limits_{x\ra a} (\lambda(x)  g(x)) = \lambda_0 g_0$
\item $\lim\limits_{x\ra a} (f(x) - g(x)) = f_0 - g_0$
\item $\lim\limits_{x\ra a} \left\|f(x)\right\| = \left\|f_0\right\|$
\item $\lim\limits_{x\ra a} \left<f(x), g(x)\right> = \left<f_0, g_0\right>$
\end{enumerate}
\end{theorem}
\begin{proof}
Возьмём любые сходящиеся к $a$ последовательности. Для них будет справедлива теорема об арифметических действиях с пределами последовательности.
\end{proof}

\begin{theorem}{Арифметические свойства предела функции}
$f, g\colon E \subset \ra \R$, $a$ предельная точка $E$.
\begin{enumerate}
\item $\lim\limits_{x\ra a} (f(x) \pm g(x)) = f_0 \pm g_0$
\item $\lim\limits_{x\ra a} (f(x) g(x)) = f_0 g_0$
\item $\lim\limits_{x\ra a} \left|f(x)\right| = \left|f_0\right|$
\item $\lim\limits_{x\ra a} \cfrac{f(x)}{g(x)}=\cfrac{f_0}{g_0}$
\end{enumerate}
\end{theorem}
\begin{proof}
Аналогично.
\end{proof}

\begin{Rem}
Арифметические свойства расширяются на бесконечности.
\end{Rem}

\begin{theorem}{Предельный переход в неравенстве}
$f, g\colon E \ra Y$, $a$ предельная точка $E$, $\forall x \in E\setminus \{a\} f(x) \leqslant g(x)$. Тогда 
$$f_0 \leqslant g_0$$
\end{theorem}

\begin{theorem}{О двух миллиционерах}
$f, g, h\colon E \ra Y$, $a$ предельная точка $E$, $f(x) \leqslant g(x) \leqslant h(x)$, $\lim\limits_{x\ra a} f(x) = \lim\limits_{x\ra a} h(x) = b$. Тогда
$$\lim_{x\ra a} g(x) = b$$
\end{theorem}

\begin{Def}
Пределы слева и справа. $f\colon E \cap \R \ra Y$.
$$\lim\limits_{x\ra a-} = \lim\limits_{x\ra a-0} \eqDef \lim\limits_{x\ra a} f\mid_{E\cap(-\inf,a)}$$
$$\lim\limits_{x\ra a+} = \lim\limits_{x\ra a+0} \eqDef \lim\limits_{x\ra a} f\mid_{E\cap(a,+\inf)}$$
\end{Def}

\begin{theorem}{Существование предела возрастающей и ограниченой функции}

\end{theorem}

\begin{theorem}{Критерий Коши}
Функция с полной областью значений имеет предел в точке тогда и только тогда, когда для любого разброса существует выколотый шарик вокруг предельной точки, все расстояния в котором малы.
\end{theorem}

%\input{theory-09.tex}
\begin{conseq}
$\sin$ и $\cos$ непрерывны. 
\end{conseq}
\begin{proof}
$$\left|\sin x - \sin y\right| = 2 \left|\sin \frac{x-y}2\right| \left|\cos \frac{x+y}2\right| \leqslant \left|x - y\right|$$
\end{proof}

\begin{conseq}
$\tg$ и $\ctg$ непрерывны. 
\end{conseq}

\begin{conseq}
$$\sin \uparrow   \left[-\frac\pi2,\frac\pi2\right]$$
$$\cos \downarrow \left[0,\pi\right]$$
$$\tg  \uparrow   \left(-\frac\pi2,\frac\pi2\right)$$
\end{conseq}

\begin{Def}
$$ \arcsin = \left(\sin \mid_{\left[-\frac\pi2,\frac\pi2\right]}\right)^{-1} $$
$$ \arccos = \left(\cos \mid_{\left[0,\pi\right]}\right)^{-1} $$
$$ \arctg  = \left(\tg  \mid_{\left(-\frac\pi2,\frac\pi2\right)}\right)^{-1} $$
\end{Def}

\begin{theorem}{}
$$\lim_{x \ra 0} \frac{\sin x}x = 1$$
\end{theorem}
\begin{proof}
$0 < x < \frac\pi2$:
$$\sin x < x < \tg x \Ra \frac{\sin x}x < 1 < \frac1{\cos x} \frac{\sin x}x \Ra \cos x < \frac{\sin x}x < 1 \xra{x\ra0} 1 \leqslant \lim_{x\ra0} \frac{\sin x}x \leqslant 1$$
\end{proof}

\subsection{Степенная функция}
$$x^n \quad x \in [0;+\infty); n \in \N$$
Больше нуля, непрерывна, инфимум 0, супремум бесконечен, строго монотонная.
$$x^\frac1n\text{ обратная}$$
Тоже непрерывна.
$$x^{\frac{m}n} = \left(x^\frac1n\right)m$$
$$x^{-\frac{m}n}=\frac1{x^{\frac{m}n}}$$

\begin{assertion}
Определение корректно.
\end{assertion}
\begin{assertion}
Свойства степени выполняются.
\end{assertion}

\begin{lemma}
$$\lim_{n\ra+\infty} a^{\frac1n}$$
\end{lemma}
\begin{proof}
$a \geqslant 1$:
$$(1+\epsilon)^n \geqslant 1 + \epsilon n > \epsilon n > \epsilon N > a$$
$$N > \frac{a}\epsilon \Ra \forall n > N\; (1+\epsilon)^n > a \Ra 1 + \epsilon > a^{\frac1n} \geqslant 1^{\frac1n} = 1$$
$0 < a < 1$:
$$\lim_{n\ra+\infty} a^{\frac1n} = \frac1{\lim_{n\ra+\infty} \left(\frac1a\right)^{\frac1n}} = 1$$
\end{proof}

\begin{theorem}{}
Пусть $\lim_{n\ra+\infty} x_n = x$, $x_n \in \Q$, $a > 0$. Тогда последовательноть $a^{x_n}$ имеет предел, зависящий только от $x$ и $a$.
\end{theorem}
\begin{proof}
$$a^{x_n} - a^{x_m} = a^{x_n}\left(a^{x_m-x_n} - 1\right)$$
$$\forall n\; |x_n| \leqslant M \Ra a^{x_n} \in \left[a^{-M}; a^M\right]$$
Т.о.
$$\left|a^{x_n} - a^{x_m}\right| \leqslant \underbrace{a^M}_{\lrh C} \left(a_{x_n-x_m} - 1\right) < C\epsilon$$
По лемме 
$$\exists N\colon \forall k > N\; |a^{\frac1n} < 1| < \epsilon$$
$$|x_n - x_m| < \frac1N \ra -\epsilon < a^{-\frac1N} < a_{x_n-x_m} - 1 < a^{\frac1N} - 1 < 1 + \epsilon$$
Т.о. предел существует.

Пусть теперь 
$$\lim_{n\ra+\infty} x_n = \lim_{n\ra+\infty} y_n = x \quad \lim_{n\ra+\infty} a^{x_n} \ne \lim_{n\ra+\infty} a^{y_n}$$
Но рассмотрим
$$\left\{z_n\right\} = \left\{x_1, y_1, x_2, y_2, \ldots\right\} \ra x$$
Но тогда $a^{z_n}$ не имеет предела, что противоречит доказанному выше.
\end{proof}

\begin{Def}
$$a^x = \lim_{\substack{x_n \ra x \\ x_n \in \Q}}$$
\end{Def}

Свойства степени:
\begin{enumerate}
\item Для $x \in \Q$ корректно.
\item $x^a x^b = x^{a + b}$
\item $\left(x^a\right)^b = x^{ab}$
\item $x^ay^a = (xy)^a$
\item $x < y \land a > 0 \ra x^a < y^a$
\begin{proof}
$$a_n \ra a > 0 \Ra a_n > 0\text{ с какого-то места}$$
$$x^a_n < x^b_n \Ra x^a \leqslant x^b$$
Теперь хотим строгое
$$\left(\frac{x}y\right)^n < 1$$
$$z\lrh\frac{x}y$$
$$z^{a_n} < 1 \land z^{a_n} \downarrow \Ra z_a < 1$$
\end{proof}
\item $x^a < x^b$ при $x>1 \land a < b$ или $0<x<1 \land a > b$
\begin{proof}
$x>1 \land a < b$:
$$a < p < q < b \quad p,q \in Q$$
$$x^{a_n} < x^p < x^q < x^{b_n}$$
$$x^a \leqslant x^p < x^q \leqslant x^b$$
\end{proof}
\end{enumerate}

\begin{lemma}
$$a > 0 \Ra \lim_{x \ra 0} a^x = 1$$
\end{lemma}
\begin{proof}
$$\forall \epsilon > 0\; \exists N\colon \forall n > N\; \left|a^{\frac1n} - 1\right| < \epsilon$$
$$\forall |x| < \frac1N 1 - \epsilon < \frac1{1 + \epsilon} < a^{-\frac1N} < a^x < a^{\frac1N} < 1 + \epsilon$$
Возьмём $\delta = \frac1N$
\end{proof}

\begin{theorem}{}
$$a > 0 \Ra f(x)\lrh a^x\text{ непрерывна}$$
\end{theorem}
\begin{proof}
Надо доказать, что $a^{\lim_{n\ra+\infty} x_n} = \lim_{n\ra+\infty} a^{x_n}$

$x_0 \lrh \lim_{n\ra+\infty} x_n$
$$a^{x_n}-a^{x_0} = a^{x_0}\left(a^{x_n-x_0} - 1\right) \ra 0$$
\end{proof}

\begin{conseq}
Есть обратная $$\log_a x$$
\end{conseq}

\begin{theorem}{}
$$\lim_{x\ra\infty} \left(1+\frac1x\right)^x = e$$
\end{theorem}
\begin{proof}
$x_n \ra +\infty$. $[x_n] = k$
$$\left(1 + 1\frac{k + 1})^k\right) \leqslant \left(1 + \frac1{x_n}\right)^x_n \leqslant \left(1+\frac1k\right)^{k+1}$$
$x_n \ra +\infty$. $y_n = -x_n$
$$f(x_n) = \left(1 + \frac1{-y_n}\right)^{-y_n} = \left(1 + \frac1{y_n-1}\right)^{y_n} \ra e$$
А для смеси возьмём две части, в каждой есть хороший номер.
\end{proof}
$$\lim_{\epsilon\ra 0} \frac{\sin(x - \epsilon) - \sin x}{\epsilon} = \lim_{\epsilon\ra 0} \frac {2\sin\frac{h}2 \cos (x + \frac{\epsilon}2)} = \cos x$$

\section{Теоремы о среднем}
\begin{theorem}{Теорема Ферма}
$f\colon \left<a, b\right>$, $x_0 \in (a, b)$, $f$ дифференцируема в $x_0$, $x_0$~--- точка экстремума. Тогда
$$f'(x_0) = 0$$
\end{theorem}
\begin{proof}
Пусть $x > x_0$.
$$\lim_{x\ra x_0} \frac{f(x) - f(x_0)}{x - x_0} \geqslant 0$$
Пусть $x < x_0$.
$$\lim_{x\ra x_0} \frac{f(x) - f(x_0)}{x - x_0} \leqslant 0$$
Но тогда
$$f'(x_0) = 0$$
\end{proof}

\begin{theorem}{Теорема Ролля}
$f\colon [a, b] \in \R$, $f$ непрерывна, $f$ дифференцируема на $(a, b)$, $f(a) = f(b)$.
Тогда
$$\exists c\in(a, b)\colon f'(c) = 0$$
\end{theorem}
\begin{proof}
Если функция константна, то всё доказано. Иначе есть глобальный максимум и минимум, причём они не могут быть оба в концах.
\end{proof}

\begin{conseq}
Между корнями функции есть корень производной.
\end{conseq}

\begin{theorem}{Теорема Лагранжа}
$f\colon [a, b] \in \R$, $f$ непрерывна, $f$ дифференцируема на $(a, b)$.
$$\exists c \in (a, b)\colon f(b) - f(a) = (b-a) f'(c)$$
\end{theorem}
\begin{theorem}{Теорема Коши}
$f, g\colon [a, b] \in \R$, $f$ непрерывна, $f$ дифференцируема на $(a, b)$, $g'(x) \ne 0 \ne g(b) - g(a)$.
$$\exists c\colon \frac{f(b) - f(a)}{g(b) - g(a)} = \frac{f'(c)}{g'(c)}$$
\end{theorem}
\begin{proof}
$h(x) = f(x) - Kg(x)$, $h(a) = h(b)$.
$$K = \frac{f(b) - f(a)}{g(b) - g(a)}$$
Тогда
$$\exists c\colon h'(c) = 0$$
$$h'(c) = 0 \Ra K = \frac{f'(c)}{g'(c)}$$
\end{proof}

\begin{conseq}
$f\colon [a, b] \in \R$, $f$ непрерывна, $f$ дифференцируема на $(a, b)$, $|f'(x)| \leqslant M$.
Тогда
$$\forall x,y\in(a,b)\; |f(x) - f(y)| \leqslant M|x-y|$$
\end{conseq}
\section{Формула Тейлора}

\begin{theorem}{Формула Тейлора}
$$T(x) = \sum{i=0}^n \frac{T^{(i)} (x_0)}{i!} (x-x_0)^i$$
\end{theorem}
\begin{proof}
$$T(x) = \sum_{i=0}^n a_k (x-x_0)^k$$
$$((x-x_o)^k)^{(m)} = \begin{cases}0 & k < m \\ m! & k = m \\ k(k-1)(k-2)\cdots(k-m+1)(x-x_0)^{k-m} & k > m\end{cases}$$
$$T(x)^{(m)} = \sum_{i=m}^n a_k k(k-1)(k-2)(k-3)\cdots(k-i+1)(x-x_0)^{k-m}$$
$$T(x_0)^{(m)} = a_m m! $$
$$a_m = \frac {T^{(m)}(x_0)}{m!} $$
\end{proof}

\begin{Def}
$f$ дифференцируема $n$ раз в точке $x_0$. Тогда многочленом Тейлора функции $f$ в точке $x_0$ есть
$$T_{n,x_0} f(x) = \sum_{i=0}^n \frac{f^{(i)} (x)}{k!} (x-x_0)^k$$
\end{Def}

\begin{Def}
Формула Тейлора:
$$f(x) = T_{n, x_0} f(x) + R_{n, x_0} f(x)$$
\end{Def}

\begin{lemma}
$g$ дифференцируема $n$ раз в $x_0$. $g(x_0) = g'(x_0) = g''(x_0) = \cdots = g^{(n)}(x_0) = 0$. Тогда
$$g(x) = o\left((x - x_0)^n\right) при x \ra x_0$$
\end{lemma}
\begin{proof}
$$\lim_{x \ra x_0} \frac{g(x)}{(x-x_0) ^ n} = \lim{x \ra x_0} \frac{g'(x)}{n (x-x_0)^{n-1}} = \cdots = \lim_{x\ra x_0} \frac{g^(n-1)}{n! (x-x_0)}$$
$g^{(n-1)}$ дифференцируема в $x_0$, а значит
$$g^{(n-1)}(x) = g^{(n-1)}(x_0) + g^{(n)}(x_0) (x-x_0) + o(x-x_0) = o(x-x_0)$$
Т.о.
$$\lim_{x\ra x_0} \frac{g^(n-1)}{n! (x-x_0)} = 0$$
Тогда 
$$g(x) = o\left((x-x_0)^n\right)$$
\end{proof}

\begin{theorem}{Формула Тейлора с остатком в форме Пеано}
$f$ дифференцируема $n$ раз в $x_0$.
$$f(x) = T_{n, k} f(x) + o((x - x_0)^n) \quad x \ra x_0$$
\end{theorem}
\begin{proof}
$$g(x) = f(x) - T_{n, k} f(x)$$
$$\forall k \leqslant n\; g^{(k)} (x_0) = f^{(k)} (x_0) - \left(T_{n, x_0} f\right)^{(k)} (x_0) = 0$$
Пользуемся леммой.
\end{proof}

\begin{conseq}
$$\exists! P \in \R[x]\colon f(x) = P(x) + o((x - x_n)^k) \quad x \ra x_0$$
\end{conseq}
\begin{proof}
$x \ra x_0$:
$$T_{n, x_0} f(x) + o\left((x-x_0)^n\right) = f(x) = P(x) + o\left((x-x_0)^n\right)$$
$$q(x) \lrh T_{n, x_0} f(x) - P(x) = o\left((x-x_0)^k\right)$$
$$q(x_0) = 0$$
$q \in \R[x]$
$$q(x) = (x - x_0) q_1(x)$$
$$q_1(x) = o\left((x - x_0)^{n-1}\right)$$
$$q_1(x_0) = 0$$
$$\vdots$$
$$q_n(x_0) = o(1)$$
$$q_n \equiv 0$$
$$q \equiv 0$$
$$P \equiv T_{n, x_0} f$$
\end{proof}

\begin{theorem}{Формула Тейлора с остатком в форме Лагранжа}
$f$ дифференцируема $n/ + 1/$ раз в $x_0$, $f^{(n)}$ непрерывна на $[x, x_0]$.
$$\exists c \in (x, x_0)\colon f(x) = T_{n, x_0} f(x) + \frac{f^{(n+1)}(c)}{(n+1)!} (x-x_0)^{n+1}$$
\end{theorem}
\begin{Rem}
Теорема Лагража~--- частный случай для $n = 0$.
$$\exists c\in(x,x_0)\colon f(x) = f(x_0) + f'(c)(x-x_0)$$
\end{Rem}
\begin{proof}
$$f(x) = T_{n, x_0} f(x) + M \frac{(x-x_0)^{n+1}}{(n+1)!}$$
Надо доказать, что в форме
$$\exists c\in(x,x_0)\colon M = \frac{f^{(n+1)}(c)}{(n+1)!}$$
$$g(t) \lrh f(t) - T_{n, x_0} f(t) - M(t-x_0)^{n+1}$$
$$g^{(k)} (t) = f^{(k)}(t) - (T_{n, x_0})^{(k)} (t) - M(n+1)(n+2)(n+3)\cdots(n-k+2)(t-x_0)^{n-k+1}$$
$$g^{(k)} (x_0) = 0$$
Тогда у функции $g$ первые $n$ производных равны нулю, а также $g(x) = 0$, значит
$$g(x_0) = g(x) = 0$$
По теореме Ролля
$$\exists x_1\in(x, x_0)\colon g'(x_1) = 0$$
$$g'(x_0) = g'(x_1) = 0$$
По теореме Ролля
$$\exists x_2\in(x, x_1)\colon g'(x_2) = 0$$
$$\vdots$$
$$\exists x_{n+1}\in(x, x_0)\colon g^{(n+1)}(x_{n+1}) = 0$$
$$g^{(n+1)}(t) = f{(n-1)}(t) - M(n+1)!$$
$$c = x_{n+1}$$
\end{proof}

\begin{conseq}
$f\colon [a, b] \ra \R$, $n+1$ раз дифференцируема на $[a, b]$, $x_0 \in (a, b)$, $\left|f^{(n+1)} (t)\right| \leqslant M$.
$$\left|f(x) - T_{n, x_0} f(x)\right| \leqslant \frac{M \left|x-x_0\right|^{n+1}}{(n+1)!} = O\left((x-x_0)^n\right)$$
\end{conseq}
\begin{proof}
$$\exists c \in (x,x_0)\colon \left|f(x) - T_{n, x_0} f(x)\right| = \left|\frac{f^{(n+1)}(v)}{(n+1)!}(x-x_0)^{n+1}\right|$$
\end{proof}

\begin{conseq}
$f\colon [a, b] \ra \R$, $n+1$ раз дифференцируема на $[a, b]$, $x_0 \in (a, b)$, $forall n\; \left|f^{(n+1)} (t)\right| \leqslant M$.
$$\lim_{n\ra\infty} T_{n, x_0} = f(x)$$
\end{conseq}
\begin{proof}
$$\left|f(x) - T_{n, x_0} f(x)\right| \leqslant \frac{M \left|x-x_0\right|^{n+1}}{(n+1)!} \ra 0$$
\end{proof}

$x_0 = 0$:
\begin{align*}
e^x &= 1 &+ x &+ \frac{x^2}{2!} &+ \frac{x^3}{3!} &+ \frac{x^4}{4!} &+ \cdots &+ o(x^n) \\
e^x &= 1 &+ x &+ \frac{x^2}{2!} &+ \frac{x^3}{3!} &+ \frac{x^4}{4!} &+ \cdots &+ \frac{e^cx^{n+1}}{(n+1)!}
\end{align*}
\begin{align*}
\sin x &= 0 &+ x &+ 0 &- \frac{x^3}{3!} &+ 0 &+ \cdots &+ o(x^{2n+1}) \\
\cos x &= 1 &+ 0 &+ \frac{x^2}{2!} &+ 0 &- \frac{x^4}{4!} &+ \cdots &+ o(x^{2n+1}) \\
\ln (x+1) &= 0 &+ x &- \frac{x^2}2 &+ \frac{x^3}3 &- \frac{x^4}4 &+ \cdots &+ o(x^n) \\
(x+1)^p &= 1 &+ px &+ \frac{p(p-1)}{2!}x^2 &+ \frac{p(p-1)(p-2)}{3!}x^3 &+ \frac{p(p-1)(p-2)(p-3)}{4!}x^4 &+ \cdots &+ o(x^n)
\end{align*}

\begin{Def}
$a_n \in \R$
$$\sum_{i=0}^\infty \eqDef \lim_{n\ra\infty} \sum{i=0}^n a_n$$
\end{Def}

\begin{conseq}
$\forall x\ in \R\;$
$$e^x = \sum_{n=0}^\infty \frac{x^n}{n!}$$
$$\sin x = \sum_{n=0}^\infty \frac{(-1)^n x^{2n+1}}{(2n+1)!}$$
$$\sin x = \sum_{n=0}^\infty \frac{(-1)^n x^{2n}}{(2n)!}$$
\end{conseq}

\begin{theorem}{Иррациональность $e$}
$$e \notin \Q$$
\end{theorem}
\begin{proof}
$$\left(1+\frac1n\right)^n \leqslant e \leqslant \left(1+\frac1{n+1}\right)^n$$
$$2 < e < 3$$
Пусть $e = \frac{m}{n}$
$$e^1 = 1 + 1 + \frac1{2!} + 1\frac{3!} + \cdots + \frac{e^c}{(n+1)!} = \frac{m}{n} \Ra$$
$$\Ra \underbrace{n!(1 + 1 + \frac1{2!} + 1\frac{3!} + \cdots)}_{\in \N} + \frac{e^c}{n+1} = \underbrace{m(n-1)!}_{\in N} \Ra$$
$$\Ra \frac{e^c}{n+1} \in \N$$
$$0<c<1 \Ra 1 < e^c < 3$$
$$0 < \frac{1}{n+1} < \frac{e^c}{n+1} < \frac{3}{n + 1} < 1$$
Т.о. $e \ne \frac{m}{n}$
\end{proof}

\section{Экстремумы функции}

\begin{Def}
$f\colon \left<a, b\right> \ra \R$, $x_0 \in (a, b)$. 
$x_0$~--- точка строгого локального минимума, если
$$\exists \delta>0\colon \forall x \in (x - \delta, x + \delta) \ \{x_0\} f(x) > f(x_0)$$
$x_0$~--- точка нестрогого локального минимума, если
$$\exists \delta>0\colon \forall x \in (x - \delta, x + \delta) f(x) \geqslant f(x_0)$$

$x_0$~--- точка строгого локального максимума, если
$$\exists \delta>0\colon \forall x \in (x - \delta, x + \delta) \ \{x_0\} f(x) < f(x_0)$$
$x_0$~--- точка нестрогого локального максимума, если
$$\exists \delta>0\colon \forall x \in (x - \delta, x + \delta) f(x) \leqslant f(x_0)$$

Точка локального максимума или минимума также называется точкой локального экстремума.
\end{Def}
\begin{theorem}{Необходимое условие экстремума}
$f\colon \left<a, b\right> \ra \R$, $x_0 \in (a, b)$, $f$ дифференцируема в $x_0$.
$$x_0\text{~--- экстремум} \Ra f'(x_0) = 0$$
\end{theorem}
\begin{proof}
Сузим до окрестности, там по теореме Ферма всё работает.
\end{proof}
\begin{Rem}
Обратное неверно, смотри $f(x) = x^3$.
\end{Rem}\
\begin{theorem}{Достаточное условие экстремума}
$f\colon \left<a, b\right> \ra R$, $x_0 \in (a, b)$, $f$ непрерывна на $(x_0-\delta, x_0+\delta)$$f$ дифференцируема на $(x_0-\delta, x_0)\cup(x_0+\delta)$. Тогда
\begin{itemize}
\item $f'((x_0-\delta, x_0)) > 0 \land f'((x_0, x_0 + \delta)) < 0 \Ra x_0\text{~--- точка максимума}$
\item $f'((x_0-\delta, x_0)) < 0 \land f'((x_0, x_0 + \delta)) > 0 \Ra x_0\text{~--- точка минимума}$
\end{itemize}
\end{theorem}
\begin{proof}
$$f'((x_0-\delta, x_0)) > 0 \Ra f\text{ возрастает на } (x_0-\delta, x_0) \Ra f(x_0) > f((x_0-\delta, x_0))$$
$$f'((x_0, x_0+\delta)) < 0 \Ra f\text{ убывает на } (x_0, x_0+\delta) \Ra f(x_0) > f((x_0, x_0+\delta))$$
\end{proof}

\begin{theorem}{Достаточное условие экстремума через вторую производную}
$f\colon \left<a, b\right> \ra R$, $x_0 \in (a, b)$, $f$ дважды дифференцируема в $x_0$ и $f'(x_0) = 0$. Тогда
\begin{itemize}
\item $f''(x_0) < 0 \Ra x_0\text{~--- точка максимума}$
\item $f''(x_0) > 0 \Ra x_0\text{~--- точка минимума}$
\end{itemize}
\end{theorem}
\begin{theorem}{Достаточное условие экстремума через $n$-ую производную}
$f\colon \left<a, b\right> \ra R$, $x_0 \in (a, b)$, $f$ дифференцируема $n$ раз в $x_0$ и $f'(x_0) = f''(x_0) \cdots = f^{(n-1)}(x_0) = 0$. Тогда
\begin{itemize}
\item $2 \mid n \land f''(x_0) < 0 \Ra x_0\text{~--- точка максимума}$
\item $2 \mid n \land f''(x_0) > 0 \Ra x_0\text{~--- точка минимума}$
\item $2 \not \mid 2 \land f''(x_0) \ne 0 \Ra x_0\text{~--- не экстремум}$
\end{itemize}
\end{theorem}
\begin{proof}
$$f(x) = f(x_0) + \frac{f'(x_0)}{1!}(x-x_0) + \cdots + \frac{f^{(n-1)}}{(n-1)!}(x-x_0)^{n-1} + \frac{f^{(n)}}{n!}(x-x_0)^{n} + o((x-x_0)^n)$$
$$f(x) - f(x_0) = \frac{f^{(n)}}{n!}(x-x_0)^{n} + o((x-x_0)^n) = (x-x_0)^n \left(\frac{f^{(n)}(x_0)}{n!} + o(1)\right)$$
$2 \div n \land f^{(n)}(x_0) > 0 \Ra \exists \epsilon>0\colon \forall x\in(x_0-\epsilon,x_0)\cup(x_0, x_0+\epsilon)\; f(x) - f(x_0) > 0$
$2 \div n \land f^{(n)}(x_0) < 0 \Ra \exists \epsilon>0\colon \forall x\in(x_0-\epsilon,x_0)\cup(x_0, x_0+\epsilon)\; f(x) - f(x_0) < 0$
$2 \not\div n \land f^{(n)}(x_0) \ne 0 \Ra \exists \epsilon>0\colon \forall x\in(x_0-\epsilon,x_0)\cup(x_0, x_0+\epsilon)\; sign(f(x) - f(x_0)) = sign(x-x_0)$
\end{proof}

\section{Выпуклость}

\begin{Def}
$f\colon \left<a, b\right> \ra \R$. \\
$f$ выпукла вниз, если 
$$\forall x, y \in \left<a, b\right>\; \forall \lambda \in (0, 1) f(\lambda x + (1-\lambda)y) \leqslant \lambda f(x) + (1-\lambda) f(y)$$
$f$ строго выпукла вниз, если 
$$\forall x, y \in \left<a, b\right>\colon x \ne y\; \forall \lambda \in (0, 1) f(\lambda x + (1-\lambda)y) < \lambda f(x) + (1-\lambda) f(y)$$
$f$ выпукла вверх, если 
$$\forall x, y \in \left<a, b\right>\; \forall \lambda \in (0, 1) f(\lambda x + (1-\lambda)y) \geqslant \lambda f(x) + (1-\lambda) f(y)$$
$f$ строго выпукла вверх, если 
$$\forall x, y \in \left<a, b\right>\colon x \ne y\; \forall \lambda \in (0, 1) f(\lambda x + (1-\lambda)y) > \lambda f(x) + (1-\lambda) f(y)$$
\end{Def}
Абсолютно эквивалентная запись, геом. смысл...	0,0301
10.12	

\begin{Rem}
Сумма выпуклых и выпуклая, умноженная на положительную, выпуклы.
\end{Rem}

\begin{lemma}
О трёх хордах. $f\colon \left<a, b\right> \ra R$~--- выпуклая, $u < v < w$, $u,v,w \in \left<a, b\right>$. Тогда
$$\frac{f(v) - f(u)}{v - u} \leqslant \frac{f(w) - f(u)}{w-u} \leqslant \frac{f(w) - f(v)}{w-v}$$
\end{lemma}
\begin{proof}
$$\frac{f(v) - f(u)}{v - u} \leqslant \frac{f(w) - f(u)}{w-u} \Lra (w-u)(f(v) - f(u)) \leqslant (v-u)(f(w) - f(u)) \Lra$$
$$\Lra (w-u)f(v) - (w-u)f(u) \leqslant (v-u)f(w) - (v-u)f(u) \Lra (w-u)f(v) \leqslant (v-u)f(w) + (w-v)f(u)$$
\end{proof}

\begin{theorem}{}
$f\colon \left<a, b\right> \ra R$~--- выпуклая. Тогда 
$$\forall x \in (a, b)\; f'_-(x) \leqslant f'_+(x)$$
\end{theorem}
\begin{proof}
$u_1 < u_2 < x < v$
$$\frac{f(x)-f(u_1)}{x-u_1} \leqslant \frac{f(x)-f(u_2)}{x-u_2} \leqslant \frac{f(x)-f(v)}{x-v}$$
Тогда $\frac{f(x)-f(u)}{x-u}$ растёт и ограничено, т.е. предел $f'_-(x)$ существует.
Аналогично существует $f'_+(x)$, она убывает. Как видно, они в правильном порадке.
\end{proof}

\begin{theorem}{}
$f$~--- выпуклая на $\left<a, b\right>$ тогда и только тогда, когда
$$\forall x, x_0 \in \left<a, b\right>\; f(x) \geqslant f(x_0) + (x-x_0)f'(x_0)$$
\end{theorem}
\begin{proof}
$\Ra$:\\
$x > x_0$, $y \in (x_0, x)$
$$\frac{f(x_0) - f(y)}{x_0-y} \leqslant \frac{f(x_0)-f(x)}{x_0 - x}$$
$$f'(x_0) = \lim_{y \ra x_0} \frac{f(x_0) - f(y)}{x_0-y} \leqslant \frac{f(x_0) - f(x)}{x_0 - x}$$
$x_0 - x > 0$
$$f'(x_0)(x-x_0) \leqslant f(x_0) - f(x)$$
Аналогично $x < x_0$, $y \in (x, x_0)$
$$\frac{f(x) - f(x_0)}{x-x_0} \leqslant \frac{f(y)-f(x_0)}{y - x_0}$$
$\La$:\\
$u<v<w$
$$\forall x\; f(x) \geqslant f(v) + (x-v) f'(v)$$
$$f(u) \geqslant f(v) + (u-v) f'(v)$$
$$f(w) \geqslant f(v) + (w-v) f'(v)$$
Сложим с правильными коэффициентами:
$$(w-v)f(u) \geqslant (w-v) f(v) + (w-v)(u-v) f'(v)$$
$$(v-u)f(w) \geqslant (v-u) f(v) + (w-v)(v-u) f'(v)$$
$$(w-v)f(u) + (v-u) f(w) \geqslant (w-u) f(v)$$
\end{proof}

\begin{theorem}{Критерий выпуклости}
$f\colon \left<a, b\right> \ra \R$, $f$ дифференцируема на $(a, b)$.
$$f\text{ (строго) выпукла} \Lra f' \text{ (строго) возрастает}$$
\end{theorem}
\begin{proof}
$\Ra$: $x_1 < x_2$
$$f(x) \geqslant f(x_1) + (x - x_1) f'(x_1)$$
$$f(x) \geqslant f(x_2) + (x - x_2) f'(x_2)$$
Подставим
$$f(x_2) \geqslant f(x_1) + (x_2 - x_1) f'(x_1)$$
$$f(x_1) \geqslant f(x_2) + (x_1 - x_2) f'(x_2)$$
$$f'(x_1) \leqslant \frac{f(x_2)-f(x_1)}{x_2-x_1} \leqslant f'(x_2)$$
$La$:
Нужно проверить, что 
$$\frac{f(u)-f(v)}{u-v} \leqslant \frac{f(v) - f(w)}{v-w}$$
По теороеме Лагранжа, есть точки $\xi < \eta$
$$\frac{f(u)-f(v)}{u-v} = f'(\xi) \leqslant f'(\eta) = \frac{f(v) - f(w)}{v-w}$$
\end{proof}

\begin{theorem}{Критерий выпуклости через вторую производную}
$f\colon \left<a, b\right> \ra \R$, $f$ дважды дифференцируема на $(a, b)$.
$$f\text{ выпукла} \Lra f'' > 0$$
\end{theorem}
\begin{proof}
Смотрим на теоремы о монотонности.
\end{proof}

\begin{theorem}{Неравенство Денсена}
$f\colon \left<a, b\right> \ra \R$ выпукла.
$$\forall \{x_i\}_{i=1}^n \subset \left<a, b\right> \forall \{\lambda_i\}_{i=1}^n \subset [0, 1]\colon \sum_{i=1}^n \lambda_i = 1$$
$$f\left(\sum_{i=1}^n \lambda_i x_i\right) \leqslant \sum_{i=1}^n \lambda_i f(x_i)$$ 
\end{theorem}
\begin{proof}
Метод математической индукции. Теорема при $n = 2$ совпадает с определением выпуклости.
$$f\left(\underbrace{\sum_{i=1}^n \lambda_i x_i}_{\lrh y} + \lambda x_{n+1} x_{n+1}\right) = f((1 - \lambda_{n+1})y + \lambda_{n+1} x_{n+1}) \geqslant $$
$$ \geqslant (1-\lambda_{n+1}) f(y) + \lambda_{n+1} f(x_{n+1}) = (1 - \lambda_{n+1}) f\left(\sum_{i=1}^n \frac{\lambda_i}{1 - \lambda_{n+1}} x_i\right) \leqslant (1 - \lambda_{n+1}) \sum_{i=1}^n \frac{\lambda_i}{1-\lambda_{n+1}} f(x_i) + \lambda_{n+1} f(x_{n+1}) = $$
$$= \sum_{i=1}^n \lambda_i f(x_i) + \lambda_{n+1} f(x_{n+1})$$
\end{proof}

\begin{conseq}
Неравенство о средних~--- достаточно рассмотреть $$f(x) = -\ln x$$
\end{conseq}
\begin{conseq}
Неравенство Гельдера:
$$x_1, \ldots, x_n, y_1, \ldots, y_n \in \R \quad p,q > 1 \quad \frac1p + \frac1q = 1$$
$$\left|\sum_{i=1}^n x_iy_i\right| \leqslant \left(\sum_{i=1}^n |x_i|^p\right)^{\frac1p} \left(\sum_{i=1}^n |x_i|^q\right)^{\frac1q}$$
\end{conseq}
\begin{proof}
Если есть нули или отрицательные~--- перейдём к модулям.
$$f(x) = x^p$$
$$f\left(  \right) = $$
$$ \lambda_i a_i = \frac{x_iy_i}{(\sum_{i=1}^n y_i^p) ^ {\frac1q}}$$
\end{proof}

\begin{conseq}
Неравентсво Минковского
\end{conseq}
\chapter{Интегральное исчисление}

\section{Неопределённый интеграл}

\begin{Def}
$f\colon \left<a, b\right> \ra \R$. Функция $F\colon \left<a, b\right> \ra \R$ называется первообразной $f$, если
$$F' = f$$
\end{Def}

Не для всех $f$ существует $F$. Например,
$$f(x) = \begin{cases}1 & x \geqslant 0 \\ 0 & x < 0\end{cases}$$
\begin{proof}
Пусть есть $F' = f$. Тогда по теореме Дарбу
$$\forall a, b \in (-1, 1), c \in (F'(a), F'(b))\; \exists c \in (a, b)\colon F'(c) = C$$
\end{proof}

\begin{theorem}{О существовании первообразной}
Для любой непрерывной $f\colon \left<a, b\right> \ra \R$ есть первообразная $F$.
\end{theorem}
Докажем в следующем семестре.

\begin{theorem}{}
$f, F\colon \left<a, b\right> \ra \R$, $F$~--- первоообразная. Тогда
\begin{enumerate}
\item $F + с, c \in \R$ также первообразная.
\item $\Phi$~--- певрообразная только если $\Phi = F + c$.
\end{enumerate}
\end{theorem}
\begin{proof}
$$(F+c)' = F' + 0 = f$$
Рассмотрим $G = \Phi - F$. Она дифференцируема и
$$G' = (\Phi - F)' = \Phi' - F' = f - f = 0$$
Но тогда $$G = const$$
\end{proof}

\begin{Def}
Неопределённым интегралом функции $f$ называется множество её первообразных.
$$\int f(x)\d x$$
\end{Def}
Пока стоит воспринимать все символы интеграла как некоторые ,,скобки''.

Если есть некоторая первообразная $F$, то
$$\int f(x)\d x = \left\{F(x) + c \mid c \in \R\right\}$$
Тот же смысл имеют записи
$$\int f(x)\d x = F(x) + c$$
$$\int f\d x = F + c$$

Для того, чтобы найти неопределённый интеграл, достаточно найти какую-то первообразную, а для проверки первообразной достаточно взять от неё производную.

Таблица интегралов:
\begin{align*}
\int 0\d x &= c \\
\int x^p\d x &= \frac{x^{p+1}}{p + 1} + c \\
\int \frac{\d x}{x} &= \ln |x| + c \\
\int a^x \d x &= \frac{a^x}{\ln a} + c \\
\int \sin x \d x &= -\cos x + c \\
\int \cos x \d x &= \sin x + c \\
\int \frac{\d x}{\cos^2 x} &= \tg x + c \\
\int \frac{\d x}{\sin^2 x} &= -\ctg x + c \\
\int \frac{\d x}{\sqrt{1 - x^2}} &= \arcsin x + c \\
\int \frac{\d x}{1 + x^2} &= \arccos x + c \\
\int \frac{\d x}{1 - x^2} &= \frac12 \ln \left|\frac{1+x}{1-x}\right| + c\\
\int \frac{\d x}{\sqrt{x^2 \pm 1}} &= \ln \left|x + \sqrt{x^2 \pm 1}\right| + c
\end{align*}

\subsection{Арифметические действия с интегралами}.

\begin{Def}
Пусть $A, B$~--- множества. Тогда
$$A + B = \left\{a + b \mid a \in A \land b \in B\right\}$$
$$A - B = \left\{a - b \mid a \in A \land b \in B\right\}$$
$$\alpha A = \left\{\alpha a \mid a \in A\right\}$$
\end{Def}

\begin{theorem}{Об арифметических операциях с интегралами}
$$\int (f \pm g) \d x = \int f \d x \pm \int g \d x$$
$\alpha \ne 0$
$$\int \alpha f \d x = \alpha \int f \d x$$
\end{theorem}
\begin{Rem}
Именно из-за того, что константы в записи нет, мы исключаем ноль.
\end{Rem}
\begin{proof}
$F, G$~--- первообразные соотвественно $f, g$.
$$\int f \d x = \left\{F + c_1\right\}$$
$$\int g \d x = \left\{G + c_2\right\}$$
$$\int f \d x \pm \int g \d x = \left\{F + c_1\right\} \pm \left\{G + c_2\right\} = \left\{F+G+c_3\right\} = $$
$(F+G)' = f + g$
$$ = \int (f+g)\d x$$
$$\alpha \int f \d x = \alpha \left\{F + c_1\right\} = \left\{\alpha F + c_2\right\} = $$
$(\alpha F)' = \alpha f$
$$ = \int \alpha f\d x$$
\end{proof}

\begin{theorem}{Замена переменной в неопределённом интеграле}
$f\colon \left<a, b\right> \ra \R$ непрерывна, $\phi\colon \left<c, d\right> \ra \left<a, b\right>$ непрерывно дифференцируема.
$$\int f(\phi(t)) \phi'(t) \d t = F(\phi(t)) + c$$
\end{theorem}
\begin{proof}
$$\left(F(\phi(t)) + c\right)' = (F(\phi(t)))' = F'(\phi(t)) \phi'(t) = f(\phi(t)) \phi'(t)$$
\end{proof}
\begin{conseq}
$$\int f(\alpha x + \beta) \d x = \frac1{\alpha}F(\alpha x + \beta) + c$$
\end{conseq}

Примеры:
$$\int \frac{\ln^2 x}{x}\d x$$
$f = x^2, \phi = \ln x$
$$\int \frac{\ln^2 x}{x} \d x = \int (\ln x)^2 (\ln x)' \d x = \frac{(\ln x)^3}3 + c = \frac{\ln^3 x}3 + c$$
$a>0$
$$\int \frac{\d x}{x^2 + a^2} = \frac1{a^2} \int \frac{\d x}{\left(\frac{x}a\right)^2 + 1} = \frac1{a^2} \frac1{\frac1a} \arctg \frac{x}a + c=$$
$$= \frac1a \arctg \frac{x}a$$
$f=\frac1{x^2+1}$

\begin{theorem}{Интегрирование по частям}
$f, g$~--- дифференцируемые, $f'g$~--- интегрируемая.
$$\int fg'\d x = fg - \int f'g \d x$$
\end{theorem}
\begin{proof}
$\Phi$~--- первообразная $f'g$.
$$(fg - \Phi + c)' = fg' + f'g - f'g = fg'$$
\end{proof}

Пример:
$$\int x^2 e^x \d x = x^2 e^x - \int 2x e^x \d x = x^2 e^x - 2 \int x e^x \d x = $$
$$ = x^2 e^x - 2\left(x e^x - \int e^x \d x\right) = x^2 e^x - 2x e^x + 2e^x + c$$

Есть термин ,,берущеися'' интегралы. Это интегралы, выражаемые через элементарные функции. Их, вообще говоря, мало. К ним относятся рациональные функции (отношение многочленов), произведение тригинометрических функций, $x\sqrt{ax^2 + bx + c}$. Не берутся, например, 
$$\int e^{x^2} \d x$$
$$\int \frac{e^x}x \d x$$
$$\int \frac{\sin x}x \d x$$
$$\int \frac{\cos x}x\d x$$
$$\int \frac{\d x}{\ln x}$$
\end{document}

Свойства фундаментальных последовательностей:
\begin{enumerate}
\item Ограничена
\item Если есть сходящаяся подпоследовательность, то она сходится.
\begin{proof}
$$\forall \epsilon > 0\: \exists K\colon \forall k > K\: \rho(x_{n_k}, a) < \epsilon$$
$$\forall \epsilon > 0\: \exists N\colon \forall n,m > K\: \rho(x_n, x_m) < \epsilon$$

Т.о.
$$\exists n_k > M = \max\{N, K\}\colon \forall n > n_k \rho(x_n, a) \leqslant \rho(x_{n_k}, a) + \rho(x_{n_k}, x_k) < 2\epsilon$$
\end{proof}
\end{enumerate}

\begin{Def}
Пространство называют полным, если любая фундаментальная последовательность имеет предел.
\end{Def}   

\begin{theorem}{О сходимости фундаментальных последовательностей}
\begin{enumerate}
\item Любая сходящаяся последовательность фундаментальна.
\item В $\R^d$ фундаментальная последовательность всегда сходится.
\end{enumerate}
\end{theorem}
\begin{proof}
$\lim x_n = a$
$$\forall \epsilon > 0\: \exists N\colon \begin{aligned}\forall n>N \rho(x_n, a) &< \epsilon \\ \forall m>N \rho(x_m, a) &< \epsilon\end{aligned} \Ra \forall \epsilon>0\: \exists N\colon \forall n,m>N\: \rho(x_m, x_n) < 2\epsilon$$

$x_n$~--- фундаментальная последовательность в $\R^d$. $E_n \lrh \{x_n, x_{n+1}, \ldots\}$~--- ограниченно.
$\cl E_n$~--- ещё и замкнуто. Т.е. компактно.
$$\cl E_1 \supset \cl E_2 \supset \cl E_3 \supset \cdots$$
$$\diam \cl E_n = \diam E_n \ra 0$$

Т.о.
$$\exists! a\colon a \in \bigcap_{i=1}^{\infty} \cl E_n$$
$$a \in \cl E_n \Ra \forall i>n\:0\leqslant\rho(a, x_i) \leqslant \diam E_n \ra 0$$

Т.о $x_n \ra a$.
\end{proof}
                                                
\begin{Rem}
$\R^d$ полно. $\left<\Q, \rho\right>$ не полно. Пространство с дискретной метрикой полно.
\end{Rem}

\begin{theorem}{О полноте компактного пространства}
Компактное метрическое пространство полно.
\end{theorem}
\begin{proof}
В компакте у любой последовательности есть сходящаяся подпоследовательность. А значит любая фундаментальная последовательность имеет сходящуюся подпоследовательность.
А значит она сама сходится. А значит пространство полно.
\end{proof}

\section{Верхний и нижний предел}

\begin{Def}
Верхний и нижний предел
$$\liminf x_n = \varliminf x_n = \lim_{x\ra\infty}\inf_{k>n} x_k$$
$$\limsup x_n = \varlimsup x_n = \lim_{x\ra\infty}\sup_{k>n} x_k$$
\end{Def}
\begin{Rem}
$y_n \lrh \inf_{k>n} x_n$, $z_n \lrh \sup_{k>n} x_n$.
$$y_n<x_n<z_n$$
$$y_n \nearrow{}; z_n \searrow$$
\end{Rem}

\begin{Def}
$a$~--- частичный предел последовательности, если $a$ предел подпоследовательности.
\end{Def}

\begin{lemma}
Если $x_n$ монотонно возрастает и неограничена, то $\lim x_n = +\infty$
\end{lemma}

\begin{theorem}{Существование верхнего и нижнего пределов}
У любой последовательности есть верхний и нижний предел в $\bar\R$, при этом
$$\varliminf x_n \leqslant \varlimsup x_n$$
\end{theorem}
\begin{proof}
$y_n \lrh \inf_{k>n} x_n$, $z_n \lrh \sup_{k>n} x_n$.
Если $x_n$ ограниченно, то и $y_n$ ограниченно. Если $x_n$ не ограниченно снизу, то и $y_n$ не ограниченно снизу. Т.о. $\lim y_n = \varliminf x_n$. Аналогично 
существует верхний предел.
\end{proof}

\begin{theorem}{Верхний и нижний предел и частичные пределы}
\begin{enumerate}
\item $\limsup$~--- наибольший частичный предел.
\item $\liminf$~--- наименьший частичный предел.
\item $\lim\text{ существует}\Lra\varlimsup = \varliminf$
\end{enumerate}
\end{theorem}
\begin{proof}
\begin{enumerate}
\item 
$a = \limsup x_n$. Покажем, что $a$~--- частичный предел.
$$z_n \searrow{} \Ra \sup_{k>n} x_k \geqslant a$$

Выберем $$x_{k_m}\colon x_{k_m} > a - \frac1m; k_{m+1} > k_m$$
Оно стремится к $a$.

Пусть есть больший частичный предел. Но тогда с какого-то места последовательность, сходящаяся к $b$, уйдёт выше супремума, что плохо.
\item Аналогично.
\item Два миллиционера.
\end{enumerate}
\end{proof}

\begin{theorem}{Определение верхнего и нижнего предела через $N$ и $\epsilon$}
\begin{enumerate}
\item $$a = \varliminf x_n \Lra \begin{cases}\forall \epsilon > 0\: \exists N\colon \forall n>N\: x_n > a - \epsilon \\ \forall \epsilon > 0\: \forall N\colon \exists n>N\: x_n < a + \epsilon\end{cases}$$
\item $$a = \varlimsup x_n \Lra \begin{cases}\forall \epsilon > 0\: \forall N\colon \exists n>N\: x_n > a - \epsilon \\ \forall \epsilon > 0\: \exists N\colon \forall n>N\: x_n < a + \epsilon\end{cases}$$
\end{enumerate}
\end{theorem}
\begin{proof}
\begin{enumerate}
\item Запишем в терминах $y_n$:
$$\forall \epsilon > 0\: \exists N\colon \inf_{n>N} > a-\epsilon ; \forall \epsilon > 0\: \exists N\colon \inf_{n>N} < a+\epsilon$$
Уже видно, что эти условия и задают предел.
\item Аналогично.
\end{enumerate}
\end{proof}

\begin{theorem}{О предельном переходе в неравенстве}
$$a_n \leqslant b_n \Ra \begin{cases}\varliminf a_n \leqslant \varliminf b_n \\ \varlimsup a_n \leqslant \varlimsup b_n\end{cases}$$
\end{theorem}                                                                                                                       
\begin{proof}
Просто сводим к пределам инфимумов. 
\end{proof}

\begin{theorem}{Неравенство Бернулли}
$$\forall x>-1\: \forall n \in \N\: (1+x)^n \geqslant 1 + nx$$
\end{theorem}
\begin{proof}
Индукция: база очевидна. Пусть $(1+x)^k \geqslant 1 + nk$. Тогда
$$(1+x)^{k+1} = \underbrace{(1+x)^k}_{>0} (1+x) \geqslant (1+kx)(1+x) = 1 + kx + x + kx^2 \geqslant 1 + (k+1)x$$
\end{proof}
\begin{conseq}
Если $|t| > 1$, то $\lim t^n = +\infty$. Если $|t| < 1$, то $\lim t^n = 0$.
\end{conseq}

\begin{theorem}{Предел убывающей по отношению}
$x_n > 0$, $\lim \frac{x_{n+1}}{x_n} < 1$. Тогда $x_n \ra 0$.
\end{theorem}
\begin{proof}
С какого-то места отношение довольно мало (меньше 1).
\end{proof}
\begin{conseq}
$$\lim_{n\ra\infty} \frac{n^k}{a^n} = 0\quad a>1$$
\end{conseq}
\begin{proof}
$$x_n = \frac{n^k}{a^n}$$
$$\frac{x_{n+1}}{x_n} = \left(\frac{n+1}n\right)^k \frac1a < 1$$
\end{proof}
\begin{conseq}
$$\lim \frac{a^n}{n!} = 0$$
\end{conseq}

Определим число $e$:
$$x_n = \left(1+\frac1n\right)^n; y_n=\left(1+\frac1n\right)^{n+1}$$
Покажем, что $x_n \uparrow; y_n \downarrow$.
\begin{proof}
$$x_n < x_{n+1} \La \frac{(n+1)^n}{n^n} < \frac{(n+2)^{n+1}}{(n+1)^{n+1}} \La \frac{n+1}{n+2} < \frac{n^n(n+2)^n}{(n+1)^{2n}} \La $$
$$ \La \frac{n+1}{n+2} < \left(1-\frac1{n^2+2n+1}\right)^n \La 1 - \frac{1}{n+2} < 1 - \frac{n}{n^2+2n+1} \leqslant \left(1-\frac1{n^2+2n+1}\right)^n$$

$$y_n < y_{n-1} \La \frac{(n+1)^{n+1}}{n^{n+1}} < \frac{n^n}{(n-1)^n} \La \frac{n+1}{n} < \frac{n^{2n}}{(n-1)^n(n+1)^n} \La $$
$$\La \frac{n+1}{n} < \left(1+\frac1{n^2-1}\right)^n \La 1 + \frac1n < 1 - \frac{n}{n^2-1} \leqslant \left(1-\frac1{n^2-1}\right)^n$$
\end{proof}

Заметим, что при этом $x_n < y_n$. Собственно, тогда $\lim x_n$ существует.
$$\lim \left(1+\frac1n\right)^n \lrh e$$

Свойства:
\begin{enumerate}
\item $\lim y_n = e$
\item $x_n < e < y_n$
\end{enumerate}

\begin{conseq}
$$\lim \frac{n!}{n^n} = 0$$
\end{conseq}
\begin{proof}
$$x_n = \frac{n!}{n^n}$$
$$\frac{x_{n+1}}{x_n} = (1+\frac1n)^{-n} \ra \frac1e < 1$$
\end{proof}

\begin{theorem}{Теорема Штольца для бесконечно малых}
$0<y_n<y_{n-1}$, $\lim x_n = \lim y_n = 0$, $\lim \frac{x_n-x_{n-1}}{y_n-y_{n-1}} = a \in \bar\R$.
Тогда 
$$\lim \frac{x_n}{y_n} = a$$
\end{theorem}
\begin{proof}
\begin{enumerate}
\item $a = 0$.
$$\epsilon_n = \frac{x_n - x_{n - 1}}{y_n - y_{n - 1}} \to 0$$
$n > m$:
$$x_n - x_m = \sum_{k = m + 1}^{n}(x_k - x_{k - 1}) = \sum_{k = m + 1}^{n}\epsilon_k(y_k - y_{k - 1})$$

$$|x_n - x_m| = \left|\sum_{k = m + 1}^{n}\epsilon_k(y_k - y_{k - 1})\right| = \sum_{k = m + 1}^{n} |\epsilon_k|(y_{k-1} - y_k) < $$

$$\forall \epsilon < 0\: \exists N\colon \forall k > N\: |\epsilon_k| < \epsilon$$

Тогда при $n > m > N$
$$< \sum^{n}_{k = m + 1}\epsilon(y_{k - 1} - y_k) = \epsilon\sum_{k = m + 1}^{n}(y_{k - 1} - y_k) = \epsilon(y_m - y_n)$$

$$|x_n - x_m| < \epsilon|y_n - y_m|$$
$n \ra \infty$

$$|x_m| < \epsilon y_m$$
$$\frac{x_m}{y_m} < \epsilon$$

\item $a \in \R$: $\tilde x_n = x_n - a y_n$.
\item $a = +\infty$: Поменяем местами $x_n$ и $y_n$. Проверим свойство для $x$:
$$\lim \frac{x_n - x_{n-1}}{y_n - y_{n-1}} = +\infty \Ra \frac{x_n - x_{n-1}}{y_n - y_{n-1}} > 1$$
$$x_n - x_{n-1} < y_n - y_{n-1} < 0$$
\end{enumerate}
\end{proof}

\begin{theorem}{Теорема Штольца для бесконечно больших}
$0<y_n<y_{n+1}$, $\lim x_n = \lim y_n = +\infty$, $\lim \frac{x_n-x_{n-1}}{y_n-y_{n-1}} = a \in \bar\R$.
Тогда 
$$\lim \frac{x_n}{y_n} = a$$
\end{theorem}
\begin{proof}
\begin{enumerate}
\item $a = 0$:
$$\epsilon_n \lrh \frac{x_n - x_{n-1}}{y_n - y_{n-1}}$$
$$x_n = x_1 + \sum_{i=2}^n (x_i - x_{i-1}) = x_1 + \sum_{i=2}^n \epsilon_i(y_i - y_{i-1})$$
$$\frac{x_n}{y_n} = \frac{x_1}{y_n} + \sum_{i=2}^n \epsilon_i \frac{y_i - y_{i-1}}{y_n} = $$
$$\forall \epsilon > 0\: \exists N\colon \forall n > N\: |\epsilon_n| < \epsilon$$
$$= \frac{x_1}{y_n} + \sum_{i=2}^N \epsilon_i \frac{y_i - y_{i-1}}{y_n} + \sum_{i=N+1}^n \epsilon_i \frac{y_i - y_{i-1}}{y_n}$$

$$\left|\sum_{i=N+1}^n \epsilon_i \frac{y_i - y_{i-1}}{y_n}\right| = \sum_{i=N+1}^n |\epsilon_i| \frac{y_i - y_{i-1}}{y_n} < 
\sum_{i=N+1}^n \epsilon \frac{y_i - y_{i-1}}{y_n} <$$
$$< \frac{\epsilon}{y_n}\sum_{i=N+1}^n (y_i - y_{i-1}) = \epsilon \frac{y_n - y_N}{y_n} < \epsilon$$

$$\sum_{i=2}^N \epsilon_i \frac{y_i - y_{i-1}}{y_n} \leqslant \frac{1}{y_n}\sum_{i=2}^N \epsilon_i(y_i - y_{i-1}) < \epsilon$$

$$\frac{x_1}{y_n} < \epsilon$$

Т.о.
$$\left|\frac{x_n}{y_n}\right| < \epsilon \Ra \frac{x_n}{y_n} \ra 0$$

\item $a\in\R$:
$\tilde x_n = x_n - a y_n$. Фактом $x_n\ra\infty$ мы не пользовались.

$$\frac{\tilde x_n - \tilde x_{n - 1}}{y_n - y_{n - 1}} = \frac{(x_n - ay_n) - (x_{n - 1} - ay_{n - 1})}{y_n - y_{n - 1}} = \frac{x_n - x_{n - 1}}{y_n - y_{n - 1}} - a \to 0$$

\item $a=+\infty$: Поменяем местами $x_n$ и $y_n$. Проверим, что $x_n$ монотонно растёт и не ноль.
$$\frac{x_n - x_{n-1}}{y_n - y_{n-1}} = +\infty \Ra \frac{x_n - x_{n-1}}{y_n - y_{n-1}} > 1$$
$$x_n - x_{n-1} > y_n - y_{n-1} > 0$$

\item $a=-\infty$: Сменим знаки $x_n$.
\end{enumerate}
\end{proof}

\chapter{Пределы и непрерывность отображений}
\section{Пределы функций}

\begin{Def}
$(X, \rho_x)$ и $(Y, \rho_y)$~--- метрические пространства. $E \subset X$, $a$~--- предельная точка $E$. $f\colon E \ra Y$.
Тогда $b$ является пределом $f$ в $a$ по Коши
$$\lim_{x \ra a} f(x) = b$$
если $b \in Y$ и
$$\forall \epsilon>0\: \exists \delta > 0\colon \forall x\: x \in \dot B_\delta(a) \cap E \Ra f(x) \in B_\epsilon (b)$$
или, что то же самое
$$\forall \epsilon>0\: \exists \delta > 0\colon \forall x\in E \: (x \ne a \land \rho(x, a) < \delta) \Ra \rho(f(x), b) < \epsilon$$
\end{Def}

\begin{Rem}
Для бесконечности на $\R$ есть частные случаи.
\end{Rem}

\begin{Def}
По Гейне,
$$\lim_{x \ra a} f(x) = b \Lra \forall \{x_n\}\subset E\colon x_n \ne a\: \lim_{n\ra \infty} x_n = a \Ra \lim_{n \ra \infty} f(x_n) = b$$
\end{Def}

\begin{theorem}{Равносильность определений предела функции}
Определения равносильны.
\end{theorem}
\begin{proof}
Коши $\Ra$ Гейне:
$$\forall \epsilon>0\: \exists \delta > 0\colon \forall x\: x \in \dot B_\delta(a) \cap E \Ra f(x) \in B_\epsilon (b)$$
Рассмотрим произвольную $\{x_n\} \subset E \setminus \{a\}$, $\lim_{n\ra\infty} x_n = a$. Для неё выполнено указанное выше. Тогда $\{f(x_n)\}$ сходится к $b$.

Гейне $\Ra$ Коши: от противного. Пусть
$$\exists \epsilon>0\colon \forall \delta > 0\: \exists x\: x \in \dot B_\delta(a) \cap E \land f(x) \notin B_\epsilon(b)$$
Возьмём данный $\epsilon$ и выберем последовательность $\delta_n = \frac1n$. Тогда получим, что
$$\exists \{x_n\}\colon x_n /ne a \land x_n \ra a \land f(x_n) \nrightarrow b$$
что противоречит Гейне.
\end{proof}
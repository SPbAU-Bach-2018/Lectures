\chapter{Интегральное исчисление}

\section{Неопределённый интеграл}

\begin{Def}
$f\colon \left<a, b\right> \ra \R$. Функция $F\colon \left<a, b\right> \ra \R$ называется первообразной $f$, если
$$F' = f$$
\end{Def}

Не для всех $f$ существует $F$. Например,
$$f(x) = \begin{cases}1 & x \geqslant 0 \\ 0 & x < 0\end{cases}$$
\begin{proof}
Пусть есть $F' = f$. Тогда по теореме Дарбу
$$\forall a, b \in (-1, 1), c \in (F'(a), F'(b))\; \exists c \in (a, b)\colon F'(c) = C$$
\end{proof}

\begin{theorem}{О существовании первообразной}
Для любой непрерывной $f\colon \left<a, b\right> \ra \R$ есть первообразная $F$.
\end{theorem}
Докажем в следующем семестре.

\begin{theorem}{}
$f, F\colon \left<a, b\right> \ra \R$, $F$~--- первоообразная. Тогда
\begin{enumerate}
\item $F + с, c \in \R$ также первообразная.
\item $\Phi$~--- певрообразная только если $\Phi = F + c$.
\end{enumerate}
\end{theorem}
\begin{proof}
$$(F+c)' = F' + 0 = f$$
Рассмотрим $G = \Phi - F$. Она дифференцируема и
$$G' = (\Phi - F)' = \Phi' - F' = f - f = 0$$
Но тогда $$G = const$$
\end{proof}

\begin{Def}
Неопределённым интегралом функции $f$ называется множество её первообразных.
$$\int f(x)\d x$$
\end{Def}
Пока стоит воспринимать все символы интеграла как некоторые ,,скобки''.

Если есть некоторая первообразная $F$, то
$$\int f(x)\d x = \left\{F(x) + c \mid c \in \R\right\}$$
Тот же смысл имеют записи
$$\int f(x)\d x = F(x) + c$$
$$\int f\d x = F + c$$

Для того, чтобы найти неопределённый интеграл, достаточно найти какую-то первообразную, а для проверки первообразной достаточно взять от неё производную.

Таблица интегралов:
\begin{align*}
\int 0\d x &= c \\
\int x^p\d x &= \frac{x^{p+1}}{p + 1} + c \\
\int \frac{\d x}{x} &= \ln |x| + c \\
\int a^x \d x &= \frac{a^x}{\ln a} + c \\
\int \sin x \d x &= -\cos x + c \\
\int \cos x \d x &= \sin x + c \\
\int \frac{\d x}{\cos^2 x} &= \tg x + c \\
\int \frac{\d x}{\sin^2 x} &= -\ctg x + c \\
\int \frac{\d x}{\sqrt{1 - x^2}} &= \arcsin x + c \\
\int \frac{\d x}{1 + x^2} &= \arccos x + c \\
\int \frac{\d x}{1 - x^2} &= \frac12 \ln \left|\frac{1+x}{1-x}\right| + c\\
\int \frac{\d x}{\sqrt{x^2 \pm 1}} &= \ln \left|x + \sqrt{x^2 \pm 1}\right| + c
\end{align*}

\subsection{Арифметические действия с интегралами}.

\begin{Def}
Пусть $A, B$~--- множества. Тогда
$$A + B = \left\{a + b \mid a \in A \land b \in B\right\}$$
$$A - B = \left\{a - b \mid a \in A \land b \in B\right\}$$
$$\alpha A = \left\{\alpha a \mid a \in A\right\}$$
\end{Def}

\begin{theorem}{Об арифметических операциях с интегралами}
$$\int (f \pm g) \d x = \int f \d x \pm \int g \d x$$
$\alpha \ne 0$
$$\int \alpha f \d x = \alpha \int f \d x$$
\end{theorem}
\begin{Rem}
Именно из-за того, что константы в записи нет, мы исключаем ноль.
\end{Rem}
\begin{proof}
$F, G$~--- первообразные соотвественно $f, g$.
$$\int f \d x = \left\{F + c_1\right\}$$
$$\int g \d x = \left\{G + c_2\right\}$$
$$\int f \d x \pm \int g \d x = \left\{F + c_1\right\} \pm \left\{G + c_2\right\} = \left\{F+G+c_3\right\} = $$
$(F+G)' = f + g$
$$ = \int (f+g)\d x$$
$$\alpha \int f \d x = \alpha \left\{F + c_1\right\} = \left\{\alpha F + c_2\right\} = $$
$(\alpha F)' = \alpha f$
$$ = \int \alpha f\d x$$
\end{proof}

\begin{theorem}{Замена переменной в неопределённом интеграле}
$f\colon \left<a, b\right> \ra \R$ непрерывна, $\phi\colon \left<c, d\right> \ra \left<a, b\right>$ непрерывно дифференцируема.
$$\int f(\phi(t)) \phi'(t) \d t = F(\phi(t)) + c$$
\end{theorem}
\begin{proof}
$$\left(F(\phi(t)) + c\right)' = (F(\phi(t)))' = F'(\phi(t)) \phi'(t) = f(\phi(t)) \phi'(t)$$
\end{proof}
\begin{conseq}
$$\int f(\alpha x + \beta) \d x = \frac1{\alpha}F(\alpha x + \beta) + c$$
\end{conseq}

Примеры:
$$\int \frac{\ln^2 x}{x}\d x$$
$f = x^2, \phi = \ln x$
$$\int \frac{\ln^2 x}{x} \d x = \int (\ln x)^2 (\ln x)' \d x = \frac{(\ln x)^3}3 + c = \frac{\ln^3 x}3 + c$$
$a>0$
$$\int \frac{\d x}{x^2 + a^2} = \frac1{a^2} \int \frac{\d x}{\left(\frac{x}a\right)^2 + 1} = \frac1{a^2} \frac1{\frac1a} \arctg \frac{x}a + c=$$
$$= \frac1a \arctg \frac{x}a$$
$f=\frac1{x^2+1}$

\begin{theorem}{Интегрирование по частям}
$f, g$~--- дифференцируемые, $f'g$~--- интегрируемая.
$$\int fg'\d x = fg - \int f'g \d x$$
\end{theorem}
\begin{proof}
$\Phi$~--- первообразная $f'g$.
$$(fg - \Phi + c)' = fg' + f'g - f'g = fg'$$
\end{proof}

Пример:
$$\int x^2 e^x \d x = x^2 e^x - \int 2x e^x \d x = x^2 e^x - 2 \int x e^x \d x = $$
$$ = x^2 e^x - 2\left(x e^x - \int e^x \d x\right) = x^2 e^x - 2x e^x + 2e^x + c$$

Есть термин ,,берущеися'' интегралы. Это интегралы, выражаемые через элементарные функции. Их, вообще говоря, мало. К ним относятся рациональные функции (отношение многочленов), произведение тригинометрических функций, $x\sqrt{ax^2 + bx + c}$. Не берутся, например, 
$$\int e^{x^2} \d x$$
$$\int \frac{e^x}x \d x$$
$$\int \frac{\sin x}x \d x$$
$$\int \frac{\cos x}x\d x$$
$$\int \frac{\d x}{\ln x}$$
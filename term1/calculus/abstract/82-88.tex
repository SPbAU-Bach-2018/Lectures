\section{} % 82
(Не)строгий максимум "--- в окрестности всё меньше. Теорема (необход. усл.): если дифф. в окрестности
экстремума, то в нём производная равна нулю.

\section{} % 83
Теорема: непрерывна на окрестности, дифференцируема на проколотой окрестности,
тогда если слева производная больше нуля, а справа меньше, то максимум; симметрично.
Теорема: если $f'(x)=0$ и $f''(x)>0$, то имеем максимум; симметрично.
Теорема: если $f'(x)=f''(x)=\dots=f^{(n-1)}(x)=0$, то при $n \vdots 2$ зависит от
очередного знака, а при $n \not\vdots 2$ не экстремум (по Тейлоре-Пеано).

\section{} % 84
Выпуклость вниз/вверх (не)строгая, если отрезок между любыми двумя точками лежит выше/ниже графика.
Выпуклость меняется на вогнутость сменой знака, сумма выпуклых и домножение на $\alpha > 0$ "--- ок.
Определения: через $\lambda \in [0,1]$, через $a < x < b$, через ориентацию угла $BAX$.
Лемма о трёх хордах: рисуем три хорды между точками $a<x<b$, смотрим на углы. Замечание:
если взять одно из трёх неравенств в лемме и сказать, что оно для всех точек, то получаем выпуклость.

\section{} % 85
Есть выпуклая на $(a,b)$, тогда $\forall x$ есть $f'^-(x)$ и $f'^+(x)$ и $f'^-(x)\le f'^+(x)$
(из непрерывность следует). Теорема: $f(x) \ge f(x_0) + (x-x_0)f'(x_0)$ $\iff$ $f$ "--- выпуклая (д-во: \TODO).

\section{} % 86
Пусть $f$ дифференцируема, тогда $f$ "--- выпукла $\iff$ $f'$ возрастает (строго куда надо).
Через вторую производную: $f''>0$.

\section{} % 87
Йенсен: $f$ выпукла, взяли $\lambda_i \in [0,1]$ с суммой 1, тогда $f\left(\sum\lambda_i x_i\right) \le \sum\lambda_if(x_i)$.
Д-во: по индукции от $n=2$ (две точки)

\section{} % 88
Нер-во о средних следует из Йенсена: $\frac{\sum x_i}{n} > \sqrt[n]{x_1x_2\dots x_n}$ (д-во: $f(x)=-\ln x$,
$\lambda_i=-\ln\frac1n$.
Нер-во Гельдера: $x_n, y_n \in \R$, $p,q>0$, $\frac1p+\frac1q=1$, тогда $|\sum x_iy_i| \le \sqrt[p]{\sum|x_i|^p)} \sqrt[q]{\sum|y_i|^q)}$
(при $p=q=2$ это КБШ; д-во: перешли к модулям, $f(x)=x^p$, взяли с весами $y_k^q$).
Нер-во Минковского: $p\ge1$, тогда $\sqrt[p]{\sum |x_k+y_k|^p} \le \sqrt[p]{\sum x_k^p} + \sqrt[p]{\sum y_k^p}$ (д-во: через Гельдера).

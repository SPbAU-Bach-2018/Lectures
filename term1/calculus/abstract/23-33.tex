\section{} % 23
Покрытие (открытыми) множествами, компакт "--- из любого покрытия можно выбрать конечное
подпокрытие. Теорема: $K$ "--- компакт в $Y \subset X$ $\iff$ $K$ "--- компакт в $X$.

\section{} % 24
Теорема: компакт замкнут и ограничен, д-во: взяли вокруг точки шарик, покрыли,
выбрали конечное числа шариков.
Теорема: замкнутое подмн-во компакта "--- компакт.

\section{} % 25
Теорема: любое конечное пересечение компактов из семейства не пусто, тогда все тоже
пересекаются (перешли к дополнениям).
Следствие: если есть цепочка вложенных непустых компактов, то пересечение непусто.

\section{} % 26
Паралеллепипед в $\R^n$ (замкнутый и открытый).
Теорема: есть вложенные, пересечение непусто (применили по каждой координате вложенные отрезки).

\section{} % 27
Гейне-Бореля: замкнутый паралеллепипед компакт (д-во: от противного, делимся пополам, каждый гиперкуб
не может быть покрыт конечным числом, потом взяли точку в пересечении, открытое множество, в котором
она лежит, шарик, содержит какой-то мелкий куб целиком, упс, противоречие).

\section{} % 28
Если есть предел последовательности, то подпоследовательность имеет тот же. Объединение
двух подпоследовательностей с общим пределом даст тот же предел.

\section{} % 29
Секвенциальная компактность: из любой последовательности можно выбрать сходящуюся к некоторой
точке множества подпоследовательность. В $\R^d$ равносильно: компактность, замкнутость
и ограниченность, секвенциальная компактность.

\section{} % 30
В $\R^d$ компактность $K$ равносильна наличию предельной точки для любого подмножества.
Больцано-Вейерштрасса: из любой ограниченной последовательности в $\R^d$ можно
выбрать сходящуюся.
Следствие: в любой последовательности в $\R$ можно найти сходящуюся к $\bar \R$.

\section{} % 31
Диаметр: $\diam A = \sup \rho(x, y)$. Свойства: от замыкания не меняется, если есть последовательность
вложенных компактов и их диаметр $\to 0$, то пересечение "--- одна точка.

\section{} % 32
Фундаментальная: $\forall \eps > 0$ существует хвост диаметром меньше $\eps$. Свойства:
ограничена, если есть сходящаяся подпоследовательность $\Rightarrow$ сходится.
Пространство полное, если любая фундаментальная сходится. Любая сходящаяся фундаментальна.

\section{} % 33
Теорема: в $\R^d$ фундаментальные сходятся ($\Rightarrow \R^d$ полно). Теорема: компактное
метрическое полно.

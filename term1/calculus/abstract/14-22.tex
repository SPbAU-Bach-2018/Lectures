\section{} % 14
$x* = \lim\limits_{n\to\infty} x_n$, если любой шар "--- ловушка. Стационарная последовательность.
Предел может исчезать, если пространство маленькое (или есть метрика дискретная).
Свойства: можно заменить шар на окрестность; если есть предел, то один; если есть, то $\{x_n\}$ ограничена,
если $x \in A'$, то $\exists \{x_n\}\to x$ (при этом м.б. $x \notin A$), при этом можно выбрать все точки различные;
если $\{x_n\} \to x*$ и они все различны, то $x* \in \{x_n\}'$, если $\{x_n\} \in A$, то $\lim x_n \in \clos A$.

\section{} % 15
Работаем с $(\R, |x-y|)$. Если $x_n \le y_n$, то $\lim x_n \le \lim y_n$. Можно требовать отношение
только между хвостами, строгие нер-ва не сохраняются. Следствия: $x_n \le a \Rightarrow \lim x_n \le a$ и
симм. и с двух сторон.

\section{} % 16
Пусть $x_n \le y_n \le z_n$, есть пределы $\lim x_n = \lim z_n$, тогда предел $\{y_n\}$ существует и равен им.
Следствие: если модуль ограничен сходящейся к нулю последовательностью, то предел "--- ноль.

\section{} % 17
(Не)строгое монотонное возрастание/убывание. Вейерштрасс: монотонная ограничена тогда и только тогда,
когда есть предел. Док-во: взяли супремум.

\section{} % 18
Вектор "--- кортеж, определили сложение, умножение на число. Сложение "--- абелева группа (комм.),
$\alpha(x+y)=\alpha x+\alpha y$, $(\alpha+\beta)x=\alpha x + \beta x$, $(\alpha\beta)x=\alpha(\beta x)$,
$1x=x$. Общее определение "--- то же самое, без конкретики. Скалярное (Евклидово) произведение: неотрицательно,
линейно (умножаем один на число, складываем), симметрично. Общее определение: то же самое
(скажем, можно всё домножить на константу). Евклидова норма: $||x||=\sqrt{<x,x>}$. $||x||\ge0$ (равно только если
ноль), $||\lambda x||$, $|<x,y>|\le||x||\cdot||y||$ (КБШ), $||x+y||\le ||x|| + ||y||$ (д-во: возвели
в квадрат, КБШ), $||x-y||\le||x-z||+||z-y||$ (Минковский), $||x-y|| \ge |~||x||-||y||~|$ (из треуг. и $||x-y||=||y-x||$).
Т.о. получили метрику: $\rho(x,y)=||x-y||$. Общая норма: неотр., линейна, треугольник. Примеры: $||x||_1$ "--- Манхэттен,
$||x||_\infty$ "--- максимум по координатам.

\section{} % 19
Есть $\{\vec x_n\}$ и её предел равен нулю (в смысле метрики на пространстве) $\iff$ предел норм равен нулю.
Арифметические свойства (из пределов векторов во что-то другое): предел суммы, домножение на $\lambda$, разность, предел скалярного, предел норм.
Свойства пределов на вещественных: сумма, домножение на $\lambda$, разность, модуль, произведение, отношение (если не ноль).

\section{} % 20
Всё ясно: в две стороны

\section{} % 21
Пределы $\pm\infty$ и $\lim = \infty$ (когда модуль растёт, то есть неограничена, например $\lim(-2)^n=\infty$).
Единственность предела и миллиционеры справедливы, если расширяем на $\pm\infty$ (но не на $\infty$), тогда
получим $\bar \R$ (куча свойств про умножение и сложение констант и бесконечностей).
Бесконечно большая последовательность $\iff \lim = \infty$ 
Бесконечно малая $\iff \lim = 0$.

\section{} % 22
Теоремы: $x_n \to \infty \iff \frac{1}{x_n} \to 0$; если $x_n$ и $y_n$ б.м., то сумма б.м. и можно
домножать на ограниченную. Для б.б можно прибавлять ограниченные с одной стороны, домножать
не не бесконечно малые последовательности одного знака, делить в разных комбинациях бесконечности,
ограниченные, отделённые от нуля.

\section{} % 69
Ферма: $f$ дифференцируема в $x_0$ и $x_0$ "--- глобальный экстремум, тогда $f'=0$ (посмотрели
на левую и правую, одна $\ge0$, другая $\le0$).
Ролля: дифференцируема, $f(a)=f(b)$, тогда существует в интервале точка, где $f'(x)=0$
(д-во: $f$ "--- либо константа, либо есть максимум/минимум, не являющийся концом отрезка).
Следствие: между корнями ф-ции есть корень производной. Следствие: если есть $n$ корней,
то хотя бы $n-1$ корень производной.

\section{} % 70
Лагранж (конечные приращения): обобщение Ролля: есть касательная, параллельная отрезку.
Коши: двумерный аналог на плоскости (считаем, что $g' \neq 0$): есть вектор скорости, параллельный
перемещению. Д-во: вводим $h(x) = f(x) - \frac{f(b)-f(a)}{g(b)-g(a)}g(x)$

\section{} % 71
Если ограничили производную $M$, то $|f(x)-f(y)| \le M|x-y|$ и $f$ равномерно непрерывна.
Если $f'=0$ на интервале, то $f$ "--- константа. Теорема: $f'\ge0$ на интервале $\iff$ то монотонна (можно строго;
симметрично для $f'\le 0$).

\section{} % 72
Дарбу: есть функция дифференцируема на $(a,b)$, то достигаются все значения производных
между $f'(a)$ и $f'(b)$. Д-во: если хотим найти $f'(x)=0$, то найдём по Вейерштрассу
минимум и максимум, достаточно показать, что хотя бы один из них не в конце (от противного).
В общем: $g(x)=f(x)-Cx$.

\section{} % 73
Пусть $-\infty \le a < b \le +\infty$, f и g дифф. на $(a,b)$, $g(x) \neq 0$, а
также $\lim f(x) = \lim g(x)$ (в какой-то точке, предел односторонний), при этом $\lim f(x) \in \{0, \infty\}$.
Тогда если $\lim \frac{f'}{g'} = b$, то $\lim \frac{f}{g} = b$. Д-во: взяли последовательность, бабах Штольцем.
Примеры: $\frac{\ln x}{x^p}$, $\frac{x^p}{a^x}$ при $a>0$, $x^x$ ($x\ln x = \frac{\ln x}{1/x}$).

\section{} % 74
$f^{(n)}(x) = (f^{(n-1)}(x))' = \frac{\d^n f(x)}{(\d x) ^n}$ "--- $n$-я производная.
Пусть $C(a, b)$ "--- мн-во функций, непрерывных на $(a, b)$ (аналогично для отрезков и мн-в),
а $C^n(E)$ "--- мн-во функций, которые можно дифф. $n$ раз. $C^{\infty}$ "--- пересечение
всех, есть включения: $C^n(E) \supset C^{n+1}(E)$, $C(E) \supset C^n(e)$, все вкл. строгие
(пример: $x^n\sqrt[3]{x}$, получим не дифф. $\sqrt[3]{x}$).

\section{} % 75
Линейная комбинация, правило Лейбница ($(fg)^{(n)} = \sum \binom{n}{k} f^{(k)}g^{(n-k)}$),
еще бывает формула Фаа-ди-Бруно (выражает сложную ф-цию), но там адец и у нас не было.
Примеры: $x^p$, $\ln x$, $a^x$, $\sin^{(n)}(x) = \sin(x+\frac{\pi n}{2})$, $f(ax+b)^{(n)}=a^nf^{(n)}(ax+b)$.

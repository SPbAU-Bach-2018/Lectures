\section{} % 41
Есть метрические пространства $X$ и $Y$, $E \subset X$, a "--- предельная точка $E$, $f: X \to Y$,
тогда по Коши: $\lim\limits_{x\to a} f(x) = b$ если $b \in Y$ и $\eps$-$\delta$ определение,
но берём точки из $\dot B_\delta(a) \cap E$. Если у нас $Y=\R$, то можно брать в качестве предела
и $\pm\infty$ (это частный случай).
По Гейне: $\forall \{x_n\} \to a$ (где $x_n\in E \setminus \{a\}$) если $x_n \to a$, то $f(x_n) \to b$.

\section{} % 42
Теорема: они равносильны, д-во: (Коши $\Rightarrow$ Гейне "--- просто; обратно "--- от противного).
Замечание: в Гейне можно рассматривать только пос-ти из разных элементов, которые строго приближаются к пределу.
Замечание: в Гейне можно просто доказать, что для любой последовательности существует предел $f(x)$, тогда
они все окажутся равны (иначе перемешаем $x_n$ и $y_n$).

\section{} % 43
Предел $f$ в точке единственен (из Коши). Если он есть, то: в некотором $\dot B_r(x)$ $f$ ограничена,
если предел не ноль, то еще и отделена от нуля.

\section{} % 44
В $\R^d$ можно переходить к пределу в: $f(x) \pm g(x)$, $\lambda f(x)$, $||f(x)||$, $<f(x), g(x)>$.
В $\R$ еще произведение и деление.

\section{} % 45
Можно делать предельный переход и миллиционеров, как в последовательностях.

\section{} % 46
Предел слева: $\lim\limits_{x\to a-} = \lim\limits_{x\to a-0}$ "--- предел $f|_{E\cap(-\infty,a)}$,
справа симметрично.
Аналогично последовательностям: если функция ограничена и монотонна, то в любой предельной точке $a$ у функции существуют левый и правый пределы.

\section{} % 47
Коши: про фундаментальные последовательности в общем виде: есть предел функции в $a$ (какое-то $b$) $\iff$ если отойти от $a$ несильно, то любые два значения будут отличаться
несильно.

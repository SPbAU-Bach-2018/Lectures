\section{} % 76
Пусть $T$ "--- многочлен степени $n$, тогда $T(x)=\sum\limits_{k=0}^n \frac{T^{(k)}(x_0)}{k!}(x-x_0)^k$.

\section{} % 77
Пусть $f$ имеет $n$ производных в точке $x_0$, тогда м-н Тейлора есть
$T_{n,x_0} f(x) = \sum\limits_{k=0}^n\frac{f^{(k)}(x_0)}{k!}(x-x_0)^k$,
если $f$ "--- многочлен степени $n$, то совпадает.
Формула Тейлора: $f(x) = T_{n,x_0}f(x) + R_{n,x_0} f(x)$, где $R$ "--- остаток.
Лемма: если $T_{n,x_0}f(x) = 0$,то $f(x)=o(x-x_0)$ при $x\to x_0$ (д-во: Лопиталь $n-1$ раз,
потом не хватает дифференцируемости в окрестности, вспоминаем производную через $o()$). Теперь в форме Пеано:
$f(x) = T_{n,x_0}f(x) + o((x-x_0)^n)$. Следствие: существует единственный многочлен степени не больше
$n$, который так приближает. Док-во: \TODO. % нафиг

\section{} % 78
Пусть есть $n+1$ производная на $(x,x_0)$ и $n$-я производная непрерывна,
тогда есть точка $c \in (x,x_0)$ такая, что $R_{n,x_0}f(x) = \frac{f^{(n+1)}(c)}{(n+1)!}(x-x_0)^{n+1}$.
При $n=0$ это формула конечных приращений (т.Лагранжа). Д-во: введём M: $R_{n,x_0}f(x)=M(x-x_0)^{n+1}$,
введём g: $g(t)=R_{n,x_0}f(t)-M(t-x_0)^{n+1}$, у неё первые $n$ производных равны нулю, Роллем пару раз.

\section{} % 79
Помним.

\section{} % 80
Определение: $\sum\limits_{n=0}^{\infty} a_n = \lim \sum\limits_{k=0}^n a_k$ (если существует).
Тогда написали разложение чего надо в ряд. \TODO из конспекта (стр.81).

\section{} % 81
Знаем, что $2 < e < 3$, разложили по Тейлору (по Лагранжу последний член), домножили на $n!$,
получили $\in \mathbb{N}$, $\frac{e^c}{n+1} \in \mathbb{N}$ (а $c \in (0,1)$, упс.


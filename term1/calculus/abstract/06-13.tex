\section{} % 6
Есть неотрицательная метрика на парах: если ноль, то равны; симметричность; правило треугольника.
Пара $(X, \rho: X\times X \to \R_+)$ "--- метрическое пространство. Дискретная (0 или 1), модуль,
расстояния на плоскости, Манхэттен, на сфере, французкая железнодорожная (есть Париж и радиальные дороги;
либо едем через Париж, либо по одной ветке), $\R^n$ (среднеквадратичное отклонение, по КБШ и Минковскому).
Открытый шар "--- $B_r(a)$, закрытый шар "--- $\overline B_r(a)$. Можно пересекать (с одним центром),
существуют непересекающиеся шары в разных центрах (Хаусдорф). Подпространство.

\section{} % 7
КБШ: $\left(\sum a_ib_i\right)^2 \le \sum a_i^2 \cdot \sum b_i^2$
д-во: $f(t)\coloneqq \sum(a_it-b_i)^2$, раскрыли скобки, не более одного корня, а еще равенство только если вектора пропорциональны.
Минковский: $\sqrt{\sum(a_i+b_i)^2}\le\sqrt{\sum a_i^2}+\sqrt{\sum b_i^2}$, возвели в квадрат, по КБШ.

\section{} % 8
Открытое мн-во "--- каждая точка лежит с открытым шаром. Свойства: $\varnothing$ и $X$ открыты, объединение
любого количество открыто, пересечение конечного открыто (иначе приблизили точку), открытый шар открыт.
Окрестность $a$ "--- открытое, содержащее $a$.

\section{} % 9
Внутренняя точка лежит вместе с шаром ($\iff$ есть открытое подмн-во, содержащее $a$).
Внутренность $\inter A$ "--- объединение всех вложенных открытых ($\iff$ все внутренние точки $A$),
она открыта, можно пересекать, внутренности, смотреть на вложенность, для открытого $A$ она равна $A$.

\section{} % 10
Замкнутое множество (дополнение открыто). Есть четыре свойства по аналогии с открытыми.
Замыкание ($\clos A$) "--- пересечение замкнутых, содержащих $A$; есть аналогичные свойста с внутренностью.
Теорема: $X\setminus \clos A=\inter(X\setminus A)$ и симметрично.

\section{} % 11
$Y \subset X$ "--- пр-ва.
Теорема: $A$ открыто в $Y$ $\Rightarrow \exists G: A \subset G \subset X$ и $G$ открыто; симметрично для замкнутых.

\section{} % 12
$x \in \clos A \iff \forall r>0 B_r(x) \cap A \neq \varnothing$.
Следствие: если $U$ открыто и $U \cap A = \varnothing$, то $U \cap \clos A = \varnothing$.
Проколотая окрестность $\dot B_r(x)$. Точка предельная, если любая проколотая окрестность пересекается с множеством.
$A' \coloneqq$ множество предельных точек $A$. Свойства: $\clos A = A \cup A'$, $A \subset B \Rightarrow A' \subset B'$, $(A\cup B)'=A'\cup B'$.
Теорема об окрестности предельной точки: в любом открытом шаре бесконечно много точек множества.

\section{} % 13
Теорема для замкнутых: если ограничено сверху, то $\sup A \in A$, для инфиума аналогично.

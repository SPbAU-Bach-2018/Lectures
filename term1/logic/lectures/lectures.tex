\documentclass[12pt]{article}

\usepackage[T2A]{fontenc}
\usepackage[utf8]{inputenc}
\usepackage[russian]{babel}
\usepackage{amsthm, amsmath, amssymb}
\usepackage[russian]{hyperref}
\usepackage{datetime}
\usepackage{cmap}
\usepackage{hologo}
\usepackage{enumerate}
\usepackage{euscript}
%\usepackage{color}
\usepackage{picture}


\renewcommand{\epsilon}{\varepsilon}

\ifx\pdfoutput\undefined
\usepackage{graphicx}
\else
\usepackage[pdftex]{graphicx}
\fi

\voffset=-20mm
\textheight=220mm

\hoffset=-20mm
\textwidth=175mm

\def\EPS{\varepsilon}
\def\SO{\Rightarrow}
\def\EQ{\Leftrightarrow}
\def\t{\texttt}
\def\O{\mathcal{O}}

\newcounter{theorem}[section]
\renewcommand{\thetheorem}{\thesection.\arabic{theorem}}
\newcommand*{\theoremheader}[1]{\par\refstepcounter{theorem}%
\textbf{Теорема \thetheorem.} #1}
\newenvironment*{theorem}[1]{
	\theoremheader{#1}%
}{%
\par
}

\newenvironment{MyTabbing}{
\t\bgroup
%\vspace*{-\baselineskip}
\begin{tabbing}
aaaa\=aaaa\=aaaa\=aaaa\=aaaa\kill
}{
\end{tabbing}
\t\egroup
}


\newcommand\Label[1]{\item {\bf #1}}
\def\Begin{$\bullet$\hspace{0.5em}}

\newenvironment{MyList}{
  \begin{enumerate}[1.]
  \setlength{\parskip}{-5pt}
  \setlength{\itemsep}{5pt}
}{
  \end{enumerate}
}

\newlength{\myskip}
\setlength{\myskip}{0.5em}

\newcommand\URL[1]{{\footnotesize{\url{#1}}}}

\begin{document}

\begin{center}
  {\Large \bf Основы математической логики и дискретной математики} \\ 
  \vspace{0.5em}
  {\Large \bf Семестр 1} \\
  \vspace{0.5em}
  {\Large Лектор: Ицыксон Дмитрий Михайлович} \\
  \vspace{0.5em}
  {\large Автор конспекта: Ольга Черникова} \\
  \vspace{0.5em}
  {Собрано {\today} в {\currenttime}}
\end{center}

\vspace{-1em}
\noindent \underline{\hbox to 1\textwidth{{ } \hfil{ } \hfil{ } }}

\vspace{1em}
\tableofcontents
\pagebreak

\section{Пропозициональные формулы}

\subsection{Пропозициальные формулы}

(Формулы вычисления высказывания)

Г - множество пропозициональных переменных ($x_1, x_2, x_3, \ldots$)

\begin{description}
\item {\bf Определение} пропозициональная формула:
\begin{enumerate}
\item Пропозициональная переменная - это формула
\item A - формула $\Rightarrow$ $\neg A$ - формула
\item A, B - формулы $\Rightarrow$ $(A \vee B), (A \wedge B), (A \to B)$ - формулы
\end{enumerate}
Пропозициальные формулы - минимальное множество строк, которые удовлетворяют 1, 2, 3 условиям.
\end{description}

\subsection{Интерпретации}

0 - False

1 - True

Дизъюнкция:
\begin{tabular}{c c|c}
x & y & x $\vee$ y \\
0 & 0 & 0\\
0 & 1 & 1\\
1 & 0 & 1\\
1 & 1 & 1\\
\end{tabular}

Конъюнкция:
\begin{tabular}{c c|c}
x & y & x $\wedge$ y \\
0 & 0 & 0\\
0 & 1 & 0\\
1 & 0 & 0\\
1 & 1 & 1\\
\end{tabular}

Импликация:
\begin{tabular}{c c|c}
x & y & x $\to$ y \\
0 & 0 & 1\\
0 & 1 & 1\\
1 & 0 & 0\\
1 & 1 & 1\\
\end{tabular}

$\Phi$ ~--- пропозициальная формула от n переменных. 

\subsection{Булева функция}

$\{0, 1\}^n \to \{0, 1\}$ ~---  булева функция.

Пропозициальная формула  $\leftrightarrow$ булева функция.

\subsection{Представление булевой функции в ДНФ и КНФ}

{\bf Литерал} - это переменная или отрицание переменной $x, \neg x, y, \neg y$

{\bf Конъюнкт(терм)}  $l_1 \wedge l_2 \wedge \ldots \wedge l_n$

{\bf Формула в дизъюнктивной нормальной форме(ДНФ):} $c_1\vee c_2 \vee \ldots \vee c_k$, где $c_i$ - конъюнкт.

{\bf Дизъюнкт(clouse(клоз)):} $l_1 \vee l_2 \vee \ldots \vee l_n$, где $l_i$ ~--- литерал.

{\bf Формула в конъюктивной нормальной форме(КНФ):} $d_1 \wedge d_2 \wedge \ldots \wedge d_k$, где $d_i$ ~--- дизъюнкт.

\begin{description}
\item [\bf Теорема:] любая булевая функция представляется в виде КНФ и ДНФ.
\item [\bf Доказательство:] 

ДНФ

\begin{tabular}{c|c}
$x_1 \ldots x_n$ & \\
\hline
$0 \ldots 0$ & \\
\hline
$ \ldots $ & 1 \\
\hline
$ \vdots $ &  \\
\hline
$ \ldots $ & 1 \\
\hline
$ 1 \ldots 1$ &  \\
\end{tabular}

Для каждой строчки, где стоит 1 запишем соответствующий конъюнкт. $(\neg x_1 \wedge x_2 \wedge \ldots \wedge x_n) \vee \ldots$

$\neg x_i$ ~--- если $x_i = 0$

$x_i$ ~--- если $x_i = 1$

КНФ

Рассмотрим строчки, где записаны 0. Они все не должны выполняться. 

\end{description}

\subsection{Эквивалентные формулы}

\begin{description}
\item {\bf Определение} две формулы эквивалентные, если они задают одну и ту же булеву функцию.
\end{description}


\begin{description}
\item {\bf Формулы де Морга} 

$\neg (x \vee y) \sim \neg x \wedge \neg y$

$\neg (x \wedge y) \sim \neg x \vee \neg y$
\end{description}

$\neg(c_1 \vee c_2 \vee \ldots \vee c_n) \sim \neg c_1 \wedge \neg c_2 \ldots \wedge \neg c_n$

$c_1 = l_1 \wedge l_2 \wedge \ldots \wedge l_k$

$\neg c_1 = \neg l_1 \vee \neg l_2 \vee \ldots \vee \neg l_k$

$x \wedge (y \vee z) \sim (x \wedge y) \vee (x \wedge z)$

$x \to y \sim \neg x \vee y$

Алгоритм приведение в ДНФ:
\begin{enumerate}
\item избавится $\to$
\item перенести отрицание к переменным
\item раскрыть скобки пользуясь дистрибутивностью.
\end{enumerate}

\section{Выполнимость формулы}

\subsection{Тавтологии, противоречия, выполнимые формулы}
 \begin{description}
\item {\bf Определение} Формула - тавтология, если она истена, при всех значениях переменной.
\end{description}


 \begin{description}
\item {\bf Определение} Формула - противоречива, если она ложна, при всех значениях переменной.
\end{description}

$\Phi$ ~--- выполнимая формула, если она не является противоречивой.
$\exists$ значение переменных, что значение формулы истина.

\subsection{Выполнимость КНФ}

Задача SAT ~---- выполнима ли формула в КНФ.

\begin{description}
\item {\bf Теорема.} По любой формуле можно за быстро построить формулу в КНФ, выполнимость которой эквивалентна выполнимости исходной.
\item {\bf Доказательство.} 

$(x_1 \vee x_2) \wedge ((x_3 \to x_4) \vee x_5)$

Для формулы построим дерево разбора. 

\begin{center} 
\includegraphics[width=2in, keepaspectratio]{im1.jpg} 
\end{center}

Для промежуточных вершин, заведем переменные $P_1, P_2, \ldots, P_k$.

Формула выполняется, если выполняется система.

$$
\left\{
\begin{aligned}
P_4 = P_1  \wedge P_2\\
P_1 = x_1 \vee x_2\\
P_2 = x_5 \vee P_3\\
P_3 = x_3 \to x_4\\
\end{aligned}
\right.
$$

Каждое уравнение можно представить как несколько дизъюнктов. 
\item {\bf Следствие из доказательства:} В полученной формуле в КНФ в каждой дизъюнкт входит $\le 3$ литерала. 3-КНФ.
\end{description}

\section{Резолюционное исчисление}

$\Phi$ ~--- тавтология $\Leftrightarrow \neg \Phi$ ~---невыполнима.

$\neg \Phi \sim \Psi$ в КНФ.

$\neg \Phi$ невыполнимо $\Leftrightarrow \Psi$ невыполнима.

КНФ: $d_1 \wedge d_2 \wedge \ldots \wedge d_k$

$d_i = (l_1 \vee l_2 \vee \ldots \vee l_m)$

$S = \{d_1, d_2, \ldots, d_k\}$

\begin{description}
\item{\bf Правило резолюции}
$\cfrac{(x \vee A) \_ (\neg x \vee B)}{A \vee B (\text{резольвента})}$
\end{description}

{\bf Утверждение} Если С ~--- резольвента дизъюнктов D и E, то любое значение переменных, который выполняет D и E, выполняет и C.  

$\cfrac{x \_ \neg x}{\blacksquare}$

\begin{description}
\item{\bf Определение} $\Phi$ ~--- формула в КНФ. Резолюционным опровержением формулы $\Phi$ называется последовательность дизъюнктов $c_1, c_2, \ldots, c_m$.
\begin{enumerate}
\item $c_m$ ~--- пустой дизъюнкт.
\item $\forall i$ от 1 до m $c_i$ ~--- либо дизъюнкт формулы $\Phi$, либо $c_i$ ~--- резольвента $c_k$ и $c_l$,  где $k, l < i$
\end{enumerate}
\end{description}

\begin{description}
\item{\bf Теорема} $\Phi$ ~--- формула в КНФ. $\Phi$ невыполнима $\Leftrightarrow \exists$ резолюционное опровержение формулы $\Phi$
\item {$\Leftarrow$ \bf Корректность} $c_1, c_2, \ldots, c_m$ ~--- резалюционное опровержение $\Phi$.

Пусть набор значений $\sigma$ выполняет $\Phi$.

По индукции можно доказать $\sigma$ выполняется $c_i \forall i$

$c_i$ ~--- дизъюнкт $\Phi$ очевидно.

$\cfrac{c_k\_c_l}{c_i} k, l < i$ по индукционному предположению $\sigma$ выполняет $c_k$ и $c_l$ $\Rightarrow \sigma$ выполняет $c_i \Rightarrow c_m = \blacksquare$ выполняет $\sigma$, противоречие. 

\item {$\Rightarrow$ \bf Полнота}

Индукция по числу n переменных в $\Phi$.

{\bf База} n = 1.

$(x \vee \neg x) \to $ заменим на 1

$x \vee x \vee x \to$ заменим на x

дизъюнкты на будут повторяться. 

$x \wedge \neg x$ ~--- единственный не выполнимый вариант $\Rightarrow$  получим $\blacksquare$.

{\bf Переход} $n \to n + 1$

$x$ ~--- переменная.

разобьем формулы на 3 группы. 

\begin{enumerate}
\item $S_1 = $A
\item $S _2 = x \vee A$
\item $S_3 = \neg x \vee A$
\end{enumerate}

$\Phi|_{x = 0}$(подставим x = 0) $S_1 \wedge S_2'$

$S_2' = $ дизъюнкт из $S_2$ без x.


$\Phi|_{x = 1}$ $S_1 \wedge S_3'$

$\Phi_{x = 0}$ ~--- невыполнима, на одну переменную меньше. По индукционному предположению существует опровержение. 

Вернем в опровержение x. Тогда получим или пустой  дизъюнкт, или x. 

Аналогично, для $\Phi_{x = 1}$. Получим $\neg x$ или опровержение.

Или получили противоречие, либо $\cfrac{x\_\neg x}{\blacksquare}$

\item{\bf Замечание} Если в $d_1$ и $d_2$ входит $le$ 2 литералов, то и в резальвенту входит $\le$ 2 литералов.

\item{\bf Пример}
$(\neg x \vee y) \wedge (\neg y \vee x) \wedge (\neg y \vee z) \wedge (\neg z \vee y) \wedge (x \vee z) \wedge (\neg x \vee \neg z)$

$\cfrac{(\neg x \vee y) \_ (x \vee z)}{\cfrac{(y \vee z) \_ (\neg y \vee z)}{\cfrac{z \_ (\neg z \vee y)}{\cfrac{y\_(\neg y \vee x)}{\cfrac{x\_(\neg x \vee \neg z)}{ \cfrac{\neg z \_ z}{\blacksquare} } }}}}$
\end{description}

\section{Алгоритм проверяющий выполнимость формулы 2-КНФ}

\begin{enumerate}
\item пока можем вывести новую резальвенту ~--- выводим.
\item остановка:
\begin{enumerate}
\item вывели $\blacksquare$
\item больше ничего не можем вывести.
\end{enumerate}

\end{enumerate}

Время работы ~--- $\O(n^2)$

Количество дизъюнктов:

\begin{enumerate}
\item дизъюнктов из 1 литерала ~---- 2n
\item из 2 ~--- $\frac{2n(2n - 1)}{2}$
\end{enumerate}

\section{Построение резолюционного доказательства по дереву расщепления}

Построим дерево расщепление. 

\begin{center} 
\includegraphics[width=2in, keepaspectratio]{im2.jpg} 
\end{center}

В каждом листе написан дизъюнкт, который опровергается подстановкой от листа до корня. 

Заполняем все дерево. Если в каком-то листе ничего не написано, значит вормула выполнима. 

Построим резолюционное опровержение по дереву. 


\begin{center} 
\includegraphics[width=2in, keepaspectratio]{im3.jpg} 
\end{center}

Пока есть что заменять, будем выводить резольвенту из двух братьев и записывать в их предка. 

В каждой вершине окажется дизъюнкт, который опровергается подстановкой переменных от вершины до корня. 

В корне должен оказаться пустой дизъюнкт. 

\section{Схемы из функциональных элементов}

\subsection{Ориентированный граф без циклов и топологическая сортировка.}

Ориентированный граф без циклов(DAG)

{\bf Утверждение} G ~--- DAG, тогда $\exists$ вершина без исходящих ребер, $\exists$ вершина без входящих ребер. 

\begin{description}
\item {\bf Лемма(о топологической сортировке)}

G - DAG, V ~--- множество вершин, тогда $\exists h: V \to \{1, 2, \ldots. |V|$

\begin{enumerate}
\item биекция
\item $(u, v)$ ~--- ребро $\Rightarrow h(u) < h(v)$ 
\end{enumerate} 

\item {\bf Доказательство}

Индукция по числу вершин.

{\bf База} одна вершина

{\bf Переход} пусть v ~--- вершина без исходящих ребер.

h(v) = |V|

Выкидываем вершину v из G и получаем G'. По предположению индукции можем построить топологическую мортировку для G'. 

Определим h на $V/\{v\}$  совпадающей с h'.
\end{description}

\subsection{Схемы}

$B = \{f_1^{(k_1)},f_2^{(k_2)}, \ldots, f_l^{(k_l)} \}$

$f_i^{(k_i)}:\{0, 1\}^{k_i} \to \{0, 1\}$

\begin{description}
\item{\bf Схема под базисом B:}

DAG

Вершины, в которые ничего не входит, называются входами $x_1, x_2, \ldots, x_n$

Вершин, из которых ничего не выходит ~--- выходы.

Вершины кроме входов ~--- внутренние(gates).

Каждая внутренняя вершина помечена $f_i^{(k_i)} \in B$ и имеет вход степени $k_i$. Входящие ребра пронумерованы.

Выполнение схемы:

\begin{enumerate}
\item топологически сортируем
\item задаем начальные значения
\item считаем значения в порядке топологической сортировки.
\end{enumerate}

Если у схемы n входов и m выходов, то она задает функцию $\{0, 1\}^n \to \{0, 1\}^m$

{\bf Определение} Базис B называется полным, если для любой булевой функции существует схема над B выражающая ее.

Размер схемы ~---- число вершин в графе.

Глубина схемы ~---- длина максимального пути от входа до выхода.

$f:\{0, 1\}^n \to \{0, 1,\}^k$

$size_B(f)$ ~--- min размер схемы в базисе B, которые вычисляют f.
\end{description}

\subsection{Эквивалентность различных базисов}

\begin{description}
\item{\bf Лемма} $B_1, B_2$ ~--- полные базисы. Тогда $\exists C > 0: \forall n, k \forall f:\{0, 1\}^n \to \{0, 1\}^k size_{B_1}(f) \le C size_{B_2}(f)$
\item{\bf Доказательство}
$B_1 = \{h_1, h_2, \ldots, h_t\}$

$B_2 = \{g_1, g_2, \ldots, g_m\}$

$g_i^{k_i}$ задается схемой в базисе $B_1$ 

f в базисе $B_2$ заменяем $g_i^{k_i}$ на схему в базисе $B_1$, которая вычисляет $g_i^{k_i}$ 

Получим схему для f в $B_1$

$C$ ~--- размер максимального представления $g_i$ в виде $B_1$ схемы.
\end{description}

\section{Схема умножения}

\subsection{Схема для сложения}
 
$P_n, \ldots, p_1$

$...., x_{n - 1}, x_{n - 2}, \ldots, x_0$


$...., y_{n - 1}, y_{n - 2}, \ldots, y_0$

$P_1 = x_0 \wedge y_0$

$P_2 = (x_1 \wedge y_1) \vee (y_1 \wedge P_1) \vee (x_1 \wedge P_1)$

$\ldots$

Размер $\O(n)$

Глубина $\O(n)$

\subsection{Схема умножения}

Размер $\O(n^{\log_2(3)})$

Глубина $\O(n \log n)$ $T(n) = cn + T(\frac{n}{2})$

$n = 2^k$ 

n ~--- длина числа.

$x = a * 2^{\frac{n}{2}} + b$

$y = c * 2^{\frac{n}{2}} + d$

$xy = ac2^n + (ad + bc)2^{\frac{n}{2}} + bd$

$S(n) = 4S(\frac{n}{2}) + cn$

$S(n) = \O(n^2)$

$(a + b)(c + d) - ac - bd$

$S(n) = 3S(\frac{n}{2}) + cn$

$S(n)  = n^{\log_2(3)}$

\section{Существование булевой функции, которая не вычисляется схемой размера $\frac{2^n}{Cn}$}

\begin{description}
\item [Теорема:] $f:\{0, 1\}^n \to \{0, 1\}$

B - полный базис.

Тогда $size_B(f) = \O(2^nn)$

\item [Доказательство:] рассмотрим $B_1 = \{\neg, \wedge, \vee \}$

ДНФ для f $\O(\frac{2^n}{n})$

Количество функций $\{0, 1\}^n \to \{0, 1\} = 2^{2^n}$

Количество схем размер $\le S$

Пусть все формулы имеют арность $\le 2(\{\vee, \wedge, \neg\})$

Для каждой вершины указываем номер вершины из которой в нее ведут ребра.  Что бы это указать, достаточно $\O(\log S)$ битов.

Значит для шифрования схемы достаточно $\O(S \log S)$.

Количество схем размера $\le S$ не больше, чем число битовых строк длины $\O(S \log S)$ = $2^{CS\log S}$

\item[Следствие:]

$\exists$ константа $D \forall n$

$\exists f:\{0, 1\}^n \to \{0, 1\}$

$size(f) \ge \frac{2^n}{Dn}$

$S = \frac{2^n}{Dn}$ Число схем размера $\le s \le 2^{C\frac{2^n}{Dn}n} = 2^{\frac{C}{D}2^{n}}$

Если D > C, то число схем размера $\le \frac{2^n}{Dn}$ меньше общего числа функций.


\end{description}

\section{Предикатные формулы}

\begin{description}
\item[Определение:]

 $M \ne 0$ k-местным предикатом на M называется $P:M^k \to \{0, 1\}$

$k \in \{0, 1, 2\ldots\}$

k-ичная функция $f:M^k \to M$

\item[Сигнатура:] 
   
$\EuScript F = \{f_1^{(k_1)}, f_2^{(k_2)}, \ldots\}$

   $f_i^{(k_i)}$ ~--- к-местная функция. 
  
   $\EuScript P = \{p_1^{(l_1)}, p_2^{(l_2)}, \ldots\}$

\item[Пример:]

$\EuScript P = \{=^{(2)}\}$

$\EuScript F = \{+^{(2)}, *^{(2)}\}$

Г $= \{x_1, x_2, \ldots\}$ ~--- множество предметных переменных.

\item[Определение:] Терм

\begin{enumerate}
\item x ~--- предметная переменная, то x ~--- терм.
\item $f^{(k)} \in \EuScript, t_1, t_2, \ldots, t_k$ ~--- термы, тогда $f^{(k)}(t_1, t_2, \ldots, t_k)$ ~--- терм. 
\item Множество термов наименьшее множество строк, удовлетворяющие 1, 2.
\end{enumerate}

\item[Определение:] Атомарная формула.

    Если $p^{(k)} \in \EuScript P, t_1, t_2, \ldots, t_k$ ~--- термы

    атомарная формула ~--- $p^{(k)}(t_1, t_2, \ldots, t_k)$

\item[Определение:] Предикатная формула.

\begin{enumerate}

\item атомарная формула ~---- предикатная формула.

\item $\Phi$ ~--- предикатная формула, то $\neg \Phi$ ~--- тоже предикатная формула.

\item Если $\Phi$ и $\Psi$ предикатные формулы, то $(\Phi \vee \Psi), (\Phi \wedge \Psi), (\Phi \to \Psi)$

\item $\Phi$ ~--- формула, x ~--- предметная переменная $\forall x(\Phi), \exists x(\Phi)$

\item множество формул минимальное множество, удовлетворяющие 1-4.

\end{enumerate}
\end{description}

Область действия квантора.

Связанное вхождение переменной находится в области действия квантора на этой переменной. 

Свободная переменная ~--- не связанная.

Формулы без свободных вхождений переменных ~--- замкнутая.

\begin{description}
\item[Интерпретация:] для сигнатуры($\EuScript P, \EuScript F$) носитель M $\ne 0$

$p^{(k)} \in \EuScript P \leftrightarrow M^k \to {0, 1}$

$f^{(k)} \in \EuScript F \leftrightarrow M^k \to M$

Оценка для множества переменных $\Gamma \to M$

\item Значение формулы в данной интерпретации при данной оценке.

Терм с k свободными переменными задает отображение из $M^k \to M$

\begin{enumerate}
\item x ~--- переменная, то это тождественное отображение.

\item $f^{(k)}(t_1, \ldots, t_k)$ ~--- композиция функций.

Атомарная формула с k переменными задает предикат.

$\Phi$ ~--- предикат.

$\neg \Phi$ ~--- отрицание предиката.

$\Phi, \Psi, (\Phi \vee \Psi), (\Phi \wedge \Psi), (\Phi \to \Psi)$

$\forall x \Phi$,
$\exists x \Phi$ ~--- k-1 предикат
\end{enumerate}

\item[Определение] I - интерпретация сигнатуры $(\EuScript F, \EuScript P)$ с носителем M.

Предикат $P = M^k \to \{0, 1\}$ называется выразимым в I, если его можно задать формулой с k свободными переменными. 

\item Замкнутая формула называется тавтологией, если она истина при всех интерпретациях.

\end{description}

\subsection{Арифметика}

$\EuScript P = \{=\}$


$\EuScript F = \{+, *\}$


$N  \{0, 1, 2, \ldots\}$

\begin{enumerate}

\item "x = 0" x + x = x
\item "x = 1" $(x * x = x) \cap \neg(x + x = x)$
\item "x $\ge$ y" $\exists z (z + y = x)$
\item "x = 179" $\exists y (x = y + y + \ldots + y \cap y = 1)$
\item "x mod y == 0" $\exists z (zy = x)$
\item "x - простое" $\forall y((x mod y == 0) \to (y = 1) \vee (y = x)) \wedge \neg(x = 1)$
\item "x - степень 2" $\forall y((x mod y == 0) \wedge (y \text{~--- простое}) \to y = 2)$
\item "x - степень 4" $\exists y(y * y = x x - \text{степень двойки})$

$\tilde k$ = переводим k + 1 в двоичную систему и удаляем первую цифру. 

\item $\tilde x$ из нулей (x + 1) - степень двойки.
\item Строки $\tilde x$ и $\tilde y$ имеют одинаковую длину $\forall$ c ((c - степень 2) $\to (x+ 1 \le c) \leftrightarrow (y + 1 \le c))$
\item $\tilde z = \tilde x \tilde y$

$\exists t$ ((t - состоит из нулей) $\wedge (|\tilde t| = |\tilde y|) \wedge z = (x + 1)(t + 1) + (y - t) - 1$)

\item $\tilde x$ - начало строки $\tilde y$ $\exists t:\tilde y = \tilde x \tilde t$
\item $\tilde x$ - конец $\tilde y$
\item $\tilde x$ - подслово $\tilde y$ $\exists t$(($\tilde x$ конец $\tilde t$) $\wedge$ ($\tilde t$ начало $\tilde y$))
\item $\tilde x$ короче $\tilde Y$ $\exists zt (t = \tilde z \tilde x) \wedge (z \ne 0) \wedge |\tilde t| = |\tilde y|$

\end{enumerate}

\section{Кодирование конечных множеств в арифметике} 

\begin{description}
\item [Теорема:] Существует 3-местный выразимый предикат S(x, a, b):
\begin{enumerate}
\item $\forall a, b \in \mathbb N S_{a, b} = \{x | S(x, a, b) = 1\}$ конечно.
\item $\forall X \subset \mathbb N, X \text{~--- конечно } \exists a, b \in \mathbb N: X = S_{a, b}$
\end{enumerate}
\item [Доказательство:]

S(x, a, b) = $\tilde x \tilde x$ короче $\tilde a$  и  $\tilde a \tilde x \tilde a$ подстрока $\tilde b$

\begin{enumerate}
\item $S_{a, b}$ - конечно.
\item $X = \{x_1, x_2, \ldots, x_n\}$

$a: \tilde a$ длиннее всех $\tilde x_i \tilde a = 10\ldots01$

$b: \tilde b = \tilde a \tilde x_1 \tilde a \tilde x_2 \ldots \tilde x_n \tilde a$  

\end{enumerate}

\item [x - степень 6]

$\exists a, b (S(x, a, b) \wedge \forall y (S(y, a, b) \to ((y = 1) \vee \exists t((6*t = y) \wedge S(t, a, b) ) ) )$

{\bf $x = 6^n$}

$[x, y] = (x + y)^2 + x$

$first(x, p) \forall z((z^2 \le p) \wedge \forall t((t > z) \to (t^2 > p))) \to (x + z = p))$

$x = 6^n$

$\exists a, b (S([x, n], a, b) \wedge \forall y(S(y, a, b) \to \exists z, m y = [z, m] \wedge (z = 1 \wedge m = 0) \vee \exists k z = 6k \wedge S([k, m - 1], a, b)$

\end{description}


\section{Доказательство непрерывности методом автоморфизмов}

\begin{description}

\item $\mathbb Z, =, +$ невыразимо x < y.

P(x, y)

$\downarrow$

P(-x, -y) поведение не должно было изменится. 

\item[Определение] I - интерпретация с носителем M.

$\alpha : M \to M$ называется автоморфизмом I.

\begin{enumerate}
\item $\alpha$ ~--- биекция
\item $\forall p^{(k)} \in \EuScript P^{(k)}$ устойчиво по $\alpha$ $p^{(k)}(\alpha(x_1), \ldots, \alpha(x_n)) = p^{(k)}(x_1, x_2, \ldots, x_n)$
\item $\forall f^{(k)} \in \EuScript F$

$f^{(k)}$ устойчиво относительно $\alpha$ 

$f^{(n)}(\alpha(x_1), \ldots, \alpha(x_n)) = \alpha(f^{(k)}(x_1, \ldots, x_n))$
\end{enumerate}

\item[Теорема]
    Если $P:M^k \to \{0, 1\}$ выразим в I, $\alpha$ ~--- автоморфизм I $\Rightarrow$ P устойчиво относительно автоморфизмов. 
\item[Доказательство]
\begin{enumerate}
\item Термы задают устойчивые относительно $\alpha$ функции.
\item Атамарные формулы задают устойчивые предикаты. 
\item $\neg \Phi$ 

$\Phi_1 \vee \Phi_2$

$\Phi_1 \wedge \Phi_2$

$\Phi_1 \to \Phi_2$

\item $\forall x \Phi(x)$

$\exists x \Phi(x)$

$P(x, y_1, y_2, \ldots)$

$P(\alpha(x), y_1, \ldots)$ ~--- так как биекция $\alpha (x)$ пробегает все значения $M \Rightarrow$ истина.
\end{enumerate}
\item[Примеры]
\begin{enumerate}
\item $(\mathbb Z,  =, <) x = 0$

$\alpha (x) = x - 1$

\item $(\mathbb Q,  =, <, +) x = 1$

$\alpha (x) = 2x$

\item $(\mathbb R,  =, <, 0, 1) x = \frac{1}{2}$

$\alpha (x) = x *|x|$
\end{enumerate}

\end{description}

\section{Конечные множества}

$\mathbb N = \{1, 2, \ldots \}$

$[n] = \{1, 2, \ldots, n\}$

Число подмножеств [n] = $2^n$

$A^k_n = |\{(s_1, \ldots, s_k) \subset [n] * \ldots *[n]\} s_i \ne s_j$ при всех $i, j|$

$A^1_n = n$

$A^k_n = n A^{k - 1}_{n - 1}$

$A^k_n = n(n - 1)\ldots(n - k + 1) = \cfrac{n!}{(n - k)!}$

$C_n^k = \left(\begin{smallmatrix} n \\ k\end{smallmatrix}\right) = |{s \subset [n]| |s| = k}|$ число сочетаний из n по k.

$A_n^k = C_n^k * k!$

$C_n^k = C_{n - 1}^{k - 1} + C_{n - 1}^{k}$ (число подмножеств, содержащих 1 + число подмножеств не содержащих 1)

$C_n^k = C_n^{n - k} = \cfrac{n!}{k!(n - k)!}$

{\bf Треугольник Паскаля}

\begin{tabular}{c c c c c c c}
&&&$C_1^0$&&&\\
&&$C_1^0$&&$C_1^1$&&\\
&$C_2^0$&&$C_2^1$&&$C_2^2$&\\
$C_3^0$&&$C_3^1$&&$C_3^2$&&$C_3^3$\\
\end{tabular}

\begin{tabular}{c c c c c c c}
&&&$1$&&&\\
&&$1$&&$1$&&\\
&$1$&&$2$&&$1$&\\
$1$&&$3$&&$3$&&$1$\\
\end{tabular}

{\bf Бином Ньютона}

$(a + b)^n = \sum_k C_n^k a^k b^{n - k}$

{\bf База} a + b = a$C_1^0 + b C_1^1$ 

{\bf Переход} $(a + b)^{n + 1} = (a + b)^n(a + b) = (\sum_{k = 0}^{n} C_n^k a^k b^{n - k})(a + b) = \sum_{k = 0}^{n + 1} C_{n + 1}^{k} a^kb^{n - k + 1}$


$(1 + 1)^n = C_n^0 + C_n^1 + \ldots$

$(1 - 1)^n = C_n^0 - C_n^1 + \ldots$

\section{Характеристическая функция}

$\mathbb \chi A \subset X$

Характеристическая функция: $\chi_A: x \to \{0, 1\}$

$\chi_A(x) = \begin{cases} 1, &x \in A\\ 0, & \text{иначе} \\ \end{cases}$

$\chi_{A\cap B} (x) = \chi_A(x)\chi_B(x)$

$\chi_{X/A}(x) = 1 - \chi_A(x)$

$\chi_{A\cup B}(x) = X_{\overline{\overline{A} \cap \overline{B}}} = 1 - (1 - \chi_{A})(1 - \chi_B) = \chi_{A} + \chi_{B} - \chi_A\chi_B$

$\chi_{A_1\cup A_2 \cup \ldots \cup A_n} = 1 - (1 - \chi_{A_1})(1 - \chi_{A_2})\ldots(1 - \chi_{A_n})$

$|A| = \sum_x\chi_A(x)$

\subsection{Формула включений-исключений}

$|A_1\cup A_2 \cup \ldots \cup A_n| = \sum_x\chi_{A_1\cup \ldots \cup A_n}(x) = \sum_{i = 1}^{n}|A_i| - \sum_{i \ne j}|A_i\cap A_j| + \sum_{i \ne j \ne k}|A_i\cap A_j \cap A_k| - \ldots$

\section{Количество счастливых билетов}

Счастливый билет, у которого $a_1 + a_2 + a_3 = a_4 + a_5 + a_6$.

$\overline{a_1a_2a_3a_4a_5a_6} \leftrightarrow \overline{a_1a_2a_3(9 - a_4)(9-a_5)(9-a_6)}$

$|\{\text{количество счастливых билетов}\}|= |\{\text{билеты с суммой цифр 27}\}|$


Из метода шаров и перегородок количество разбиений $C_{32}^5 - |c_1 \cup c_2 \cup \ldots \cup c_6|$

$c_1$ ~--- множество разбиений числа 27 на 6 неотрицательных слагаемых у которого $a_1 \ge 10$

$c_2$ ~--- множество разбиений числа 27 на 6 неотрицательных слагаемых у которого $a_2 \ge 10$

$\ldots$

\section{Равномощные множества}

\begin{description}
\item[Определение:] множества A и B равномощны, если $\exists$ биекция $f:A \to B$
\end{description}

\begin{enumerate}
\item равномощность двух отрезков.

$[a, b] \to [c, d]$

$x \to (x - a)(d - c)/(b - a) + c$ 

\item равномощность множества последовательностей из 0 и 1 и множества натуральных чисел. 

$S \subset N$

$x_n = \begin{cases} 
1, & \text{если } n \in S\\
0, & \text{если } n \notin S\\
 \end{cases}$
\end{enumerate}

\subsection{счетные множества}

\begin{description}
\item[Определение:] множество называется счетным, если оно равномощно $\mathbb N$

$\mathbb N \to^{f} S = \{f(1), f(2), f(3), \ldots \}$

\item[Свойства счетных множеств:]

\begin{enumerate}
\item Любое подмножество счетного множества конечно, либо счетно. 

A - счетно.

$A =\{f(1)(g(1)), f(2), f(3), f(4)(g(2)), \ldots \}$

$g(k) = \text{первый элемент в последовательности} A \text{после} g(k - 1)$

\item Объединение конечного или счетного числа конечных множеств конечно или счетно.

$A_1 f_1(1) f_1(2) f_1(3) f_1(4) \ldots$

$A_2 f_2(1) f_2(2) f_2(3) f_2(4) \ldots$

$A_3 f_3(1) f_3(2) f_3(3) f_3(4) \ldots$

$A_4 f_4(1) f_4(2) f_4(3) f_4(4) \ldots$

$\ldots$

$f_1(1) f_1(2) f_2(1) f_1(3) f_2(2) f_3(1) \ldots$

\item Любое бесконечное множество содержит счетное подмножество. 

$x_1, x_2, x_3, \ldots$  если не можем выбрать $\Rightarrow$ множество конечно.
\end{enumerate}

\end{description}

\section{Бесконечное множество}

\subsection{Примеры счетных множеств}

\begin{enumerate}
\item $\mathbb Q = \cfrac{p}{q}$

$\cfrac{1}{1}, \cfrac{2}{1}, \cfrac{3}{1}, \cfrac{4}{1}, \ldots$

$-\cfrac{1}{1}, \cfrac{2}{1}, \cfrac{3}{1}, \cfrac{4}{1}, \ldots$

$\cfrac{1}{2}, \cfrac{2}{2}, \cfrac{3}{2}, \cfrac{4}{2}, \ldots$

$\ldots$

Объединение счетного числа счетных множество ~---- счетно.

\item $\mathbb N^k$ ~--- счетно.

Индукция по k.

{\bf База} $\mathbb N^2$ объединение счетного числа счетных множеств.

{\bf Переход} $k \to k + 1$

$\mathbb N^{k + 1} (a, x)$

$a \in \mathbb N^k$

$x \in \mathbb N$

Оба множества счетны. Можем занумировать их декартово произведение.

\item множество конечных последовательностей натуральных 

Количество последовательностей длины 1 ~--- $\mathbb N$

Количество последовательностей длины 2 ~--- $\mathbb N^2$

$\ldots$

Объединение счетно. 

\item алгебраических чисел ~--- счетно.

Количество уравнений ~--- счетно

Корней у каждого уравнения конечно. 

$\Rightarrow$ их объединение счетно. 
\end{enumerate}

\subsection{Объединение бесконечного и счетного множества}

\begin{description}
\item[Теорема:] A ~--- бесконечное, B ~---- счетное или конечное, то $A \cup B$ равномощно A.
\item[Доказательство:] 

B' = B/A

B' = счетное или конечное

$B' \cap A = 0$

$A \cup B' = A \cup B$

A ~--- бесконечное $\Rightarrow$ в A есть счетное подмножество Q.

$A = Q \cup (A/Q)$

$A \cup B' = (Q \cup B') \cup (A / Q)$

Q равномощно B'
\end{description}

\subsection{Равномощность [0, 1] и множество бесконечных последовательностей из 0 и 1}

\begin{description}
\item[Теорема:] [0, 1] равномощен множеству бесконечных последовательностей из 0 и 1.

\item[Доказательство:]
$\alpha \in [0, 1]$

Если $\alpha < \frac{1}{2}$ на первое место последовательности ставим 0, иначе 1. Переходим к отрезку, где лежит $\alpha$

Это биекция.
\end{description}



\subsection{Равномощность квадрата и отрезка}

\begin{description}
\item[Теорема:] [0, 1] $\times$ [0, 1] равномощен [0, 1].

\item[Доказательство:]
$(\alpha, \beta) \in [0, 1] \times [0, 1]$

$\alpha \leftrightarrow a_1a_2a_3 \ldots$

$\beta \leftrightarrow b_1b_2b_3 \ldots$

$(\alpha, \beta) \leftrightarrow a_1b_1a_2b_2a_3b_3\ldots$

\end{description}

\section{Теорема Кантора-Бернштейна}

\begin{description}
\item[Теорема:] Если A равномощно подмножеству B, B равномощно подмножеству A, то A и B равномощны.

\item[Доказательство:]
    \begin{description}
    \item[Лемма:] $A_0 \supset A_1 \supset A_2$ 

    $A_0$ равномощно $A_2$, тогда $A_0$ равномощно $A_1$.

    \item[Доказательство:]
    $f:A_0 \to A_2$ ~--- биекция.

    $f(A_1) = A_3 \subset A_2$

    $f(A_2) = A_4 \subset A_3$

    $\ldots$

    $A_{n + 2} = f(A_n)$

    $A_0 \supset A_1 \supset A_2 \supset A_3 \ldots$

    $c_0 = A_0/A_1$

    $c_1 = A_1/A_2$

   $c_2 = A_2/A_3$

    $\ldots$

     $A_0 = c_0 \cup c_1 \cup c_2 \cup \ldots$

    $A_1 = c_1 \cup c_2 \cup \ldots$

     $f(c_i) = f(A_i/A_{i + 1}) = f(A_i)/f(A_{i + 1}) = A_{i + 2} / A_{i + 3} = c_{i + 2}$

     Биекция:

    $c_0 = c_2$

    $c_1 = c_1$

    $c_2 = c_4$

    $c_3 = c_3$

    $\ldots$
   \end{description}
  
  $f:A \to B_1  B_1 \subset B, f$~--- биекция.

  $g:B \to A_1  A_1 \subset A, g$~--- биекция.

   $g(B_1) = A_2 \subset A_1$

   $B_1 $~--- равномощно $A_2$

    $A $ ~---- равномощно $B_1$ 

    $\Rightarrow$ A равномощно $A_2$

    $A \supset A_1 \supset A_2 \Rightarrow A_1$ равномощно $A \Rightarrow A$ равномощно $B$.  

\end{description}


\section{Теорема Кантора}  

\begin{description}
\item[Теорема Кантора:] [0, 1] несчетно.
\item[Доказательство:] 

Пусть пронумировали.

$1: x_{11}, x_{12}, x_{13}, \ldots$

$2: x_{21}, x_{22}, x_{23}, \ldots$

$3: x_{31}, x_{32}, x_{33}, \ldots$

$\ldots$

$\neg x_{11}, \neg x_{22}, \neg x_{33}, \ldots$ - не пронумировали.

\item[Следствие:] множество $2^{\mathbb N}$~--- несчетно.

\item [Обобщенная теорема Кантора:] X не равномощно множеству своих подмножеств $2^x$
\item [Доказательство:] Пусть f ~--- биекция $x \to 2^x$.

$D = \{a \in X| a \notin f(a)\}$

$D \subset X$

Пусть $f(d) = D$

\begin{enumerate}
\item $d \in D \Rightarrow d \notin f(d)$ ~--- противоречие
\item $d \notin D \Rightarrow d \in f(d)$ ~---- противоречие
\end{enumerate}
\end{description}

\subsection{Континум}

\begin{description}
\item[Определение:] Множество имеет мощность континум если оно равномощно [0, 1]
\item[Пример:] Существует неалгебраическое вещественное число. 
\item[Пример:] Существует характеристическая функция не вычисляемая программой. 

Количество программ счетно, количество множеств континум.
\end{description}

\section{Введение в графы}

\begin{description}
\item[Ориентированный граф:] (V, E), V ~--- множество

$E \subset V \times V$
\item[Петля:] $(u, u) \in E$

\item[Входящая степень:] $d_{in}(u) = |\{(v, u) \in E | v \in V\}|$

\item[Исходящая степень:] $d_{out}(u) = |\{(u, v) \in E | v \in V\}|$ 


\item[Неориентированный граф:] (V, E), 
$E \subset \{\{v, u\}|v\in V, u \in V\}$

\item[Степень вершины:] $deg(v) = |\{e \in E | v \in e\}|$

\item[Простой граф:] ~--- неориентированный граф без петель и кратных ребер. 
\end{description}

\subsection{Компоненты связности, пути и циклы}

\begin{description}
\item[Путь в ориентированном/неориентированном графе:] 

$V_1, V_2, V_3, \ldots, V_n \in V: \forall i \in [n - 1]  (V_i, V_{i + 1}) \in E$

\item[Простой путь:] ~---- путь в котором все вершины различны. 

\item[Длина пути:] ~--- $u_1, \ldots, u_n = n - 1$
\item[Определение:] вершины u и v связаны путем, если существует путь $w_1 = u, w_2, \ldots, w_k = v$

\item[Замечание:] Если u и v связаны путем, то они связаны простым путем. 

\item[Доказательство:] самый короткий путь ~--- простой. 

$u, \ldots, w, \ldots, w, \ldots v \to u, \ldots, v$

\item[Утверждение:] Отношение быть связным путем в неориентированном графе ~---- отношение эквивалентности.
\end{description}

В ориентированных графах $u \sim v$ из u в v есть путь и из v в u есть путь.

\begin{description}
\item[Определение:]
Разбиение на классы эквивалентности в неориентированном графе ~--- компоненты связности

\item[Определение:]
Разбиение на классы эквивалентности в ориентированном графе ~--- компоненты сильной связности

Фактор граф на отношение эквивалентности ~--- компоненты сильной связности С. Есть ребро между $c_i$ и $c_j$ если $\exists u \in C_i, v \in c_j (u, v) \in E$

\item[Утверждение:] Фактор граф - DAG(граф без циклов)

В фактор графе нет петель, по определению. Путь есть цикл и в цикле лежит $C_i$ и $C_j$. Рассмотрим вершины u из $C_i$ и v из $C_j$, тогда существует путь из u в v и из v в u, значит они должны лежать в одном классе эквивалентности. 

\item[Цикл] ~--- это путь $v_1, \ldots, v_n: v_n = v_1$

\item[Длина цикла] ~--- n - 1

\item[Простой цикл] $v_1, \ldots, v_{n- 1}$ ~--- различны. 

$(v_1, v_2), \ldots (v_{n - 1}, v_n)$ ~--- различные ребра. 
\end{description}

\subsection{Деревья}

Неориентированный граф. 

\begin{description}
\item[Определение:] Граф связный, если в нем одна компонента связности.
\item[Определение:] Дерево ~--- это связный граф без простых циклов.

\item[Утверждение:] Если в дереве $\ge 2$ вершины, то в нем $\ge$ 2 вершины степени 1(висячие вершины).

\item[Доказательство:] Пусть $u_1, u_2, \ldots, u_k$ ~--- простой путь максимальной длинны.  $u_1$ и $u_k$ имеют степень 1.

\item[Утверждение:] Если в дереве n вершин, то в нем $n - 1$ ребро.

\item[Доказательство:] Индукция по числу вершин.

{\bf База:} n = 1

{\bf Переход:} Пусть u вершина степени 1. Выкинем ребро. (G/u) ~--- дерево, по предположению индукции в нем n - 2 ребра $\Rightarrow$ в G n  - 1 ребро. 

\item[Теорема:] Следующий утверждения эквивалентны. 

\begin{enumerate}
\item G ~--- дерево
\item cвязны граф n - 1 ребро.
\item G ~--- граф без циклов, в котором n - 1 ребро. 
\item G ~--- граф без циклов, но при добавление любого ребра появляется цикл. 
\item G ~--- связный граф, при удаление любого ребра связность теряется.
\end{enumerate}

\item[Доказательство:] 1) $\to$ 2) доказали

2) $\to$ 3)

Пусть в G есть цикл. Будем удалять по ребру из цикла, пока циклы не закончатся. 

Получилось дерево $\Rightarrow$ количество ребер n - 1 $\Rightarrow$ ничего не удалили.

$3) \to 1)$

Если граф не связен можем добавить ребро между компонентами связности и циклов не появится. Добавляем пока не станет деревом, а в дереве n - 1 ребро, значит, мы ничего не добавили. 

$1) \to 4)$

Между любыми двумя вершинами есть простой путь, добавим ребро и получим цикл.

$4) \to 1)$ 

Если бы граф не был связан смогли бы добавить ребро между компонентами.


$1) \to 5)$

Пусть не теряется, тогда когда вернем ребро, получим цикл.

$5) \to 1)$

Если бы в графе был цикл, то могли бы удалить ребро.

\item[Остовное дерево:] Из любого связного графа можно выкинуть несколько ребер так, что бы он стал деревом.

Дерево, которое получилось ~--- остовное дерево. 

\item[Доказательство:] Пока есть цикл, удаляем в цикле ребро. 

\item[Лес]~--- граф, каждая компонента связности которого ~--- дерево. 

\item[Лемма:] Если G ~--- неориентрованный связный граф, то |E| $\ge$ |V| - 1
\item[Доказательство:] Если G ~--- дерево, то |E| = |V| - 1

Рассмотрим остовное дерево G', в нем |V| - 1 ребро, в исходном графе ребер больше. 
\end{description}

\section{Теорема Келли}

\begin{description}
\item[Теорема Келли:] число деревьев с V = [n] равняется $n^{n - 2}$
\item[Доказательство(код Прюффера)]

\begin{center} 
\includegraphics[width=2in, keepaspectratio]{im4.jpg} 
\end{center}

Находи лист с минимальным номером, выкидываем, записываем, к чему прикрепляется. 

Повторяем, пока число вершин $\ge 2$

2, 3, 2, 2, 4

Получилось n - 2 числа от 1 до n.

Это биекция.

Индукцией по n показываем, что каждому элементу из $n^{n - 2}$ соответствует ровно одно дерево. 

{\bf База:} n = 2
{\bf Переход} Восстанавливаем первый лист и удаляем из последовательности первый элемент. По предположению индукции дерево восстанавливается однозначно.

\section{Эйлеров путь, цикл. Раскраски графов}
\subsection{Эйлеров цикл}

\begin{description}
\item[Эйлеров цикл] ~--- цикл, который проходит по всем ребрам ровно один раз. 

\item[Теорема:] Пусть G ~--- связный граф. В G есть эйлеров цикл $\Leftrightarrow$ степени всех вершин четны. 

\item[Доказательство:] $\Rightarrow$ У каждой вершины на каждое входящее ребро, есть исходящее. 

$\Leftarrow$ 

Рассмотрим самый длинный цикл, в котором не повторяются ребра C. Выкинем из G все ребра цикла C получился граф G'. В G' тоже все степени четные. 

Цикл обязательно закончится в начальной вершин. Пойдем по ребру, найдем еще один цикл. 

Если E' = 0, то все доказано. 

Пусть E' != 0

\begin{enumerate}
\item Из связности G следует, что хотя бы из одной вершины С выходит ребро в E'.
\item Начинаем путь в G' по этому ребру, получаем цикл C'.
\item Склеиваем С и C' в большой цикл. 
\end{enumerate}

Противоречие с максимальностью C.
\end{description}
\end{description}

\subsection{Эйлеров путь}
\begin{description}

\item [Эйлеров путь] ~--- это путь проходящий по всем ребрам один раз. 

\item [Теорема:] G ~--- связный граф. В G есть эйлеров путь $\Leftrightarrow$ в G либо 0, либо 2 вершины нечетной степени. 

\item [Доказательство:] $\Rightarrow$ все понятно

$\Leftarrow$ Если 0, то есть Эйлеров цикл, если 2, соединим ребром. 
\end{description}

\subsection{Раскраска графов}


\begin{description}
\item[Правильная раскраска графов:] G(V, E) неориентированный граф. 

Правильная раскраска в k цветов.

Двудольный (2-дольный)
\item[Теорема:] Граф двудольный $\Leftrightarrow в нем нет циклов нечетной длины$

\item[Доказательство:] $\Rightarrow$ очевидно, так как вершины цикла обязаны менять цвет. 

$\Leftarrow$ Пусть нет нечетных циклов. 

В каждой компоненте раскрасим отдельно. 

Теперь G - связный граф $u \in V$ 

Определим раскрасим $h(v) = \begin{cases} 1, & \text{если путь из u в v имеет нечетную длину}\\
2, &\text{если четно}\end{cases}$

Если раскраска не однозначна, то существует цикл нечетной длины.

Пусть h неправильная раскраска, то существует цикл нечетной длинны. 

\item[Лемма:] Если в G нет простых нечетных циклов, тот там нет нечетных циклов. 

\item[Доказательство:] Рассмотрим самый короткий нечетный цикл. 

Пусть он не простой. $u, \ldots, v, \ldots, v, \ldots, u$

В центе нечетный цикл, или если выкинуть получится нечетный. Значит, нечетный цикл не самый короткий. 
\end{description}

\section{Конечная теория вероятностей}

Конечное вероятностное пространство. 

$\Omega$ ~--- конечное множество(пространство элементарных событий)

$p: 2^{\Omega} \to [0, 1]$

Вероятностная мера:
\begin{enumerate}
\item $P(\Omega) = 1$
\item $A, B \subset \Omega, A \cap B = 0, P(A \cup B) = P(A) + P(B)$
\end{enumerate}

Элементы множества $\Omega$ ~--- элементарные события.

$A \subset \Omega$ A ~--- событие.

$P(A)$ ~--- вероятность события.

Свойства конечного вероятностного пространства. 
\begin{enumerate}
\item P(0) = 0, $P(\Omega) + P(0) = P(\Omega)$
\item $A \subset B$, то $P(A) < P(B), P(B) = P(A) + P(B/A) \ge 0$
\item $\Omega = \{\omega_1, \omega_2, \ldots, \omega_n\}$

$P_1 = P(\{\omega_1\})$

$P_2 = P(\{\omega_2\})$

$\ldots$

$P_n = P(\{\omega_n\})$

$P(A) = \sum_{\omega_i \in A} P_i$

\item $P(A_1 \cup A_2 \ldots A_n) \le \sum_{i = 1}^{n} P(A_i)$
\item Формула включений/исключений.

$P(A_1 \cup A_2 \ldots \cup A_n) = \sum_{i =1}^{n}P(A_i)  - \sum_{i != j}^{n} P(A_i \cap A_j) + \ldots$

\end{enumerate}


 
\subsection{Задача о галстуках}

В каждом кружке d человек. Всего кружков $\le 2^{d - 1}$
\begin{description}
\item[Утверждение] Можно выдать галстуки так, что бы в каждом кружке были как с галстуком, так и без.

\item[Доказательство] Рассмотрим случайный способ раздачи галстуков, что бы все способы были равновероятны. 

$A_i$ ~--- в i-ом кружке либо все дети с галстуком, либо без. 

$P(A_i) = (2^{n - d} + 2^{n - d}) \cfrac{1}{2^n} = 2^{1 - d}$

$P(\exists$ кружок, в котором либо все в галстуке, либо все без$) = P(A_1 \cup A_2 \cup \ldots \cup A_k) \le 2^{d - 1} * 2^{1 - d} = 1$

$P(A_i \cap A_j) > 0 \Rightarrow P < 1$ 

\end{description}

\section{Теорема Эрдеша-Ко-Радо}

\begin{description}

\item[Теорема Эрдеша-Ко-Радо]

$S = \{0, 1, \ldots, n - 1\}$

$\EuScript F \subset 2^s$

$\forall A \in \EuScript F |A| = k k \le \cfrac{n}{2}$

$\forall A, B \in \EuScript F A \cap B \ne 0$

Тогда $|\EuScript F| \le C_{n-1}^{k - 1}$

\item[Доказательство:]

    $A_s = \{s, s  +1, \ldots, s+ k -1\}$ mod n 
    \begin{description}
    \item[Лемма:] $\EuScript F$ содержит $\le k$ элементов $A_s$
    \item[Доказательство:] $A_s \in \EuScript F$ 

      Рассмотрим элементы, которые пересекаются с $A_s$ их 2k - 2

      Разбиваем на пары:

     $A_{s - k +1}  - A_{s + 1}$

    $\ldots$

     $A_{s - 1}  - A_{s + k - 1}$   

     Из каждой пары можем взять не более одного элемента. 

    $\Downarrow$

    Кроме $A_s$ может быть $\le k - 1$ элемента.

   $\Downarrow$

   $\EuScript F$ содержит $\le k$ элементов $A_s$ 
    \end{description}
   $\sigma:[n]\to [n]$ ~--- биекция.
    
    $i \in [n]$

   $A_{\sigma, i} = \{\sigma (i), \sigma (i + 1), \ldots, \sigma (i + k - 1)\}$

    $\Omega = \{(\sigma ,i)| \sigma \in S_n, i \in n\}$
   
     $|\Omega| = n n!$

   $P(\{\sigma, i\}) = \cfrac{1}{n n!}$

     $X \subset \Omega$

      $X = \{(\sigma, i)|A_{\sigma, i} \in \EuScript F\}$

     
     $P(x)\le \cfrac{k n!}{n n!}$


     $P(x)\le \cfrac{k}{n}$

    $P(x) = \cfrac{|\EuScript F|}{C_n^k}$

      $|\EuScript F| \le \cfrac{k}{n}C_n^k = C_{n -1}^{k - 1}$
\end{description}

\section{Математическое ожидание}

\subsection{Случайная величина}

\begin{description}
\item[Случайная величина:] $\xi : \Omega \to \mathbb R$

\item[Примеры:]

\begin{enumerate}
\item $\Omega = \{0, 1\}^3$

$\xi (\omega)$ = число единиц в $\omega$

\item $\Omega = \{1, 2, 3, 4, 5, 6\}$

$\xi (\omega) = \omega$

\item $\Omega = \{$ множество простых графов на n вершинах$\}$

$|\Omega| = 2^{C_n^2}$

$P(\omega) = \cfrac{1}{2^{C_n^2}}$

$\xi (\omega)$ ~--- число ребер в $\omega$
\end{enumerate}
\end{description}

\subsection{Математическая ожидание}

\begin{description}

\item[Математическое ожидание:] $E[\xi] = \sum_{\omega \in \Omega} P(\{\omega \}) \xi{\omega}$

\item[Лемма(Линейность):] $\alpha, \beta \in \mathbb R$

Тогда $E[\alpha \xi + \beta \eta] = \alpha E[\xi] + \beta E[\eta]$

\item[Доказательство:] 

$E[\alpha \xi + \beta \eta] = \sum_{\omega \in \Omega} P(\omega)(\alpha \xi(\omega) + \beta \eta(\omega)) = \alpha \sum_{\omega}P(\omega)\xi (\omega) + \beta \sum_{\omega}P(\omega)\eta (\omega) = \alpha E[\xi] + \beta E[\eta]$

\item[Пример:]

$\xi (\omega)$ ~--- число ребер в графе $\omega$

$\xi(\omega_e) = \begin{cases} 1, & \text{ребро e есть в} \omega \\ 0, & \text{иначе} \end{cases}$

$\xi = \sum_e \xi_e$

$E[\xi_e] = \sum_{\omega, e  \text{~--- ребро}}P(\{\omega\}) = \cfrac{1}{2}$

$E[\xi] = \sum_e E[\xi_e] = C_n^2 \cfrac{1}{2}$

\item $\xi$ принимает значения $a_1, a_2, \ldots, a_n$

$P[\xi \in A] = P(\xi_{\omega} | \xi(\omega) \in A)$

$A \subset \mathbb R$

$q_i = P[\xi = a_i]$

$E[\xi] = a_1 q_1 + a_2 q_2 + \ldots + a_n q_n$

\item[Лемма(принцип усреднения):]

$\xi: \Omega \to \mathbb R E[\xi] = \mu$

Тогда $\exists \omega \in \Omega$, что $\xi (\omega)\ge \mu$ и $\exists \omega' \in \Omega \xi (\omega') \le \mu$

\item[Доказательство:] 
   Пусть все $\xi (\omega) < \mu$

Тогда $E[\xi] = \sum_{\omega \in \Omega}\xi (\omega) P(\{\omega \}) < \mu (\sum_{\omega}P(\{\omega\}) = 1)$

\end{description}

\subsection{Турнир с большим числом гамильтоновых путей}
\begin{description}
\item[Турнир~---] ориентированный граф на n вершин между любыми 2-мя вершинами ровно одно ребро. 

\item[Гамильтонов путь ~---] простой путь, который проходит по каждой вершине ровно один раз.

\item[Утверждение:] $\exists$ турнир на n вершинах, в которых $\ge \cfrac{n!}{2^{n - 1}}$ гамильтонов путь. 

\item[Доказательство:] 

$\Omega = \{$ множество турниров на n вершинах$\}$

$|\Omega| = 2^{C_n^2}$

$P(\omega) = \cfrac{1}{2^{C_n^2}}$

$X: \Omega \to \mathbb R$

$X(\omega)$ ~--- число гамильтоновых путей в $\omega$

Пусть $\sigma \in S_n$

$X_{\sigma}(\omega) = \begin{cases} 1, & \text{если $\sigma(1), \sigma(2), \ldots$ гамильтонов путь} \\ 0, & \text{иначе}\end{cases}$

$X = \sum_{\sigma} X_{\sigma}$

$E[X_{\sigma}] = P(\{\omega | \sigma(1), \sigma(2), \ldots \text{~--- гамильтонов путь}\})$

$E[X_{\sigma}] = \cfrac{2^{C_n^2 - (n - 1)}}{2^{c_n^2}} = \cfrac{1}{2^{n - 1}}$

$E[x] = \cfrac{n!}{2^{n - 1}}$ 
\end{description}

\section{Набор выполняющий $\frac{7}{8}$ дизъюнктов 3-КНФ. Неравенство Маркова}

\subsection{Набор выполняющий $\frac{7}{8}$ дизъюнктов 3-КНФ}

\begin{description}

\item[Пример:] 3-КНФ

e ~--- формула в 3-КНФ, в каждый дизъюнкт входит 3 различных переменных . 

$(x\vee y \vee \neg z) \wedge (x \vee z \vee \neg y) \wedge \ldots$

m ~---  число  дизъюнктов. 

$\exists$ набор значений переменных, который выполнит $\ge \cfrac{7}{8} m дизъюнктов$ 

\item[Доказательство:]

$\Omega = \{$множество наборов значений переменных$\}$

$X(\omega) =$ число выполненных дизъюнктов.  

С ~--- дизъюнкт

$X_c(\omega) = \begin{cases} 1, & \text{если с выполняет $\omega$} \\ 0, & \text{иначе} \\ \end{cases}$

$E[X_c] = \cfrac{7}{8}$

$E[X] = \sum_{c} E[X_c] = \cfrac{7}{8} m \Rightarrow$ есть такое значение, которое выполняет $\cfrac{7}{8} m$ дизъюнктов.
\end{description}

\subsection{Неравенство Маркова}

\begin{description}
\item[Теорема(Неравенство Маркова):]

$X: \Omega \to \mathbb R, X \ge 0, E[x] > 0$

Тогда $\forall c > 0$ 

$P[x \ge c E[x]] = P(\{\omega | X(\omega) \ge cE[x]\}) \le \cfrac{1}{c}$

\item[Доказательство:]

Пусть $P[x \ge c E[c]] > \frac{1}{c}$

$E[x] = \sum_{\omega} X(\omega)P(\{\omega\}) \ge \sum_{\omega: X(\omega) \ge c E[x]} X(\omega)P(\{\omega\}) \ge c E[x] \sum_{\omega| X(\omega) \ge c E[x]} P(\{ \omega\}) > E[x]$ 

\end{description}

\subsection{Алгоритм, который находит набор.}

\begin{description}

\item $\Phi, m$ дизъюнктов

y ~--- число не выполненных дизъюнктов.

$E[y] = \cfrac{1}{8} m$

$P[y > \cfrac{1}{8} m] = P[8y > m] = P[8y \ge m + 1] = P[y \ge (\cfrac{1}{8} m) \cfrac{m + 1}{m}] \le \cfrac{m}{m + 1} = 1 - \cfrac{1}{m + 1}$

\item[Алгоритм]

Повторить t раз. 

Взять случайный набор и проверить подходит ли он.

$P[\text{Алгоритм не нашел набор, выполняющий $\ge \cfrac{7}{8} m$}] \le (1 - \cfrac{1}{m + 1}) ^ t = ((1 - \cfrac{1}{m + 1})^{m + 1})^{N} < \cfrac{1}{e^{N}}$

$t = N(m + 1)$
\end{description}

\section{Независимые событие}

\subsection{Независимые события}

\begin{description}
\item[Определение:] $\Omega, p$

$A_1, A_2, \ldots, A_n \subset \Omega$

События $A_1, A_2, \ldots, A_n$ независимы, если $P(A_1 \cap A_2 \cap \ldots A_n) = P(A_1)*P(A_2)*\ldots P(A_n)$
\end{description}

\subsection{независимые случайные величины}

\begin{description}
\item[Определение:] $X_1, X_2, \ldots, X_n: \Omega \to \mathbb R$ независимые, если 

$\forall a_1, a_2, \ldots, a_n \in \mathbb R$

$P[x_1 = a_1, x_2 = a_2, \ldots, x_n = a_n] = P[x_1 = a_1]P[x_2 = a_2] \ldots P[x_n = a_n]$ ~--- независимые события.

\item[Свойства:] 

\begin{enumerate}
\item $x_1, x_2, \ldots, x_n$ ~--- независимые.

$A_1, A_2, \ldots, A_n \in \mathbb R$

Тогда $P[x_1 \in A_1, x_2 \in A_2, \ldots x_n \in A_n] = \prod_{i = 1}^{n} P(x_i \in A_i)$

\item $x_1, x_2, \ldots, x_n$ ~--- независимые

$f_1, f_2, \ldots, f_n: \mathbb R \to \mathbb R$

Тогда $f_1(x_1), f_2(x_2), \ldots, f_n(x_n)$ ~--- независимые.
\end{enumerate}


\end{description}


\subsection{Распределение Бернули:}

\begin{description}
\item[Распределение Бернули:] 

$X: \Omega \to \mathbb R$

$p[x = 1] = p$


$p[x = 0] = 1 - p$

\item[Биномиальное распределение:]

$x_1, x_2, \ldots, x_n$ ~--- независимые

$p[x_i = 1] = p, p[x_i = 0] = 1 - p$

$x = x_1 + x_2 + \ldots + x_n$

$p[x = k] = C_n^k p^k (1 - p)^{n - k}$

$E[x_i] = p$

$E[x] = \sum_{1}^n E[x_i] = pn$
\end{description}

\subsection{Закон больших чисел для распределения Бернули}

\begin{description}
\item[Теорема(Закон больших чисел для распределения Бернули)]

$\forall p, \epsilon \exists c < 1$

$x_1, \ldots, x_n$ ~--- независимые

$E[x_i] = p$

$p[|\cfrac{x_1 + \ldots + x_n}{n} - p| \ge \epsilon] \le 2c^n$

\item[Доказательство:]

$A = X_1 \circ X_2 \circ \ldots \circ X_n$

$A: \Omega \to \{0, 1\}^n$

$Y_1, Y_2, \ldots, Y_n$ ~---независимые.

$P[Y_i = 1] = p + \epsilon$

$P[Y_i = 0] = 1 - p - \epsilon$

B = $Y_1 \circ Y_2 \circ \ldots \circ Y_n$

$S \in \{0, 1\}^n$

$\omega(s)$ ~--- число единиц в S.

$P[A = S] =p^{\omega (s)} (1 - p)^{n - \omega (s)}$

$P[B = S] =(p+\epsilon)^{\omega (s)} (1 - p - \epsilon)^{n - \omega (s)}$ 

Пусть $\omega(s) \ge (p + \epsilon) n$

$P[A = S] \le P[B = S]$

$P[A = S] = P[B = S](\cfrac{p}{p + \epsilon})^{\omega(s)}(\cfrac{1 - p}{1 - p - \epsilon})^{n - \omega(s)} \le P[B = S](\cfrac{p}{p + \epsilon})^{(p + \epsilon)n}(\cfrac{1 - p}{1 - p - \epsilon})^{{(1 - p - \epsilon)}n} = P[B = S]c^n$, где $с = (\cfrac{p}{p + \epsilon})^{p + \epsilon}(\cfrac{1 - p}{1 - p - \epsilon})^{1 - p - \epsilon}$

Позже покажем, что с < 1

$P[\cfrac{\sum x_i}{n} \ge p + \epsilon] = \sum_{s, w(s) \ge (p + \epsilon)n} P[A = S] \le \sum _{s, w(s) \ge (p + \epsilon) n} P[B = S] c^n \le c^n$

Аналогично $p[\cfrac{\sum{x_i}}{n} \le p - \epsilon] \le c'^n$

$P[|\cfrac{\sum x_i}{n} - p| \ge \epsilon] \le c'^n + c^n \le 2 max(c', c)^n$


$\ln(x) \le x - 1$

$\ln(c) = (p + \epsilon) \ln{\cfrac{p}{p + \epsilon}} + (1 - p - \epsilon)\ln(\cfrac{1 - p}{1 - p - \epsilon}) \le (p + \epsilon)(\cfrac{p}{p + \epsilon} - 1) + (1 - p - \epsilon)(\cfrac{1 - p}{1 - p - \epsilon} - 1) = p - (p + \epsilon) + ((1 - p) - (1 - p - \epsilon)) = 0$
\end{description}

\section{Дисперсия}
\subsection{Математическое ожидание произведения независимых случайных величин}

\begin{description}
\item[Лемма] $X, Y$ ~--- независимы, тогда $E[xy] = E[x]E[y]$

\item[Доказательство]

$E[xy] = \sum_{a \in \mathbb R \text{конечная сумма}} = \sum_{a_1(\text{значение x}, a_2(\text{значение y}))} a_1 a_2 P[x = a_1, y = a_2] = \sum a_1 a_2 P[x = a_1]P[y = a_2] = (\sum a_1 P[x = a_1])(\sum a_2 P[y = a_2]) = E[x] E[y]$
\end{description}

\subsection{Дисперсия}

\begin{description}

\item[Определение:] $D[x] = E[(x - E[x])^2]$ ~--- дисперсия случайной величины X.

$D[x] = E[x^2 + (E[x])^2 - 2xE[x]] = E[x^2] + (E[x])^2 - 2E[x]E[x] = E[x^2] - (E[x])^2 \ge 0$

\item[Примеры:] 

$p[x = 1] = p$

$p[x = 0] = 1 - p$

$D[x] = E[x^2] - (E[x])^2 = p - p^2 = p(1 - p)$

\item[Свойства дисперсии:]
\begin{enumerate}
\item $D[\alpha x] = \alpha^2D[x]$

$E[(\alpha x)^2] = (E[\alpha x])^2 = \alpha^2(E[x^2] - (E[x])^2)$

\item $x_1, x_2, \ldots, x_n$ ~--- случайные величины.

$\forall i \ne j, x_i$ и $x_j$ ~--- независимы(попарно независимы)

Тогда $D[x_1 + \ldots + x_n] = \sum_{i = 1}^{n} D[x_i]$
\begin{description}
\item [Доказательство]

$D[x_1 + \ldots + x_n] = E[(x_1 + \ldots + x_n)^2] - (E[x_1 + \ldots + x_n])^2 = \sum E[x_i^2] + \sum_{i \ne j}E[x_i x_j] - \sum(E[x_i])^2 - \sum_{i \ne j}E[x_i]E[x_j] = \sum_{i = 1}^{n}E[x_i^2] - (E[x_i])^2 = \sum_{i = 1}^nD[x_i]$
\end{description}
\end{enumerate}

\end{description}

\section{Неравенство Чебышева}

\subsection{Неравенство Чебышева}

\begin{description}
\item[Теорема. Неравенство Чебышева]

$p[|x - E[x]| \ge \epsilon] \le \cfrac{D[x]}{\epsilon^2}$

\item[Доказательство:]

$P[|x - E[x]| \ge \epsilon] = P[(x - E[x])^2 \ge \epsilon^2] = P[(x - E[x])^2 \ge \cfrac{\epsilon^2 D[x]}{D[x]}] \le \cfrac{D[x]}{\epsilon^2}$
\end{description}

\subsection{Закон больших чисел для попарно независимых случайных величин}

\begin{description}

\item [Теорема.]
$x_1, x_2, \ldots, x_n$ ~--- попарно независимые.

$p[x_i = 1] = p, p[x_i = 0] = 1 - p$

Тогда $P[|\cfrac{\sum x_i}{n} - p| \ge \epsilon] \le \cfrac{C}{\epsilon^2 n}$

\item[Доказательство:]

$y = \cfrac{\sum x_i}{n}$

$E[y] = p$

$D[y] = \cfrac{D[\sum x_i]}{n^2} = \cfrac{n p(1 - p)}{n^2} = \cfrac{p (1 - p)}{n}$

$P[|y - p| \ge \epsilon] \le \cfrac{p(1 - p)}{\epsilon^2 n} \le \cfrac{1}{4 \epsilon^2 n}$

$\sqrt{p(1 - p)} \le \cfrac{p + (1 - p)}{2}$

\end{description}

\section{Условная вероятность}

\subsection{Условная вероятность}

\begin{description}
\item[Условная вероятность:]

$\Omega, p A \subset \Omega P(A) > 0$

$B \subset \Omega$

$P(B|A) = \cfrac{P(B \cap A)}{P(A)}$

\begin{enumerate}
\item A, B ~--- независимые. $P(B | A) = P(B)$
\item (Формула Байеса) $P(B \cap A) = P(B|A)P(A)$
\item (Формула полной вероятности) $A_1 \cup A_2 \ldots A_n = \Omega$

$A_i \cap A_j = 0 \forall i, j$

Тогда $P(B) = \sum_{i = 1}^{n}P(B|A_i) P(A_i)$

\end{enumerate}

\end{description}

\section{Лемма Фаркаша}

Имеет ли система решение?

$\begin{cases} a_{11}x_1 + a_{12} + \ldots + a_{1n}x_n \le b_1\\ $\ldots$ \\ a_{m1}x_1 + a_{m2} + \ldots + a_{mn}x_n \le b_n \\ \end{cases}$

Можно сложить с коэффициентами и получить противоречие.

\begin{description}

\item[Лемма Фаркаша] 

Если система линейных неравенств несовместима $\Leftrightarrow \exists \alpha_1, \alpha_2, \ldots, \alpha_m \ge 0:$

$\alpha_1(a_{11}x_1 + a_{12}x_2 + \ldots) + \alpha_2() + \ldots = 0$

$\alpha_1 b_1 + \ldots + \alpha_m b_m = -1$

\item[Доказательство:]

$\Leftarrow$ очевидно. 

$\Rightarrow$ 

Индукция по n.

{\bf База} n = 1

$x_1 \le c_1$

$\ldots$

$x_1 \le c_k$

$x_1 \ge d_1$

$\ldots$

$x_1 \ge d_l$

$\exists i, j: c_i < d_j$

$0 \le c_i - d_j$

{\bf Переход}

неравенство без $x_1$ (**)

$x1 \le f_1(x_2 \ldots x_n)$

$x1 \le f_2(x_2 \ldots x_n)$

$\ldots$

$x1 \le f_k(x_2 \ldots x_n)$


$x1 \ge g_1(x_2 \ldots x_n)$

$x1 \ge g_2(x_2 \ldots x_n)$

$\ldots$

$x1 \ge g_t(x_2 \ldots x_n)$

$\begin{cases} (**) \\ g_i(x_2, \ldots, x_n) \le f_j(x_2, \ldots, x_n)\\ i \in [l], j \in [k] \\ \end{cases}$

Эта система  не содержит $x_i$

Новые неравенства ~--- это линейная комбинация старых с неотрицательными коэффициентами. 
$x_1 \le f_j(x_2, \ldots, x_n)$

$-x_1 \le -g_i(x_2, \ldots, x_n)$ 
\end{description}

\section{Задача линейного программирования. Двойственная задача.}

\subsection{Задача линейного программирования.}

$c_1 x_1 + c_2 x_2 + \ldots + c_n x_n \to \max$

$\begin{cases} a_{11}x_1 + \ldots + a_{1n}x_n \le b_1\\ \ldots \\  a_{m1}x_1 + \ldots + a_{mn}x_n \le b_n\\ \end{cases}$

Множество решений системы (*) ~--- множество допустимых решений.

$c_1 x_1 + \ldots + c_n x_n$ ~--- целевая функция.

{\bf Факт} Если целевая функция ограничена на множестве допустимых решения, то задача линейного программирования имеет оптимальное решение. 

\subsection{Двойственная задача}

$y_1 b_1 + y_2 b_2 + \ldots + y_m b_m \to \min$

$\begin{cases} a_{11}y_1 + \ldots + a_{m1}y_m = c_1\\ \ldots \\  a_{1n}y_1 + \ldots + a_{mn}y_m = c_m\\ y_1, \ldots, y_m \ge 0 \\ \end{cases}$

\begin{description}
\item[Теорема]
\begin{enumerate}
\item $x_1, \ldots, x_n$ ~--- допустимое решение.

$y_1, \ldots, y_m$ ~--- допустимое решение.

то $c_1 x_1 + c_2 x_2 + \ldots + c_n x_n \le y_1 b_1 + \ldots + y_m b_m$

\item Если множество допустимых решений (1) и (2) не пусто, то $\exists$ оптимальное решение $x_1^*, \ldots, x_n^*(1)$ 

$y_1^*, \ldots, y_n^* (2)$

$c_1 x_1^* + \ldots + c_n x_n^* = b_1 y_1^* + \ldots + b_m y_m^*$

\item Если $x_1^*, \ldots, x_n^*, y_1^*, \ldots, y_m^*$  оптимальное решение и $a_{i1}x_1^* + \ldots + a_{in}x_n^* < b_i \Rightarrow y_i^* = 0$
\end{enumerate}

\item[Доказательство:]

\begin{enumerate}
\item $y_1b_1 + y_2b_2 + \ldots + y_m b_m \ge \sum_{i = 1}^{m}y_i\sum_{j - 1}^{n}a_{ij}x_j = \sum_{j = 1}^{n}x_j \sum_{i = 1}^{m}y_i a_{ij} = \sum_{j = 1}^{n}x_jc_j$

\item $c_1 x_1 + \ldots + c_n x_n \le y_1 b_1 + \ldots + y_m b_m$

$c_1 x_1 + \ldots + c_n x_n \le inf_{y \text{~--- допустимые решения (2)}}(y_1 b_1 + \ldots + y_m b_m) = d_2$

Докажем, что $\exists x_1^*, \ldots, x_n^*$ ~--- допустимые решение(1)

$c_1 x_1^* + c_2 x_2^* + \ldots + c_n x_n^* = d_2$

От противного, пусть такого решения нет. 

$\begin{cases} (*)\\ -c_1x_1 - c_2x_2 - \ldots - c_nx_n \le -d_2 \\ \end{cases}$

несовместимая система.

$\exists y_1, \ldots, y_m:$

$\sum y_i a_i1 = c_1$

$\ldots$

$\sum y_i a_in = c_n$

$(\sum_{i = 1}^{m} y_i b_i) - d_2 < 0$

$y_i ~---$ допустимое решение (2)

$y_1b_1 + \ldots + y_mb_m < d_2 = inf$ противоречие. 

Аналогично $\exists y_1^*, \ldots,y_m^* $

$\sum b_iy_i^* = d1 = sup()$
\item Если стоит строгое равенство в решение и существует $\sum_{j}a_{ij}x_j < b_j $, то $y_j  = 0$
\end{enumerate}

\end{description}

\section{Поток в графе}

\subsection{Поток}

$G$ ~--- ориентированный граф $G(V, E)$

$s, t \in V$

$s$ ~--- исток.

$t$ ~--- сток. 

$c: v \times v \to \mathbb R_{+}$ ~--- пропускная способность

Если $(u, v) \notin E$, то С(u, v) = 0

Пусть $\EuScript P$ ~--- множество простых путей из s в t.

Поток $\{f_p\}_{p \in \EuScript P}$

$f_p \ge 0$

$\forall e \in E \sum_{p \in \EuScript P}f_p \le c(e)$

Размер потока $\sum_{p \in \EuScript P}f_p$

$\sum_{p \in \EuScript P} f_p \to max$

$\begin{cases} f_p \ge 0\\ \sum_e f_p \le c(e)\\ \end{cases}$

\subsection{Двойственная задача}

$\sum_e l_e c(e) \to min$

$\begin{cases} \sum_{l \in p}l_e - \gamma_p = 1\\ l_e \ge 0 \\ \gamma_p \ge 0\end{cases}$

\subsection{разрез}

$S, T \subset V S\cap T = 0, s \in S, t \in T, S \cup T = V$

$(S, T)$ ~--- разрез

Пусть $(S, T)$ разрез определим $l_e = \begin{cases} 1, &\text{если e ведет из S в T}\\ 0, &\text{иначе}\\ \end{cases}$

$\sum f_p \le \sum l_e c(e)$ ~--- суммарный поток меньше любого разреза.

\subsection{Теорема Форда-Фолкерсона}

\begin{description}
\item[Теорема Форда-Фолкерсона] Размер максимального потока равен минимальной пропускной стоимости разреза.

\item[Доказательство] Пусть $f_p^*$ ~--- оптимальное решение прямой задачи, $l_e^*, \gamma_p^*$ ~--- оптимальное решение двойственной задачи. 

\begin{enumerate}

\item $\sum_{p \in P}f^*_p = \sum_e l_e^* c(e)$
\item $f_p^* >0$, то $\gamma_p^* = 0 \sum_e l_e^* = 1$

\item $\sum f_p < c(e)$, то $l^*_e = 0$
\end{enumerate}

Строим разрез по $l^*_e$

S ~--- множество всех вершин, в которое можно дойти из S по ребрам $e: l^*_e = 0$

T ~--- все остальные. $T = V - S$

Пусть e ведет из S в T.

$c(e) = \sum f^*_p$, так как $l^*_e > 0$

$\sum_{l \text{из S в T}}c(e) \le \sum_{p} f^*_p$, если было бы верно, что $\forall p: f^*_p > 0$ пересекает S, T ровно 1 раз.

Пусть $f_p^* > 0$ и p пересекает S, T несколько раз $f_p^* > 0$, то $\sum_{l \in p}l_e^* = 1$

\end{description}

\section{Целочисленный поток}
\begin{description}
\item[Теорема:]

Если $\forall e c(e) \in \mathbb Z_+$

Тогда $\exists$ целочисленный максимальный поток.

\item[Доказательство:]

Размножим ребра, теперь у всех ребер пропускная способность = 1.

Найдем путь из S в T пустим поток и удалим ребро. 

Проблема, решение может быть не оптимально. Можем пройти через разрез несколько раз. 

Правильный алгоритм:

\begin{enumerate}
\item Ищем путь из S в T
\item Пускаем по нему поток.
\item удаляем ребро.
\item добавляем обратное ребро.
\end{enumerate}

Нет пути $\Rightarrow$ поток совпадает с разрезом $\Rightarrow$ он максимальный. 
\end{description}

\section{Паросочетания}
\subsection{Паросочетания}
\begin{description}
\item [Определение:] Неориентированный граф.

$G(V, E)$

$E' \subset E$ E' ~--- паросочетание, если ребра из E' не имеет общих концов.  

\end{description}

\subsection{Теорема Кенинга}

G ~--- двудольный граф.

Из каждый вершины исходит хотя бы одно ребро.

\begin{description}
\item[Теорема Кенинга] Размер максимального паросочетания в G = размер минимального покрывающего множества. 

$S \subset V$ покрывающие множество $\forall e \in E$ имеет конец в S.

\item[Доказательство:] Размер любого паросочетания $\le$ размера любого покрывающего множества(любое ребро паросочетания нужно покрыть)

\begin{center} 
\includegraphics[width=3in, keepaspectratio]{im5.jpg} 
\end{center}



Добавим вершины S и T, ребра из S в первую долю с пропускными способностями 1, из ребер второй доли в T с пропускными способностями 1 и между долями пропускные способности +inf.

Рассмотрим минимальный разрез. Пусть он равен $T_1$ + $S_2$, тогда из $S_1$ в $T_2$ нет ребер. 

Паросочетание ~--- ребра, по которым течет поток. 

 $T_1$ и $S_2$ ~--- покрытие, поскольку нет ребер из $S_1$ в $T_2$;


Размер паросочетания = размеру потока = $|T_1|+| S_2|$ = размеру покрытия.

\end{description}

\subsection{Теорема Холла}

\begin{description}
\item [Теорема Холла] G ~--- двудольный граф, тогда в G есть паросочетание размера |M| $\Leftrightarrow \forall S \subset M |\Gamma (S)| \ge |S|$

$\Gamma(S) = \{v \subset N: \exists u \in S (u, v) \in E\}$

\item[Доказательство:] 

$\Rightarrow$ очевидно.

$\Leftarrow$ От противного. Пусть есть покрывающее множество размера < |M|

$\forall v \in M/S_1$ соединяется только с вершинами из $S_2$ 

$\Gamma (M/S_1) \subset S_2$

$\Leftarrow |S_2| \ge |M/S_1|$

$\Leftarrow |S_1 + S_2| \ge |S_1| + |M/S_1| = |M|$


\end{description}


\section{Частично упорядоченные множества}

\subsection{Частично упорядоченные множества}
\begin{description}
\item[Определение:]  M ~--- множество. 

$\le$ бинарное отношение.

$\le$ частичный порядок, если

\begin{enumerate}
\item транзитивно:

$a \le b, b \le c \Rightarrow a \le c$

\item антисимметрично

$a \le b, b \le a \Rightarrow a = b$

\item рефлексивность

$a \le a$
\end{enumerate}
\end{description}

\subsection{Цепь}

$(M, \le)$

\begin{description}
\item[Определение:] Цепь ~--- такое $S \subset M$

$\forall x, y \in S (x \le y $\text{ или }$ y \le x)$
\end{description}

$(M, \le)$

Покрытие цепями $M = S_1 \cup S_2 \cup \ldots \cup  S_k$

$S_i \cap S_j = 0 \forall S_i$ ~--- цепь.

k ~--- размер покрытия.

\subsection{Антицепь} 

\begin{description}
\item[Определение:] Антицепь ~---  $S \subset M, S$ ~--- антицепь , если $\forall x, y \in S$

$x \ne y \Rightarrow \begin{cases} \neg (x \le y)\\ \neg (y \le x)\\ \end{cases}$

\end{description}

\subsection{Теорема Дилвортса}
\begin{description}
\item [Теорема Дилвортса:] 

$(M, \le)$

Размер максимальной антицепи равен минимальному размеру покрытия M цепями. 

\item[Доказательство:]

Размер $\forall$ покрытия цепями $\ge$ размер $\forall$ антицепями. (Так как цепь не может содержать 2 элемента антицепи)

Создадим копию каждого элемента множества M, M'.

Проведем ребро между $x_i$ и $x_j'$, если $x_i < x_j$

Рассмотрим максимальное паросочетание в двудольном графе. Пусть его размер K. 

Нарисуем ребра, которые вошли в паросочетание в исходном графе. 

Получилось: из каждой вершины выходит $\le 1$ ребра и $\le 1$ ребра входит. 

По транзитивности циклов нет. 

Получили покрытие n - k цепями. (при добавление ребра каждые две цепи объединяются)

В двудольном графе есть минимальное покрывающее множества размера k(по теореме Кенга)

$S \subset M: \forall x \in S x, x'$ не входит в покрывающее множество. 

$\forall x, y \in S$ нет ребра между (x, y') и (y, x') $\Rightarrow S$ антицепь размера $S \ge n - k$ 

\end{description}

\section{Теорема Менгера} 
\begin{description}
\item[Реберная теорема Менгера] $G(V, E)$ ~--- простой неориентированный граф.

$s, t \in V$

$\mu$ ~--- максимальное количество не пересекающихся по ребрам  путей из $s$ в $t$.

$\nu$ ~--- минимальное количество ребер, которое нужно удалить, чтобы s и t оказались в разных компонентах связности.

$\mu = \nu$

\item[Доказательство:] 

 $\mu < \nu$, так как из каждого пути нужно удалить хотя бы одно ребро. 

Из s все ребра исходящие. 

В t все ребра входящие. Все остальные ребра ориентированные и туда и туда. 

Все пропускные способности равны 1.

\begin{description}
\item[Утверждение:] Пропускная способность минимального размера = $\nu$ 
\end{description}


Разрез достаточно удалить, что бы вершины оказались в разных компонентах. 

По теорема Форда-Фолкинсона $\exists$ поток, размер которого равен $\nu$

Поскольку все пропускные способности целочисленные, то поток состоит из путей по которым течет 1. 
\end{description}

\section{Код Хемминга}
\subsection{Игра с угадыванием числа}

\begin{description}
\item[Задача:] $k = \lceil \log_2 n \rceil$ ~--- необходимое количество вопросов, что бы угадать число.

\item[Алгоритм:] бинарный поиск.

\item[Доказательство оптимальности ответа:]  Построим дерево ответов. Листья ~--- возможные ответы. 

m ~--- вопросов. 

$2^m$ ~--- листьев. 

$2^m \ge log_2 n$

$m \ge \lceil log_2 n \rceil$

\end{description}

\subsection{Игра с одной ошибкой}

\begin{description}
\item [n] один ответ может быть не верный. 
\item m вопросов. Каждое число должно быть записано m + 1 раз. 

$2^m \ge (m + 1)n$

$m \ge log_2(m + 1) + log_2(n)$
\end{description}

\subsection{Код Хемминга}
\begin{description}
\item[Определение:] Расстояние по Хеммингу

 $x, y \in \{0, 1\}^n$

$\delta (x, y) = количество \{i| x_i \ne y_i\}$

\item[Определение:] $C: \{0, 1\}^k \to \{0, 1\}^m$ называется кодом исправляющем ошибку с расстоянием d, если 

$\forall x, y \in \{0, 1\}^k$

$\delta (C(x), C(y)) \ge d$

$x \ne y$

Если $2r + 1 \le d$, то говорят, что код исправляет r ошибок. 

\item[Утверждение:] Пусть код $C:\{0, 1\}^k \to \{0, 1\}^m$ исправляет 1 ошибку(расстояние 3).

Тогда в предыдущей игре можно угадать число за m вопросов $(n \le 2^k)$

$[n] \to  \{0, 1\}^k$

находим кодовое слово C(y) которое max на расстояние 1 от $z_1 z_2 \ldots z_m$. Ответ y.

\item[Теорема] $C:\{0, 1\}^k \to \{0, 1\}^m$ ~--- код, исправляющий 1 ошибку(код на расстояние 3), то $2^m \ge (m + 1)2^k$

\item[Код Хемминга]

Пусть $2^m \ge (m + 1)2^k$. Тогда существует код $C:\{0, 1\}^k \to \{0, 1\}^m$, исправляющий 1 ошибку. 

$2m \ge 2^{m - k} \ge m + 1$

$2^{m - k} - 1 \ge m$

Выпишем все не нулевые числа.

\begin{tabular}{|c|c|c|c|c|c|c|c|}
\hline
&&&&m&&&\\
\hline
&1&0&1&0&1&1&\\
m-k&0&1&1&0&0&1&\\
&0&0&0&1&1&1&\\
&0&0&0&0&0&0&\\
\hline
\end{tabular}

$y \in \{0, 1\}^m$ ~--- кодовое слово, если скалярное произведение по mod 2 на каждую строку таблицы равно 0.

Столбцы, в которых одна 1~--- дополнительные, остальные информационные. 

\section{Теорема Рамсея}

$R(M, N)$ ~--- это минимальное такое число K, что $\forall$ полного графа на k вершинах, ребра которого раскрашены в два цветы, найдутся либо m вершин, все ребра покрашены в 1 цвет, либо n вершин, все ребра между ними покрашены во 2 цвет. 

$R(n, n) \le C_{2n - 2}^{n - 1} \le 2^{2n - 2}$
 
\subsection{Верхняя оценка}
$R(2, n) = n$

$m, n \ge 3$

R(m, n) = R(n, m)

$R(m, n) \le R(m - 1, n) + R(m, n - 1)$

Пусть в графе k = $R(m - 1, n) + R(m, n - 1)$ вершин.

Выбираем 1 вершину у нее $R(m - 1, n) + R(m, n - 1)$ - 1 сосед. $\Rightarrow$ 
\begin{enumerate}
\item У этой вершины $\ge R(m - 1, n)$ соседей первого цвета.
\item $\ge R(m, n - 1)$ соседей 2 цвета. 
\end{enumerate}

$\Rightarrow$ R(m, n) ~--- конечно(по индукции)

$R(m, n) \le C_{m + n - 2}^{m - 1}$

\item[Доказательство:] индукция по m+n 

{\bf База} n = 2 R(m, 2) = m

{\bf Переход} $R(m, n) \le R(m - 1, n) + R(m, n - 1) \le C_{m + n - 3}^{m - 2} + C_{m + n - 3}^{m - 1} = C_{m + n - 2}^{m - 1}$
\end{description}

\section{Обобщение чисел Рамсея}

R(s, m, n) ~--- это наименьшее такое число k, что при любой раскраске всех s ~--- элементных подмножеств [k] в 2 цвета найдется, либо m элементное множество, все s-элементные подмножества покрашены в 1 цвет, либо n элементное подмножество, все s-элементы которого покрашены во второй цвет. 

$R(1, m, n) = m + n - 1$

$R(2, m, n) = R(m, n)$ 

$R(2, m, n) \le R(2, m -1, n) + R(2, m, n - 1) - 1 + 1$

$R(s, m, n) \le R(s - 1, R(s, m - 1, n), R(s, m, n - 1)) + 1$

Пусть k  = R(s - 1; R(s, m - 1, n), R(s, m, n - 1)) + 1

Выберем $u \in [k]$

Все (s - 1) ~--- элементные подмножества $[k]/\{u\}$

$A \subset [k]/\{u\} |A| = s - 1$ красим A в цвет $A \cup \{u\}$

Среди $[k] /\{u\}$ есть либо 
\begin{enumerate}
\item R(S; m - 1, n) элементов: $\forall (s - 1)$ подмножество покрашенное в цвет 1
\item $R(S; m, n - 1)$ элементов: $\forall (s - 1)$ элементов подмножества покрашенные в цвет 2.

\subsection{Для раскраски во много цветов}

$R_r(S, n_1, n_2, \ldots, n_r)$

$R_r(S, n_1, \ldots, n_r) \le R_r(s - 1, R_r(s, n_1 - 1, \ldots), R_r(s, n_1, n_2 - 1, \ldots), \ldots, R_r(s, n_1, \ldots, n_r - 1)) + 1$
\end{enumerate}

\section{Нижняя оценка на R(k, k). Бесконечный вариант теоремы Рамсея}

\subsection{Нижняя оценка на R(k, k)}

\begin{description}
\item[Предложение:] $R(n, n) \ge 2^{\frac{n}{2}}$ при $n \ge 3$
\item[Доказательство:] Возьмем полный граф на k вершинах и покрасим его ребра в 2 цвета случайным образом. 

$\rho$ ~--- подмножество из n вершин.

p[все ребра множества S покрашены в 1 цвет] = $22^{-C_n^2}$

$p[\exists$ n вершин множества, что все ребра покрашены в 1 цвет$] \le \sum_{\rho \text{~--- n вершин множества}}$P[все ребра в S покрашены в 1 цвет] $\le C_k^n 2^{1 - C_n^2} < 1(хотим)$

Пусть k = $\lfloor 2^{\frac{n}{2}} \rfloor$

$C_n^k2^{1 - C_n^2} < \frac{k^n}{n!}\frac{2^{\frac{n}{2} + 1}}{2^{\frac{n^2}{2}}} \le \frac{2^{\frac{n}{2} + 1}}{n!} < 1 n \ge 3$

$1 - C_n^2 = 1 - \frac{n(n - 1)}{2}$

\end{description}


\subsection{Бесконечный вариант теоремы Рамсея}

\begin{description}
\item[Теорема:] Если ребра бесконечного графа покрашены в r цветов, то $\exists$ бесконечное число вершин, что все ребра между ними покрашены в 1 цвет. 
\item[Доказательство:] Достаточно доказать для двух цветов(в противном случае склеиваем цвета)

Рассмотрим два случая.

\begin{enumerate}
\item Рассмотрим только вершины,у которых соседей первого цвета конечно. Если таких бесконечное количество. 

Выбираем вершину, выкидываем всех соседей первого цвета. Оставшихся вершин бесконечно. Повторим операцию.

\item 
Если таких вершин конечно. 

Выбираем вершину, у которой бесконечное количество соседей первого цвета и оставляем только ее соседей. Если можем повторять операцию бесконечное количество раз, все ок. Иначе в какой-то момент получим первый случай. 

\end{enumerate}

\end{description}

\section{Примеры использования теоремы Рамсея}

\subsection{Теорема Эрдеша-Секереша}
\begin{description}
\item [Теорема:] $\exists k$, что из k точек на плоскости, ни какие 3 не лежат на одной прямой, можно найти выпуклый n-угольник.
\item[Доказательство:]
$k = R_2(3, n, n)$

$\chi(A, B, C) = \begin{cases} 1, & \text{если в треугольнике ABC лежит нечетное число точек}\\ 0, & \text{иначе}\\\end{cases}$

$\chi(A, B, C) = \chi(B, D, C) + \chi(A, D, C) + \chi(A, B, D) + 1$

Значит можем выбрать n точек, что в любом треугольник не лежат точки из подмножества. 

\end{description}

\subsection{Раскраска натуральных чисел}
\begin{description}
\item[Раскраска натуральных чисел:] $\forall r \exists k: \forall $ раскраски [k] в r цветов $\exists a, b, c:$ покрашенные в один цвет и a + b = c
\item[Доказательство:]

$C:[k] \to [r]$ ~--- раскраска.

$\chi(a, b) = C(|a - b|)$

$K = R_r(2; 3, 3, 3, \ldots, 3)$


$a \le b \le d c(b - a) = c(d - b) = c(d - a)$

d - a = d - b + b - a

\end{description}
\noindent \underline{\hbox to 1\textwidth{{ } \hfil{ } \hfil{ } }}

\begin{center}
  \t{\Large{\bf КОНЕЦ}}
\end{center}

\end{document}

\section{} % 20
По четыре свойства для сложения и умножения (ассоц., $\exists 1$, $\exists -a$ с двух сторон, $a+b=b+a$), дистрибутивность с двух сторон.
Кольцо "--- по сложению Абелева группа, умножение ассоц. Бывает <<с 1>>, бывает <<коммутативное>>, в поле всё ок, тело "--- поле без $ab=ba$.
0, -a, 1 единственны, если есть левый и правый обратные по умножению, то совпадают. Если есть $\ge 2$ левых, то их бесконечно (без доказательства).
$0\cdot a=a\cdot0=0$, $a(-b)=(-a)b=-(ab)$, если $0=1 \Ra $ скучно. Гомоморфизм/изоморфизм колец. Если $c=ab$, то $a$ "--- левый делитель;
нетривиальные делители нуля; область целостности (и $ab=0 \Ra a = 0 \lor b = 0$); поле "--- область целостности; в таких можно сокращать $ab=ac$.

\section{} % 21
$A^*$ "--- обратимые элементы ($ab=ba=1$), образуют мультипликативную группу, надо показать замкнутость и единицу.

\section{} % 22
Многочлен "--- последовательность, почти все нули; сложение и умножение (проверить, что почти все нули); получили кольцо.

\section{} % 23
Определили <<$a$>>, <<$x$>>, <<$x^k$>>, записали в нормальном виде; макс. ненулевой "--- степень (или $-\infty$). Теорема:
степень суммы $\le \max$, степень произведения $\le$ суммы (в области целостности "--- равно). Следствие: если $A$ "--- обл.цел.,
то $A[x]$ "--- обл.цел. Если $A$ "--- обл.цел., то $(A[x])^*=A^*$ (два включения).

\section{} % 24
Старший коэф. делителя из $A^*$, тогда можно на него делить с остатком (у остатка степень маленькая). Индукция по степени делимого. 
Если $A$ "--- обл. целостности, то единственно: от противного, попереносили в стороны, степень многочленово с разных сторон равенства разная.

\section{} % 25
Можно считать $f(c)$, если $c$ из $B$ "--- комм. надкольцо $A$ с $1$ (например, в $\Z[x]$ можно подставить из $\R$). Многочлен задаёт
отображение, можно раскрывать скобки в $(f+g)(c)$ и $(f\cdot g)(c)$. Безу: $f(x)=(x-c)q(x) + f(c)$ (поделили с остатком).
Ввели корень, следствие: корень $\iff (x - c) \mid f$.

\section{} % 26
Характеристика кольца $n$ "--- $\underbrace{1 + \dots + 1}_n = 0$ ($\FChar A = n$), если такого нет, то $\FChar A = 0$.
В поле $\FChar A = 0$ или $\FChar A = p$ (от противного по дистрибутивности и области целостности).

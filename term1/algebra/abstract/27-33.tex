\section{} % 27
Определили в кольце $n \cdot a$ ($n \in \Z$), конструктивно производную (+фикт. слаг.). $(f+g)'$, $(cf)'$, $(fg)'$ (сначала для $x^a$ и $y^b$, потом одно многочлен, потом два многочлена), $(fgh\dots)')$, $(f^k)'$.
Для поля: если $f'=0$, то в зависимости от $\FChar$ либо константа, либо степени $x$ делятся на $p$.

\section{} % 28
Кратный корень: $(x-c)^k\mid f$, $(x-c)^{k+1}\nmid f$. Для поля с $\FChar A=0$ и $f\neq 0$: кратный корень $\iff$ корень $f$ и корень $f'$ кратности $k-1$
(туда "--- расписали; обратно "--- есть кратность, применили <<туда>>, значит, сошлось). Плохой пример: в $\mathbb{F}_2$ смотрим $x^7+x^2$ и 0.
Следствие: корень кратности $\ge k$, если обнуляет $f, f', \dots, f^{k-1}$.

\section{} % 29
$A$ "--- обл.цел. Кратность корня у произведения многочленов через две кратности. Если $f=(x-c)^kg$ и $g(c)\neq 0$, то при $d\neq c$: $(x-d)^a\mid f \iff (x-d)^a\mid g$ ($\Ra$ индукция по $a$ от 1).
Кол-во корней с кратностью не выше степени (при $f\neq 0$): индукция по $\deg f$. Вне обл.цел, например в $\Z/\Z2$: $2x$. Следствие: если в обл. цел. два многочлена
совпадают во всех точках и $|A|>\max\{\deg f, \deg g\}$, то они равны. Следствие при $|A|=\infty$.

\section{} % 30
Любая не-константа: имеет корень $\iff$ делится на линейный многочлен $\iff$ имеет $\deg f$ корней $\iff$ полностью раскладывается на линейные. $1 \iff 2$ "--- Безу,
$1 \Ra 3$ и $2 \Ra 4$ "--- индукция по $\deg f$.

\section{} % 31
Найдём интерполяционный многочлен степени меньше $n$ (тогда не более 1 решения). Добавляем по точке: если $f$ решение, то $f+(x-x_1)\dots(x-x_n)\cdot h$ "--- тоже.

\section{} % 32
Сначала решим подзадачу, когда во всех точках ноль, в одной "--- единица. $L_i$ "--- такое частное решение, скомбинируем.
Красивый ответ: $g(x) = \prod (x-x_i)$, тогда $L_k = \frac{g(x)}{x-x_k} \div g'(x_k)$.

\section{} % 33
Кольцо многочленов от $n$ переменных: $((A[x_1])[x_2])=((A[x_2])[x_1])=A[x_1,x_2]$. Ввели $\binom{n}{k}$ (через формулу, показали
треугольник Паскаля и симметричность по $k$). Расписали бином Ньютона по индукции. Следствия: $\sum \binom{n}{k}$, $\sum \binom{n}{2k}$,
$\sum \binom{n}{2k+1}$.

\section{} % 01
Область определения ($\Dom$) и назначения ($\coDom$), $(g \circ f)(x)=g(f(x))$, проверить области в теореме.

\section{} % 02
$f$ обратимо справа, если $f \circ g = id_B$, бывает слева, бывает обратимо. Если обратимо слева и справа, то обратимо и $\exists! f^{-1}$.
Свойства: обратимость композиции трёх видов. Композиция сюръекций, инъекций, биекций. Теорема: $(f^{-1})^{-1} = f$.

\section{} % 03
Композиция с тождественным. Теорема: если $g \circ f = id_A$, то $f$ "--- инъекция, $g$ "--- сюръекция.

\section{} % 04
$\Dom \neq \varnothing$. В одну сторону "--- вопрос 3, в другую "--- построим $h_1 : f(A) \to A$ и расширим на $B$, получим $h \circ f = id_A$.
Следствие из вопросов 4-5: обратимость $\iff$ биективность.

\section{} % 05
Аксиома выбора: существует отображение, выбирающее по элементу как-нибудь. Теорема в одну сторону очевидна, в другую: выбираем
по аксиоме для каждого элемента элемент из его прообраза, показываем, что всё хорошо с $\Dom$ и $\coDom$.
Следствие из вопросов 4-5: обратимость $\iff$ биективность.

\section{} % 06
От противного ($f$ "--- не сюръекция): строим последовательность для плохого $a$ ($a$, $f(a)$, $f(f(a))$, \dots), зациклились, показали отсутствие предпериода (по индукции по меньшему номеру),
нашли для $a$ прообраз.

\section{} % 07
$\forall a, \exists n_a: (f\circ f\circ \dots \circ f)(a) = a$ (строим последовательность $a$, $f^{-1}(a)$, $f^{-1}(f^{-1}(a))$, \dots и индукция по меньшему номеру);
$\exists n, \forall a: (f \circ dots \circ f)(a) = a$ (взяли произведение $n_a$); взяли два элемента с равными образами, наприменяли $f$ еще $n-1$ раз, $a=b$.

\section{} % 08
Пустое, полное, (анти)рефлексивность, симметричность, асимметричность ($<$; нет одновременно с двух сторон), антисимметричность ($\le$; если есть с двух
сторон "--- равенство), транзитивность, всего шесть. Нестрогий частичный порядок: $\le$, строгий: $<$. ЧУМ "--- частично упорядоченное множество (есть какой-то порядок).

\section{} % 09
$a \sim b$ рефлексивно, симметрично, транзитивно, определили $[a]$, они либо не пересекаются, либо совпадают, получили разбиение на классы, мн-во классов "--- фактормножество.

\section{} % 10
Ассоциативная бинарная операция, есть нейтральный и обртный элемент (с двух сторон), в Абелевой "--- коммутативность.
Единственны нейтральный и обратный; можно решать $ax=b$ и $xa=b$ единственным образом, обратный к $a_1a_2\dots a_n$.

\section{} % 11
Подгруппа, если группа относительно сужения. Достаточно замкнутости относительно двух операций. Пересечение "--- подгруппа.

\section{} % 12
<<Замыкание $M$ относительно операций>> и <<пересечение подгрупп, содержащих $M$>> определяют одно и то же.

\section{} % 13
Гомоморфизм $G \xrightarrow{f} H$: $f(ab)=f(a)f(b)$; изоморфизм "--- биекция. Гомоморфизм сохраняет единичный, обратные,
образ группы "--- подгруппа, композиция гомоморфизмов, обратный к изоморфизму. Изоморфизм "--- эквивалентность.

\section{} % 14
Циклическая порождается одним элементом ($\Z$; тривиальная из $e$), можно вводить целые степени.
Кстати, умеем делить с остатком (взяли целую часть; доказали единственность от противного и рассмотрев модули с двух сторон).
Циклическая группа изоморфна либо $\Z$ (если много элементов), либо $C_n$ "--- просто взяли последовательность $a$, $a^2$, $a^3$, \dots.
Порядок группы, порядок элемента ($a^n=e$; порядок $\left<a\right>$).

\section{} % 15
Есть подгруппа $H$. Отношение: $a^{-1}b \in H$, $[a]$ "--- это элементы вида $ah$ (левый класс). Левые классы либо не пересекаются,
либо совпадают.

\section{} % 16
Лемма: отображение $h \to ah$ ($H$ "--- подгруппа) есть биекция, то есть $|H|=|aH|$. Индекс: $[G:H]$ "--- число левых классов.
Лагранж: $|G|=[G:H]\cdot|H|$. Следствия: левых и правых классов одинаково, порядок подгруппы делит группы, порядок элемента делит,
группа порядка $p$ циклична.

\section{} % 17
См. 18. Потом: неподвижная точка, теорема: индукция по числу подвижных точек, получили нецепляющиеся циклы. Следствия:
$S_n$ порождается всеми циклами; можно $\sigma^n$, возведя циклы по отдельности (они коммутируют); порядок $\sigma$ "--- НОК
порядков циклов. Цикловой тип $\sigma$ "--- неупорядоченный набор длин циклов (плюс неподвижные).

\section{} % 18
$S_n$ "--- биекции, произведение "--- композиция. Цикл $i_1 \to i_2 \to \dots \to i_l \to i_1$, его порядок. Транспозиция,
незацепляющиеся циклы (они коммутируют). Потом порождаем транспозициями циклы, а циклами "--- перестановки (из 17 билета).

\section{} % 19
Множество инверсий $Inv(\sigma)$, четность, транспозиция её меняет (с двух сторон). Если разложить $\sigma$ в $\prod$ транспозиций,
то чётность сойдётся. Чётность произведения, знакопеременная группа $A_n$ (чётные перестановки).

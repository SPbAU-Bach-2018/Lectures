\section{Производная многочлена}

\begin{Def}
	Пусть $A$ "--- коммутативное кольцо с 1. Тогда определим
	умножение на натуральное число $k$:

	\begin{align*}
	k \cdot a &= \underbrace{a+a+\dots+a}_{k\text{~слагаемых}} = \\
	          &=(\underbrace{1+1+\dots+1}_{k\text{~слагаемых}})a
	\end{align*}

	Для удобства также положим $0\cdot a = 0$.
\end{Def}

\begin{Def}
	Для многочлена $f=a_nx^n+a_{n-1}x^{n-1}+\dots+a_0x^0$ определим производную:

	\begin{align*}
	f' &= (n \cdot a_n) x^{n-1} + \dots + (2\cdot a_2)x + a_1 = \\
	   &= \sum_{k = 1}^nka_kx^{k-1}
	   &= \sum_{k = 0}^nka_kx^{k-1} && \text{при $k=0$ фиктивное слагаемое}
	\end{align*}

	Несмотря на то, что $x^{-1}$ неопределено, фиктивное слагаемое при $k=0$ имеет коэффициент ноль
	(считаем, что $0 \cdot x^{-1} = 0$)	и иногда удобно считать, что оно есть.
\end{Def}

\begin{theorem}{Свойства производной}
	\begin{enumerate}
		\item $(f + g)' = f' + g'$ и более общий случай: $(f_1 + \dots + f_k)' = f_1' + \dots + f_k'$.
		\item $\forall c \in A \colon (c\cdot f)' = c\cdot f'$
		\item $(fg)' = f'g + fg'$
		\item $(f_1 \cdot f_2 \cdot \dots \cdot f_k)' = f_1' \cdot f_2 \cdot \dots \cdot f_k + f_1\cdot f_2'\cdot \dots \cdot f_k + \dots + f_1 \cdot \dots \cdot f_k'$
		\item $(f^k)' = kf^{k - 1}f'$ 
		\item Если $A$ "--- поле и $f \in A[x]$, то:
		\begin{itemize}
			\item Если $\mathrm{char} A = 0$, то
			    \[f' = 0 \Lra f = c_0 \in A = \mathrm{const} \]
			\item Если $\mathrm{char} A = p > 0$, то
				\[f' = 0 \Lra f \in A[x^p]\]
				то есть все ненулевые коэффициенты $f$ находятся при степенях вида $x^{kp}$.
		\end{itemize}
	\end{enumerate}
\end{theorem} 

\begin{proof}
	\begin{enumerate}
		\item Расписать по определению.
		\item Расписать по определению.
		\item Доказываем в несколько приёмов:
		\begin{itemize}
			\item Пусть $f = x^n$, $g = x^m$:
				\[(fg)' = (x^{n + m})' = (n + m)x^{n + m -1} = nx^{n-1}x^m + x^n mx^{m-1} = f'g + fg'\]
			\item Пусть $f = x^n$, $g = \sum_{k = 0}^m c_k x^k$:
				\begin{align*}
				(fg)' &= \left(\sum_{k=0}^m c_k x^n x^k\right)' = \sum_{k=0}^m c_k(x^n x^k)' = \\
				 &= \sum_{k=0}^m c_k(f'x^k + fkx^{k-1}) = f' \sum_{k=0}^m c_k x^k + f \sum_{k=0}^m kx^{k-1} = f'g + fg'
				\end{align*}
			\item Пусть $f = \sum_{k = 0}^n a_k x^k$, а $g$ "--- произвольный многочлен:
				\begin{align*}
				(fg)' &= \sum_{k = 0}^n a_k(x^k g)' = \sum_{k = 0}^n a_k(kx^{k-1}g + x^kg') =
				 &= g\sum_{k = 0}^n ka_kx^{k-1} + g'\sum_{k=0}^n a_kx^k = f'g + fg'
				\end{align*}
		\end{itemize}
		\item По индукции, несколько раз применяем пункт 3.
		\item Следует из пункта 4.
		\item 
		\begin{itemize}
			\item Пусть $\mathrm{char} A = 0$. Докажем $\Ra$:
				\begin{gather*}
					f = c_0 + c_1x + \dots + c_nx^n \\
				0 = f' = c_1 + 2c_2x + \dots + nc_nx^{n-1} \\
				\forall k>0 \colon kc_k = \underbrace{(1 + 1 + \dots + 1)}_{\neq 0, k\text{~единиц}}c_k = 0 \Ra c_k = 0 \\
				\Ra f = c_0 = \mathrm{const}
				\end{gather*}
				$\La$ очевидно.

			\item Пусть $\mathrm{char} A = p > 0$. Тогда $\underbrace{1+1+\dots+1}_p=0$.
				\begin{enumerate}
				\item $\Ra$:
	   				\begin{gather*}
   					f' = \sum_{k = 1}^nkc_kx^{k-1} \\
   					\forall k \geq 1 \colon kc_k = 0
   					\end{gather*}

   		 			Рассмотрим такое $k$, что $p \nmid k$ $\Ra$ $k = pq + r$, $1 \le r <p$:
   					\begin{align*}
	   		 		kc_k &= \underbrace{(1+1+\dots+1)}_{k}c_k = \\
	   		 		     &= (\underbrace{\underbrace{1+\dots+1}_p+ \dots + \underbrace{1+\dots+1}_p}_q + \underbrace{1+\dots+1}_r)c_k = \\
   			 		     &= \underbrace{(1+\dots+1)}_{r\text{~раз}, \neq 0} c_k = 0 \\
   		 			\Ra c_k = 0
   		 			\end{align*}
   		 		\item $\La$:	
   		 			\begin{align*}
		 			f &= c_0 + c_px^p + c_{2p}x^2p + \dots \in A[x^p] = \\
		 			  &= \sum\limits_{j=0}^rc_{jp}x^{jp}; \\
		 			f' &= \sum\limits_{j=0}^rjpc_{jp}x^{jp-1} = \\
		 			   &= \sum\limits_{j=0}^rj\cdot 0 \cdot c_{jp}x^{jp-1} = 0;
		 			\end{align*}
			 	\end{enumerate}
		\end{itemize}	
	\end{enumerate}
\end{proof}

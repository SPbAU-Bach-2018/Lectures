\section{Алгебраическая форма записи комплексного числа. Комплексное сопряжение. Свойства комплексного сопряжения}

$\R \mapsto \C\colon a \mapsto (a, 0)$ - инъективный гомоморфизм колец:
\begin{align*}
	\phi(a+b) &= \phi(a) + \phi(b) \\
	\phi(ab) &= \phi(a) \phi(b)
\end{align*}
\begin{gather*}
\C \supset \phi(\R) = \{(a, 0) \mid a \in \R\} \\
\phi(\R) \cong \R
\end{gather*}
Поэтому говорят, что $\R \subseteq \C$, имея в виду, что $\phi(\R) \subseteq \C$.
\begin{gather*}
i = (0, 1) \Ra i^2 = (-1, 0) = -1
\end{gather*}
\begin{Def}
Алгебраическая форма записи:
\begin{gather*}
(a, b) = (a, 0)*(1, 0) + (b, 0)*(0, 1) = a + bi
\end{gather*}
$a$ называется вещественной частью комплексного числа, $b$ "--- мнимой частью.
\begin{gather*}
a = \Re z \quad b = \Im z
\end{gather*}
\end{Def}

\begin{Def}
$z \in \C$, $z = a + bi$, $a, b \in \R$. $\bar z$ называется комплексно сопряжённым с $z$, если $\bar z = a - bi$.
\end{Def}
\begin{Rem}
Сопряжение $\sim$ симметрия относительно вещественной оси.
\begin{center}
\def\svgwidth{6.0cm}
\input{theory35-vector.pdf_tex}
\end{center}
\end{Rem}

\underline{Cвойства:}
\begin{itemize}
\item[1.] $\bar{\bar z} = z$
\item[2.] $z = \overline{z} \Lra z \in \R$
\item[3.] $\overline{z_1 + z_2} = \bar z_1 + \bar z_2$
\item[3'.] $\overline{z_1 + z_2 + \dots + z_n} = \bar z_1 + \bar z_2 + \cdots + \bar z_n$ (По индукции из свойства 3)
\item[4.] $\overline{z_1z_2} = \bar z_1 \bar z_2$
\item[4'.] $\overline{z_1 z_2 \cdots z_n} = \bar z_1 \bar z_2 \cdots \bar z_n$ (По индукции из свойства 4)
\item[5.] $f \in \R[x]~f = a_0 + a_1x + a_2x^2 + \dots + a_nx^n$ Тогда: $\overline{f(z)} = f(\bar z)$
\item[6.] \begin{itemize}
			\item[\bullet] $z + \bar z \in \R$ 
			\item[\bullet] $z \bar z \in \R$, $z \bar z \ge 0$
			\item[\bullet] $z \bar z = 0 \Lra z = 0$
		 \end{itemize}	
Два последних пункта следуют из того, что $z \bar z = a^2 + b^2$.
\begin{proof}
Только 5 свойство:
\begin{gather*}
f(z)=a_0 + a_1z + \cdots + a_nz^n \\
\overline{f(z)} = \overline{a_0 + a_1z + \cdots + a_nz^n} = \bar a_0 + \overline{a_1z} + \cdots + \overline{a_nz^n} = \bar a_0 + \bar a_1 \bar z + \cdots + \bar a_n \overline{z^n} = \\
= a_0 + a_1\bar {z} + \cdots + a_n\bar z^n = f(\overline{z})
\end{gather*}
\end{proof}   

$\bar z\colon \C \ra \C$ "--- гомоморфизм:
\begin{align*}
\overline{z_1 + z_2} &= \bar z_1 + \bar z_2 \\
\overline{z_1 z_1} &= \bar z_1 \bar z_2
\end{align*}

$\overline{z} \circ \overline{z} = id_\C$, поэтому сопряжение "--- нетождественный изоморфизм из $\C$ на себя (автоморфизм).
\begin{Def}
Автоморфизм "--- изоморфизм поля с самим собой.
\end{Def}
\item[7.] $z \ne 0$: $z \bar z = |z|^2$, $|z| \ne 0$.
\begin{gather*}
z \frac{\bar z}{|z|^2} = 1 \Ra z^{-1} = \frac{\bar z}{|z|^2} = \frac{a - bi}{a^2 + b^2}
\end{gather*}
PS: определение и проч. про модуль в следующем вопросе.
\end{itemize}


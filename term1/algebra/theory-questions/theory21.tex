\section{Мультипликативная группа кольца}

\begin{Def}	 
	Пусть $A$ - кольцо с 1.
	Введём множество $A^*$:
	\[A^* = \{ a \in A \mid \exists b \in A \colon ab = ba = 1 \}\]
	Тогда $\left<A^*, \cdot\right>$ "--- мультипликативная группа кольца\\
\end{Def}

\begin{theorem}{}
	$A^*$ - группа по умножению\\
\end{theorem}
\begin{proof}
	Существование единицы очевидно: $1_A \in A^*$, останется единицей и в $A^*$.
	Покажем замкнутость относительно операции умножения:
	\[a_1, a_2 \in A^* \iff \exists b_1, b_2 \in A \colon a_1b_1=b_1a_1=a_2b_2=b_2a_2 = 1\]
	Теперь покажем, что $a_1a_2$ тоже лежит в $A^*$ по определению:
	\begin{gather*}
	(a_1a_2)(b_1b_2) = a_1(a_2b_2)b_1 = a_1b_1 = 1; \\
	(b_2b_1)(a_1a_2) = b_2(b_1a_1)a_2 = b_2a_2 = 1;
	\end{gather*}
	Покажем замкнутость относительно взятия обратного:
	\[ a \in A^* \Ra \exists b \colon ab = ba = 1 \Ra b \in A^* \Ra a^{-1}_{A^*} = b \]
\end{proof}
	 
\textbf{ Примеры: }
\begin{itemize}
	\item $K$ "--- поле, тогда $K^* = K \setminus \{ 0 \}$
	\item $\Z^* = {-1, 1}$
\end{itemize}

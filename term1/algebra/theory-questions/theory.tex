%% texify: XeLaTeX + MakeIndex + BibTeX, modificated
%% texify
%% --pdf
%% --engine=xelatex
%% --tex-option=$synctexoption // You may delete this, affords to skip from preview to code in one click, that i seldom do
%% --tex-option=-8bit // Else minted fails on tabs
%% --tex-option=--shell-escape // For minted to live
%% $fullname
%% I prefer to build with TeXworks for better view of errors and warnings. It's hard to read all log file. 

\documentclass[12pt,a4paper]{article}
\usepackage{polyglossia} %% Better than babel on XeLaTeX
\usepackage{amsmath, amssymb} %% Cool math!
\usepackage{color} %% Coloring almost anything
\usepackage[russian]{hyperref} %% Clickable links is pdf
\usepackage{indentfirst}
\usepackage{ifthen}
\usepackage[left=1cm,right=1cm,top=2cm,bottom=2cm]{geometry}
\usepackage{wrapfig}
\usepackage{datetime}
%% WARNING: latest minted is used. Download from github!
%% Works fine, though
%\usepackage{minted} %% Highlighting code. Installation is hard: requires python2 and script Pygments. Look at documentation for help!
\usepackage[math-style=ISO,vargreek-shape=unicode]{unicode-math} %% MAGIC! INCLUDE AS LAST!
\usepackage{extarrows}

\setdefaultlanguage[spelling=modern,babelshorthands=true]{russian} %% Languages for polyglossia
\setotherlanguage{english}

\defaultfontfeatures{Ligatures={TeX}} %% Fonts and ligatures.
\setmainfont{CMU Serif} %% There are original Knuth's fonts in Unicode, called Computer Modern Unicode. Download anywhere, just install them
\setsansfont{CMU Sans Serif}
\setmonofont{CMU Typewriter Text}  
\setmathfont{Latin Modern Math} %% Download too. You may change it :)
\AtBeginDocument{\def\setminus{\mathbin{\backslash}}}
%\setmathfont{XITS}

%% Magic as black as my working table
%\DeclareSymbolFont{cyrletters}{\encodingdefault}{\familydefault}{m}{it}
%\newcommand{\makecyrmathletter}[1]{%
%  \begingroup\lccode`a=#1\lowercase{\endgroup
%  \Umathcode`a}="0 \csname symcyrletters\endcsname\space #1
%}
%\count255="409
%\loop\ifnum\count255<"44F
%  \advance\count255 by 1
%  \makecyrmathletter{\count255}
%\repeat
%% Simpy adds cyrillic to maths!

\frenchspacing %% One space before sentence, not two!
\allowdisplaybreaks[4]

%% Shortcuts:
\def\la{\leftarrow}
\def\ra{\rightarrow}
\def\lra{\leftrightarrow}
\def\La{\Leftarrow}
\def\Ra{\Rightarrow}
\def\Lra{\Leftrightarrow}
\def\lrh{\leftrightharpoons}
\def\xra{\xrightarrow} 
\def\btu{\bigtriangleup} 
\def\rat{\rightarrowtail}
\def\thra{\twoheadrightarrow}
\def\thrat{\twoheadrightarrowtail}

\def\ub{\underbrace}

\def\N{\mathbb{N}}
\def\Z{\mathbb{Z}}
\def\Q{\mathbb{Q}}
\def\R{\mathbb{R}}
\def\C{\mathbb{C}}
\def\F{\mathbb{F}}

\def\LraDef{\stackrel{\mathrm{Def}}{\Lra}}
\def\eqDef{\stackrel{\mathrm{Def}}{=}}
\def\d{\mathup{d}}

% ======================================

%% Change Chapter and Section numeration style
%%\renewcommand{\thechapter}{\Roman{chapter}}
%%\renewcommand{\thesection}{\thechapter.\arabic{section}}

%% Indent for first par in chapter
\makeatletter
%%\renewcommand{\chapter}{\clearpage %% no double page, only
\thispagestyle{empty}%% not plain, empty. wanna number of page!
\global\@topnum=0
\@afterindenttrue %% Set to true!
%%\secdef\@chapter\@schapter}
\makeatother

%% Environment for theorem body
\newcounter{theorem}[section]
\renewcommand{\thetheorem}{\thesection.\arabic{theorem}}
\newcommand*{\theoremheader}[1]{\par\refstepcounter{theorem}%
\textbf{Теорема \thetheorem.\ifthenelse{\equal{#1}{}}{}{ #1.}}}
\newenvironment*{theorem}[1]{
\theoremheader{#1}%
}{%
\par%
}

%% Environment for consequence body
\newcounter{conseq}[theorem]
\renewcommand{\theconseq}{\thetheorem.\arabic{conseq}}
\newcommand*{\conseqheader}{\par\refstepcounter{conseq}%
\textit{Следствие \theconseq.}}
\newenvironment*{conseq}{
\conseqheader%
}{%
\par%
}

\newcounter{lemma}[section]
\renewcommand{\thelemma}{\thesection.\arabic{lemma}}
\newcommand*{\lemmaheader}{\par\refstepcounter{lemma}%
\textit{Лемма \thelemma.}}
\newenvironment*{lemma}{
	\lemmaheader%
}{%
}

\newenvironment{assertion}{%
\par\textbf{Утверждение. }%
}{%
\par%
}

%% Environment for proof body. I like this style, but you are free to change it.
\newenvironment{proof}{%
\par$\blacktriangleright$%
}{%
\hfill$\blacktriangleleft$%
}

%% Environment for definitions. Pretty raw one.
\newenvironment{Def}{%
\par$\mathfrak{Def\colon}$%
}{%
\par%
}

%% Environment for remarks.
\newenvironment{Rem}{%
\par\textit{REM: }%
}{%
\par%
}

%% Environment for examples.
\newenvironment{exmp}{%
\par\textbf{Пример: }%
}{%
\par%
}

\setcounter{MaxMatrixCols}{40}

% ==================================

%% In-line code highlighting. Using: \py|a = input()|
%\newmintinline[cinl]{c}{} %\c is defined :(
%\newmintinline[cpp]{cpp}{}
%\newmintinline[python]{python}{}
%\newmintinline[bash]{bash}{}
%\newmintinline[make]{make}{}

%% Escaped code highlighting. Using: \begin{cppcode} ... \end{cppcode}
%\setminted{obeytabs,tabsize=4,linenos,texcomments}
%\newminted{c}{}
%\newminted{cpp}{}
%\newminted{python}{}
%\newminted{bash}{}
%\newminted{make}{}

% ==================================
\DeclareMathOperator{\Int}{int}
\DeclareMathOperator{\cl}{cl}
\DeclareMathOperator{\diam}{diam}
\DeclareMathOperator{\Dom}{Dom}
\DeclareMathOperator{\coDom}{coDom}
\DeclareMathOperator{\Char}{char}
\AtBeginDocument{\let\Re\relax}
\AtBeginDocument{\newcommand{\Re}{\mathop{\mathrm{Re}}\nolimits}}
\AtBeginDocument{\let\Im\relax}
\AtBeginDocument{\newcommand{\Im}{\mathop{\mathrm{Im}}\nolimits}}
\DeclareMathOperator{\Arg}{Arg}
\newcommand{\emod}[1]{\mathop{\equiv}\limits_{#1}}

\newcommand{\Choose}[2]{{\left(#1 \atop #2\right)}}

\begin{document}
\begin{center}
  {\Large \bf Лекции по алгебре} \\ 
  \vspace{0.5em}
  {\Large \bf Лектор: Всемирнов~Максим~Александрович} \\
  \vspace{0.5em}
  {\large Авторы конспекта: Егор~Суворов, Дмитрий~Лапшин, Ольга~Черникова, Надежда~Бугакова, Всеволод~Степенов, Глеб~Валин, Елизавета~Третьякова} \\
  \vspace{0.5em}
  {Собрано: \today~\currenttime} \\

\end{center}

\vspace{-1em}
\noindent \underline{\hbox to 1\textwidth{{ } \hfil{ } \hfil{ } }}

\vspace{1em}
\tableofcontents
\pagebreak

\section{Множества}

Не любая совокупность элементов --- множество. Про каждый объект можно сказать, принадлежит ли он множеству ($x \in A$) или нет ($x \notin A$).

\begin{Def}
Множество $A$ - подмножество $B$, если все элементы $A$ содержатся и в $B$. 
$$ A \subset B \LraDef \forall x \in A\; x \in B $$
\end{Def}
\begin{Def}
Множества называются равными, если они содержатся друг в друге.
$$ A = B \LraDef A \subset B \land B \subset A $$
\end{Def}
\begin{Def}
Пустое множество --- это множество без элементов.
$$ \forall x\: x \notin \emptyset $$
\end{Def}
\begin{Def}
$2^A$ --- множество всех подмножеств $A$.
$$ 2^A \eqDef \left\{B \mid B \subset A \right\} $$
\end{Def}

\begin{itemize}
\item $\N$ --- множество натуральных чисел. 
\item $\Z$ --- множество целых чисел.
\item $\Q$ --- множество рациональных чисел.
\item $\R$ --- множества вещественных чисел.
\item $\C$ --- множества комплексных чисел.
\end{itemize}

Задание множеств:
\begin{itemize} 
\item $\left\{a,b,c\right\}$
\item $\left\{a_1, a_2, \ldots, a_n\right\}$
\item $\left\{a_1, a_2, \ldots\right\}$
\item $\left\{x \in A \mid \Phi(x)\right\}, \Phi(x) - \text{условие}$.
\end{itemize} 
Например, $\left\{p \in \N \mid p \text{ имеет ровно 2 натуральных делителя}\right\}$.

Бывают некорректно заданные <<множества>>. Например, множество художественных произведений на русском языке --- плохо заданное множество. Рассмотрим 
$\Phi(n)$ --- истина, если n нельзя записать в не более чем тридцать слов русского языка. Тогда
$\left\{n \in \N \mid \Phi(n)\right\}$~--- не множество. Если бы это было множеством, то в нём есть наименьший элемент, 
который описывается как <<Наименьший элемент множества...>>

\begin{Def}
Пересечение двух множеств~--- множество, состоящие из всех элементов, находящихся одновременно в обоих множествах.
$$ A \cap B \eqDef \left\{x \in A \mid x \in B \right\} $$
\end{Def}
\begin{Def}
Объединение двух множеств~--- множество, состоящее из элементов обоих множеств.
$$ A \cup B \eqDef \left\{x \mid x \in A \lor x \in B \right\} $$
\end{Def}
\begin{Def}
Разность множеств~--- это множество тех элементов, которые лежат в первом, но не во втором.
$$ A \setminus B \eqDef \left\{ x \in A \mid x \notin B \right\}$$
\end{Def}
\begin{Def}
Симметрическя разность~--- объединение разностей.
$$ A \btu B \eqDef \left(A \setminus B\right) \cup \left(B \setminus A\right) $$
\end{Def}

Объединение и пересечение множно записать для многих множеств.
$$ \bigcup_{i \in I} A_i = \left\{x \mid \exists i \in I\colon x \in A_i\right\}; 
\bigcap_{i \in I} A_i = \left\{x \mid \forall i \in I\: x \in A_i \right\} $$

Свойства операций со множествами:
\begin{enumerate}
\item Ассоциативность
$$ A \cap B = B \cap A; A \cup B = B \cup A $$
\item Коммутативность
$$ \left(A \cap B \right) \cap C = A \cap \left(B \cap C \right); \left(A \cup B \right) \cup C = A \cup \left(B \cup C \right) $$
\item Рефлексивность
$$ A \cap A = A; A \cup A = A $$
\item Дистрибутивность
$$ A \cap \left(B \cup C \right) = \left(A \cap B\right) \cup \left(A \cap C \right) $$
$$ A \cup \left(B \cap C \right) = \left(A \cup B\right) \cap \left(A \cup C \right) $$
\item Нейтральный элемент
$$ A \cap \emptyset = \emptyset$$
$$ A \cup \emptyset = A$$
\end{enumerate}

\begin{theorem}{Правила де Моргана}
$ A, B_\alpha, \alpha \in I $.
Тогда 
$$ A \setminus \bigcup_{\alpha \in I} B_\alpha = \bigcap_{\alpha \in I} \left(A \setminus B_\alpha\right) ; 
A \setminus \bigcap_{\alpha \in I} B_\alpha = \bigcup_{\alpha \in I} \left(A \setminus B_\alpha\right) $$
\end{theorem} 
\begin{proof}
$$
x \in A \setminus \bigcup_{\alpha \in I} B_{\alpha} \Lra \left\{\begin{aligned}x &\in A \\ x &\notin \bigcup_{\alpha \in I} B_{\alpha}\end{aligned}\right. \Lra 
\left\{\begin{aligned} x &\in A \\ \forall \alpha \in I\: x &\notin B_\alpha \end{aligned}\right. \Lra
\forall \alpha \in I\: \left\{\begin{aligned} x &\in A \\ x &\notin B_\alpha \end{aligned}\right.  
\Lra x \in \bigcap_{\alpha \in I} \left(A \setminus B_\alpha\right) 
$$
$$
x \in A \setminus \bigcap_{\alpha \in I} B_{\alpha} \Lra \left\{\begin{aligned}x &\in A \\ x &\notin \bigcap_{\alpha \in I} B_{\alpha}\end{aligned}\right. \Lra 
\left\{\begin{aligned} x &\in A \\ \lnot \forall \alpha \in I\: x &\in B_\alpha \end{aligned}\right. \Lra
\exists \alpha \in I\colon \left\{\begin{aligned} x &\in A \\ x &\notin B_\alpha \end{aligned}\right.  
\Lra x \in \bigcup_{\alpha \in I} \left(A \setminus B_\alpha\right) 
$$
\end{proof}

\begin{theorem}{Обобщение дистрибутивности}
$ A, B_\alpha, \alpha \in I $.
Тогда 
$$ A \cap \bigcup_{\alpha \in I} B_\alpha = \bigcup_{\alpha \in I} (A \cap B_\alpha) $$
$$ A \cup \bigcap_{\alpha \in I} B_\alpha = \bigcap_{\alpha \in I} (A \cup B_\alpha) $$
\end{theorem}
\begin{proof}
$$
x \in A \cap \bigcup_{\alpha \in I} B_{\alpha} \Lra \left\{\begin{aligned}x &\in A \\ x &\in \bigcup_{\alpha \in I} B_{\alpha}\end{aligned}\right. \Lra 
\left\{\begin{aligned} x &\in A \\ \exists \alpha \in I\colon x &\in B_\alpha \end{aligned}\right. \Lra
\exists \alpha \in I\colon \left\{\begin{aligned} x &\in A \\ x &\in B_\alpha \end{aligned}\right.  
\Lra x \in \bigcup_{\alpha \in I} \left(A \cap B_\alpha\right) 
$$
$$
x \in A \cup \bigcap_{\alpha \in I} B_{\alpha} \Lra \left[\begin{aligned}x &\in A \\ x &\in \bigcap_{\alpha \in I} B_{\alpha}\end{aligned}\right. \Lra 
\left[\begin{aligned} x &\in A \\ \forall \alpha \in I\: x &\in B_\alpha \end{aligned}\right. \Lra
\forall \alpha \in I\: \left[\begin{aligned} x &\in A \\ x &\in B_\alpha \end{aligned}\right.  
\Lra x \in \bigcap_{\alpha \in I} \left(A \cup B_\alpha\right) 
$$
\end{proof}

\begin{Def}
Упорядоченная пара $\langle a, b \rangle$ или $(a, b)$ --- объект
$$ (a_1; b_1) = (a_2; b_2) \LraDef a_1 = a_2 \land b_1 = b_2 $$
\end{Def}
\begin{Def}
Упорядоченная $n$-ка, или кортеж --- объект
$$ (a_1, a_2, \ldots, a_n) = (b_1, b_2, \ldots, b_n) \LraDef \forall i=1..n\: a_i = b_i $$
\end{Def}

\begin{Def}
Декартого произведение множеств --- множество кортежей, состоящих из элементов соответствующих множеств.
$$ \left(a_1, a_2, \ldots, a_n\right) \in A_1 \times A_2 \times \ldots \times A_n \LraDef \forall i=1..n\: a_i \in A_i $$
\end{Def}
\section{Бинарные отношения}

\begin{Def}
Отношение на множествах $A$ и $B$ --- произвольное подмножество их декартова произведения.
$$ a \mathop{R} b \LraDef (a, b) \in R $$
\end{Def}
\begin{Def}
Область определения отношения 
$$ \beta_R = dom_R = \{a \in A \mid \exists b \in B\colon (a,b) \in R\} $$
\end{Def}
\begin{Def}
Обсласть значения отношения 
$$ \rho_R = ran_R =\{b \in B \mid \exists a \in A\colon (a, b) \in R\}$$
\end{Def}
\begin{Def}
Обратное отношение
$$R^{-1} \colon \beta_{R^{-1}} = \rho_R; \rho_{R^{-1}} = \beta_R; b \mathop{R^{-1}} a \LraDef a \mathop{R} b$$
\end{Def}
\begin{Def}
Композиция отношений
$$ R_1\colon A \ra B; R_2\colon B \ra C $$
$$ R_1 \circ R_2 = \{(a, c) \mid a \mathop{R_1} b \land b \mathop{R_2} c\} $$
Про значок ~--- его использовать не будем
\end{Def}

Пример композиции: $<\colon \N \ra \N$. $$< \circ < = \{(a, b) \mid b - a \geqslant 2\}$$

\begin{Def}
Функция (отображение) ~--- такое отношение, что первый ключ уникален.
$$f\colon A \ra B$$
$$ a \mathop{f} b_1 \land a \mathop{f} b_2 \Ra b_1 = b_2 $$
$$ a \mathop{f} b \LraDef f(a) = b $$
$$ A = \beta_f \quad \text{($A$~--- область определения)}$$
\end{Def}

\begin{Def}Свойтва отображеий:
\begin{enumerate}
\item Рефлексивность $a \mathop{R} a$
\item Cимметричность $a \mathop{R} b \Lra b \mathop{R} a$
\item Транзитивность $a \mathop{R} b \land b \mathop{R} c \Ra a \mathop{R} c$
\item Иррефлексивность $\lnot a \mathop{R} a$
\item Антисимметричность $a \mathop{R} b \land b \mathop{R} a \Ra a = b$
\end{enumerate}
\end{Def}

Примеры:
\begin{itemize} 
\item $=$: 1, 2, 3, 5
\item $\emod{5}$: 1, 2, 3
\item $\leqslant$: 1, 3, 5
\item $<$: 3, 4, 5
\item $\subset$: 1, 3, 5
\end{itemize}
\section{Тождественное отображение. Теорема о свойствах отображений, композиция которых есть
тождественное отображение}

\subsection{Тождественное отображение}

\begin{Def} Пусть есть $A$. Тогда тождественное отображение $id_{A}: A \to A$ задаётся как:

\[\forall a \in A\colon id_A(a) = a\]

$\Gamma_{id_A}$ есть диагональ $A \times A$, то есть $\Gamma_{id_a} = \{(a, a) \mid a \in A\}$
\end{Def}

\begin{theorem}{}
$f: A \to B$, тогда:

\[f \circ id_A = f =  id_B \circ f\]
\end{theorem}

\begin{proof}

Области определения и назначения совпадают.

Пусть $a \in A$. Проверим первое равенство:
\[(f \circ id_A)(a) = f(id_A(a)) = f(a)\]
и второе:
\[(id_B \circ f)(a) = id_B(f(a)) = f(a)\]
\end{proof}

\subsection{Инъекция, сюръекция, биекция}
\begin{Def}
$f: A \to B$. Тогда $f$ "--- инъективное отображение (инъекция), если:

\begin{enumerate}
\item $\forall a_1, a_2 \in A, \exists b\colon (a_1, b) \in \Gamma_f \wedge (a_2, b) \in \Gamma_f \Ra a_1 = a_2$
\item $\forall a_1, a_2 \in A\colon f(a_1) = f(a_2) \Ra a_1 = a_2$
\end{enumerate}

Обозначается $f: A \rat B$.
\end{Def}

\begin{Def}
Отображение $f: A \to B$ назывется сюръективным (сюрекцией, или <<отображение \textit{на} $B$>>), если:
\[\forall b \in B, \exists a \in A\colon b = f(a)\]

Обозначние: $f: A \thra B$
\end{Def}

\begin{Def}
$f$ называется биективным (или биекцией), если $f$ и сюръективно, и инъективно.

Обозначение: $f: A \thrat B$.
\end{Def}

\begin{Def} 
Образ $C \subset A$: $f(C) = \{b \in B \mid \exists c \in C b = f(c)\}$.
\end{Def}

\begin{Def} 
Полный прообраз $D \subset B$: $f^{-1}(D) = \{a \in A \mid f(a) \in D\}$.
\end{Def}

$f(f^{-1}(D)) \subseteq D$, но может не совпадать.

$f$ инъективно $\iff$ прообраз любого одноэлементного множества содержит не более одного элемента.

$f : A \to B$ сюръективно $\iff$ $f(A) = B$.

\subsection{Теорема}
\begin{theorem}{}
$f:A \to B$, $g:B \to A$. Если $g \circ f = id_A$, то $f$ "--- инъективно, $g$ "--- сюръективно.
\end{theorem}

\begin{proof}
\begin{enumerate}
\item Пусть $a_1, a_2 \in A\colon f(a_1) = f(a_2)$. Тогда:
\begin{gather*}
g(f(a_1)) = g(f(a_2)); \\
(g \circ f)(a_1) = (g \circ f)(a_2); \\
id_A(a_1) = id_A(a_2); \\
a_1 = a_2; \\
\Ra f \text{ "--- инъекция}
\end{gather*}

\item Пусть $a \in A$ и $b=f(a)$, тогда:
\begin{gather*}
g(f(a)) = (g \circ f)(a) = id_A(a) = a; \\
a = g(b); \\
\Ra \forall a \in A, \exists b \in B\colon a = g(b) \Ra g \text{ "--- сюръекция}
\end{gather*}

\end{enumerate}
\end{proof}

\section{Равносильность инъективности и обратимости слева}

\begin{theorem}{}
Пусть $f:A \to B$ и $A \ne \varnothing$. Тогда $f$ обратима слева $\iff$ $f$ инъективна.
\end{theorem}

\begin{proof}
\begin{enumerate}
\item $\Ra$

$\exists g \colon g \circ f = id_A \Ra f$ инъективно.

\item $\La$

Пусть $C = f(A)$. Построим $h_1: C \to A$ такое, что
\[(c, a) \in \Gamma_{h_1} \Lra (a, c) \in \Gamma_{f}\]. Проверим, что это график:

\begin{enumerate}
\item Определённость для $c \in C$:
\begin{gather*}
\forall c \in C, \exists a \in A \colon (a, c) \in \Gamma_{f}; \\
\forall c \in C, \exists a \in A \colon (c, a) \in \Gamma_{h_1}; \\
\end{gather*}
\item Однозначность. Знаем, что $f$ инъективно. 
\begin{gather*}
\forall a_1, a_2 \in A, \exists b \in B \colon (a_1, b) \in \Gamma_{f} \wedge (a_2, b)\in \Gamma_{f} \Ra a_1 = a_2; \\
\forall a_1, a_2 \in A, \exists b \in C \colon (a_1, b) \in \Gamma_{f} \wedge (a_2, b)\in \Gamma_{f} \Ra a_1 = a_2; \\
\forall a_1, a_2 \in A, \exists b \in C \colon (b, a_1) \in \Gamma_{h_1} \wedge (b, a_2)\in \Gamma_{h_1} \Ra a_1 = a_2;
\end{gather*}
\end{enumerate}

$\Ra \Gamma_{h_1}$ "--- график.

Теперь построим $h: B \to A$. Для этого выберем произвольный $a \in A$ и положим:

$h(b) = \begin{cases} h_1(b), & \text{если~} b \in C\\ a, &\text{если~} b \notin C\end{cases}$

Проверим, что $h \circ f = id_A$. Рассмотрим $x \in A$:
\[(h \circ f)(x) = h(f(x)) = h_1(f(x)) = x\]
\end{enumerate}
\end{proof}

\section{Равносильность сюръективности и обратимости справа}

\begin{Def}{\bf Аксиома выбора}
Пусть есть множество $B$ и семейство $X_b$: $\forall b \in B \colon X_b \neq \varnothing$. Тогда
\[\exists \Phi: B \to \bigcup\limits_{b \in B}X_b\]
такое, что
\[\forall b \in B \colon \Phi(b) \in X_{b}\]
\end{Def}

\begin{theorem}{}

$f : A \to B$ "--- обратимо справа $\Lra$ $f$ "--- сюръективно.

\end{theorem}

\begin{proof}
\begin{enumerate}
\item $\Ra$

Очевидно, так как $f(g(B)) = B$.

\item $\La$

Введём семейство $X_b$ для $b \in B$: $X_b = f^{-1}(\{b\}) \ne \varnothing$.

По аксиоме выбора существует $g: B \to \bigcup\limits_{b \in B}X_b$ (обозначим область назначения $g$ за $C$). $f(g(b))=b$, так
как $g(b) \in f^{-1}(\{b\})$.

Покажем, что $C = A$.
Так как $X_b = f^{-1}(\{b\}) \subset A$, то $C \subset A$.
Обратное включение: возьмём $a \in A$. Знаем, что $a \in X_{f(a)} \Ra a \in C \Ra A \subset C \Ra A = C$.
Таким образом $g: B \to A$.

По построению знаем, что:
\begin{gather*}
\forall b \in B \colon f(g(b)) = b; \\
\forall b \in B \colon (f \circ g)(b) = b; \\
f \circ g = id_{B}; \\
\Ra f \text{~обратима справа}
\end{gather*}
\end{enumerate}
\end{proof}

\begin{conseq}{}

$f$ "--- обратимо $\iff$ $f$ "--- биективно.

\end{conseq}

\section{Метрические пространства}

\begin{Def}
Пусть есть множество $X$ и отображение $\rho \colon X \times X \ra \left[0; +\infty\right) $. Тогда $\rho$ называется метрикой, если:
\begin{enumerate}
\item $\rho(x, y) = 0 \Lra x = y$
\item $\rho(x, y) = \rho(y, x)$
\item $\rho(x, y) + \rho(y, z) \geqslant \rho(x, z)$
\end{enumerate} 
Также пара $(X, \rho)$ называется метричесикм пространством.
\end{Def}

Примеры:
\begin{enumerate}
\item Дискретная метрика
$\rho(x, y) = \begin{cases}0 & x \ne y \\ 1 & x = y\end{cases}$
\item $\rho(x, y) = \left|x - y\right|$
\item Евклидовская метрика. $\rho$ --- длина отрезка на плоскости между точками
\item Манхеттанская метрика. $\rho\left((x_1, y_1), (x_2, y_2)\right) = |x_1 - x_2| + |y_1 - y_2|$
\item Расстояния на сфере.
\item Французская железнодорожная метрика. Есть центр --- точка $O$. Тогда для точек на одном луче из $O$ расстояние $\rho(A, B) = |AB|$, иначе $\rho(A, B) = |AO| + |BO|$
\item Пространство $\R^n$, метрика $$\rho(x, y) = \sqrt{\sum_{i=1}^n \left(x_i-y_i\right)^2}$$
\end{enumerate}

\begin{Def}
Пусть $(X, \rho)$ --- метрическое пространство. Тогда $(Y, \rho|_{Y \times Y})$ --- подпространство X. $Y \subset X$.
\end{Def}


\begin{Def}
$B_r(a) = \left\{x \in X \mid \rho(x, a) < r\right\}$ --- открытый шар.
\end{Def}
\begin{Def}
$\bar B_r(a) = \left\{x \in X \mid \rho(x, a) \leqslant r\right\}$ --- замкнутый шар.
\end{Def}

Свойства:
\begin{enumerate}
\item $B_{r_1}(a) \cap B_{r_2}(a) = B_{\min\{r_1, r_2\}}(a)$
\item $x \ne y \Ra \exists r > 0\colon B_r(x) \cap B_r(y) = \varnothing$
\begin{proof}
Рассмотрим $r = \frac13 \rho(x,y) > 0$.
\end{proof}
\end{enumerate}
         


\section{Неравентсва Коши-Буняковского и Минковского}

\begin{theorem}{Неравенство Коши-Буняковского}
$a_1, a_2, \ldots a_n, b_1, b_2, \ldots, b_n \in \R$
$$\left(\sum_{k=1}^n a_kb_k\right){}^2 \leqslant \sum_{k=1}^n a_k^2 \sum_{k=1}^n b_k^2 $$
\end{theorem}
\begin{proof}
$$f(t) \lrh \sum_{k=1}^n(a_kt-b_k)^2 = \left(\underbrace{a_1^2 + a_2^2 + \ldots + a_n^2}_{\lrh A}\right)t^2 - 
2\left(\underbrace{a_1b_1 + \ldots + a_nb_n}_{\lrh C}\right)t + \left(\underbrace{b_1^2 + \ldots + b_2^2}_{\lrh B}\right)$$
$f$ имеет не более 1 корня, следовательно
$$ (2C)^2 - 4AB \leqslant 0 \Ra 4\left(C^2 - AB\right) \leqslant 0 \Lra C^2 \leqslant AB$$
Можно считать, что все числа не равны 0~--- иначе всё тривиально.
\begin{Rem}
Равентсво в случае, если числа пропорциональны.
\end{Rem}
\begin{proof}
$$a_i = \alpha b_i$$

$\Lra$

$$C^2 = AB \Lra \text{есть корень} t_0 \Lra \forall a_k t_0 - b_k = 0$$
\end{proof}
\end{proof}

\begin{theorem}{Неравенство Минковского}
$$\sqrt{\sum_{i=1}^n (a_i+b_i)^2} \leqslant \sqrt{\sum_{i=1}^k a_i^2} + \sqrt{\sum_{i=1}^k b_i^2}$$
\end{theorem}
\begin{proof}
Возведём в квадрат
$$ \sqrt{\sum_{i=1}^n (a_i+b_i)^2} \leqslant \sqrt{\underbrace{\sum_{i=1}^k a_i^2}_{\lrh A}} + \sqrt{\underbrace{\sum_{i=1}^k b_i^2}_{\lrh B}} \Lra \sum_{i=1}^n (a_i + b_i)^2 \leqslant A + 2\sqrt{AB} + B \Lra$$
$$ \Lra A + B + 2\sum_{i=1}^n a_ib_i \Lra A + B + 2\sqrt{AB} \Lra \sum_{i=1}^n a_ib_i \leqslant \sqrt{AB} \La$$
$$ \La \text{Неравенство Коши-Буняковского}$$
\begin{Rem}
Равентсво в случае, если числа пропорциональны.
\end{Rem}
\end{proof}      
\section{Открытые множества}
\begin{Def}
$(X, \rho)$ --- метрическое пространство. $G \subset X$ --- открытое множество, если $$\forall x \in G\: \exists r > 0\colon B_r(x) \subset G$$
\end{Def}

\begin{theorem}{О свойтсвах открытых множеств}
Пусть $(X, \rho)$ --- метрическое пространство.
\begin{enumerate}
\item $\varnothing$ и $X$ --- открыты.
\item Объединение открытых открыто.
\item Пересечение \textbf{конечного числа} открытых открыто.
\item $B_r(a)$ открыт.
\end{enumerate}
\end{theorem}
\begin{proof}
\begin{enumerate}
\item Очевидно.
\item $$x \in \bigcup G_\alpha \Ra \exists \alpha_0 \colon x \in G_{\alpha_0} \Ra \exists r > 0: B_r(x) \in \bigcup G_\alpha$$
\item $x \in \bigcap_{k=1}^n G_k$ 
$$ \forall k=1..n\: x \in G_k \Ra \forall k=1..n\: \exists r_k > 0\colon B_{r_k}(x) \in G_k \Ra \exists r = \min{r_k}\colon G_r \in \bigcap_{k=1}^n G_k$$
\item $$\forall x \in B_r(a)\: \exists r_x = \frac12 \left(r - \rho(a, x)\right)$$
$$y \in B_{r_x}(x) \Ra \rho(y, x) < r_x \Ra \rho(y, x) + \rho(a, x) < r_x + \rho(a, x) \Ra \rho(y, a) < r$$
\end{enumerate}
\end{proof}

\begin{Rem}
$$\bigcap_{n=1}^\infty \left(0; 1 + \frac1n\right) = \left(0;1\right] \text{ --- не открытое множество}$$
\end{Rem}
\section{Внутренние точки и внутренность множества}
\begin{Def}
$x \in A$ --- внутренняя точка $A$, если $\exists r > 0\colon B_r(x) \in A$
\end{Def}
\begin{Rem}
$x$ --- внутренняя точка $A$ эквивалентно тому, что в $A$ содержится некое открытое множество, содержащее x.
\end{Rem}
\begin{Def}
Внутренность множества $A$:
$$A^0 = \Int A \eqDef \bigcup_{\substack{G \text{ открыто} \\ G \subset A}} G$$
\end{Def}

Свойства:
\begin{enumerate}
\item $\Int A \subset A$
\item $\Int A$ --- множество всех внутренних точек.
\item $\Int A$ открыто.
\item $A \text{ открыто} \Lra A = \Int A$
\item $A \subset B \Ra \Int A \subset \Int B$
\item $\Int (A \cap B) = \Int A \cap \Int B$
\item $\Int \Int A = \Int A$
\end{enumerate}
\section{Группы. Простейшие следствия из аксиом группы}
\begin{Def}
	Бинарная операция на $A$ "--- отображение $f : A \times A \ra A$
\end{Def}

\begin{Def}
Пусть $G \neq \emptyset$, $\cdot : G \times G \ra G$.
Тогда $\left<G, \cdot \right>$ "--- группа, если выполняются следующие свойства:
\begin{enumerate}
	\item Ассоциативность: $(a \cdot b) \cdot c = a \cdot (b \cdot c)$ (для любых троек)
	\item $\exists$ нейтральный елемент $e$ такой, что $\forall a \in G \colon a \cdot e = e \cdot a = a$ 
	\item $\forall a \in G, \exists a^{-1} \colon a \cdot a^{-1} = a^{-1} \cdot a = e$		
	\item $\forall a, b \in G \colon a \cdot b = b \cdot a$ "--- если это свойство выполняется, то группа Абелева (коммутативная).
\end{enumerate}
\end{Def}

В дальнейшем под записью $ab$ будет пониматься $a \cdot b$, если, разумеется, не сказано другого.

\begin{exmp}
Пусть $X$ "--- множество. $S(X)$ "--- множество биекций из X в X.
Если взять операцию <<композиция>>, получится группа. Ассоциативность композиции знаем,
нейтральный элемент "--- $id_X$, обратимость биекции тоже знаем. Коммутативность отсутствует.
\end{exmp}

\begin{exmp}
Пусть $X = \{1, \ldots, n\}$. $S_n = S(\{1, \ldots n\})$ "--- симметрическая группа степени $n$,
группа перестановок чисел от 1 до $n$.
\end{exmp}
	
\begin{theorem}{Простейшие свойства групп}
	\begin{enumerate}
		\item Единственность нейтрального
		\item Единственность обратного
		\item 
			Уравнения вида $ax=b$, $ya=b$ имеют решение, причем единственное
		\item $(a_1 a_2 \dots a_n)^{-1} = a_n^{-1} a_{n-1}^{-1} \dots a_1^{-1}$
   	\end{enumerate}
\end{theorem}
\begin{proof}
	\begin{enumerate}
		\item
			Пусть существуют два нейтральных элемента $e_1, e_2$:
			\[e_2 = e_1 e_2 = e_1 \Ra e_2 = e_1\]
		\item
			Пусть $a', a''$ "--- обратные к $a$, тогда:
			\begin{gather*}
			a'aa'' = (a'a)a''= ea'' = a''; \\
			a'aa'' = a'(aa'') = a'e = a'; \\
			a' = a'';
			\end{gather*}
		\item
			$ax = b \iff b^{-1}ax = e \Ra x = (b^{-1}a)^{-1}$.
			Так мы доказали одним махом и существование, и единственность (так как обратный элемент существует и единственен).
		\item
			$(a_1 \dots a_n) (a_1^{-1} \dots a_n^{-1}) = (a_1 \dots a_{n-1})(a_n a_n^{-1})(a_{n-1}^{-1} \dots a_1^{-1})$.
			Центральная скобка равна $e$. Таким образом избавляемся от всех переменных, получаем, что правая скобка обратна левой. 
	\end{enumerate}
\end{proof}
	

\section{Подгруппы. Критерий того, что непустое подмножество группы является подгруппой. Пересечение подгрупп}
\begin{Def}
	Пусть $f: A \to B$ и $C \subset A$. Введём $g$ "--- сужение $f$ на $C$:
	\begin{gather*}
	g: C \ra B \\
	\forall c \in C\colon g(c) = f(c)
	\end{gather*}
\end{Def}

\begin{Def}
	$H \subset G$ "--- подгруппа в $G$, если она является группой относительно сужения операции в $G$ на $H$.
\end{Def}


\begin{Def}
	Множество $A$ замкнуто относительно операции $\cdot$, если $\forall a, b \in A \colon a \cdot b \in A$

	Множество $A$ замкнуто относительно операции взятия обратного, если $\forall a \in A \colon a^{-1} \in A$ \\
\end{Def}

\begin{theorem}{Достаточные условия для подгруппы}
Для того, чтобы доказать, что $H$ "--- подгруппа $G$ ($H \neq \varnothing$, $H \subset G$), достаточно проверить только замкнутость относительно операциий $\cdot$ и взятия обратного элемента. \\
\end{theorem}
\begin{proof}
Ассоциативность к нам переходит из исходной группы $G$.

Если существует обратный элемент, то $a a^{-1} = e \in H$

Так как есть замкнутость, то операция $\cdot$ действует из $H \times H$ в $H$.
\end{proof}

\begin{conseq}
Пусть $\varnothing \ne H \subset G$ и $\forall a, b \in H\colon ab^{-1} \in H \Ra H$ "--- подгруппа.  
\end{conseq}
\begin{proof}
Нейтральный элемент есть: $a \in H \Ra aa^{-1} \in H \Ra e \in H$.

Замкнутость относительно взятия обратного: $\forall a \in H \colon ea^{-1} \in H \Ra a^{-1} \in H$.

Замкнутость относительно операции $\cdot$: $\forall a, b \in H \Ra a, b^{-1} \in H \Ra a\left(b^{-1}\right)^{-1} \in H \Ra ab \in H$.
\end{proof}

\begin{theorem}{}
	$H_\alpha$ "--- подгруппы в $G$. Тогда $\bigcap H_\alpha$ "--- подгруппа в $G$.
\end{theorem}
\begin{proof}
    \begin{gather*}
	H = \bigcap H_\alpha; \\
	e \in H_{\alpha} \Ra e \in H \Ra H \ne \varnothing; \\
	a, b \in H \Ra \forall \alpha\colon a, b \in H_\alpha \Ra ab \in H_\alpha \Ra ab \in H; \\
	a \in H \Ra \forall \alpha \colon a \in H_\alpha \Ra a^{-1} \in H_\alpha \Ra a^{-1} \in H;
	\end{gather*}
\end{proof}

\section{Два эквивалентных определения подгруппы, порожденной множеством}
\begin{Def}
	Замыканием множества относительно операции "--- множество всех элементов, получаемых из элементов этого множества применением данной операции.

	Аналогично определяется замыкание относительно множества операций.
\end{Def}

В частности, для группы это замыкание $A$ будет выглядеть как: \[\{a_1^{z_1}a_2^{z_2}\dots | a_i \in A, z_i = \pm 1 \}\], где $A$ "--- данное множество.

\begin{Def}
	Подгруппа, порожденная множеством "--- замыкание этого множества относительно операций $\cdot$ и взятия обратного элемента
\end{Def}

\begin{Def}
	Подгруппа, порожденная множеством "--- пересечение всех подгрупп, содержащих это множество 
\end{Def}

\begin{Rem}
	Предыдущие два определения действительно задают подгруппы, именно это мы доказывали в предыдущих теоремах 
\end{Rem}

\begin{theorem} {Равносильность определений подгруппы, порожденной множеством}
	Пусть $M \subset G$, $G$ "--- группа, $A$ "--- замыкание $M$ относительно операций $\cdot$ и взятия обратного элемента,
    $B = \bigcap \{ H \in 2^G \mid H \supset M \land H \text{"--- подгруппа } G\}$.

	Тогда $A = B$.
\end{theorem}
\begin{proof}
	$A \subset B$, так как любая подгруппа, содержащая $M$, должна содержать замыкание $M$.

	Но так как $A$ "--- подгруппа, то $B \subset A$.

	$\Ra A = B$.
\end{proof}

\section{Гомоморфизмы групп. Свойства гомоморфизмов}
\begin{Def}
	$H$, $G$ "--- группы, $f: G \to H$
	\begin{enumerate}
		\item $f$ "--- гомоморфизм, если $\forall a, b \in G \colon f(ab) = f(a)f(b)$
		\item $f$ "--- изоморфизм, если $f$ "--- и гомоморфизм, и биекция
	\end{enumerate}
\end{Def}

\begin{Def}
	$H$, $G$ "--- группы. Если между $H$, $G$ есть изоморфизм, то группы называются изоморфными: $H \cong G$
\end{Def}						

\begin{theorem}{Свойства гомоморфизма}
	\begin{enumerate}
		\item $f(e_G) = e_H$
		\item $f(x^{-1}) = (f(x))^{-1}$
		\item $f(G)$ "--- подгруппа $H$
		\item $G \xrightarrow{f} H \xrightarrow{g} K$, $f$, $g$ "--- гомоморфизмы, тогда $g \circ f: G \to K$ "--- тоже гомоморфизм
		\item $f: G \to H$ "--- изоморфизм, тогда $f^{-1}: H \to G$ "--- изоморфизм
	\end{enumerate}
\end{theorem}
\begin{proof}
	\begin{enumerate}
		\item $f(e_G) = f(e_G e_G) = f(e_G)f(e_G) = e_H e_H = e_H$
		\item $f(e_G) = f(x x^{-1}) = f(x)f(x^{-1}) = e_H \Ra f(x^{-1}) = (f(x))^{-1}$
		\item
		    Покажем замкнутость относительно операции $\cdot$
		    \begin{gather*}
			e_H \in f(G); \\
			c, d \in f(G) \iff \exists a, b \in G \colon c = f(a), d = f(b); \\
			cd = f(a)f(b) = f(ab) \Ra cd \in f(G); \\
			\end{gather*}

			Покажем замкнутость относительно взятия обратного (из пункта 2 теоремы):
			\[f(a)^{-1} = f(a^{-1}) \in f(G)\]

			Таким образом, $f(G)$ "--- подгруппа.
		\item Пусть $a, b \in G$, тогда:
			\[(g \circ f)(ab) = g(f(ab)) = g(f(a)f(b)) = g(f(a))g(f(b)) = (g \circ f)(a)(g \circ f)(b)\]
		\item
			\begin{gather*}
			c, d \in H; \\
			\exists a, b \in G\colon c = f(a); d = f(d); \\
			a = f^{-1}(c); b = f^{-1}(d); \\
			f^{-1}(cd) = ab = f^{-1}(c)f^{-1}(d);
			\end{gather*}
			Тогда $f^{-1}$ "--- изоморфизм
	\end{enumerate}   
\end{proof}		

\begin{conseq}
    \begin{enumerate}
    \item  $G \cong G$ (взяли $id_G$).
    \item  $G \cong H \Ra H \cong G$
    \item  $G \cong H$, $H \cong K$ $\Ra$ $G \cong K$
    \end{enumerate}

\end{conseq}

\begin{exmp}
Пусть $G=\left< \mathbb{R}, +\right>$, $H=\left< \mathbb{R}_+, \cdot\right>$, тогда $x \ra e^x$ "--- изоморфизм.
\end{exmp}

\section{Циклические группы. Теорема о классификации циклических групп с точностью до изоморфности}
\begin{Def}
	Группа $G$ "--- циклическая группа, если она порождена одним элементом (то есть замыкание этого элемента
	в группе порождает группу).
	Обозначение: $G = \left<a\right>$
\end{Def}

Примеры:
\begin{enumerate}
	\item $\left<\mathbb{Z}, +\right> = \left<1\right>$
	\item ${e} = \left<e\right>$ "--- тривиальная группа
	\item $n \in \mathbb{N}$, $C_n$ "--- группа поворотов на угол $\frac{2\pi k}{n}$, $C_n = \left<\frac{2\pi}{n}\right>$
\end{enumerate}

\begin{Def}{Степени}
\begin{gather*}
i > 0: a^i = \underbrace{aa \ldots a}_{i} \\
a^{-i} = \underbrace{a^{-1}a^{-1} \ldots a^{-1}}_{i} \\
a^{m + n} = a^ma^n
\end{gather*}

\end{Def}
\begin{theorem}{Лемма о делении с остатком}
	$\forall n \in \mathbb{N}, a \in \mathbb{Z} \colon \exists!$ (существуют и единственны) $q, r \in \mathbb{Z} \colon 0 \leqslant r < n \colon a = qn + r$
\end{theorem}
\begin{proof}
	Существование:
	\begin{gather*}
	q = \left\lfloor \frac{a}{n} \right\rfloor\\
	nq \leqslant a < nq + n \\
	r = a - nq
	\end{gather*}
	Единственность (от противного):
	\begin{gather*}
	a = nq_1 + r_1 = nq_2 + r_2; 0 \leqslant r1, r2 < n;\\
	0 = n * (q_1 - q_2) + (r_1 - r_2), n|q_1 - q_2| = |r_1 - r_2|
	\end{gather*}

	Так как $n \geqslant 1$, то левая часть хотя бы $n$, а правая "--- строго меньше $n$. Противоречие.
\end{proof}

\begin{theorem}{Теорема о классификации циклических групп с точностью до изоморфности}
	Всякая циклическая группа изоморфна либо $(\mathbb{Z}, +)$, либо $C_n$, $n \in \mathbb{N}$
\end{theorem}
\begin{proof}
	$G = \left<a\right>$. Рассмотрим отображение $i \ra a^i$, $i \in \mathbb{Z}$.
	\begin{itemize}
	\item Если все элементы различны, то получили изоморфизм $(i + j \lra a^{i+j} = a^i a^j)$
	\item В противном случае $\exists i > j\colon a^i = a^j \Ra a^{i-j}=e$.
	      Пусть $n$ "--- наименьшее число такое, что $a^n = e$. Тогда $G = \{e = a^0, a^1, a^2, \dots, a^{n-1} \}$, $G \lra C_n$, $a^k a^l = a^{(k+l) \bmod n}$
	\end{itemize}
\end{proof}

\begin{Def}
	$G$ "--- группа. Если $G$ конечна, то порядком $G$ называют число элементов в ней, иначе порядок равен $\infty$
\end{Def}

\begin{Def}
	$a \in G$ и $n \in \mathbb{N}$ "--- минимальное число такое, что $a^n = e$. Тогда $n$ - порядок элемента $a$. Если такого элемента нет, то порядок $a$ равен $\infty$
\end{Def}
\begin{Rem}
	Альтернативное определение: порядок элемента равен порядку циклической группы, порожденной этим элементом
\end{Rem}

\section{Классы смежности}

Есть группа $G$ и ее подгруппа $H$. Введем отношение $\sim$: $a\sim b \iff a^{-1}b \in H$.
Докажем, что это отношение "--- отношение эквивалентности:

\begin{gather*}
a^{-1}a = e \in H \Ra a \sim a; \\
a \sim b \Lra a^{-1}b \in H \Lra (a^{-1}b)^{-1} = b^{-1}a \in H \Lra b \sim a; \\
\left.
\begin{align*}
  a \sim b \Lra a^{-1}b \in H; \\
  b \sim c \Lra b^{-1}c \in H; 
\end{align*}
\right\} \Ra a^{-1}c = a^{-1}bb^{-1}c \in H \Ra a \sim c;
\end{gather*}

Рассмотрим класс эквивалентности элемента $a$:
\begin{align*}
[a] &= \{b \in G \mid a^{-1}b \in H\} = \\
    &= \{b \in G \mid \exists h \in H \colon a^{-1}b = h\} = \\
    &= \{b \in G \mid \exists h \in H \colon b = ah\}
\end{align*}

Словами: класс эквивалентности $a$ "--- это образ отображения $f: H \to G$, где $f(h)=ah$.

\begin{Def}
	$aH = [a]$ "--- левый класс смежности по подгруппе $H$, $Ha$ "--- правый класс смежности по подгруппе $H$ (определяется симметрично).
\end{Def}
\begin{Rem}
	Левые классы смежности для разных элементов или не пересекаются, или совпадают, так как $aH$ "--- класс эквивалентности. Аналогично для
	правых классов.
\end{Rem}

\begin{exmp}
$G = S_3$ "--- группа перестановок из трех элементов. Пусть $H = \left<(12)\right> = \{e, (12)\}$.
Тогда $(13)H = \{(13), (123)\}, H(13) = \{(13), (132)\}$.
\end{exmp}

\section{Теорема Лагранжа и следствия из нее}
\begin{lemma}
	$H \subset G$, $f: H \ra aH$, $f(h)=ah$, $H$ "--- подгруппа. Тогда $f$ "--- биекция. В частности, из этого будет следовать, что $|H| = |aH|$, то есть мощности всех левых классов смежности равны друг другу
\end{lemma} 
\begin{proof}
	Заметим, что это отображение "--- сюрьекция по определению $aH$ (у каждого элемента есть прообраз). Докажем, что это инъекция.

	От противного: пусть $ah_1 = ah_2$, домножим слева на $a^{-1}$, получим $h_1 = h_2$.

	Таким образом, $f$ "--- биекция.
\end{proof}

\begin{Def}
	Число левых классов смежности по $H$ называется индексом $H$ в $G$. Обозначение: $[G:H]$
\end{Def}

\begin{theorem}{Теорема Лагранжа}
	Пусть $G$ - конечная группа, возьмём элемент $a$ и его левые классы смежности $aH_{\alpha}$.
	Очевидно, что $G = \bigcup aH_{\alpha}$ и $H_{\alpha_1} \cap H_{\alpha_2} = \emptyset$.
	Тогда $|G| = [G : H] \cdot |H|$
\end{theorem}
\begin{proof}
	Все эти классы имеют одинаковую мощность, равную $|H|$ (по лемме). Тогда $|G| = [G : H] \cdot |H|$, так как эти классы не пересекаются.
\end{proof}
\begin{conseq}
	Количество правых и левых классов смежности одинаково (достаточно провести аналогичные действия для правых классов смежности)
\end{conseq}
\begin{conseq}
	Порядок любой подгруппы делит порядок конечной группы.
\end{conseq}
\begin{conseq}
	Порядок любого элемента делит порядок конечной группы (рассмотрим циклическую подгруппу, порожденную этим элементом)
\end{conseq}
\begin{conseq}
	Группа порядка $p$ (где $p$ "--- простое число) циклична, так как порядок любого элемента равен либо 1 ($e$), либо $p$ (все остальные), а тогда все элементы кроме $e$ порождают всю группу порядка $p$.
\end{conseq}

\section{Предел монотонной последовательности}

\begin{Def}
$(x_n)$ нестрого монотонно возрастает, если $$x_1 \leqslant x_2 \leqslant x_3 \leqslant \cdots$$

$(x_n)$ строго монотонно возрастает, если $$x_1 < x_2 < x_3 < \cdots$$

$(x_n)$ нестрого монотонно убывает, если $$x_1 \geqslant x_2 \geqslant x_3 \geqslant \cdots$$

$(x_n)$ строго монотонно убывает, если $$x_1 > x_2 > x_3 > \cdots$$
\end{Def}

\begin{theorem}{Теорема Вейерштрасса}
Монотонная последовательность ограниченна тогда и только тогда, когда имеет предел.
\end{theorem}
\begin{proof}
$\La$: Очевидно.

$\Ra$: Пусть $(x_n)$ возрастает. Она ограниченна, значит есть супремум. Докажем, что это и есть предел. Возьмём $\epsilon > 0$.
$$a = \sup \{x_n\} \Ra \exists x_k\colon x_k > x - \epsilon \Ra a - \epsilon < x_k \leqslant x_{k+1} \leqslant \ldots \leqslant a$$
Тогда $$\forall n \geqslant k\: |x_n - a| < \epsilon$$
\end{proof}
\section{Симетрическая группа. Порождение симметрической группы транспозициями}

\begin{Def}
$S_n$ "--- биекции на $\{1, \dotsc, n\}$.
$S_n$ "--- симметрическая группа (группа перестановок) степени $n$.
Произведение двух перестановок "--- это их композиция: $\sigma \cdot \tau = \sigma \circ \tau$,
то есть первой выполняется биекция $\tau$.
\end{Def}

\begin{exmp}

\begin{tabular}{ c c }
  \begin{tabular}{|c|c|c|c|c|c|c|}
  \hline
  $x$       & 1 & 2 & 3 & 4 & 5 & 6 \\ \hline
  $\tau(x)$ & 5 & 1 & 6 & 4 & 2 & 3 \\ \hline
  \end{tabular}
  &
  \begin{tabular}{|c|c|c|c|c|c|c|}
  \hline
  $x$       & 1 & 2 & 3 & 4 & 5 & 6 \\ \hline
  $\sigma(x)$ & 2 & 4 & 5 & 1 & 3 & 6 \\ \hline
  \end{tabular}
  \\
  \rule{0pt}{4ex}
  \begin{tabular}{|c|c|c|c|c|c|c|}
  \hline
  $x$                    & 1 & 2 & 3 & 4 & 5 & 6 \\ \hline
  $(\tau\cdot\sigma)(x)$ & 1 & 4 & 2 & 5 & 6 & 3 \\ \hline
  \end{tabular}
  &
  \begin{tabular}{|c|c|c|c|c|c|c|}
  \hline
  $x$                    & 1 & 2 & 3 & 4 & 5 & 6 \\ \hline
  $(\sigma\cdot\tau)(x)$ & 3 & 2 & 6 & 1 & 4 & 5 \\ \hline
  \end{tabular}
\end{tabular}
\end{exmp}

\begin{Def}
	Цикл $(i_1, i_2, \dotsc, i_k)$ "--- это перестановка $\sigma$ такая, что:
	\begin{gather*}
	\sigma(i_1) = i_2; \\ 
	\sigma(i_2) = i_3; \\
	\dotsc \\
	\sigma(i_k) = i_1; \\
	\sigma(j) = j, j \notin \{i_1, \dotsc, i_k\}
	\end{gather*}
	То есть цикл переводит $i_1\to i_2 \to \dotsc \to i_k \to i_1$ и оставляет остальные элементы на месте.
	$k$ называется длиной цикла.
\end{Def}

\begin{Rem}
    \[(i_1, i_2, \dotsc, i_k) = (i_2, i_3, \dotsc, i_k) = \dotsc = (i_s, i_{s+1}, \dotsc, i_k, i_1, \dotsc, i_{s - 1})\]
\end{Rem}
\begin{Rem}
    Порядок цикла длины $k$ равен $k$. В частности, $(i_1, i_2, \dotsc, i_k)^k = id$.
\end{Rem}

\begin{Def}
	Транспозиция "--- цикл длины 2.
	$(i, j)$ "--- $i$ и $j$ меняются местами, остальные остаются на месте.
\end{Def}

\begin{Def}
	Пусть $(i_1, i_2, \dotsc, i_k)$, $(j_1, j_2, \dotsc, j_s)$ "--- циклы.
	Эти циклы называются незацепляющимися (непересекающимися), если $\{i_1, i_2, \dotsc, i_k\} \bigcap \{j_1, j_2, \dotsc, j_s\} = \varnothing$.
\end{Def}

\begin{Rem}
    Если $\sigma, \tau$ "--- непересекающиеся циклы, то $\sigma\tau = \tau\sigma$, легко перемножать.
\end{Rem}

В билете 17 показывается, что любая перестановка представляется как произведение циклов. Да, тут
как бы идёт 17 билет.

\begin{theorem}{$S_n$ порождается транспозициями}
\end{theorem}
\begin{proof}
Покажем, что любой цикл $(i_1, i_2, \dotsc, i_k)$ есть произведение транспозиций $(i_1 i_2)(i_2 i_3)\dotsc(i_{k - 1}i_k)$.
Заметим, что если применить это произведение к перестановке (вот просто взять и последовательно применить справа налео все транспозиции аккуратно), то мы и получим тот же самый результат, как и от применения исходного цикла.

Осталось лишь заметить, что каждая из остальных $n - k$ точек любо неподвижная, либо также лежит на каком-то цикле "--- а бить циклы на произведение транспозиций мы только что научились.
\end{proof}

\section{Четность перестановки. Теорема об изменении четности перестановки при умножении на транспозицию. Следствия из неё}

\begin{Def}
Пусть $\sigma \in S_n$ и $i, j = 1..n$. Тогда $i$ и $j$ образуют инверсию относительно $\sigma$, если $i < j$, а $\sigma(i) > \sigma(j)$.
\end{Def}
\begin{Def}
$Inv(\sigma)$ "--- множество всех инверсий относительно $\sigma$.
\end{Def}
\begin{Def}
$\sigma$ "--- четная, если $|Inv(\sigma)|$ четно.
\end{Def}
\begin{Def}
$\sigma$ "--- нечетная, если $|Inv(\sigma)|$ нечетно.
\end{Def}

\begin{theorem}{}
$\sigma \in S_n, \tau = (i~j)$ "--- транспозиция. Тогда $\sigma$ и $\sigma\tau$ имеют различную четность.
\end{theorem}
\begin{proof}
Не умаляя общности, будем считать, что $i < j$. Тогда
\begin{gather*}
\sigma=\left(
\begin{matrix}
1 & \cdots & i & \cdots & j & \cdots & n \\
\sigma_1 & \cdots & \sigma_i & \cdots & \sigma_j & \cdots & \sigma_n \\
\end{matrix}
\right) \\
\sigma\tau=\left(
\begin{matrix}
1 & \cdots & i & \cdots & j & \cdots & n \\
\sigma_1 & \cdots & \sigma_j & \cdots & \sigma_i & \cdots & \sigma_n \\
\end{matrix}
\right)
\end{gather*}

$$
	\begin{tabular}{|c|c|c|c|}
	\hline
	\multicolumn{2}{|c|}{$\sigma$} & \multicolumn{2}{c|}{$\sigma\tau$}\\
	\hline
	$(k~l), \{k, l\} \bigcap \{i, j\} = \varnothing$ & есть инверсия & $(k~l), \{k, l\} \bigcap \{i, j\} = \varnothing$ & есть инверсия\\
	\hline
	$(k~l), \{k, l\} \bigcap \{i, j\} = \varnothing$ & есть инверсия & $(k~l), \{k, l\} \bigcap \{i, j\} = \varnothing$ & есть инверсия\\
	\hline
	$(k~i), k < i$ & есть & $(k~j), k < i$ & есть\\
	\hline
	$(k~i), k < i$ & нет & $(k~j), k < i$ & нет\\
	\hline
	$(k~i), k > j$ & есть & $(k~j), k > j$ & есть\\

	\hline
	$(k~j), k < i$ & есть & $(k~i), k < i$ & есть\\
	\hline
	$(k~j), k < i$ & нет & $(k~i), k < i$ & нет\\
	\hline
	$(k~j), k > j$ & есть & $(k~i), k > j$ & есть\\
	\hline
	$(k~j), k > j$ & нет & $(k~i), k > j$ & нет\\
	
	\hline
	\multicolumn{4}{|c|}{$\sigma(\sigma(i)\sigma(k)\sigma(j))$}\\
	\hline
	\multicolumn{4}{|c|}{$\sigma\tau(\sigma(j)\sigma(k)\sigma(i))$}\\
	\hline
	$(k~i), i < k < j$ & есть & $(k~j), i < k < j$ & нет\\
	\hline
	$(k~i), i < k < j$ & нет & $(k~j), i < k < j$ & есть\\
	\hline
	$(k~j), i < k < j$ & есть & $(k~i), i < k < j$ & нет\\
	\hline
	$(k~j), i < k < j$ & нет & $(k~i), i < k < j$ & есть\\

	\hline
	(i~j) & есть & (i~j) & нет\\
	\hline
	(i~j) & нет & (i~j) & есть\\
	\hline

	\end{tabular}
$$	
Как глубокоуважаемый читатель уже догадался, самое интересное здесь "--- это последние 6 строк.
\begin{gather*}
r = |\{k \mid i < k < j \land (i~k)\text{ образует инверсию}\}| \\
s = |\{k \mid i < k < j \land (k~j)\text{ образует инверсию}\}| \\
|Inv(\sigma\tau)| = |Inv(\sigma)| - r + (j - i - 1 - r) - s + (j - i - 1 - s) \pm 1
\end{gather*}
Таким обазом, чётность изменилась.
\end{proof}

Следствия:
\begin{enumerate}
\item Пусть $\sigma = \prod_{j = 1}^r \tau_j$, где $\tau_j$ "--- транспозиция.
Тогда $\sigma$ четна (нечетна) $\iff$ число сомножителей четно (нечетно).
  \begin{proof}
  \begin{itemize}
  \item $\Leftarrow$: $id$ четна. При каждом добавлении $\tau_j$ четность меняется.
  \item $\Rightarrow$: $\sigma$ четна, $r$ "--- число транспозиций, и если у него другая четность, то приходим к противоречию.
  \end{itemize}
  \end{proof}
\item $\sigma \in S_n$, $\tau$ "--- транспозиция. $\sigma$ и $\tau\sigma$ имеют различную четность. (из первого следствия)
\item При перемножении двух перестановок их четность меняется так же, как при суммировании их четностей как чисел.
\item Множество четных перестановок "--- подгруппа $S_n$. Называется <<знакопеременная группа>>, обозначается $A_n$.
  \begin{proof}
  \begin{enumerate}
  \item Непусто ($id$ четна)
  \item Замкнуто (по третьему следствию)
  \item Обратный элемент к $\sigma = \prod_{j = 1}^{2r} \tau_j$ "--- это $\sigma^{-1} = \prod_{j = 1}^{2r} \tau_{2r + 1 - j}$ 
  \end{enumerate}
  \end{proof}
\end{enumerate}

\section{Кольца, тела, поля}
\begin{Def}
Есть $A \neq \varnothing$, на нём заданы две бинарные операции:
\begin{enumerate}
\item $+: A \times A \to A$ (сложение)
\item $\cdot: A \times A \to A$ (умножение)
\end{enumerate}

Определим интересные свойства, которыми такая структура $\left<A, +, \cdot\right>$ может обладать:

\begin{enumerate}
\item Ассоциативнсть сложения:
	\[ \forall a, b, c \in A \colon (a + b) + c = a + (b + c) \]
\item Существование нейтрального элемента по сложению (обозначается $0$):
	\[ \exists 0 \in A, \forall a \in A \colon a + 0 = 0 + a = a \]
\item Существование обратного элемента по сложению (обозначается $-a$):
	\[ \forall a \in A, \exists (-a) \in A \colon a + (-a) = (-a) + a = 0 \]
\item Коммутативность сложения:
	\[ \forall a, b \in A \colon a + b = b + a \]
\item Ассоциативность умножения:
	\[ \forall a, b, c \in A \colon (a \cdot b) \cdot c = a \cdot (b \cdot c) \]
\item Коммутативность умножения:
	\[ \forall a, b \in A \colon a \cdot b = b \cdot a \]
\item Существование нейтрального элемента по умножению (обозначается $1$):
	\[ \exists 1 \in A, \forall a \in A \colon a \cdot 1 = 1 \cdot a = a \]
\item Существование обратного элемента по умножению (обозначается $a^{-1}$):
 	\[ \forall a \in A \setminus \{0\}, \exists a^{-1} \in A \colon a \cdot a^{-1} = a^{-1} \cdot a = 1 \]
 	Также иногда говорят про левый обратный $a^{-1}_l$ ($a^{-1}_la = 1$) и правый обратный $a^{-1}_r$ ($aa^{-1}_r=1$),
 	но если существуют оба, то они совпадают (см. свойства).
\item
    \begin{itemize}
	\item Дистрибутивность слева: $ \forall a, b, c \in A \colon a \cdot (b + c) = a \cdot b + a \cdot c $\\
	\item Дистрибутивность справа: $ \forall a, b, c \in A \colon (a + b) \cdot c = a \cdot c + b \cdot c $\\
	\end{itemize}
	Просто <<дистрибутивность>> "--- дистрибутивность и слева, и справа.
\end{enumerate}
\end{Def}

\begin{Def}
	Кольцо "--- тройка $\left<A, +, \cdot\right>$, удовлетворяющая свойствам 1--5, 9 (a, b)
\end{Def}
\begin{Def}
	Кольцо, в котором выполнено свойство 6 "--- коммутативное кольцо
\end{Def}
\begin{Def}
	Кольцо, в котором выполнено свойство 7 "--- кольцо с единицей
\end{Def}
\begin{Def}
	Тело "--- кольцо с $1$, в котором $1 \neq 0$ и выполнена аксиома 8 (то есть всё, кроме 6)
\end{Def}
\begin{Def}
	Поле "--- коммутативное кольцо с $1$, в котором $1 \neq 0$ и выполнена аксиома 8 (т.е. все 9 аксиом)
\end{Def}

\begin{Rem}
	Иногда кольца, для которых выполнены аксиомы 1--4 и 9, называют ассоциативными кольцами
\end{Rem}

\begin{Rem}
Если $\left<A, +, \cdot\right>$ "--- кольцо, то $\left<A, +\right>$ "--- абелева группа
\end{Rem}

\textbf{ Примеры: }
\begin{itemize}
\item $\Z$ "--- коммутативное кольцо с $1$, но $2$ не имеет обратного в $\Z$ $\Ra$ не поле
\item $\mathbb{N}$ "--- не кольцо
\item $2\Z$ (все чётные целые числа) "--- кольцо без $1$
\item $\Q, \R$ - поля
\end{itemize}

\subsection{Простейшие свойства колец}

\begin{enumerate}
\item $0$ единственен, так как $0 = 0 + \cdot 0' = 0'$
\item $-a$ единствененю
	\begin{proof}
	Пусть существуют два ($b$ и $c$), являющихся обратными по сложению к $a$:
	\begin{gather*}
		a + b = b + a = a + c = c + a = 0; \\
		a + b = a + c = 0; \\
		(b + a) + b = (b + a) + c = 0; \\
		0 + b = 0 + c ; \\
		b = c; 
	\end{gather*}
	\end{proof}
\item $1$ единственна (если существует), так как $1 = 1 \cdot 1' = 1'$
\item Если у $a$ есть и левый, и правый обратный по умножению, то они совпадают: \[a^{-1}_l = a^{-1}_l(aa^{-1}_r) = (a^{-1}_la)a^{-1}_r = a^{-1}_r\]
\item Если в кольце с 1 у элемента $a$ есть 2 левых обратных, то левых обратных к $a$ бесконечно много (упражнение).
Подсказка: построим инъективное отображение множества левых обратных в своё подмножество. Говорят, это может
быть сложно даже в качестве задачи на пятёрку, потому что надо творчески построить отображение.
	
\item $0 \cdot a = a \cdot 0 = 0 $
	\begin{proof}
		\begin{align*}
		a \cdot 0 + a \cdot 0 &= a(0 + 0) = a\cdot0 & \text{добавим} -(a\cdot0); \\
		a \cdot 0 + a \cdot 0 + (-(a \cdot 0)) &= a \cdot 0 + (-(a \cdot 0)); \\
		a \cdot 0 &= 0; \\
		\end{align*}
		Аналогичным образом с другой стороны.
	\end{proof}
\item $ a(-b) = (-a)b = -(ab)$
	\begin{proof}
		\begin{gather*}
		a + (-a) = 0; \\
		ab + (-a)b = (a + (-a))b = 0 \cdot b = 0; \\
		-(ab) + ab + (-a)b = -(ab) + 0; \\
		(-a)b = -(ab); \\
		\end{gather*}
		Аналогичным образом с другой стороны.
	\end{proof}
\item Если $0 = 1$, то $|A| = 1$, $A = \{0\}$:
	\begin{proof}
		Рассмотрим произвольный $a \in A$:
		\[a = 1 \cdot a = 0 \cdot a = 0\]
	\end{proof}
\end{enumerate}

\begin{Def}
Пусть $A$ "--- кольцо (тело, поле).
Тогда непустое $B \subset A$ "--- подкольцо (подтело, подполе), если является кольцом (телом, полем) относительно сужения операций на $B$
\end{Def}


\begin{Rem}
\begin{itemize}
	\item $A \supset B \neq \varnothing$ "--- подкольцо в $A$, если оно замкнуто относительно умножения, сложения, взятия обратного по сложению
	\item $B$ "--- подтело, если оно подкольцо и замкнуто по взятию обратного ненулевого элемента по умножению и содержит элементы, отличные от нуля, то есть:
	\begin{gather*}
	\forall a, b \in B \colon a + b \in B; \\
	\forall a \in B \colon (-a) \in B; \\
	\forall a, b \in B \colon ab \in B; \\
	\forall a \in (B \setminus \{0\}) \colon a^{-1} \in B; \\
	\exists a \in B \colon a \neq 0;
	\end{gather*}
\end{itemize}
\end{Rem}

\subsection{Гомоморфизмы колец}

\begin{Def}
	Пусть $A$, $B$ "--- кольца и есть отображение $f: A \to B$.
	$f$ "--- гомоморфизм, если:
	\begin{enumerate}	
	\item $\forall a_1, a_2 \in A \colon f(a_1 + a_2) = f(a_1) + f(a_2)$
	\item $\forall a_1, a_2 \in A \colon f(a_1a_2) = f(a_1)f(a_2)$
	\end{enumerate}
\end{Def}

\begin{Def}		
	$f$ "--- изоморфизм, если $f$ "--- и гомоморфизм, и биекция
\end{Def}

\begin{Def}		
	$A$ и $B$ изоморфны, если между ними сущесвует изоморфизм,
	обозначение: $A \cong B$.
\end{Def}
\begin{Rem}
	Если $f$ "--- гомоморфизм и $f(0_A)$ обратим по сложению, то $f(0_A) = 0_B$
\end{Rem}
\begin{proof}
	$f(0_A) = f(0_A + 0_A) = f(0_A) + f(0_A)$, добавили обратный по сложению к $f(0_A)$,
	получили $0_B = f(0_A)$.
\end{proof}

\begin{Rem}
	Если $f$ "--- гомоморфизм и $f(1_A)$ обратим в $B$ то $f(1_A)=1_B$
\end{Rem}
\begin{proof}
	$f(1_A) = f(1_A \cdot 1_A) = f(1_A)f(1_A)$, домножаем на обратный к $f(1_A)$,
	получаем $1_B = f(1_A)$.
\end{proof}
	
\subsection{Делимость в кольцах}

\begin{Def}
Пусть $A$ "--- кольцо и $a, b, c \in A$ таковы, что $c = ab$.
Тогда $a$ "--- левый делитель $c$, $b$ --- правый делитель $c$.
\end{Def}

\begin{Rem}
Любой элемент является и левым, и правым делителем нуля: $0 = a \cdot 0 = 0 \cdot a$.
\end{Rem}
    
\begin{Def}
	$a$, $b$ "--- нетривиальные делители нуля, eсли $0 = ab$, $a \neq 0$, $b \neq 0$.
\end{Def}
	
\begin{Def}	
	Область целостности "--- коммутативное кольцо с $1$, в котором отсутствуют нетривиальные делители нуля.
\end{Def}

\begin{Rem}
    В области целостности если $ab=0$, то либо $a=0$, либо $b=0$.
\end{Rem}
\begin{Rem}
	Поле "--- область целостности (потому что не-ноль всегда можно домножить на обратный).
\end{Rem}
	
\begin{theorem}{}
	$A$ "--- область целостности и $a \in A \setminus \{ 0 \}$. Тогда
	\[ab = ac \Ra b = c\]
\end{theorem}
	
\begin{proof}
	\begin{gather*}
	\underbrace{a}_{\neq 0}(b - c) = 0; \\
	a^{-1}a(b - c) = a^{-1} \cdot 0; \\
	(b - c) = 0; \\
	b - c + c = 0 + c; \\
	b = c; \\
	\end{gather*}
\end{proof}

\section{Мультипликативная группа кольца}

\begin{Def}	 
	Пусть $A$ - кольцо с 1.
	Введём множество $A^*$:
	\[A^* = \{ a \in A \mid \exists b \in A \colon ab = ba = 1 \}\]
	Тогда $\left<A^*, \cdot\right>$ "--- мультипликативная группа кольца\\
\end{Def}

\begin{theorem}{}
	$A^*$ - группа по умножению\\
\end{theorem}
\begin{proof}
	Существование единицы очевидно: $1_A \in A^*$, останется единицей и в $A^*$.
	Покажем замкнутость относительно операции умножения:
	\[a_1, a_2 \in A^* \iff \exists b_1, b_2 \in A \colon a_1b_1=b_1a_1=a_2b_2=b_2a_2 = 1\]
	Теперь покажем, что $a_1a_2$ тоже лежит в $A^*$ по определению:
	\begin{gather*}
	(a_1a_2)(b_1b_2) = a_1(a_2b_2)b_1 = a_1b_1 = 1; \\
	(b_2b_1)(a_1a_2) = b_2(b_1a_1)a_2 = b_2a_2 = 1;
	\end{gather*}
	Покажем замкнутость относительно взятия обратного:
	\[ a \in A^* \Ra \exists b \colon ab = ba = 1 \Ra b \in A^* \Ra a^{-1}_{A^*} = b \]
\end{proof}
	 
\textbf{ Примеры: }
\begin{itemize}
	\item $K$ "--- поле, тогда $K^* = K \setminus \{ 0 \}$
	\item $\Z^* = {-1, 1}$
\end{itemize}

\section{Кольца многочленов}

\begin{Def}
$A$ "--- коммутативное кольцо с $1$.
Введём кольцо многочленов от одной переменной $A[x]$ над кольцом $A$:
\begin{gather*}
A[x] = \{ \underbrace{a_0, a_1, a_2, \ldots, a_n, \ldots}_{\text{счётная последовательность}} \mid a_i \in A \land\text{почти все нули}\}
\end{gather*}
\end{Def}
		 
\begin{Rem}
<<Почти все>> "--- все, кроме конечного числа.
\end{Rem}
	 
\begin{Def}	
Сложение многочленов: 
\begin{gather*}
+\colon A[x] \times A[x] \to A[x] \\ 
(a_0, a_1, a_2, \ldots) + (b_0, b_1, b_2, \ldots) = (a_0 + b_0, a_1 + b_1, a_2 + b_2, \ldots)
\end{gather*}
\end{Def}
	
\begin{Rem}
Сложение тоже порождает многочлен: пусть начиная с позиции $n$ в $a_i$ все нули,
а начиная с позиции $m$ в $b_i$ все нули, тогда начиная с $\max(n, m)$ элементы $a_i+b_i$ все нули.
\end{Rem}
	 
\begin{Def}	
Умножение многочленов:
\begin{gather*}
\cdot\colon A[x] \times A[x] \to A[x] \\
(a_0, a_1, a_2, \ldots) \cdot (b_0, b_1, b_2, \ldots) = (c_0, c_1, c_2, \ldots) \\
c_n = \sum_{i=0}^n a_ib_{n - i} = \sum_{i + j = n} a_ib_j
\end{gather*}
\end{Def}
		 
\begin{Rem}
$\exists n, m\colon \forall i>n, \forall j>m \colon a_i = 0 \land b_j = 0 \Ra \forall k > n + m\colon c_k = 0 $
\end{Rem}
	
\begin{proof}
\begin{gather*}
c_k = \sum_{i=0}^k a_ib_{k - i} = \underbrace{\sum_{i=0}^n a_ib_{k - i}}_{0\leq i\leq n \Ra k-i \geq k-n \geq n+m-n = m \Ra b_{k - i} = 0 \Ra \sum = 0} +
\underbrace{\sum_{i=n+1}^k a_ib_{k-i}}_{i>n \Ra a_i = 0 \Ra \sum = 0} = 0
\end{gather*}
\end{proof}	
	
\begin{theorem}{}
$\left<A[x], +, \cdot\right>$ "--- коммутативное кольцо с $1$.
\end{theorem}

\begin{proof}
\begin{enumerate}
\item Аксиомы 1"---4 покомпонентно выполнены в $A$
\item $\exists 0 = (0, 0, 0, \ldots)$
\item $\exists 1 = (1, 0, 0, \ldots)$ \\
По определению операции умножения:
\begin{gather*}
(1, 0, 0, \ldots)(a_0, a_1, a_2, \ldots) = (a_0, \underbrace{0a_0 + 1a_1}_{=a_1}, \underbrace{0a_0 + 0a_1 + 1a_2}_{=a_2}, \ldots) = (a_0, a_1, a_2, \ldots)
\end{gather*}

\item Коммутативность:
$\beta = (b_0, b_1, ...)$, $\alpha = (a_0, a_1, ...)$. $\alpha\beta = \beta\alpha$.
\begin{gather*}
\alpha\beta = (c_0, c_1, ...) \Ra c_k = \sum_{i = 0}^k a_ib_{k - i} \\
\beta\alpha = (d_0, d_1, ...) \Ra d_k = \sum_{i = 0}^k b_ia_{k - i} = \sum_{j = 0}^k b_{k - j}a_j = \\
j = k - i, i = k - j \\
= \sum_{i = 0}^k b_{k - i}a_i \xlongequal{\text{$A$ коммутативно}} \sum_{i = 0}^k a_ib_{k - i} = c_k
\end{gather*}
\item Дистрибутивность "--- упражнение
\item Ассоциативность: 
$\alpha = (a_0, a_1, ...)$, $\beta = (b_0, b_1, ...)$, $\gamma = (c_0, c_1, ...)$. $\alpha\beta = d$, $(\alpha\beta)\gamma = e$. $\beta\gamma = f$, $\alpha(\beta\gamma) = g$. $e=g$.
\begin{gather*}
e_k = \sum_{i = 0}^k f_ic_{k-i} = \sum_{i = 0}^k \left(\sum_{j = 0}^i a_jb_{i-j}\right)c_{k-i} = \\
 = \sum_{j = 0}^k \left(\sum_{i = j}^k a_jb_{i - j}c_{k - i}\right) = \sum_{j = 0}^k a_j \left(\sum_{i = j}^k b_{i - j}c_{k - i}\right) = \\
l = i - j \\
 = \sum_{j = 0}^k a_j\left(\sum_{l = 0}^{k - j} b_lc_{k - l - j}\right) = \sum_{j = 0}^k a_jf_{k-j} = g_k
\end{gather*}
\end{enumerate}
\end{proof}
\section{Степень многочлена}	
\begin{Rem}
	Альтернативное обозначение многочленов (тут $a \in A$, а $x$ "--- переменная):
	\begin{align*}
	a &= (a, 0, 0, \dots); \\
	x &= (0, 1, 0, \dots); \\
	x^i &= (0, \dots, \underbrace{1}_{i-\text{я позиция}}, \dots); \\
	(a_0, a_1, a_2, \dots) &= (a_0, 0, 0, \dots) + (0, a_1, 0, 0, \dots) + \dots =  \\
    &= (a_0, 0, 0, \dots) \cdot (1, 0, 0, \dots) + (0, a_1, 0, 0, \dots) \cdot (0, 1, 0, 0, \dots) + \dots = \\
    &= a_0 + a_1x + a_2x^2 + \dots + a_nx^n
	\end{align*}
	Последняя строка "--- общий способ записать многочлен.
\end{Rem}
	
\begin{Def}
	Альтернативное определение кольца многочленов:
	\[A[x] = \lbrace a_0 + a_1x + \dots + a_nx^n \mid n \in \N \cup \lbrace 0 \rbrace \wedge a_i \in A \rbrace\]
\end{Def}
	
\begin{Def}
	Пусть $f = a_0 + a_1x + \dots + a_nx^n$, $a_n \neq 0$ и $f \neq 0$.
	Тогда $n$ "--- степень многочлена $f$, обозначается $n = \deg f$.
	Если $f = 0$, то положим $\deg f = -\infty$.
\end{Def}
	 
\begin{theorem}{}
	\begin{enumerate}
	\item $deg(f + g) \leq \max(\deg f, \deg g)$
	\item $deg(fg)  \leq \deg f + \deg g$
	\begin{Rem}
		Если $A$ - область целостности, то $\deg (fg) = \deg f + \deg g$
	\end{Rem}
	\end{enumerate}		
\end{theorem}			 

\begin{proof}
	\begin{enumerate}
	\item Следует из доказательства замкнутости относительно сложения (просто сложили и посмотрели на определение степени):
	\begin{gather*}
	f = a_0 + \dots + a_nx^n \wedge a_n \neq 0;\\
	g = b_0 + \dots + b_mx^m \wedge a_m \neq 0;
	\end{gather*}
	\item Раскрыли скобки по дистрибутивности кольца многочленов, обнаружили, что там не может возникнуть
	      $x^k$, где $k > \deg f + \deg g$.
	\item Для области целостности. Рассмотрим $a_n\neq 0$ и $b_m \neq 0$ и пусть $fg$ имеет вид
	\[ c_0x^0 + c_1x^1 + \dots + c_{n+m}x^{n+m} \]
	Так как $c_{n+m}=a_nb_m$, то $c_{n+m} \neq 0$ (иначе есть нетривиальные делители нуля), то
	есть $\deg (fg) = \deg f + \deg g$.
	Если же один из многочленов ноль, то его степень равна $-\infty$, сумма степеней тоже равна
	$-\infty$, а результат ноль $\Ra$ его степень равна $-\infty$.
	\end{enumerate}
\end{proof}	 

\begin{conseq}
	Если $A$ "--- область целостности, то и $A[x]$ "--- область целостности
\end{conseq}
\begin{proof}
	Пусть $f, g \neq 0$, тогда:
	\[
	\deg f, \deg g \geq 0
	\Ra \deg (fg) \geq 0
	\Ra  fg \neq 0	
	\]
\end{proof}
	 
\begin{exmp}
	Возьмём $A = \Z / 4 \Z$ (кольцо остатков по модулю 4, не область целостности)
	положим $f = 2x$, $g = 2x^2$, $fg = 4x^2 = 0$, то есть $A[x]$ "--- не область целостности.
\end{exmp} 
	 
\begin{conseq}
	$A$ "--- область целостности $\Ra (A[x])^* = A^*$ 
\end{conseq}	 
\begin{proof}
\begin{itemize}
\item $\subset$: пусть $fg=1$ (при этом $f,g \neq 0$, иначе точно необратимо), тогда:
	\begin{equation*}
	\left.
		\begin{aligned}
		\deg f + \deg g &= 0 \\
		\deg f &\ge 0 \\
		\deg g &\ge 0
		\end{aligned}
	\right\} \Ra \deg f = \deg g = 0
	\end{equation*}
\item $\supset$: если элемент обратим в кольце $A$, то он обратим и в кольце многочленов
\end{itemize}
\end{proof}

\section{Теорема о деление с остатком}
\begin{theorem}{}
	$A$ "--- коммутативное кольцо с $1$б
	$f, g \in A[x]$ и
	\[f = a_0 + a_1x + a_2x^2 + \dots + a_nx^n\]
	
	где $n = \deg f$ и $a_n \in A^*$. Тогда $\exists q, r \in A[x] \colon  g = qf + r, \deg r < \deg f$
	(разделили $g$ на $f$ с остатком).
\end{theorem}

\begin{Rem}
	Если $A$ "--- область целостности, то такое представление единственно
\end{Rem}

\begin{proof}
\begin{itemize}
\item \textbf{Существование}:
	Индукция по $m = \deg g$.
	\begin{itemize}
	\item
	\textbf{База:} $m < n$. Тогда положим $q=0, r=g$.
	\item
	\textbf{Переход:} доказали для всех многочленов $\deg g < m$, докажем для $m$
	\begin{gather*}
	g = b_mx^m + \dots + b_0; \\
	g_1 = g - b_m a_n^{-1}x^{m-n}f;
	\end{gather*}
	коэффицент при $x^m$ в $g_1$: $b_m - b_m a_n^{-1} a_n = 0 \Ra \deg g_1 < m$\\
	по предположению индукции $g_1 = fq_1 + r_1, \deg r_1	< \deg f $, тогда разделим $g_1$ на $f$ с остатком и положим:
	\begin{gather*}
		r = r_1; \\
		q = q_1 + b_m a_n^{-1} x^{m-n}; \\
		g = fq	+ r;
	\end{gather*}
	\end{itemize}

\item	
	\textbf{Единственность:}
	Знаем, что $A$ "--- область целостности. Пусть представление не единственно.

	\begin{gather*}
	g = fq + r = f \tilde{q} + \tilde{r}; \\
	\deg r, \deg \tilde{r} < f; \\
	f (q - \tilde{q}) = \tilde{r} - r;
	\end{gather*}

	Если $q - \tilde{q} \neq 0$, то степень левого многочлена $\geq \deg f$ и степень правого $< \deg f$, а
	это невозможно. $\Ra$
	$q -\tilde{q} = 0, r - \tilde{r} = 0 \Ra q = \tilde{q}, r = \tilde{r}$
\end{itemize}
\end{proof}

\begin{Rem}
	Условие обратимости старших коэффицентов существенно, пример:
	$A = \Z$, $f = 2x, g = x^2 + 1$. Разложения в $\Z[x]$ не существует, а вот в $\R[x]$ существует
	и единственно: $g=0.5x \cdot f + 1$.
\end{Rem}

\section{Теорема Безу}

\begin{Def}{\bf Значение многочлена в точке}
Пусть есть $A \subset B$, и $B$ "---  коммутативное кольцо с 1 (будем уметь подставлять
в многочлены из $\Z[x]$ не только целые числа). Есть многочлен:
\[f \in A[x], f = a_nx^n + \ldots + a_0, c \in B\]
Тогда значение многочлена в точке $c$ есть
\[f(c) = a_nc^n + \ldots + a_0 = \sum_{k = 0}^{n}a_kc^k\]
\end{Def}

\begin{Rem}
Таким образом мы многочленом $f \in A[x]$ задали отображение
$\tilde f : B \to B$, которое переводит $c \to f(c)$. Имеются свойства:
\begin{enumerate}
\item $(f+g)(c)=\tilde f(c) + \tilde g(c)$
\item $(f\cdot g)(c)=\tilde f(c) \cdot \tilde g(c)$
\end{enumerate}
\end{Rem}

\begin{theorem}{}
\textbf{(Безу)}
	Пусть $A$ "--- коммутативное кольцо с $1$, $c \in A$ и $f \in A[x]$. Тогда:
	\[\exists q \in A[x] \colon f(x) = (x - c)q(x) + f(c)\]
\end{theorem}

\begin{proof}
	Рассмотрим $x - c$, по теореме о делении многочленов с остатком получаем:
	\begin{gather*}
	f(x) = (x - c)q(x) + r_0; \\
	\deg r_0 < \deg (x - c) = 1; \\
	\Ra r = r_0 \in A; \\
	f(x) = (x - c)q(x) + r_0; \\
	f(c) = (c - c)q(c) + r_0; \\
	f(c) = r_0; \\
	\Ra f(x) = (x - c)q(x) + r_0; \\
	\end{gather*}
\end{proof}

\begin{Def}
	$A$, $B$ "--- коммутативные кольца с 1 и $A \subseteq B$; $f \in A[x]$. Тогда
	$c \in B$ "--- корень $f$, если $f(c) = 0$.
\end{Def}

\begin{conseq}
	$c$ "--- корень $\Lra$ $(x - c) \mid f$\\
\end{conseq}
\begin{proof}
	\begin{itemize}
	\item $\La$: $f(x) = (x - c)g(x) \Ra f(c) = (c-c)g(c) = 0$ $\Ra$ $c$ - корень
	\item $\Ra$: $f(c) = 0 \Ra$ теорема Безу $\Ra f(x) = (x-c)g(x) + f(c) = (x-c)g(x) \Ra (x-c) | f$
	\end{itemize}
\end{proof}

\section{Характеристика кольца. Теорема о характеристике поля}
\begin{Def}
	Характеристика кольца $A$ с единицей "--- это наименьшее $n > 0$ такое, что $\underbrace{1+\dots+1}_{n} = 0$.
	Обозначается $\Char A = n$.
	Если такого $n$ нет, то считается, что $\Char A = 0$.
\end{Def}

\begin{exmp}
	$\Char \Z = \Char \Q = \Char \R = 0$
\end{exmp}
\begin{exmp}
	Пусть $\mathbb{F}_2$ и $\mathbb{F}_3$ "--- поля из 2 и 3 элементов, соответственно. Тогда:
	\begin{gather*}
	\Char \mathbb{F}_2 = 2 \\
	\Char \mathbb{F}_3 = 3
	\end{gather*}
\end{exmp}

\begin{theorem} 
	Если $A$ "--- поле, то $\Char A$ либо $0$, либо простое число.
\end{theorem}
\begin{proof}
	Два случая:
	\begin{enumerate}
	\item $\forall n > 0 \colon \underbrace{1 + \dots + 1}_{n} \neq 0 \Ra \mathrm{char} A = 0$
	\item $\Char A > 0 \colon \underbrace{1 + \dots + 1}_{n} = 0$. Тогда $n > 1$, так как в поле $1 \neq 0$.
	От противного: пусть $n = ab, 1 < a, b < n$.
	Рассмотрим $\underbrace{1+\dots+1}_a = A$ и $\underbrace{1+\dots+1}_b = B$, по дистрибутивности
	знаем, что $AB=0$. Так как поле "--- область целостности, то $A=0$ или $B=0$, то
	есть меньшим количеством единиц можно получить ноль, противоречие.
	\end{enumerate}
\end{proof}

\section{Производная многочлена}

\begin{Def}
	Пусть $A$ "--- коммутативное кольцо с 1. Тогда определим
	умножение на натуральное число $k$:

	\begin{align*}
	k \cdot a &= \underbrace{a+a+\dots+a}_{k\text{~слагаемых}} = \\
	          &=(\underbrace{1+1+\dots+1}_{k\text{~слагаемых}})a
	\end{align*}

	Для удобства также положим $0\cdot a = 0$.
\end{Def}

\begin{Def}
	Для многочлена $f=a_nx^n+a_{n-1}x^{n-1}+\dots+a_0x^0$ определим производную:

	\begin{align*}
	f' &= (n \cdot a_n) x^{n-1} + \dots + (2\cdot a_2)x + a_1 = \\
	   &= \sum_{k = 1}^nka_kx^{k-1}
	   &= \sum_{k = 0}^nka_kx^{k-1} && \text{при $k=0$ фиктивное слагаемое}
	\end{align*}

	Несмотря на то, что $x^{-1}$ неопределено, фиктивное слагаемое при $k=0$ имеет коэффициент ноль
	(считаем, что $0 \cdot x^{-1} = 0$)	и иногда удобно считать, что оно есть.
\end{Def}

\begin{theorem}{Свойства производной}
	\begin{enumerate}
		\item $(f + g)' = f' + g'$ и более общий случай: $(f_1 + \dots + f_k)' = f_1' + \dots + f_k'$.
		\item $\forall c \in A \colon (c\cdot f)' = c\cdot f'$
		\item $(fg)' = f'g + fg'$
		\item $(f_1 \cdot f_2 \cdot \dots \cdot f_k)' = f_1' \cdot f_2 \cdot \dots \cdot f_k + f_1\cdot f_2'\cdot \dots \cdot f_k + \dots + f_1 \cdot \dots \cdot f_k'$
		\item $(f^k)' = kf^{k - 1}f'$ 
		\item Если $A$ "--- поле и $f \in A[x]$, то:
		\begin{itemize}
			\item Если $\mathrm{char} A = 0$, то
			    \[f' = 0 \Lra f = c_0 \in A = \mathrm{const} \]
			\item Если $\mathrm{char} A = p > 0$, то
				\[f' = 0 \Lra f \in A[x^p]\]
				то есть все ненулевые коэффициенты $f$ находятся при степенях вида $x^{kp}$.
		\end{itemize}
	\end{enumerate}
\end{theorem} 

\begin{proof}
	\begin{enumerate}
		\item Расписать по определению.
		\item Расписать по определению.
		\item Доказываем в несколько приёмов:
		\begin{itemize}
			\item Пусть $f = x^n$, $g = x^m$:
				\[(fg)' = (x^{n + m})' = (n + m)x^{n + m -1} = nx^{n-1}x^m + x^n mx^{m-1} = f'g + fg'\]
			\item Пусть $f = x^n$, $g = \sum_{k = 0}^m c_k x^k$:
				\begin{align*}
				(fg)' &= \left(\sum_{k=0}^m c_k x^n x^k\right)' = \sum_{k=0}^m c_k(x^n x^k)' = \\
				 &= \sum_{k=0}^m c_k(f'x^k + fkx^{k-1}) = f' \sum_{k=0}^m c_k x^k + f \sum_{k=0}^m kx^{k-1} = f'g + fg'
				\end{align*}
			\item Пусть $f = \sum_{k = 0}^n a_k x^k$, а $g$ "--- произвольный многочлен:
				\begin{align*}
				(fg)' &= \sum_{k = 0}^n a_k(x^k g)' = \sum_{k = 0}^n a_k(kx^{k-1}g + x^kg') =
				 &= g\sum_{k = 0}^n ka_kx^{k-1} + g'\sum_{k=0}^n a_kx^k = f'g + fg'
				\end{align*}
		\end{itemize}
		\item По индукции, несколько раз применяем пункт 3.
		\item Следует из пункта 4.
		\item 
		\begin{itemize}
			\item Пусть $\mathrm{char} A = 0$. Докажем $\Ra$:
				\begin{gather*}
					f = c_0 + c_1x + \dots + c_nx^n \\
				0 = f' = c_1 + 2c_2x + \dots + nc_nx^{n-1} \\
				\forall k>0 \colon kc_k = \underbrace{(1 + 1 + \dots + 1)}_{\neq 0, k\text{~единиц}}c_k = 0 \Ra c_k = 0 \\
				\Ra f = c_0 = \mathrm{const}
				\end{gather*}
				$\La$ очевидно.

			\item Пусть $\mathrm{char} A = p > 0$. Тогда $\underbrace{1+1+\dots+1}_p=0$.
				\begin{enumerate}
				\item $\Ra$:
	   				\begin{gather*}
   					f' = \sum_{k = 1}^nkc_kx^{k-1} \\
   					\forall k \geq 1 \colon kc_k = 0
   					\end{gather*}

   		 			Рассмотрим такое $k$, что $p \nmid k$ $\Ra$ $k = pq + r$, $1 \le r <p$:
   					\begin{align*}
	   		 		kc_k &= \underbrace{(1+1+\dots+1)}_{k}c_k = \\
	   		 		     &= (\underbrace{\underbrace{1+\dots+1}_p+ \dots + \underbrace{1+\dots+1}_p}_q + \underbrace{1+\dots+1}_r)c_k = \\
   			 		     &= \underbrace{(1+\dots+1)}_{r\text{~раз}, \neq 0} c_k = 0 \\
   		 			\Ra c_k = 0
   		 			\end{align*}
   		 		\item $\La$:	
   		 			\begin{align*}
		 			f &= c_0 + c_px^p + c_{2p}x^2p + \dots \in A[x^p] = \\
		 			  &= \sum\limits_{j=0}^rc_{jp}x^{jp}; \\
		 			f' &= \sum\limits_{j=0}^rjpc_{jp}x^{jp-1} = \\
		 			   &= \sum\limits_{j=0}^rj\cdot 0 \cdot c_{jp}x^{jp-1} = 0;
		 			\end{align*}
			 	\end{enumerate}
		\end{itemize}	
	\end{enumerate}
\end{proof}
 
\section{Кратные корни. Теорема о кратности корня многочлена и его производной}
Напоминание: если $A$ "--- поле, $f \in A[x]$, $f \neq 0$, то $c$ "--- корень $f$ в $A$ $\iff$ когда $(x - c) \mid f$ в $A[x]$ (по теореме Безу).

\begin{Def}
Если для некоторого $k \ge 2$ имеем $(x - c)^k \mid f$, но $(x - c)^{k + 1} \nmid f$, то говорим, что $c$ "--- корень $f$ кратности $k$.
\end{Def}

\begin{Rem}
Переформулировка: $c$ "--- корень $f$ кратности $k$, если:
\begin{gather*}
\begin{cases}
f(x) = (x - c)^{k}g(x) \\
(x - c) \nmid g(x) \Lra g(c) \ne 0
\end{cases}
\end{gather*}
\end{Rem}

\begin{theorem}{}
$A$ "--- поле, $\Char A = 0$, $f \in A[x]$, $f \ne 0$. Тогда $c$ "--- корень $f$ кратности $k \ge 1$ тогда и только тогда, когда
выполняются два условия:
\begin{enumerate}
\item $c$ "--- корень $f$.
\item $c$ "--- корень $f'$ кратности $k - 1$.
\end{enumerate}
\end{theorem}
\begin{proof}
\begin{itemize}
\item[$\Ra$:] то, что $c$ "--- корень $f$, очевидно. Далее:
\begin{gather*}
\text{$c$ "--- корень кратности $k$} \iff
  \begin{cases}
  f = (x - c)^{k}g(x), \\
  g(c) \ne 0
  \end{cases}
\end{gather*}
\begin{align*}
f' &= k(x - c)^{k - 1}g + (x - c)^{k}g' = \\
   &= (x - c)^{k - 1}(kg + (x - c)g') \\
\Ra (x - c)^{k - 1} \mid f' \\
\end{align*}

Покажем, что $c$ "--- не корень $(kg + (x - c)g')$ (то есть $c$ "--- корень $f'$ кратности ровно $k-1$):
\[kg(c) + (c - c)g'(c) = kg(c) \neq 0\]
так как $k\neq 0$ и $g(c) \neq 0$.

\item[$\La$:]
$c$ "--- корень $f$. Пусть $c$ имеет в $f$ кратность $l \ge 1$, по предыдущей часте теоремы $c$ "--- корень $f'$ кратности $l - 1$.
Таким образом, $l-1=k-1 \iff l=k$.
\end{itemize}
\end{proof}

\begin{Rem}
Предположение $\Char A = 0$ существенно:
\[ f = x^7 + x^2 \in \F_2[x] \]
Здесь $0$ "--- корень кратности $2$.
\[ f' = x^6 \]
Здесь $0$ "--- кратности $6$.
\end{Rem}

\begin{Rem}
Будем обозначать $k$-ю производную как $f^{(k)}$:
\begin{gather*}
f^{(0)} = f \\
f^{(1)} = f' \\
\dots \\
f^{(k)} = (f^{(k - 1)})'
\end{gather*}
\end{Rem}
              
\begin{conseq}
$A$ "--- поле характеристики $0$. $0 \ne f \in A[x]$, $c$ "--- корень $f$ кратности $\ge k$ тогда и только тогда, когда:
\begin{gather*}
0 = f(c) = f'(c) = \ldots = f^{(k - 1)}(c) \\
\end{gather*}
\end{conseq}
\section{Секвенциальная компактность}

\begin{theorem}{Компактность в $\R^d$}
Следующее в $\R^d$ равносильно:
\begin{enumerate}
\item Компактно
\item Замкнуто и ограниченно
\item Для любой последовательности в множестве можно выбрать подпоследовательность, сходящуюсю к некоторой точке множества (\textit{секвенциально компактно})
\end{enumerate}
\end{theorem}
\begin{proof}
$2 \Ra 1$: $К$ ограниченно, значит можно его ограничить кубом, значит оно подмножество компактного и замкнуто, значит компактно.

$1 \Ra 3$:
Возьмём последовательность $\{x_n\}\lrh E$ элементов множества $F$. Если множество элементов $E$ конечно, то какой-то элемент повторился бесконечно. Возьмём новую стационарную последовательность ровно из этого элемента, имеющую предел. Если же оно бесконечно, докажем, что у него есть предельная точка.

Пусть ни одна точка не предельна. Значит 
$$\forall x \in X\: \exists r_x > 0\colon \dot B_{r_x}(x) \cap F = \emptyset$$
Но тогда возьмём покрытие
$$\bigcup_{x\in X} B_{r_x} (x)$$
В нём есть конечное подпокрытие. Возьмём его
$$\bigcup_{i=1}^k \dot B_{r_{y_i}} \supset K \supset E$$
Но также
$$\bigcup \dot B_{r_{y_i}} \cap E = \varnothing$$
Значит 
$$E \subset \bigcup_{i=1}^k \{y_i\}$$
Получили, что $E$ конечное. 

Таким образом предельная точка существует, а значит можно выбрать подпоследовательность можно.

$3 \Ra 2$:
Пусть $K$ не замкнуто. Возьмём предельную точку, которой нет в $K$. Значит есть последовательность, сходящаяся к ней. Из неё нельзя выбрать подпоследовательность, сходящуюся к элементу $K$.

Пусть $K$ не ограничено. Значит есть точка, не лежащая в данном шарике.
$$K \nsubset B_1(a) \Ra \exists x_1\colon \rho(x_1, a) > 1$$
$$K \nsubset B_{\rho(a, x_1) + 1}(a) \Ra \exists x_2\colon \rho(x_2, a) > \rho(x_1, a) + 1$$
$$ \vdots $$
Рассмотрим сходящуюся подпоследовательность. Она ограничена шариком радиуса $R$. Но
$$\rho(a, x_n) > \rho(a, x_{n-1}) + 1 > \cdots > n$$
$$R > \rho\left(b, x_{n_k}\right) > \rho\left(a, x_{n_k}\right) + \rho(a, b) > n_k + \rho(a, b) \ra \infty$$
Значит $K$ ограниченно.  
\end{proof}

\begin{Rem}
$1\Ra 3; 3\Ra 2; 1 \Ra 2$ справедливы для всех пространств. $2 \Ra 1$ ломается, например, на $\R$ с дискретной метрикой.
\end{Rem}

\section{Теорема Больцано-Вейерштрасса и другие следствия}
\begin{conseq}
В $\R^d$ компактность $K$ равносильна наличию предельной точки для любого подмножества.  
\end{conseq}
\begin{proof}
В одну сторону просто по теореме.
Обратно: возьмём часть доказательства, объясняющее взятие подпоследовательности.
\end{proof}

\begin{conseq}{Теорема Больцано-Вейерштрасса.}
Из любой ограниченной последовательности в $\R^d$ можно выбрать сходящуюся подпоследовательность.
\end{conseq}
\begin{proof}
Множество значений ограниченно, значит его замыкание компактно, значит в компактном есть сходящаяся подпоследовательность.
\end{proof}

\begin{conseq}
В любой последовательности в $\R$ есть сходящаяся в $\bar \R$ подпоследовательность.
\end{conseq}
\begin{proof}
Если ограничена, то см. предыдущее. Иначе она стремится к бесконечности. Тогда выберем бесконечную подпоследовательность, стремящуюся к бесконечности. В ней бесконечное число положительных или бесконечное число отрицательных.
\end{proof}
\section{Метод Ньютона}
\begin{Def}
$A$ "--- поле, $x_i \ne x_j$.
\[
\begin{array}{c|c|c|c}
x_1 & x_2 & \cdots & x_n\\
\hline
y_1 & y_2 & \cdots & y_n\\
\end{array}
\]
Интерполяционная задача: найти многочлен $f$, что $\deg f < n$ и
\[ \forall i=1..n\: f(x_i) = y_i \]
\end{Def}

Пусть есть решение $f$
\[ g = (x - x_1) \cdots (x - x_n) \]
Тогда $f_1 = f + gh$ "--- тоже решение.
\[ f_1(x_i) = f(x_i) + g(x_i)h(x_i) = f(x_i) = y_i \]

\begin{theorem}{Единственность}
В данной постановке задача имеет не более одного решения. 
\end{theorem}
\begin{proof}
Пусть $f$, $f_1$ "--- решения одной задачи.
\[ f(x_i) = f_1(x_i) = y_i, \deg f, \deg f_1 < n \]
$f - f_1$ принимают 0 в $x_1, \ldots, x_n$.
\[ \deg(f - f_1) < n \Ra f - f_1 = 0 \Ra f = f_1 \]
\end{proof}

\textbf{Метод Ньютона}: $f_i(x), \deg f_i < i$, $f_i$ решает интерприционную задачу на первых $i$ точках.

\begin{itemize}
\item[$i = 1$:] $f_1(x) = y_1$.
\item[$i \to i + 1$:] $f_i \to f_{i + 1}$
\begin{gather*}
f_{i + 1}(x) = f_i(x) + c_i(x - x_1)\ldots(x - x_i) \\
y_{i + 1} = f_{i + 1}(x_{i + 1}) = f_i(x_{i + 1}) + c_i(x_{i + 1} - x_1) \ldots (x_{i + 1} - x_i) \\
c_i = \frac{y_{i + 1} - f_i(x_{i + 1})}{(x_{i + 1} - x_{1}) \ldots (x_{i + 1} - x_{i})} \\
\deg f_{i + 1} < i + 1 \\
\end{gather*}
\end{itemize}

\begin{Rem}
\[ c_1 = \frac{y_2 - y_1}{x_2 - x_1} \]
\end{Rem}
\section{Метод Лагранжа}
Решим сначала подзадчу:
\[
\begin{array}{c|c|c|c|c|c}
x_1 & x_2 & \cdots & x_i & \cdots & x_n \\
\hline
0 & 0 & \cdots & 1 & \cdots & 0 \\
\end{array}
\]
$\deg L_i < n$
\[ L_i = \frac{(x - x_1)(x-x_2) \cdots (x - x_{i - 1})(x - x_{i + 1}) \cdots (x - x_n)}{(x_i - x_1)(x_i - x_2) \cdots(x_i-x_{i-1})(x_i-x_{i+1}) \cdots (x_i - x_n)} \]

Перейдём к общему случаю:
\begin{gather*}
\begin{array}{c|c|c|c}
x_1 & x_2 & \ldots & x_n \\
\hline
y_1 & y_2 & \ldots & y_n \\
\end{array} \\
f = y_1L_1 + y_2L_2 + \cdots + y_n L_n \\
f(x_i) = \sum_{k = 1}^{n} y_kL_k(x_i) = y_i L_i(x_i) = y_i \\
f = \sum_{k = 1}^{n} y_kL_k \\
L_i(x) = \frac{(x - x_1)(x - x_2) \cdots (x-x_{i-1})(x-x_{i+1}) \cdots (x - x_n)}{(x_i - x_1)(x_i - x_2) \cdots(x_i-x_{i-1})(x_i-x_{i+1}) \cdots (x_i - x_n)}
\end{gather*}
Теперь выпишем дробь в более общем виде. 
\[ g(x) = \prod_{j=1}^{n} (x-x_j) \]
Числитель $L_k$ равен $\frac{g(x)}{(x - x_k)}$, знаменатель же равен $g'(x_i)$:
\begin{gather*}
g'(x) = \\
= 1(x-x_2)(x-x_3)\cdots(x-x_n) + (x-x_1)1(x-x_3)\cdots(x-x_n) + \cdots + (x-x_1)(x-x_2)\cdots(x-x_{n-1})1 \\
g'(x_i) = 0 + 0 + \cdots + (x_i - x_1)(x_i - x_2) \cdots(x_i-x_{i-1})1(x_i-x_{i+1}) \cdots (x_i - x_n) + \cdots + 0 \\
f(x) = \sum_{i=1}^{n} f(x_i) \frac{g(x)}{(x - x_i) g'(x_i)} \\
\deg g = n \Ra \deg f \le n - 1
\end{gather*}

\section{Полнота компактных метрических пространств}

\begin{theorem}{О сходимости фундаментальных последовательностей}
\begin{enumerate}
\item Любая сходящаяся последовательность фундаментальна.
\item В $\R^d$ фундаментальная последовательность всегда сходится.
\end{enumerate}
\end{theorem}
\begin{proof}
$\lim x_n = a$
$$\forall \epsilon > 0\: \exists N\colon \begin{aligned}\forall n>N \rho(x_n, a) &< \epsilon \\ \forall m>N \rho(x_m, a) &< \epsilon\end{aligned} \Ra \forall \epsilon>0\: \exists N\colon \forall n,m>N\: \rho(x_m, x_n) < 2\epsilon$$

$x_n$~--- фундаментальная последовательность в $\R^d$. $E_n \lrh \{x_n, x_{n+1}, \ldots\}$~--- ограниченно.
$\cl E_n$~--- ещё и замкнуто. Т.е. компактно.
$$\cl E_1 \supset \cl E_2 \supset \cl E_3 \supset \cdots$$
$$\diam \cl E_n = \diam E_n \ra 0$$

Т.о.
$$\exists! a\colon a \in \bigcap_{i=1}^{\infty} \cl E_n$$
$$a \in \cl E_n \Ra \forall i>n\:0\leqslant\rho(a, x_i) \leqslant \diam E_n \ra 0$$

Т.о $x_n \ra a$.
\end{proof}
                                                
\begin{Rem}
$\R^d$ полно. $\left<\Q, \rho\right>$ не полно. Пространство с дискретной метрикой полно.
\end{Rem}

\begin{theorem}{О полноте компактного пространства}
Компактное метрическое пространство полно.
\end{theorem}
\begin{proof}
В компакте у любой последовательности есть сходящаяся подпоследовательность. А значит любая фундаментальная последовательность имеет сходящуюся подпоследовательность.
А значит она сама сходится. А значит пространство полно.
\end{proof}
\section{Конструкция комплексных чисел, как множества пар}
\begin{gather*}
\R^2 = \{(a, b) \mid a, b \in \R\}
\end{gather*}
\begin{itemize} 
\item $+$~: $\R^2 \times \R^2 \ra \R^2$
\begin{gather*}
(a, b) + (c, d) \mapsto (a + c, b + d)
\end{gather*}
\item $*$~: $\R^2 \times \R^2 \ra \R^2$
\begin{gather*}
(a, b)*(c, d) \mapsto (ac - bd, ad + bc)
\end{gather*}
\end{itemize}
\begin{theorem}{}
$\R^2$ с введёнными операциями является полем.
\end{theorem}
\begin{Def}
Это поле называется полем комплексных чисел $\C$ (Complex).
\end{Def}
\begin{proof}
\begin{enumerate}
\item $0_c = (0, 0)$
\item $-(a, b) = (-a, -b) $
\item $(1, 0) * (a, b) = (a, b) $
\item$ (a, b) \ne 0$, $(a, b)^{-1}$
\begin{gather*}
(a, b)^{-1} = (c, d) \Lra (a, b)*(c, d) = (1, 0) \Lra \\
\Lra \left\{\begin{aligned}
ac - bd &= 1 \\
bc + ad &= 0 \\
\end{aligned}\right.\Lra \\
\Lra \left\{\begin{aligned}
(a^2 + b^2) c &= a \\
(a^2 + b^2) d &= -b
\end{aligned}\right.
\end{gather*}
Найденные значения корректны, так как $(a, b) \ne 0 \Ra a^2 + b^2 > 0$.
\end{enumerate}
\end{proof}

\section{Алгебраическая форма записи комплексного числа. Комплексное сопряжение. Свойства комплексного сопряжения}

$\R \mapsto \C\colon a \mapsto (a, 0)$ - инъективный гомоморфизм колец:
\begin{align*}
	\phi(a+b) &= \phi(a) + \phi(b) \\
	\phi(ab) &= \phi(a) \phi(b)
\end{align*}
\begin{gather*}
\C \supset \phi(\R) = \{(a, 0) \mid a \in \R\} \\
\phi(\R) \cong \R
\end{gather*}
Поэтому говорят, что $\R \subseteq \C$, имея в виду, что $\phi(\R) \subseteq \C$.
\begin{gather*}
i = (0, 1) \Ra i^2 = (-1, 0) = -1
\end{gather*}
\begin{Def}
Алгебраическая форма записи:
\begin{gather*}
(a, b) = (a, 0)*(1, 0) + (b, 0)*(0, 1) = a + bi
\end{gather*}
$a$ называется вещественной частью комплексного числа, $b$ "--- мнимой частью.
\begin{gather*}
a = \Re z \quad b = \Im z
\end{gather*}
\end{Def}

\begin{Def}
$z \in \C$, $z = a + bi$, $a, b \in \R$. $\bar z$ называется комплексно сопряжённым с $z$, если $\bar z = a - bi$.
\end{Def}
\begin{Rem}
Сопряжение $\sim$ симметрия относительно вещественной оси.
\begin{center}
\def\svgwidth{6.0cm}
\input{theory35-vector.pdf_tex}
\end{center}
\end{Rem}

\underline{Cвойства:}
\begin{itemize}
\item[1.] $\bar{\bar z} = z$
\item[2.] $z = \overline{z} \Lra z \in \R$
\item[3.] $\overline{z_1 + z_2} = \bar z_1 + \bar z_2$
\item[3'.] $\overline{z_1 + z_2 + \dots + z_n} = \bar z_1 + \bar z_2 + \cdots + \bar z_n$ (По индукции из свойства 3)
\item[4.] $\overline{z_1z_2} = \bar z_1 \bar z_2$
\item[4'.] $\overline{z_1 z_2 \cdots z_n} = \bar z_1 \bar z_2 \cdots \bar z_n$ (По индукции из свойства 4)
\item[5.] $f \in \R[x]~f = a_0 + a_1x + a_2x^2 + \dots + a_nx^n$ Тогда: $\overline{f(z)} = f(\bar z)$
\item[6.] \begin{itemize}
			\item[\bullet] $z + \bar z \in \R$ 
			\item[\bullet] $z \bar z \in \R$, $z \bar z \ge 0$
			\item[\bullet] $z \bar z = 0 \Lra z = 0$
		 \end{itemize}	
Два последних пункта следуют из того, что $z \bar z = a^2 + b^2$.
\begin{proof}
Только 5 свойство:
\begin{gather*}
f(z)=a_0 + a_1z + \cdots + a_nz^n \\
\overline{f(z)} = \overline{a_0 + a_1z + \cdots + a_nz^n} = \bar a_0 + \overline{a_1z} + \cdots + \overline{a_nz^n} = \bar a_0 + \bar a_1 \bar z + \cdots + \bar a_n \overline{z^n} = \\
= a_0 + a_1\bar {z} + \cdots + a_n\bar z^n = f(\overline{z})
\end{gather*}
\end{proof}   

$\bar z\colon \C \ra \C$ "--- гомоморфизм:
\begin{align*}
\overline{z_1 + z_2} &= \bar z_1 + \bar z_2 \\
\overline{z_1 z_1} &= \bar z_1 \bar z_2
\end{align*}

$\overline{z} \circ \overline{z} = id_\C$, поэтому сопряжение "--- нетождественный изоморфизм из $\C$ на себя (автоморфизм).
\begin{Def}
Автоморфизм "--- изоморфизм поля с самим собой.
\end{Def}
\item[7.] $z \ne 0$: $z \bar z = |z|^2$, $|z| \ne 0$.
\begin{gather*}
z \frac{\bar z}{|z|^2} = 1 \Ra z^{-1} = \frac{\bar z}{|z|^2} = \frac{a - bi}{a^2 + b^2}
\end{gather*}
PS: определение и проч. про модуль в следующем вопросе.
\end{itemize}


\section{Модуль комплексного числа. Мультипликативность модуля. Произведение двух сумм двух квадратов}
\begin{gather*}
\forall z = a+bi \in \C\colon z\bar z = a^2 + b^2 
\end{gather*}
\begin{Def}
Модуль комплексного числа
\begin{gather*}
|z| = \sqrt{z\overline{z}} = \sqrt{a^2 + b^2}
\end{gather*}
\end{Def}

\underline{Свойство}: $|z_1z_2|^2 = |z_1|^2|z_2|^2$
\begin{proof}
\begin{gather*}
z_1 = a + bi,~z_2 = c + di \\
(a^2 + b^2)(c^2 + d^2) = (ac - bd)^2 + (ad + bc)^2
\end{gather*}
\end{proof}
\begin{Rem}
	Для $\Z[a, b, c, d]$(кольцо многочленов) тоже верно.
\end{Rem}

\underline{Напоминание}: $\phi$ называется мультипликативной, если $\phi(ab) = \phi(a)\phi(b)$. Таким образом, модуль мультипликативен.

\underline{Вопрос}: При каких $k$ имеет место
\begin{gather*}
(a_1^2 + \cdots + a_k^2)(b_1^2 + \cdots + b_k^2) = c_1^2 + \dots + c_k^2
\end{gather*}
где $c_i$ "--- полиномы от $a_j$ и $b_l$?

\underline{Ответ}: Только для $k \in \{1, 2, 4, 8\}$.
\begin{description}
\item[$k = 1$:] мультипликативность |$\R$|
\item[$k = 2$:] мультипликативность |$\C$|
\item[$k = 4$:] мультипликативность модуля кватернионов
\item[$k = 8$:] мультипликативность модуля октав
\end{description}
\section{Число e}

Определим число $e$:
$$x_n = \left(1+\frac1n\right)^n; y_n=\left(1+\frac1n\right)^{n+1}$$
Покажем, что $x_n \uparrow; y_n \downarrow$.
\begin{proof}
$$x_n < x_{n+1} \La \frac{(n+1)^n}{n^n} < \frac{(n+2)^{n+1}}{(n+1)^{n+1}} \La \frac{n+1}{n+2} < \frac{n^n(n+2)^n}{(n+1)^{2n}} \La $$
$$ \La \frac{n+1}{n+2} < \left(1-\frac1{n^2+2n+1}\right)^n \La 1 - \frac{1}{n+2} < 1 - \frac{n}{n^2+2n+1} \leqslant \left(1-\frac1{n^2+2n+1}\right)^n$$

$$y_n < y_{n-1} \La \frac{(n+1)^{n+1}}{n^{n+1}} < \frac{n^n}{(n-1)^n} \La \frac{n+1}{n} < \frac{n^{2n}}{(n-1)^n(n+1)^n} \La $$
$$\La \frac{n+1}{n} < \left(1+\frac1{n^2-1}\right)^n \La 1 + \frac1n < 1 - \frac{n}{n^2-1} \leqslant \left(1-\frac1{n^2-1}\right)^n$$
\end{proof}

Заметим, что при этом $x_n < y_n$. Собственно, тогда $\lim x_n$ существует.
$$\lim \left(1+\frac1n\right)^n \lrh e$$

Свойства:
\begin{enumerate}
\item $\lim y_n = e$
\item $x_n < e < y_n$
\end{enumerate}
\section{Сравнение скорости роста возрастания последовательностей}

\begin{theorem}{Предел убывающей по отношению}
$x_n > 0$, $\lim \frac{x_{n+1}}{x_n} < 1$. Тогда $x_n \ra 0$.
\end{theorem}
\begin{proof}
С какого-то места отношение довольно мало (меньше 1).
\end{proof}
\begin{conseq}
$$\lim_{n\ra\infty} \frac{n^k}{a^n} = 0\quad a>1$$
\end{conseq}
\begin{proof}
$$x_n = \frac{n^k}{a^n}$$
$$\frac{x_{n+1}}{x_n} = \left(\frac{n+1}n\right)^k \frac1a < 1$$
\end{proof}
\begin{conseq}
$$\lim \frac{a^n}{n!} = 0$$
\end{conseq}

\begin{conseq}
$$\lim \frac{n!}{n^n} = 0$$
\end{conseq}
\begin{proof}
$$x_n = \frac{n!}{n^n}$$
$$\frac{x_{n+1}}{x_n} = (1+\frac1n)^-n \ra \frac1e < 1$$
\end{proof}


\section{Теорема Штольца}

\begin{theorem}{Теорема Штольца}
$0<y_n<y_{n-1}$, $\lim x_n = \lim y_n = 0$, $\lim \frac{x_n-x_{n+1}}{y_n-y_{n+1}} = a \in \bar\R$.
Тогда $\lim \frac{x_n}{y_n} = a$.
\end{theorem}
\begin{proof}
\begin{enumerate}
\item Пусть a = 0.

$$\epsilon_n = \frac{x_n - x_{n - 1}}{y_n - y_{n - 1}} \to 0$$

$$x_n - x_m = \sum_{k = m + 1}^{n}(x_k - x_{k - 1}) = \sum_{k = m + 1}^{n}\epsilon_k(y_k - y_{k - 1})$$

$$|x_n - x_m| = |\sum| \le \sum_{k = m + 1}^{n}|\epsilon_k|(y_{k - 1} - y_k)$$

Выберем N, такое что $\forall k > N |\epsilon_k| < \epsilon$, тогда при n и m > N 

$$< \sum^{n}_{k = m + 1}\epsilon(y_{k - 1} - y_k) = \epsilon\sum_{k = m + 1}^{n}(y_{k - 1} - y_k) = \epsilon(y_m - y_n)$$

$$|x_n - x_m| < \epsilon|y_n - y_m|$$

устремим n  к бесконености.

$$|x_m| < \epsilon(y_m)$$

$$\frac{x_m}{y_m} < \epsilon \text{при $m > N$}$$

\item $a \in \R$

$$\tilde x_n = x_n - a y_n$$
\end{enumerate}
\end{proof}

\section{Теорема Штольца}
\begin{theorem}{Теорема Штольца}
$0<y_n<y_{n+1}$, $\lim x_n = \lim y_n = +\infty$, $\lim \frac{x_n-x_{n+1}}{y_n-y_{n+1}} = a \in \bar\R$.
Тогда $\lim \frac{x_n}{y_n} = a$.
\end{theorem}
\begin{proof}
$a = 0$:
$$\epsilon_n \lrh \frac{x_n - x_{n+1}}{y_n - y_{n+1}}$$
$$x_n = x_1 + \sum_{i=2}^n (x_i - x_{i-1}) = x_1 + \sum_{i=2}^n \epsilon_i(y_i - y_{i-1})$$
$$\frac{x_n}{y_n} = \frac{x_1}{y_n} + \sum_{i=2}^n \epsilon_i \frac{y_i - y_{i-1}}{y_n} = $$
$$\forall \epsilon > 0\: \exists N\colon \forall n > N\: |\epsilon_n| < \epsilon$$
$$= \frac{x_1}{y_n} + \sum_{i=2}^N + \sum_{i=N+1}^n$$

$$\left|\sum_{i=N+1}^n \epsilon_i \frac{y_i - y_{i-1}}{y_n}\right| \leqslant \sum_{i=N+1}^n |\epsilon_i| \frac{y_i - y_{i-1}}{y_n} < 
\sum_{i=N+1}^n \epsilon \frac{y_i - y_{i-1}}{y_n} <$$
$$< \frac{\epsilon}{y_n}\sum_{i=N+1}^n (y_i - y_{i-1}) = \frac{\epsilon}{y_n} (y_n - y_N) < \epsilon$$

$$\sum_{i=2}^N \epsilon_i \frac{y_i - y_{i-1}}{y_n} \leqslant \frac{1}{y_n}\sum_{i=2}^N \epsilon_i(y_i - y_{i-1}) < \epsilon$$

$$\frac{x_1}{y_n} < \epsilon$$

Т.о.
$$\left|\frac{x_n}{y_n}\right| < \epsilon \Ra \frac{x_n}{y_n} \ra 0$$

$a\in\R$:
$\tilde x_n = x_n - a y_n$. Фактом $x_n\ra\infty$ мы не пользовались.

$$\frac{\tilde x_n - \tilde x_{n - 1}}{y_n - y_{n - 1}} = \frac{(x_n - ay_n) - (x_{n - 1} - ay_{n - 1})}{y_n - y_{n - 1}} = \frac{x_n - x_{n - 1}}{y_n - y_{n - 1}} - a \to 0$$

$a=+\infty$: Поменяем местами $x_n$ и $y_n$. Проверим, что $x_n$ монотонно растёт и не ноль.
$$\frac{x_n - x_{n-1}}{y_n - y_{n-1}} = +\infty \Ra \frac{x_n - x_{n-1}}{y_n - y_{n-1}} > 1 x_n - x_{n11} > y_n - y_{n-1} > 0$$

$a=-\infty$: Сменим знаки $x_n$.
\end{proof}
\section{Пределы функций}

\begin{Def}
$(X, \rho_x)$ и $(Y, \rho_y)$~--- метрические пространства. $E \subset X$, $a$~--- предельная точка $E$. $f\colon E \ra Y$.
Тогда говорят, что
$$\lim_{x \ra a} f(x) = b$$
если $b \in Y$ и
$$\forall \epsilon>0\: \exists \delta > 0\colon \forall x \in \dot B_\delta(a)\: \cap E \Ra f(x) \in B_\epsilon (b)$$
или, что то же самое
$$\forall \epsilon>0\: \exists \delta > 0\colon \forall x \in E \: (x \ne a \land \rho(x, a) < \delta) \Ra \rho(f(x), b) < \epsilon$$
\end{Def}

\begin{Rem}
Для бесконечности на $\R$ есть частные случаи.
\end{Rem}

\begin{Def}
По Гейне,
$$\lim_{x \ra a} f(x) = b \Lra \forall \{x_n\}\subset E\colon x_n \ne a\: \lim_{n\ra \infty} x_n = a \Ra \lim_{n \ra \infty} f(x_n) = b$$
\end{Def}


\section{Равносильность определения по Коши и по Гейне}

\begin{theorem}{Равносильность определений предела функции}
Определения равносильны.
\end{theorem}

\begin{proof}
\begin{enumerate}
\item Коши $\Ra$ Гейне

$$\forall \epsilon>0\: \exists \delta > 0\colon \forall x \in \dot B_\delta(a)\: \cap E \Ra f(x) \in B_\epsilon (b)$$

Пусть $lim x_n = a$, $x_n \in E, x_n \ne a$

По $\delta$ выберем N $\forall n > N x_n \in B_{\delta}(a)$, тогда $f(x_n) \in B_{\epsilon}(b)$

Нашли номер N при котором $f(x_n) \in B_{\epsilon}(b) \Ra lim f(x_n) = b$ 

\item Гейне $\Ra$ Коши

от противного.

По Коши $\to \forall \epsilon>0\: \exists \delta > 0\colon \forall x \in \dot B_\delta(a)\: \cap E \Ra f(x) \in B_\epsilon (b)$

$$\exists \epsilon>0\: \forall \delta > 0\colon \exists x \in \dot B_\delta(a)\: \cap E \Ra f(x) \notin B_\epsilon (b)$$

$$\delta = \frac{1}{n}$$

Выберем последовательность $\{x_n\}$

$$x_n \in \dot B_{\frac1n}(a)$$

$$\rho(f(x_n), b) \ge \epsilon \Ra lim(f(x_n)) \ne b$$

Противоречие с определением по Гейне
\end{enumerate}

\end{proof}

\begin{Rem}
В определение по Гейне можно рассматривать только те последовательности, в которых все $x_n$ различны.
\end{Rem}
\begin{Rem}
Можно рассматривать лишь такие последовательности, что $\rho(x_n, a)$ убывает.
\end{Rem}



\section{Многочлены Чебышева}
\begin{theorem}{}
Существуют многочлены $T_n(x)$, $U_n(x)$ такие, что:
\begin{itemize}
\item $\cos{n \phi} = T_n(\cos \phi)$
\item $\frac{\sin{n \phi}}{\sin \phi} = U_n(\cos \phi)$ при $\phi \ne 2 \pi k$
\item $T_0(x) = 1$, $T_1(x) = x$, $T_{n + 1}(x) = 2xT_n(x) - T_{n - 1}(x)$
\item $U_0(x) = 0$, $U_1(x) = 1$, $U_{n + 1}(x) = 2xU_n(x) - U_{n - 1}(x)$
\end{itemize}
\end{theorem}
\begin{Def} 
Многочлены $T_n$ и $U_n$ называются многочленами Чебышева первого и второго рода соответсвенно.
\end{Def}

\underline{Пример:} 
\begin{align*}
T_0(x) &= 1 & \cos 0\phi &= 1 \\ 
T_1(x) &= x & \cos 1\phi &= \cos \phi\\ 
T_2(x) &= 2x^2 - 1 & \cos 2\phi &= 2 \cos^2 \phi - 1 \\
T_3(x) &= 4x^3 - 3x & \cos 3\phi &= 4 \cos^3 \phi - 3\cos \phi \\
U_0(x) &= 0 & \frac{\sin 0\phi}{\sin \phi} &= 0\\ 
U_1(x) &= 1 & \frac{\sin 1\phi}{\sin \phi} &= 1\\ 
U_2(x) &= 2x & \frac{\sin 2\phi}{\sin \phi} &= 2\cos \phi \\
U_3(x) &= 4x^2 - 1 & \sin 3\phi &= (4\cos^2 \phi - 1)\sin \phi \\
\end{align*}

\begin{proof}
$U_n, T_n$ "--- многочлены с целыми коэффициентами по их заданию. $\deg T_n = n$, $\deg U_n = n - 1$ (индукция по $n$).

Докажем $\cos n\phi = T_n(\cos\phi)$ индукцией по $n$.

\underline{База}: $n = 0, 1$.

\underline{Переход:}
\begin{gather*}
z = \cos\phi + i \sin\phi \\
\cos (n + 1)\phi = \frac{z^{n + 1} + z^{-(n + 1)}}{2} = (z + z^{-1})\frac{z^n + z^{-n}}{2} - \frac{z^{n - 1} + z^{-(n - 1)}}{2}= \\
= 2\cos\phi\cos{n\phi} - \cos{(n - 1)\phi} = 2 \cos \phi\, T_n(\cos \phi) - T_{n - 1}(\cos\phi) = T_{n + 1}(\cos \phi)
\end{gather*}
Таким образом, для $T_n$ всё доказано.

Доказательство для $U_n$: аналогично через
\begin{gather*}
\frac{\sin{n\phi}}{\sin\phi} = \frac{z^n - z^{-n}}{z - z^{-1}}
\end{gather*}
\end{proof}
\section{Теорема о пересечении высот треугольника}
\begin{theorem}{Высоты треугольника} 3 высоты треугольника пересекаются в одной точке.
\end{theorem}
\begin{proof}
\begin{center}
\def\svgwidth{6.0cm}
\input{theory44-altitudes.pdf_tex}
\end{center}

$(a, b), (c, d)$ "--- точки. Тогда $(a, b) \bot (c, d) \Lra ac + bd = 0$ "--- скалярное произведение.

$$z_1 = a + bi, z_2 = c + di, \Re (z_1\bar{z_2}) = ac + bd$$

Известно, что $(z_1 - w) \bot z_2$ и $(z_2 - w) \bot z_1$.

Надо доказать, что $w \bot z_1 - z_2$.
$$\left\{
\begin{aligned}
	z_1 - w \bot z_2 &\Lra \Re((z_1 - w)\bar{z_2}) = 0 \\
	z_2 - w \bot z_1 &\Lra \Re((z_2 - w)\bar{z_1}) = 0
\end{aligned}\right.
\Ra \Re(z_1\bar{z_2} - z_2\bar{z_1} + w(\bar{z_1} - \bar{z_2})) = 0 \Lra
$$
$$\Lra \Re(z_1\bar{z_2} - z_2\bar{z_1}) + \Re(w(\overline{z_1 - z_2})) = 0$$
$z_1\bar{z_2} - z_2\bar{z_1}$ "--- чисто мнимое, поэтому 
$$\Re(z_1\bar{z_2} - z_2\bar{z_1}) = \Ra \Re(w(\overline{z_1 - z_2})) = 0$$
\end{proof}

\underline{Упражнение:}
\begin{enumerate}
\item Медианы.
\item Биссектрисы.
\end{enumerate}
\section{Теорема о предельном переходе в неравенствах. Теорема о двух милиционерах}


\begin{theorem}{Предельный переход в неравенстве.}
$f, g\colon E \ra Y$, $a$ предельная точка $E$, $\forall x \in E\setminus \{a\} f(x) \leqslant g(x)$. Тогда $f_0 \leqslant g_0$.
\end{theorem}

\begin{theorem}{О двух миллиционерах}
\end{theorem}

\section{Матричная конструкция поля коплексных чисел}

$$M(2, \R)$$
$$\mathcal{C} = \left\{\left(\begin{matrix}a & -b \\ b & a\end{matrix}\right) \mid a, b \in \R\right\}$$

\begin{assertion}
$\mathcal{C}$ "--- коммутативное кольцо с единицей.
\end{assertion}
\begin{proof}
Операции замкнуты:
$$\left(\begin{matrix}a_1 & -b_1 \\ b_1 & a_1\end{matrix}\right) + \left(\begin{matrix}a_2 & -b_2 \\ b_2 & a_2\end{matrix}\right) = \left(\begin{matrix}a_1+a_2 & -b_1-b_2 \\ b_1+b_2 & a_1+a_2\end{matrix}\right)$$
$$\left(\begin{matrix}a_1 & -b_1 \\ b_1 & a_1\end{matrix}\right) \left(\begin{matrix}a_2 & -b_2 \\ b_2 & a_2\end{matrix}\right) = \left(\begin{matrix}a_1a_2-b_1b_2 & -a_1b_2-a_2b_1 \\ a_2b_1+a_1b_2 & -b_1b_2+a_1a_2\end{matrix}\right) = \left(\begin{matrix}a_1a_2-b_1b_2 & -(a_1b_2+a_2b_1) \\ a_1b_2+a_2b_1 & a_1a_2-b_1b_2\end{matrix}\right)$$
Как видно, операции и коммутативны.
Единица есть:
$$\left(\begin{matrix}1 & 0 \\ 0 & 1\end{matrix}\right) = \left(\begin{matrix}1 & -0 \\ 0 & 1\end{matrix}\right)$$
Таким образом, $\mathcal{С}$ "--- коммутативное подкольцо с единицей.
\end{proof}

\begin{assertion}
$\mathcal{C}$ "--- поле.
\end{assertion}
\begin{proof}
Найдём обратный:
$$\left(\begin{matrix}a & -b \\ b & a\end{matrix}\right)\left(\begin{matrix}a' & -b' \\ b' & a'\end{matrix}\right) = \left(\begin{matrix}1 & 0 \\ 0 & 1\end{matrix}\right) 
\Lra \left\{\begin{aligned}aa'-bb' &= 1 \\ ab'+a'b &= 0\end{aligned}\right.\xLongleftrightarrow{a, b \ne 0} \left\{\begin{aligned}a' &= a\frac1{a^2+b^2} \\ b' &= -b\frac1{a^2+b^2}\end{aligned}\right.$$
\end{proof}

\begin{assertion}
$$\C \sim \mathcal{C}$$
\end{assertion}
\begin{proof}
Отображение очевидно:
$$(a, b) \lra \left(\begin{matrix}a & -b \\ b & a\end{matrix}\right)$$
Все операции переходят друг в друга, базовые операции (сложение, умножение на скаляр, перемножение) переходят в себя, сопряжение "--- в транспонирование.
\end{proof}

\section{Критерий Коши для отображений и для функций}

\begin{theorem}{Критерий Коши}

$$f: E \subset X \to Y, a \text{~--- предельная точка E, Y ~--- полное}$$

$$\exists \lim_{x \to a} f(x) \Lra \forall \epsilon > 0 \exists \delta > 0 \forall x, y \in \dot B_{\delta}(a) \cap E \rho(f(x), f(y)) < \epsilon$$

\end{theorem}
\begin{proof}
 \begin{enumerate}
 \item $\Ra$ 
 Если $\lim_{x \to a}f(x) = b \Ra \forall \epsilon > 0 \exists \delta > 0$
   $$\forall x \in \dot B_{\delta}(a) \forall y \in \dot B_{\delta}(a)$$

   $$f(x) \in B_{\epsilon}(b), f(y) \in B_{\epsilon}(b)$$

   $$\rho(f(x), b) < \epsilon, \rho(f(y), b) < \epsilon \Ra \rho(f(x), f(y)) \le \rho(f(x), b) + \rho(f(y), b) < 2\epsilon$$
 \item $\La$

     Берем любую последовательность $x_n$ $x_n \ne a \in E \to a$

     $$\exists N \forall n > N x_n \in B_{\delta}(a)$$

     $$\Ra x_n \in \dot B_{\delta}(a) \cap E$$

     $$\rho(f(x_n), f(x_m)) \forall n, m > N$$

     $$\Ra f(x_n) \text{~--- фундументальная последовательность точек из Y}$$

     $$\Ra \exists lim(f(x_n)) \text{полнота Y}$$
 \end{enumerate}


\end{proof}
\section{Непрерывные отображения. Непрерывность слева и справа}

\begin{Def} (По Коши) $f: E \subset x \to y a \in E$
$$f \text{~--- непрерывно в точке a, если } \forall \epsilon > 0\: \exists \delta > 0\colon \forall x \in B_{\delta}(a)\: f(x) \in B_{\epsilon}(f(a))$$
\end{Def}

\begin{Def} (По Гейне)
$$\forall \{x_n\} \subset E \land x_n \to a\: f(x_n) \to f(a) \Lra f \text{~--- непрерывно в точке а}$$
\end{Def}

\begin{Def}
$f: E \subset \R \to Y, a \in E$

$$f \text{~--- непрерывно слева в точке a} \Lra g = f|_{(-\infty , a] \cap E} \text{ ~--- непрерывно в точке a}$$
\end{Def}

\begin{Def}
$f: E \subset \R \to Y, a \in E$

$$f \text{~--- непрерывно справа в точке a} \Lra g = f|_{[a, +\infty ) \cap E} \text{ ~--- непрерывно в точке a}$$
\end{Def}
\section{Арифметические действия с непрерывными функциями}

\begin{theorem}{Арифметические действия с непрерывными функциями}
$f, g\colon E \subset X \to \R^d$, $a \in E$, $f, g$ непрерывны в точке $a$. Тогда

\begin{enumerate}
\item $f(x)+g(x)$ непрерывно в точке $a$
\item $сf(x)$ непрерывно в точке $a$
\item $f(x) - g(x)$ непрерывно в точке $a$
\item $\| f(x) \|$ непрерывно в точке $a$
\item $\left<f(x), g(x)\right>$ непрерывно в точке $a$
\end{enumerate}
\end{theorem}

\begin{theorem}{Арифметические действия с непрерывными вещественными функциями}
$f, g\colon E \subset X \to \R$, $a \in E$, $f, g$ непрерывны в точке $a$. Тогда
\begin{enumerate}
\item $f(x) + g(x)$ непрерывно в точке $a$
\item $f(x)g(x)$ непрерывно в точке $a$
\item $f(x) - g(x)$ непрерывно в точке $a$
\item $|f(x)|$ непрерывно в точке $a$
\item Если $g(a) \ne 0$, то $\frac{f(x)}{g(x)}$ непрерывно в точке $a$
\end{enumerate}
\end{theorem}

\begin{theorem}{О стабильном знаке}
$f\colon E \subset X \to \R$, $a \in E$, $f$~--- непрерывно в точке $a$ и $f(a) \ne 0$. Тогда

$$\exists B_{\delta}(a)\colon \forall x \in B_{\delta}(a)\: sign(f(x)) = sign(f(a))$$
\end{theorem}
 
\begin{proof}
$$\epsilon = \frac{|f(a)|}{2}$$
\end{proof}

\begin{theorem}{О непрерывности композиции}
$f\colon E_1 \subset X \to Y$, $g\colon E_2 \subset Y \to Z$, $f(E_1) \subset E_2$, $a \in E_2$, $f$ непрерывна в точке $a$, $g$ непрерывна в точке $f(a)$. Тогда $g \circ f$ непрерывна в точке $a$.
\end{theorem}
\begin{proof}
Надо проверить, что 
$$\forall \epsilon > 0\: \exists \delta > 0\colon \forall x \in B_{\delta}(a) \cap E_1\: g(f(x)) \in B_{\epsilon}(g(f(a)))$$
Берем $\epsilon$
$$\exists \gamma > 0\colon \forall y \in B_{\gamma}(f(a)) \cap E_2\: g(y) \in B_{\epsilon}(g(f(a))) \quad \text{по непрерывности $g$ в точке $f(a)$}$$
$$\exists \delta > 0\colon \forall x \in B_{\delta}(a) \cap E_1\: f(x) \in B_{\gamma}(f(a)) \quad \text{по непрерывности $f$ в точке $a$}$$
Тогда
$$g(f(x)) \in B_{\epsilon}(g(f(a))$$
\end{proof}

\section{Тождество Эйлера: две суммы четырёх квадратов}

Можно показать, что модуль мультипликативен.

Тождество Эйлера для произведений двух сумм четырёх квадратов (правая часть "--- квадрат модуля произведения
двух кватернионов):
\begin{gather*}
|\alpha \beta| = |\alpha||\beta| \Lra (a_1^2+b_1^2+c_1^2+d_1^2)(a_2^2+b_2^2+c_2^2+d_2^2) = \\
= (a_1a_2 - b_1b_2 - c_1c_2 - d_1d_2)^2 + (a_1b_2 + a_2b_1 + c_1d_2 - c_2d_1)^2 + \\
+ (a_1c_2 + a_2c_1 - b_1d_2 + b_2d_1)^2 + (a_1d_2 + a_2d_1 + b_1c_2 - b_2c_1)^2
\end{gather*}

\noindent \underline{\hbox to 1\textwidth{{ } \hfil{ } \hfil{ } }}
\end{document}

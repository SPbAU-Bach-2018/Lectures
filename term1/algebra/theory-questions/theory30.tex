\section{Алгебраические замкнутые поля}
\begin{Def}
Поле $A$ алгебраически замкнуто, если любой $f \in A[x] \setminus A$ имеет в $A$ хотя бы 1 корень.
\end{Def}

\begin{theorem}{}
Следующие условия равносильны. 
\begin{enumerate}
\item $A$ "--- алгебраически замкнуто. 
\item $\forall f \in A[x]\colon \deg f \ge 1$ делится на линейный многочлен. 
\item $\forall f \in A[x]\colon \deg f \ge 1$ имеет $\deg f$ корней (с учетом кратности).
\item $\forall f \in A[x]\colon \deg f \ge 1$ полностью раскладывается на линейные в кольце многочленов.
\end{enumerate}
\end{theorem}

\begin{proof}
$1 \Lra 2$: Следствие теоремы Безу.

$3 \Ra 1$: Очевидно.

$1 \Ra 3$ Индукция по $\deg f$.
\begin{itemize}
\item[База:] $\deg f = 1$: $ax = b$. $x = \frac{b}{a}$ "--- корень.
\item[Переход:] $\deg f \ge 2$.
\[ \exists c \in A\colon \text{$c$ "--- корень $f$ кратности $k \ge 1$} \Ra f = (x - c)^kg \]
По индукционному предположению число корней $g$ равно $\deg g$. 
Все корни $f$, отличные от $c$, это в точности корни $g$, причем той же кратности.
Тогда число корней $f$ есть
\[ C(f) = k + \deg g = \deg f \]

$4 \Ra 2$ Очевидно.

$2 \Ra 4$ Индукция по $\deg f$.
\end{itemize}
\end{proof}
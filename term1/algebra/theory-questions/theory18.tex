\section{Симетрическая группа. Порождение симметрической группы транспозициями}

\begin{Def}
$S_n$ "--- биекции на $\{1, \dotsc, n\}$.
$S_n$ "--- симметрическая группа (группа перестановок) степени $n$.
Произведение двух перестановок "--- это их композиция: $\sigma \cdot \tau = \sigma \circ \tau$,
то есть первой выполняется биекция $\tau$.
\end{Def}

\begin{exmp}

\begin{tabular}{ c c }
  \begin{tabular}{|c|c|c|c|c|c|c|}
  \hline
  $x$       & 1 & 2 & 3 & 4 & 5 & 6 \\ \hline
  $\tau(x)$ & 5 & 1 & 6 & 4 & 2 & 3 \\ \hline
  \end{tabular}
  &
  \begin{tabular}{|c|c|c|c|c|c|c|}
  \hline
  $x$       & 1 & 2 & 3 & 4 & 5 & 6 \\ \hline
  $\sigma(x)$ & 2 & 4 & 5 & 1 & 3 & 6 \\ \hline
  \end{tabular}
  \\
  \rule{0pt}{4ex}
  \begin{tabular}{|c|c|c|c|c|c|c|}
  \hline
  $x$                    & 1 & 2 & 3 & 4 & 5 & 6 \\ \hline
  $(\tau\cdot\sigma)(x)$ & 1 & 4 & 2 & 5 & 6 & 3 \\ \hline
  \end{tabular}
  &
  \begin{tabular}{|c|c|c|c|c|c|c|}
  \hline
  $x$                    & 1 & 2 & 3 & 4 & 5 & 6 \\ \hline
  $(\sigma\cdot\tau)(x)$ & 3 & 2 & 6 & 1 & 4 & 5 \\ \hline
  \end{tabular}
\end{tabular}
\end{exmp}

\begin{Def}
	Цикл $(i_1, i_2, \dotsc, i_k)$ "--- это перестановка $\sigma$ такая, что:
	\begin{gather*}
	\sigma(i_1) = i_2; \\ 
	\sigma(i_2) = i_3; \\
	\dotsc \\
	\sigma(i_k) = i_1; \\
	\sigma(j) = j, j \notin \{i_1, \dotsc, i_k\}
	\end{gather*}
	То есть цикл переводит $i_1\to i_2 \to \dotsc \to i_k \to i_1$ и оставляет остальные элементы на месте.
	$k$ называется длиной цикла.
\end{Def}

\begin{Rem}
    \[(i_1, i_2, \dotsc, i_k) = (i_2, i_3, \dotsc, i_k) = \dotsc = (i_s, i_{s+1}, \dotsc, i_k, i_1, \dotsc, i_{s - 1})\]
\end{Rem}
\begin{Rem}
    Порядок цикла длины $k$ равен $k$. В частности, $(i_1, i_2, \dotsc, i_k)^k = id$.
\end{Rem}

\begin{Def}
	Транспозиция "--- цикл длины 2.
	$(i, j)$ "--- $i$ и $j$ меняются местами, остальные остаются на месте.
\end{Def}

\begin{Def}
	Пусть $(i_1, i_2, \dotsc, i_k)$, $(j_1, j_2, \dotsc, j_s)$ "--- циклы.
	Эти циклы называются незацепляющимися (непересекающимися), если $\{i_1, i_2, \dotsc, i_k\} \bigcap \{j_1, j_2, \dotsc, j_s\} = \varnothing$.
\end{Def}

\begin{Rem}
    Если $\sigma, \tau$ "--- непересекающиеся циклы, то $\sigma\tau = \tau\sigma$, легко перемножать.
\end{Rem}

В билете 17 показывается, что любая перестановка представляется как произведение циклов. Да, тут
как бы идёт 17 билет.

\begin{theorem}{$S_n$ порождается транспозициями}
\end{theorem}
\begin{proof}
Покажем, что любой цикл $(i_1, i_2, \dotsc, i_k)$ есть произведение транспозиций $(i_1 i_2)(i_2 i_3)\dotsc(i_{k - 1}i_k)$.
Заметим, что если применить это произведение к перестановке (вот просто взять и последовательно применить справа налео все транспозиции аккуратно), то мы и получим тот же самый результат, как и от применения исходного цикла.

Осталось лишь заметить, что каждая из остальных $n - k$ точек любо неподвижная, либо также лежит на каком-то цикле "--- а бить циклы на произведение транспозиций мы только что научились.
\end{proof}

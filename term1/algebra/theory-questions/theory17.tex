\section{Теорема о разложении перестановки в произведение непересекающихся циклов}

Все нужные определения находятся в билете 18.

\begin{Def}
	$\sigma \in S_n$. Говорим, что $j$ "--- неподвижная точка относительно $\sigma$, если $\sigma(j) = j$
\end{Def}

\begin{theorem}{Всякая перестановка может быть представлена в виде произведения непересекающихся циклов}
\end{theorem}
\begin{proof}
	Индукция по $m$ "--- числу подвижных точек $\sigma$ (число подвижных точек = $n$ - число неподвижных точек).
	\begin{itemize}
\item \textbf{База}: $m = 0 \Leftrightarrow n $ "--- число неподвижных точек, то есть $\forall j \in \{1, \dotsc, n\} \colon \sigma(j) = j$, то есть $\sigma = id$
\item \textbf{Переход}: $m > 0 \Rightarrow \exists i\colon \sigma(i) \ne i$ -- с него и начнём:
\begin{gather*}
i_1 = i; \\
i_2 = \sigma(i_1); \\
i_3 = \sigma(i_2) = \sigma^2(i_1); \\
\dotsb;
\end{gather*}
И так до тех пор, пока не встретим повторение (а его мы обязательно встретим, потому что чисел у нас всего $n$ "--- конечное число).

$i_1 \mapsto i_2 \mapsto \dotsc \mapsto i_k \mapsto i_{k+1}$ "--- уже встречался.
Заметим, что $i_{k+1} \ne i_j, \forall j \in \{2, \dotsc, n\}$ в силу инъективности $\sigma$ (иначе у какого-то элемента было бы два различных прообраза "--- $i_{j-1}$ и $i_{k+1}$) $\Rightarrow i_{k+1} = i_1$.

Мы получили цикл $\tau = (i_1, i_2, \dotsc, i_k)$. Рассмотрим $\sigma\tau^{-1}$. 

Неподвижные точки $\sigma\tau^{-1}$ "--- это неподвижные точки $\sigma $ плюс $i_1, i_2, \dotsc, i_k \Rightarrow \sigma\tau^{-1}$ и $\tau$ "--- незацепляющиеся $\Rightarrow$ (по индукционному предположению) $\sigma\tau^{-1} = \prod_{j = 1}^r \tau_j$. 

Домножим обе части равества на $\tau$ и получим, что $\sigma = (\prod_{j = 1}^r \tau_j)\tau$ -- произведение незацепляющихся циклов.\\
\end{itemize}
\end{proof}
\begin{theorem}{Следствие}
$S_n$ порождается всеми циклами (так как для каждой $\sigma$ можно выбрать свой набор).
\end{theorem}

\begin{Rem}
Небольшой забавный бонус:
$\sigma = \prod_{j = 1}^r \tau_j$, где $\tau_j$ -- попарно непересекающиеся циклы. Тогда $\sigma^m = (\prod_{j = 1}^r \tau_j)^m$, так как непересекающиеся циклы коммутируют.\\
\end{Rem}

\begin{Def}
Порядок цикла "--- его длина.
\end{Def}
\begin{Rem}
Порядок $\sigma$ "--- наименьшее общее кратное длин циклов при разложении $\sigma$ на непересекающиеся циклы.
\end{Rem}
\begin{Def}
Цикловым типом $\sigma \in S_n$ называется набор длин её непересекающихся циклов, упорядоченных по неубыванию, плюс набор единиц для неподвижных элементов.
\end{Def}

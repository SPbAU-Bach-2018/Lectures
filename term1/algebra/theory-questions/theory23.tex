\section{Степень многочлена}	
\begin{Rem}
	Альтернативное обозначение многочленов (тут $a \in A$, а $x$ "--- переменная):
	\begin{align*}
	a &= (a, 0, 0, \dots); \\
	x &= (0, 1, 0, \dots); \\
	x^i &= (0, \dots, \underbrace{1}_{i-\text{я позиция}}, \dots); \\
	(a_0, a_1, a_2, \dots) &= (a_0, 0, 0, \dots) + (0, a_1, 0, 0, \dots) + \dots =  \\
    &= (a_0, 0, 0, \dots) \cdot (1, 0, 0, \dots) + (0, a_1, 0, 0, \dots) \cdot (0, 1, 0, 0, \dots) + \dots = \\
    &= a_0 + a_1x + a_2x^2 + \dots + a_nx^n
	\end{align*}
	Последняя строка "--- общий способ записать многочлен.
\end{Rem}
	
\begin{Def}
	Альтернативное определение кольца многочленов:
	\[A[x] = \lbrace a_0 + a_1x + \dots + a_nx^n \mid n \in \N \cup \lbrace 0 \rbrace \wedge a_i \in A \rbrace\]
\end{Def}
	
\begin{Def}
	Пусть $f = a_0 + a_1x + \dots + a_nx^n$, $a_n \neq 0$ и $f \neq 0$.
	Тогда $n$ "--- степень многочлена $f$, обозначается $n = \deg f$.
	Если $f = 0$, то положим $\deg f = -\infty$.
\end{Def}
	 
\begin{theorem}{}
	\begin{enumerate}
	\item $deg(f + g) \leq \max(\deg f, \deg g)$
	\item $deg(fg)  \leq \deg f + \deg g$
	\begin{Rem}
		Если $A$ - область целостности, то $\deg (fg) = \deg f + \deg g$
	\end{Rem}
	\end{enumerate}		
\end{theorem}			 

\begin{proof}
	\begin{enumerate}
	\item Следует из доказательства замкнутости относительно сложения (просто сложили и посмотрели на определение степени):
	\begin{gather*}
	f = a_0 + \dots + a_nx^n \wedge a_n \neq 0;\\
	g = b_0 + \dots + b_mx^m \wedge a_m \neq 0;
	\end{gather*}
	\item Раскрыли скобки по дистрибутивности кольца многочленов, обнаружили, что там не может возникнуть
	      $x^k$, где $k > \deg f + \deg g$.
	\item Для области целостности. Рассмотрим $a_n\neq 0$ и $b_m \neq 0$ и пусть $fg$ имеет вид
	\[ c_0x^0 + c_1x^1 + \dots + c_{n+m}x^{n+m} \]
	Так как $c_{n+m}=a_nb_m$, то $c_{n+m} \neq 0$ (иначе есть нетривиальные делители нуля), то
	есть $\deg (fg) = \deg f + \deg g$.
	Если же один из многочленов ноль, то его степень равна $-\infty$, сумма степеней тоже равна
	$-\infty$, а результат ноль $\Ra$ его степень равна $-\infty$.
	\end{enumerate}
\end{proof}	 

\begin{conseq}
	Если $A$ "--- область целостности, то и $A[x]$ "--- область целостности
\end{conseq}
\begin{proof}
	Пусть $f, g \neq 0$, тогда:
	\[
	\deg f, \deg g \geq 0
	\Ra \deg (fg) \geq 0
	\Ra  fg \neq 0	
	\]
\end{proof}
	 
\begin{exmp}
	Возьмём $A = \Z / 4 \Z$ (кольцо остатков по модулю 4, не область целостности)
	положим $f = 2x$, $g = 2x^2$, $fg = 4x^2 = 0$, то есть $A[x]$ "--- не область целостности.
\end{exmp} 
	 
\begin{conseq}
	$A$ "--- область целостности $\Ra (A[x])^* = A^*$ 
\end{conseq}	 
\begin{proof}
\begin{itemize}
\item $\subset$: пусть $fg=1$ (при этом $f,g \neq 0$, иначе точно необратимо), тогда:
	\begin{equation*}
	\left.
		\begin{aligned}
		\deg f + \deg g &= 0 \\
		\deg f &\ge 0 \\
		\deg g &\ge 0
		\end{aligned}
	\right\} \Ra \deg f = \deg g = 0
	\end{equation*}
\item $\supset$: если элемент обратим в кольце $A$, то он обратим и в кольце многочленов
\end{itemize}
\end{proof}

\section{Формула Муавра}

\begin{theorem}{}
$z = r(\cos \phi + i\sin \phi)$, $z \ne 0$, $n \in \Z$. Тогда 
\begin{gather*}
z^n = r^n(\cos n\phi + i\sin n\phi)
\end{gather*}
\end{theorem} 
\begin{proof}
\begin{itemize}
\item Если $n \in \N$, то очевидно из умножения в тригинометрической форме.
\item Если $n = 0$, то написано $1 = 1$.
\item Если $n = -1$:
\begin{gather*} 
z^{-1} = \frac{\bar z}{|z|^2} = \frac{r\overline{(\cos \phi + i \sin \phi)}}{r^2} = r^{-1}(\cos \phi - i \sin \phi) = r^{-1}(\cos(-\phi) + i\sin(-\phi))
\end{gather*}
\item Если $n < 0$:
\begin{gather*}
z^n = \left(z^{-1}\right)^n = \left(r^{-1}(\cos{-\phi} + i \sin{-\phi})\right)^{|n|} = \\ 
= r^{-|n|}(\cos(-|n| \phi) + i \sin(-|n| \phi)) = r^n(\cos{n \phi} + i \sin{n \phi})
\end{gather*}
\end{itemize}
\end{proof}



\section{Аргумент комплексного числа. Тригонометрическая форма записи. Арифметические операции над комплексными числами в тригонометрической форме}

\begin{center}
\def\svgwidth{6.0cm}
\input{theory37-vector.pdf_tex}
\end{center}
$z \in \C,~z = a + bi \Ra (a, b)$ "--- координата в декартовой системе координат. В полярной системе координат два других параметра: $r$ "--- радиус, $\phi$ - угол.
\begin{gather*}
\left\{
\begin{aligned}
	a = r\cos \phi  \\
	b = r\sin \phi
\end{aligned}\right.
\end{gather*}

Пары $(r, \phi)$ и $(r, \phi + 2 \pi k)$ определяют одну и ту же точку на комплексной плоскости.
\begin{gather*}
\sim\colon \phi_1 \sim \phi_2 \Lra \phi_1 - \phi_2 = 2 \pi k, k\in \Z
\end{gather*}
Это эквиваленция. Тогда все равнозначные значения образуют классы эквиваленции.
\begin{Def}
Аргументом комплексного числа $z$ называют множество возможных значений $\phi$. Главным аргументом называют представителя этого множества из диапазона $[0,2\pi)$
\begin{gather*}
\Arg z = [\phi] = \left\{\phi \Re z = |z| \cos \phi \land \Im z = |z| \sin \phi\right\}\\
\arg z \in [\phi] \cap [0,2\pi)
\end{gather*}
\end{Def}
\begin{Rem}
Собственно, в каком полуинтревале выбирать главный аргумент "--- вопрос выбора.
\end{Rem}

Пусть $z = a + bi$, $|z| = \sqrt{a^2 + b^2}$. $arg z = ?$:
\begin{enumerate}
\item $a > 0, b \ge 0$ 
\begin{gather*}
\frac{b}{a} = \tg \phi, \phi \in \left[0, \frac{\pi}2\right) \Ra \arg z = \arctg \frac{b}{a}
\end{gather*}
\item $a > 0, b < 0$ 
\begin{gather*}
\frac{b}{a} = \tg \phi, \phi \in \left(\frac{3\pi}2, 2\pi\right) \Ra \arg z = 2\pi + \arctg \frac{b}{a}
\end{gather*}
\item $a < 0$ 
\begin{gather*}
\frac{b}{a} = \tg \phi, \phi \in \left(\frac{\pi}2, \frac{3\pi}2\right) \Ra \arg z = \pi + \arctg \frac{b}{a}
\end{gather*}
\item $a = 0, b > 0$
\begin{gather*}
\arg z = \frac\pi2
\end{gather*}
\item $a = 0, b < 0$
\begin{gather*}
\arg z = \frac{3\pi}2
\end{gather*}
\end{enumerate}

\begin{Def}{Тригонометрическая форма записи числа}
\begin{gather*}
z = a + bi = r\cos \phi + ir\sin \phi = r(\cos \phi + i\sin \phi)
\end{gather*}
где $r$ "--- модуль ($r \ge 0$), а $\phi$ "--- аргумент комплексного числа.
\end{Def}
\begin{gather*}
|\cos \phi + i \sin \phi| = \sqrt{\cos^2 \phi + \sin^2 \phi} = 1
\end{gather*}

В тригинометрической форме числа трудно складывать и вычитать, но легко умножать и делить: 
$z_1 = r_1(\cos \phi_1 + i \sin \phi_1)$, $z_2 = r_2(\cos \phi_2 + i \sin \phi_2)$. Тогда:
\begin{gather*}
z_1z_2 = r_1r_2(\cos \phi_1 \cos \phi_2 - \sin \phi_1 \sin \phi_2 + i(\cos \phi_1 \sin \phi_2 + \cos \phi_2 \sin \phi_1)) = \\
= r_1r_2(\cos(\phi_1 + \phi_2) + i \sin(\phi_1 + \phi_2)) \\
|z_1z_2| = r_1r_2 = |z_1||z_2| \quad \Arg (z_1z_2) = \Arg z_1 + \Arg z_2
\end{gather*}

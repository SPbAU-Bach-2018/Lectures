\section{Кольца, тела, поля}
\begin{Def}
Есть $A \neq \varnothing$, на нём заданы две бинарные операции:
\begin{enumerate}
\item $+: A \times A \to A$ (сложение)
\item $\cdot: A \times A \to A$ (умножение)
\end{enumerate}

Определим интересные свойства, которыми такая структура $\left<A, +, \cdot\right>$ может обладать:

\begin{enumerate}
\item Ассоциативнсть сложения:
	\[ \forall a, b, c \in A \colon (a + b) + c = a + (b + c) \]
\item Существование нейтрального элемента по сложению (обозначается $0$):
	\[ \exists 0 \in A, \forall a \in A \colon a + 0 = 0 + a = a \]
\item Существование обратного элемента по сложению (обозначается $-a$):
	\[ \forall a \in A, \exists (-a) \in A \colon a + (-a) = (-a) + a = 0 \]
\item Коммутативность сложения:
	\[ \forall a, b \in A \colon a + b = b + a \]
\item Ассоциативность умножения:
	\[ \forall a, b, c \in A \colon (a \cdot b) \cdot c = a \cdot (b \cdot c) \]
\item Коммутативность умножения:
	\[ \forall a, b \in A \colon a \cdot b = b \cdot a \]
\item Существование нейтрального элемента по умножению (обозначается $1$):
	\[ \exists 1 \in A, \forall a \in A \colon a \cdot 1 = 1 \cdot a = a \]
\item Существование обратного элемента по умножению (обозначается $a^{-1}$):
 	\[ \forall a \in A \setminus \{0\}, \exists a^{-1} \in A \colon a \cdot a^{-1} = a^{-1} \cdot a = 1 \]
 	Также иногда говорят про левый обратный $a^{-1}_l$ ($a^{-1}_la = 1$) и правый обратный $a^{-1}_r$ ($aa^{-1}_r=1$),
 	но если существуют оба, то они совпадают (см. свойства).
\item
    \begin{itemize}
	\item Дистрибутивность слева: $ \forall a, b, c \in A \colon a \cdot (b + c) = a \cdot b + a \cdot c $\\
	\item Дистрибутивность справа: $ \forall a, b, c \in A \colon (a + b) \cdot c = a \cdot c + b \cdot c $\\
	\end{itemize}
	Просто <<дистрибутивность>> "--- дистрибутивность и слева, и справа.
\end{enumerate}
\end{Def}

\begin{Def}
	Кольцо "--- тройка $\left<A, +, \cdot\right>$, удовлетворяющая свойствам 1--5, 9 (a, b)
\end{Def}
\begin{Def}
	Кольцо, в котором выполнено свойство 6 "--- коммутативное кольцо
\end{Def}
\begin{Def}
	Кольцо, в котором выполнено свойство 7 "--- кольцо с единицей
\end{Def}
\begin{Def}
	Тело "--- кольцо с $1$, в котором $1 \neq 0$ и выполнена аксиома 8 (то есть всё, кроме 6)
\end{Def}
\begin{Def}
	Поле "--- коммутативное кольцо с $1$, в котором $1 \neq 0$ и выполнена аксиома 8 (т.е. все 9 аксиом)
\end{Def}

\begin{Rem}
	Иногда кольца, для которых выполнены аксиомы 1--4 и 9, называют ассоциативными кольцами
\end{Rem}

\begin{Rem}
Если $\left<A, +, \cdot\right>$ "--- кольцо, то $\left<A, +\right>$ "--- абелева группа
\end{Rem}

\textbf{ Примеры: }
\begin{itemize}
\item $\Z$ "--- коммутативное кольцо с $1$, но $2$ не имеет обратного в $\Z$ $\Ra$ не поле
\item $\mathbb{N}$ "--- не кольцо
\item $2\Z$ (все чётные целые числа) "--- кольцо без $1$
\item $\Q, \R$ - поля
\end{itemize}

\subsection{Простейшие свойства колец}

\begin{enumerate}
\item $0$ единственен, так как $0 = 0 + \cdot 0' = 0'$
\item $-a$ единствененю
	\begin{proof}
	Пусть существуют два ($b$ и $c$), являющихся обратными по сложению к $a$:
	\begin{gather*}
		a + b = b + a = a + c = c + a = 0; \\
		a + b = a + c = 0; \\
		(b + a) + b = (b + a) + c = 0; \\
		0 + b = 0 + c ; \\
		b = c; 
	\end{gather*}
	\end{proof}
\item $1$ единственна (если существует), так как $1 = 1 \cdot 1' = 1'$
\item Если у $a$ есть и левый, и правый обратный по умножению, то они совпадают: \[a^{-1}_l = a^{-1}_l(aa^{-1}_r) = (a^{-1}_la)a^{-1}_r = a^{-1}_r\]
\item Если в кольце с 1 у элемента $a$ есть 2 левых обратных, то левых обратных к $a$ бесконечно много (упражнение).
Подсказка: построим инъективное отображение множества левых обратных в своё подмножество. Говорят, это может
быть сложно даже в качестве задачи на пятёрку, потому что надо творчески построить отображение.
	
\item $0 \cdot a = a \cdot 0 = 0 $
	\begin{proof}
		\begin{align*}
		a \cdot 0 + a \cdot 0 &= a(0 + 0) = a\cdot0 & \text{добавим} -(a\cdot0); \\
		a \cdot 0 + a \cdot 0 + (-(a \cdot 0)) &= a \cdot 0 + (-(a \cdot 0)); \\
		a \cdot 0 &= 0; \\
		\end{align*}
		Аналогичным образом с другой стороны.
	\end{proof}
\item $ a(-b) = (-a)b = -(ab)$
	\begin{proof}
		\begin{gather*}
		a + (-a) = 0; \\
		ab + (-a)b = (a + (-a))b = 0 \cdot b = 0; \\
		-(ab) + ab + (-a)b = -(ab) + 0; \\
		(-a)b = -(ab); \\
		\end{gather*}
		Аналогичным образом с другой стороны.
	\end{proof}
\item Если $0 = 1$, то $|A| = 1$, $A = \{0\}$:
	\begin{proof}
		Рассмотрим произвольный $a \in A$:
		\[a = 1 \cdot a = 0 \cdot a = 0\]
	\end{proof}
\end{enumerate}

\begin{Def}
Пусть $A$ "--- кольцо (тело, поле).
Тогда непустое $B \subset A$ "--- подкольцо (подтело, подполе), если является кольцом (телом, полем) относительно сужения операций на $B$
\end{Def}


\begin{Rem}
\begin{itemize}
	\item $A \supset B \neq \varnothing$ "--- подкольцо в $A$, если оно замкнуто относительно умножения, сложения, взятия обратного по сложению
	\item $B$ "--- подтело, если оно подкольцо и замкнуто по взятию обратного ненулевого элемента по умножению и содержит элементы, отличные от нуля, то есть:
	\begin{gather*}
	\forall a, b \in B \colon a + b \in B; \\
	\forall a \in B \colon (-a) \in B; \\
	\forall a, b \in B \colon ab \in B; \\
	\forall a \in (B \setminus \{0\}) \colon a^{-1} \in B; \\
	\exists a \in B \colon a \neq 0;
	\end{gather*}
\end{itemize}
\end{Rem}

\subsection{Гомоморфизмы колец}

\begin{Def}
	Пусть $A$, $B$ "--- кольца и есть отображение $f: A \to B$.
	$f$ "--- гомоморфизм, если:
	\begin{enumerate}	
	\item $\forall a_1, a_2 \in A \colon f(a_1 + a_2) = f(a_1) + f(a_2)$
	\item $\forall a_1, a_2 \in A \colon f(a_1a_2) = f(a_1)f(a_2)$
	\end{enumerate}
\end{Def}

\begin{Def}		
	$f$ "--- изоморфизм, если $f$ "--- и гомоморфизм, и биекция
\end{Def}

\begin{Def}		
	$A$ и $B$ изоморфны, если между ними сущесвует изоморфизм,
	обозначение: $A \cong B$.
\end{Def}
\begin{Rem}
	Если $f$ "--- гомоморфизм и $f(0_A)$ обратим по сложению, то $f(0_A) = 0_B$
\end{Rem}
\begin{proof}
	$f(0_A) = f(0_A + 0_A) = f(0_A) + f(0_A)$, добавили обратный по сложению к $f(0_A)$,
	получили $0_B = f(0_A)$.
\end{proof}

\begin{Rem}
	Если $f$ "--- гомоморфизм и $f(1_A)$ обратим в $B$ то $f(1_A)=1_B$
\end{Rem}
\begin{proof}
	$f(1_A) = f(1_A \cdot 1_A) = f(1_A)f(1_A)$, домножаем на обратный к $f(1_A)$,
	получаем $1_B = f(1_A)$.
\end{proof}
	
\subsection{Делимость в кольцах}

\begin{Def}
Пусть $A$ "--- кольцо и $a, b, c \in A$ таковы, что $c = ab$.
Тогда $a$ "--- левый делитель $c$, $b$ --- правый делитель $c$.
\end{Def}

\begin{Rem}
Любой элемент является и левым, и правым делителем нуля: $0 = a \cdot 0 = 0 \cdot a$.
\end{Rem}
    
\begin{Def}
	$a$, $b$ "--- нетривиальные делители нуля, eсли $0 = ab$, $a \neq 0$, $b \neq 0$.
\end{Def}
	
\begin{Def}	
	Область целостности "--- коммутативное кольцо с $1$, в котором отсутствуют нетривиальные делители нуля.
\end{Def}

\begin{Rem}
    В области целостности если $ab=0$, то либо $a=0$, либо $b=0$.
\end{Rem}
\begin{Rem}
	Поле "--- область целостности (потому что не-ноль всегда можно домножить на обратный).
\end{Rem}
	
\begin{theorem}{}
	$A$ "--- область целостности и $a \in A \setminus \{ 0 \}$. Тогда
	\[ab = ac \Ra b = c\]
\end{theorem}
	
\begin{proof}
	\begin{gather*}
	\underbrace{a}_{\neq 0}(b - c) = 0; \\
	a^{-1}a(b - c) = a^{-1} \cdot 0; \\
	(b - c) = 0; \\
	b - c + c = 0 + c; \\
	b = c; \\
	\end{gather*}
\end{proof}

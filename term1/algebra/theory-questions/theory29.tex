\section{Число корней многочлена}
\begin{lemma}
$A$ "--- область целостности. $0 \ne f, g \in A[x]$. $c$ "--- корень $f$ кратности $k$, корень $g$ кратности $l$.
Тогда $c$ "--- корень $fg$ кратности $k + l$.
\end{lemma}
\begin{proof}
\begin{gather*}
f = (x - c)^kf_1, f_1(c) \ne 0 \\
g = (x - c)^lg_1, g_1(c) \ne 0 \\
fg = (x - c)^{k + l}f_1g_1, f_1(c)g_1(c) \ne 0
\end{gather*}
\end{proof}

\begin{lemma}
$A$ "--- область целостности.
Какие бы ни были $c \ne d \in A$, $0 \ne f, g \in A[x]$, $a, k \in \N$, такие, что $f = (x - c)^{k}g$ и $g(c) \ne 0$,
\[ (x - d)^{a} \mid f \Lra (x - d)^{a} \mid g \]
\end{lemma}

\begin{proof}
\begin{itemize}
\item[$\La$:]
\[ (x - d)^{a} \mid g \Ra (x - d)^{a} \mid f \]
\item[$\Ra$:]
Индукция по $a$.
\begin{itemize}
\item[База:] $a= 1$
\begin{gather*}
x - d \mid f \Ra f(d) = 0 \\
(c - d)^k g(d) = 0 \Ra g(d) = 0 \Ra (x - d) \mid g
\end{gather*}
\item[Переход:] $a - 1 \to a$. 
$a - 1$ для всех $f$ и $g$ удовлетворяет условию леммы.
\begin{gather*}
f = (x - c)^{k}g \\
(x - d)^{a} \mid f \Ra (x - d)^{a - 1} \mid f \\
f = (x - d)^{a}f_1 \\
g = (x - d)^{a - 1}g_1 \\
(x - d)^af_1 = (x - c)^{k}(x - d)^{a - 1}g_1 \\
(x - d)f_1 = (x - c)^kg_1 \\
\Ra (x - d) \mid g_1
\end{gather*}
\end{itemize}
\end{itemize}
\end{proof}

\begin{theorem}{}
$A$ "--- область целостности. $0 \ne f \in A[x]$. Тогда число корней $f$ с учетом кратности не превосходит $\deg f$.
\end{theorem}
\begin{proof}
Индукция по $\deg f$
\begin{itemize}
\item[База:] 
$\deg f = 0$, $f = const \ne 0$ "--- нет корней.
\item[Переход:] 
$c$ "--- корень $f$ кратности $k$. $f = (x - c)^{k}g$, $g(c) \ne 0$, $c$ "--- не корень $g$.
Все корни $g$ "--- это в точности все корни $f$, кроме $c$, причем кратность сохраняется. Тогда число корней $g$ с учетом кратности не более $\deg g$. Тогда число корней $f$ 
\[ C(f) = k + C(g) \le k + \deg g = \deg f \]
\end{itemize}
\end{proof}

\begin{Rem}
Предположение, что $A$ "--- область целостности, существенно.
\end{Rem}

\begin{Def}
$A$, $f \in A[x]$.
\[ \tilde f\colon A \to A\colon c \mapsto f(c) \]
\end{Def}
\begin{exmp}
$A = \F_2$
\begin{gather*}
f = 0, g = x^2 + x \\
\tilde f\colon 0 \mapsto 0, 1 \mapsto 0 \\
\tilde g\colon 0 \mapsto 0, 1 \mapsto 0
\end{gather*}
\end{exmp}

\begin{conseq}
$A$ "--- область целостности. $f, g \in A[x]$, $|A| > \max\{\deg f, \deg g\}$.
Тогда, если $\tilde f = \tilde g$, то $f = g$.
\end{conseq}
\begin{proof}
Рассмотрим $f - g \ne 0$. Тогда $\tilde f - \tilde g$ "--- тождественно не нулевое отображение.
\[ \forall c \in A, f(c) - g(c) = 0 \]
Тогда число корней $f - g$
\[ C(f-g) > \deg (f - g) \]
Тогда $f - g = 0$.
\end{proof}

\begin{conseq}
Если $A$ "--- область целостности, $|A| = \infty$ и $\tilde f = \tilde g$, то и $f = g$.
\end{conseq}
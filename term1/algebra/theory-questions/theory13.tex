\section{Гомоморфизмы групп. Свойства гомоморфизмов}
\begin{Def}
	$H$, $G$ "--- группы, $f: G \to H$
	\begin{enumerate}
		\item $f$ "--- гомоморфизм, если $\forall a, b \in G \colon f(ab) = f(a)f(b)$
		\item $f$ "--- изоморфизм, если $f$ "--- и гомоморфизм, и биекция
	\end{enumerate}
\end{Def}

\begin{Def}
	$H$, $G$ "--- группы. Если между $H$, $G$ есть изоморфизм, то группы называются изоморфными: $H \cong G$
\end{Def}						

\begin{theorem}{Свойства гомоморфизма}
	\begin{enumerate}
		\item $f(e_G) = e_H$
		\item $f(x^{-1}) = (f(x))^{-1}$
		\item $f(G)$ "--- подгруппа $H$
		\item $G \xrightarrow{f} H \xrightarrow{g} K$, $f$, $g$ "--- гомоморфизмы, тогда $g \circ f: G \to K$ "--- тоже гомоморфизм
		\item $f: G \to H$ "--- изоморфизм, тогда $f^{-1}: H \to G$ "--- изоморфизм
	\end{enumerate}
\end{theorem}
\begin{proof}
	\begin{enumerate}
		\item $f(e_G) = f(e_G e_G) = f(e_G)f(e_G) = e_H e_H = e_H$
		\item $f(e_G) = f(x x^{-1}) = f(x)f(x^{-1}) = e_H \Ra f(x^{-1}) = (f(x))^{-1}$
		\item
		    Покажем замкнутость относительно операции $\cdot$
		    \begin{gather*}
			e_H \in f(G); \\
			c, d \in f(G) \iff \exists a, b \in G \colon c = f(a), d = f(b); \\
			cd = f(a)f(b) = f(ab) \Ra cd \in f(G); \\
			\end{gather*}

			Покажем замкнутость относительно взятия обратного (из пункта 2 теоремы):
			\[f(a)^{-1} = f(a^{-1}) \in f(G)\]

			Таким образом, $f(G)$ "--- подгруппа.
		\item Пусть $a, b \in G$, тогда:
			\[(g \circ f)(ab) = g(f(ab)) = g(f(a)f(b)) = g(f(a))g(f(b)) = (g \circ f)(a)(g \circ f)(b)\]
		\item
			\begin{gather*}
			c, d \in H; \\
			\exists a, b \in G\colon c = f(a); d = f(d); \\
			a = f^{-1}(c); b = f^{-1}(d); \\
			f^{-1}(cd) = ab = f^{-1}(c)f^{-1}(d);
			\end{gather*}
			Тогда $f^{-1}$ "--- изоморфизм
	\end{enumerate}   
\end{proof}		

\begin{conseq}
    \begin{enumerate}
    \item  $G \cong G$ (взяли $id_G$).
    \item  $G \cong H \Ra H \cong G$
    \item  $G \cong H$, $H \cong K$ $\Ra$ $G \cong K$
    \end{enumerate}

\end{conseq}

\begin{exmp}
Пусть $G=\left< \mathbb{R}, +\right>$, $H=\left< \mathbb{R}_+, \cdot\right>$, тогда $x \ra e^x$ "--- изоморфизм.
\end{exmp}

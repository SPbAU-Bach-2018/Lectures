\setauthor{Егор Суворов}

% 22.05.2015
%\section{Взаимная индукция}
%\subsection{Конденсатор с катушкой}
	Теперь давайте опишем это дело формулами.
	В любой момент времени напряжение на катушке и на конденсаторе равны, так как они соединены проводами:
	\[ U_L(t) = U_C(t) \]
	Теперь вспомним, что такое напряжение на конденсаторе:
	\[ U_C(t) = \frac{q(t)}{C} \]
	И что такое напряжение на катушке:
	\[ U_L(t) = -L \frac{\d I(t)}{\d t}\]
	(напоминаем, что знак минус из-за того, что катушка стремится противодействовать идущему току).
	Вспоминаем определение тока: это количество зарядов, прошедших через проводник:
	\[ U_L(t) = -L \frac{\d}{\d t}\left(\frac{\d q}{\d t}\right) = -L \frac{\d^2 q}{\d t ^2}\]
	Приравниваем:
	\begin{align*}
	-L \frac{\d^2 q}{\d t ^2} &= \frac{q(t)}{C} \\
	\frac{\d^2 q}{\d t ^2} + \frac{1}{LC} q(t) &= 0 \\
	\end{align*}
	Получилось дифференциальное уравнение на $q(t)$,
	мы его уже знаем "--- это уравнение гармонических колебаний (знакомились в механике).
	Найдём решение и подставим:
	\begin{align*}
	q(t) &= q_0 \sin(\omega t + \phi) \\
	\dot q(t) &= q_0 \omega \cos(\omega t + \phi) \\
	\ddot q(t) &= -q_0 \omega^2 \sin(\omega t + \phi) \\
	-q_0 \omega^2 \sin(\omega t + \phi) &+ \frac{1}{LC} q_0 \sin(\omega t + \phi) = 0 \\
	-\omega^2 &+ \frac{1}{LC} = 0 \\
	\omega^2 &= \frac{1}{LC} \\
	\end{align*}
	$\omega$ "--- это частота колебаний, период колебаний "--- $\frac{2\pi}{\omega}$.
	Что такое $q_0$?
	Это исходный заряд на конденсаторе, то есть $q_0 = CU_0$.
	А фазу $\phi$ удобно брать равной $\pi 2$, чтобы всё началось в момент времени $t=0$.
	\[
	q(t) = CU_0 \sin (\omega t + \frac \pi 2) = CU_0 \cos (\omega t)
	\]
	Ток, как мы помним, есть производная заряда:
	\[
	I(t) = -CU_0 \omega \sin (\omega t)
	\]
	Минус соответствует тому, что в нулевой момент времени заряд с конденсатора утекает и ток отрицательный.
	% график для тока
	Можно еще проверить, что энергия действительно сохраняется:
	\begin{gather*}
	E_C = \frac{q^2}{2C} = \frac C 2 U_0^2 \cos^2 (\omega t ) \\
	E_L = \frac{LI^2}{2} = \frac{LC^2U_0^2}2 \sin^2 (\omega t) = \frac C 2 U_0^2 \sin^2 (\omega t) \\
	E_C + E_L = \frac C 2 U_0^2
	\end{gather*}
	% двойной график для энергий конденсатора и катушки

	\subsubsection{Радио}
		Если есть внешнее электрическое поле, то в конденсаторе это поле меняет напряжение,
		получается изменяющийся ток.
		Получается что-то вроде примера из механики: грузик на пружине, к которому прикладывают изменяющуюся силу.
		То есть если удачно совпадёт частота, то получим резонанс и усиление внешнего поля.
		Почти так работает радио: берём внешнее поле и усиливаем.

\subsection{Два конденсатора}
	% картинка: заменили катушку на конденсатор
	% у левого плюс сверху, минус снизу, ток пойдёт по часовой стрелке
	Пусть левый конденсатор изначально заряжем.
	Подсоединяем правый конденсатор (той же ёмкости), изначально никак не заряженный.
	Вопрос: что будет?

	Давайте посчитаем.
	Попробуем точно так же: мы знаем, что напряжения равны: $U_L=U_R$.
	Так как $U=\frac q C$, то $q_L=q_R$.
	Время куда-то потерялось, нехорошо.
	Утверждается, что это из-за того, что мы пренебрегли сопротивлением провода,
	а именно оно не позволит мгновенно сравнять заряды.
	% картинка: добавили резистор
	Теперь имеем $U_L=U_R + IR$ (последнее слагаемое "--- напряжение на резисторе).
	Появилось время:
	\[
		\frac{q_L}{C} = \frac{q_R}{C} + \frac{\d q_R}{\d t}R
	\]
	Также знаем, что $q_L + q_R = CU_0$ (поскольку заряд никуда не девался).
	Решим систему, получим диффур и решим, разделив переменные:
	\begin{gather*}
		\frac{CU_0}{C} - \frac{q_R}{C} = \frac{q_R}{C} + \frac{\d q_R}{\d t} R \\
		U_0 = \frac{2q_R}{C} + \frac{\d q_R}{\d t} R \\
		\d t = \frac{\d q_R}{\frac{U_0}{R} - \frac{2q_R}{RC}} \\
		\int_0^T \d t = \int_0^Q \frac{\d q_R}{\frac{U_0}{R} - \frac{2q_R}{RC}} \\
		T = -\frac {RC} 2 \ln \left(\frac{U_0}{R} - \frac{2q_R}{RC}\right) \mid|_0^Q \\
		T = -\frac {RC} 2 \left( \ln \left(\frac{U_0}{R} - \frac{2q_R}{RC}\right) - \ln \left(\frac{U_0}{R}\right) \right) \\
		T = -\frac {RC} 2 \ln \left(1 - \frac{2q_R}{U_0C}\right) \\
		\frac{-2T}{RC} = \ln \left(1 - \frac{2q_R}{U_0C}\right) \\
		e^{\frac{-2T}{RC}} = 1 - \frac{2q_R}{U_0C} \\
		\frac{2q_R}{U_0C} = 1 - e^{\frac{-2T}{RC}} \\
		q_R = \left(1 - e^{\frac{-2T}{RC}}\right) \cdot \frac{U_0C}{2} \\
	\end{gather*}
	В момент времени ноль получаем нулевой заряд на правом конденсаторе, а на бесконечности экспонента быстро зануляется и получаем равномерное распределение заряда.
	% график с двумя зарядами конденсаторов

\subsection{Один конденсатор и переменный ток}
	Есть конденсатор, есть однородное внешнее электрическое поле $E=E_0 \sin \omega t$, направленное перпендикулярно обкладкам.
	Вопрос: что будет внутри?

	Если есть изменяющееся электрическое поле, то появляется магнитное поле.
	Третье уравнение Максвелла говорит:
	\[ \oint \vec E \cdot \d \vec l = -\partd{}{t} \Phi \]
	Но это не очень поможет, оно может вычислить электрическое по магнитному, а мы хотим наоборот.
	Зато поможет четвёртое:
	\[ \oint \vec B \cdot \d \vec l = \underbrace{\frac{4\pi}{c} \vec j}{=0} + \frac1c \partd{}{t}\int \vec E \cdot \d \vec S\]
	Второе слагаемое ноль, так как у нас никакие заряды внутри конденсатора не переносятся.
	Соорудим контур радиуса $r$, лежащий в той же плоскости, что и обкладки конденсатора:
	% картинка

	Подумаем, какое вообще может быть магнитное поле по направлению.
	Так как электрическое поле всегда вертикально, то магнитное поле закручивается вокруг него,
	то есть параллельно контуру.
	Значит, в левой части уравнения у нас поле, умноженное на длину окружности (скалярное произведение равно единице).
	\begin{gather*}
	B \cdot 2\pi r = \frac1c \partd{}{t} E_0 \sin (\omega t) \cdot \pi r^2 \\
	B = \frac{\omega r}{2c} E_0 \cos (\omega t)
	\end{gather*}
	Почти успех.
	Но, вообще говоря, это магнитное поле порождает новое электрическое.
	Посчитаем по третьему закону Максвелла.
	Для этого надо выбрать контур.
	Возьмём в качестве контура прямоугольник, проходящий через ось конденсатора и край:
    %     │
    %  ───┴───
    %     ╔══╗     высота - h, ширина контура - r, циркуляция по часовой стрелке
    % o   ╚══╝ x
    %  ───┬───
    %     │
    Верхнее и нижнее ребро взаимоуничтожатся.
    На левом ребре поле будет ноль: взяли симметричный относительно центра контур, на нём циркуляция точно ноль, устремили его ширину к нулю, получаем, что поле не может иметь какой-то знак, отличный от нуля.
    % дорисовать картинку
    Значит, вклад даёт только правое ребро, вклад равен $- E(r) \cdot h$.
    Получили:
	\begin{align*}
	-E(r) h &= -\frac1c \partd{}{t} \int B \cdot \d \vec S = -\frac1c \partd{}{t} h \int_0^r B(r) \d r = \\
		&= -\frac1c \partd{}{t} \left( h \frac{\omega r^2}{4c}E_0 \cos \omega t \right) = \\
		&= \frac{\omega^2 r^2}{4c^2} E_0 \sin ( \omega t ) \cdot h
	\end{align*}
	Итого получили, что первая итерация нам дала поправку к электрическому полю:
	\[
	E(r) = - \frac{\omega^2 r^2}{4c^2} E_0 \sin ( \omega t )
	\]
	Полное поле получается
	\[
	E(r) = E_0 \sin (\omega t) \left(1 - \frac{\omega^2r^2}{4c^2} \right)
	\]
	Но надо провести есть бескеонечность итераций: ведь поправка тоже создаёт магнитное поле и так далее...
	Давайте хотя бы поймём, что даёт вторая поправка.
	На самом деле, можно просто поправку в скобках <<запихнуть>> в $E_0$ и применить такой же алгоритм.
	Будьте осторожны: из интегрирования по $r$ вылезут факториалы в знаменателе.
	Итого получим ряд:
	\[
	E(r) = E_0 \sin (\omega t) \left(1 - \frac{1}{(1!)^2} \left(\frac{\omega r}{2c} \right)^2 + - \frac{1}{(2!)^2} \left(\frac{\omega r}{2c} \right)^4 - \dots \right)
	\]
	Этот ряд сходится, но его так просто не просуммировать.
	Называется функцией Бесселя нулевого рода, обозначается так:
	\[
	J_0(x) = 1 - \frac{1}{(1!)^2} \left(\frac x 2\right) ^ 2 + \frac{1}{(2!)^2} \left(\frac x 2\right) ^ 4 - \dots
	\]
	Например, в каждый момент времени у нас электрическое поле выглядит вот так:
	% картинка с графиком

	Что забавно, есть и положительное, и отрицательное, и даже нулевое электрическое поле внутри конденсатора.
	А всё из-за того, что мы конденсатор облучили внешним гармонически колеблющимся электрическим полем.

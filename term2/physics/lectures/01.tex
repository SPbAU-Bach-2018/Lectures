\chapter{Заряды и поля}
\setauthor{Анастасия Гайдашенко}

\begin{Def}
Электрический заряд "--- характеристика тела, определяющая силу электрического взаимодействия (способность вступать в электрическое взаимодействие).
\end{Def}

Заряды бывают:
\begin{itemize}
\item <<+>>, положительные (заряд больше нуля)
\item <<–>>, отрицательные (заряд меньше нуля)
\item <<0>>, нейтральные   (заряд равен нулю)
\end{itemize}

Возможные взаимодействия материальных точек (заряженных):
\begin{itemize}
\item притягиваются (разноименные заряды)
\item отталкиваются (одноименные заряды)
\end{itemize}
Это свойство, позволяющее ввести различие между зарядами.

Из соображений симметрии, сила, возникающая между зарядами, может действовать только по оси, которая соединяет заряды.

По третьему закону Ньютона можем определить силу $F_21$, если знаем $F_12$: $\overrightarrow{F_21} = – \overrightarrow{F_12}$

В ходе эксперимента можно установить, что возникающая между зарядами сила прямо пропорциональна зарядам и обратно пропорциональна квадрату расстояния между ними, "--- закон Кулона.

С системе СГС: $|\overrightarrow{F_12} | = |\overrightarrow{F_21}| = \frac{|q_1||q_2|}{r^2}$

Как второй заряд чувствует наличие первого?

Поле "--- субстанция, окружающая все заряженные тела. Обозначается: $\overrightarrow{E}$

Фотоны "--- «переносчики» электрического поля (двигаются не быстрее скорости света).

Сила, действующая на заряд со стороны электрического поля: $\overrightarrow{E}q = \overrightarrow{F}$

Электрическое поле одного точечного заряда: $\overrightarrow{E} = \frac{q}{r^2}$

Электрические поля подчиняются принципу суперпозиции: складываются, но между собой не взаимодействуют: $\overrightarrow{E_p} = \sum{\overrightarrow{E_i}}$

Таким образом, мы можем посчитать электрическое поле не только от точечных зарядов, но и от заряженных тел (для этого разобьем на маленькие кусочки).

\begin{exmp}
  Палочка длины $L$ с зарядом $Q > 0$. Точка $P$, на расстоянии $x$ от правого конца палочки.
  Решение: разбиваем на кусочки длины $dl$, тогда: $dE = \frac{\frac{Qdl}{L}}{(l + x)^2}$; $E_p = \int{dE}_0^L = \int{\frac{Q}{L}\frac{dl}{(l + x)^2}}_0^L = \frac{Q}{L}\int{\frac{dl}{(l + x)^2}}_0^L  = \frac{Q}{L}(\frac{1}{x} – \frac{1}{x + L}) = \frac{Q}{x(x + L)}$
\end{exmp}

\img{1}

Силовые линии всегда выходят из положительных зарядов и заканчиваются в отрицательных. Силовая линия проходит так, чтобы касательная к ней была направлением силы (если $\overrightarrow{F} = 0$, то силовой линии нет).

\begin{exmp}
\img{2} $E = \frac{2q}{x^2} – \frac{q}{(x + L)^2} = 0 \Ra x = – \frac{L}{sqrt(2) – 1}$
\end{exmp}

\img{3}

\img{4}

Густота силовых линий пропорциональна модулю электрического поля: $\frac{N}{S} = n \sim |\overrightarrow{E}|$

\img{5}

Число линий, прошедших через площадки $S_1$ и $S_2$, – одинаково. $n_1 = \frac{N}{S_1}; n_2 = \frac{N}{S_2}; \frac{n_1}{n_2} = \frac{S_2}{S_1} = (\frac{r_2}{r_1})^2 \Ra n \sim \frac{1}{r^2} \sim E$

\img{6}

Поток электрического поля: $N = nS \sim |\overrightarrow{E}|S$ "--- количество силовых линий, прошедших через нашу площадку. $\overrightarrow{E}\overrightarrow{S} = ES\cos{\alpha} = \Phi$

\img{7}

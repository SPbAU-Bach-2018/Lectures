\setauthor{Егор Суворов}

% 27.03.2015
\section{Закон Ампера}
	Цель: хотим вычислять магнитное поле проще, чем по закону Био-Савара-Лапласа.

	Мы умеем считать магнитное поле вокруг конечного провода, но оно не зависит от длины $2L$.
	Пусть провод бесконечный.
	Тогда у нас синус (или косинус?) вырождается в единицу и получаем $B=\frac{2I}{CR}$.
	При этом нам уже неважно, где находится точка "--- она всегда лежит на каком-то серединном перпендикуляре.
	Вокруг длинного провода поле выглядит закручено вокруг него:
	% картинка, в каждой точке окружности вокруг провода есть поле-касательная
	% окружность обозначена C
	Легко посчитать интеграл по окружности (потому что поле является касательной окружности ):
	\[ \oint_C \vec B \cdot \d \vec l = B \cdot 2\pi R = \frac{4\pi}{C} I\]
	Утверждается, что это формула верна всегда, в некотором смысле независимо
	от формы провода и контура интегрирования.
	Чуть точнее: возьмём поверхность, через неё течёт какой-то ток:
	% картинка: взяли поверхность, посчитали через неё ток I, проинтегрировали магнитное поле вокруг контура поверхности
	Если мы проинтегрируем магнитное поле по контуру поверхности и подставим определение тока, то получим:
	\[ \oint_С \vec B \cdot \d \vec l = \frac{4\pi}{C}I = \frac{4\pi}{C} \oint \vec j \cdot \vec n \cdot \d S\]
	\begin{Def}
		$\oint _C \vec B \cdot \d \vec l$ "--- циркуляция.
		Показывает, насколько круто закручено наше поле.
	\end{Def}
	Получили такой аналог теоремы Гаусса, называется законом Ампера.
	Напомним её:
	\[ \oint_S \vec E \cdot \d \vec S = 4 \pi \sum Q\]
	Справа стоит то, что создаёт нечто, а слева "--- создаваемое.
	Заряды создают электрические поля, а ток создаёт магнитные поля.

	% лирическое отступлене
	Что интересно, эти законы топологические, то есть от формы не зависят.
	Внутри электромагнитных законов есть красивая симметрия, которая нам пока недоступна.
	Некоторые топологические задачи (распутывание узлов или задача зацепления)
	решались введением похожих физических величин.
	% конец лирического отступления

	% и еще одно
	В электростатике интеграл электрического поля по замкнутому контуру был равен нулю.
	В электродинамике интеграл магнитного поля наоборот, не равен нулю, если есть ток.
	Такая вот симметрия
	% онец

\section{Соленоид}
	Соленоид "--- это такая катушка из намотанного провода.
	Аналог конденсатора: <<запасает>> магнитное поле.
	% картинка: катушка слева направо; нарисовали только сечение
	% два слоя кружочков: сверху и снизу (это сечения проводов)
	% сверху из доски, снизу - в доску
	Будем считать, что катушка бесконечна влево и вправо
	Давайте вычислим магнитное поле в центре катушки.
	Чтобы не брать интеграл, хотим применить закон Ампера.

	Но сначала определимся с направлением.
	Из-за симметрии оно идёт либо вправо, либо влево
	(кстати, не в центре поле может отклоняться вверх/вниз).
	По правилу правой руки можно понять, что оно идёт вправо
	(просто рассмотрели каждый виток по отдельности, везде поле в центре катушки идёт вправо).

	Пусть катушка большой длины $L$, в ней есть много витков ($N$).
	Тогда <<плотность витков>> будет равна $n=\frac NL$, считаем, что это что-то небольшое.
	Возьмём замкнутый контур:
	% картинка: взяли прямоугольник вокруг верхних проводов, низ проходит через центр катушки.
	Понятно, что на левом и правом ребре суммарная циркуляция будет ноль, потому что магнитное
	поле ведёт себя на них одинаково, а направление разное.
	Поверим на слово: снаружи катушки магнитное поле убывает очень быстро (времени объяснять, почему, нет),
	поэтому на верхнем ребре интеграл тоже ноль.
	Другое объяснение: можно взять верхнее ребро очень далеко, там точно будет ноль.
	Таким образом, вклад в наш интеграл идёт только от ребра по центру катушки.
	Пусть длина этого ребра равна $h$.
	В любой точке ребра магнитное поле одинаково (потому что катушка очень длинная).
	Получаем:
	\begin{gather*}
		B \cdot h = \oint_C \vec b \cdot \d \vec l = \frac{4\pi}{C} I \cdot nh \\
		B = \frac{4\pi}{C} n I
	\end{gather*}

\section{Векторный потенциал}
	В электростатике был обычный потенциал для электрического поля.
	Тут мы умножали $\vec \nabla$ скалярно на $\phi$:
	% : $\vec E = -\vec \nabla \phi$
	\[ \vec E = -\vec \nabla \phi = -\begin{pmatrix}\partd{}x\\\partd{}y\\\partd{}z\end{pmatrix} \cdot \phi\]
	Давайте попробуем умножить $\vec \nabla$ векторно на $\vec A$.
	Просто по формуле для векторного произведения в координатах:
	\[
	\vec \nabla \times \vec A
	=
	\left|\begin{matrix}
	\vec i & \vec j & \vec k \\
	\partd{}x & \partd{}y & \partd{}z \\
	A_x & A_y & A_z \\
	\end{matrix}\right|
	=
	\begin{pmatrix}
		\partd{A_z}{y} - \partd{A_y}{z} \\
		\partd{A_x}{z} - \partd{A_z}{x} \\
		\partd{A_y}{x} - \partd{A_x}{y}
	\end{pmatrix}
	\]
	\begin{Def}
		$\vec \nabla \times \vec A = \rot \vec A$ "--- ротор $\vec A$.
	\end{Def}
	\begin{Def}
		Векторный потенциал магнитного поля:
		$\vec A(\vec r) = \frac 1 C \int \frac{\vec j(\vec r')}{|\vec r - \vec r'|} \d V$
	\end{Def}
	Смысл такой: выделили объёмный кусок провода, посчитали ток через этот кусок, поделили на $|\vec r - \vec r'|$, проинтегрировали.
	% картинка со здоровенным проводом-змеёй, точкой \vec r, точкой \vec r' в выделенном куске и т.п.
	\begin{theorem}
		$\vec B(\vec r) = \vec \nabla \times \vec A(\vec r) = \rot \vec A(\vec r)$.
	\end{theorem}
	\begin{proof}
		Давайте распишем $\vec A$ в координатах.
		Единственная векторная вещь в определении "--- $\vec j$.
		Поэтому расписывается очень просто:
		\[
		\vec A(\vec r) = \begin{pmatrix}
		\frac 1 C \int \frac{j_x(\vec r')}{|\vec r - \vec r'|} \d V \\
		\frac 1 C \int \frac{j_y(\vec r')}{|\vec r - \vec r'|} \d V \\
		\frac 1 C \int \frac{j_z(\vec r')}{|\vec r - \vec r'|} \d V \\
		\end{pmatrix}
		\]
		Посчитаем $(\rot \vec A)_x$.
		Для этого сначала посчитаем $\partd{A_z}{y}$.
		Под интегралом у нас от $y$-компоненты (то есть от $r_y$) зависит только знаменатель, $r'$ "--- это переменная, по которой мы интегрируем.
		Выпишем явную формулу (сразу поменяв местами производную и интеграл, так можно делать):
		\[
		\partd{A_z}{y}
		= \frac 1 C \int \partd{}{y} \left(\frac{j_z(\vec r')}{|\vec r - \vec r'|} \d V\right)
		= \frac 1 C \int \left( j_z(\vec r') \d V \partd{}{y} \frac{1}{|\vec r - \vec r'|} \right)
		\]

		Так сразу считать сложно, поупражняемся:
		\[
		\partd{}{y} \frac{1}{|r|}
		= \partd{}{y} \frac{1}{\sqrt{x^2+y^2+z^2}}
		= - \frac{y}{(x^2+y^2+z^2)^{3/2}}
		= - \frac{r_y}{|\vec r|^3}
		\]

		Возвращаемся к большой частной производной:
		\[
		\partd{}{y} \frac{1}{|r-r'|}
		= - \frac{(r-r')_y}{|\vec r - \vec r'|^3}
		\]
		И подставляем в формулы:
		\begin{gather*}
		\partd{A_z}{y} = - \frac 1 C \int \frac{j_z(\vec r')(\vec r - \vec r')_y}{|\vec r - \vec r'|^3} \d V \\
		\partd{A_y}{z} = - \frac 1 C \int \frac{j_y(\vec r')(\vec r - \vec r')_z}{|\vec r - \vec r'|^3} \d V \\
		\end{gather*}
		\begin{align*}
		(\rot \vec A)_x &= - \frac 1 C \int \frac{\d V}{|\vec r - \vec r'|^3} \cdot \left( j_z(\vec r')(\vec r - \vec r')_y - j_y(\vec r')(\vec r - \vec r')_z \right) = \\
		                &= - \frac 1 C \int \frac{\d V}{|\vec r - \vec r'|^3} \cdot (-(\vec j \times (\vec r - \vec r'))_x) = \\
		                &= \phantom{-} \frac 1 C \int \frac{\d V}{|\vec r - \vec r'|^3} \cdot (\vec j \times (\vec r - \vec r'))_x \\
		\rot \vec A     &= \frac 1 C \int \frac{(\vec j \times (\vec r - \vec r')) \d V}{|\vec r - \vec r'|^3}
		\end{align*}
		Почти получили следствие из закона Био-Савар-Лапласа:
    	\begin{gather*}
			\d \vec B(\vec r) = \frac 1 C \frac{\vec I \d l \times (\vec r - \vec r')}{|\vec r - \vec r'|^3} \\
			\vec B(\vec r) = \frac 1 C \int \frac{\vec I \d l \times (\vec r - \vec r')}{|\vec r - \vec r'|^3} \\
		\end{gather*}
		Заметим, что $\vec j \d V = \vec I \d l$, неясно как.
		Тогда сошлось, что и требовалось доказать:
		\begin{gather*}
    		\vec B(\vec r) = \rot \vec A(\vec r)
		\end{gather*}
	\end{proof}

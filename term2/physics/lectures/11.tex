\setauthor{Егор Суворов}

% 15.05.2015
\section{Лирическое отступление: принцип наименьшего действия}
% Переписать нетезисно:
За этот семестр мы с грехом пополам вывели уравнения Максвелла.
В прошлом семестре занимались механикой.
Действуют какие-то разные законы, физика выглядит как сборник независимых фактов...
Жутко неудобно, хочется найти что-нибудь общее.
Что общего между, например, теорией Максвелла и механикой?
В первой какие-то поля двигаются, а во второй "--- частицы.
Может, всё-таки есть какой-то общий принцип?

% везде ниже прочитать и дорисовать картинки
Есть, называется принцип наименьшего действия.
Посмотрим на одну частицу, пусть она как-то движется по закону $q(t)$, по какой-то траектории.
Тогда у неё есть скорость "--- $\dot q(t)$.
Теперь сопоставим каждой тройке $\langle q(t), \dot q(t), t\rangle$ некоторое число.
Получили функционал $L(q, \dot q, t))$.
Он принимает на вход две функции и момент времени, функции статичны для траектории.
Пусть движение началось в момент времени $0$, а закончилось в момент времени $T$.
Проинтегрируем функционал по $t$:
\[
\int_0^T L(q(t), \dot q(t), t) \d t = S
\]
Назовём это $S$ \textit{действием}.
Обращаем внимание, что $L$ в каждый момент интегрирования <<знает>>, что было раньше и что будет дальше "--- первые два параметра являются неизменяющимися функциями.
\begin{assertion}
	Если у нас фиксированы начальная и конечные точки, то в физике возможны не все траектории,
	а только те, которые минимизируют действие.
\end{assertion}
\begin{Rem}
	Обычно в физике если минимум локальный, то он же и глобальный.
	Но не всегда, тогда надо уточнять.
\end{Rem}

Конечно, стоит выяснить, что же такое функционал $L$, тогда научимся решать любую задачу.
Обычно их можно написать из соображений симметрии мира и простых соображений, но это отдельная тема.

\subsection{Механика}
	Например, в механике имеем:
	\[
	L = \frac{m{\dot q}^2(t)}{2} - \underbrace{П}_{\text{потенциальная энергия}}(q(t))
	\]
	Обращаем внимание, что это не энергия "--- там была бы сумма энергий, а не разность.
	Также этот функционал (называется функционалом Лагранжа) не пользуется своим знанием о прошлом и будущем.
	Если проделать вычисления, то окажется, что минимальность этого функционала эквивалентна второму закону Ньютона:
	\[
	m\ddot q(t) = - \partd{П(q(t))}{q}
	\]
	Но зато принцип намного более общий, чем второй закон Ньютона.
	Например, физики на втором курсе учатся писать функционал Лагранжа для систем частиц, для чего-то еще более общего, и вообще для всех задач механики.
	Тогда можно находить уравнения движений.

\subsection{Электродинамика}
	В механике мы изучали частицы, а тут "--- поля.
	Оказывается, что $L$ можно написать и в терминах полей:
	\[
		L(\vec E(t), \vec B(t), \dot{\vec E(t)}, \dot{\vec B(t)}, t)
	\]
	Дальше говорим, что поля у нас не абы какие, а такие, на которых действие минимизируется.
	Если минимизировать функционал, то мы получим уравнения Максвелла.
	Они имеют такой же смысл, как законы Ньютона в механике: это уравнения движения, которые минимизируют действие
	Зато получается намного удобнее: мы пользуемся одним постулатом, из него выводим вообще всё,
	а не мучаемся с выводом уравнений Максвелла в течение семестра с кучей допущений.

\subsection{Квантовая механика}
	Ключевая задача квантовой механики: есть какая-то система, с какой вероятностью она перейдёт
	из некоторого начального состояния в некоторое конечное?
	Её даже в какой-то момент научились решать примерно так же, как и мы <<вывели>> уравнения Максвелла.
	Потом приходит Фейнман и говорит, что эта искомая вероятность пропорциональная буковкам (Фейнмановский интеграл):
	\[
		P \sim \int \mathrm{D}q e^{\left(\frac{i S(q(t),\dot q(t), t)}{\bar h}\right)}
	\]
	Здесь $i$ "--- мнимая единица, $\bar h$ "--- постоянная Планка (это, кстати, единственное место от квантовой механики в этом интеграле),
	а $\mathrm{D}q$ "--- в некотором смысле дифференциал, мы же интегрируем не по длине, а по всем возможным функциям.

	Физический смысл: берём начальное и конечное состояние, соединяем их всеми возможными гипотетическими траекториями (необязательно физическими).
	На каждой считаем действие, берём какую-то функцию от этого, интегрируем по всем траекториям.
	
	Вообще говоря, этот интеграл <<по всем траекториям>> математически не могут определить уже лет пятьдесят.
	Только в каких-то очень-очень специфичных случаях.
	Зато физики могут их считать (не задумываясь о сходимости и прочих мелочах) и почему-то всегда сходится со здравым смыслом и экспериментом.
	Примерно как наивная теория множеств.
	Сюда же относится афоризм: <<Shut up and calculate!>>.

\subsection{Разные силы}
	Этим интегралом смогли успешно объединить квантовую механику, электродинамику (получилась квантовая электродинамика),
	потом добавили слабое и сильное взаимодействие.
	А вот с гравитацией еще не справились.

\subsection{Математика}
	Если забить на сходимость Фейнмановского интеграла и просто поизучать с точки зрения математики,
	то обнаружится, что он также умеет решать кучу чисто математических задач.
	Например, самая известная: топологическая задача различения многообразий или узлов.

	\subsubsection{Поиск геодезических}
		Задача: есть две точки (например, на плоскости), хотим найти кратчайший путь.
		Пишем, что такое длина кривой:
		\begin{align*}
			\d s &= \sqrt{\d x ^ 2 + \d y ^ 2} \\
			L &= \int \d s
			  = \int \sqrt{\d x ^ 2 + \d y ^ 2} \\
			  &= \int \d x \sqrt{1 + \left(\frac{\d y}{\d x}\right) ^ 2}
			  = \int \d x \sqrt{1 + {y'}^ 2}
		\end{align*}
		Это какой-то функционал от $y$.
		Минимизируем "--- найдём ответ (прямую).
		Но есть бонус: аналогичным образом можно искать кратчайшие пути (геодезические) между точками на сфере, на чём-то еще более страшном...

	\subsubsection{Мыльная плёнка}
		Погрузили два проволочных кольца в мыльную воду, достали их параллельными друг другу, получили между ними мыльную плёнку.
		Вопрос: какой формы?
		Утверждение: с минимальной энергией, то есть с минимальной площадью.
		Перефразируем на математическом языке: найти функцию $y(x)$, для которой минимальна площадь поверхности вращения.
		Тоже можно решить, получим гмперболический косинус.

		Можно сказать, что у нас не кольца, а что-то более сложное и даже не параллельное друг другу "--- получим нерешённую задачу в математике.

	\subsubsection{Самая первая задача}
		С этой задачи Лагранж придумал всю эту теорию.
		Задача: есть две точки, соединены проволокой, по ней скользит бусинка массы $m$.
		Хотим, чтобы она как можно быстрее доскользила от одной точки до другой.
		Ответ: циклоида.

\section{Взаимная индукция}
	В прошлый раз начинали обсуждать задачу про две катушки.
	% картинка
	Давайте порассуждаем.
	Пропускаем по первой катушке ток $I_1(t)$.
	Тогда порождается магнитное поле $B_1(t) = \frac{4\pi}{c} n_1 I$ (где $n_1$ "--- сколько витков на единицу длины).
	Теперь считаем ЭДС второй катушки.
	Для этого нужен магнитный поток через вторую катушку: $\Phi_2(t) = N_2 \cdot S \cdot B_1(t)$ ($N_2$ "--- число витков во второй катушке).
	Тогда
	\begin{align*}
		\mathcal{E} &= -\frac1c \partd{\Phi_2(t)}{t}
			= -\frac{N_2S}{c} \partd{B_1(t)}{t}
			= -\underbrace{\frac{4\pi n_1 N_2S}{c^2}}_{\text{взаимная индукция}} \partd{I_1(t)}{t}
	\end{align*}
	Вот этот \textit{коэффициент взаимной индукции} обозначается $L_{12}$.
	Очевидно, что в простом случае (если катушки целиком зацепляются) у нас $L_{12}=L_{21}$, потому что $n_1N_2=N_1n_2$.
	Можно доказать (мы не будем), что даже если у нас какие-то более сложные витки (разные по форме, не совсем катушки и прочее),
	то всё равно будет верно $L_{12}=L_{21}$, даже если мы не сможем напрямую сосчитать ЭДС и всё остальное.

	Кстати, мы тут только один шаг решения задачи выполнили.
	Ведь после второго ЭДС во второй катушке возникнет ток $I_2(t)=\frac{\mathcal{E}_2}{R_2}$,
	он по такой же схеме что-то породит в первой катушке и так далее.
	Можно составить систему из диффуров, решить, получить точное решение.
	А еще можно руками сделать несколько итераций и должно стать ясно, что надо оставить а чем можно пренебречь.

	\subsection{Одна катушка}
		Пусть у нас теперь одна катушка, сама на себя действует.
		Получим порождённое ЭДС:
		\[
			\mathcal{E} = -L \partd{I(t)}{t}
		\]
		Это $L$ называется \textit{коэффициентом самоиндукции}.
		Такая характеристика геометрии катушки, единица измерения называется Генри (Гн).
		Когда мы пытаемся загнать в катушку ток, она сопротивляется.

		Уравнение несколько напоминает второй закон Ньютона:
		\[
		F = m \partd{v}{t}
		\]
		Это неспроста.
		Можно проводить аналогии между механикой и катушками.
		Например: сила в механике -- ЭДС в катушке; скорость в механике -- ток (действительно, скорость движения зарядов).
		Кинетическая энергия в механике: $\frac{mv^2}{2}$, энергия для катушки "--- $\frac{LI^2}{2}$ (доказывать не будем, просто по аналогии).

	\subsection{Конденсатор с катушкой}
		Взяли конденсатор, зарядили.
		Подключили к катушке.
		Пошёл ток, конденсатор начал разряжаться, куда-то денется энергия.
		Денется она в катушку, и так пока конденсатор полностью не разрядится (всё это время будет идти ток).
		Если посмотреть на график тока от времени, то ясно, что в начале ток ноль, в катушке ноль энергии,
		значит, вся она в конденсаторе, он генерирует ток.
		Ток в системе растёт, растёт энергия катушки, конденсатор разряжается.
		В какой-то момент ток достигнет максимума.
		Значит, вся энергия в катушке, а конденсатор разряжен.
		Но ток продолжает течь (sic!).
		Конденсатор перезаряжается в другую сторону.
		Получаем колебания.

%Liza Tretyakova

\begin{section}{Потоки}

	\begin{center}
	\includegraphics[width=0.5\textwidth]{02/first}
	\end{center}

Через замкнутую поверхность поток электрического поля равен нулю(сколько силовых линий вошло, столько и вышло). 
Для убедительности что-нибудь посчитаем:
\begin{enumerate}

\item Кубик с однородным электрическим полем снаружи

	\begin{center}
	\includegraphics[width=0.5\textwidth]{02/second}
	\end{center}

	\[\Phi_{\text{через кубик}} = \Phi_1 + \Phi_2 + \cdots + \Phi_6\]
	\[\Phi_{1, 2, 3, 6} = 0\]
	\[\Phi_4 = E \cdot a^2 \cdot (-1)\]
	\[\Phi_5 = E \cdot a^2\]
	То есть
	\[\Phi_{\text{через весь кубик}} = 0\]
	\begin{Rem}
	Если бы внутри куба был заряд, то поток через все грани был бы положителен.
	\end{Rem}
	
\item Сфера с центром в заряде

	\begin{center}
	\includegraphics[width=0.5\textwidth]{02/third}
	\end{center}

	Чтобы получить суммарный поток, разобьём сферу на много маленьких зарядов и посмотрим на какой-нибудь.
	\[E = \frac{q}{r^2}\]
	и так в каждой площадке
	\[\Phi_{\Delta S} = E \cdot \Delta S = \frac{q}{r^2}\Delta S\]
	\[\Phi_{\text{через сверху}} = \sum \Phi_{\Delta S} = \sum \frac{q}{r^2}\Delta S = \frac{q}{r^2}\sum\Delta S 
	= \frac{q}{r^2}4\pi r^2 = 4 \pi q\]
	
\item Кривая поверхность
	
	\begin{center}
	\includegraphics[width=0.5\textwidth]{02/fourth}
	\end{center}

	Количетсво силовых линий, проходящих через эти поверхности, равны. $\Phi \sim N$, где $N$ "--- это самое количество.
	\[\Phi_{S'} = \Phi_S = 4\pi q\]
	\[\Phi = 4\pi\sum q_i\] "--- теорема Гаусса.

\end{enumerate}

И ещё больше примеров:
\begin{enumerate}

\item
	Заряженная ниточка, её линейный заряд равен $\lambda$. Как посчитать её $\vec{E}$?
	\begin{center}
	\includegraphics[width=0.5\textwidth]{02/thread}
	\end{center}

	Рассмотрим цилиндр. Посчитаем поток через него. Для этого заметим, что силовые линии строго перпендикулярны ниточке(из соображений
	симметрии).
	\[\Phi_{\text{полный}} = \Phi_{\text{торцы}} + \Phi_{\text{боковая поверхность}}\]
	\[\Phi_{\text{торцов}} = 0\], так как \[\vec{E} \bot \vec{n}\]
	
	В каждой точке |E| одинаков и $\vec{E} \upuparrows \vec{n}$, то $\Phi_{\text{боковой поверхности}} = \delta S$, то есть
	\[\Phi_{\text{боковой поверхности}} = \sum E \delta S = E \cdot \sum \delta S = 2 \pi r h E\]
	Но по теореме Гаусса суммарный поток $\Phi = 4\pi\sum q_i$, поэтому
	\[\Phi_{\text{боковой поверхности}} = \frac{4\pi\lambda h}{\sum q_i}\]
	\[2\pi r h E = 4 \pi \lambda h\]
	\[E = \frac{2 \lambda}{r}\]
	
\item
	Посчитаем поле от заряженной бесконечной плоскости.
	
	Плотность заряда $\sigma$, что означает, что участок площадью в один квадратный метр имеет заряд в $\sigma \text{Кл}$.
	Снова рассмотрим цилиндр над каким-либо круглым участком плоскости(так как она бесконечная, то все такие участки не будут
	ничем отличаться друг от друга).
	\begin{center}
	\includegraphics[width=0.5\textwidth]{02/sixth}
	\end{center}

	\[\Phi = 2E\pi r^2 = 4\pi\sigma\pi r^2\] "--- из теоремы Гаусса
	\[E = 2\pi\sigma\]
	
	Мы получили, что $E$ не зависит от расстояния до поверхности, потому что для бесконечной плоскости масштаба нет 
	и мы не можем понять, на расстоянии ли мы одного метра либо десяти. С ниткой такое рассуждение не прокатывает, потому что по другой 
	координатной оси нам будет с чем сравнивать.

\end{enumerate}

\end{section}

\begin{section}{Проводник в электростатике}

	Все вещества мы делим на две группы:
	\begin{itemize}
	
	\item
		Проводники. У них есть свободные электрончики, которые по этому самому проводнику способны передвигаться. 
		Примеры: золото, никель, платина, медь, алюминий.
	
	\item
		Диэлектрики. Примеры: фарфор, керамика, пластика.
		
	\item
		Полупроводники - но вы этого сейчас не слышали :)
	
	\end{itemize}
	
	И интересуют нас теперь проводники.
	
	Если бы внутри проводника было электрическое поле, то свободные заряды бы двигались в нём. То есть
	\begin{enumerate}
	
	\item
		В электростатике в проводнике нет электрического поля.
		
	\item
		Проводник можно зарядить. 
		
		Все заряды, которые есть в проводнике, сосредоточены на его поверхности
		\begin{center}
		\includegraphics[width=0.5\textwidth]{02/seventh}
		\end{center}

		\[4\pi\sum q_i = \Phi = \Phi_{\text{через поверхность} = 0} \rightarrow \sum q_i = 0\]
		
		А это и означает, что внутри поверхности зарядов нет, то есть вес заряды находятся на поверхности.
	
	\end{enumerate}

	Рассмотрим пластинку из проводника.
	\begin{center}
	\includegraphics[width=0.5\textwidth]{02/eighth}
	\end{center}

	Одноимённые заряды скапливаются на одной стороне пластинки под действием силы $q \vec{E}$, и поле разбежавшихся зарядов компенсирует
	внешнее, чтобы внутри полчить ноль.
	
	Теперь рассмотрим проводящий шар.
	\begin{center}
	\includegraphics[width=0.5\textwidth]{02/nineth}
	\end{center}

	Силовые линии должны заканчиваться на отрицательных зарядах или на бесконечности; силовые линии должны входить 
	в поверхность проводника перпендикулярно - исходя из этих соображений, мы и получили картинку выше. 
	
	Почему перпендикулярно поверхности? Пусть входят по касательной. Тогда у электрического поля будет проекция на эту поверхность, из-за
	 этого появится ненулевая сила и заряды начнут двигаться под её действием "--- но этого не может быть, потому что у нас электростатика.

\end{section}

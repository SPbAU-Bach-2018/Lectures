\setauthor{Егор Суворов}

% 24.04.2015
\section{Динамика}
	Помним, что $\oint \vec E \cdot \d \vec l = 0$ (по замкнутому контуру).
	Вопрос: так ли это в динамике?

	Порассуждаем.
	Пусть у нас в какой-то неподвижной системе отсчёта заряд двигается с постоянной скоростью $\vec v$.
	От этого заряда возникает какое-то магнитное поле $\vec B$.
	В любой точке в этой системе отсчёта будут два поля: $\vec E$ и $\vec B$.
	Давайте теперь рассмотрим систему отсчёта, которая двигается вместе с зарядом.
	В ней есть только поле $\vec E'$, а магнитного поля нет.
	
	\textit{Лирическое отступление}:
	да, магнитное поле действительно зависит от системы отсчёта.
	На самом деле, отдельно электрического и отдельно магнитного поля не бывает.
	Есть только электромагнитное, которое мы видим по-разному в зависимости от системы отсчёта.
	С этой точки зрения мы рассматривать теорию не будем: у нас мало времени и мотивации.
	Чтобы это увидеть, надо приплести теорию относительности и порассматривать четырёхмерное пространство-время.

	\textit{Лирическое отступление}:
	вообще говоря, представление об электрическом поле со школы довольно ограниченное, хоть
	и верное: его можно пощупать, посмотреть (зарядили расчёску, она притянула).
	И сложно представить, что от изменения системы отсчёта что-то поменяется.
	А магнитное поле "--- какая-то искусственная штука, которая сильно нарушает симметрию пространства.

	Посмотрим на силу, которая действует на какой-то пробный заряд $q_0$.
	Она от системы отсчёта зависеть не должна.
	Запишем её в двух системах (слева "--- неподвижная с подвижным зарядом, справа "--- с неподвижным зарядом):
	\begin{align*}
	q_0 (\vec E + \frac{\vec v}{c} \times \vec B) &= q_0 \vec E' \\
	\vec E + \frac{\vec v}{c} \times \vec B &= \vec E'
	\end{align*}
	Кстати, $\vec E'$ не меняется от времени, так как в той системе у нас полная электростатика.
	А раз так, можем написать $\oint \vec E' \cdot \d \vec l = 0$ в любой момент времени.
	По теореме Стокса это то же самое, что:
	\begin{gather*}
	\nabla \times \vec E' = 0  \\
	\nabla \times (\vec E + \frac{\vec v}{c} \times \vec B) = 0 \\
	\nabla \times \vec E + \frac{1}{c} \nabla \times (\vec v \times \vec B)) = 0
	\end{gather*}
	Вспомним формулу: $\vec a \times (\vec b \times \vec c) = \vec b (\vec a \vec b) - \vec c (\vec a \vec b)$
	Применим:
	\begin{gather*}
	\nabla \times \vec E + \frac{1}{c} (\vec v (\nabla \vec B) - \vec B (\nabla \vec v)) = 0 \\
	\nabla \times \vec E + \frac{1}{c} (\vec v (\nabla \vec B) - (\vec \nabla) \vec B) = 0
	\end{gather*}
	Вспомним, что такое дивергенция электрического поля ($\nabla \vec E$) "--- это сколько электрических зарядов внутри.
	Дивергенция магнитного "--- это сколько <<магнитных зарядов>> внутри, но таких как бы нет, поэтому ноль,
	то есть $\nabla \vec B = 0$ (это, кстати, еще одно уравнение Максвелла).
	<<Как бы нет>> "--- это экспериментальные данные: обычно там ноль.

	\textit{Лирическое отступление}:
	наблюдение из детского сада: если разрезать магнит, получим не северный и южный полюса, а два магнита.
	То есть нет таких точек, из которых бы выходили силовые линии магнитного поля, а они все замкнутные.
	На самом деле такие <<магнитные заряды>> предсказаны: называется \textit{монополь},
	но его еще никто не поймал.

	Осталось два слагамых:
	\[
	\nabla \times \vec E - \frac{1}{c} (\vec \nabla) \vec B = 0
	\]
	В текущий переход поверим на слово (времени мало, но вроде это просто :):
	\begin{gather*}
	\nabla \times \vec E + \frac{1}{c} \partd{\vec B}{t} = 0 \\
	\nabla \times \vec E = -\frac{1}{c} \partd{\vec B}{t}
	\end{gather*}
	То есть при изменении магнитного поля во времени электрическое поле закручивается.
	Последнее "--- это еще одно уравнение Максвелла (в народе <<закон Фарадея>>).
	Если вспомним, что интеграл ротора равен циркуляции, то:
	\[
	\oint \vec E \d \vec l = \int (\nabla \times \vec E) \d \vec S = -\frac 1 c \int \partd{\vec B}{t} \d \vec S
	\]
	А еще можно вытащить производную:
	\[
	\oint \vec E \d \vec l = \int (\nabla \times \vec E) \d \vec S = -\frac 1 c \partd{}{t} \underbrace{\int \vec B \d \vec S}_{\text{поток~}\vec B}
	\]

	\subsection{ЭДС}
	\begin{exmp}
	Пусть есть проводник и на нём лежит перемычка-проводник, двигается вправо со скоростью $\vec v$.
	Также есть какое-то магнитное поле $B$ слева от проводника:
	% ┌──────╫────   ^
	% │на нас║-->	| h
	% │	  ║ v	 |
	% └──────╫────   v
	Из-за силы Лоренца
	\[ \vec F = q \frac{\vec v} c \times \vec B\]
	у нас положительные заряды уедут вниз, отрицательные "--- вверх.
	Хотим посчитать разность потенциалов между двумя концами.
	Это работа, которую мы должны совершить для перетаскивания заряда.
	А сопротивляется нам только сила Лоренца.
	То есть получим:
	\[ \phi_B - \phi_A = |F| h = \frac{v}{c}B h\]
	Кстати, так делать вообще не очень хорошо: у нас же уже электродинамика, странно говорить про потенциалы, если циркуляция по замкнутому контуру не ноль.

	Давайте теперь возьмём наш закон Фарадея.
	Возьмём контур против часовой стрелку через нашу перемычку и проводник.
	Магнитный поток меняется, потому что меняется площадь поверхности, поток меняется так:
	$\partd{\text{~поток}}{t} = B h \partd{l}{t} = Bhv$
	Получили циркуляцию тока в контуре:
	\[ \mathcal{E} = -\frac{v}{c} B h\]
	Да, не тот знак, видимо, с направлением напутали.
	Такая циркуляция электрического поля называется <<ЭДС>>.
	\end{exmp}

	\begin{exmp}
	Пусть есть два магнита, между ними прямые линии поля, между ними вращается проводик-рамочка.
	Магнитный поток меняется.
	Посчитаем ЭДС:
	\[
	\mathcal{E} = -\frac{1}{c} \partd{\Phi}{t} = -\frac 1 c \partd{}{t} (B S \cos \phi)
	            = -\frac 1 c \partd{}{t} (BS \cos (\omega t)) = BS \frac{\omega}{c} \sin (\omega t)
	\]
	\end{exmp}

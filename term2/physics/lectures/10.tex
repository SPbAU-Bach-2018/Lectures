\setauthor{Егор Суворов}

% 08.05.2015
\section{Обобщение закона Ампера}
	Итак, у нас есть законы Максвелла:
	\begin{enumerate}
		\item
			Теорема Гаусса:
			$\vec \nabla \cdot \vec E = 4\pi\rho$
		\item
			Нет источников магнитного поля:
			$\vec \nabla \cdot \vec B = 0$
		\item
			Закон индукции Фарадея (только что вывели):
			$\rot \vec E = -\frac1c \partd{\vec B}{t}$
		\item
			Закон Ампера для магнитного поля:
			$\rot \vec B = \frac{4\pi}{c} \vec j$
	\end{enumerate}

	Давайте проверим предположения, которыми мы пользовались, когда их выводили.
	Теорему Гаусса выводили в предположении статики, но она верна всегда (верим на слово).
	Второе уравнение тоже верно всегда.
	Третье уравнение выводили уже в динамика (в статике ротор электрического поля ноль, потому что интеграл по контуру ноль).
	Самое интересное "--- четвёртое уравнение.
	Давайте проверим, нет ли противоречия: мы там всё-таки предполагали, что заряды движутся, но довольно медленно, не влияя на электрическое поле (плотность зарядов не меняется, <<стационарный ток>>).

	Кстати, еще бывает пятое уравнение (когда заряд утекает из объёма через поверхность, через неё идёт ток):
	\[
		\int \vec j \cdot \d \vec S = -\partd{q}{t}
	\]
	Левую часть изменим по теореме Стокса, а правую перепишем через плотность:
	\begin{gather*}
		\int \vec \nabla \cdot \vec j \cdot \d V = -\partd{}{t} \underbrace{\int \rho \cdot \d V}_{q} \\
		\vec \nabla \cdot \vec j = -\partd{\rho}{t} \\
	\end{gather*}
	Получили управнение непрерывности.
	Мы им, кстати, не пользовались при выводе наших законов.
	Вопрос: не является ли система противоречивой?
	Чуть точнее: пусть у нас есть поля, токи и заряды (вся информация), не может ли быть так, что они удовлетворяют уравнениям Максвелла, но противоречат уравнению непрерывности?

	В первых трёх уравнениях ошибок, надо сказать, нет.
	Противоречие есть между законом Ампера и уравнением непрерывности.
	Давайте скалярно умножим на наблу закон Ампера:
	\begin{gather*}
		\vec \nabla \cdot (\vec\nabla \times \vec B) = \vec\nabla \cdot \frac{4\pi}{c} \vec j = \frac{4\pi}{c} \underbrace{\vec \nabla \cdot \vec j}_{\text{левая часть}} = -\frac{4\pi}{c}\partd{\rho}{t}
	\end{gather*}
	Слева у нас скалярное произведение $\vec \nabla$ на что-то, перпендикулярное ему, то есть ноль.
	\[
		0 = -\frac{4\pi}{c} \partd{\rho}{t}
	\]
	Получили предположение, поторым пользовались при выводе четвёртого уравнения.
	Как-то не очень хорошо: оно же не всегда выполняется.
	Надо бы поправить четвёртое уравнение, чтобы оно работало и в случае нестационарных токов (когда плотность меняется).
	Для этого давайте что-нибудь добавим в его правую часть.
	Что это может быть?
	Определённо какой-то вектор, потому что там уже вектор стоит.
	Какой там еще не задействован?
	Электрическое поле $\vec E$.
	Чтобы как-то зависеть от динамики, возьмём производную.
	Утверждается, что вот такое уравнение уже будет верно:
	\[
		\rot \vec B = \frac{4\pi}{c} \vec j + \frac1c \partd{\vec E}{t}
	\]
	Новое второе слагаемое называется током смещения (первое "--- это как бы <<настоящий ток>>).
	Проверим аналогичным образом (применим наблу):
	\[
		0 = \frac{4\pi}{c} \vec \nabla \cdot \vec j + \frac1c \partd{\vec \nabla \cdot \vec E}{t} 
	\]
	Первое слагаемое перепишем по уравнение непрерывности, а второе "--- по Гауссу:
	\[
		0 = -\frac{4\pi}{c}\partd{\rho}{t} + \frac1c 4\pi \partd{\rho}{t}
	\]

	Теперь получили законы, которые полностью описывают всё происходящее с полями.
	Можно читать в разные стороны: например, по полям получать заряды или наоборот.
	В разных областях нужно разное и с разными приближениями.

\section{Конденсатор с изменяющимися зарядами}
	Пусть у нас есть плоский конденсатор, у него как-то меняется заряд на пластинах.
	% картинка: пластинка сверху (+) и снизу (-)
	Возникает какое-то вихревое магнитное поле.
	Если бы мы в законе Ампера не учитывали новую добавку, то между обкладками магнитного поля бы не было, так как там не тока.
	Но добавка есть и она зависит от поля.
	А поле зависит от заряда на обкладках.
	То есть внутри обкладок магнитное поле есть: тока нет, зато есть меняющееся электрическое поле.
	А снаружи обкладок электрического поля нет (значит, оно не меняется), зато есть ток и, следовательно, магнитное тоже есть.

	Давайте найдём магнитное поле внутри обкладок.
	Оно должно как-то закручиваться.
	Ось одна "--- по центру пластины, где втыкается провод.
	% картинка: сверху заряд +Q(t), снизу -Q(t), поле E(t) сверху вниз
	Давайте возьмём какой-нибудь симметричный контур, в котором посмотрим на магнитное поле:
	% картинка: две пластины, окружность радиуса r
	Единственный вклад по закону Ампера даёт второе слагаемое, так как тока нет.
	Вернём закон Ампера в интегральную форму (первое слагаемое уже ноль, будем смотреть на поверхность, ограниченную контуром):
	\begin{gather*}
		\nabla \times \vec B = \frac 1c \partd{\vec E}{t} \\
		(\nabla \times \vec B) \d \vec S = \frac 1c \partd{\vec E}{t} \d \vec S \\
		\int (\nabla \times \vec B) \d \vec S = \frac 1c \partd{}{t} \int \vec E \d \vec S \\
	\end{gather*}
	Теперь по теореме Стокса (слева) и по знанию о поле в конденсаторе:
	\begin{align*}
		\int (\nabla \times \vec B) \d \vec S &= \frac 1c \partd{}{t} \int \vec E \d \vec S \\
		\int \vec B \d \vec l &= \frac 1c \partd{}{t} \int 4\pi\sigma \d S =\\
			&= \frac 1c \partd{}{t} 4\pi\sigma \int \d S \\
			&= \frac 1c \partd{}{t} 4\pi\sigma \pi r^2 \\
			&= \frac{4\pi^2r^2}c \partd{\sigma}{t} \\
	\end{align*}
	Смотрим на левую часть.
	Контур "--- это окружность.
	Магнитное поле везде одинаковое, направлено по касательной (чтобы закручивалось),
	поэтом интеграл слева равен $2 \pi r B$:
	\begin{align*}
		2\pi r B &= \frac{4\pi^2r^2}c \partd{\sigma}{t} \\
		B &= \frac{2\pi r}c \partd{\sigma}{t} \\
		B &= \frac{2\pi r}c \partd{}{t} \frac{Q(t)}{S} \sim I \\
	\end{align*}

\section{Катушка внутри катушки}
	Пусть есть две длинные вертикальные катушки, одна внутри другой:
	%    AAAAA
	%    A
	%BBBBAB
	%    AB
	%BBBBAB
	%    A
	%    AAAAA
	Пустили по первой катушке ток $I_1(t)$.
	Что будет?
	Во-первых, в первой катушке появится магнитное поле $B_1(T)$ (мы его даже вычисляли, оно пропорционально току).
	Значит, будет меняться магнитый поток через вторую катушку $\Phi_2(t)$.
	Значит, в ней возникнет ЭДС $\mathcal{E}_2(t)$.
	Значит, в ней возникает ток $I_2(t)$.
	Значит, опять возникает магнитное поле от второй катушки $B_2(t) \sim I_2(t)$.
	Оно пронизывает первую катушку, в ней возникнет ЭДС, потом возникнет ток.
	Круг замкнулся.

	На самом деле, даже если у нас одна катушка, то всё равно подобный цикл будет.
	Пустили ток, появилось магнитное поле, появился переменный магнитный поток, оно породил ЭДС, которое порождает ток.
	Возникший ток получается обратным.
	\begin{exmp}
		Взяли длинный провод, пустили ток.
		Магнитное поле маленькое.
		Потом свернули провод в катушку.
		Магнитное поле увеличелось, обратный ток увеличился, суммарный ток уменьшился.
		Это так называемое реактивное сопротивление.
		С ним активно борятся во всяких крутых аудиокабелях.
	\end{exmp}

\setauthor{Егор Суворов}

% 10.04.2015
\chapter{Теоремы Стокса и уравнения Максвелла}
\section{Закон Гаусса в дифференциальной форме}
	\subsection{Теорема Стокса}
		У нас частенько встречаются интегралы по замкнутой поверхости, например:
		\[ \int \vec j \cdot \vec n \cdot \d \vec S \]
		\begin{theorem}[Стокса]
			Пусть есть векторное поле $\vec \Omega$.
			То есть $\Omega$ "--- это какой-то вектор от точки.
			Пусть есть поверхность $\Sigma$, которая ограничивает объём.
			Тогда:
			\[ \int \vec \Omega \cdot \d \vec S = \int \underbrace{\vec \nabla \cdot \vec \Omega}_{\text{дивергенция}} \d V\]
			То есть поток через поверхность равен интегралу дивергенции по объёму.
		\end{theorem}
		\begin{proof}
			Возьмём кубик размером $\d x \times \d y \times \d z$, растущий из точки $(x, y, z)$
			Хотим посчитать поток $\vec \Omega$ через поверхность кубика.
			% картинка: кубик, x на нас, y вправо, z вверх
			% самая правая грань заштрихована
			Можно посчитать поток через каждую грань.
			Например, поток через выделенную грань равен $\d x \cdot \d z \cdot \Omega_y(x, y + \d y, z)$,
			если считаем, что в пределах грани поток не меняется.
			Если посчитаем еще и для симметричной грани и сложим, получим:
			\[ \d x \d z (\Omega_y(x, y + \d y, z) - \Omega_y(x, y, z)) = \d x \d z \left( \partd{\Omega_y}{y} \cdot \d y \right) \]
			Аналогично для оставшихся двух пар граней, получим полный поток:
			\begin{gather*}
			\d x \d y \d z \left( \partd{\Omega_x}{x} + \partd{\Omega_y}{y} + \partd{\Omega_z}{z} \right) \\
			\d V \vec \nabla \cdot \vec \Omega \\
			\end{gather*}
			Любой объём можно разбить на такие кубики и сложить.
			Получим в точности интеграл:
			\[ \int \vec \nabla \cdot \vec \Omega \d V\]
		\end{proof}
	
	\subsection{Уравнение}
		Теорема Гаусса говорит:
		\[ \oint \vec E \cdot \d \vec S = 4\pi Q = 4 \pi \int \rho \d V\]
		Если применим теорему Стокса, то получим следующее:
		\begin{align*}
		\oint \vec E \cdot \d \vec S &= \int \vec \nabla \cdot \vec \Omega \d V \\
		\int 4 \pi \rho \d V &= \int \vec \nabla \cdot \vec \Omega \d V \\
		4 \pi \rho &= \vec \nabla \cdot \vec \Omega
		\end{align*}
		Последнее "--- в точности первое уравнение Максвелла.
		Такая переформулировка теоремы Гаусса.
	
\section{Незакрученность поля в дифференциальной форме}
	\section{Теорема Стокса для другого измерения}
		Пусть есть векторное поле $\vec V$ и хотим сосчитать его цикруляцию.
		\begin{theorem}[Стокса]
			\[
			\underbrace{\int_C \vec V \cdot d l}_{\text{циркуляция}}
			=
			\int \left(\vec \nabla \times \vec V\right) \cdot \d \vec S
			=
			\int \rot \vec V \cdot \d \vec S
			\]
			То есть циркуляция поля равна потоку ротора через ограниченную контуром поверхность.
		\end{theorem}
		\begin{proof}
			<<Докажем>> только для маленьких квадратиков, потом оно честно-честно обощается.
			Взяли квадратик, растущий из $(x, y)$ размером $\d x \times \d y$.
			% картинка, ориентирован против часовой стрелки
			Аналогично, возьмём одно из четырех слагаемых цикруляции (для правого отрезка):
			\[ V_y(x + \d x, y) \d y \]
			Добавив второе слагаемое получим частную производную:
			\[ V_y(x + \d x, y) \d y - V_y(x, y) \d y = \d y \partd{V_y}{x} \d x \]
			Сложив вообще все слагаемые получим (обращаем внимание на минус, потому что $x$ и $y$ несимметричны):
			\[ \d x \d y \left(\partd{V_y}{x} - \partd{V_x}{y}\right) = \d x \d y \left(\rot \vec V\right)_z\]
			То есть получили в точности скалярное произведение $\rot \vec V$ на площадку $\d \vec S$.
			Но контур у нас не обязательно лежит целиком в плоскости, поэтому для симметрии мира
			добавим еще слагаемых, получим скалярное произведение $\rot \vec V$ на $\d \vec S$.
		\end{proof}
	\section{Уравнение}
		Возьмём интеграл электрического поля по замкнутому контуру (в электростатике):
		\[ \oint \vec E \cdot d \vec l = 0 \]
		Применим Стокса:
		\[ \oint \rot \vec E \cdot d \vec S = 0 \]
		Отсюда понято, что $\rot \vec E = 0$, потому что уравнение выше верно для абсолютно любого контура.
		Получили второе уравнение Максвелла:
		\[ \vec \nabla \times \vec E = 0 \]

		\begin{Rem}
			Для динамики в этом уравнении справа получится не ноль.
		\end{Rem}
		\begin{Rem}
			Есть теорема, что это уравнение и предыдущее однозначно определяют поле.
		\end{Rem}

\section{Формальные упражнения со значками}
	Напоминание:
	\begin{gather*}
	\vec \nabla \times \vec E = \rot \vec E \\
	\vec \nabla \cdot \vec E = \divergence \vec E \\
	\vec \nabla \times \phi = \grad \phi
	\end{gather*}
	Мы когда-то выясняли, что для электрического поля
	\[ \vec E = -\grad \phi \]
	Давайте теперь чисто формально посмотрим на $\vec \nabla \times (\vec \nabla \vec \phi)$.
	У нас векторное произведение двух векторов, параллельных $\nabla$, то есть оно равно нулю,
	то есть $\rot \vec E = 0$ "--- получили еще раз.

	Теперь возьмём $\vec E = - \vec \nabla \phi$ и $\vec \nabla \cdot \vec E = 4 \pi \rho$ и подставим одно в другое:
	\begin{gather*}
	\vec \nabla \cdot (-\vec \nabla \phi) = 4 \pi \rho \\
	\vec \nabla ^2 \phi = -4\pi \rho \\
	\underbrace{\Delta}_{оператор Лапласа} \phi = -4 \pi \rho
	\end{gather*}
	Это уравнение Пуассона.
	Оказывается, что существует единственное решение, если мы знаем граничное условие.
	Используется при моделировании потенциалов, когда нам сложно в каждой точке считать интегралы.

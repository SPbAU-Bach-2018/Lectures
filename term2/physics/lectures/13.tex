\setauthor{Егор Суворов}

% 29.05.2015
\section{Приближённые вычисления}
Вся лекция "--- лирическое отсутпление на тему, которая может пригодиться.


\subsection{Настройщики пианино}
Придумал Ферми, для иллюстрации того, как можно работать с приближёнными оценками.
Сколько в Чикаго есть настройщиков пианино?
Единственный факт, который мы возьмём из внешнего мира "--- население Чикаго, три миллиона.

Давайте сначала поймём, сколько в Чикаго семей.
Для этого посчитаем, сколько в семье в среднем человек.
Подняли руки в аудитории, посчитали братьев/сестёр, сказали, что примерно по четыре человека в семье.
Значит, $750\,000$ семей.
Дальше посчитали в аудитории, у кого есть пианино.
Внезапно, у половины.
Говорим, что тут университет и мы культурные, поэтому в среднем всё-таки будет поменьше: скажем, у трети.
Пианино положено настраивать раз в год.
Значит, в год надо сделать $250\,000$ настроек.
Посчитаем, сколько дней в году работает средний настройщик.
Выходных порядка сотни (50 недель плюс праздники), значит, 250 дней работает.
В день он может настроить, скажем, три пианино.
То есть один настройщик за год делает 750 настроек.
То есть настройщиков будет $\frac{250\,000}{750} = 300$.
Лезем в вольфрам, получаем число 290 (правда, для <<musical instrument repairers and tuners>>, но в порядок попали).

\subsection{Извержение вулкана}
Это пример из 2010 года, когда в Исландии извергался Эйяфьятлайокудль.
На дворе десятое апреля, вулкан извергает пыль, самолёты не летают.
Интересно: будут ли летать через две недели?
Чуть точнее: за какое время осядет вся пыль от вулкана?

Читаем новости.
Говорят, что максимальная высота, на которую выброшена пыль, равна $H=30$ км.
В другой передаче говорят, что частицы настолько крупные, что достигают радиуса $r=10^{-6}$ м.
Хотим посчитать время оседания, давайте для этого найдём установившуюся скорость падения.
Взяли пылинку, на неё действует сила тяжести ($mg$) и сила сопротивления.
С тяжестью всё ясно, давайте найдём сопротивление.
На первом курсе нашему лектору рассказывали, что есть число Рейнольдса, которое характеризует вязкость:
\[
Re = \frac{v \cdot r}{\nu}
\]
Если оно большое, то один тип движения, если оно маленькое (сильно меньше единицы), то другой тип движения.
Чуть точнее: если мало, то сила сопротивления примерно равна
\[ F=\rho v \nu r \]
Если велико, то равно
\[ F=\rho v^2 r^2 \]
Число Рейнольдса можно посчитать, зная скорость.
Замкнутый круг.
Давайте разомкнём: предположим что-нибудь случайное (мало/велико), посчитаем, проверим, что сошлось.
Например, что мало.
Тогда
\[ 
v
= \frac{mg}{\rho\nu r}
= \frac{\frac43\pi r^3 \rho_{\text{пыли}} g}{\rho_{\text{воздуха}} \nu r}
= \frac 43\pi \frac{\rho_{\text{пыли}} r^2 g}{\rho_{\text{воздуха}} \nu}
\]
Прикинем плотность пыли: пыль "--- это кусок вулкана.
Если бросить в воду "--- утонет, значит, плотнее.
Но не адски тяжёлый, наверное.
Возьмём $3000\frac{кг}{м^3}$.
\[
v
= 4 \cdot \frac{3000 \frac{кг}{v^3} \cdot 10^{-12} м^2 \cdot 10 \frac{м}{с^2}}{1 \frac{кг}{м^3} \cdot 1.5 \cdot 10^{-5} \frac{м^2}{с}}
= 4 \cdot \frac{3 \cdot 10^3 \cdot 10^{-12} \cdot 10}{1 \cdot 1.5 \cdot 10^{-5}}
= 10^{-2} \frac{м}{с}
\]
Теперь проверим число Рейнольдса:
\[
Re = \frac{10^{-2} \cdot 10^{-6}}{10^{-5}} = 10^{-3}
\]

Итого пыль падает со скоростью $1\frac{см}{с}$.
Довольно реалистично.
Посчитаем время:
\[
T = \frac{H}{v} = \frac{3 \cdot 10^4 м}{10^{-2} \frac{м}{с}} = 3 \cdot 10^6 c = 1 месяц
\]
И сошлось, как ни странно.
Понятно, что чем дальше от Исландии, тем меньше эффект.
Но в самой Исландии самолёты начали летать через месяц.
Понятно, что мы тут всё делали с точностью до порядка, поэтому если надо, скажем, планировать поездку через две недели, то не прокатит.
Но потом прикольно.

\subsection{Челябинский метеорит}
2013 год.
Челябинск.
Взрывается метеорит.
Говорят, что в радиусе $R=8 км$ выбиты все окна.
Хочется узнать, какая была масса у метерорита (когда он еще летел).
Понятно, что в момент взрыва вся энергия $E$ метеорита перешла в энергию для выбиванию окон.

Надо понять, какое должно быть ударное давление на фронте ударной волны, чтобы выбить стекло.
Давайте сначала попробуем понять, на каком расстоянии будет ударная волна (которая распространяется в воздухе плотностью $\rho$) через время $T$ после взрыва.
Попробуем подогнать под размерности (называется методом размерностей).
\begin{gather*}
[E] = Дж = Н \cdot м = \frac{кг \cdot м}{с} \cdot m = \frac{кг \cdot м^2}{с} \\
[\rho] = кг \cdot м^3 \\
[r] = м \\
[t] = с \\
\end{gather*}
Давайте угадывать.
Слева надо написать $r$ (размерности метр), а справа "--- какую-то комбинацию.
Сначала сократим килограммы, поделив $E$ на $\rho$ (а не наоборот, потому что хотим метры в числителе).
Получили $\frac{м^5}{с^2}$.
Взяли корень пятой степени, домножили на время, получили
\[
r = \left(\frac{E}{\rho}\right)^{1/5} \cdot t^{2/5}
\]
Как ни странно, она правильная.

А выяснили эту формулу, когда надо было по опубликованным фотографиям ядерного взрыва (всё остальное засекречно) посчитать мощность бомбы.
Это проделал английский учёный Тейлор, опубликовал статью, потом на него чуть ли не дело завели, думая, что информацию он не вычислил, а получил.

Продолжим выяснять давление.
Сначала скорость:
\[ v = \frac{r}{t} = \left(\frac E \rho\right)^{1/5} \cdot r^{-3/5} = \left(\frac E \rho\right)^{1/2} \cdot r^{-3/2} \]
Теперь давление:
\[ p = \rho v^2 = \frac{E}{r^3} \]

Выясним давление $P_*$, при котором разбивается стекло.
Например, знаем, что если кинуть молоток в стекло, то оно разобьётся.
У него масса, скажем, килограмм, скорость метр в секунду (довольно много, попробуйте представить), если он передаст стеклу импульс за время $\delta t$, то получим вот такое давление:
\[
P_* = \frac{mv}{S\delta t}
\]
Скажем, стекло имеет площадь 1 кв. м, оценим $\delta t$.
Это примерно время, за которое молоток пройдёт расстояние, равное толщине стекла.
\[
P_* = \frac{1~кг \cdot 1\frac{м}{с}}{1~м^2\cdot\frac{10^{-2}м}{1 м/с}} = 10^2~Па
\]
Итого знаем, что $\frac E {r^3} = P_*$.

Что такое энергия метеорита?
Это изначально была только потенциальная энергия гравитационного взаимодействия с землёй, которая потом потихоньку переходит в кинетическую.
Со школы знаем формулу (где $m$ "--- масса метеорита, $M$ "--- масса Земли, $R$ "--- её радиус):
\[ E = G \frac{mM}{R} \]
Получили уравнение:
\begin{gather*}
\frac{G\frac{mM}{R}}{r^3} = P_*
\Ra
m = \frac{P_*r^3R}{MG}
= \frac{10^2 \cdot (8 \cdot 10^3)^3 \cdot (6 \cdot 10^6)}{6 \cdot 10^{-11} \cdot 10^{24}}
= \frac{10^2 \cdot (8 \cdot 10^3)^3 \cdot (6 \cdot 10^6)}{6 \cdot 10^{-11} \cdot 10^{24}} = \\
= \frac{10^2 \cdot 5\cdot10^2 \cdot 10^9 \cdot 10^6}{10^{13}}
= 5 \cdot 10^6~кг
\end{gather*}
Открываем английскую Википедию, видимо, что масса астероида была 10 килотонн.

Наглядно видно, что мы забили на безразмерные константы, но почему-то всё равно сошлось по порядку.
ВНЕЗАПНО всякие такие константы почему-то имеют порядок <<единица>>.
Загадка, но удобно.

Хотя вот к скорости молотка у нас чувствительно: если увеличим её на порядок, то масса метеорита вырастет на два порядка.

\subsection{Рэлей, Максвелл и Эверест}
Письмо Максвелла к Релею: <<сижу на террасе в гостинице, хорошая погода, море, видно Эверест>>.
Максвелл также удивляется, почему видно: до Эвереста 150 км, что довольно много.

Послушаем, отчего свет вообще затухает.
Смысл такой: вот есть молекулы-диполи (у которых два полюса на расстоянии), при облучении этих полюсов-зарядов переменным электрическим полем (светом), рассеивают энергию.
Вводят величину <<поляризация>>: это количество молекул в единице объёма ($N$, в $\frac{шт}{см^3}$), умноженное на дипольный момент ($d=\alpha E$, где $\alpha$ "--- константа, $E$ "--- поле).
Два точных факта:
\begin{enumerate}
\item $\varepsilon = 1 + 4\pi N \alpha$ (диэлектрическая проницаемость связана с $\alpha$).
\item $n^2 = \varepsilon \Ra n^2 - 1 = 4\pi N \alpha$ (показатель преломления зависит от диэлектрической проницаемости).
\end{enumerate}
Выведем мощность рассеивания $M_1$ диполя длиной $d$, который колеблется с частотой $\omega$:
\[ M_1 \sim \frac{\omega^4}{c^3}d^2 \]
На самом деле, там коэффициент есть и его надо учитывать.
Но, внезапно, коэффициент тоже около единицы "--- $\frac 23$.
Соответственно, $N$ диполей в одном кубическом сантиметре рассеят столько энергии:
\[ M=NM_1 = \frac 23 N \frac{\omega^4}{c^3} \alpha^2 E^2 \]

Откуда это рассеиваемая энергия берётся?
По формуле, было электрическое поле $E$, поток энергии (энергия на единичную площадку) есть:
\[ F = \frac1{4\pi} c E^2 \]
Возьмём производную по длине путешествия "--- это сколько потока энергии рассеется за единицу длины.
Мы знаем, сколько энергии рассеивается, приравняем:
\[ \frac{\d F}{\d x} = -\frac 23 N \frac{\omega^4}{c^3} \alpha ^2 E^2 \]
$E^2$ "--- это почти $F$, подставим:
\[
\frac{\d F}{\d x} = -\frac{8\pi}{3} \frac{\omega^4}{c^4} \alpha^2 N F
\Ra
F \sim e^{-\frac{8\pi}{3} \frac{\omega^4}{c^4} \alpha^2 N x}
\]
Физический смысл такой: свет затухает в $e$ раз по прошествии $L=\frac{3}{8\pi} \left(\frac{c}{\omega}\right)^4 \frac{1}{\alpha^2 N}$
(называется <<характерная длина>>).
Отсюда мы знаем всё: скорость света, число молекул, частоту изменения диполя (это просто частота видимого света), кроме $\alpha$.
Но мы знаем, что $\alpha$ связана с показателем преломления, можно подставить:
\[
L = \frac{3}{8\pi} \left(\frac{c}{\omega}\right)^4 \frac{(4\pi)^2}{(n^2-1)^2 N}
= 6\pi \frac{N}{(n^2-1)^2} \left(\frac{c}{\omega}\right)^4
\]
Показатель преломления воздуха, увы, близок к единице.
Поэтому разложим $n^2-1=(n+1)(n-1)=2(n-1)$.
\[
L = \frac32\pi \frac{N}{(n-1)^2} \left(\frac{c}{\omega}\right)^4
\]
Посчитаем $N$:
\[
N=\frac{N_A}{22.4} \cdot 10^{-3} = \frac{6\cdot 10^{23}}{22.4} \cdot 10^{-3} = 6.7 \cdot 10^{19}
\]
Показатель преломления равен $n-1=0.29 \cdot 10^{-3}$.

Разберёмся с $\frac{c}{\omega}$.
В горах обычно синий свет, его длина волны равна $\lambda=5.8 \cdot 10^{-5}~см$.
Можно вывести частоту: $\lambda = \frac{2\pi c}{\omega}$.
Тут у нас кончилось время на лекции, можно всё подставить и должно получиться
\[ L = 120 км \]

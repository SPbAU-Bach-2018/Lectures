\setauthor{Егор Суворов}

% 13.03.2015
\section{Рассуждения про энергию конденсатора}
	Посмотрим на пружину.
	Когда мы её растягиваем, то силы взаимодействия между частями пружины
	меняются, возникает потенциальная энергия.
	А где сидит энергия в конденсаторе?
	Она находится в электрическом поле между пластинами.
	\[ W = \frac{CU^2}2 = \frac{C(Ed)^2}2 = \frac{S}{4\pi d}\cdot\frac{E^2d^2}{2} = Sd \frac{E^2}{8\pi} \]
	Тут $Sd=V$ "--- это объём поля. 
	\begin{Def}
		Плотность энергии электрического поля: $\mathcal{W} = \frac{E^2}{8\pi}$
	\end{Def}
	\begin{Rem}
		Мы это выяснили для плоского конденсатора, но вообще это верно для любого поля.
	\end{Rem}

	Задача: подключили батарейку к конденсатору.
	Что будет, если мы попробуем раздвинуть обкладки конденсатора?
	\cautoimg{05/01}
	Два соображения:
	\begin{enumerate}
	\item
		Раздвигаем обкладки, расстояние растёт.
		По формуле $C=\frac{S}{4\pi d}$ ёмкость падает, напряжение постоянное, энергия растёт.
		Конечная энергия меньше начальной.
		То есть система совершила работу.
	\item
		Так как пластины заряжены по-разному, то они стремятся соединиться.
		Раздвигая обкладки, мы работаем против них.
		Значит, работу совершаем мы.
	\end{enumerate}
	Разрешение парадокса: мы во втором пункте ничего не сказали про заряд.
	Из определения ёмкости знаем, что $Q=CU$ (тут $Q$ "--- заряд на обкладке).
	Так как $U$ постоянно, а ёмкость падает, то $Q$ тоже падает.
	То есть заряд перетаскивается с одной обкладки на другую, это делает батарейка и совершает работу.
	% Интересно, а меняется ли энергия конденсатора-таки? Я не помню

\section{Диэлектрики}
	\subsection{Определение и свойства}
		\begin{Def}
			В диэлектрике заряды свободно перемещаться не могут, хотя могут немножко отходить от своих положений.
		\end{Def}
		Давайте посмотрим, как диэлектрики себя ведут в электрическом поле.
		Пусть есть картинка:
		\cautoimg{05/02}
		Есть силовые линии сверху вниз.
		При этом в проводнике электрического поля нет.
		Таким образом, на поверхности проводника создаётся электрический заряд, компенсирующий внешнее поле.
		\cautoimg{05/03}

		Теперь пусть у нас не проводник, а диэлектрик.
		На нём уже таких ограничений нет, электрическое поле внутри может существовать.
		\cautoimg{05/04}
		Но идеальных диэлектриков, которые не чувствуют поле, всё-таки не бывает.
		Поэтому внизу всё равно образуются положительные заряды, а наверху "--- отрицательные.
		Но их будет недостаточно для полной компенсации поля.
		Это явление называется \textit{поляризацией диэлектрика}.
		\cautoimg{05/05}

	\subsection{Поле внутри диэлектрика}
		Пусть плотность зарядов на поверхности диэлектрика равна $\sigma_p$.
		Пусть пластины заряжены на $\pm Q$, плотность заряда на каждой равна $\sigma$.
		Сначала вычислим поле между пластинами:
		\[E_{\text{между пластинами}} = 4\pi \sigma\]
		Поле, которое генерируется диэлектриком, направлено в противоположную сторону.
		Значит, суммарое поле в диэлектрике:
		\[ E_\text{в диэлектрике}
			= 4\pi\sigma - 4\pi\sigma_p
			= 4\pi(\sigma - \sigma_p)
			= 4\pi\frac{\sigma - \sigma_p}{\sigma}\sigma
			= \frac{4\pi\sigma}{\epsilon}
		\]
		\begin{Def}
			$\epsilon = \frac{\sigma - \sigma_p}{\sigma}$ "--- \textit{диэлектрическая проницаемость}.
		\end{Def}
		\begin{Rem}
			Пока верим, что $\epsilon$ "--- константное свойство вещества.
		\end{Rem}
		\begin{exmp}$\epsilon_\text{фарфора} = 6$\end{exmp}
		\begin{exmp}$\epsilon_\text{стекла} = 10$\end{exmp}
		\begin{exmp}$\epsilon_\text{дерева} = 4$\end{exmp}
		\begin{Rem}
			Полностью погасить поле диэлектрик не может "--- заряды-то несильно перемещаются.
			А если сильно "--- это уже проводник.
		\end{Rem}

	\subsection{Диэлектрик в конденсаторе}
		\subsubsection{Задача}
			Пусть есть плоский конденсатор ёмкостью $C_0$.
			Внесли внутрь диэлектрик с известным $\epsilon$, который полностью заполнил пространство между пластинами:
			\cautoimg{05/06}
			Вопрос: что будет?

		\subsubsection{Без батарейки}
			У нас в $\epsilon$ раз уменьшилось поле, а заряды остались прежними.
			То есть разность потенциалов уменьшилась в $\epsilon$ раз ($U=Ed$).
			А ёмкость из формулы $C=\frac{Q}{U}$ увеличилась в $\epsilon$ раз.

		\subsubsection{Добавили батарейку}
			Пусть изначально был заряд $Q_0=C_0U$.
			Внесли диэлектрик.
			Напряжение остаётся прежним (его сохраняет батарейка), то есть из формулы $U=Ed$ (где $d$ "--- расстояние между пластинами)
			знаем, что суммарное поле внутри конденсатора (с учётом диэлектрика) должно остаться прежним.
			Но у нас есть диэлектрик, он всегда уменьшает порождаемое обкладками поле в $\epsilon$ раз.
			То есть это порождаемое поле должно увеличиться в $\epsilon$ раз для компенсации этого эффекта,
			единственный способ "--- увеличить заряды в $\epsilon$ раз.
			То есть $Q_1=\epsilon Q_0$, заряд обкладок поменялся.
			\begin{Rem}
				Заряд мог поменяться только через батарейку.
				Через неё протекло $\Delta Q = Q_1 - Q_0 = (\epsilon - 1)C_0U$.
			\end{Rem}

		\subsubsection{Считаем заряды на диэлектрике внутри конденсатора}
			Пусть есть две пластины, без диэлектрика.
			\[ E = 4\pi \sigma = 4\pi \frac{Q_0}S = \frac U d\]
			Внесли диэлектрик.
			Пусть на обкладках теперь заряды $\pm Q_1$, а в диэлектрике заряды $\pm Q_p$.
			\cautoimg{05/07}
			С одной стороны, поле всё еще $\frac U d = \mathrm{const}$, а с другой есть формула $E = 4\pi \sigma$:
			\[ \frac U d = E = 4\pi \frac{Q_1-Q_p}{S}\]
			Таким образом $Q_1-Q_p=Q_0 \iff Q_p=Q_1-Q_0=\Delta Q = (\epsilon-1)C_0U$.

	\subsection{Неполное заполнение диэлектриком}
		Пусть конденсатор подключен к батарейке и в него частично внесли диэлектрик:
		\cautoimg{05/08a}
		Вопрос: что будет происходить с диэлектриком?

		В зависимости от $x$ у нас меняется потенциальная энергия (при полном погружении ёмкость и энергия одни,
		при отсутствии "--- другие).
		То есть в системе <<конденсатор--диэлектрик>> возникает сила, заставляющая его втягиваться в конденсатор:
		\[ F = - \frac{\partial W}{\partial x} \]
		\begin{Rem}
			Можно доказать, что диэлектрик всегда стремится туда, где поле больше
			(с учётом того, что в диэлектрике поле уменьшается)
		\end{Rem}

		Знаем, что $W = \frac{C(x)U^2}{2}$.
		Напряжение постоянно, а ёмкость меняется.
		\[F = -\frac{\partial\left(\frac{C(x)U^2}{2}\right)}{\partial x} = -\frac{U^2}{2} \cdot \frac{\partial C}{\partial x}\]
		Давайте распилим конденсатор на два, в одном диэлектрика нет, в другом "--- полностью есть.
		\cautoimg{05/08b}
		Эти конденсаторы не просто находятся рядом, а соединены параллельно, так как напряжения на обкладках одинаковы.
		Их ёмкости "--- $C_1$ и $C_2$.
		Суммарная ёмкость равна $C_1+C_2$.
		Пусть площади обкладок равны $S_1$ и $S_2$, а ширина обкладок равна $H$:
		\cautoimg{05/09}
		\begin{gather*}
		C_1=\frac{S_1}{4\pi d} = \frac{xH}{4\pi d} \\
		C_2=\epsilon\frac{S_2}{4\pi d} = \epsilon\frac{(L-x)H}{4\pi d} \\
		C(X) = \frac{H}{4\pi d}(x+\epsilon(L-x)) \\
		F(X) = -\frac{U^2}{2} \cdot \frac{H}{4\pi d} \cdot \frac{\partial}{\partial x}(x + \epsilon(L-x))
		     = \frac{U^2}{2} \cdot \frac{H}{4\pi d}(\epsilon-1)
		     = \frac{U^2H}{8\pi d} (\epsilon-1)
		\end{gather*}

\chapter{Потенциалы}
\setauthor{Егор Суворов}

% 06.03.2015
\section{Напоминание про потенциалы}
  Уже есть понятие потенциала и разности потенциалов.
  Если есть поле $\vec E$ и хотим посчитать разность потенциалов (aka напряжение) между точками $A$ и $B$ (порядок важен),
  то надо взять пробный заряд $q_0$ и протащить его по какой-нибудь траектории из $A$ в $B$,
  посчитали какую-то работу:
  \[A = -q_0 \int_A^B \vec E \cdot \d\vec l\]
  А отсюда как-то определили разность потенциалов в поле:
  \[\phi_B-\phi_A = \frac{A}{q_0} = -\int_A^B \vec E \cdot \d \vec l\]
  Если поле статическое, то интеграл от траектории не зависит, зависи только от положения концов.

  Также посчитали разность потенциалов в случае поля точечного заряда $Q$ (не очень точно со знаками):
  \[ \phi_B-\phi_A = \frac{Q}{r_b} - \frac{Q}{r_a}\]

\section{Разность потенциалов в однородном поле}
  Пусть в любой точке пространства поле направлено в одну сторону и имеет модуль $E$.
  Рассмотрим две точки: $A$ и $B$, на расстоянии $d$.
  Хотим посчитать разность потенциалов.
  Сначала предположим, что $\vec{AB} \upuparrows \vec E$.
  Тогда скалярное произведение в интеграле заменяется на обычное:
  \[\phi_B - \phi_A = -\int_A^B \vec E \cdot \d \vec l = -\int_A^B E \d l = -E \int_A^B \d l = -Ed \]
  Теперь пусть у нас $\vec{AB}$ идёт под наклоном.
  Давайте возьмём траекторию в виде ломаной: сначала мы идём вдоль силовых линий поля, а затем перпендикулярно.
  На втором отрезке работа равна нулю "--- мы идем по эквипотенциальной поверхности.
  На первом отрезке работу уже посчитали, потому что он направлен вдоль силовых линий.
  % A левее, чем B
  % \phi_b - \phi_a = -Ed, где $d$ "--- расстояние по силовым линиям
  \begin{Rem}
    Потенциал <<убывает>> в ту же сторону, что и направлены силовые линии.
  \end{Rem}

\section{Потенциальная энергия системы двух зарядов}
  Пусть есть заряд $Q$, а на расстоянии $r$ от него "--- заряд $q$.
  Хотим посчитать потенциальную энергию такой системы (обозначим как $W_{пот}$)
  Наверное, стоит сначала понять, что такое потенциальная энергия такой системы.
  \begin{Def}
    Потенциальная энергия системы двух зарядов "--- работа, требуемая, что притащить
    эти заряды с бесконечности на расстояние $r$ друг к другу.
  \end{Def}
  \begin{Rem}
    То же самое с другим знаком: какая работа требуется, чтобы растащить два заряда друг от друга на бесконечность.
  \end{Rem}
  Будем считать, что заряд $Q$ фиксирован.
  Тогда перемещается заряд $q$.
  Посчитаем разность потенциалов между его двумя положениями:
  \[W_{пот} = q \Delta \phi = q \frac{Q} r = \frac{qQ}{r}\]
  Заметим, что получилось симметрично по зарядам, что хорошо (они ничем не отличаются).

\section{Ёмкость}
  Пусть есть проводник, который мы зарядили зарядом $Q$.
  У него появился потенциал.
  \begin{Def}
    Потенциал проводника "--- разность потенциалов между бесконечностью (где потенциал ноль) и точкой на его поверхности.
    Напоминаем, что во всех точках поверхности проводника потенциалы одинаковы.
  \end{Def}
  Хотим понять, как потенциал проводника зависит от $Q$.
  Наверое, линейно: на поверхности есть много небольших точечных зарядов, каждый из которых генерирует потенциал линейно от своего заряда.
  \begin{Def}
    Ёмкость проводника $C=\frac{Q}{\phi}$, $[C]=\Phi$ (Фарады).
  \end{Def}

  \subsection{Ёмкость сферы}
    Считаем по определению ёмкость сферы радиусом $R$ и зарядом $Q$.
    Надо определить потенциал на поверхности сферы.
    \subsubsection{Поле сферы}
      Пусть есть точка на расстоянии $r$ от центра ($r>R$), хотим найти $E(r)=?$.
      Поле, очевидно, будет сферически симметрично, то есть направлено по радиусу.
      Воспользуемся теоремой Гаусса: взяли концентрическую сферы радиусом $r$, это поверхность.
      Поле везде перпендикулярно поверхности и этой сферы.
      Посчитаем поток через неё (скалярные произведения в интеграле уходят):
      \[E \cdot (4\pi r^2)\]
      По теореме Гаусса:
      \[E 4 \pi r^2 = 4\pi Q \Ra E = \frac{Q}{r^2}\]
      Так как сфера проводник, то внутри сферы (при $r<R$) поле равно нулю.
      % график
    \subsubsection{Поле шара}
      В точке на расстоянии $r>R$ тоже можем воспользоваться теоремой Гаусса и получить $E=\frac{Q}{r^2}$.
      Интересно внутри, при $r<R$.
      Введём плотность заряда в шаре: $\rho = \frac{Q}{\frac43\pi R^3}$ (будем считать, что заряд равномерен).
      Опять взяли поверхность радиуса $r$ и посчитали поток через неё: $\Phi = E \cdot 4\pi r^2$.
      Осталось найти заряд внутри и приравнять теоремой Гаусса:
      \begin{gather*}
      \Phi = E \cdot 4\pi r^2
      = 4\pi \left(\rho \frac43\pi r^3\right)
      = 4\pi \left(\frac{Q}{\frac 43\pi R^3} \frac43\pi r^3\right)
      = 4\pi \cdot \frac{Qr^3}{R^3} \\
      E
        = \frac{\Phi}{4\pi r^2} = \frac{4\pi Qr^3}{R^3} \cdot \frac{1}{4\pi r^2}
        = \frac{Q}{R^3} \cdot r
      \end{gather*}
      \begin{Exercise}
        Мы когда-то доказывали, что поля внутри проводника нет.
        А вот внутри шара, получается, есть.
        Как так?
      \end{Exercise}
      % график
    \subsubsection{Потенциал сферы}
      Ищем потенциал сферы "--- это, по определению, разность потенциалов между  бесконечностью и поверхностью.
      То есть надо посчитать интеграл:
      \begin{gather*}
      \phi_{R} - \phi_{\infty}
        = -\int_{\infty}^{R} \frac{Q}{r^2} \d r
        = \int_{R}^{\infty} \frac{Q}{r^2} \d r
        = \left(-\frac{Q}{r}\right)\biggr|_R^\infty
        = \frac{Q}{R}
      \end{gather*}
    \subsubsection{Ёмкость сферы}
      Теперь делим заряд на потенциал:
      \begin{gather*}
      C = \frac{Q}{\phi} = \frac{Q}{\frac QR} = R
      \end{gather*}

\section{Конденсаторы}
  \subsection{Ёмкость системы проводников}
    Пусть есть два проводника, подсоединили батарейку к ним с напряжением $U$.
    Батарейка должна как-то создать разность потенциалов в проводниках.
    Для этого она будет \textit{перетаскивать заряды с одного проводника на другой}.
    Внутри самой батарейки про заряды мы ничего не знаем.
    Теперь проводники как-то зарядились, будем для простоты считать, что на них заряды $\pm Q$
    (могут быть и разные, но пока не будем про это думать).
    \begin{Def}
      Обобщение ёмкости на систему из двух проводников с зарядами $\pm Q$:
      $C=\frac QU$.
    \end{Def}

  \subsection{Плоский конденсатор}
    Есть две пластины площади $S$ с зарядами $\pm Q$ на расстоянии $d$
    (такая система и называется \textit{плоским конденсатором}).
    Для простоты будем считать, что сами пластины бесконечно большие, то есть $d \ll \sqrt S$.

    Поле одной пластины мы уже считали, оно однородно.
    Для положительно заряженой пластины силовые линии исходят из неё и направлены перпендикулярно поверхости,
    при этом величина поля постоянна и не зависит от расстояния.
    Для отрицательно заряженной "--- наоборот (силовые линии в неё втыкаются).
    Посмотрим на сумму.
    Снаружи пластин поле равно нулю, а внутри поле постоянно и равно
    \[E_{\text{внутри}} = 2\cdot(\text{поле от пластины}) = 2\cdot(2\pi \sigma) = 4\pi\sigma\]
    где $\sigma$ "--- поверхностная плотность заряда ($\frac Q S$).
    Теперь надо посчитать разность потенциалов между пластинами.
    Там однородное поле, напряжение в нём мы уже считали, получаем (с точностью до знака?)
    \[U = 4\pi\sigma \cdot d\]

    Считаем ёмкость системы:
    \begin{gather*}
    C = \frac Q U = \frac Q {\Delta \phi} = \frac Q {4 \pi \frac{Q}{S} d} = \frac{S}{4\pi d}
    \end{gather*}

  \subsection{Система из двух конденсаторов (параллельно)}
    Пусть есть конденсаторы с ёмкостью $C_1$ и $C_2$, с зарядами $q_1$ и $q_2$.
    Соединим параллельно (плюс с плюсом, минус с минусом).
    На самом деле, левые пластины конденсаторов "--- это один проводник, с зарядом $q_1+q_2$.
    Правые пластины "--- с зарядом $-(q_1+q_2)$.
    Ёмкость по определению равна $C = \frac{q_1+q_2}{U}$.
    Заметим, что напряжение между парами пластин одинаково
    (потому что потенциал на поверхностях пластин одинаковый для каждой пластины).
    Тогда:
    \[ C = \frac{q_1 + q_2}{U} = \frac{q_1}{U} + \frac{q_2}{U} = C_1+C_2\]
    \begin{Rem}
    Ёмкость двух последовательно соединённых конденсаторов чуть сложнее, мы этого делать не будем.
    \end{Rem}

  \subsection{Потенциальная энергия конденсатора}
    Попытаемся его посчитать.
    Пусть изначально были нейтральные пластины.
    Будем перетаскивать заряды по одному, каждый размером $\Delta q$.
    Сначала никакой работы не требуется.
    Потом пластины зарядятся на $\pm \Delta q$, между ними возникнет поле.
    Для следующего перетаскивания надо совершить работу для преодоления поля.
    Посчитаем, сколько работы в сумме потребуется.

    Пусть у нас на пластинах сейчас $\pm q$, хотим перетащить заряд $\Delta q$.
    Надо совершить работу по перетаскиванию одного заряда:
    \[\Delta A=\Delta q \cdot E d = \Delta q \cdot U = \Delta q \frac q C\]
    Посчитаем полную работу:
    \[A = \int_0^Q \d q \cdot \frac q C = \frac 1C \cdot \frac{Q^2}{2} = \frac{Q^2}{2C}\]
    Это и есть энергия конденсатора ($W_к$, русское <<к>>).
    Можно переписать:
    \[W_к = \frac{Q^2}{2C} = \frac{CU^2}{2}\]

  \subsection{Замыкание конденсаторов с разной полярностью}
    Пусть есть два конденсатора (один с перевёрнутой полярностью) с ёмкостями $C_1 > C_2$, оба заряжены до напряжения $U$.
    Потом соединили так:
\begin{verbatim}
┌─ +|C_1|   ─┐
│            │
└─  |C_2|+  ─┘
\end{verbatim}
    Вопрос: какое будет напряжение между обкладками конденсатора?

    Посмотрим на изначальные заряды на обкладках (буква $i$ от <<initial>>):
    \begin{align*}
      q_{1i} &= C_1 U_i \\
      q_{2i} &= -C_2 U_i \\
      q_i &= (q_{1i}+q_{2i}) = (C_1-C_2) U_i > 0
    \end{align*}
    После соединения заряды перераспределятся так, чтобы потенциалы на каждый обкладке были одинаковы.
    Так как $C_1>C_2$, то слева будет сплошой плюс, а справа "--- сплошной минус.
    Получили два параллельно соединённых конденсатора, можно посчитать их ёмкость: $C_1+C_2$.
    Пусть в конце напряжение равно $U_f$.
    При этом заряд никуда не убежал, поэтому на левой обкладке системы он по-прежнему равен $q_i$ с одной стороны.
    С другой стороны его можно посчитать, потому что мы знаем все параметры конденсатора в результате:
    \[ (C_1+C_2)U_f = (C_1-C_2)U_i \Ra U_f = U_i\frac{C_1-C_2}{C_1+C_2}\]

    Посчитаем энергию такой системы в начале и в конце:
    \begin{gather*}
    W_i = \frac{C_1U_i^2}{2} + \frac{C_2U_i^2}{2} = \frac{(C_1+C_2)U_i^2}{2} \\
    W_f = \frac{(C_1+C_2)U_f^2}{2} = \frac{(C_1-C_2)U_i^2}{2(C_1+C_2)}
    \frac{W_f}{W_i} = \left(\frac{C_1-C_2}{C_1+C_2}\right)^2 < 1
    \end{gather*}
    Видно, что энергия куда-то делась, нехорошо.
    На самом деле, если аккуратно учесть сопротивление провода, то станет видно, что энергия ушла на его нагрев.

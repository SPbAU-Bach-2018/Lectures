\chapter{Электродинамика}
\setauthor{Егор Суворов}

% 20.03.2015
\section{Введение}
  Эксперимент со школы: если компас рядом с проводом, в котором течёт ток, то стрелка отклоняется.
  Это показывает, что ток создаёт некое магнитное поле.

  Теория электромагнетизма существует с XIX века, она очень красивая и есть куча интерпретаций.
  Но мы не физики, времени у нас мало, поэтому у нас будут только огрызки этой теории, в которых,
  впрочем, всё равно можно увидеть некоторую симметрию.

\section{Электрический ток}
  Пусть имеется площадка площадью $\d S$, к этой площадке проведена нормаль $\vec n$.
  Введём вектор $\vec j$, который показывает количество и направление зарядов,
  которые протекают через эту площадку за единицу времени.

  Поясним физический смысл этого вектора.
  Пусть есть объёмная плотность заряда $\rho$, заряды движутся со средней скоростью $\vec v$.
  Как посчитать, сколько зарядов протекает через площадку ($\d q$)?
  Надо посчитать суммарный <<объём>> этих зарядов:
  \cautoimg{06/01}
  \begin{gather*}
    \d q = \rho \cdot \d V \\
    \d V = |v| \cdot \d t \cdot \d S
  \end{gather*}
  Если скорость скошена относительно нормали, то формулы чуть меняются:
  \cautoimg{06/02}
  \begin{gather*}
    \d V = \vec v \cdot \vec n \cdot \d t \cdot \d S \\
    \d q = \rho \cdot \d V = \rho \cdot \vec v \cdot \vec n \cdot \d t \cdot \d S
    \frac{\d q}{\d t} = \underbrace{\rho \cdot \vec v}_{\vec j} \cdot \vec n \cdot \d S
  \end{gather*}
  Таким образом, физический смысл $\vec j$ "--- сколько зарядов и куда течёт (кажется, это называется <<плотность тока>>).

  \begin{Def}
    Электрический ток $I$:
    \[ I = \oint_S \vec j \cdot \vec n \cdot \d S\]
    Физический смысл: с какой скоростью за единицу времени заряд утекает из объёма, ограниченного поверхностью.
  \end{Def}

  \begin{theorem}[закон сохранения электрического заряда]
    Сколько утекло из объёма "--- на столько внутри заряд и уменьшился.
    \[\oint_S \vec j \cdot \vec n \cdot \d S = -\frac{\d Q_\text{внутри}}{\d t}\]
  \end{theorem}

  Теперь подумаем, что такое <<заряд внутри>>.
  Если есть объёмная плотость $\rho(\vec r)$, то его посчитать просто:
  \[ Q_\text{внутри} = \oint \rho(\vec r) \d V\]
  Подставим:
  \begin{align*}
    \oint_S \vec j \cdot \vec n \cdot \d S
      &= -\frac{\d}{\d t} \oint \rho(\vec r) \d V \\
      &= -\oint \frac{\d\rho}{\d t} \d V
  \end{align*}
  Утверждение из матанализа (теорема Стокса):
  \begin{align*}
    \oint_S \vec j \cdot \vec n \cdot \d S &= \oint_V \vec\nabla \cdot \vec j \cdot \d V
  \end{align*}
  Напоминаем: $\vec\nabla$ "--- оператор Набла, мы уже когда-то градиент с его помощью обозначали.
  Напоминаем, что такое \textit{дивергенция векторной функции} $\vec\nabla \cdot \vec j$:
  \[ \vec\nabla \cdot \vec j = \partial{j_x}{x} + \partial{j_y}{y} + \partial{j_z}{z}\]

  Пояснение про теорему Стокса:
  Мы ей уже пользовались в матан при вычислении определённого интеграла:
  \[ \int_a^b f(x) \d x = F(b) - F(a)\]
  Геометрический смысл: определяем интеграл <<по отрезку>> как разность двух значених первообразных на нульмерных границах этого отрезка.
  То есть свели нечто по одномерному объекту к нульмерному.
  А в нашем случае мы переписали интеграл по объему как по поверхности (точнее, конкретно в нашем случае пошли с другой стороны).

  Результат применения теоремы Стокса:
  \begin{gather*}
    \oint \vec\nabla \cdot \vec j \cdot \d V = -\oint \frac{\d \rho}{\d t} \d V
    \Rightarrow
    \vec\nabla \cdot \vec j = -\partd{\rho(\vec r, t)}{t}
  \end{gather*}
  Это \textit{уравнение непрерывности}.
  Вообще говоря, оно верно для любых потоков, например, гидродинамических.
  Частный случай "--- когда скорость не зависит от времени.
  Тогда частная производная в правой части равна нулю и $\vec\nabla \cdot \vec j = 0$.

\section{Ток в магнитном поле}
  Пусть снаружи есть магнитное поле $\vec B$ (что такое <<магнитное поле>>, выясним позже).
  Магнитное поле легко заметить: оно действует на движущийся со скоростью $\vec v$ заряд $q$ силой Лоренца:
  \[ F_\text{Лоренца} = q (\vec v \times \vec B)\]

  Если у нас есть ток в проводе, то можно его рассматривать просто как поток зарядов.
  Если снаружи есть магнитное поле, то оно действует на каждый из зарядов.
  Рассчитаем силу, которая будет действовать на заряды тока $I$ со стороны магнитного поля $\vec B$.
  Для этого посмотрим на кусок провода объёмом $\d V$ и с концентрацией зарядов $n$ (сколько зарядов на единицу объёма)
  \cautoimg{06/03}
  Очевидно, в куске провода $n \d V$ зарядов.
  Если каждый из них имеет заряд $q$, то суммарный заряд равен $\d Q = q n \d v = \rho \d V$
  (можно еще выражать через объёмную плотность, да).
  Рассчитаем силу:
  \begin{gather*}
  \d \vec F = \d Q (\vec v \times \vec B) = (\underbrace{\rho \vec v}_{\vec j} \times \vec B) \d V = (\vec j \times \vec B) \d V
  \end{gather*}
  Пусть провод прямой, сечение равно $S$, длина равна $\d l$.
  Тогда
  \[ \d \vec F = (\vec j \times \vec B) \cdot S \cdot \d l = (\vec j \cdot S)\ \times (\vec B \cdot \d l) \]
  Так как направление течения тока перпендикулярно сечению, то:
  \[ \d \vec F = \vec I \times \vec B \d l \]
  \cautoimg{06/04}

\section{Порождение магнитного поля зарядами}
  Пусть есть два провода без магнитого поля, по ним течёт одинаковый ток в одинаковом направлении
  \cautoimg{06/05}
  Оказывается, что они будут вести себя так, как если бы они находились в магнитном поле.
  В данном случае провода будут притягиваться.

  \begin{theorem}[закон Био-Савара-Лапласа]
  Заряд $q$, движущийся со скоростью $\vec v$, создаёт вокруг себя магнитное поле:
  \[ \vec B(\vec r) = \frac 1 c \cdot q \cdot \frac{\vec v \times \vec r}{r^3}\]
  \cautoimg{06/06}
  Здесь $c$ "--- это некоторая константа (в СГС это электродинамическая постоянная, равная скорости света; в СИ вместо неё будет $4\pi$).
  На больших расстояниях магнитное поле одного заряда убывает, причём с такой же скоростью, как электрическое.
  Этакий аналог закона Кулона.
  \end{theorem}

  Давайте посчитаем, какое магнитное поле порождает текущий ток.
  Пусть есть кусочек провода с длиной и направлением $\Delta \vec r'$.
  По нему протекает со скоростью $\vec v(\vec r')$ заряд.
  Хотим определить магнитное поле в точке $\vec r$ (потом сложим по всем кусочкам).
  \cautoimg{06/07}
  \[ \d \vec B(\vec r) = \frac 1 c \d q \frac{\vec v(\vec r) \times (\vec r - \vec r')}{|\vec r - \vec r'|^3}\]
  Вспоминаем, что такое ток по кусочку:
  \[ \vec I = \rho \vec v \cdot S \]
  И какой у нас заряд внутри кусочка:
  \[ \d q = \rho \cdot S \cdot \d |l|\]
  Отсюда (так как считаем $\vec I \upuparrows \d \d l$):
  \begin{gather*}
    I \cdot \d \vec l = \d q \cdot \vec v \\
    \d \vec B(\vec r) = \frac I c \cdot \frac{\d \vec l \times (\vec r - \vec r')}{|\vec r - \vec r'|^3}
  \end{gather*}
  Ура, перешли от движеия каждого заряда к току по кусочку.
  Заметка: считаем $\vec I \upuparrows \vec \d l$, поэтому можно писать вектор только в одном из двух.
  Интегрируем по длине провода:
  \[
    \vec B = \frac I C \int \frac{\d \vec l \times (\vec r - \vec r')}{|\vec r - \vec r'|^3}
  \]

  \subsection{Пример: прямой провод}
  Пусть есть провод длины $2L$, по которому течёт ток $I$.
  Хотим найти магнитое поле в точке на серединном перпендикуляре к проводу на расстоянии $R$ от него.
  % картинка, слева направо течёт ток, точка внизу
  По формуле нужно разбить провод на много маленьких кусочков $\d \vec l$,
  вычислить расстояния от них до искомой точки и посчитать интеграл.
  % картинка, на которой видны направления, и угол \theta между куском и направлением на точку
  Направление для всех кусочков будет одно: в доску.
  Надо только посчитать модуль поля.
  \begin{gather*}
  | \vec r - \vec r' | = \frac R{\sin \theta} \\
  | \d \vec l \times (\vec r - \vec r') | = \d l \cdot \frac{R}{\sin \theta} \cdot \sin \theta = \d l \cdot R
  B = \frac I C \int \frac{\d l \cdot R}{\left(\frac R{\sin \theta}\right)^3}
    = \frac I {CR^2} \int \sin^3 \theta \d l
  \end{gather*}
  Хотим проинтегрировать, да не можем сразу "--- там разные переменные $\theta$ и $l$, надо установить зависимость между ними.
  
  Давайте считать, что $l$ меняется от $0$ до $2L$.
  % картинка для вычисления угла
  Отсюда наглядно видно, что
  \begin{gather*}
    \ctg \theta = \frac{L-l}{R} \\
    - \frac 1 {\sin^2\theta} \d \theta = -\frac{\d l}{R} \\
   \d l = \frac{R}{\sin^2 \theta} \d \theta \\
    B = \frac{I}{CR^2} \int \frac{R}{\sin^2 \theta} \d \theta \cdot \sin^3 \theta
      = \frac{I}{CR} \int \sin \theta \d \theta
  \end{gather*}
  Теперь разберёмся с границей интегрирования.
  Интегрируем от $\alpha$ до $-\alpha$
  % картинка, слева угол \alpha, справа - в другую сторону
  \[ B = \frac{I}{CR} (\cos \alpha - \cos(\pi-\alpha)) = \frac{2I}{CR} \cos \alpha\]
  Можно пойти дальше: выразить $\cos \alpha$ через $R$ и $L$.

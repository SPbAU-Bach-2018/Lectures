\documentclass[10pt,a4paper]{article}
\usepackage{cmap}
\usepackage[T2A]{fontenc}
\usepackage[utf8]{inputenc}
\usepackage[english,russian]{babel}
\usepackage{fancyhdr}
%\usepackage{minted}
\usepackage{hyperref}
\usepackage{indentfirst}
\usepackage{amssymb,amsmath,amsthm}
\usepackage{mathtools}
%\usepackage[top=0.5cm,bottom=0.5cm,left=1cm,right=1cm,marginparwidth=0.5cm,marginparsep=0cm,headsep=0cm,includehead,includefoot]{geometry}
\usepackage[top=0.5cm,bottom=0.5cm,left=0.5cm,right=0.5cm,includehead,includefoot]{geometry}
\usepackage[page]{totalcount}
\usepackage{color}
\usepackage{datetime}
\pagestyle{fancy}

\newcommand{\env}[2]{\begin{#1}#2\end{#1}}
\newcommand{\eps}{\varepsilon}
\newcommand{\R}{\mathbb{R}}
\newcommand{\Q}{\mathbb{Q}}
\newcommand{\TODO}{\textbf{\textcolor{red}{TODO}}}
\renewcommand{\d}{\mathrm{d}}
\DeclareMathOperator{\rot}{rot}
\DeclareMathOperator{\grad}{grad}

%\renewcommand{\section}[1]{%
%\noindent\makebox[\linewidth]{\color[rgb]{0.8,0.8,0.8}\rule{\paperwidth}{0.4pt}}%
%\refstepcounter{section}%
%\reversemarginpar\hspace{0.01mm}\marginpar{\textbf{\arabic{section}}}\reversemarginpar%
%}%

\renewcommand{\t}[1]{\ifmmode{\mathtt{#1}}\else{\texttt{#1}}\fi}
\renewcommand{\O}{\mathcal{O}}

\begin{document}
\lhead{Физика, II семестр}
\rhead{СПб АУ, бакалавриат}
\cfoot{\thepage~из~\totalpages}
\rfoot{\today~\currenttime}

Теория любит СГС, поэтому почти везде используется СГС.

\section{Электростатика}
\begin{enumerate}
\item Закон Кулона: $F = k\frac{q_1q_2}{r^2}$. В СГС $k=1$, в CИ $k=\frac{1}{4\pi\epsilon_0}$.
\item $\epsilon_0$ "--- электромагнитная постоянная, $\epsilon_0 \approx 8.8541 \cdot 10^{-12} \frac{Ф}{м}$ (Фарады на метр).
\item Поле от заряда: $|E| = \frac{q}{r^2}$.
\item Поток поля $\Phi$: интеграл по площади скалярного произведения поля на нормаль к поверхности.
\item Теорема Гаусса: если поверхность ограничивает суммарный заряд $q$, то поток через поверхность равен $4\pi q$ (СГС) или $\frac{q}{\epsilon_0}$ (СИ).
\item Для тонкой нитки электрическое поле на расстоянии $R$ равно $E=\frac{2\lambda}{r}$ (СГС)
\item Для плоскости: $E=2\pi \sigma$ ($\sigma$ "--- поверхностная плотность заряда) в СГС.
\item Внутри проводника зарядов нет, они на поверхности.
\item Потенциалы: $\phi_B - \phi_A \int_A^B \vec E \cdot \d \vec l$.
\item $\vec E = -\grad \phi$ (потенциал уменьшается в сторону силовых линий).
\item Потенциальная энергия системы двух зарядов: сколько надо, чтобы притащить их с бесконечности друг к другу, равна $\frac{q_1q_2}{r}$.
\item Потенциал проводника: $\phi = \phi_\infty - \phi_X = -\phi_X$, где $X$ "--- точка на поверхности
\item Ёмкость проводника: $C=\frac{Q}{\phi}$, $Q$ "--- заряд, $\phi$ "--- потенциал, меряется в Фарадах.
\item Ёмкость сферы: $R$
\item Ёмкость системы двух проводников по определению (заряды на них $\pm Q$, разность потенциалов $U$): $C=\frac{Q}{C}$.
\item Поле внутри плоского конденсатора: $4\pi \sigma$
\item Ёмкость плоского конденсатора: $C=\frac{S}{4\pi d}$, где $d$ "--- расстояние, сильно меньше площади
\item Параллельно подключили конденсаторы "--- ёмкость сложилась
\item
	Потенциальная энергия конденсатора: изначально не заряжен, по одному перетаскиваем заряды с одной обкладки на другую, сопротивляемся возрастающему полю.
	Получаем $W=\frac{Q^2}{2C}=\frac{CU^2}{2}$.
\item
	Если замкнуть два конденсатора <<плюс с минусом>>, то заряд перераспределиться, потенциальная энергия упадёт (уйдёт на нагрев провода)
\item
	Энергию конденсатор хранит в поле, говорим, что по определению плотность энергии поля есть $\mathcal{W}=\frac{E^2}{8\pi}$ (на единицу объёма).
\item
	Диэлектрик поляризуется в поле (заряды внутри распихиваются по его сторонам), внутри возникает электрическое поле, которое немного компенсирует внешнее.
\item
	$\epsilon$ "--- диэлектрическая проницаемость, безразмерная.
	\[ E_{суммарное с диэлектриком} = \frac{E_{без диэлектрика}}{\epsilon} \]
	Для фарфора "--- 6, для стекла "--- 10, для дерева "--- 4, для керосина "--- 2.
\item
	Добавили диэлектрик в плоский конденсатор, ёмкость увеличилась в $\epsilon$ раз.
\end{enumerate}

\section{Электродинамика и магнетизм}
\begin{enumerate}
\item Ток $I$ "--- это поток $\vec j$ через поверхность, где $\vec j$ "--- это, условно, скорость и направление движения тока
\item Сколько тока $I$ через поверхность утекло, на столько заряд внутри и уменьшился. Например, если течёт вдоль поверхности, то никуда ничего не исчезает.
\item Теорема Стокса (штука под вторым интегралом называется дивергенция):
\[ \oint_S \vec j \cdot \vec n \cdot \d S = \int_V \vec \nabla \cdot \vec j \cdot \d V \]
Поток через поверхность равен дивергенции по ограниченном поверхностью объёму.
\item Сила Лоренца от магнитного поля: $F = q(\vec v \times \vec B)$ (и в СИ, и в СГС).
\item Заряд $q$, движущийся со скорость $\vec v$ создаёт вокруг себя магнитное поле (считаем, что заряд в центре координат):
\[ \vec B(\vec r) = \frac{1}{c} \cdot q \cdot \frac{\vec v \times \vec r}{|r|^3} \]
Это в СГС, в СИ надо заменить коэффициент на $\frac{\mu_0}{4\pi}$.
\item Закон Ампера: взяли контур, циркуляция по нему пропорциональна потоку тока через поверхность (СГС):
\[ \oint_C \vec B \cdot \d l = \frac{4\pi}{c} I \]
\item Выводить так: посчитали магнитное поле для точки, лежащей на серединном перпендикуляре провода, сделали провод бесконечно длинный, взяли окружность вокруг провода
\item В центре бесконечного соленоида (цилиндр такой с намотанным проводом): $B = \frac{4\pi}{c} n I$, где $n$ "--- плотность витков на длину
\item Векторный потенциал: $\vec A(\vec r) = \frac{1}{c}\int \frac{\vec j(\vec r')}{|\vec r - \vec r'|} \d V$, причём $\vec B(\vec r) = \vec \nabla \times \vec A(\vec r) = \rot \vec A(\vec r)$.
Не пользовались.
\end{enumerate}

\section{Максвелл}
\begin{enumerate}
\item Теорема Стокса, если есть поле $\vec \Omega$:
\[ \int \vec \Omega \cdot \d \vec S = \int \vec \nabla \cdot \vec \Omega \d V \]
Поток через поверхность равен дивергенции по объёму
\item Тогда можно перевести Гаусса по этой теореме (не забыть заменить заряд на интеграл по плотности): $4\pi \rho = \vec\nabla \cdot \vec \Omega$, тут $\Omega$ "--- какое-то поле, которое вылезло из плотности, кажется.
\item Другая Стокса: циркуляция поля равна потоку ротора через поверхность
\item Отсюда выводится, что $\vec \nabla \times \vec E = \rot \vec E = 0$
\item Магнитное поле зависит от системы отсчёта, увы.
\item Взяли заряд с постоянной скоростью, поместили в центр системы отсчёта, взяли пробный заряд, приравняли действующие в двух системах отсчёта силы.
Потом еще применили $a \times (b \times c) = b(ac)-c(ab)$.
Получили
\[ \vec \nabla \times \vec E = -\frac1c \frac{\partial \vec B}{\partial t} \]
Это еще одно уравнение, закон Фарадея.
\item Ток смещения: умножили наблу на простой закон Ампера (набла слева), получили, что плотность тока не меняется, неудобно.
Поправили:
\[ \rot \vec B = \frac{4\pi}{c} \vec j + \frac{1}{c} \frac{\partial \vec E}{\partial t} \]
Второе слагаемое "--- и есть ток смещения.
\item Магнитное поле внутри конденсатора с изменяющимся зарядом есть, так как электрическое поле меняется, оно закручивается вокруг <<точки втыкания>> проводов в пластины.
\end{enumerate}

\section{Катушки}
\begin{enumerate}
\item Взяли две катушки, в одной пустили ток, возникло магнитное поле, во второй возникла ЭДС:
\[ \mathcal{E} = -\frac1c \frac{\partial \Phi_2(t)}{\partial t} \]
Переписали, заменили производную на производную тока в первой по времени, оставшийся коэффициент "--- коэффициент взаимной индукции.
\item Самоиндукция: сама на себя подействовала, $L$ "--- коэффициент (измеряется в Генри, Гн)
\[ \mathcal{E} = -L \frac{\partial I(t)}{\partial t} \]
\item Можно провести аналогию с механикой: сила "--- ЭДС, скорость "--- ток, энергия катушки "--- $\frac{LI^2}{2}$, энергия кинетическая "--- $\frac{mv^2}{2}$.
\item Конденсатор с катушкой будут колебаться:
\begin{gather*}
\omega ^2 = \frac{1}{LC} \\
I(t) = -CU_0 \omega \sin(\omega t)
\end{gather*}
\item Если заряжать конденсатор переменным током, то внутри у поля будет какое-то безумие с функцией Бесселя, в зависимости от расстояния до центра конденсатора:
\[ E(r) = E_0\sin(\omega t) J_0(\omega r) \]
\[
J_0(x) = 1 - \frac{1}{(1!)^2} \left(\frac x 2\right) ^ 2 + \frac{1}{(2!)^2} \left(\frac x 2\right) ^ 4 - \dots
\]
\end{enumerate}

\end{document}

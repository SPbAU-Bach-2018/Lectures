\chapter{Метод Гаусса, Гаусс по не простому модулю, Базисы}
\setauthor{Игорь Лабутин} 

\section{Метод Гаусса}

\subsection{Основное}
У матрицы можно менять строки и столбцы местами, складывать строки и столбцы. При этом определитель понятно как меняется.\\
Чтобы привести матрицу к треугольному виду, будем по очереди занулять элементы $i$-ого столбца под главной диагональю, вычитая $i$-ую строку, домноженную на нужный коэффициент. Чтобы привести к трапециевидной, будем вычитать элементы над и под главной диагональю.\\
К треугольному виду приводим за $\sum_{k=1}^{n}{k^2} = \frac{n^3}{3}$ умножений, к трапециевидному за $\sum_{k=1}^{n}{nk} = \frac{n^3}{2}$.
\subsection{Определитель}
Определитель треугольной матрицы = произведение элементов на диагонали.
\subsection{Система}
Если привести матрицу к трапециевидной, решением будет для $i\le n$ $x_i=\frac{b_i-\sum_{j>i}{x_ja_{ij}}}{a_{ii}}$, если к диагональной, то $x_i=\frac{b_i-\sum_{j\ge n}{x_ja_{ij}}}{a_{ii}}$, $x_j$ при $j>n$ -- свободные переменные. Если хотим менять столбцы (а иначе у нас может получиться столбец нулей, портящий трапециевидность/диагональность), то надо поддерживать перестановку переменных.\\
У нас могут быть свободные переменные, все решения образуют линейное пространство, базис которого мы нашли.\\
На вещественных числах у Гаусса плохая точность на некоторых тестах, например матрица Гильберта: $a_{ij}=\frac{1}{i+j}$. Чтобы улучшить точность можно ставить максимальный по модулю элемент из правой нижней подматрицы на место того, на который собираемся делить. Еще можно использовать $long\ double$ или $BigDecimal$.
\section{Гаусс по не простому модулю}
Все ранее написанное работает в поле (где можно делить). Теперь посмотрим, что произойдет в кольце по не простому модулю.\\
Вместо деления будем с помощью вычитания строки домноженной на коэффициент (просто домножать строку на константу нельзя) заменять строки, начинающиеся на $a$ и $b$ на эквивалентные, начинающиеся на $gcd(a, b)$ и $0$. Это можно делать за $O(n^3T(gcd))$, вычитая и домножая строки по ходу алгоритма Евклида ($a\%b=a-b\cdot \lfloor\frac{a}{b}\rfloor$) или за $O(n^2(n+T(gcd)))$, найдя расширенным алгоритмом Евклида линейное представление $gcd$ и нуля (новые строки получаются как суммы старых с найденными коэффициентами). Так можно найти определитель.\\ Привести матрицу к диагональному виду не получится (например на матрице $((1, 1), (0, 2)))$, т.к. мы не можем домножать строку на число.\\
Заметим, что если применить этот метод к кольцу $Z$, то числа могут расти экспоненциально.
\section{Базисы}
\subsection{Основное}
Можно найти базис линейного пространства (подпространства $R^n$) Гауссом, приведя матрицу из векторов-строк пространства к трапециевидной. Первые $rank A$ строк полученной матрицы будут базисом ($dim U = rank A$).\\
\begin{Def}
Базис ортогональный, если $\forall i \ne j$ $<v_iv_j> = 0$ а $<v_iv_i> \ne 0$.
\end{Def}
\begin{Def}
Базис ортонормированный, если он ортогональный и $\forall i <v_iv_i>=1$.
\end{Def}
В ортонормированном базисе любой вектор $z$ можно представить в виде суммы $\alpha_iv_i$, где $\alpha_i = <z, v_i>$.\\
\subsection{Ортогонализация Грама-Шмидта}
Хотим за $O(n^3)$ получить из произвольного базиса ортонормированный. Рассматриваем вектора по очереди, делаем каждый следующий ортогональным предыдущему и делим на длину (нормируем). $u_k = norm(v_k-\sum_{i=1}^{k-1}{<u_iv_k> u_i})$
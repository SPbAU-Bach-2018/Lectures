\section{Первообразные корни}
\setauthor{Егор Суворов}

Рассмотрим систему вычетов по простому модулю $p$.
Мультипликативная группа по модулю $p$ состоит из ровно $p-1$ элемента "--- это просто все не-нули.
Однако с точки зрения умножения её так описывать не слишком удобно.

\begin{Def}
$g$ называется \textit{первообразным корнем} по модулю $p$, если любой ненулевой вычет $x$ можно представить в виде $x=g^k$ для некоторого $k$.
\end{Def}

\begin{theorem}\label{PrimitiveRootExists}
По простому модулю $p$ существует хотя бы один первообразный корень.
\end{theorem}
\begin{Rem}
На самом деле первообразные корни также существуют и по модулям 2, 4, $p^k$ и $2p^k$.
По модулю 2 его найти просто "--- это единица.
\end{Rem}

\begin{theorem}\label{PrimitiveRootIff}
$g$ "--- первообразный корень тогда и только тогда, когда среди чисел $g^1$, $g^2$, \dots, $g^{p-1}$ единица встречается в первый раз на месте $g^{p-1}$.
Заметим, что $g^{p-1}=1$ всегда по малой теореме Ферма (или теореме Эйлера).
\end{theorem}
\begin{proof}
\begin{description}
\item[$\Ra$]:
	Если для некоторого $k$ имеем $g^k=1$, то у нас $g^x$ принимает всего $k$ различных значений (так как после $k$ всё зацикливается).
	Значит, $k \ge p - 1$ по определению первообразного корня.
	С другой стороны, $k \le p - 1$, так как $g^{p-1}=1$.
\item[$\La$]:
	Пусть $g$ не является первообразным корнем, тогда какое-то число не встретилось в последовательности степеней, значит, какое-то встретилось дважды,
	то есть:
	\begin{gather*}
	\exists 1 \le k_1 < k_2 \le p - 1\colon g^{k_1} = g^{k_2} \\
	g^{k_1} = g^{k_2} \\
	g^{k_1-k_1}= g^{k_2-k_1} \\
	g^0 = g^{k_2-k_1} \\
	1 = g^{k_2-k_1} \\
	\end{gather*}
	То есть получаем, что единица должна была встретиться уже на месте $k_2-k_1$, что строго меньше $p-1$, противоречие.
\end{description}
\end{proof}

\begin{theorem}
В мультипликативной группе по модулю $p$ есть ровно $\phi(p-1)$ первообразных корней.
\end{theorem}
\begin{proof}
По теореме \ref{PrimitiveRootExists} хотя бы один корень точно есть, назовём его $\omega$.
Давайте возьмём какое-нибудь число из мультипликативной группы, скажем, $\omega^k$.
По теореме \ref{PrimitiveRootIff} оно будет являться первообразным корнем тогда и только тогда, когда в последовательности $(\omega^k)^1$, $(\omega^k)^2)$, $\dots$ единица в первый раз встречается на месте $p-1$:
\begin{gather*}
(\omega^k)^a = 1 \mod p \\
\omega^{ka} = 1 \mod p \\
\omega^{ka} = \omega^0 \mod p \\
ka = 0 \mod (p-1)
\end{gather*}
Минимальное ненулевое $a$, являющееся решением, равно $\frac{p-1}{\gcd(k, p-1)}$, что равно $p-1$ только в случае взаимной простоты $k$ и $p-1$.
Таким образом, существует всего $\phi(p-1)$ различных $k$, которые являются первообразными корнями.
Обращаю внимание, что тут нам совершенно неважно, какой именно первообразный корень взять в начале "--- мы пользуемся тем, что существует какой-нибудь и что остальные через него выражаются.
\end{proof}

Давайте научимся искать первообразный корень.
\subsection{Поиск <<в лоб>>}
	Например, можно искать его <<в лоб>>: перебираем вычеты от меньших к большим и для каждого проверяем условие теоремы \ref{PrimitiveRootIff}.
	Очевидно, что это будет работать за $O(pg)$, где $g$ "--- минимальный первообразный корень, но на самом деле это даже $O(p)$ (без доказательства).

\subsection{Быстрая проверка}
	Чтобы ускорить, давайте для начала научимся быстрее проверять первообразность.
	В каких вообще случаях может так случиться, что $g^a=1$ (и $a$ минимально), при условии, что $g^{p-1}=1$?
	Вспомним доказательство теоремы \ref{PrimitiveRootIff}: такое $a$ имеет вид $\frac{p-1}{d}$, где $d$ "--- некоторый делитель $p-1$.
	Но тогда $g^{ak}=1$ для любого $k$.
	То есть, вообще говоря, достаточно будет проверить, что $g^a \neq 1$ для всех $a$ вида $\frac{p - 1}{p_i}$, где $p_i$ "--- простые делители $p-1$.
	Если нам известна факторизация $p-1$ и в ней $m$ различных простых делителей, то быстрым возведением в степень для каждого мы можем проверить, что $g^a \neq 1$,
	получим асимптотику $O(m\log p)$.
	Очевидно, что $m = O(\log p)$ (так как каждый простой делитель хотя бы 2) и получается оценка $O(\log^2 p)$.
	Хотя на самом деле, конечно, различных простых сильно меньше логарифма (в среднем их $O(\log \log n)$, теорема Харди-Рамануджана).
\begin{cppcode}
int powmod(int a, int b, int mod);

int p;
vector<int> p1_divisors; // простые делители p-1

bool check(int g) {
    for (int x : p1_divisors) {
        if (powmod(g, (p - 1) / x, p) == 1) {
            return false;
        }
    }
    return true;
}
\end{cppcode}

	Теперь мы можем находить ответ за $O(g \log^2 p)$.
	Факт: первообразный корень не больше, чем $O(\log p)$ (без доказательства).
	То есть уже умеем за $O(log^3 p)$:
\begin{cppcode}
int g = 2;
while (!check(g)) g++;
\end{cppcode}

\subsection{Рандомизированный алгоритм}
	Так как мы знаем, что корней всего $\phi(p-1)$, то тыкая в случайное число, мы попадаем с вероятностью $\frac{\phi(p-1)}{p-1}$.
	Значит, матожидание числа шагов "--- $\frac{p - 1}{\phi(p-1)}$.
	Если $p-1=p_1^{k_1}p_2^{k_2}\dots$, то $\frac{p-1}{\phi(p-1)} = \frac{p_1p_2}{(p_1-1)(p_2-1)\dots}$, что довольно немного.
	Непример, если у $p-1$ мало простых делителей (например, такое бывает при выборе модуля для Фурье вида $p=2^k \cdot l$), то это быстро.
	На каждом шаге проверка за $O(\log^2 p)$, вообще красота.
\begin{cppcode}
for (;;) {
    g = randint(1, p - 1);
    if (check(g)) break;
}
\end{cppcode}

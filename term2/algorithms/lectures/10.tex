\chapter{Паросочетания}
\setauthor{Юрий Кравченко}

Максимальная клика, максимальное независимое множество и минимальное покрывающее множество - одна и ты же задача с точностью замены графа на обратный.

В произвольном графе проверить, что существует клика заданного размера - $NP$-полная задача.

\begin{Def}
    Паросочетание "--- набор попарно несмежных ребер.

    Максимальное паросочетане "--- набор наибольшего размера.
\end{Def}

\begin{theorem}
Если существует дополняющая чередующаяся цепь, то можно увеличить размер паросочетания. Обратное тоже верно
\end{theorem} 
\begin{proof}
$M$ "--- текущее паросочетание

$P$ "--- какое-то паросочетание

Пусть $\exists |P| > |M|$ рассмотрим $P \Delta S$. Граф разбился на 4 типа объектов.
КАРТИНКА
т.к.  $|P| > |M| \Ra \exists$ объект типа 2, потому что только в нём рёбер из $P$ больше, чем рёбер из $M$. Его-то мы и возьмём в качестве дополняющей цепи.
\end{proof}

\textbf{Алгоритм:}
\begin{enumerate}
\item найти цепь
\item применить её
\end{enumerate}

Код для двудольного графа:
\begin{cppcode}
bool dfs(v) {
	u[v] = 1;
	for (int x: G[v]) {
		if (pair[x] == -1 || (u[pair[x]] == 0 && dfs(pair[x])) {
			pair[x] = v;
			return true;
		}
	}
	return false;
}
\end{cppcode}

Тупой алгоритм
\begin{cppcode}
while(good) {
	good = false;
	u = {0};
	for v {
		if (u[v] == 0 && is[v] && dfs(v)) {
			good = true;
			break;
		}
	}
}
\end{cppcode}

Алгоритм Куна $O(VE)$
\begin{cppcode}
for v {
	u = 0;
	dfs(v);
}
\end{cppcode}
\begin{theorem}
Кун корректен
\end{theorem}
\begin{proof}
Пусть не существовало дополняющего пути из вершины $v$. Мы применили некоторый путь и путь из вершины $v$ появился. Посмотрим внимательно на картинку и поймём, что он уже был до этого.
КАРТИНКА
\end{proof}
Оптимальный Кун $O(|M|E)$
\begin{cppcode}
u = {0};
for v {
	if dfs(v) {
		u = {0};
	}
}
\end{cppcode}
Жадная инициализация
\begin{cppcode}
random_shuffle(Edges);
for (e: Edges) {
	if (!w[a[e]] && !w[b[e]]) {
		w[a[e]] = w[b[e]] = true;
		A += {e};
	}
}
\end{cppcode}
$|A| \ge \frac{Max}{2}$ потому что каждое ребро могло помешать не больше, чем двум рёбрам из ответа. Теперь оптимальный Кун работает хотя бы в 2 раза быстрее.

Пусть есть Max Matching, получим Cover и IS(независимое множество) за O(E)
\begin{theorem}{теорема Кёнига}
$max |M| = min |C|$

В двудольном графе максимальное паросочетание равно числу вершин в минимальном вершинном покрытии.
\end{theorem}
\begin{proof}
$|C| \ge |M|$ (Пусть вершинное покрытие меньше, тогда какая-та вершина "покрывает" хотя бы два ребра. Противоречие).

Предъявим $|M| = |C|$.

Возьмем вершины, которые не лежат в паросочетание в левой доле.

Запустим из них Куна. Мы дошли из них до какого-то множества вершин. 

$A^+$ "--- вершины из первой доли, до которых мы дошли

$B^+$ "--- вершины из второй доли, до которых дошел dfs

$A^-$ "--- вершины из первой доли, до которых мы не дошли.
 
$B^-$ "--- вершины из второй доли, до которых мы не дошли. 

Cover = $A^- \cup B^+$

IS = $A^+ \cup B^-$
\item из $A^+$ нет рёбер в $B^-$
\item $|A^- \cup B^+| = |M|$ потому что $A^-$ и $B^+$-концы рёбер из $M$
\end{proof}
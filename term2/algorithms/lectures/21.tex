\setauthor{Дима Бойкий}
\section{Хеши}
\subsection{Строка Туэ-Морса}

\textbf{Def}: s[i] = bitcount(i) \% 2 \\
\textbf{Def2}: $s_{k+1} = s_k$ + not $s_k$ \\ \\
Обозначим <P,M> : $\sum s_i \cdot p^i $ $mod $ $M$ \\
Пример использования: антихеш-тест для mod = $2^{m}$ (для любого P) \\\\
$H_1$ = $Hash(s[0,2^k])$, $H_2$ = $Hash(s[2^k,2^{k+1}])$ \\
$H_1$ = $H_2$ для $k \ge$ ? \\\\
$H_1 - H_2 = 0 \mod 2^m$ \\
$s[2^k,2^{k+1}]$ = not $s[0,2^k]$ $\Rightarrow$ $\sum p^i \cdot (\pm1) = 0 \mod$ $2^{m}$  \\
$\sum p^i \cdot (\pm1) = -p^0 + p^1 + p^2 - p^3 + ... = -(1-p)(1-p^2)(1-p^4)...(1 - p^{2^{k-1}})$ \\
$1-p^4 = (1-p^2)(1+p^2) = (1+p^2)(1+p)(1-p)$, каждая из скобок четная $\Rightarrow $ получаем $1+2+3+..k = $ $k(k+1)\over{2}$ $ \ge$ $k^2\over{2}$ $\Rightarrow $ при $Len = 2^k$ хеши равны по модулю $2^{k^2\over{2}}$\\\\
$m = 2^{k^2/2} \Rightarrow k = \sqrt{2m}$ \\
$m = 64 \Rightarrow k = \sqrt{128} \approx 12$, $2^k = 4096$\\
Нужна строка длины $2^{k+1}$, значит 8192 хватит. \\
\textbf{Замечание}: $p = 2k$: $p^m \equiv 0 \mod 2^m$ \\

\subsection{Алгоритм построения коллизии для <P,M> хеша}

$Hash(s_1,..,s_k) = Hash(t_1,..,t_k)$, $\sum = \{a,b\} $ \\
$\sum p^i \cdot (-1,0,1) = 0 \mod M$ \\
$A_0 = \{p^0, p^1,...,p^{k-1}\}$  //все по модулю M \\
Если среди элементов $A_i$ есть 0, то все хорошо. \\
Получение $A_{i+1}: $ \\
1. Sort($A_i$) \\
2. $A_{i+1} = \{A_i[1]-A_i[0], A_i[3] - A_i[2],..\}$
  
$|A_{i+1}| =$ $|A_i|\over{2}$ \\
Диапазон $A_0$: $0.. M-1$ \\\\
$A_1$: $k$ случайных чисел, $E(x_i)=$ $M \cdot i\over{k+1}$, диапазон $A_1 \sim$  $M\over{k+1}$  \\ 
$A_2$: $|A_1| =$ $k\over{2}$ $\Rightarrow$ диапазон $A_2 \sim$ $M\over{(k+1)(k/2 + 1)}$ \\
Для $A_i$ диапазон делится на $k\over{2^{i-1}}$ $+ 1$ \\\\
$M \le$ $k\over{1}$ $\cdot$ $k\over{2}$ $\cdot$ $k\over{4}$ ... $\approx$ ${k^{log k}}\over{2^{({log^2 k})\over{2}}} $ = $2^{({log^2 k})\over{2}}$ $\Rightarrow M =$ $2^{({log^2 k})\over{2}}$, $log M =$ $ ({log^2 k})\over{2}$\\
$log k = \sqrt{2logM}$\\
$k = 2^{\sqrt{2logM}}$\\
$M = 2^{64}$: $k = 2^{\sqrt{2 \cdot 64}} \approx 2^{12} = 4096$ \\
\textbf{Пример}: $1 - p + p^2 = 0$, $1 + p^2 = p$: $bab = aba$ // $b$ обозначает 1, $a$ -- 0 \\

\subsection{Леммы
про вероятность коллизии при выборе многочлена, или точки, или обоих случайными} 	

P - точка, M - модуль\\
S - многочлен(строка) с коэффициентами 0..M-1\\
k - степень многочлена(длина строки) \\\\
\textbf{Lm1}: <P,M> - fixed, s - random, тогда Pr[s(P) = 0] = ${1}\over{M}$, Hash(s) = 0\\\\
s(x) = (x-p)t(x) + $a_0 \leftarrow$ равномерно распределены от 0 до M-1, значит верно\\\\ 
\textbf{Lm2}: <s,M> - fixed, P - random, тогда Pr[s(P) = 0] $\le$ ${k}\over{M}$ \\\\
Pr[s(P) = 0] = ${ root number \over{M}}$, M - простое $\Rightarrow$ корней $\le k$\\\\
\textbf{Замечание}: M = $2^{32}$, $x^{32} = 0 \mod M$ имеет $2^{31}$ корней \\\\
\textbf{Lm3}: M - fixed, <s,P> - random. Pr[s(P) = 0] = ${1}\over{M}$ $\Leftrightarrow$ у random s(x) $\sim$ 1 корень \\\\
$\sum_{P=0}^{M-1}$ $1\over{M}$ $\cdot$ $1\over{M}$ $=$ $1\over{M}$ через Lm1\\\\
\textbf{Выводы}:
\begin{enumerate}
	\item[1.] M -- простое  $\sim$ $10^9+7$
	\item[2.] P$_1$ - random
	\item[3.] P$_2$ - random
\end{enumerate}   

\subsection{Хеш-таблица}
m списков, всего n чисел: Len $\sim$ $n\over{m}$\\
\textbf{Lm1}: E(Len$_i$ = 1) (бессмысленна, т.к все равно могут быть все элементы в одном списке) \\
\textbf{Lm2}: E(max Len$_i$) = $O(log n)$\\\\
Hash = Random\\
Pr\{len$_i$ = k\} = $\frac{C^n_k \cdot (n-1)^{n-k}}{n^n} \le \frac{n^k}{k!}\cdot\frac{n^{n-k}}{n^n} = \frac{1}{k!}$ \\
\textbf{Неправильное док-во}: E(Len$_i$ $\ge$ k) $\le$ n($1\over{k!}$+$1\over{(k+1)!}$+...) $\le$ $2n\over{k!}$ \\
E(Len$_i \ge k$) = 1: 2n = k!\\
2n = ($k\over{e}$) $^k$ $\cdot \frac{1}{\sqrt{2 \pi k}}$, $k^k = 2^{klogk}$, 2n = $2^{klogk}$ \\
$\frac{log(2n)}{loglog(2n)} \le k \le log(2n)$ \\
Получаем 1 + o(1/n)\\\\

\subsection{Двойное хеширование}
$i_1 = Hash_1(x)$, $i_2 = Hash_2(x)$\\\\
\textbf{Add(x)}\\
if (|List[i$_1$| $\le$ |List[i$_2$|) \\
List[i$_1$] $\leftarrow$ x 	\\
else\\
List[i$_2$] $\leftarrow$ x \\\\
\textbf{Lm3}: E(max Len$_i$) = $\Theta(log log n)$ 	// без док-ва \\\\
\textbf{Двойное хеширование для открытой адресации}\\\\
i = h$_1$(x)\\\\
\textbf{while}\\
i++  // в обычном \\\\
\textbf{while}\\
i = (i + h$_2$(x)) $\mod n$ // в двойном\\\\
\textbf{Совершенное хеширование} \\
Даны $x_1,..,x_n$. Найти h: x$_i\rightarrow [0..m)$, что  $\forall$ i $\neq$ j: h($x_i$) $\neq$ h($x_j$)\\\\
\textbf{Алго1}: Time = O(n), m = n$^2$\\
Pr\{коллизии\} $\le$ $\frac{\frac{n(n-1)}{2}}{n^2}$ $\le$ $1\over{2}$\\\\
\textbf{while} 1\\
h = hashFunc\\
\textbf{if} (good(h))\\
break;\\\\
\textbf{Алго2}: Time  = O(n), m $\le$ 4n\\\\
2-Level\\
1. Random n = m$_1$\\
n$_i$ - количество одинаковых хешей для разных элементов\\
Тогда E\{$\sum n_i^2$\} = E\{кол-во коллизий\} = $\sum(h[i] = h[j])$ = n + n(n-1)$\cdot \frac{1}{n}$ $\le$ 2n// n при i = j, n(n-1) при i$\neq$j \\\\
\textbf{Lm Маркова}: E(X) $\le$ A $\Rightarrow$ Pr[A $\ge$ 2X] $\le$ $1\over{2}$\\\\
$\Rightarrow$ Pr[$\sum n_i^2 \ge 4n$] $\le$ $1\over{2}$ $\Rightarrow$ Pr[$\sum n_i^2 < 4n$] $\ge$ $1\over{2}$\\\\
\textbf{Фильтр Блюма}\\
add(x) $x_1,..,x_n$
isAdded(x) m бит памяти \\\\
\textbf{Алгоритм}: $h_1,..,h_k$; $h_i:$ $x_j \rightarrow cell_{ij}$\\\\
init(): обнуление. bit $\leftarrow$ 0\\\\
add($x_j$)\\
 |\textbf{FOR} i = 1..k\\
 |   bit[cell$_{ij}$] = 1\\\\
isAdded($x_j$)\\
 |\textbf{FOR} i = 1..k\\
 |\textbf{if} bit[cell$_{ij}$] = 0:\\
 |	 	 \textbf{return} 0;\\
 |\textbf{return} 1;\\\\
Получился алгоритм с односторонней ошибкой.\\
Pr[Error] = Pr[isAdded(x$_i$) = 1, но x$_i$ нет] //! с оценкой этого на лекции были проблемы 
 
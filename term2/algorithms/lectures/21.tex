\setauthor{Дима Бойкий}
\section{Хеши}
\subsection{Строка Туэ-Морса}

\begin{Def} 
   Два определения строки Туэ-Морса:
   \begin{enumerate}
       \item s[i] = bitcount(i) \% 2
       \item $s_{k+1} = s_k$ + not $s_k$
   \end{enumerate} 
\end{Def}


Обозначим <P,M> : $\sum s_i \cdot p^i $ $mod $ $M$ 

Строки Туэ-Морса можно использовать для антихеш-тест для  $M = 2^{m}$ (для любого P)


$H_1$ = $Hash(s[0,2^k])$, $H_2$ = $Hash(s[2^k,2^{k+1}])$ 

$H_1$ = $H_2$ для $k \ge$ ? 

Проверим для какой длины случится коллизия.

$$H_1 - H_2 = 0 \mod 2^m$$
$$s[2^k,2^{k+1}] = \not s[0,2^k] \Rightarrow \sum p^i \cdot (\pm1) = 0 \mod 2^{m}$$  
$$\sum p^i \cdot (\pm1) = -p^0 + p^1 + p^2 - p^3 + ... = -(1-p)(1-p^2)(1-p^4)...(1 - p^{2^{k-1}})$$
$1-p^4 = (1-p^2)(1+p^2) = (1+p^2)(1+p)(1-p)$, каждая из скобок четная $\Rightarrow $ получаем $1+2+3+..k = $ $k(k+1)\over{2}$ $ \ge$ $k^2\over{2}$ $\Rightarrow $ при $Len = 2^k$ хеши равны по модулю $2^{k^2\over{2}}$\\\\
$m = 2^{k^2/2} \Rightarrow k = \sqrt{2m}$ \\
$m = 64 \Rightarrow k = \sqrt{128} \approx 12$, $2^k = 4096$\\
Нужна строка длины $2^{k+1}$, значит 8192 хватит. \\
\begin{Rem}
    $p = 2k$: $p^m \equiv 0 \mod 2^m$ 
\end{Rem}

\subsection{Алгоритм построения коллизии для <P,M> хеша}
Хотим найти стороки s и t, такие что $Hash(s_1, \cdots ,s_k) = Hash(t_1,..,t_k)$
 
Алфавит $\Sigma = \{a,b\}$ 

Если мы сможем подобрать такие коэффициенты(не все 0), то сможем найти исходные строчки(как, покажем потом):
$$\sum p^i \cdot (-1,0,1) = 0 \mod M$$

$A_0 = \{p^0, p^1, \cdot ,p^{k-1}\}$  //все по модулю M 

Если среди элементов $A_i$ есть 0, то все хорошо. )

Получение $A_{i+1}: $ 
\begin{enumerate} 
 \item Sort($A_i$) 
 \item $A_{i+1} = \{A_i[1]-A_i[0], A_i[3] - A_i[2], \cdots \}$
\end{enumerate}
  
Размер множетсва уменьшился в два раза $|A_{i+1}| = \frac{|A_i|}{2}$

Диапазон:
\begin{description}{
    \item [$A_0$:] $0 \cdots M-1$ 
    \item [$A_1$:] $k$ случайных чисел, $E(x_i)=$ $M \cdot i\over{k+1}$, диапазон $A_1 \sim$  $M\over{k+1}$ 
    \item [$A_2$:] $|A_1| =$ $k\over{2}$ $\Rightarrow$ диапазон $A_2 \sim$ $M\over{(k+1)(k/2 + 1)}$ 
    \item [$A_i$:] диапазон делится на $k\over{2^{i-1}}$ $+ 1$ 
}
\end{description}

$$M \le \frac{k}{1} \cdot \frac{k}{2} \cdot \frac{k}{4} \cdots \approx \frac{k^{log k}}{2^{\frac{\log^2 k}{2}}} =
 2^{\frac{log^2 k}{2}} \Ra M = 2^{({log^2 k})\over{2}}, \log M = \frac{{\log^2 k}}{2}$$
$$\log k = \sqrt{2 \log M}$$
$$k = 2^{\sqrt{2\log M}}$$
$M = 2^{64}$: $k = 2^{\sqrt{2 \cdot 64}} \approx 2^{12} = 4096$ 
\begin{exmp}
Элементы $A_i$ можно расписать как многочлен.

$1 - p + p^2 = 0, 1 + p^2 = p\colon bab = aba$ ($b$ обозначает 1, $a$ "--- 0)
\end{exmp}
\subsection{Леммы
про вероятность коллизии при выборе многочлена, или точки, или обоих случайными} 	

P - точка, M - модуль

S - многочлен(строка) с коэффициентами 0..M-1

k - степень многочлена(длина строки) 
\begin{lemma}{}
 <P,M> - fixed, s - random, тогда Pr[s(P) = 0] = ${1}\over{M}$, Hash(s) = 0

s(x) = (x-p)t(x) + $a_0 \leftarrow$ равномерно распределены от 0 до M-1, значит верно
\end{lemma}

\begin{lemma}{} 
<s,M> - fixed, P - random, тогда Pr[s(P) = 0] $\le$ ${k}\over{M}$ 

Pr[s(P) = 0] = ${ root number \over{M}}$, M - простое $\Rightarrow$ корней $\le k$
\end{lemma}

\begin{Rem}
 M = $2^{32}$, $x^{32} = 0 \mod M$ имеет $2^{31}$ корней 
\end{Rem}

\begin{lemma}{}
 M - fixed, <s,P> - random. Pr[s(P) = 0] = ${1}\over{M}$ $\Leftrightarrow$ у random s(x) $\sim$ 1 корень

$\sum_{P=0}^{M-1}$ $\frac{1}{M}$ $\cdot$ $1\over{M}$ $=$ $1\over{M}$ через Lm1
\end{lemma}

\textbf{Выводы}:
\begin{enumerate}
	\item M -- простое  $\sim$ $10^9+7$
	\item P$_1$ - random
	\item P$_2$ - random
\end{enumerate}   

\subsection{Хеш-таблица}
m списков, всего n чисел: длина одного списка $\sim \frac{n}{m}$

Пусть $m = n$

\begin{lemma}{}
E($Len_i$ = 1) (бессмысленна, т.к все равно могут быть все элементы в одном списке) 
\end{lemma}{}

\begin{lemma}{}
 E(max $Len_i$) = $O(\log n)$
\end{lemma}

Hash = Random

$$Pr\{len_i = k\} = \frac{C^n_k \cdot (n-1)^{n-k}}{n^n} \le \frac{n^k}{k!}\cdot\frac{n^{n-k}}{n^n} = \frac{1}{k!}$$

\textbf{Неправильное док-во}: 
$$E(Len_i \ge k) \le n(\frac{1}{k!}+\frac{1}{(k+1)!}+ \cdots) \le \frac{2n}{k!}$$
$$E(Len_i \ge k) = 1\colon 2n = k!$$
$$2n = (\frac{k}{e})^k \cdot \frac{1}{\sqrt{2 \pi k}}, k^k = 2^{k\log k}, 2n = 2^{k\log k}$$
$$\frac{\log(2n)}{\log\log(2n)} \le k \le \log(2n)$$
Получаем 1 + o(1/n)

\subsection{Двойное хеширование}
$$i_1 = Hash_1(x), i_2 = Hash_2(x)$$
\begin{cppcode}
void Add(x):
    if (|List[i[i]| <= |List[i[2]|) {
        List[i[1]].pb(x)
    } else {
        List[i[2]].pb(x)
    }
\end{cppcode}

\begin{lemma}{}
 $$E(max Len_i) = \Theta(\log \log n)$$
 
 Без доказательства
\end{lemma}


\subsection{Двойное хеширование для открытой адресации}

\begin{cppcode}
i = h1(x)
while(...) i++  // в обычном 

while(...) i = (i + h2(x)) mod n // в двойном
\end{cppcode}

\subsection{Совершенное хеширование} 

Даны $x_1,..,x_n$. Найти $h\colon x_i\rightarrow [0..m)$ такое, что  $\forall i \neq j\colon h(x_i) \neq h(x_j)$
\begin{description}
\item[Алгоритм 1]: Time = O(n), $m = n^2$
Pr\{коллизии\} $\le \frac{\frac{n(n-1)}{2}}{n^2} \le \frac{1}{2}$

\begin{cppcode}
while(1) {
    h = hashFunc
    if (good(h)) {
        break;
    }
}
\end{cppcode}

\item[Алгоритм 2]: 

Space = O(n)

Time = O(n)

\brgin{enumerate}
\item Распределили n по случайным образом по n ячейкам . 
\item  Пусть в i-й ячейке a[i] элементов, используем первый алгоритм, 
получаем a[i]^2 дополнительных ячеек. 
\end{enumerate}

$n_i$ "--- количство элементов в i-ом списке.

$$E[\sum n_i^2] = E[\text{количество коллизий}] = \sum(h[i] = h[j]) = n + \frac{n(n - 1)}{n} \le 2n$$

\begin{lemma}{\bf Маркова}
$$E[X] \le A \Ra Pr[A \ge 2X] \le \frac{1}{2}$$
\end{lemma}

$$\Ra P[\sum n_i^2 \ge 4n] \le \frac{1}{2}$$
\end{description} 


\subsection{Фильтр Блюмa}

Занимает m бит памяти. 

Есть k хешфункций. Число добавлено, если во всех битиках, которые выдают
хешфункции стоит 1.

При добавление числа ставим 1 во всех соответсвующих клетках.  

$$h_1,..,h_k$$

$$h_i \colon x_j \rightarrow cell_{ij}$$

\begin{cppcode}
void init() {
    bit = {0};
}

void add(j) {
    for(int i = 1; i <= k; ++i) {
         bit[cell[i][j]] = 1;
    }
}

bool isAdded(j) {
    for(int i = 1; i <= k; ++i) {
         if (bit[cell[i][j]] = 0) { 
             return 0;
         }
     }
     return 1;
}
\end{cppcode}

Получился алгоритм с односторонней ошибкой.

Pr[Error] = Pr[isAdded(x[i]) = 1, но x[i] нет] //! с оценкой этого на лекции были проблемы 


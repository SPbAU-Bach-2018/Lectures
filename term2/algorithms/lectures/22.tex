\chapter{Ретроанализ и функция Гранди}
\setauthor{Виталий Пластинин}

\section {Множества. bitset}
n элементов, целые $1 \dots c$
\begin{enumerate}
\item sorted array $O(n + m)$\\
\item bitset<c> b $O(c/w)$\\
\end{enumerate}
$w$ -- размер машинного словa.\\
$unsigned \underbrace{long \; long}_{w = 64}$\\

Объединение множеств:
\begin{cppcode}
for (int i = 0; i < c / w; i++) {
    a1[i] |= a2[i];
}
\end{cppcode}

bitset умеет:
\begin{enumerate}
\item or, and, xor, not
\item $<<$, $>>$
\item .count() (\_\_buildin\_popcount $O(\log w \cdot \frac{c}{w})$)
\item
\begin{cppcode}
x = b[i];
b[i] = x;
\end{cppcode}
\end{enumerate}

Так же его можно кастить:\\
\begin{cppcode}
unsigned long long *a = (unsigned long long*)&b;
\end{cppcode}

Умножение матриц над $\mathbb{F}_2$:
\begin{cppcode}
forn(i, n) {
    forn(j, n) {
        forn(k, n) {
            c[i][j] ^= a[i][k] & b[k][j];
        }
    }
}
\end{cppcode}

Заметим, что этот код эквивалентен следующему:
\begin{cppcode}
forn(i, n) {
    forn(k, n) {
        if (a[i][k]) {
            forn(j, n) {
                c[i][j] ^= b[k][j];
            }
        }
    }
}
\end{cppcode}

Cоптимизируем, используя битсеты, до $O(\frac{n^3}{w})$, вместо $O(n^3)$:
\begin{cppcode}
forn(i, n) {
    forn(j, n) {
        if (a[i][j]) {
            c[i] ^= b[j];
        }
    }
}
\end{cppcode}

\section{Игры}

\subsection{DAG}
Пусть есть симметричная игра на ацикличном ориентированном графе:\\
Есть начальная вершина, игроки ходят по очереди. Проигрывает тот, кто не может сделать ход.\\

Решение:\\
Понятно, что вершины с исходящей степенью 0 изначально проигрышны($L$). Дальше просто запускаем \texttt{dfs} из начальной вершины.
Условия выигрышности/проигрышности вершины:\\
$\exists$ ход в $L \Rightarrow W$\\
$\forall$ ход в $W \Rightarrow L$\\
Т.е. $f[v] = not(and f[x_i])$\\

Работает, очевидно, за $O(V + E)$\\

\subsection{Функция Гранди}

\begin{Def}
$$ f[v] = mex \{f[x_i]\} $$
$$ mex A = min \; x: x \in (\mathbb{N} \cup 0) \wedge x \notin A $$
\end{Def}

\begin{Rem}
\begin{enumerate}
\item $f[v] = 0 \Leftrightarrow Lose$
\item $mex \; \emptyset = 0$
\item $0 \in A \Rightarrow mex \; A \neq 0$
\end{enumerate}
\end{Rem}

\subsection{Прямая сумма игр}

$V = \langle V_1, V_2 \rangle$\\

$\langle v_1, v_2 \rangle \rightarrow \langle x_1, v_2 \rangle$\\

$\langle v_1, v_2 \rangle \rightarrow \langle v_1, x_2 \rangle$\\

$f[\langle v_1, v_2 \rangle] = f[v_1] xor f[v_2]$

\subsection{Игра ним}
Есть кучка из $n$ камней. Игроки по очереди берут из нее любое количество камней. Проигрывает тот, кто не может сделать ход.\\

Понятно, что для одной кучки верно, что
\begin{enumerate}
\item $n = 0 \Rightarrow lose$
\item $n > 0 \Rightarrow win$
\end{enumerate}

Сумма: $n_1$ и $n_2$\\
$$lose \Leftrightarrow f[n_1] \bigoplus f[n_2] = 0$$

\begin{theorem}{}
$$f[nim(n1), nim(n2)] = n_1 \bigoplus n_2$$
\end{theorem}

\begin{proof}
%тут, наверно, хорошо бы нарисовать большую картинку и вставить
$f[\langle v_1, v_2 \rangle] = mex\{a_i \bigoplus f[v_2], f[v_1] \bigoplus b_i\}$\\
$a_i \neq f[v_i] \Rightarrow a_i \bigoplus f[v_2] f[v_1] \bigoplus f[v_2]$\\
Аналогично $b_i \neq f[v_2]$
\end{proof}

\begin{lemma}{}
$$f[nim(n1), nim(n2)] = n_1 \bigoplus n_2$$
\end{lemma}

\begin{proof}
Переход $2^k \rightarrow 2^{k + 1}$\\
\begin{tabular}{c|c|c|}
\hline
      & $A + 2^n$ & $A$\\
\hline
$2^n$ & $A$ & $A + 2^n$\\
\hline
      & $2^n$ &
\end{tabular}
\end{proof}

Следствие из теоремы -- ним на $k$ кучках:
$$f[\langle x_1, \dots, x_k \rangle] = x_1 \bigoplus \dots \bigoplus x_n$$

\subsection{Ретроанализ}
Теперь симмитричная игра на ориентированном графе с циклами.\\
$w \overset{\exists}{\rightarrow} l$\\
$l \overset{\forall}{\rightarrow} w$\\

пока $\exists$ непомеченная, для которой можем сказать -- ставим.

\begin{lemma}
Непомеченные -- ничейные.
\end{lemma}

\begin{proof}
Из непомеченных всегда есть цикл (если бы не было, была бы конечная вершина, для которой можно было бы определить состояние).
\end{proof}

Алгоритм быстрого ретроанализа:\\
\begin{enumerate}
\item Добавим все вершины, которые изначально проигрышные (их степень равна нулю. Здесь и дальше под степенью вершины подразумиваем
кол-во исходящих из нее ребер).
\item Запустим \texttt{bfs}.
\item Достаем очередную вершину из очереди. Если она $L$, то помечаем все вершины, из которых есть переход в данную как $W$ и добавляем
их в очередь. Если же она была $W$, то удаляем из графа (даже не удаляем, а просто уменьшаем счетчик степень у вершин, которые имели 
ребро в эту) все ребра, что в нее входили. Если у какой-то вершины после этого стала степень 0, то помечаем ее как $L$ и добавляем в очередь.
\item Те вершины, которые мы не смогли пометить, являются ничейными.
\end{enumerate}
Время работы, как и алгоритма \texttt{bfs} составляет $O(V + E)$\\

\subsection{BinaryTree}
Slow: $O(TreeSize)$\\

Fast:\\
\begin{cppcode}
go(v) {
    if (rand() % 2) {
        swap(v.left, v.right);
    }
    if (go(v.left) == LOSE)
        return WIN;
    if (go(v.right) == LOSE)
        return WIN;
    return LOSE;
}
\end{cppcode}

$W(h) = L(h - 1) + \frac{1}{2} \cdot W(h - 1)$\\

$L(h) = 2 \cdot W(h - 1)$\\

$W(h) = \frac{1}{2} \cdot W(h - 1) + 2 \cdot W(h - 2)$\\

Это примерно равно $O(1.687^h)$ (вместо $2^h$).

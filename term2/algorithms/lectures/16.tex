\section{Строки. Базовые алгоритмы.}
\setauthor{Денис Галеев}

\subsection{Общие понятия}

Пусть дана строка $s$ длины $n$. 
Тогда подстрокой строки $s$ называется строка $s[i \dots j]$, где $0 \le i \le j < n$.
Префиксом называются подстроки, где $i = 0$, а суффиксом подстроки, где $j = n - 1$.

Период строки это целое положительное число $t$, такое что $\forall i \in [0 \dots n-t) \colon s[i] = s[i + t]$.
Целым периодом называется такой период, что он делит длину строки нацело. 
Существует так же минимальный период, это минимум из всех периодов.

\subsection{Префикс-функция}
Пусть дана строка $s$. 
Префикс-функция $p[i]$ для строки $s$, длина наибольшего префикса строки $s[0 \dots i)$(не совпадающий со всей строкой), который одновременно является её суффиксом.
Более формально $p[i] = \max n < i \colon s[0 \dots n) = s[i - n \dots i)$.

Научимся префикс-функцию для всей строки $s$.

\begin{cppcode}
strings s;
int n = (int) s.length();
vector<int> p (n);
int k = 0;

for (int i = 1; i < n; ++i) {
  k = pi[i];
  while (k > 0 && s[i] != s[k])
    k = pi[k];
  if (s[i] == s[k]) ++k;
  pi[i + 1] = k;
}
\end{cppcode}

Поймем как и почему это работает. 
%Для начала очевиден факт, что $p[i - 1] \ge p[i] - 1$, если бы это было не так, то для префикс функции $s[0 \dots i)$, выбрана не наибольший префикс совпадающий с суфиском, например подходит большее значение $p[i] - 1$. Допустим префикс функция вычислена до $i$. 
Надо дописать.

Почему работает за $\O(n)$.
Заметим что увеличение префикс функции происходит всего лишь в одной строке \cpp"if (s[i] == s[k]) ++k;", всего таких увеличений не больше $n$, а количество уменьшений строка \cpp"k = pi[k];" не может первосходить количество увеличений, следовательно общее время работы это $\O(n)$.

Как ускорить префикс-функцию.
Можно убрать строку $k = pi[i]$, так как $k$ уже равно $p[i]$, если не верите посмотрите на последнюю строку цикла.

\subsection{КМП}
\subsection{z-функция}
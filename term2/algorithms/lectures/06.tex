\section{Дерево отрезков снизу.}
\setauthor{Бугакова Надежда}

\begin{cppcode}
for (L+=n, R+=n; L <= Rl L/=2, R/=2)
{
	if (L%2 == 1) sum += get(L++);
	if (R%2 == 0) sum += get(R--);
}
\end{cppcode}

За каждую итерацию цикла рассматриваемый отрезок становится $\ge$ в 2 раза короче $\Rightarrow$ цикл сделает $O(\log(R - L))$ операций.

На каждой итерации цикла $L$ и $R$ - отрезок вершин, которые нужно обработать.

\underline{Переход от запросов сверху к запросам снизу:}
\begin{enumerate}
	\item Одномерное: $\log n \rightarrow \log(R - L)$
	\item Двумерное: $\log^2n \rightarrow \log(R - L)\cdot BS + \log n \rightarrow \log(R - L)\cdot BS + \log(R - L)$
\end{enumerate}

<<+>> дерева отрезков снизу:
\begin{itemize}
	\item Нет рекурсии
	\item $O(\log(R - L))$
\end{itemize} 

\begin{center} \text{\bf Задачи.} \end{center}
Можно сопоставить массиву точки: $(i, a_i)$. 
\begin{equation*}
	\text{Тогда, чтобы найти решение такой задачи:} 
	\begin{cases}
		L \le i \le R \\
		X \le a_i \le Y
	\end{cases}	,
\end{equation*}
достаточно найти точку в прямоугольнике. Для того, чтобы из точек получить массив, надо определить их порядок. Есть несколько способов.
\begin{enumerate}
	\item $P(p_x, p_y), Q(q_x, q_y)$
		Можно задать два отношения:
		\begin{itemize}
			\item $P <_1 Q \text{, если } ((p_x < q_x) || (p_x == q_x \&\& p_y < q_y))$
			\item $P <_2 Q \text{, если } p_y < q_y$
		\end{itemize}
		Тогда то, что точка $P$ лежит в прямоугольнике равносильно следующему:
		\begin{equation*}
			\begin{cases}
				A \le_1 P \le_1 B \\
				A \le_2 P \le_2 B
			\end{cases}
		\end{equation*}
	\item Sort по $x$-координате, забивая на одинаковые. Тогда условие $L_x \le x \le R_x$ превратиться в условие на индекс 
	$lower\_bound(L_x) \le i \le upper\_bound(R_x)$.
\end{enumerate}

Для многомерной задачи дерево отрезков нужно хранить в дереве отрезков. $x_1, \dots, x_d$ и  условия $L_i \le x_i \le R_i$. 
Одним деревом отрезков мы понижаем размерность на 1: $d \rightarrow d - 1$

Sort $x_d$. Строим на массиве дерево отрезков. В каждой вершине дерева отрезков, соответсвующей $[L..R]$, 
хранится ещё одно дерево отрезков на точках, хранящихся c индексами c $L$ по $R$ на одну размерность меньше.

\underline{Как посчитать ответ:} BS(находим L и R индексы в массиве) + Д.О. + Запросы к соответсвующим вершинам.
  
Д.О. $\rightarrow$ Д.О. $\rightarrow \dots \rightarrow$ Д.О. $\rightarrow$ массив. Каждый переход осуществляется за $O(\log n)$, но
мы умеем последние два перехода делать за $O(\log n)$ при запросах.

\underline{Итог:}
\begin{itemize}
	\item $O(n\log^{d - 1}n)$ памяти
	\item $O(\log^d n)$ Change
	\item $O(\log^{d - 1} n)$ Count, Sum, Min   
\end{itemize}

\section{Дерево Фенвика}

\begin{cppcode}
get(i) { //sum[0, i], $O(\log n)$
		sum = 0;
		for (; i >= 0; i &= (i + 1), i--)
			sum += a[i];
    return sum;
}
\end{cppcode}

\begin{cppcode}
add(i, x){
	for (; i < n; i |= (i + 1))
		a[i] += x;
}
\end{cppcode}

<<+>> Дерева Фенвика:
\begin{itemize}
	\item Код короткий.
	\item Нет дополнительной памяти(Inplace).
	\item Константа меньше.
\end{itemize}

В ячейке массива $a[i]$ хранится сумма чисел с индексами $[i\&(i + 1) \dots i]$

\begin{center} \text{\bf Корректность дерева Фенвика.} \end{center}

\begin{itemize}
	\item Корректность add.
		\begin{enumerate}
			\item Для любого i, учатсвующего в for: $i\&(i + 1) \le i_0 \le i$, то есть прибавляем только к тому, к чему надо.
			\item Все нужные числа переберём.
		\end{enumerate}
	\item Коррекнтность get.	
\end{itemize}

\begin{center} \text{\bf Ещё один <<+>> дерева Фенвика: 2D-деревья.} \end{center}

$O(\log^2 n)$

\begin{cppcode}
	get(i_0, j_0) { // sum [0, i_0][0, j_0]
		sum = 0;
		for (i = i_0; i >= 0; i &= (i + 1), i--)
			for (j = j_0; j >= 0; j &= (j + 1), j--)
				sum += a[i][j];
		return sum;
   }
\end{cppcode}

\begin{cppcode}
	add(i_0, j_0, x) { 
		for (i = i_0; i < n; i |= (i + 1))
			for (j = j_0; j < n; j |= (j + 1))
				a[i][j] += x;
		return sum;
   }
\end{cppcode}

$\sum(L \dots R] = $get($R$) - get($L$) одномерный случай.

$\sum(x_1, x_2]\times(y_1, y_2] = $get($x_2$, $y_2$) - get($x_1$, $y_2$) - get($x_2$, $y_1$) + get($x_1$, $y_1$) двумерный случай

$k$-мерное: $(\log n)^k2^k = O((\log n)^k)$. ($2^k$ - количество слагаемых)

\section{SkanLine}

\begin{enumerate}
	\item $n$ точек, $m$ прямоугольников. Для каждого прямоугольника  $\sum$ значений точек/количество точек. За $O((n + m)\log(n + m))$
	\begin{enumerate}
		\item Сжатие координат: $0 \le x, y \le n + 2m$
		\item Переходим к полоскам(стаканам: $[-\inf; x]\times[y_1; y_2]$). Прямоугольник: 1 полоска - 2 полоска.
		\item ScanLine: идём слева направо по $x$(Sort Events по $x$). Дерево отрезков или Фенвик по $y$.

		События:

		1) Встретили точку с координатой $y$: Add($y$);

		2) sum($y_1$, $y_2$);

		Приоритет: Add < Sum.
	\end{enumerate}
	\item Последовательность точек, возрастающая по обеим координатам, длиннейшая.

	f[i] - максимальное количество точек у последовательности с концом в этой точке.

	f[i] = $\max\limits_{y_j < y_i} f[j] + 1$; $\leftarrow$ Д.О. с max.

	PS: есть решение с динамикой + BS.
\end{enumerate}

\section{Алгортим Диница}
Алгоритм Диница позволяет найти максимальный 
поток за $O(VE + VK)$, где к - количество найденных путей из истока в сток. Если добавить scaling, будет $O(VE \log U).$ 
 
Сеть кратчайших путей

Все вершины графа можно разбить части(слои) так, что длина минимального пути от стока до всех вершин одного слоя с одинакова. 

Теперь из ребер исходного графа выберем те, которые соединяют вершины из "соседних" слоев. Эта констркуция и называется сетью кратчайших путей, а любой путь из истока в сток по ребрам сети кратчайших путей является кратчайшим.


\subsection{Описание алгоритма}
\begin{itemize}
\item
Получаем сеть кратчайших путей, найдя алгоритмом Форда-Беллмана расстояние от истока до каждой из вершин.
\item 
Жадно находим дополняющие пути и пускаем по ним потоки. Если ни одн путь не нашелся, алгоритм завершается.
\item
Строим остаточную сеть и возвращаемся к первому пункту. 
\end{itemize}

Псевдокод
\begin{cppcode}
while (bfs(s, t)) {
  while dfs(s)
    UpdateFlow();
    
dfs(v) {
  if(v == t) return 1;
  while exists (v, u) в сети кратчайших путей {
      if (dfs(u))
          return 1;
      else 
          удаляем ребро из v в u
  return 0;
} 
\end{cppcode}

С каждым шагом алгоритма минимальное расстояние $d$ от каждой вершины до истока не уменьшается
Пусть это не так и для вершины $v$ оно уменьшилось. Но тогда найденный до этого путь не был кратчайшим.

Докажем, что итоговая сложность $O(VE + K)$.

$\Sigma (Vk_i + E) \leq VE$ -- для одной фазы

$V$  -- фаз всего.

При единичной пропускной способности Диниц работает за $O(VE)$ 

Каждое принадлежит ровно одному пути.

\begin{theorem}{Первая теорема Карзанова}

Если $P = \Sigma  min(c_in, c_out)$, то Диниц работает за $O(V \sqrt{P})$
\end{theorem}

\begin{theorem}{Вторая теорема Карзанова}

Если нет кратных ребер, количество фаз -- $V^\frac{2}{3} U^\frac{1}{3}$.
\end{theorem}
\begin{proof}
$U \cdot min|V_i||V_{i + 1}| \leq (\frac{V}{L}^2$

$Flow \leq Cut \leq  min|V_i||V_{i + 1}| \Rightarrow $

 $V^\frac{2}{3}U^{1}{3}\leq L \Rightarrow$
 количество фаз $\leq V^\frac{2}{3} U^\frac{1}{3}$ ч.т.д.
\end{proof}

\subsection{LR-поток}

\textbf{\color{red} я не уверена в том, что написано в этом разделе. Пусть кто-нибудь проверит}
\begin{enumerate}
\item решим задачу, когда есть много 
истоков и много стоков. И хотим пустить максимальный поток из
истоков в стоки. 

    \begin{proof}
    Создадим вспомогательный исток S и вспомогательный сток T.
    
    Соединим S со всеми основными стоками  ребрами пропускной
    способности $\infty$ и все стоки со стоком $T$ пропускная
    способность опять $\infty$.

    Пустим поток из $S$ в $T$. 
    \end{proof}
\item Немного усложним задачу. Теперь у истоков
есть избытки $s_i$ и у всех стоков недостатки $t_j$. 
Известно, что $\sum s_i = \sum t_j$. Нужно "переправить"
все избытьки в недостатки .
    \begin{proof}
        Такая же конструкция, как и в прошлый раз, 
        но пропускная способность ребер не $\infty$, 
        а $s_i$ и $t_j$ соответственно.
    \end{proof}
\item Теперь хотим найти $LR$ циркуляцию. У каждого ребра
пропускная способнасть ограничена с двух сторо $L \le f_e \le R$
Хотим пустить какую-то циркуляцию, которая удовлетворяет этим условиям.
    \begin{proof}
        По всем ребрам пустим поток $L_e$. В результате чего в
        каких-то вершинах окажутся избытки, в каких-то недостатки,
        а пропускная способность ребер будет $R-L$.

        Теперь нужно переноправить избытки в недостатки. Свели задачу к предыдущей. 
    \end{proof} 
\item $LR$ "--- поток. 
    \begin{proof}
        Проводим ребро $TS$ с пропускной способнотью $\infty$ и ищем циркуляцию.
    \end{proof}
\item $max LR$ "--- поток. 
    \begin{proof}
     $f_{max} - f$ "--- поток в $G_f$.
     То есть найдем какой-то $L-R$ поток. То есть теперь у вершине не будет избытков. Они все уже
     будут перенаправлены и в таком новом графе найдем максимальный поток. 
    \end{proof}
\end{enumerate} 
 

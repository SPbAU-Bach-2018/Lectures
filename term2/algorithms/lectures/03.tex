\chapter{Операции на отрезке, Rope, SkipList, 2-3-Tree, B-Tree, RB-Tree, AA-Tree}
\setauthor{Степанов Владимир}

\section {Неявный ключ: операции на отрезке $[L, R]$}

\begin{enumerate}
	\item \cpp'Sum(L, R)'
	\item \cpp'Min/Max(L, R)'
	\item \cpp'Reverse(l, R)'
	\item \cpp'Add(L, R, d)'
	\item \cpp'Set(L, R, val)'
\end{enumerate}

$[L, R]$, $L \le x \le R$

<<Выспличиваем отрезок>> от $L$ до $R$.
\cimg{image03_01.jpg}{}

В вершине храним флаг, присвоено ли что-нибудь, если да, то храним значение, иначе None.

\cpp'Root->value = ...'

\subsection{pushDown}
\begin{cppcode}
pushDown(v)
{
	if (v->value != None)
	{
		v->l->value = v->value;
		v->r->value = v->value;
		v->value = None;
		x = v->value;
		v->value = None;
	}
}
\end{cppcode}

Update:
\begin{cppcode}
	v->sum = v->value != None ? v->value * v.size :
	v->l->sum + v->r->sum + v->x;
\end{cppcode}

\begin{Rem}
	Update не нужен, если перед заходом в сыновей всегда делать \cpp'pushDown()'.
\end{Rem}

\subsection{Reverse}
\begin{cppcode}
if (Rev == 1)
{
	L->Rev ^= 1;
	R->Rev ^= 1;
	swap(L, R);
	Rev = 0;
}
\end{cppcode}

\subsection{Sum}

\begin{cppcode}
getSum(v, vL, vR, L, R) //v, [vL, vR] -- отрезок соответствующий v, [L, R]
{
	if (v == NULL)
		return 0;
	if (vL > R || vR <  L)
		return 0;
	if (vL >= L && vR <= R)
		return v->sum;
	return getSum(v->L, vL, v->x - 1, L, R)
		+ getSum(v->R, v->x + 1, vR, L, R)
			+ (L <= v->x <= R ? v-> x : 0);
}
\end{cppcode}

Асимптотика $O(\log N)$, так как на каждом уровне рассмотрим не более 4 отрезков.

\section{Rope}

\cimg{image03_02.jpg}{}

\begin{Def}
	Rope -- интерфейс.
	\begin{enumerate}
		\item \cpp'Access(i)'
		\item \cpp'Link'
		\item \cpp'Cut'
	\end{enumerate}
\end{Def}

Реализация: treap+ reverse, так как чтобы сделалть правильный Link нужно уметь переворачивать отрезки,
потому что можем соединять любой конец одного отрезка с любым концом другого.

\cimg{image03_03.jpg}{}
 
\section{SkipList}
 
\cimg{image03_04.jpg}{}

\subsection{Operations}
 
\begin{enumerate}
	\item \cpp'Find = O(k)'
	\item \cpp'Add = Find + O(1)'
	\item \cpp'Delete = Find + O(1)'
	\item \cpp'Split = Find + O(1)'
\end{enumerate}

\subsection{Find = $O(\sqrt{N})$}

Разобьем массив на $\sqrt{N}$ кусочков и будем хранить в первом элементе отрезка ссылку на первый элемент следующего отрезка, также запомним $L_{i}$ -- на сколько элементов мы при этом прыгаем. При добавлении элемента просто изменяем $L_{i}$ для начала текущего отрезка. Если $L_{i} \geqslant 2 \sqrt{N}$, тогда Rebuild. Теперь Fnd за $O(\sqrt{N})$\\

\cimg{image03_05.jpg}{}


\subsection{Ideal Model}

\cimg{image03_06.jpg}{}

Всего $\log{N}$ уровней и на каждом уровне не более одного раза мы идем в право, значит Find = $O(\log{N})$. Остальные операции не работают.


\subsection{Skip-List}

Храним $\log{N}$ списков, в первом лежат все элементы, во втором каждый элемент из первого листа присутствует с вероятностью $\frac{1}{2}$, на третьем каждый элемент из второго списка лежит с вероятностью $\frac{1}{2}$ и т.д.

$M_{i, k}$ "--- Мат. ождиание того, что элемент $i$ окажется на $k$-ом уровне.$ P = \frac{1}{2^{k}}$, \xi = 1, значит $M_{i,k} = P \xi = \frac{1}{2^{k}}$.
Значит, матожидание количества вершин на $k$-ом уровне равно $\frac{N}{2^{k}}$, т.к. матожидания складываются.

Всего $\log{N}$ переходов вниз и на каждом уровне Мат. ожидание на ход вправо $O(1)$.

Если мы сделаем $k$ прыжков в право, значит $k$ элементов подряд не смогли прыгнуть на уровень выше, вероятность этого $\frac{1}{2^{k}}$. Мат. ожиание $k$ прыжков в право равно $\frac{1}{2^{k}}$.

$X = \frac{1}{2} + \frac{1}{4} + \dots = 1$. Сложили мат. ожидания.

\section{2-3-Tree}

\cimg{image03_07.jpg}{}

Все ключи различны. $A < X < B < Y < C$. Все листья имеют одинаковую глубину.

Добавление: спускаемся по дереву до листа и приклеиваем вершину. Могли получить 4-вершину.

\cimg{image03_08.jpg}{}

Возможно, придется сделать так и выше, возможно изменится корень.

Удаление: избавляемся от пустых вершин.

\cimg{image03_09.jpg}{}


\section{B-Tree}

K-Tree == B-Tree.\\

$[2-3] \longrightarrow [k, 2k)$\\

Пусть $k$ = 1024, тогда получим очень маленькую высоту дерева.

$\log_{2k}{N} \leqslant h \leqslant \log_{k}{N}$
\[ \frac{\log_{k}{N}}{\log_{2k}{N}} = \frac{N/\log{k}}{\log{N}/\log{2k}} = \frac{\log{k} + \log{2}}{\log{k}} = 1 + \log_{k}{2} \]
$k = 1024$, $h \leqslant 3$, $N = 1e9$

Легче хранить на диске, можно читать вершины с диска.

Диск: $\log_{k}{N}$. Время: $k\log_{k}{N} \longrightarrow \log_{2}{k} \log_{k}{N}$.

\section{RB-Tree}

\subsection{Описание}
\begin{enumerate}
	\item Бинарное дерево.
	\item Вершины покрашены в красные и черный цвет, причем нет двух красных вершин соединенных ребром.
	\item Все фиктивные листья имеют одинаковую черную глубину.
	\item $2^{h} - 1 \leqslant size \leqslant 2^{2h} - 1$, где h "--- черная глубина дерева.
\end{enumerate}

\cimg{image03_10.jpg}{0.5}

\subsection{RB-Tree и 2-3-4-Tree}

RB-Tree $\longleftrightarrow$ 2-3-4-Tree\\
Объединим красные вершины с их родителями и получим RB $\longrightarrow$ 2-3-4.

\cimg{image03_11.jpg}{0.5}

Следующими преобразованиями получим RB $\longleftarrow$ 2-3-4.

\cimg{image03_12.jpg}{0.5}

\subsection{AA-Tree}

AA-Tree "--- вид RB-Tree, так же  AA-Tree $\longleftrightarrow$ 2-3-Tree, т.к. есть только такие конструкции:

\cimg{image03_13.jpg}{}

$2^{h} - 1 \leqslant size \leqslant sqrt{3}^{h} - 1$, где $h$ "--- черная высота дерева.\\

\subsection{$B^{+}$-Tree}

Является 2-3-Tree.

Все ключи хранятся в листьях, а в вершинах храним Max/Min в поддереве.\\

\subsection{$B^{*}$-Tree}

$[2-3] \longrightarrow [k, 1.5k )$
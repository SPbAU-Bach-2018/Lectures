\setauthor{Иван Дегтяренко}

\section{Preflow-push поток}
Конспект лекции можно увидеть \href{http://acm.math.spbu.ru/~sk1/mm/lections/2015-04-08-preflow-push-2.pdf}{тут}.

\section{Глобальный реберный разрез}

Научимся выбирать минимальный реберный разрез для всех позиций $s$ и $t$. 
Алгоритм Штор-Вагнера:
\begin{description}
\item Алгоритм состоит из $n - 1$ фазы.
\item На каждой фазе в множество $A$ начинаем по одной добавлять вершину, такую что $ \omega(w) = \sum_{u \in A} c(v, u) = max$.
\item Две последние добавленные вершины называем истоком $s$ и стоком $t$, а в качестве размера минимального разреза между ними берем значение $w(t)$.
\item Обновляем ответ и объединяем вершины $s$ и $t$.  
\end{description}

Алгоритм Каргера-Штейна:
\begin{cppcode}
while (n > 2) {
	e = Random Edge;
	Join(e.begin, e.end);
}
\end{cppcode}

\begin{assertion}
Вероятность того, что ребра, оставшиеся после алгоритма, будут ребрами минимального разреза, равна $\Theta(\frac{1}{n^2})$.
\end{assertion}

\begin{proof}
\begin{enumerate}
\item Обозначим размер ответа за $ans$, минимальную степень вершины за $k$. Заметим, что $ans < k + 1$.
\item На $i$-м шаге у нас $n - i$ вершин, вероятность попасть в ребро не из разреза - $1 - \frac{k}{E}$.
\item Тогда, так как $E > \frac{(n - i)k}{2}$, вероятность ни разу не попасть в ребро из разреза:
$$p_n > \prod_{i = 0}^{n - 2} (1 - \frac{2}{n - i})  = \frac{2}{n(n - 1)}$$. 
\end{enumerate}
\end{proof}

Эту вероятность можно улучшить до $\Theta(\frac{1}{n^2})$ следующим образом:
\begin{description}
\item Пусть в начале у нас было $n$ вершин. Будем объединять вершины, пока их не станет $\frac{n}{2}$ 
\item Теперь выберем не одно, а два случайных ребра, и перейдем к шагу один для графа со сжатым первым ребром, аналогично для второго.
\end{description}


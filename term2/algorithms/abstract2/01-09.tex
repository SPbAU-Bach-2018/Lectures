\section{} % 01
Есть неориентированный граф, хотим найти паросочетание "--- множество рёбер такое, что никакие два не имеют общего конца.
Максимальное по размеру, является совершенным, если размера $\frac{|V|}{2}$.
Лемма для всех графов: есть чередующаяся цепь $\iff$ можно увеличить размер паросочетания, вправо очевидно, влево "--- рассмотрели симметрическую разность текущего и оптимального, там есть циклы и как-то чередующиеся цепочки, точно есть подходящая.
В двудольном умеем искать за $\O(VE)$ (Кун), $\O(\sqrt V E)$ (Хопкрофт-Карп, Диниц).
В произвольном можно искать за $\O(VE)$ и $\O(\sqrt VE)$ (Эдмондс), мы не учились.

\section{} % 02
Кун: ищем дополняющие цепи dfs'ом по одной (поиск за $E$, цепей $V$).
Можно обнулять достижимость только когда поменяли граф (тогда $O(|M|E)$), можно жадно инициализировать (хотя бы половину ответа наберём).
Кёнига: в двудольном макспарсоч равен минимальному вершинному покрытию.
С одной стороны не меньше, а с другой стороны "--- взяли последнюю итерацию Куна, разбили вершины на (не)достижимые в каждой доли, взяли какие-то 2 из 4 множеств, успех.
За $\O(K^2D)$ из конспекта не нужно.

\section{} % 03
Максимальная клика $\iff$ максимальное независимое множество (инвертировали рёбра).
Множество независимо $\iff$ его дополнение есть покрытие.
В произвольном графе проверить существование клики размера $k$ "--- NP-полная задача.
В двудольном используем теорему Кёнига.

\section{} % 04
\TODO 9.7 % доказывать Холла надо, из Куна следует

\section{} % 05
\TODO

\section{} % 06
Лемма Холла: для любого подмножества вершин левой доли смежное подмножество вершин правой доли должно быть не меньше $\iff \exists$ совершенный парсоч.
В регулярном двудольном графе условие Холла выполняется (посчитали количество концов рёбер слева и справа).
В двудольном регулрном графе: пусть хотим раскрасить рёбра так, чтобы смежные рёбра были разных цветов, надо хотя бы $\max \deg$ цветов.
Взяли парсоч, покрасили, повторить $k$ раз.
Если граф нерегулярный, то надо добавить лишних рёбер (можно кратные) и вершин, решить для регулярного, убрать лишнее из ответа.
Можно разделяй-и-властвуй: если в регулярном графе степень чётна, то выделим Эйлеровы циклы в компонентах, разделим задачу на две.
Если нечётна "--- выделили парсоч один раз, стало чётно.

\section{} % 07
Хотим красить вершины, смежные "--- разного цвета.
Очевидно, что в $\max \deg +1$ цвет можно покрасить.
Теорема Брукса: в $\max \deg$ цветов нельзя покрасить только нечётные циклы и полные (из связных графов).
Теорема Визинга: если красим рёбра, то любой граф всегда можно покрасить в $\max \deg + 1$ цветов (а хотя бы $\max \deg$ точно нужно).
А еще любой планарный граф можно покрасить в четыре цвета.
Задача vertex-list-coloring: для каждой вершины есть цвета, в которые можно её красить.
Если для каждой вершины список хотя бы длины $\deg + 1$, то умеем делать.

Покраска в пять цветов планарного: всегда есть вершина с $\deg \le 5$ (по формуле Эйлера и тому, что грань ограничена хотя бы трёмя рёбрами).
Удалили, покрасили остаток, вернули, в худшем случае все пять цветов заняты соседями.
Рассмотрим пары соседей $(1,3)$, $(2,4)$, хотя бы одна из них не соединена между собой цепочкой по планарности, взяли и заменили там целиком один цвет на другой.
Алгоритм, хороший иногда: пусть есть множество уже покрашенных,
выбираем вершину, у которой максимальное число покрашенных соседей (если таких несколько "--- рандомную), красим ее в минимально возможный цвет.

\section{} % 08
Перманент "--- как определитель, только знак всегда плюс.
Перманент матрицы смежности "--- количество совершенных паросочетаний в двудольном графе.
А по модулю два перманент и определитель "--- одно и то же.

\section{} % 09
Есть $n$ мальчиков и $m$ девочек, у каждого есть список предпочтений из кого-то другого пола от <<очень нравится>> к <<не очень>>.
Надо найти стабильное паросочетание (хоть какое-то).
Нестабильно, если есть две пары $(a_1, b_1)$ и $(a_2, b_2)$ такие, что $a_2$ с большей радостью будет с $b_1$, а $b_1$ "--- с большей радостью будет с $a_2$, чем с $a_1$.
Тут мы говорим, что <<остаться без пары>> всегда хуже, чем какая-нибудь пара, поэтому пустое обычно нестабильно.
Можно находить нестабильности и исправлять, меняя пары, но это может не закончиться.
Алгоритм: мальчики предлагают по очереди девочкам от первой в списке (предлагает и ждёт ответа, в случае отказа "--- следующей), девочки выбирают лучшее предложение и говорят <<может быть>>, остальным отказывают.
Когда перестало происходить что-либо "--- девочки говорят <<да>>, процесс завершён.
Это точно завершится, так как каждый выполнит конечное число операций.
И нестабильностей не будет (от противного).

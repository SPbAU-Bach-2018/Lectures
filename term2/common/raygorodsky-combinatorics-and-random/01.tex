\section{Задача Данцера и Гранбаума}

\textit{
Есть обычная плоскость $\R^2$.
Плоскость "--- это же просто множество пар вещественных чисел.
Как много можно взять точек на плоскости, чтобы любой треугольник, образованный этими точками, был остроугольным?
}

Вопрос абсолютно понятен, и при чём же тут вероятность?
А ни при чём: ответ три.
Если взять уже четыре точки, то они или образуют выпуклый четырёхугольник "--- а там есть тупой угол,
"--- или треугольник с точкой внутри, и там тоже всё плохо.

\begin{Rem}
	Три точки на прямой дают угол $180^\circ$
\end{Rem}

А в пространстве $\R^3$? можно взять пирамидку (тетраэдр).
Но этого не достаточно: можно совместить две пирамидки, это уже 5 точек.

\begin{Exercise}
	Докажите, что больше пяти уже нельзя.
\end{Exercise}

\begin{Rem}
	Тряпка с доски, как ни странно, стирает мел!
\end{Rem}

Задачка на IQ: если взять большую размерность, что будет дальше, в $\R^n$? $2n-1$?
Авторы предполагали также...

Кому-нибудь не понятно, что такое $\R^n$?
Ну всё равно расскажу.
Мы рассматриваем точки как последовательности чисел
\[ \R^n = \left\{ \bar x = (x_1, x_2, \dots, x_n) \mid x_i \in \R \right\}\]
Но нужно ещё углы.
Что такое скалярное произведение?
\[ (\bar x, \bar y) = x_1 y_1 + x_2 y_2 + \dots + x_n y_n \]
Но что нам даёт оно?
Оно даёт косинус угла:
\[ (\bar x, \bar y) = |x| |y| \cos \widehat{(\bar x \bar y)} \]
И даже теорема косинусов есть:
\[ |\bar x \bar y| = (\bar x \bar x) + (\bar y \bar y) - 2 (\bar x \bar y) \]

Теперь собственно ведём функцию
\[ f(n) = \max \left\{ |X| \mid X \subset \R^n \land \forall \bar x, \bar y, \bar z \in X\; \text{$\bigtriangleup \bar x\bar y\bar z$ остроугольный}\right\}\]
Предполодили, что
\[ f(n) = 2n - 1 \]

И с 1963 года гипотезу не могли обосновать.
А в 1983 произошёл облом.
И сейчас вероятноть нам поможет.
Пришли математики Эрдеш и Фюреди, оба венгры.
Эрдеш написал за 60 лет полторы тысячи статей!

\begin{theorem}
	\[f(n) > \left\lfloor \frac12 \left(\frac23\right)^n\right\rfloor\]
\end{theorem}

Заметим, что эта оценка обгоняет $2n-1$ только аж при $n \ge 35$!!! Говорят, что при $n=4..34$ оценка $2n-1$ тоже неверна...

Ну да ладно. Докажем, что при заданном количестве точек есть вероятность существования хорошего множества точек! Тогда получим, что оно есть, но найти кроме как перебором мы не сможем...

\begin{proof}
	\[ m \lrh \left\lfloor \frac12 \left(\frac23\right)^n\right\rfloor \]
	Построим случайную точку $A_1 = (x_1, \dots, x_n)$.
	Что значит случайная?
	Мы выбираем случайную точку так: берём монетку (были какие-то введения математической монеты, не будем этим заниматься).
	Бросаем её.
	С её помощью расставляем нули и единицы в $x_i$.
	\[ P(A_1) = \frac1{2^n} \]
	Теперь ровно также составим случайную точку $A_2$ той же монетой.
	Также составим всего $2m$.

	\begin{assertion}
		Там нет тупых углов "--- они же на вершинах гиперкуба!
	\end{assertion}
	\begin{proof}
		Ну действительно, рассмотрим тройку точек $A_1, A_2, A_3$.
		Рассмотрим скалярное произведение
		\[ (A_1 - A_3, A_2 - A_3) \]
		Рассмотрим знак. Посмотрим на каждую координату
		\[
			\begin{matrix}
				A_1   & 1 & 1 & 1 & 1 & 0 & 0 & 0 & 0 \\
				A_2   & 1 & 1 & 0 & 0 & 1 & 1 & 0 & 0 \\
				A_3   & 1 & 0 & 1 & 0 & 1 & 0 & 1 & 0 \\
				\prod & 0 & 1 & 0 & 0 & 0 & 0 & 1 & 0
			\end{matrix}
		\]
		Действительно, каждое слагаемое неотрицательно, и всё скалярное произведение неотрицательно!
	\end{proof}

	Остались прямые углы... Все знают, что такое математическое ожидание? Пусть есть случайная величина "--- функция от исхода
	\[ \xi(A_1, ..., A_{2m}) \in \R \]
	"--- количество <<прямых>> углов, образованных точками $A_i$.
	Почему в кавычках?
	А если точки совпадут?
	Там произведение ноль, а угла-то нет!
	Ну давайте представим, что точки бесконечно близко, но не совпадают, и образуют прямой угол.
	То есть мы написали именно ту функцию, которую хотели!

	Математическое ожиданее величины $\xi$ есть
	\[ \mathup{M}X = \sum_{i=1}^n y_i P(X = y_1) \]
	А оно ещё и линейно!
	\[ \mathup{M}(c_1X_1 + c_2X_2) = c_1\mathup{M}X_1 + c_2\mathup{M}X_2 \]

	Посчитаем с помощью его линейности матожидание $\xi$. Разложим $\xi$ в сумму
	\[ \xi = \xi_1 + \xi_2 + \dots + \xi_k \]
	где $k$ "--- количество углов, порождённых точками $A_3$. Как заментно,
	\[ k = A_{2m}^3 = 3C_{2m}^3 \]
	Тогда
	\[ \xi_i(A_1, A_2, \dots, A_n) = \begin{cases} 1 & \text{угол с индексом $i$ <<прямой>>} \\ 0 & \text{иначе}\end{cases} \]
	Чему равно его матожидание?
	Оно просто равно вероятности того, что этот угол <<прямой>>.
	Глянем на таблицку наверху.
	Там в двух из восьми случаем в произведение летит единичка, делая угол прямым. Тогда вероятность того, что угол прямой, есть
	\[ P(\xi_i = 1) = \left(\frac68\right)^n = \left(\frac34\right)^n \]
	Тогда
	\begin{gather*}
		\mathup{M}\xi = 3C_{2m}^3 \left(\frac34\right)^n = \frac{3\cdot2m(2m-1)(2m-2)}6 \left(\frac34\right)^n < 4m^3 \left(\frac34 \right)^n = \\
		= 4m \cdot m^2 \cdot \left(\frac34\right)^n \le 4m \frac14 \left(\frac43\right)^n \left(\frac34\right)^n = m
	\end{gather*}
	Тогда получили
	\[ \mathup{M}\xi < m \Ra \exists A_1, \dots, A_{2m}\colon \xi(A_1, \dots, A_{2m}) < m \]
	Берем углы, и выкидываем его вершину. Останется хотя бы $m$ различных точек с острыми углами.
\end{proof}

Есть и более точные оценки.

\begin{Exercise}
	(*) Докажите, что
	\[f(n) \leqslant 2^n \]
	Звёздочка одна, это уже решено.
\end{Exercise}

Весь пафос в том, что геометрия не в силах решить эту задачу!

\section{Гиперграфы}

Что такое $n$"---однородный гиперграф?
\[ H = (V, E), E \subset 2^V, \forall e \in E\; |e| = n \le 2 \]

\begin{Rem}
	Есть фонд Династия. Лектор там получил грант и участвовал в конференции победителей. Куча людей рассказывали что-то о алгебре, что-то сложное, всем понятоно... Рагородский начал лекцию с слов <<Хроматическое число>> и никто их не понял...
\end{Rem}

\begin{Def}
	Хроматичеким числом называется минимальное число цветов, что есть раскраска, 
	при которой каждое ребро соединяет вершины только различных цветов.
\end{Def}

Для гиперграфа тоже можно это определить! Мы скажем так:
\begin{Def}
	Хроматичеким числом называется минимальное число цветов, что есть раскраска, 
	при которой каждое ребро не является одноцветным.
\end{Def}

\begin{exmp}
	Вот есть аудитория. 
	Выберем пять лучших специалистов по каждой специальности. 
	Получим гиперграф. 
	Если хроматическое число всего 2, то можно разбить аудиторию на две части, что в каждой есть специалист в каждой области!
\end{exmp}

Тот же самый Эрдеш совместно с Хайналом пронаблюдали следующее:

\begin{theorem}
	Если у $n$"---однородного гиперграфа $H$ число рёбер менее $2^n$, то его хроматическое число не больше двух.
\end{theorem}

\begin{proof}
	Рассмотрим случайную расскраску.
	Её вероятность ($v = |V|$)
	\[ P(\chi) = \frac1{2^v} \] 
	В нём введём событие $A_e$ "--- ребро $e$ одноцветное. 
	Его вероятность
	\[ P(A_e) = \frac1{2^{n-1}} \]
	Теперь
	\[ P\left(\bigcup_{e \in E} A_e\right) \le \sum_{e \in E} P(A_e) = |E| \frac1{2^{n-1}} < 1 \]
	Значит 
	\[ P\left(\overline{\bigcup_{e \in E} A_e}\right) > 0 \]
	то есть есть случай, когда нет одноцветных рёбер.
\end{proof}

Давайте ещё порассуждаем.

\begin{Def}
	\[m(n) = \min\{m \mid \text{$\exists H=(V, E)$ "--- гиперграф с $m$ рёбрами, что $\chi(H) > 2$}\} \]
\end{Def}

Нет ни одной оценки без вероятностей:
\begin{description}
\item[1973, Бек:]
	\[ m(n) \ge c \sqrt[3]{\frac{n}{\ln n}} 2^n, c > 0 \]

\item[2000, Радхакришнан и Сринимашан (?)]
	\[ m(n) \ge c \sqrt{\frac{n}{\ln n}} 2^n, c' > 0 \]

\item[1963, Эрдеш]
	\[ m(n) \le \frac{c \ln 2}{4} n^2 2^n (1 + o(1)) \]
	Ничего лучше нет
\end{description}

Зато есть обобщение.
\begin{Def}
	\[m(n, r) = \min\{m \mid \text{$\exists H=(V, E)$ "--- гиперграф с $m$ рёбрами, что $\chi(H) > r$}\} \]
\end{Def}
Там стооолько оценочек... Данила Черкашин, студент из Санкт-Петербурга по прозвищу Лось, добился хорошей оценки в позапрошлом году
\[ m(n, r) > c \left(\frac{n}{\ln n}\right)^\frac{r-1}r r^{n-1} \]
Уподобляйтесь Лосю!


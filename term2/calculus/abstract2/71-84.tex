\section{} % 71
Разбиение $\tau$ измеримого множества $E$ есть конечный набор измеримых множеств, которые попарно не пересекаются и в объединении дают $E$.
Мелкость (ранг) $|\tau|$ есть наибольший диаметр.
$r'$ подчинено $r$ (оба разбивают $E$), если мы сначала разбили на $r$, а потом доразбили до $r'$.
Подчинимость транзитивна, для любых двух разбиений если общий подчинённый (взяли попарные пересечения).
Оснащение $\xi$ "--- набор точек из множеств.

\section{} % 72
Сумма Римана для функции и оснащённого разбиения "--- сумма $f(\xi_k)\mu E_k$.
Интеграл Римана $I=\int_E f$, если $\forall \epsilon > 0\colon \exists delta$ такое, что все разбиения мелкого ранга независимо от оснащения отличаются от $I$ несильно.
$\idotsint$ "--- пока что просто обозначение.
Можно еще требовать от разбиения не дизюнктивность, а пересечение по мере ноль.

\section{} % 73
В одномерном: если $f$ на отрезке интегрируема, то ограничена.
В многомерном: если есть последовательность разбиений $E$ (с бесконечно мелким рангом), все множества всех разбиений имеют положительную меру.
Тогда если $f$ интегрируема на $E$, то ограничена.
Простой ограниченности нет (можно взять квадрат с хвостом, на хвосте неограничено, но интеграл-то ноль).
Из одномерности сложное условие следует (кроме $[a,a]$, но тогда одна точка и всё ок).

Доказательство: взяли оснащённое разбиение из последовательности такое, что от интеграла отличается на единицу или меньше.
Взяли множество из разбиения, на котором $f$ неограничена, у него положительная мера, вынесли это слагаемое, подвигали точку из оснащения, получили сумму Римана бесконечно большую (положительную или отрицательную).
Тут мы требовали именно последовательность, чтобы ограничение на $E$ не зависело от функции и можно было сразу по $E$ сказать, что всё будет ок.
Так-то нам для конкретной функции хватает <<посмотреть на одно хорошее разбиение>>.

\section{} % 74
Колебание на $E$ есть супремум минус максимум (или супремум разности), обозначается $\omega(f, E)$.
Критерий: есть измеримое $E$.
$f$ интегрируема $\iff$ $\forall \epsilon > 0 \colon \exists \delta > 0$ такая, что для всех мелких разбиений $\omega(f, E_k)\mu E_k < \epsilon$ (не забыли про $\mu E_k$).
$\Ra$: получили $\delta$ такое, что все оснащённые разбиения хорошо приближают интеграл.
Зафиксировали разбиение $\tau$, сначала в каждом элементе разбиения выбрали хорошее приближение супремума (если разброс в куске равен $\gamma$, то с точностью в $\frac \gamma 4$),
а потом везде выбрали хорошее приближение инфиума.
Разница значений в каждом куске хотя бы $\frac \gamma 2$, то есть можно оценить <<интеграл колебаний>> как удвоенную разность.
Но разность не более $2\epsilon$, поэтому колебания тоже малы.
$\La$: сначала показали, что если одно разбиение подчинено другому и мелкость б\'ольшего $< \epsilon$, то на каждом элементе из большого разбиения распишем модуль разности сумм Римана,
он в сумме по элементам будет мелкий.
Теперь рассмотрели последовательность разбиений с мелкостью $\to 0$, верен критерий Коши для сумм Римана (с некоторого места разница между двумя любыми мала), есть предел $I$.
И понятно, что этот предел хорошо приближается.

\section{} % 75
Верхняя сумма Дарбу $S_{\tau}(f)$ и нижняя ($s_{\tau}(f)$).
Как интеграл, но в верхней вместо значения в точке оснащения берём супремум, а в нижней "--- инфиум.
Верхняя не меньше нижней, разница равна сумме $\omega(f, E_k)\mu E_k$, интеграл зажат между ними.
Если $f$ ограничена, то интегрируема на $E$ $\iff$ если брать достаточно мелкие разбиения, то суммы Дарбу всегда будут мало отличаться.
Это ясно из билета 74: отличие между суммами Дарбу есть в точности <<интеграл колебаний>>.
Следствие: интерал зажат между суммами Дарбу (взяли разбиение, зажали интеграл между суммами, перешли к пределу по последовательности подчинённых).
Непрерывная функция на измеримом компакте интегрируема, так как она ограничена и равномерно непрерывна (колебания на каждом куске малы).
Упражнение: на открытом измеримом $f$ тоже интегрируема, если непрерывна и ограничена: взяли разбиение, в каждом элементе взяли очень большой компакт, на нём ок, а мера остатка стремится к нулю.

\section{} % 76
Если $E$ измеримо, то $\int_E 1 = \mu E$ ($f(x)=1$).
Если $E \supset E'$ измеримы и $f$ интегрируема на $E$, то интегрируема на $E'$ (достроили, сохраняя мелкость, до разбиения $E$).
Аддитивность: если $E'$ и $E''$ измеримы, $f$ ограничена, то $f$ интегрируема на $E' \sqcup E''$ $\iff$ интегрируема на них обоих
($\Ra$ уже очевидно, $\La$: $E$ точно измеримо, взяли разбиение, пересекли с $E$ и $E''$)
\TODO[доказательство, ограниченность $f$]

\section{} % 77
На нулевой мере интегрируема и интеграл равен нулю (у элементов разбиения мера ноль).
Если сохраняя ограниченность поменять значения $f$ над подмножестве нулевой меры, интеграл не поменяется (по аддитивности).
$f \colon \cl E \to \R$ ограничена $E$ измеримо, тогда интегрируемости на $E$, $\cl E$ и $\int E$ равносильны и интегралы равны.
Линейность: сложили две функции с коэффициентами, интегралы так же сложились (через оценку колебаний и в лоб).

\section{} % 78
$f$, $g$ интегрируемы и ограничены $\Ra$ $fg$ интегрируема (в лоб по колебаниям, дописали слагаемых, вынесли за скобку).
Если еще и $\inf_E {g} > 0$, то $\frac fg$ интеграруема (проверяем ограниченность и интегрируемость $\frac 1g$ колебаниями).
$|f|$ интегрируема и новый интеграл не меньше модуля старого (взяли разбиение, написали неравенств).

\section{} % 79
Если интегрируемы и $f \ge g$, то $\int f \ge \int g$ (взяли последовательность оснащённых разбиений).
Если $G$ открыто, $a \in G$, $f$ неотрицательна, а $f(a)>0$, то $\int f > 0$ (по теореме о стабилизации знака взяли окрестность, взяли функцию $g$ в ячейке из окрестности).

\section{} % 80
Есть последовательность $E_k \subset E$, они измеримы, мера стремится к $E$, $f$ интегрируема и ограничена, тогда $\int_{E_k} f \to \int_E f$ ($\int_E = \int_{E_k} f + \int_{E\setminus E_k} f$, мера $E \setminus E_k$ ограничена, $f$ ограничена).
Пояснение: если $E=\sqcup E_k$ (все измеримы) и частичные суммы мер стремятся к $\mu E$ (иногда не так, привет от кривой Пеано), а также $f$ интегрируема на $E$, то предел интегралов $f$ на префиксах равен $\int_E f$.

\section{} % 81
Пусть $f, g \colon E \to \R$, причём $f, g$ интегрируемы, $g \ge 0$, а $m \le f \le M$.
Тогда существует $\lambda \in [m, M]$ такое, что $\lambda \int_E g = \int_E fg$.
Доказательство: $mg \le fg \le Mg$, проинтегрировали, разделили на $\int_E g$ (если смогли, иначе $\int_E g = 0$ $\Ra$ вообще тривиально).

\section{} % 82
$E$ линейно связно, если любые две точки можно соединить ломаной.
Если есть линейно связный компакт, непрерывная $f$ и интегрируемая $g \ge 0$, то есть $c \in K \colon \int_K fg = f(c) \int_K g$ (по вопросу 81, потом взяли ломаную между минимумом/максимумом, нашли нужное значение $f(c)$).
$E$ просто связно, если нельзя представить в виде дизъюнктного объединения двух множеств, каждое из которых и открыто и замкнуто в $E$ одновременно ($E=два квадратика$ не связно).
Следствие: в линейно связном компакте можно представить $\int_K f = f(c) \mu K$ (связность существенна).
Геометрический смысл интеграла: есть интегрируемая и ограниченная $f \ge 0$, взяли <<подграфик>>, тогда интеграл есть его объём.
Взяли $\epsilon > 0$, взяли мелкое разбиение (а каждое множество из разбиения зажали клеточными так, чтобы суммарная погрешность $< \epsilon$).
Зажали график между клеточными множествами (элемент разбиения $\times$ супремум/инфиум).
Раскрывая скобки в лоб, верхнее клеточное множество оценили сверху как верхнюю сумму Дарбу плюс мелочь (там могут ячейки пересекаться), а нижнее "--- снизу как нижнюю сумму минус мелочь.

\section{} % 83
Есть две функции (одна не больше другой), они заданы на $\cl E$ (измеримо), возьмём график <<между ними>>, это <<элементарное множество>> (обобщение криволинейной трапеции).
Такое множество измеримо: взяли минимум функций, вычли, взяли графики, вычли, потом вернули обратно график нижней (который выкинули с подграфиком).
Мера такого множества есть интеграл разности функций, так как при сдвиге параллельно осям ячейки не меняются $\Ra$ мера не меняется.
Отсюда также следует, что конечное объединение элементарных множеств измеримо.

\section{} % 84
Если мы не в одномерном пространстве, то мера спрямляемой кривой равна нулю.
В $\R^1$ у нас мера равна длине отрезка, а если кривая неспрямляема, то есть безумная кривая Пеано, заметающая весь квадратик меры один.
Доказательство: нарезали на $m$ кусочков равномерно, в границах кусочков взяли ячейки с центром в точке и достаточной стороной, посчитали сумму мер, она стремится к нулю.

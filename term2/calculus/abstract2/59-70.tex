\section{} % 59
Ячейка "--- полуоткрытый параллелепипед, мера "--- её объём.
Мера неотрицательна, аддитивна для дизъюнктного объединения ячеек (разрезали по всем существующим координатам, получили такой кубик, нарезанный на ячейки, при разрезе сумма мер не меняется).

\section{} % 60
Клеточное множество "--- \textit{конечное} дизъюнктное объединение ячеек.
Пересечение "--- пересекли все пары.
Разность "--- нарезали по границам, пересобрали.
$A \cup B = (A \setminus B) \cup B$.
Мера клеточного "--- сумма мер ячеек.
Надо проверить, что если два клеточных множества равны как множества, то меры равны (взяли все попарные пересечения).

\section{} % 61
Монотонность: $A \subset B \Ra \mu 0 \le A \le \mu B$.
Полуаддитивность: $\mu(A\cup B) \le \mu A + \mu B$ (при этом $\mu(A \cup B) = \mu A + \mu B + \mu (A \cap B$).
Аддитивность для дизъюнктных: $\mu(A \cup B) = \mu A + \mu B$.
Мера разности для вложенных: $\mu(B \setminus A) = \mu B - \mu A$.
Доказательства: $3 \Ra 1, 4 \Ra 2, 2'$.

\section{} % 62
Для ограниченных множеств: внешняя мера $E$ есть инфиум $\mu A$, где $A$ "--- клеточное и покрывает $E$.
Внутренняя $\mu_* E$ "--- супремум вложенных клеточных.
Теорема: для ограниченных $0 \le \mu_* E \le \mu^* E < +\infty$, меры монотонны по отдельности, \textit{внешняя} меря полуаддитивна.
Доказательство: рассмотрели пустое; зажали $A \subset E \subset B$ и написали по очереди супремум/инфиум; сверху ограничили какой-нибудь ячейкой (так как $E$ ограничено).
Монотонность: для внутренней супремум пойдёт по б\'ольшему множеству, для внешней "--- аналогично.
Полуаддитивность внешней: взяли два клеточных надмножества, взяли по ним инфиумы.
Если $E$ ограничено и внутренняя мера равна внешней, то оно измеримо по Жордану.

\section{} % 63
Множество, зажатое между двумя одинаковыми параллелепипедами (один открытый, другой замкнутый) измеримо.
Из конечного числа точек "--- измеримо, мера ноль.
Обратные к натуральным "--- меря ноль.
Рациональные точки на $[0,1]$ неизмеримы (внутренняя "--- 0, внешняя "--- 1).
В $\R^2$ горизонтальный отрезок имеет меру ноль (хотя бесконечен).
Связное открытое неизмеримое: занумеровали рациональные числа на $(0,1)$, взяли отрезки $(r_k - \frac{\epsilon}{2^k}, r_k + \frac{\epsilon}{2^k})\cap(0,1)$,
прикреили над ним цилиндры, снизу приклеили полоску для связности.
Оно неизмеримо (внешнюю меру оцениваем, дописыв замыкания; нижнюю "--- спроецировав точки на прямую).

\section{} % 64
Ограниченное измеримо тогда и только тогда, когда $\forall \epsilon > 0$ можно зажать между двумя клеточными такими, что они мало отличаются по мере.
$\Ra$ очевидно, $\La$ по теореме о милиционерах.

\section{} % 65
Расстояние до точки от множества есть инфиум расстояний.
$\delta$-окрестность "--- все точки строго ближе $\delta$.
Если $E$ измеримо, то $\mu^*(U_\delta(E)) \to \mu E$ при $\delta \to 0$ (измеримо и стремится).
Доказали по очереди для $E$-клеток, для клеточных множеств, для измеримых.
Следствие: если $E$ нулевой меры, то есть клеточное $C$ сколько угодно малой меры такое, что $E \subset \Int C$ (потому что взяли окрестность $E$, а потом покрыли клеточным).

\section{} % 66
Граница "--- те точки, что в любом шаре есть точки и из множества, и не из множества.
Можно показать, что $\partial E = \cl E - \Int E$.
Лемма 1: если $x \in E$, $y \notin E$, то на отрезке между ними есть точка с границы (бинпоиском нашли цепь сходящихся отрезков).
Лемма 2: если $D$ "--- клеточное, а $E$ ограничено и $\partial E \subset D$, то $E \cup D$ тоже клеточное.
Доказательство: взяли клетку $Q \supset E \cup D$, вычли $D$, получили клеточное множество $Q \setminus D$.
Взяли оттуда клетку $P_k$. Если она не пересекается с $E$ "--- выкинули, иначе клетка вложена в $E$ (так как если есть точка не из $E$, то по предыдущей лемме можно найти граничную точку внутри $P_k$, а такие выкинули вместе с $D$).
То есть $E \cup D = D \cup \bigcup P_k$ (с учётом выкинутых), что и требовалось (доказывается двумя включениями).

\section{} % 67
Если $E$ ограничено, то оно измеримо тогда и только тогда, когда граница имеет нулевую меру.
$\Ra$: зажали между клеточными (разность их мер "--- $\epsilon$), граница есть $\cl \setminus \Int$, это зажато между $\cl B$ и $\Int A$.
Первое из них оценим другим клеточным множеством $B'$ сверху (с маленькой ошибкой, так как измеримо), с $\Int A$ аналогично снизу построим $A'$.
Тогда $\mu (B' \setminus A') < 3\epsilon$ (и разность клеточная), а в него зажата граница, то есть граница меры ноль.
$\La$: возьмём покрывающее границу клеточное множество $C_{\epsilon}$ маленькой меры.
Тогда по лемме 2 $E \cup C_{\epsilon}$ клеточное, зажмём $E$ между $B_{\epsilon} E \cup C_{\epsilon} \supset E \supset B_{\epsilon}\setminus C_{\epsilon} = E \setminus C_{\epsilon}$, измерилось.

\section{} % 68
Граница объединения/пересечения/разности есть объединение/... границ.
Тогда если $E$ и $F$ измеримы, то измеримы замыкания, внутренности, пересечения, объединения, разности.

\section{} % 69
Монотонность (знаем для верхней меры), полуаддитивность (знаем для верхней меры), аддитивность для дизъюнктных (взяли два клеточных подмножества, пусть не пересекаются, выписали неравенство, перешли к супремумам).
Аддитивности отдельно по верхней/нижней нет, возьмём рациональные и иррациональные на $[0,1]$.

\section{} % 70
$f$ "--- непрерывная функция на компакте из $\R^n$ в $\R$, её график имеет меру ноль (в $\R^{n+1}$, не путайтесь, где работаете).
Доказательство: она равномерно непрерывна, весь компакт содержится в ячейке, распилили его на кучу ячеек диаметром не больше $\delta$.
Взяли часть графика, которая лежит над ячейкой $P_k$, по равномерной непрерывности её можно зажать в полоску толщиной $2\epsilon$ (фиксируем произвольную $x_k$ оттуда, остальные по высоте отличаются несильно),
считаем суммарный объём таких полосок, он не больше $2\epsilon \mu P$, то есть мера графика меньше любого $\epsilon \cdot C$, то есть ноль.

Измеримы все криволинейные трапеции вида <<взяли точки над компактов высоты от 0 до $f(x)$>> ($f(x) \ge 0$ и непрерывна).
Покажем, что граница трапеции $G_f$ имеет меру ноль, она состоит из кусков: график функции, компакт, торчащие вверх цилиндры над $\partial K$.
График имеет меру ноль, компакт "--- график функции $f=0$ (тоже ноль), границу компакта зажмём клеточным множеством маленькой меры, вырастим над ним цилиндр,
так как $f$ ограничена, то высота цилиндра ограничена, то есть мера цилиндров над границей ноль (потому что содержится в клеточном множестве $\times (-1, \max f]$).

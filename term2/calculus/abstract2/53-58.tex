\section{} % 53
Есть линейный оператор $A \colon \R^n \to \R^n$.
$A$ обратим $\iff$ $A(\R^n)=\R^n$ $\iff$ уравнение $Ax=0$ имеет только нулевое решение $\iff$ $\det A \neq 0$
Переписали в терминах СЛУ, всё стало очевидно.

Если есть $m>0$ такое, что $\|Ax\| \ge m\|x\|$, то $A$ обратим и $\|A^{-1}\| \le \frac 1m$.
Обратимость по теореме: $Ax=0$ решается только тривиально; потом взяли норму (инфиум по подходяшим константам), заменили $A^{-1}x$ на $y$, а $x$ на $Ay$, получили норму исходную.

Обратимость близких к обратимым: есть обратимый линейный $A$, $\|A^{-1}\|=\frac 1 \alpha$ (про $\|A\|$ не знаем).
Тогда если $B \colon \R^n \to \R^n$ линеен и $\|B-A\| = \beta$ и $\beta < \alpha$, то он обратим и разность норм обратных не больше $\frac{\beta}{\alpha(\alpha-\beta)}$.
Найдём ограничение снизу на $\|Bx\|$ для всех $x$ и получим обратность: написали $Ax=Bx+(A-B)x$, вынесли из-под модуля, заменили на $\beta$, получили $m=\alpha-\beta$.
Второй шаг: $B^{-1}-A^{-1}=B^{-1}(A-B)A^{-1}$, норма произведения не больше произведения норм.

Следствие: $f(B)=B^{-1}$ непрерывна на множестве обратимых операторов.
Взяли последовательность $B_k \to A$, нашли $\delta$ по $\epsilon$ из теоремы.

\section{} % 54
Пусть $f$ непрерывно дифференцируема в окрестности $a$, $\d_af$ обратим и $b=f(a)$, тогда
у точек $a$ и $b$ есть окрестности $U$ и $V$ такие, что $f$ там биективна,
а также обратная к $f$ в этих окрестностях дифференцируема (зовём её $g$).
Если всё так, то $\d_b g = (\d_a f)^{-1}$.

Шаг 1: пусть $A=\d_a f$, пусть $\lambda = \frac{1}{4\|A^{-1}\|}$, в качестве $U$ возьмём шар с центром в $a$ такой, что норма разность дифференциалов в $x$ и в $a$ не больше $2\lambda$.
Покажем инъективность, пусть есть точки $x$ и $x+h$, отрезок $[x,x+h]$ лежит в шаре, сделали функцию $F(t)=f(x+th)-tAh$.
$\|F(1)-F(0)\|=\|F'(t)\|$ для некоторого $t$, оценили эту разность сверху как $2\lambda \|h\| = \frac{\|Ah\|}{2}$.
Теперь оценим $\|f(x+th)-f(x)\|$ снизу: прибавим и вычтем $Ah$, с плюсом вынесется за скобку, вычитаемый остаток уже оценили, получили оценку снизу в $\frac{\|Ah\|}{2}$, успех.

Шаг 2: взяли точку $x_0$ с шариком $B_r(x_0) \subset U$, покажем, что $B_{\lambda r}(f(x_0)) \subset f(U)$.
Взяли оттуда $y$, $\|y - f(x_0)\| < \lambda y$, найдём прообраз $y$ из $U$.
Положим $\phi(x) = \|y - x\| \colon B_r(x_0) \to \R$ (хотим найти корень).
Покажем, что минимум $\phi$ (и $\phi^2$) не на границе: оценили $2\lambda\|x - x_0\| < \phi(x) + \phi(x_0)$.
Взяли $x_1$ на границе шарика, $2\lambda r \le \phi(x_1) + \phi(x_0) \le \phi(x_1) + \lambda r$, отсюда $\phi(x_1) > \lambda r > \phi(x_0)$, в центре меньше.
Взяли частную производную $\phi^2$, приравняли нулю, получили СЛУ с матрицей $(\d_x f)^T$, она обратима $\Ra$ нетривиальных решений нет $\Ra$ нашли корень.

\section{} % 55
Взяли $y, y + k \in V = F(U)$, покажем $g(y) - g(y+k) = A^{-1}k + o(\|k\|)$.
Посмотрели на $x=g(y)$, $x+h=g(y+k)$, сказали $k=f(x+h)-f(x)=\d_xf(h) + r(h)$.
Взяли $B=(\d_xf)^{-1}$, $Bk=B(\d_xf(h))+B(r(h))$, выразили $h$, устремили $k \to 0$, показали малость $B(r(h))$.
Помним, что умеем оценивать $\|k\| = \|f(x+h)-f(x)\| > 2\lambda\|h\|h$.

$\d_x f$ непрерывна, $(\d_x f)^{-1}=\d_{f(x)} g$ непрерывна, $\d_y g = (\d_{g(y)} f)^{-1}$ тоже непрерывна как композиция непрерывных.

Следствие: если $f$ непрерывно дифференцируема на открытом $D$, то если $W \subset D$ открыто, то $f(W)$ тоже открыто.
Представили $W$ как кучу шариков, по второму шагу образ открытого открыт.

\section{} % 56
Есть функция $f \colon D \subset \R^{n+m} \to \R^n$, $D$ открыто, взяли $A=\d_{a,b} f$ в точке $f(a, b)=0$.
Тогде если $A$ удовлетворяет условию $A(h, 0) = 0 \Ra h = 0$, то есть окрестность $W$ точки $b$ и функция $g \colon W \to \R^n$ такая,
что $f(g(y), y) = 0$ и $g(b)=a$.
Наша $f$ "--- это ограничение на график, первые координаты "--- значение неявной, вторые "--- исходная точка.
Пример: окружность, везде ок, кроме как на оси OY.

Предыстория: если есть линейный оператор $A$ с таким свойством, то уравнение $A(x, k)=b$ имеет ровно одно решение при любых $k \in \R^m$, $b \in R^n$ (расписали покоординатно, получили СЛУ, определитель не ноль).

Доказательство теоремы: возьмём $F(x, y) = (f(x, y), y)$ и проверим у неё условие теоремы об обратной функции в точке $(a, b)$.
Пусть $f(a+h, b+k)-f(a,b)=A(h, k) + r(h, k)$, $F(a+h,b+k)-F(a,b)=B(h, k)$, тогда проверили обратимость $B$ (единственность решения однородного: $k$ сразу ноль, а по условию на $A$ и $h$ тоже ноль).
Тогда есть окрестность точки $F(a, b)=(0,b)$, где есть биекция, взяли там обратную функцию $G$.
Срезали $G$ так, чтобы первые координаты занулились.

\section{} % 57
$a$ "--- строгий локальный максимум $f \colon D \subset \R^{n+m} \to \R$ при условии $\Phi = 0$ ($\Phi \colon D \to \R^m$), если
в некоторой окрестности, пересечённой с $\Phi=0$ он строгий локальный максимум.

Метод множителей Лагранжа: если $f$ и $\Phi$ непрерывно дифференцируемы в $D$, а $a$ "--- условный экстремум,
то $\nabla f$, $\nabla (\Phi_1)$, \dots линейно зависимы.
А если $\nabla \Phi_i$ сами по себе линейно зависимы, то у нас беда и скука.
Иначе $\nabla f = \lambda_1 \nabla \Phi_1 + \dots$, где $\lambda_i$ суть множители Лагранжа.
Получили систему из $n+m$ условий на координаты градиентов и $m$ условий на $\Phi=0$, переменных тоже $n+2m$: $m$ множителей и $n+m$ координат точки
(впрочем, система обычна нелинейна).
Если выписать матрицу Якоби для $\Phi$, то градиенты $\nabla \Phi_i$ "--- её строки, их меньше столбцов $(m < n + m$), т.е. независимость $\nabla \Phi_i$ $\iff$ ранг максимален.

Доказательство метода: если $\nabla \Phi_i$ линейно зависимы, то доказывать нечего.
Пусть $a=(b, c)$, $A=\d_a \Phi$ и, соответственно, $\rang A = m$.
Возьмём определитель каких-нибудь $m$ столбцов матрицы Якоби, какой-нибудь "--- не ноль.
Пусть эти столбцы в конце (иначе переставим), тогда проверим условие теоремы об обратной функции: если $A(0_n, h) = 0_m$, то $h=0_m$, так как последние столбцы образуют невырожденную систему.
Тогда знаем, что есть $W$ "--- окрестность $b$ и функция $g$ такие, что $\Phi(x, g(x))=0$ и $g(b)=c$.
Если $(b, c)$ "--- условный локальный минимум, то $b$ "--- безусловный локальный минимум $H(x) = f(x, g(x))$ во внутренности $W$.
Значит $\nabla H(b) = 0$, расписали его через частные производные $f$ и $g$.
Аналогично расписали $\nabla \Phi_i$, они тоже нули.
Сложили с коэффициентами, вычли из $\nabla H(b)$, всё ненужное ($g_i$) сократилось, эту систему можно решить (т.к. определитель не ноль).

\section{} % 58
Взяли квадратичную форму $A$ (значение в точке $x$ равно $\langle Ax, x \rangle$), найдём экстремумы на единичной сфере, одно ограничение $\Phi$, применили метод множителей,
получили обязательное условие: $Ax = \lambda x$, где $\lambda$ "--- собственное число, а $x$ "--- отнормированный собственный вектор.
Подставили его в квадратичную форму, она выдала значение $\lambda$, то есть максимум/минимум "--- это максимальное/минимальное собственное число, достигаются в соответствующих векторах.

Утверждение: $\langle Ax, y \rangle = \langle x, A^Ty \rangle$ (достаточно проверить для базиса, так как всё линейно или же $\langle Ax, y\rangle = (Ax)^Ty = \dots$).
Отсюда выводим, что квадрат нормы оператора $A$ есть максимум квадратичной формы $A^TA$ на сфере, то есть корень из макс. собственного числа $A^TA$ ($\|x\|^2 = \langle x, x \rangle$).

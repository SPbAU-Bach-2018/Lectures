\section{} % 42
Взяли единичный $\vec h$, сузили функцию на прямую, взяли производную "--- это по направлению.
$\partd{f}{\vec h} (a) = \d_af(h)$.
Если функция скалярная: $\partd{f}{\vec h} = \d_af(h) = (\nabla f(a)) \cdot h$.
Т.е. производная по направлению по модулю меньше модуля градиента в $a$, равенство только в случае $h \parallel \nabla f(a)$ по неравенству Коши-Буняковского (там равенство только если вектора параллельны).

\section{} % 43
Частная производная "--- $\partd{f}{e_i}$ ($e_i$ "--- базисный вектор).
Переформулировка: зафиксировали все координаты, кроме одной, взяли производную.
Для скалярной функции частная по $e_k$ равна $k$-й координате градиента.
Векторная "--- это несколько независимых скалярных, отсюда знаем матрицу Якоби целиком.

\section{} % 44
Если дифференцируемо, то частные производные есть.
Но не обратно: пусть $f$ единица везде, кроме координатных осей (а там ноль), тогда частные есть, нет даже непрерывности.
По всем направлениям тоже не хватает: берём дугу единичной окружности (с выколотым началом), частные везде 0, непрерывности нет.
Надо требовать существование \textit{и непрерывность} частных производных в \textit{окрестности} $a$, тогда будет дифференцируемость.
Доказательство: надо оценить $f(a+h)-f(a)-\langle \nabla f(a), h\rangle$, расписываем скалярное произведение,
вводим $\vec b_k$ и $\vec h_k$ (каждый "--- сдвиг по одной координате от предыдущего), разность значений в соседних есть частная производная
в какой-то точке из окрестности, написали Лагранжа (разность значений есть производная где-то на длину), остаток есть скалярное $h$ на хрень из разностей частных производных в точках.
Написали Коши-Буняковского, сказали, что каждое слагаемое стремится к нулю, так как частные производные непрерывны.

\section{} % 45
Пример: $x^2+y^2$ если ровно одна координата рациональна, $0$ иначе.
Дифференцируема в нуле, но частные производные есть только в нуле, никакой непрерывности.
Если дифференциал есть и непрерывен (норма соседних близка к нули), то у нас непрерывная дифференцируемость $\iff$ частные производные есть и непрерывны везде.
$\Ra$: частные производные точно есть, покажем непрерывность: разность дифференциалов не больше скалярного произведения нормы разности дифференциалов и $\|e_k\|$.
$\La$: дифференцируемость точно есть.
Разность дифференциалов "--- это отображение, норма не больше корня из суммы квадратов коэффициентов, расписали сумму, каждое слагаемое стремится к нулю по непрерывности частных.

\section{} % 46
Ввели $g=\partd{f}{e_i}$ в окрестности $a$, взяли её частную производную "--- второй порядок.
Обычно порядок неважен, но не всегда: $xy\cdot\frac{x^2-y^2}{x^2+y^2}$ (а в нуле "--- ноль).
Если обе частные от функции $f: D \subset \R^2 \to R$ существуют в окрестности и непрерывны в $a=(x_0, y_0)$, то порядок неважен.
Доказательство: дважды Лагранжем, потом в обратном порядке, сошлось.

\section{} % 47
Пусть $D \subset \R^n \to \R$ открыто, функция непрерывно дифференцируема $r$ раз.
Тогда можно переставлять индексы при частных производных $r$-го порядка.
Достаточно доказать теорему для транспозиции $k \leftrightarrow k+1$: взяли первые $k-1$ производную (это будет новая функция),
у неё можно брать две в любом порядке (почти успех), потом добавили все производные после $k+2$.
Пример есть в вопросе 46.

\section{} % 48
Мультииндекс "--- вектор из неотрицательных, высота "--- сумма чисел, факториал "--- перемножили поэлементно.
$h^k$ (где $h$ "--- вектор, $k$ "--- мультииндекс) берём почленно, считаем $0^0 = 1$.
Производная порядка мультииндекса: $f^{(k)}(x) = \frac{\partial^{|k|} f}{\partial x_1^{k_1} \partial x_2^{k_2} \dots}$.
Лемма: есть функция $f \colon D \subset \R^n \to R$, $D$ открыто, функция $r$-гладкая (есть непрерывные частные производные порядка $r$).
Есть отрезок $[x_0, x_0+h] \subset D$, взяли функцию $F(t) = f(x_0+th)$ (композиция линейной $x_0+th$ и гладкой $f$),
тогда её $r$-я производная в точке $t$ известна (это почти частная производная $f$ по направлению $h$).

Равна сумме по мультииндексам $k$ высоты $r$: $r! \sum_{|k|=r} \frac{f^{(k)}(x_0+th)}{k!}h^k$.
В двумерном случае получили почти бином Ньютона (вместо степени "--- частные производные).
Доказательство индукцией по $r$, при $r=0$ вообще получили определение, в переходе надо продифференцировать еще раз.
Аккуратно с мультииндексами "--- они могут вылезать несколько раз из разных мест.

\section{} % 49
Если есть $(r+1)$-гладкая функция и отрезок $[x,x_0]$, то есть формула Тейлора степени $r$ с мультииндексами:
$f(x) = \sum_{|k|\le r} \frac{f^{(k)}(x_0)}{k!}(x-x_0)^k + \sum_{|k|=r+1}\frac{f^{(k)}(x_0+\theta(x-x_0))}{k!}(x-x_0)^k$.
Доказательство: взяли функцию $F(t)=f(x_0+t(x-x_0))$, для неё написали Тейлора степени $r$ в форме Лагранжа (найдя какое-то $\theta$), подставили частные производные $f$ по направлению $h=x-x_0$ с мультииндексами.

При $r=0$ получаем многомерную формулу Лагранжа: $f(x_0 + h)=f(x_0) + \langle \nabla f(x_0 + \theta h), h \rangle$.

\section{} % 50
Остаток заменили на $o(\|h\|^r)$ при $\|h\| \to 0$.
Выписали предыдущую формулу \textit{порядка $r-1$}, потом в первой сумме добавили мультииндексы высоты $r$, а справа их вычли, оцениваем разность справа.
Разность производных стремится к нулю, а $\frac{h^k}{\|h\|^k}$ ограничено единицей.

Полиномиальная формула: $\left(\sum x_i\right)^r = \sum_{|k|=r} \frac{r!}{k!} x^k$.
Не по индукции, а из Тейлоре по этой формуле.
Ввели функцию $f(x) = \left(\sum x_i\right)^r$, нашли частную производную по $x_i$ порядка $l$, посмотрели на неё в точке $x=\vec 0$.
При $l\neq r$ она занулится, а при $l=r$ получится равной $r!$, т.е. остаток в форме Лагранжа занулится, а вот основные слагаемые дадут что надо.

\section{} % 51
Минимум, если в открытой окрестности нет значений больше.
Строгий минимум, если в открытой проколотой окрестности все больше (и тут, и раньше пересекаем окрестности с областью определения).
Если $a$ "--- экстремум (минимум/максимум) и есть частная производная, то она равна нулю.
Т.е. если дифференцируема, то градиент ноль и все частные по направлениям тоже нули.
Доказательство: сузили на прямую (в окрестности), на ней тоже максимум, одномерная теорема есть.
Точка стационарна, если $\nabla f(a)=0$.

\section{} % 52
Есть симметричная матрица $C$ размера $n \times n$, квадратичная форма $Q(h) \colon \R^n \to \R$ есть $Q(h)=\sum c_{ij}h_ih_j$.
Строго положительно определена, если всегда больше нуля (кроме нуля, в котором ноль).
Критерий Сильвестра для положительноопределённости: все определители верхних левых углов матрицы ($n$ штук) строго больше нуля,
для нестрогой положительности "--- нестрого.
Для отрицательной надо рассмотреть $-C$, тогда у нечётных определителей надо будет требовать другой знак.
Лемма: если строго положительно определена, то $\exists c>0\colon Q(h) \ge c \|h\|^2$ (взяли на единичной сфере, нашли минимум).

Пусть $f$ дважды непрерывно дифференцируема в окрестности, возьмём матрицу вторых производных (Гессиан), посмотрим на $Q(h)$.
Если строго положительно/отрицательно определена, то $a$ "--- строгий максимум/минимум.
Если нестрогий минимум/максимум, то нестрого определена.
Только так, иногда могут быть проблемы (нестрого определена, экстремума нет), как и в одномерном случае, тогда головой работать.
Доказательство: если положительно определена, то нашли $c>0$, расписали разность по Тейлору, при малых $h$ имеем $\frac c2 + o(1)>0$;
если нестрогий минимум, то взяли фиксированное $h$, расписали $f(a+th)-f(a)\ge 0$ при малых $t$, перешли к пределу по $t$, получили нужное.

\setauthor{Надежда Бугакова}

\begin{exmp}
	\begin{gather*}
		\int\limits_\alpha^\beta \frac{t}{1+t^4} \d t = \\
		f(x) = \frac1{x^2} \quad \phi(t) \lrh t^2 \\
		= \int\limits_\alpha^\beta \frac12 f(\phi(t))\phi'(t) = \int\limits_{\phi(\alpha)}^{\phi(\beta)} f(x) \d x = \frac12 \int_{\alpha^2}^{\beta^2} \frac{1}{1+x^2} \d x = \frac12 (\arctg t)\bigr|_{\alpha^2}^{\beta^2} = \frac{\arctg \beta^2 - \arctg \alpha^2}2
	\end{gather*}
\end{exmp}

\begin{exmp}
	\begin{gather*}
		W_n = \int\limits_0^{\pi / 2} \sin^n x \d x = \int\limits_0^{\pi / 2} \cos^n \d x \\
		\int\limits^{\pi/2}_0 \cos^{n} x \d x = \int \limits_0^{\pi/2} \cos^{n - 1}x(\sin x)'\d x =
		\cos^{n - 1}x\sin x \biggr|_{0}^{\pi/2} - \int\limits_0^{\pi/2}(\cos^{n - 1}x)'\sin x \d x = \\
		= -\int\limits_0^{\pi/2}(\cos^{n - 1}x)'\sin x \d x = (n - 1)\int\limits_0^{\pi/2}\cos^{n - 2}x\sin^2x\d x = \\
		= (n - 1)\int\limits_0^{\pi/2}\cos^{n - 2}x(1 - \cos^2x)\d x = (n - 1)(W_{n - 2} - W_n)
		W_n = \frac{n - 1}{n}W_{n - 2} \\
		\begin{aligned}
			W_0 &= \frac\pi2 \\
			W_1 &= 1 \\
			W_{2n} &= \frac{2n - 1}{2n} \cdot \frac{2n - 3}{2n - 2} \cdot \frac{2n - 5}{2n - 4} \cdot \dots \cdot \frac{1}{2} \cdot \frac\pi2 = \frac{(2n - 1)!!}{(2n)!!} \cdot \frac\pi2 \\
			W_{2n + 1} &= \frac{2n}{2n + 1} \cdot \frac{2n - 2}{2n - 1} \cdot \dots \cdot \frac{2}{3} \cdot 1 = \frac{(2n)!!}{(2n + 1)!!}
		\end{aligned}
	\end{gather*}
	\begin{assertion}[Формула Валлиса]
		\[ \lim_{n \ra \infty} \frac{(2n)!!}{(2n - 1)!!}\cdot\frac{1}{\sqrt{2n + 1}} = \sqrt{\frac\pi2} \]
	\end{assertion}
	\begin{proof}
		\begin{gather*}
			\cos^{2n} x \ge \cos^{2n + 1}x \ge \cos^{2n + 2}x \quad\text{ при $x\in[0, \pi/2]$} \\
			W_{2n} \ge W_{2n + 1} \ge W_{2n + 2} \\
			\frac{(2n - 1)!!}{(2n)!!}\cdot\frac\pi2 \ge \frac{(2n)!!}{(2n+1)(2n - 1)!!} \ge \frac{(2n + 1)!!}{(2n + 2)!!}\frac\pi2 \\
			\frac\pi2 \ge \frac{((2n)!!)^2}{(2n + 1)((2n - 1)!!)^2} \ge \frac{2n + 1}{2n + 2} \cdot \frac\pi2 \ra \frac\pi2 \\
			\lim_{n \ra \infty} \frac{((2n)!!)^2}{(2n + 1)((2n - 1)!!)^2} = \frac\pi2
		\end{gather*}
	\end{proof}
	\begin{conseq}
		$C_{2n}^{n} \sim \frac{4^n}{\sqrt{\pi n}}$
	\end{conseq}
	\begin{proof}
		\begin{gather*}
			(2n)!! = 2^nn! \quad (2n-1)!! (2n) !! = (2n)! \\
			\frac{(2n)!!}{(2n - 1)!!} = \frac{((2n)!!)^2}{(2n)!} = \frac{(2^nn!)^2}{(2n)!} = 4^n\frac{n!n!}{(2n)!} = \frac{4^n}{C_{2n}^n} \\
			\frac{4^n}{C_{2n}^n}\cdot\frac{1}{\sqrt{2n + 1}} \sim \sqrt{\frac\pi2} \\
			C_{2n}^{n} \sim \frac{4^n}{\sqrt{\pi/2}}\cdot\underbrace{\frac{1}{\sqrt{2n + 1}}}_{\sim\frac1{\sqrt{2n}}} \sim \frac{4^n}{\sqrt{\pi n}} \Ra
				C_{2n}^n \sim \frac{4^n}{\sqrt{\pi n}}
		\end{gather*}
	\end{proof}
\end{exmp}

\begin{exmp}
	\[ H_n = \frac{1}{n!}\int\limits_0^{\pi/2} \left(\left(\frac\pi2\right)^2 - x^2\right)^n\cos x\d x \]
\end{exmp}
\begin{enumerate}
\item
	$H_n \ra 0$
	\[ 0 < H_n \le \frac{1}{n!}\frac\pi2\max_{x \in [0, \pi/2]} \left(\left(\frac\pi2\right)^2 - x^2\right)^n\cos x \le \frac{1}{n!} \frac\pi2 \left(\frac\pi2\right)^{2n} \ra 0 \]
	Так как показательная функция растёт медленнее факториала.

\item
	$H_n = (4n - 2)H_{n - 1} - \pi^2H_{n - 2}$, $H_0 = 1$, $H_1 = 2$.

	\underline{Пояснение:}
	\begin{gather*}
		n!H_n = \int\limits_{0}^{\pi/2}{({(\pi/2)}^2 - x^2)}^n(\sin x)' \d x = \\
		= (\text{Подстановка, которая равна 0}) - \int\limits_{0}^{\pi/2}({({(\pi/2)}^2 - x^2)}^n)'\sin x \d x = \\
		= -2n\int\limits_{0}^{\pi/2}{({(\pi/2)}^2 - x^2)}^{n - 1}x(\cos x)' \d x
	\end{gather*}
	Затем ещё раз интегрируем по частям, используя равенство $x^2 = {(\pi/2)}^2 - ({(\pi/2)}^2 - x^2)$
	\item $H_n = P_n(\pi^2)$, где $P_n$ - многочлен $n$-ой степени с целыми коэффициентами.
\end{enumerate}

\begin{theorem}
	$\pi$ и $\pi^2$ иррациональны.
\end{theorem}
\begin{proof}
	От противного. Пусть $\pi = \frac{a}{b}$.
	\begin{gather*}
		H_n = P_n\left(\frac{a^2}{b^2}\right) \Ra b^{2n}H_n = b^{2n}P_n\left(\frac{a^2}{b^2}\right) - \text{целое число.} \\
		H_n > 0 \Ra b^{2n}H_n > 0 \\
		1 \le b^{2n}H_n \le b^{2n}\frac\pi2\left(\frac\pi2\right)^{2n} \frac{1}{n!} = \frac\pi2\frac{{\left(b\cdot\pi/2\right)}^{2n}}{n!} \ra 0
	\end{gather*}
\end{proof}

\begin{theorem}[Формула Тейлора с остатком в интегральной форме]
	$f \in C^{n + 1}\left<a, b\right>$, $x_0 \in \left<a, b\right>$.
	Тогда
	\[ f(x) = T_{n, x_0}f(x) + \frac{1}{n!}\int\limits_{x_0}^x (x - t)^n f^{(n + 1)}(t) \d t \]
\end{theorem}
\begin{proof}
	Докажем по индукции.
	\begin{description}
	\item[База $n = 0$:]
		\[f(x) = f(x_0) + \int\limits_{x_0}^{x}f'(t)\d t = f(x_0) + f(x) - f(x_0) \]

	\item[Переход:]
		\begin{gather*}
			f(x) = T_{n, x_0}f(x) + \frac{1}{n!}\int\limits_{x_0}^x{(x - t)}^nf^{(n + 1)}(t)\d t = T_{n, x_0}f(x) - \frac{1}{(n + 1)!}
			\int\limits_{x_0}^xf^{(n + 1)}(t)({(x - t)}^{n + 1})'\d t = \\
			= T_{n, x_0}f(x) - \left(\frac{f^{(n + 1)}(t)}{(n + 1)!}(x - t)^{(n + 1)}\biggr|_{x_0}^{x} - \frac{1}{(n + 1)!}\int\limits_{x_0}^x(x - t)^{n + 1}f^{n + 2}(t)\d t \right) = \\
			= T_{n, x_0}f(x) + \frac{f^{(n + 1)}x_0}{(n + 1)!}(x - x_0)^{(n + 1)} + \frac{1}{(n + 1)!}\int\limits_{x_0}^x{(x - t)}^{n + 1}f^{(n + 2)}(t)\d t
		\end{gather*}
	\end{description}
\end{proof}

\section{Интегральные суммы.}
\begin{Def}
	$f \colon X \ra Y$, $X, Y$ "--- метрические пространства.
	\[ \omega_f(\delta) = \sup\{\rho_y(f(x_1), f(x_2)) \mid \rho_x(x_1, x_2) < \delta\} \]
	$\omega_f(\delta)$ "--- модуль непрерывности.
\end{Def}

\underline{Свойства:}
\begin{enumerate}
\item
	$\omega_f(\delta) \ge 0$

\item
	$\omega_f(0) = 0$

\item
	$\omega_f(\delta)\nearrow$

\item
	$f$ равномерно непрерывно на $X$ тогда и только тогда, когда $w_f$ непрерывна в 0.
	\begin{proof}
		\begin{description}
		\item[$\Ra$:]
			$f$ равномерно непрерывно на $X$:
			\begin{gather*}
				\forall \epsilon > 0, \exists \delta > 0 \colon \forall x_1, x_2 \in X\colon \rho_x(x_1, x_2) \le \delta, \rho_y(f(x_1), f(x_2)) < \epsilon \Ra \\
				\Ra \underbrace{\sup\{\rho_y(f(x_1, f(x_2)) \mid \rho_x(x_1, x_2) \le \delta\}}_{= w_f(\delta)} \le \epsilon \\
				\forall \epsilon > 0, \exists \delta > 0 \colon w_f(\delta) \le \epsilon \Ra 0 \le w_f(\gamma) \le w_f(\delta) \le \epsilon \quad\text{при $\gamma \le \delta$}
			\end{gather*}
			Таким образом $w_f$ непрерывна в 0.

		\item[$\La$:]
			\begin{gather*}
				\forall \epsilon < 0, \exists \delta > 0 \colon w_f(\delta) < \epsilon \Ra \\
				\Ra ((\rho_x(x_1, x_2) \le \delta) \Ra (\rho_y(f(x_1), f(x_2) < \sup\{\rho_y(f(x_1), f(x_2)) \mid \rho_x(x_1, x_2) \le \delta\} < \epsilon))
			\end{gather*}
		\end{description}
	\end{proof}

\item
	Если $X$ "--- компакт, то $f$ непрерывна на $X$ тогда и только тогда, когда $w_f$ непрерывна в 0.
\end{enumerate}

\begin{Def}
	$[a, b]$. Дробление отрезка (иногда называют пунктиром):
	\[ \tau\colon x_0 = a < x_1 < x_2 < \dots < x_n = b \]
	Мелкость дробления (ранг дробления):
	\[ |\tau| = \max_{k=1..n} \left(x_k - x_{k - 1}\right) \]
	Оснащение дробления:
	\[ \xi_1, \xi_2, \dots, \xi_n \quad \xi_k \in [x_{k - 1}, x_k] \]
\end{Def}

\begin{Def}
	Сумма Римана: $f\colon [a, b] \ra \R$
	\[ \sigma(f, \tau, \xi) = \sum_{k = 1}^n f(\xi_k) \left(x_k - x_{k - 1}\right) \]
\end{Def}

\begin{theorem}
	$f \in C[a, b]$.
	\[ \forall \epsilon > 0, \exists \delta > 0\colon \forall (\tau, \xi), |\tau| < \delta \Ra \left|\sigma(f, \tau, \xi) - \int\limits_a^bf(x)\d x\right| < \epsilon \]
	В частности, если $(\tau_n, \xi_n)$ "--- последовательность дроблений такая, что $|\tau_n| \ra 0$, то
	\[ \sigma(f, \tau_n, \xi_n) \ra \int\limits_a^bf(x)\d x \]
\end{theorem}
\begin{proof}
	\begin{gather*}
		\Delta = \sigma(f, \tau, \xi) - \int\limits_a^bf(x)\d x = \sum\limits_{k = 1}^nf(\xi_k)(x_k - x_{k - 1}) - \sum\limits_{k = 1}^n
		\int\limits_{x_{k - 1}}^{x_k}f(x) \d x = \\
		= \sum\limits_{k = 1}^n\left(\int\limits_{x_{k - 1}}^{x_k}f(\xi_k)\d x - \int\limits_{x_{k - 1}}^{x_k} f(x) \d x\right) =
		\sum\limits_{k = 1}^n\int\limits_{x_{k - 1}}^{x_k}(f(\xi_k) - f(x))\d x \\
		|\Delta| \le \sum\limits_{k = 1}^n\left|\int\limits_{x_{k - 1}}^{x_k}(f(\xi_k) - f(x))\d x \right| \le \sum\limits_{x = 1}^n
		\int\limits_{x_{k - 1}}^{x_k}|f(\xi_k) - f(x)|\d x \le \\
		\stackrel{|\xi_k - x| < \tau}{\le} \sum\limits_{k = 1}^n\int\limits_{x_{k - 1}}^{x_k}w_f(|\tau|)\d x = \sum\limits_{k = 1}^nw_f(|\tau|)(x_k - x_{k - 1}) \\
		|\Delta| \le w_f(|\tau|)(b - a) < \epsilon(b - a)
	\end{gather*}
\end{proof}

\begin{Def}
	$f:[a, b] \ra \R$.
	$f$ интегрируема по Риману, если
	\[ \exists I \colon \forall \epsilon, \exists \delta > 0\colon \forall |\tau| < \delta |\sigma(f, \tau, \xi) - I| < \epsilon \]
	и $I$ называется интегралом Римана.
	\[ \int\limits_a^bf(x)\d x \]
\end{Def}
\begin{Rem}
	В этом определении не обязательна непрерывность функции, но для не непрерывных будет непонятно, существует ли интеграл.
\end{Rem}

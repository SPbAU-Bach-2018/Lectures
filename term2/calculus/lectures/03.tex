2) Посчитаем сам итеграл.

\begin{gather*}
	\int\limits^{\pi/2}_0 \cos^{n} x \d x = \int \limits_0^{\pi/2} \cos^{n - 1}x(\sin x)'\d x =
	\cos^{n - 1}x\sin x \biggr|_{0}^{\pi/2} - \int\limits_0^{\pi/2}(\cos^{n - 1}x)'\sin x \d x = \\
	= -\int\limits_0^{\pi/2}(\cos^{n - 1}x)'\sin x \d x = (n - 1)\int\limits_0^{\pi/2}\cos^{n - 2}x\sin^2x\d x = \\
	= (n - 1)\int\limits_0^{\pi/2}\cos^{n - 2}x(1 - \cos^2x)\d x = (n - 1)(W_{n - 2} - W_n)
\end{gather*}
\[
\begin{aligned}
	&W_n = \frac{n - 1}{n}W_{n - 2}\\
	&W_0 = \pi/2 \\
	&W_1 = 1  \\
	&W_{2n} = \frac{2n - 1}{2n} \cdot \frac{2n - 3}{2n - 2} \cdot \frac{2n - 5}{2n - 4} \cdot \dots \cdot \frac{1}{2} \cdot \pi/2 
	= \frac{(2n - 1)!!}{(2n)!!} \cdot \pi/2 \\
	&W_{2n + 1} = \frac{2n}{2n + 1} \cdot \frac{2n - 2}{2n - 1} \cdot \dots \cdot \frac{2}{3} \cdot 1 = \frac{(2n)!!}{(2n + 1)!!} \\
	&\cos^{2n} x \ge \cos^{2n + 1}x \ge \cos^{2n + 2}x\text{ при }x\in[0, \pi/2] \\
	&W_{2n} \ge W_{2n + 1} \ge W_{2n + 2} \\
	&\frac{(2n - 1)!!}{(2n)!!}\cdot\pi/2 \ge \frac{(2n)!!}{(2n+1)(2n - 1)!!} \ge \frac{(2n + 1)!!}{(2n + 2)!!}\pi/2 \\
	&\pi/2 \ge \frac{((2n)!!)^2}{(2n + 1)((2n - 1)!!)^2} \ge \frac{2n + 1}{2n + 2} \cdot \pi/2 \longrightarrow \pi/2 
	\Ra \lim\limits_{n \rightarrow \infty} \frac{((2n)!!)^2}{(2n + 1)((2n - 1)!!)^2} = \pi/2 \\
	&\textbf{Формула Валлиса:} \lim\limits_{n \rightarrow \infty} \frac{(2n)!!}{(2n - 1)!!}\cdot\frac{1}{\sqrt{2n + 1}} = \sqrt{\pi/2} \\
	&(2n)!! = 2^nn! \\
	&\frac{(2n)!!}{(2n - 1)!!} = \frac{((2n)!!)^2}{(2n)!} = \frac{(2^nn!)^2}{(2n)!} = 4^n\frac{n!n!}{(2n)!} = \frac{4^n}{C_{2n}^n} \\
	&\frac{4^n}{C_{2n}^n}\cdot\frac{1}{\sqrt{2n + 1}} \sim \sqrt{\pi/2} \\
	&C_{2n}^{n} \sim \frac{4^n}{\sqrt{\pi/2}}\cdot\underbrace{\frac{1}{\sqrt{2n + 1}}}_{\sim1/\sqrt{2n}} \sim \frac{4^n}{\sqrt{\pi n}} \Ra
	C_{2n}^n \sim \frac{4^n}{\sqrt{\pi n}}
\end{aligned} 
\]

3) $H_n = \frac{1}{n!}\int\limits^{\pi/2}_0({{(\pi/2)}^2 - x^2)}^n\cos x\d x$

\begin{enumerate}
	\item $H_n \longrightarrow 0$
	\[ 0 < H_n \le \frac{1}{n!}\max\limits_{x \in [0, \pi/2]}{({(\pi/2)}^2 - x^2)}^n\cos x \le \frac{1}{n!}\pi/2{(\pi/2)}^{2n} \longrightarrow 0 \]
	Так как показательная функция растёт медленнее факториала. 
	\item $H_n = (4n - 2)H_{2n - 1} - n^2H_{n - 2} $ 

	$H_0 = 1$ $H_1 = 2$

	\underline{Пояснение:}
	\begin{gather*}
		n!H_n = \int\limits_{0}^{\pi/2}{({(\pi/2)}^2 - x^2)}^n(\sin x)' \d x = \\
		= (\text{Подстановка, которая равна 0}) - \int\limits_{0}^{\pi/2}({({(\pi/2)}^2 - x^2)}^n)'\sin x \d x = \\
		= -2n\int\limits_{0}^{\pi/2}{({(\pi/2)}^2 - x^2)}^{n - 1}x(\cos x)' \d x 
	\end{gather*}
	Затем ещё раз интегрируем по частям, используя равенство $x^2 = {(\pi/2)}^2 - ({(\pi/2)}^2 - x^2)$
	\item $H_n = P_n(\pi^2)$, где $P_n$ - многочлен $n$-ой степени с целыми коэффициентами.
\end{enumerate}
\begin{theorem}
	$\pi$ и $\pi^2$ иррациональны.
\end{theorem}
\begin{proof}
От противного. Пусть $\pi = \frac{a}{b}$.

\begin{gather*}
	H_n = P_n\left(\frac{a^2}{b^2}\right) \Ra b^{2n}H_n = b^{2n}P_n\left(\frac{a^2}{b^2}\right) - \text{целое число.} \\
	H_n > 0 \Ra b^{2n}H_n > 0                                                                     \\
	1 \le b^{2n}H_n \le b^{2n}\pi/2{(\pi/2)}^{2n}\frac{1}{n!} = \pi/2\frac{{\left(b\pi/2\right)}^{2n}}{n!} \longrightarrow 0  \\
\end{gather*}
\end{proof}
\begin{theorem}{Формула Тейлора с остатком в интегральной форме:}

	$f \in C^{n + 1}<a, b>$ 
	
	$x_0\in<a, b>$

	Тогда $f(x) = T_{n, x_0}f(x) + \frac{1}{n!}\int\limits_{x_0}^x(x - t)^nf^{(n + 1)}t \d t$
\end{theorem}
\begin{proof}
	Докажем по индукции.
	
	\begin{description}
		\item[База:] $n = 0$:
		$$f(x) = f(x_0) + \int\limits_{x_0}^{x}f'(t)\d t = f(x_0) + f(x) - f(x_0)$$
		\item[Переход:] 
		\begin{gather*}
			f(x) = T_{n, x_0}f(x) + \frac{1}{n!}\int\limits_{x_0}^x{(x - t)}^nf^{(n + 1)}(t)\d t = T_{n, x_0}f(x) - \frac{1}{(n + 1)!}
			\int\limits_{x_0}^xf^{(n + 1)}(t)({(x - t)}^{n + 1})'\d t = \\
			= T_{n, x_0}f(x) - \left(\frac{f^{(n + 1)}(t)}{(n + 1)!}(x - t)^{(n + 1)}\biggr|_{x_0}^{x} - \frac{1}{(n + 1)!}\int\limits_{x_0}^x(x - t)^{n + 1}f^{n + 2}(t)\d t \right) = \\
			= T_{n, x_0}f(x) + \frac{f^{(n + 1)}x_0}{(n + 1)!}(x - x_0)^{(n + 1)} + \frac{1}{(n + 1)!}\int\limits_{x_0}^x{(x - t)}^{n + 1}f^{(n + 2)}(t)\d t
		\end{gather*}
	\end{description}
\end{proof}

\section{Интегральные суммы.}
\begin{Def}{}
	$f \colon X \ra Y$, $X, Y$ - метрические пространства.

	$\omega_f(\delta) = \sup\{\rho_y(f(x_1), f(x_2)): \rho_x(x_1, x_2) < \delta\}$

	$\omega_f(\delta)$ - модуль непрерывности.
\end{Def}

\underline{Свойства:}
\begin{enumerate}
	\item $\omega_f(\delta) \ge 0$
	\item $\omega_f(0) = 0$
	\item $\omega_f(\delta)\nearrow$
	\item $f$ равномерно непрерывно на $X \Leftrightarrow w_f$ непрерывна в 0.

	\begin{proof}
		\begin{itemize}
			\item "$\Rightarrow$"

			$f$ - равномерно непрерывно на $X$ $\Leftrightarrow \forall \epsilon > 0, \exists \delta > 0 \colon (\forall x_1, x_2 \in X) \wedge 
			(\rho_x(x_1, x_2) \le \delta) \Rightarrow \rho_y(f(x_1), f(x_2)) < \epsilon$

			$\Rightarrow \underbrace{\sup\{\rho_y(f(x_1, f(x_2))\colon \rho_x(x_1, x_2) \le \delta\}}_{= w_f(\delta)} \le \epsilon$
   	
			$\forall \epsilon > 0, \exists \delta > 0 \colon w_f(\delta) \le \epsilon \Rightarrow 0 \le w_f(\gamma) \le w_f(\delta) \le \epsilon$
			при $\gamma \le \delta$   

			$\Rightarrow w_f$ - непрерывна в 0

			\item "$\Leftarrow$"

			$\forall \epsilon < 0, \exists \delta > 0 \colon w_f(\delta) < \epsilon \Rightarrow$

			$\Rightarrow$ Если $\rho_x(x_1, x_2) \le \delta$ $\rho_y(f(x_1), f(x_2) < \sup\{\rho_y(f(x_1), f(x_2))\colon \rho_x(x_1, x_2) \le \delta\} < \epsilon$
		\end{itemize}
	\end{proof}
	\item $X$ - компактб $f$ - непрерывна на $X \Rightarrow w_f$ непрерывна в 0.
\end{enumerate}

\begin{Def} $[a, b]$
	
	\text{\bf Дробление отрезка} $\tau \colon x_0 = a < x_1 < x_2 < \dots < x_n = b$. (Иногда называют пунктиром)

	\text{\bf Мелкость дробления} (ранг дробления) $ |\tau| = \max\limits_{k = 1, \dots, n}(x_k - x_{k - 1})$

	\text{\bf Оснащение дробления} $\colon \xi_1, \xi_2, \dots, \xi_n$, где $\xi_k \in [x_{k - 1}, x_k]$

\end{Def}

\begin{Def}{Сумма Римона} $f \colon [a, b] \longrightarrow \R$

	$\sigma(f, \tau, \xi) = \sum\limits_{k = 1}^n(x_k - x_{k - 1})$

\end{Def}

\begin{theorem}
	$f \in C[a, b]$

	$\forall \epsilon, \exists \delta > 0 \colon \forall$ дробления $(\tau, \xi), |\tau| < \delta| \Rightarrow |\sigma(f, \tau, \xi) -
	\int\limits_a^bf(x)\d x| < \epsilon$

	В частности, если $(\tau_n, \xi_n)$ - последовательность дроблений такая, что $|\tau_n| \longrightarrow 0$, то 
	$\sigma(f, \tau_n, \xi_n) \longrightarrow \int\limits_a^bf(x)\d x$
\end{theorem}
\begin{proof}
	\begin{gather*}
		\Delta = \sigma(f, \tau, \xi) - \int\limits_a^bf(x)\d x = \sum\limits_{k = 1}^nf(\xi_k)(x_k - x_{k - 1}) - \sum\limits_{k = 1}^n
		\int\limits_{x_{k - 1}}^{x_k}f(x) \d x = \\
		= \sum\limits_{k = 1}^n\left(\int\limits_{x_{k - 1}}^{x_k}f(\xi_k)\d x - \int\limits_{x_{k - 1}}^{x_k} f(x) \d x\right) =
		\sum\limits_{k = 1}^n\int\limits_{x_{k - 1}}^{x_k}(f(\xi_k) - f(x))\d x \\
		|\Delta| \le \sum\limits_{k = 1}^n\left|\int\limits_{x_{k - 1}}^{x_k}(f(\xi_k) - f(x))\d x \right| \le \sum\limits_{x = 1}^n
		\int\limits_{x_{k - 1}}^{x_k}|f(\xi_k) - f(x)|\d x \le \\
		\le \sum\limits_{k = 1}^n\int\limits_{x_{k - 1}}^{x_k}w_f(|\tau|)\d x(\text{т.к. $|\xi_k - x| < \tau$}) = \sum\limits_{k = 1}^nw_f(|\tau|)(x_k - x_{k - 1}) \\
		|\Delta| \le w_f(|\tau|)(b - a) < \epsilon(b - a)
	\end{gather*}
\end{proof}
\begin{Def}
	$f:[a, b] \longrightarrow \R$

	$f$ - интегрируема по Риману, если $\exists I \colon \forall \epsilon, \exists \delta > 0\colon \forall |\tau| < \delta |\sigma(f, \tau, \xi) - I| < \epsilon$

	и $I$ называется интегралом Римана. ($\int\limits_a^bf(x)\d x$)
\end{Def}
\begin{Rem}
	В этом определении не обязательна непрерывность функции, но для не непрерывных будет непонятно, существует ли интеграл.
\end{Rem}
\begin{conseq}
	\[ (b - a)\min f \le \int_a^b f \le (b - a)\max f \]
\end{conseq}
\begin{proof}
	Проинтегрируем неравенства $\min f \le f \le \max f$ и получим то, что надо.
\end{proof}
\begin{conseq}
	\[\left| \int_a^b f \right| \le \int_a^b |f|\]
\end{conseq}
\begin{proof}
	\begin{gather*}
		 -|f| \le f \le |f| \\
		 \int_a^b -|f| \le \int_a^b f \le \int_a^b |f| \\
		 -\int_a^b |f| \le \int_a^b f \le \int_a^b |f| \\
		 \left|\int_a^b f \right| \le \int_a^b |f|
	 \end{gather*}
\end{proof}

\begin{conseq}{Интегральная теорема о среднем}
	\[\exists c\colon \int_a^b f = (b - a)f(c)\]
\end{conseq}
\begin{proof}
	$\min f \le \frac{1}{b - a} \int_a^b f \le \max f$\\
	Заметим, что $\min f$ и $\max f$ достигаются, а значит и $\frac{1}{b - a} \int_a^b f$ достигается в некоторой точке. Возьмём её за $c$.
\end{proof}

\begin{Def}
	$\fint_a^b f$ "--- среднее значение функции $f$ на отрезке $[a, b]$\\
	\[\fint_a^b f = \frac{1}{b - a} \int_a^b f\]
\end{Def}

\begin{Def}
	$f \in C[a, b]; \Phi \colon [a, b] \ra \R$ \\
	$\Phi (x) = \int_a^x f$ "--- интеграл с переменным верхним пределом.
\end{Def}

\begin{theorem}{Барроу}
	$f \in C[a, b]$
	\[\Phi^{'}(x) = f(x)\]
\end{theorem}
\begin{proof}
	\begin{gather*}
		 \text{Пусть}  x, y \in [a, b]; x < y, тогда \\
		 \Phi^{'}(x) = \lim_{y \ra x} \frac{\Phi (y) - \Phi(x)}{y - x} \\
		 \Phi (y) - \Phi (x) = \int_a^y f - \int_a^x f = \int_x^y f \\
		 \Phi^{'} (x) = \lim_{y \ra x} \frac{1}{y - x} \int_x^y f \\
		 \exists c_y \in [x, y] \colon \lim_{y \ra x} \frac{1}{y - x} \int_x^y f = \lim_{y \ra x} f(c_y)\\
		 \text{Неформальное объяснение : пусть } y \ra x \text{, тогда } c_y \ra x \Rightarrow \lim_{y \ra x} f(c_y) = f(x) \Rightarrow \Phi^{'} = f(x) \\
		 \text{Формальное объяснение : докажем, что }  \forall \epsilon > 0, \exists \delta \colon \forall y \in (x - \delta, x + \delta), |f(с_y) - f(x)| < \epsilon \\
		 \text{по определению непрерывности в точке } x \colon \\
		 \forall \epsilon > 0, \exists \delta \colon \forall y \in (x - \delta, x + \delta), |f(y) - f(x)| < \epsilon \\
		 y \in (x - \delta, x + \delta) \Rightarrow y_c \in (x - \delta, x + \delta) \Rightarrow |f(y_c) - f(x)| < \epsilon \\
		 \text{аналогично можно доказать для } y > x
	 \end{gather*}
\end{proof}

\begin{Def}
	$f \in C[a, b]; \Psi \colon [a, b] \ra \R$ \\
	$\Psi (x) = \int_x^b f$ "--- интеграл с переменным верхним пределом.
\end{Def}

\begin{conseq}
	\[ \Psi^{'}(x) = -f(x)\]
\end{conseq}
\begin{proof}
	\begin{gather*}	
		\Phi (x) + \Psi (x) = \int_a^x f + \int_x^b f = \int_a^b f \\
		\Psi(x) = \int_a^b - \Phi(x)\\
		\Psi^{'}(x) =  0 - \Phi^{'}(x) = -f(x)
	\end{gather*}
\end{proof}

\begin{conseq}
	\[ f \in C\langle a, b \rangle \Rightarrow \exists F \colon F \text{ "--- первообразная } f\]
\end{conseq}
\begin{proof}
	\begin{gather*}	
		\text{Возьмём } c \in (a, b)\\
		\Phi \colon [c, b\rangle \ra \R; \Phi(x) = \int_c^x f; \Phi^{'}(x) = f(x)\\
		\Psi \colon \langle a, c] \ra \R; \Psi(x) = \int_x^c f; \Psi^{'}(x) = -f(x)\\
	\end{gather*}
	\[
	F(x) = \left\{
	\begin{aligned}
		\Phi (x), x \in [c, b\rangle\\
		\Psi (x), x \in \langle a, c] \\
	\end{aligned}
	\right.
	\]
\end{proof}

\begin{theorem}{Формула Ньютона "--- Лейбница}
	$f \in C[a, b]; F \text{ "--- первообразная } f\text{, тогда}$
	\[ \int_a^b f = F(b) - F(a) \]
\end{theorem}
\begin{proof}
	\begin{gather*}
		\Phi(x) = \int_a^x \text{ "--- первообразная } f\\
		F(x) = \Phi (x) + C\\
		F(b) - F(a) = \Phi (b) + C - \Phi (a) - C = \Phi (b) = \int_a^b f
	\end{gather*}
\end{proof}

\begin{conseq*}
	\[\int_a^b f \text{ не зависит от выбора площади $\sigma$, используемой в определении.} \]
\end{conseq*}

\begin{theorem}{Линейность интеграла}
	$f, g \in C[a, b]; \alpha, \beta \in \R$, тогда
	\[ \int_a^b(\alpha f + \beta g) = \alpha \int_a^b f + \beta \int_a^b g\]
\end{theorem}
\begin{proof}
	Пусть $F$ - первообразная $f$; $G$ - первообразная $g$, тогда $\alpha F + \beta G$ - первообразная $\alpha f + \beta g$ \\
	$\int_a^b(\alpha f + \beta g) = \alpha F(b) + \beta G(b) - (\alpha F(a) + \beta G(a)) = \alpha (F(a) - F(b)) + \beta (G(a) - G(b)) = \alpha \int_a^b f +\beta \int_a^b g$
\end{proof}

\begin{Def}
	$ F |_{a}^{b}$ "--- подстановка\\
	$F |_{a}^{b} = F(b) - F(a)$
\end{Def}

\begin{theorem}{формула интегрирования по частям}
	$u, v \in C^1[a, b]$, тогда
	\[ \int_a^b u v^{'} = u v |_{a}^{b} - \int_a^b u^{'} v\]
\end{theorem}

\begin{proof}
	\begin{gather*}
		(uv)^{'} = u v^{'} + u^{'} v\\
		uv = \int u v^{'} + \int u^{'} v\\
		\int u v^{'} = uv - \int u^{'} v\\
		F(x) = u(x)v(x) - G(x)
	\end{gather*}
	$\int_a^b u v^{'} = F(b) - F(a) = (u(b)v(b) - G(b)) - (v(a)u(a) - G(a)) = (u(b)v(b) - v(a)u(a)) - (G(b) - G(a)) = uv |_{a}^{b} - \int_a^b u^{'} v$
\end{proof}

\begin{exmp}
	$\int_1^n \ln x = x \ln x |_{1}^{n} - \int_1^n x (\ln x)^{'} = n \ln n - 1 \ln 1 - \int_1^n \frac{x}{x} = n \ln n - n + 1$ 
\end{exmp}

\begin{Def}
	$\int_a^b f(x) dx \\ dx$ "--- параметр интегрирования
\end{Def}

\begin{Def}
	$a > b \\ \int_a^b f = - \int_b^a f$
\end{Def}

\begin{theorem}{Замена переменных в определённом интеграле}\\
	$f \in C\langle a, b \rangle; \Phi \in C^1[ \alpha, \beta]; \Phi([\alpha, \beta]) \subset \langle a, b \rangle$, тогда
	\[ \int_{\alpha}^{\beta} f(\Phi(t)) (\Phi (t))^{'} dt = \int_{\Phi(\alpha)}^{\Phi(\beta)} f(x) dx\]
\end{theorem}
\begin{proof}
	$F$ "--- первообразная $f; F(\Phi(t))$ "--- первообразная $f(\Phi (t)) (\Phi (t))^{'}$\\
	$\int_{\alpha}^{\beta} f(\Phi (t)) (\Phi (t))^{'} dt = F(\Phi (\beta)) - F(\Phi (\alpha)) = \int_{\Phi(\alpha)}^{\Phi(\beta)} f(x) dx$
\end{proof}
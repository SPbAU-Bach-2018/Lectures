\setauthor{Юрий Кравченко}
\begin{conseq}
	\[ (b - a)\min f \le \int\limits_a^b f \le (b - a)\max f \]
\end{conseq}
\begin{proof}
	Проинтегрируем неравенства $\min f \le f \le \max f$ и получим то, что надо.
\end{proof}

\begin{conseq}
	\[ \left| \int\limits_a^b f \right| \le \int\limits_a^b |f| \]
\end{conseq}
\begin{proof}
	\begin{gather*}
		 -|f| \le f \le |f| \\
		 \int\limits_a^b -|f| \le \int\limits_a^b f \le \int\limits_a^b |f| \\
		 -\int\limits_a^b |f| \le \int\limits_a^b f \le \int\limits_a^b |f| \\
		 \left|\int\limits_a^b f \right| \le \int\limits_a^b |f|
	 \end{gather*}
\end{proof}

\begin{conseq}[Интегральная теорема о среднем]
	\[ \exists c\colon \int\limits_a^b f = (b - a)f(c) \]
\end{conseq}
\begin{proof}
	\[ \min f \le \frac{1}{b - a} \int\limits_a^b f \le \max f \]
	Заметим, что $\min f$ и $\max f$ достигаются, а значит и $\frac{1}{b - a} \int_a^b f$ достигается в некоторой точке. Возьмём её за $c$.
\end{proof}

\begin{Def}
	$\fint_a^b f$ "--- среднее значение функции $f$ на отрезке $[a, b]$
	\[ \fint_a^b f = \frac{1}{b - a} \int\limits_a^b f\]
\end{Def}

\begin{Def}
	$f \in C[a, b]$, $\Phi\colon [a, b] \ra \R$.
	$\Phi (x) = \int_a^x f$ "--- интеграл с переменным верхним пределом.
\end{Def}

\begin{theorem}[Барроу]
	$f \in C[a, b]$
	\[\Phi'(x) = f(x)\]
\end{theorem}
\begin{proof}
	Пусть $x, y \in [a, b]$, $x < y$.
	Тогда
	\begin{gather*}
		\Phi'(x) = \lim_{y \ra x} \frac{\Phi (y) - \Phi(x)}{y - x} \\
		\Phi (y) - \Phi (x) = \int\limits_a^y f - \int\limits_a^x f = \int\limits_x^y f \\
		\Phi' (x) = \lim_{y \ra x} \frac{1}{y - x} \int\limits_x^y f \\
		\exists c_y \in [x, y]\colon \lim_{y \ra x} \frac{1}{y - x} \int\limits_x^y f = \lim_{y \ra x} f(c_y)
	\end{gather*}

	Неформальное объяснение: пусть $y \ra x$, тогда $c_y \ra x \Ra \lim_{y \ra x} f(c_y) = f(x) \Ra \Phi' = f(x)$.

	Формальное объяснение: докажем, что
	\[ \forall \epsilon > 0, \exists \delta\colon \forall y \in (x - \delta, x + \delta), |f(с_y) - f(x)| < \epsilon \]
	По определению непрерывности в точке $x$:
	\begin{gather*}
		\forall \epsilon > 0, \exists \delta\colon \forall y \in (x - \delta, x + \delta), |f(y) - f(x)| < \epsilon \\
		y \in (x - \delta, x + \delta) \Ra y_c \in (x - \delta, x + \delta) \Ra |f(y_c) - f(x)| < \epsilon
	 \end{gather*}
	Аналогично можно доказать для $y > x$.
\end{proof}

\begin{Def}
	$f \in C[a, b]$, $\Psi \colon [a, b] \ra \R$.
	$\Psi (x) = \int_x^b f$ "--- интеграл с переменным нижним пределом.
\end{Def}

\begin{conseq}
	\[ \Psi'(x) = -f(x) \]
\end{conseq}
\begin{proof}
	\begin{gather*}
		\Phi (x) + \Psi (x) = \int\limits_a^x f + \int\limits_x^b f = \int\limits_a^b f \\
		\Psi(x) = \int\limits_a^b f - \Phi(x)\\
		\Psi'(x) =  0 - \Phi'(x) = -f(x)
	\end{gather*}
\end{proof}

\begin{conseq}
	\[ f \in C\langle a, b \rangle \Ra \exists F \colon F \text{ "--- первообразная } f\]
\end{conseq}
\begin{proof}
	Возьмём $c \in (a, b)$:
	\begin{gather*}
		\Phi \colon [c, b\rangle \ra \R; \Phi(x) = \int\limits_c^x f; \Phi'(x) = f(x)\\
		\Psi \colon \langle a, c] \ra \R; \Psi(x) = \int\limits_x^c f; \Psi'(x) = -f(x)\\
		F(x) =
		\begin{cases}
			\Phi (x)  & x \in [c, b\rangle\\
			-\Psi (x) & x \in \langle a, c] \\
		\end{cases}
	\end{gather*}
\end{proof}

\begin{theorem}[Формула Ньютона "--- Лейбница]
	$f \in C[a, b]$, $F$ "--- первообразная $f$.
	Тогда
	\[ \int\limits_a^b f = F(b) - F(a) \]
\end{theorem}
\begin{proof}
	\begin{gather*}
		\Phi(x) = \int\limits_a^x f \text{ "--- первообразная } f \\
		F(x) = \Phi (x) + C \\
		F(b) - F(a) = \Phi (b) + C - \Phi (a) - C = \Phi (b) = \int\limits_a^b f
	\end{gather*}
\end{proof}

\begin{conseq*}
	Определённый интеграл не зависит от выбора площади $\sigma$, используемой в определении.
\end{conseq*}

\begin{theorem}[Линейность интеграла]
	$f, g \in C[a, b]$, $\alpha, \beta \in \R$.
	Тогда
	\[ \int\limits_a^b(\alpha f + \beta g) = \alpha \int\limits_a^b f + \beta \int\limits_a^b g \]
\end{theorem}
\begin{proof}
	Пусть $F$ "--- первообразная $f$, $G$ "--- первообразная $g$.
	Тогда $\alpha F + \beta G$ "--- первообразная $\alpha f + \beta g$:
	\[ \int\limits_a^b(\alpha f + \beta g) = \alpha F(b) + \beta G(b) - (\alpha F(a) + \beta G(a)) = \alpha (F(a) - F(b)) + \beta (G(a) - G(b)) = \alpha \int\limits_a^b f +\beta \int\limits_a^b g \]
\end{proof}

\begin{Def}
	$ F \bigl|_{a}^{b}$ "--- подстановка:
	\[ F \bigl|_{a}^{b} = F(b) - F(a) \]
\end{Def}

\begin{theorem}[Формула интегрирования по частям]
	$u, v \in C^1[a, b]$.
	Тогда
	\[ \int\limits_a^b u v' = uv \biggl|_a^b - \int\limits_a^b u' v \]
\end{theorem}

\begin{proof}
	\begin{gather*}
		(uv)' = u v' + u' v\\
		uv = \int u v' + \int u' v\\
		\int u v' = uv - \int u' v\\
		F(x) = u(x)v(x) - G(x) \\
		\int\limits_a^b u v' = F(b) - F(a) = (u(b)v(b) - G(b)) - (v(a)u(a) - G(a)) = \\
		= (u(b)v(b) - v(a)u(a)) - (G(b) - G(a)) = uv |_{a}^{b} - \int\limits_a^b u' v
	\end{gather*}
\end{proof}

\begin{exmp}
	Интеграл логарифма:
	\[ \int\limits_1^n \ln x = x \ln x \biggl|_{1}^{n} - \int\limits_1^n x (\ln x)' = n \ln n - 1 \ln 1 - \int\limits_1^n \frac{x}{x} = n \ln n - n + 1 \]
\end{exmp}

\begin{Def}
	$\d x$ "--- параметр интегрирования:
	\[ \int\limits_a^b f(x) \d x \]
\end{Def}
Пока не очень будем задумываться над глубоким смыслом этой записи, воспринимаем как <<скобки>>.

\begin{Def}
	$a > b$:
	\[ \int\limits_a^b f = - \int\limits_b^a f \]
\end{Def}

\begin{theorem}[Замена переменных в определённом интеграле]
	$f \in C\langle a, b \rangle$,  $\phi \in C^1[ \alpha, \beta]$, $\phi([\alpha, \beta]) \subset \langle a, b \rangle$, тогда
	\[ \int\limits_{\alpha}^{\beta} f(\phi(t)) \phi'(t) \d t = \int\limits_{\phi(\alpha)}^{\phi(\beta)} f(x) \d x\]
\end{theorem}
\begin{proof}
	$F$ "--- первообразная $f$, $F(\phi(t))$ "--- первообразная $f(\phi (t)) \cdot \phi'(t)$.
	\[ \int\limits_{\alpha}^{\beta} f(\phi (t)) \phi'(t) \d t = F(\phi (\beta)) - F(\phi (\alpha)) = \int\limits_{\phi(\alpha)}^{\phi(\beta)} f(x) dx \]
\end{proof}
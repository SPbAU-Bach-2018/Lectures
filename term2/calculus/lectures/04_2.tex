\begin{exmp}
	\begin{enumerate}
	\item
		\begin{gather*}
			S_p(n) = 1^p + 2^p + \ldots + n^p \\
			S_p(n) = \sideset{}{'}\sum_{k = 1}^{n} k^p + \frac{1^p + n^p}{2} =
				\int_1^n x^p \d x + \frac{1}{2}\int_1^n p(p - 1)x^{p - 2}\{x\}(1 - \{x\}) + \frac{1 + n^p}{2} \\
			= \frac{1 + n^p}{2} + \frac{n^{p + 1} - 1}{p + 1} + \frac{p(p - 1)}{2}\int_1^n x^{p - 2}\{x\}(1 - \{x\})
		\end{gather*}

		Пусть $p \in (-1, 1)$, тогда
		\begin{gather*}
			0 \le \int_1^n x^{p - 2}\{x\}(1 - \{x\})\d x \le \int_1^{n}x^{p - 2} =
				\frac{n^{p - 1} - 1}{p - 1} = O(1) \\
			S_p(n) = \frac{n^{p + 1}}{p + 1} + \frac{n^p}{2} + O(1)
		\end{gather*}

		Если $p = 1$, тогда
		\[ S_p(n) = \frac{n^{p + 1}}{p + 1} + \frac{n^p}{2} + O(\log n) \]

		Если $p > 1$, тогда
		\[ S_p(n) = \frac{n^{p + 1}}{p + 1} + \frac{n^p}{2} + O(n^{p - 1}) \]

	\item
		Рассмотрим гармонические числа.
		\begin{gather*}
			H_n = \sideset{}{'}\sum_{k = 1}^n \frac{1}{k} + \frac{1 + \frac{1}{n}}{2} =
				\frac{1 + \frac{1}{n}}{2} + \int_1^{n}\frac{\d x}{x} + \frac{1}{2}\int_1^n\frac{2\{x\}(1 - \{x\})}{x^3}\d x = \\
			= \frac{1}{2} + \frac{1}{2n} + \ln n + \int_1^n\frac{\{x\}(1 - \{x\})}{x^3}\d x \\
			H_n = \ln n + O(1) \\
			H_n - \ln n = \frac{1}{2} + \frac{1}{2n} + a_n \\
			a_n = \int_1^n\frac{\{x\}(1 - \{x\})}{x^3}\d x \\
			a_n \le a_{n + 1}
		\end{gather*}
		Поймем, что $a_n$ ограничено
		\begin{gather*}
			a_n = \int_1^n\frac{\{x\}(1 - \{x\})}{x^3}\d x \le
				\int_1^{n}\frac{\d x}{x^3} = \left. -\frac{1}{2x^2}\right |_{x = 1}^{x = n} =
				\frac{1}{2} - \frac{1}{2n} \le \frac{1}{2} \\
			\Ra \exists \lim_{n \to \infty} a_n \Ra \exists \lim_{n \to \infty}(H_n - \ln n) = \gamma
		\end{gather*}
		$\gamma = 0,57721566\ldots$ "--- постоянная Эйлера.

		Итог: $H_n = \ln n + \gamma + o(1)$

	\item
		Формула Стирлинга.
		\begin{gather*}
			\ln n! = \ln 1 + \ln 2 + \ldots + \ln n \\
			\ln n! = \sideset{}{'}\sum_{k = 1}^{n} \ln k + \frac{\ln n}{2} =
				\int_{1}^{n}\ln x\d x - \frac{1}{2}\int_{1}^{n}\frac{\{x\}(1 - \{x\})}{x^2}\d x + \frac{\ln n}{2} = \\
			= n\ln n - n + \frac{\ln n}{2} + 1 - \frac{1}{2}\int_{1}^{n}\frac{\{x\}(1 - \{x\})}{x^2}\d x \\
			\int_{1}^{n}\frac{\{x\}(1 - \{x\})}{x^2}\d x = b_n \\
			b_n \le b_{n + 1} \\
			b_n = \int_{1}^{n}\frac{\{x\}(1 - \{x\})}{x^2}\d x  \le \int_1^{n}\frac{\d x}{x^2} = \left. -\frac{1}{x} \right|_{x = 1}^{x = n} = 1 -\frac{1}{n} \le 1 \\
			\Ra \exists \lim_{n \to \infty} b_n \\
			\ln n! = n \ln n + \frac{\ln n}{2} - n + C + o(1) \\
			n! = n^n e^{-n}\sqrt{n}e^ce^{o(1)} = n^{n}e^{-n}\sqrt{n}e^c(1 + o(1)) \sim
				n^n e^{-n}\sqrt{n}e^c \\
			\frac{4^n}{\sqrt{\pi n}} = C_{2n}^n = \frac{(2n)!}{(n!)^2} \sim
				\frac{(2n)^{2n}e^{-2n}\sqrt{2n}e^c}{(n^ne^{-n}\sqrt{n}e^c)^2} =
				\frac{4^n n^{2n}e^{-2n}\sqrt{2n}e^c}{n^{2n}e^{-2n}ne^{2c}} = \frac{4^n\sqrt{2n}}{ne^c} \\
			\frac{4^n}{\sqrt{\pi n}} \sim \frac{4^n \sqrt{2n}}{n e^c} \\
			e^c \sim \sqrt{2\pi} \Ra e^c = \sqrt{2\pi} \\
		\end{gather*}

		\textbf{Формула Стирлинга:}
		\[ n! \sim n^ne^{-n}\sqrt{2\pi n} \]
		\begin{Rem}
			Можно использовать для биномиальных коэффициентов. Записать их через
			факториал и расписать.
		\end{Rem}
	\end{enumerate}
\end{exmp}

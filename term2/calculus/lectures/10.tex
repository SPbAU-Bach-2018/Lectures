\begin{exmp}
	$f(t) = (\cos t, \sin t)$, $b = 2\pi$, $a = 0$:
	\begin{gather*}
		f(b) - f(a) = \vec 0 \\
		\| f(b) - f(a) \| = 0 \\
		f'(t) = (-\sin t, \cos t) \\
		\| f'(t) \| = \sqrt{\sin^2 t + \cos^2 t} = 1
	\end{gather*}
	Очевидно, равенство здесь недостижимо.
\end{exmp}

\section{Частные производные}

Картинка про сечение.

Если есть функия от многих переменных, мы можем взять сечение её графика плоскостью, и рассматривать производную оставшейся функции.

\begin{Def}
	Пусть есть функция $f\colon D \ra \R$, точка $a \in \Int D$ и вектор направления $\vec h\colon \|h\| = 1$.
	Рассмотрим функцию
	\[ F(t) = f(a + th), t \in \R \]
	Тогда производной $f$ по направлению $h$ в точке $a$ называют
	\[ \partd{f}{h}(a) = F'(0) \]
\end{Def}

\begin{Rem}
	\[ F'(0) = \lim_{t \ra 0} \frac{F(t) - F(0)}t = \lim_{t \ra 0} \frac{f(a + th) - f(a)}t \]
	Очень похоже на определение производной функции одной переменной.
\end{Rem}

\begin{assertion}
	$f\colon D \ra \R$, $a \in \Int D$, $f$ дифференцируема в $a$, $h$ "--- ненулевой вектор.
	\[ \partd{f}{h}(a) = \d_a f(h) \]
\end{assertion}
\begin{proof}
	\begin{gather*}
		\partd{f}{h}(a) = F'(0) \\
		\d_a f\colon t \mapsto Tt \\
		\d_0 F = \underbrace{\d_a f}_{t \mapsto Tt} \circ \underbrace{\d_0 (a + th)}_{t \mapsto ht} = (t \mapsto Tht) \\
		\d_0 F = (t \mapsto F'(0) \cdot t) \\
		F'(0) = Th = (t \mapsto Tt)(h) = \d_a f(h)
	\end{gather*}
\end{proof}
То есть градиент указывает, где производная максимальна.
\begin{theorem}[Экстремальное свойство градиента]
	$f\colon D \ra \R$, $a \in \Int D$, $f$ дифференцируема в $a$, $h$ "--- напралвение, $\nabla f(a) \ne 0$.
	Тогда
	\[ \left|\partd{f}{h}(a)\right| \le \|\nabla f(a) \| \]
	и равенство достигается только в случае $h \parallel \nabla f(a)$
\end{theorem}
%% ... %%
\begin{proof}
	\begin{gather*}
		\left|\partd{f}{h}(a)\right| = |\left< \nabla f(a), h\right>| \le \|\nabla f(a)\| \|h\| = \|\nabla f(a) \|
	\end{gather*}
	В неревенстве Коши"---Буняковского равенство было как раз в случае параллельности векторов $\nabla f(a)$ и $h$.
\end{proof}

Физический смысл градиента такой: пусть есть поверхность. Положим на неё шарик. Он покатится туда, куда указывает градиент.

\begin{Def}
	Частная производная "--- производная по направлению базисного вектора
	\[ e_k = (0, \dots, \underbrace{1}_{k}, \dots, 0) \]
	Записывается:
	\[ \partd{f}{e_k} (a) \]
\end{Def}
\begin{Rem}
	\begin{gather*}
		\partd{f}{e_k} (a) = \lim_{t \ra 0} \frac{f(a_1, \dots, a_k + t, \dots, a_n) - f(a_1, \dots, a_k, \dots, a_n)}t = \\
		= \lim_{t \ra 0} \frac{g(a_k + t) - g(a_k)}t = \frac{\d g}{\d x} (a_k) \quad g(x) \lrh f(a_1, \dots, x, \dots, a_n)
	\end{gather*}
	Таким образом, частная производная "--- производная всей функции по одной координате.
\end{Rem}
Обозначения:
\[
	\begin{matrix}
		\partd{f}{x_k} & D_{x_k} f & f'_{x_k}
	\end{matrix}
\]

\begin{exmp}
	$f(x, y) = x^y$:
	\begin{gather*}
		\partd{f}{x} (x, y) = y \cdot x^{y - 1} \\
		\partd{f}{y} (x, y) = \ln x \cdot x^y \\
	\end{gather*}
\end{exmp}

\begin{conseq}
	$f\colon D \subset R^n \ra \R$, $a \in \Int D$, $f$ дифференцируема в $a$.
	\[ \partd{f}{e_k} (a) = \left<\nabla f(a), e_k\right> = (\nabla f(a))_k \]
	Таким образом,
	\[ \nabla f(a) = \left( \partd{f}{e_1} (a), \dots, \partd{f}{e_n} (a) \right) \]
\end{conseq}

\begin{conseq}
	$f\colon \R^n \ra \R^m$, $a \in \Int D$, $f$ дифференцируема в $a$.
	Тогда матрица Якоби (матрица дифференциала)
	\[\left(\begin{matrix}
		\partd{f_1}{e_1}(a) & \partd{f_1}{e_2}(a)
			&\cdots& \partd{f_1}{e_n}(a) \\
		\partd{f_2}{e_1}(a) & \partd{f_2}{e_2}(a)
			&\cdots& \partd{f_2}{e_n}(a) \\
		\vdots & \vdots & \ddots & \vdots \\
		\partd{f_m}{e_1}(a) & \partd{f_m}{e_2}(a)
			&\cdots& \partd{f_m}{e_n}(a)
	\end{matrix}\right)\]
\end{conseq}
\begin{proof}
	\begin{gather*}
		f(a + h) = f(a) + \d_a f(h) + o(h) \\
		f_i(a + h) = f_i(a) + (\d_a f(h))_i + o(h) = f_i(a) + \d_a f_i(h) + o(h) = f_i(a) + \left<\nabla f_i(a), h\right> + o(h)
	\end{gather*}
	Отсюда видим, что $i$-я строка матрицы есть вектор частных производных:
	\begin{gather*}
		\left(\begin{matrix}
			\nabla f_1(a) \\
			\nabla f_2(a) \\
			\vdots \\
			\nabla f_m(a)
		\end{matrix}\right)
	\end{gather*}
\end{proof}

\begin{conseq}
	Формула для дифференциала композиции:
	$f\colon D \subset R^n \ra R^m$, $g\colon E \subset R^m \ra R^l$, $f(D) \subset E$, $a \in \Int D$, $f(a) \in \Int E$,
	$f$ дифференцируема в $a$, $g$ "--- в $f(a)$.
	Тогда
	\[ \d_a(g \circ f) = \d_f(n) g \circ \d_a f \]
	то есть матрица $\d_a (g \circ f)$ есть
	\[\left(\begin{matrix}
		\partd{g_1}{e_1}(a) & \partd{g_1}{e_2}(a)
			&\cdots& \partd{g_1}{e_n}(a) \\
		\partd{g_2}{e_1}(a) & \partd{g_2}{e_2}(a)
			&\cdots& \partd{g_2}{e_n}(a) \\
		\vdots & \vdots & \ddots & \vdots \\
		\partd{g_l}{e_1}(a) & \partd{g_l}{e_2}(a)
			&\cdots& \partd{g_l}{e_m}(a) \\
	\end{matrix}\right)\left(\begin{matrix}
		\partd{f_1}{e_1}(a) & \partd{f_1}{e_2}(a)
			&\cdots& \partd{f_1}{e_n}(a) \\
		\partd{f_2}{e_1}(a) & \partd{f_2}{e_2}(a)
			&\cdots& \partd{f_2}{e_n}(a) \\
		\vdots & \vdots & \ddots & \vdots \\
		\partd{f_m}{e_1}(a) & \partd{f_m}{e_2}(a)
			&\cdots& \partd{f_m}{e_n}(a) \\
	\end{matrix}\right)\]
	и
	\[ \partd{(g \circ f)_i}{e_k} = \sum_{j = 1}^m \partd{g_i}{e_j} (f(a)) \partd{f_j}{e_k} (a) \]
\end{conseq}

\begin{Rem}
	Наличие частных производных ничего не говорит о дифференцируемости:
	\begin{gather*}
		f(x, y) = \begin{cases} 0 & xy = 0 \\ 1 & xy \ne 0 \end{cases} \\
		f'_x(0, 0) = 0 \quad f'_y(0, 0) = 0
	\end{gather*}
	Сама функция даже не непрерывна!

	Даже если потребовать про все напраления, ситуацию это не исправит. Рассмотрим функцию, равную единице только на дуге

	Картинка 2.

	Там все производные по направлению равны нулю.
\end{Rem}

\begin{theorem}
	$f\colon D \subset \R^n \ra \R$, $a \in \Int D$, все частные производные в окрестности $a$ существуют и непрерывны.
	Тогда $f$ дифференцируема в $a$.
\end{theorem}
\setauthor{Кравченко Юрий}
\chapter{Функции нескольких переменных}
\section{Линейные операторы}

\begin{Def}
	$X$ --- линейное (векторное) пространство над $\R$ \\
	$x_1, x_2, ..., x_n \in X$\\
	$\lambda_1x_1 + \lambda_2x_2 + \dots + \lambda_nx_n$ --- линейная комбинация векторов $x_1, x_2, \dots, x_n$
\end{Def}

\begin{Def}
	$x_1, x_2, ..., x_n \in X$ --- линейно независимы, если
	$\lambda_1x_1 + \lambda_2x_2 + \dots + \lambda_nx_n = 0$ только при $\lambda_1 = \lambda_2 = \dots = \lambda_n = 0$
\end{Def}

\begin{Def}
	Размерность пространства --- максимальное число линейно независимых векторов
\end{Def}

\begin{Def}
	$x_1, x_2, ..., x_n \in X$ --- базис, если
	\begin{enumerate}
	\item
		$x_1, x_2, ..., x_n$ линейно независимы
	\item
		$\forall x \in X, \exists \lambda_1, \lambda_2, \dots, \lambda_n \colon x = \lambda_1x_1 + \lambda_2x_2 + \dots + \lambda_nx_n$
	\end{enumerate}
\end{Def}

$X = \R^n$\\
$\left\{
\begin{pmatrix}
1\\
0\\
0\\
\vdots \\
0\\
\end{pmatrix},
\begin{pmatrix}
0\\
1\\
0\\
\vdots \\
0\\
\end{pmatrix},
\dots,
\begin{pmatrix}
0\\
0\\
0\\
\vdots \\
1\\
\end{pmatrix}
\right\}$ --- базис, потому что любой вектор 
$\begin{pmatrix}
	a_1\\
	a_2\\
	\vdots \\
	a_n 
\end{pmatrix}$ можно представить как
$a_1
\begin{pmatrix}
1\\
0\\
0\\
\vdots \\
0\\
\end{pmatrix}
+a_2
\begin{pmatrix}
0\\
1\\
0\\
\vdots \\
0\\
\end{pmatrix}
+\dots+a_n
\begin{pmatrix}
0\\
0\\
0\\
\vdots \\
1\\
\end{pmatrix}$\\
\begin{Def}
$X$, $Y$ --- линейные пространства\\
$A \colon X \rightarrow Y$ --- линейный оператор, если
\begin{enumerate}
\item $A(x + y) = A(x) + A(y), \forall x, y \in X $
\item $A(\lambda x) = \lambda A(x), \forall x \in X, \forall \lambda \in \R$
\end{enumerate}
\underline{Свойства:}
\begin{enumerate}
\item $\exists O \colon O(x) = 0_y \forall x \in X$
\begin{proof}
$\lambda = 0, A(0x) = 0A(x)$
\end{proof}
\item $A(\sum_{k = 1}^n \lambda_k x_k) = \sum_{k = 1}^n \lambda_k A(x_k)$
\begin{proof}
$A(\sum_{k = 1}^n \lambda_k x_k) = \sum_{k = 1}^n A(\lambda_k x_k) = \sum_{k = 1}^n \lambda_k A(x_k)$
\end{proof}
\item Множество линейных операторов --- линейное пространство
\begin{proof}
\begin{enumerate}
\item $\forall A, B, (A + B)(x) = (B + A)(x)$
\item $\forall A, B, C ((A + B) + C)(x) = (A + (B + C))(x)$
\item $\exists O \colon \forall A, (A + O)(x) = A(x) + O(x)$
\item $\forall A, \exists -A \colon (A + -A)(x) = O(x)$
\item $\forall \alpha, \beta, A, \alpha(\beta A)(x) = (\alpha \beta)A(x)$
\item $\forall A, (1 \cdot A)(x) = A(x)$
\item $\forall \alpha, \beta, A, (\alpha + \beta)A(x) = \alpha A(x) + \beta A(x)$
\item $\forall \alpha, A, B, \alpha(A + B)(x) = \alpha A(x) + \alpha B(x)$
\end{enumerate}
Проверьте сами
\end{proof}
\end{enumerate}
\begin{Def}
Композиция линейных операторов\\
$A \colon X \rightarrow Y$ \\
$B \colon Y \rightarrow Z$ \\
$B \circ A \colon X \rightarrow Z$
$(B \circ A)(x) = B(A(x))$
\end{Def}
\begin{theorem}
Композиция линейных операторов --- линейный оператор\\
\end{theorem}
\begin{proof}
\begin{enumerate}
\item $(B \circ A)(x + y)  = B(A(x + y)) = B(A(x) + A(y) = B(A(x)) + B(A(y))  = (B \circ A)(x) + (B \circ A)(y)$
\item $(B \circ A)(\lambda x) = B(A(\lambda x)) = B(\lambda A(x)) = \lambda B(A(x)) = \lambda (A \circ B) (x)$
\end{enumerate}
\end{proof}
\begin{Def}
$A \colon X \rightarrow Y$
$B \colon Y \rightarrow X$
$B$ --- обратный оператор к $A$, если:
\begin{enumerate}
\item $\forall x, (B \circ A)(x) = x$
\item $\forall y, (A \circ B)(y) = y$
\end{enumerate}
$B = A^{-1}$
\end{Def}
\underline{Свойства:}
\begin{enumerate}
\item Если обратный оператор существует, то он единственный
\item $(\lambda A) ^ {-1} = \frac{1}{\lambda}A^{-1}$
\begin{proof}
$(\frac{1}{\lambda}A^{-1})((\lambda A) ^ {-1} )(x) = \frac{1}{\lambda} \lambda(A^{-1} \circ A)(x) = x$
\end{proof}
\end{enumerate}
Теперь рассмотрим случай $Y = X$\\
\begin{theorem}
Множество обратимых операторов образует группу относительно композиции
\end{theorem}
\begin{proof}
\begin{enumerate}
\item $\exists E \colon E(x) = x$
\item Проверим замкнутость операций. Для этого докажем, что\\
$(A \circ B) ^ {-1} (x) = (B^{-1} \circ A^{-1}) (x)$\\ 
$B{-1} \circ A{-1} \circ (A \circ B) (x) = B ^ {-1}(A^{-1}(A(B(x))) = B^{-1}(B(x)) = x$
\end{enumerate}
\end{proof}
Частный случай $X = R ^ n, Y = R ^ m$\\
$A \colon R ^ n \rightarrow R ^ m$\\
$\bar x = x_1 \bar e_1 + \dots + x_n \bar e_n$\\
$A(\bar x) = A(x_1 \bar e_1 + \dots + x_n \bar e_n) = x_1A \bar e_1 + \dots + x_nA \bar e_n$ \\
Напишем это же равенство через матрицы\\
$A(\bar x) = 
\begin{pmatrix}
a_{1, 1} & \dots & a_{1, n}\\
\vdots & \ddots & \vdots\\
a_{m, 1} & \dots & a_{m, n}\\
\end{pmatrix}
\begin{pmatrix}
x_1\\
\vdots\\
x_n\\
\end{pmatrix} = 
\begin{pmatrix}
a_{1, 1}x_1 + \dots + a_{1, n}x_n\\
\vdots\\
a_{m, 1}x_1 + \dots + a_{m, n}x_n\\
\end{pmatrix} = x_1
\begin{pmatrix}
a_{1, 1}\\
\vdots\\
a_{m, 1}\\
\end{pmatrix} + \dots + x_n
\begin{pmatrix}
a_{1, n}\\
\vdots\\
a_{m, n}\\
\end{pmatrix}$

\begin{Def}
Норма оператора
$X$ и $Y$ --- нормированные пространства \\
заданы $||$ $||_x$ и $||$ $||_y$ \\
$A \colon X \rightarrow Y$ --- линейный оператор \\
$||A|| = \sup_{||x||_x \le 1} ||A(x)||_y$\\
или то же самое\\
$||A|| = \sup \{||A(x)||_y \colon ||x||_x \le 1\}$
\end{Def}
\begin{Def}
Если $||A||  < +\inf$, то $A$ --- ограниченный оператор
\end{Def}
\underline{Свойства:}
\begin{enumerate}
\item $||A + B|| \le ||A|| + ||B||$
\begin{proof}
$||A + B|| = \sup ||(A + B)(x)||_y = \sup ||A(x) + B(x)||_y \le \sup ||A(x)||_y + \sup ||B(x)||_y = ||A|| + ||B||$
\end{proof}
\item $||\lambda A|| = |\lambda| \cdot ||A||$
\begin{proof}
$||\lambda A|| = \sup || (\lambda A)(x) ||_y = \sup || \lambda A(x) ||_y = \sup( |\lambda| \cdot ||A(x) ||_y) = |\lambda| \sup ||A(x)||_y = |\lambda| \cdot ||A||$
\end{proof}
\item $||A|| = 0 \Rightarrow A$ --- нулевой оператор
\begin{proof}
coming soon
\end{proof}
\item Норма для операторов, которую мы задали, действительно является нормой в линейном пространстве операторов
\end{enumerate}
\begin{Rem}
Ограниченный оператор $\neq$ ограниченное отображение\\
Если $A \colon X \rightarrow Y$ --- ограниченное отображение, то $A = O$
\end{Rem}
\begin{proof}
Пусть $A \neq O \Rightarrow \exists x \in X \colon A(x_0) \neq 0_y \Rightarrow ||A(x_0)||_y \neq 0$ \\
$||A(\lambda x_0)||_y = |\lambda| \cdot ||Ax_0||_y$
Если $A$ --- ограниченное отображение $\Rightarrow A(\lambda x_0)$ --- ограниченное множество $\Rightarrow ||A(\lambda x_0)|| \le M \forall \lambda \in \R$, но $||A(\lambda x_0)|| = |\lambda| \cdot ||Ax_0||$ противоречие
\end{proof}
coming soon
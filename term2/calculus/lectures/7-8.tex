\chapter{Функции нескольких переменных}
\setauthor{Кравченко Юрий}
\section{Линейные операторы}

\begin{Def}
	$X$ "--- линейное (векторное) пространство над $\R$, $x_1, x_2, \dots, x_n \in X$, $\lambda_1, \lambda_2, \dots, \lambda_n \in \R$.
	$\lambda_1x_1 + \lambda_2x_2 + \dots + \lambda_nx_n$ "--- линейная комбинация векторов $x_1, x_2, \dots, x_n$.
\end{Def}

\begin{Def}
	$x_1, x_2, ..., x_n \in X$ "--- линейно независимы, если
	\[ \lambda_1x_1 + \lambda_2x_2 + \dots + \lambda_nx_n = 0 \]
	только при
	\[ \lambda_1 = \lambda_2 = \dots = \lambda_n = 0 \]
\end{Def}

\begin{Def}
	Размерность пространства "--- максимальное число линейно независимых векторов
\end{Def}

\begin{Def}
	$x_1, x_2, ..., x_n \in X$ "--- базис, если
	\begin{enumerate}
	\item
		$x_1, x_2, ..., x_n$ линейно независимы
	\item
		\[ \forall x \in X, \exists \lambda_1, \lambda_2, \dots, \lambda_n \colon x = \lambda_1x_1 + \lambda_2x_2 + \dots + \lambda_nx_n \]
	\end{enumerate}
\end{Def}

\begin{exmp}
$X = \R^n$.

$\left\{
\begin{pmatrix}
1\\
0\\
0\\
\vdots \\
0\\
\end{pmatrix},
\begin{pmatrix}
0\\
1\\
0\\
\vdots \\
0\\
\end{pmatrix},
\dots,
\begin{pmatrix}
0\\
0\\
0\\
\vdots \\
1\\
\end{pmatrix}
\right\} $ "--- базис, потому что любой вектор 
$\begin{pmatrix}
	a_1\\
	a_2\\
	\vdots \\
	a_n 
\end{pmatrix}$ можно представить как
\[ a_1
\begin{pmatrix}
1\\
0\\
0\\
\vdots \\
0\\
\end{pmatrix}
+a_2
\begin{pmatrix}
0\\
1\\
0\\
\vdots \\
0\\
\end{pmatrix}
+\dots+a_n
\begin{pmatrix}
0\\
0\\
0\\
\vdots \\
1\\
\end{pmatrix} \]
\end{exmp}

\begin{Def}
	$X$, $Y$ "--- линейные пространства.
	$A\colon X \ra Y$ "--- линейный оператор, если
	\begin{enumerate}
		\item $\forall x, y \in X, A(x + y) = A(x) + A(y)$
		\item $\forall x \in X, \forall \lambda \in \R, A(\lambda x) = \lambda A(x)$
	\end{enumerate}
\end{Def}

\underline{Свойства:}
\begin{enumerate}
\item
	$\forall A, A(0_x) = 0_y$
	\begin{proof}
		$\lambda = 0, A(0x) = 0A(x)$
	\end{proof}

\item
	\[ A\left(\sum_{k = 1}^n \lambda_k x_k\right) = \sum_{k = 1}^n \lambda_k A\left(x_k\right) \]
	\begin{proof}
		\[ A\left(\sum_{k = 1}^n \lambda_k x_k\right) = \sum_{k = 1}^n A\left(\lambda_k x_k\right) = \sum_{k = 1}^n \lambda_k A\left(x_k\right) \]
	\end{proof}

\item
	Множество линейных операторов "--- линейное пространство
	\begin{proof}
		\begin{enumerate}
			\item $\forall A, B, (A + B)(x) \equiv (B + A)(x)$
			\item $\forall A, B, C, ((A + B) + C)(x) \equiv (A + (B + C))(x)$
			\item $\exists O \colon \forall A, (A + O)(x) \equiv A(x) + O(x)$
			\item $\forall A, \exists -A \colon (A + -A)(x) \equiv O(x)$
			\item $\forall \alpha, \beta, A, \alpha(\beta A)(x) \equiv (\alpha \beta)A(x)$
			\item $\forall A, (1 \cdot A)(x) \equiv A(x)$
			\item $\forall \alpha, \beta, A, (\alpha + \beta)A(x) \equiv \alpha A(x) + \beta A(x)$
			\item $\forall \alpha, A, B, \alpha(A + B)(x) \equiv \alpha A(x) + \alpha B(x)$
		\end{enumerate}
	Проверьте сами
\end{proof}
\end{enumerate}

\begin{Def}
	Композиция линейных операторов $A \colon X \ra Y$ и $B \colon Y \ra Z$:
	\[ B \circ A \colon X \ra Z \quad (B \circ A)(x) = B(A(x)) \]
\end{Def}
\begin{theorem}
	Композиция линейных операторов "--- линейный оператор.
\end{theorem}
\begin{proof}
	\begin{enumerate}
		\item $(B \circ A)(x + y)  = B(A(x + y)) = B(A(x) + A(y)) = B(A(x)) + B(A(y))  = (B \circ A)(x) + (B \circ A)(y)$
		\item $(B \circ A)(\lambda x) = B(A(\lambda x)) = B(\lambda A(x)) = \lambda B(A(x)) = \lambda (A \circ B) (x)$
	\end{enumerate}
\end{proof}

\begin{Def}
	$A \colon X \ra Y$, $B \colon Y \ra X$.
	$B$ "--- обратный оператор к $A$, если:
	\begin{enumerate}
		\item $\forall x, (B \circ A)(x) = x$
		\item $\forall y, (A \circ B)(y) = y$
	\end{enumerate}
	\[ B = A^{-1} \]
\end{Def}

\underline{Свойства:}
\begin{enumerate}
	\item Если обратный оператор существует, то он единственный
	\item $(\lambda A)^{-1} = \frac{1}{\lambda}A^{-1}$
	\begin{proof}
		\[ \left(\frac{1}{\lambda}A^{-1}\right) \circ \left(\lambda A\right)(x) = \frac{1}{\lambda} \lambda\left(A^{-1} \circ A\right)(x) = x \]
	\end{proof}
\end{enumerate}

\begin{theorem}
	Теперь рассмотрим случай $Y = X$.

	$A \colon X \to X$
	 
	Множество обратимых операторов образует группу относительно композиции.
\end{theorem}
\begin{proof}
	\begin{enumerate}
	\item
		$\exists E \colon E(x) = x$
	\item
		Проверим замкнутость операций. Для этого докажем, что
		\[ (A \circ B) ^ {-1} (x) = (B^{-1} \circ A^{-1}) (x) \]
		\begin{proof}
			\[ B^{-1} \circ A^{-1} \circ (A \circ B) (x) = B^{-1}(A^{-1}(A(B(x))) = B^{-1}(B(x)) = x \]
		\end{proof}
	\end{enumerate}
\end{proof}

Частный случай $X = \R^n$, $Y = \R^m$, $A \colon \R^n \ra \R^m$.
\begin{gather*}
	\bar x = x_1 \bar e_1 + \dots + x_n \bar e_n \\
	A(\bar x) = A(x_1 \bar e_1 + \dots + x_n \bar e_n) = x_1A \bar e_1 + \dots + x_nA \bar e_n
\end{gather*}
Напишем это же равенство через матрицы:
\[ A(\bar x) = 
\begin{pmatrix}
a_{1, 1} & \dots & a_{1, n}\\
\vdots & \ddots & \vdots\\
a_{m, 1} & \dots & a_{m, n}\\
\end{pmatrix}
\begin{pmatrix}
x_1\\
\vdots\\
x_n\\
\end{pmatrix} = 
\begin{pmatrix}
a_{1, 1}x_1 + \dots + a_{1, n}x_n\\
\vdots\\
a_{m, 1}x_1 + \dots + a_{m, n}x_n\\
\end{pmatrix} = x_1
\begin{pmatrix}
a_{1, 1}\\
\vdots\\
a_{m, 1}\\
\end{pmatrix} + \dots + x_n
\begin{pmatrix}
a_{1, n}\\
\vdots\\
a_{m, n}\\
\end{pmatrix}
\]

\begin{Def}
	Норма оператора:
	$X$ и $Y$ "--- нормированные пространства (заданы $\|\|_x$ и $\|\|_y$).
	$A \colon X \ra Y$ "--- линейный оператор.
	\[ \|A\| = \sup_{\|x\|_x \le 1} \|A(x)\|_y \]
	или то же самое
	\[ \|A\| = \sup \{\|A(x)\|_y \mid \|x\|_x \le 1\} \]
\end{Def}
\begin{Def}
	Если $\|A\| < +\infty$, то $A$ "--- ограниченный оператор.
\end{Def}

\underline{Свойства:}
\begin{enumerate}
\item
	$\|A + B\| \le \|A\| + \|B\|$
	\begin{proof}
		$\|A + B\| = \sup \|(A + B)(x)\|_y = \sup \|A(x) + B(x)\|_y \le \sup \|A(x)\|_y + \sup \|B(x)\|_y = \|A\| + \|B\|$
	\end{proof}

\item
	$\|\lambda A\| = |\lambda| \cdot \|A\|$
	\begin{proof}
		$\|\lambda A\| = \sup \| (\lambda A)(x) \|_y = \sup \| \lambda A(x) \|_y = \sup( |\lambda| \cdot \|A(x) \|_y) = |\lambda| \sup \|A(x)\|_y = |\lambda| \cdot \|A\|$
	\end{proof}

\item
	$\|A\| = 0 \Ra A$ "--- нулевой оператор
	\begin{proof}
		$||A|| = 0 \Ra \sup ||A(x)||_y = 0 \Ra ||A(x)||_y = 0, \forall x \colon ||x||_x \le 1$
		
		Пусть $||\widetilde{x}||_x > 1$, тогда возьмём $x = \frac{\widetilde{x}}{||\widetilde{x}||_x}$ и $||x||_x = 1$
		
		$A(\widetilde{x}) = A(||\widetilde{x}||_x \cdot x) = ||\widetilde{x}||_x \cdot A(x) = 0$
	\end{proof}

\item
	Норма для операторов, которую мы задали, действительно является нормой в линейном пространстве операторов.
\end{enumerate}

\begin{Rem}
	Ограниченный оператор не означает ограниченное отображение. Если $A \colon X \ra Y$ "--- ограниченное отображение, то $A = O$.
\end{Rem}
\begin{proof}
	Пусть $A \ne O$
	\begin{gather*}
		\exists x_0 \in X \colon A(x_0) \ne 0_y \Ra \|A(x_0)\|_y \ne 0 \\
		\|A(\lambda x_0)\|_y = |\lambda| \cdot \|Ax_0\|_y
	\end{gather*}
	Если $A$ "--- ограниченное отображение, то $A(\lambda x_0)$ "--- ограниченное множество, поэтому
	\[ \forall \lambda \in \R, \|A(\lambda x_0)\| \le M \]
	но
	\[ \|A(\lambda x_0)\| = |\lambda| \cdot \|Ax_0\| \]
	"--- противоречие
\end{proof}

\begin{theorem}
$A \colon X \rightarrow Y$ --- линейный оператор\\
$X, Y$ --- нормированное пространство\\
Тогда
\[\sup_{||x||_x \le 1} ||A(x)||_y = \sup_{||x||_x < 1} ||A(x)||_y = \sup_{||x||_x = 1} ||A(x)||_y = \sup_{x \neq 0} \frac{||A(x)||_y}{||x||_x} = \inf \{c \colon ||A(x)||_y \le c||x||_x, \forall x \in X \}\]
\end{theorem}
\begin{proof}
Для удобства обозначим части равенств $N_1, N_2, N_3, N_4, N_5$ соответственно\\
$N_1 \ge N_2$ очевидно\\
$N_1 \ge N_3$ очевидно\\
\[N_3 = N_4\]
$\frac{||A_x||_y}{||x||_x} = \frac{1}{||x||_x} \cdot ||A(x)||_y = ||A(\frac{x}{||x||_x})||_y$ \\
$\frac{x}{||x||_x}$ --- единичный вектор\\
\[N_4 = N_5\]
$\inf\{c\colon ||A(x)||_y \le c||x||_x,\forall x \in X\} = \inf \{c\colon ||A(x)||_y \le c||x||_x, \forall x \in X, x \neq 0\} = \inf \{c\colon \frac{||A(x)||_y}{||x||_x} \le c, \forall x \in X, x \neq 0\}$ --- наименьшая из верхних границ для $\frac{||A(x)||_y}{||x||_x}$ при $x \neq 0$, что является $\sup$ по определению\\
\[N_2 \ge N_1\]
Докажем, что \\
$(1+ \epsilon)N_2 \ge N_1,  \forall \epsilon > 0$
$(1+ \epsilon)N_2 = sup_{||x||_x < 1} ||A(x)||_y \cdot (1 + \epsilon)$ \\
Зафиксируем $x \in X, ||x||_x \le 1$
$\widetilde{x} = \frac{x}{1 + \epsilon}$\\
$||\widetilde{x}||_x = ||\frac{x}{1 + \epsilon}||_x = \frac{1}{1 + \epsilon} ||x||_x \le \frac{1}{1 + \epsilon} < 1$\\
$||A(x)||_y = ||A((1 + \epsilon)\widetilde{x})||_y = (1 + \epsilon)||A(\widetilde{x})||_y \le (1+\epsilon)N_2$
\[N_3 \ge N_1 \]
$||A(x)||_y = ||A(||x||_x \cdot \frac{x}{||x||_x})||_y = ||x||_x \cdot ||A(\frac{x}{||x||_x})||_y \le ||A(\frac{x}{||x||_x})||_y \le N_3$
\end{proof}
\begin{conseq} 
\begin{enumerate}
\item $||A(x)|| _ y \le ||A|| \cdot ||x||_x, \forall x \in X$ \\
\begin{proof}
$||A|| = \sup_{x \neq 0} \frac{||A(x)||_y}{||x||_x}$
\end{proof}
\item $||B \circ A|| \le ||B|| \cdot ||A||$\\
\begin{proof}
$||(B \circ A)(x)||_z = ||B(A(x))||_z \le ||B|| \cdot ||A(x)||_y \le ||B|| \cdot ||A|| \cdot ||x||_x$ \\
$||B \circ A|| = \sup_{||x||_x \le 1} ||(B\circ A)(x)||_z \le ||B|| \cdot ||A||$
\end{proof}
\end{enumerate}
\end{conseq}
\begin{theorem}
$A \colon X \rightarrow Y$ --- линейный оператор\\
Следующие условия равносильны:
\begin{enumerate}
\item $A$ ограниченный оператор
\item $A$ непрерывен в 0
\item $A$ непрерывен
\item $A$ равномерно непрерывен
\end{enumerate}
\end{theorem}
\begin{proof}
$4\Rightarrow 3 \Rightarrow 2$ --- очевидно\\
\[1 \Rightarrow 4\]
$||A(x_1) - A(x_2)||_y = ||A(x_1 - x_2)||_y \le ||A|| \cdot ||x_1 - x_2||_x$ \\
$\forall \epsilon > 0, \exists \delta > 0 \colon \forall x_1, x_2, ||x_1 - x_2||_x < \delta \Rightarrow ||A(x) - A(y)||_y < \epsilon$ \\
Возьмём $\delta = \frac{\epsilon}{||A||} $\\
$||x_1 - x_2||_x < \frac{\epsilon}{||A||} \Rightarrow ||A(x_1) - A(x_2)||_y \le ||A|| \cdot ||x_1 - x_2||_x < ||A|| \cdot \frac{\epsilon}{||A||} = \epsilon$
\[2 \Rightarrow 1\]
$\forall \epsilon > 0, \exists \delta > 0 \colon ||x - 0||_x < \delta \Rightarrow ||A(x) - A(0)||_y < \epsilon$ \\
Возьмём $\epsilon = 1$ и соответствующий ему $\delta$\\
$||x - 0||_x < \delta \Rightarrow ||A(x) - A(0)||_y < 1$
Возьмём $x\in X \colon ||x||_x < 1$, тогда \\
$\widetilde{x} = \delta x$\\
$||\widetilde{x}||_x < \delta \Rightarrow 1 > ||A \widetilde{x}||_y = ||A(\delta x)||_y = \delta ||A(x)||_y$\\
$||A(x)||_y < \frac{1}{\delta}, \forall x \in X \colon ||x||_x < 1$\\
$\sup_{||x||_x < 1} ||A(x)||_y \le \frac{1}{\delta}$
\end{proof}
\begin{theorem}
$A \colon \R^n \rightarrow \R^m$\\
\[||A|| \le (\sum_{i = 1}^m \sum_{j = 1}^n a_{i, j}^2)^{\frac{1}{2}}\]
\end{theorem}
\begin{proof}
$x = x_1e_1 + \dots + x_n e_n$ \\
$||x||^2 = x_1^2 + \dots + x_n^2$ \\
$||A(x)|| = ||A(x_1e_1 + \dots + x_n e_n)|| = ||x_1A(e_1) + \dots + x_nA(e_n)|| \le |x_1| \cdot ||A(e_1)|| + \dots + |x_n| \cdot ||A(e_n)||$ \\
$||A(x)||^2 \le (|x_1| \cdot ||A(e_1)|| + \dots + |x_n| \cdot ||A(e_n)||) ^ 2 \le (|x_1| ^ 2 + \dots + |x_n|^2) \cdot (||A(e_1)||^2 + \dots + ||A(e_n)||^2) = ||x^2|| (||A(e_1)||^2 + \dots + ||A(e_n)||^2) = ||x||^2 \cdot \sum_{i = 1}^m \sum_{j = 1}^n a_{i, j}^2$\\
$A(e_k) = 
\begin{pmatrix}
a_{1, 1} & \dots & a_{1, n}\\
\vdots & \ddots & \vdots\\
a_{m, 1} & \dots & a_{m, n}\\
\end{pmatrix}
\begin{pmatrix}
0\\
\vdots\\
0\\
1\\
0\\
\vdots\\
0\\
\end{pmatrix} = 
\begin{pmatrix}
a_{1, k}\\
\vdots\\
a_{m, k}\\
\end{pmatrix}$\\
$||A(e_k)||^2 = a_{1, k}^2 + \dots + a_{m, k}^2$\\
$||A(x)|| \le ||x||(\sum_{i = 1}^m \sum_{j = 1}^n a_{i, j}^2)^{\frac{1}{2}} \le (\sum_{i = 1}^m \sum_{j = 1}^n a_{i, j}^2)^{\frac{1}{2}}$ при $||x|| \le 1$\\
$\sup_{||x|| \le 1}||A(x)|| \le (\sum_{i = 1}^m \sum_{j = 1}^n a_{i, j}^2)^{\frac{1}{2}}$
\end{proof}
\begin{Def}
$X$ --- линейное пространство\\
Две нормы называются эквивалентными, если $\exists c_1, c_2 > 0 \colon$\\
$c_1||x||_1 \le ||x||_2 \le c_2||x||_1 \forall x \in X$\\
\end{Def}
\begin{Rem}


\begin{enumerate}
\item Если нормы эквивалентны, то сходимости по этим нормам означают одно и то же

Пусть $x_n \rightarrow x_0$ по норме  $1$, тогда

$||x_n - x_0||_2 \le c_2||x_n - x_0||_1$, но $x_n - x_0 \rightarrow 0 \Rightarrow x_n \rightarrow x_0$ по норме 2

\item Непрерывности относительно эквивалентных норм совпадают
\end{enumerate}
\end{Rem}

\begin{theorem}
В  $R^d$ все нормы эквивалентны
\end{theorem}

\begin{proof}
$||x|| = \sqrt{x_1^2 + \dots + x_n^2}$

Докажем, что все нормы эквивалентны данной

Пусть $p$ --- другая норма

$p(x) = p(\sum_{k = 1}^d x_k e_k) \le \sum_{k = 1}^d |x_k| p(e_k) \le (\sum_{k = 1}^d |x_k|^2)^{\frac{1}{2}} (\sum_{k = 1}^d p(e_k)^2)^{\frac{1}{2}} = ||x|| \cdot (\sum_{k = 1}^d p(e_k)^2)^{\frac{1}{2}}$

Возьмём $c_2 = (\sum_{k = 1}^d p(e_k)^2)^{\frac{1}{2}}$

$p(x) \le c_2 ||x||$

$|p(x) - p(y)| \le p(x - y) \le c_2 ||x - y|| \Ra p$ --- непрерывная функция

$S^{d - 1} = \{x\in \R^d \colon ||x|| = 1\}$ --- сфера в $n$-мерном пространстве $\Ra$ компакт (ограничена и замкнута) $\Ra$ по теореме Вейерштрасса, $p$ достигает наименьшее значение $p(y)$

$p(y) \neq 0$, потому что иначе $y = 0 \Ra y \notin S^{d - 1}$

Пусть $c_1 = p(y) > 0$, тогда

$с_1 \le p(\frac{x}{||x||}) = p(\frac{1}{||x||} \cdot x) = \frac{1}{||x||} \cdot p(x)$

$c_1||x|| \le p(x)$
\end{proof}

\section{Дифференцируемость в $\R^n$}

\begin{Def}
$f \colon D \subset \R^n \ra \R^m, a \in \Int D$

$f$ дифференцируема в точке а, если

$\exists T \colon \R^n \ra \R^m$ --- линейный оператор, такой что

$f(a + h)_{h \ra 0} = f(a) + T(h) + o(||h||)$
\end{Def}
\begin{Rem}
$T$ определен однозначно, зафиксируем $h$, $t > 0$

$f(a + th)_{t \ra 0} = f(a) + T(th) + o(||th||) = f(a) + T(th) + o(t)$

$T(h) + o(1) = \frac{f(a + th) - f(a)}{t}$

$T(h) = \lim_{t \ra 0} \frac{f(a + th) - f(a)}{t}$
\end{Rem}
\begin{Def}
Матрица, соответствующая оператору $T$ "--- матрица Якоби
\end{Def}
Частный случай $m = 1$

$f \colon D \subset R^n \ra R$

$f(a + h) = f(a) + T(h) + o(||h||)$

$T \colon \R^n \ra \R$

$T(h) = 
\begin{pmatrix}
t_1 & \dots & t_n
\end{pmatrix}
\begin{pmatrix}
h_1\\
\vdots \\
h_n
\end{pmatrix} = t_1h_1 + \dots + t_nh_n = \left< V, h \right>$

$V = 
\begin{pmatrix}
t_1\\
\vdots \\
t_n
\end{pmatrix}$

$\exists V \in \R^n \colon f(a + h) = f(a) + \left< V, h \right> + o(||h||)$

\begin{Def}
$V$ "--- градиент функции $f$

$V = grad f = \nabla f$
\end{Def}

\begin{theorem}
$f \colon D \subset \R^n \ra \R^m, a \in \Int D$

$f = 
\begin{pmatrix}
f_1\\
\vdots\\
f_m
\end{pmatrix}$

\[f \text{ дифференцируема в точке } a \Leftrightarrow f_k \text{ дифференцируема в точке } a \forall k = 1, \dots, m\]
\end{theorem} 

\begin{proof}
$f(a + h) = f(a) + T(h) + o(||h||)$

$\Ra$

$f_k(a + h) = f_k(a) + (T(h))_k + o(||h||)$

$\La$

$\alpha_k(h) = f_k(a+h) - f_k(a) - \left<V_k, h\right> = o(||h||)$

$\frac{\alpha_k(h)}{||h||} \ra 0$

$\alpha(h) = f(a+h) - f(a) - T(h) = o(||h||)$

$\alpha(h) = 
\begin{pmatrix}
\alpha_1(h)\\
\vdots\\
\alpha_n(h)
\end{pmatrix}$

$||\alpha(h)|| = \sqrt{\alpha_1^2(h) +\dots + \alpha_n^2(h)}$

$\frac{||\alpha(h)||}{||h||} = \sqrt{(\frac{||\alpha_1(h)||}{||h||}) ^ 2 + \dots + (\frac{||\alpha_n(h)||}{||h||}) ^ 2} \ra 0$, потому что

$\frac{\alpha_k(h)}{||h||} \ra 0$
\end{proof}


\setauthor{Кравченко Юрий}
\chapter{Функции нескольких переменных}
\section{Линейные операторы}

\begin{Def}
	$X$ "--- линейное (векторное) пространство над $\R$, $x_1, x_2, \dots, x_n \in X$, $\lambda_1, \lambda_2, \dots, \lambda_n \in \R$.
	$\lambda_1x_1 + \lambda_2x_2 + \dots + \lambda_nx_n$ "--- линейная комбинация векторов $x_1, x_2, \dots, x_n$.
\end{Def}

\begin{Def}
	$x_1, x_2, ..., x_n \in X$ "--- линейно независимы, если
	\[ \lambda_1x_1 + \lambda_2x_2 + \dots + \lambda_nx_n = 0 \]
	только при
	\[ \lambda_1 = \lambda_2 = \dots = \lambda_n = 0 \]
\end{Def}

\begin{Def}
	Размерность пространства "--- максимальное число линейно независимых векторов
\end{Def}

\begin{Def}
	$x_1, x_2, ..., x_n \in X$ "--- базис, если
	\begin{enumerate}
	\item
		$x_1, x_2, ..., x_n$ линейно независимы
	\item
		\[ \forall x \in X, \exists \lambda_1, \lambda_2, \dots, \lambda_n \colon x = \lambda_1x_1 + \lambda_2x_2 + \dots + \lambda_nx_n \]
	\end{enumerate}
\end{Def}

\begin{exmp}
$X = \R^n$.

$\left\{
\begin{pmatrix}
1\\
0\\
0\\
\vdots \\
0\\
\end{pmatrix},
\begin{pmatrix}
0\\
1\\
0\\
\vdots \\
0\\
\end{pmatrix},
\dots,
\begin{pmatrix}
0\\
0\\
0\\
\vdots \\
1\\
\end{pmatrix}
\right\} $ "--- базис, потому что любой вектор 
$\begin{pmatrix}
	a_1\\
	a_2\\
	\vdots \\
	a_n 
\end{pmatrix}$ можно представить как
\[ a_1
\begin{pmatrix}
1\\
0\\
0\\
\vdots \\
0\\
\end{pmatrix}
+a_2
\begin{pmatrix}
0\\
1\\
0\\
\vdots \\
0\\
\end{pmatrix}
+\dots+a_n
\begin{pmatrix}
0\\
0\\
0\\
\vdots \\
1\\
\end{pmatrix} \]
\end{exmp}

\begin{Def}
	$X$, $Y$ "--- линейные пространства.
	$A\colon X \ra Y$ "--- линейный оператор, если
	\begin{enumerate}
		\item $\forall x, y \in X, A(x + y) = A(x) + A(y)$
		\item $\forall x \in X, \forall \lambda \in \R, A(\lambda x) = \lambda A(x)$
	\end{enumerate}
\end{Def}

\underline{Свойства:}
\begin{enumerate}
\item
	$\exists O \colon \forall x \in X, O(x) = 0_y$
	\begin{proof}
		$\lambda = 0, A(0x) = 0A(x)$
	\end{proof}

\item
	\[ A\left(\sum_{k = 1}^n \lambda_k x_k\right) = \sum_{k = 1}^n \lambda_k A\left(x_k\right) \]
	\begin{proof}
		\[ A\left(\sum_{k = 1}^n \lambda_k x_k\right) = \sum_{k = 1}^n A\left(\lambda_k x_k\right) = \sum_{k = 1}^n \lambda_k A\left(x_k\right) \]
	\end{proof}

\item
	Множество линейных операторов "--- линейное пространство
	\begin{proof}
		\begin{enumerate}
			\item $\forall A, B, (A + B)(x) \equiv (B + A)(x)$
			\item $\forall A, B, C, ((A + B) + C)(x) \equiv (A + (B + C))(x)$
			\item $\exists O \colon \forall A, (A + O)(x) \equiv A(x) + O(x)$
			\item $\forall A, \exists -A \colon (A + -A)(x) \equiv O(x)$
			\item $\forall \alpha, \beta, A, \alpha(\beta A)(x) \equiv (\alpha \beta)A(x)$
			\item $\forall A, (1 \cdot A)(x) \equiv A(x)$
			\item $\forall \alpha, \beta, A, (\alpha + \beta)A(x) \equiv \alpha A(x) + \beta A(x)$
			\item $\forall \alpha, A, B, \alpha(A + B)(x) \equiv \alpha A(x) + \alpha B(x)$
		\end{enumerate}
	Проверьте сами
\end{proof}
\end{enumerate}

\begin{Def}
	Композиция линейных операторов $A \colon X \ra Y$ и $B \colon Y \ra Z$:
	\[ B \circ A \colon X \ra Z \quad (B \circ A)(x) = B(A(x)) \]
\end{Def}
\begin{theorem}
	Композиция линейных операторов "--- линейный оператор.
\end{theorem}
\begin{proof}
	\begin{enumerate}
		\item $(B \circ A)(x + y)  = B(A(x + y)) = B(A(x) + A(y) = B(A(x)) + B(A(y))  = (B \circ A)(x) + (B \circ A)(y)$
		\item $(B \circ A)(\lambda x) = B(A(\lambda x)) = B(\lambda A(x)) = \lambda B(A(x)) = \lambda (A \circ B) (x)$
	\end{enumerate}
\end{proof}

\begin{Def}
	$A \colon X \ra Y$, $B \colon Y \ra X$.
	$B$ "--- обратный оператор к $A$, если:
	\begin{enumerate}
		\item $\forall x, (B \circ A)(x) = x$
		\item $\forall y, (A \circ B)(y) = y$
	\end{enumerate}
	\[ B = A^{-1} \]
\end{Def}

\underline{Свойства:}
\begin{enumerate}
	\item Если обратный оператор существует, то он единственный
	\item $(\lambda A)^{-1} = \frac{1}{\lambda}A^{-1}$
	\begin{proof}
		\[ \left(\frac{1}{\lambda}A^{-1}\right)\left(\left(\lambda A\right) ^ {-1} \right)(x) = \frac{1}{\lambda} \lambda\left(A^{-1} \circ A\right)(x) = x \]
	\end{proof}
\end{enumerate}

Теперь рассмотрим случай $Y = X$.
\begin{theorem}
	Множество обратимых операторов образует группу относительно композиции
\end{theorem}
\begin{proof}
	\begin{enumerate}
	\item
		$\exists E \colon E(x) = x$
	\item
		Проверим замкнутость операций. Для этого докажем, что
		\[ (A \circ B) ^ {-1} (x) = (B^{-1} \circ A^{-1}) (x) \]
		\begin{proof}
			\[ B^{-1} \circ A^{-1} \circ (A \circ B) (x) = B^{-1}(A^{-1}(A(B(x))) = B^{-1}(B(x)) = x \]
		\end{proof}
	\end{enumerate}
\end{proof}

Частный случай $X = \R^n$, $Y = \R^m$, $A \colon \R^n \ra \R^m$.
\begin{gather*}
	\bar x = x_1 \bar e_1 + \dots + x_n \bar e_n \\
	A(\bar x) = A(x_1 \bar e_1 + \dots + x_n \bar e_n) = x_1A \bar e_1 + \dots + x_nA \bar e_n
\end{gather*}
Напишем это же равенство через матрицы:
\[ A(\bar x) = 
\begin{pmatrix}
a_{1, 1} & \dots & a_{1, n}\\
\vdots & \ddots & \vdots\\
a_{m, 1} & \dots & a_{m, n}\\
\end{pmatrix}
\begin{pmatrix}
x_1\\
\vdots\\
x_n\\
\end{pmatrix} = 
\begin{pmatrix}
a_{1, 1}x_1 + \dots + a_{1, n}x_n\\
\vdots\\
a_{m, 1}x_1 + \dots + a_{m, n}x_n\\
\end{pmatrix} = x_1
\begin{pmatrix}
a_{1, 1}\\
\vdots\\
a_{m, 1}\\
\end{pmatrix} + \dots + x_n
\begin{pmatrix}
a_{1, n}\\
\vdots\\
a_{m, n}\\
\end{pmatrix}
\]

\begin{Def}
	Норма оператора:
	$X$ и $Y$ "--- нормированные пространства (заданы $\|\|_x$ и $\|\|_y$).
	$A \colon X \ra Y$ "--- линейный оператор.
	\[ \|A\| = \sup_{\|x\|_x \le 1} \|A(x)\|_y \]
	или то же самое
	\[ \|A\| = \sup \{\|A(x)\|_y \mid \|x\|_x \le 1\} \]
\end{Def}
\begin{Def}
	Если $\|A\| < +\infty$, то $A$ "--- ограниченный оператор.
\end{Def}

\underline{Свойства:}
\begin{enumerate}
\item
	$\|A + B\| \le \|A\| + \|B\|$
	\begin{proof}
		$\|A + B\| = \sup \|(A + B)(x)\|_y = \sup \|A(x) + B(x)\|_y \le \sup \|A(x)\|_y + \sup \|B(x)\|_y = \|A\| + \|B\|$
	\end{proof}

\item
	$\|\lambda A\| = |\lambda| \cdot \|A\|$
	\begin{proof}
		$\|\lambda A\| = \sup \| (\lambda A)(x) \|_y = \sup \| \lambda A(x) \|_y = \sup( |\lambda| \cdot \|A(x) \|_y) = |\lambda| \sup \|A(x)\|_y = |\lambda| \cdot \|A\|$
	\end{proof}

\item
	$\|A\| = 0 \Ra A$ "--- нулевой оператор
	\begin{proof}
		coming soon
	\end{proof}

\item
	Норма для операторов, которую мы задали, действительно является нормой в линейном пространстве операторов.
\end{enumerate}

\begin{Rem}
	Ограниченный оператор не означает ограниченное отображение. Если $A \colon X \ra Y$ "--- ограниченное отображение, то $A = O$.
\end{Rem}
\begin{proof}
	Пусть $A \ne O$
	\begin{gather*}
		\exists x \in X \colon A(x_0) \ne 0_y \Ra \|A(x_0)\|_y \ne 0 \\
		\|A(\lambda x_0)\|_y = |\lambda| \cdot \|Ax_0\|_y
	\end{gather*}
	Если $A$ "--- ограниченное отображение, то $A(\lambda x_0)$ "--- ограниченное множество, поэтому
	\[ \forall \lambda \in \R, \|A(\lambda x_0)\| \le M \]
	но
	\[ \|A(\lambda x_0)\| = |\lambda| \cdot \|Ax_0\| \]
	"--- противоречие
\end{proof}
coming soon

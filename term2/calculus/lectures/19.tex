\setauthor{Дмитрий Лапшин}

\begin{Def}
	$A$ "--- множество, $x$ "--- точка.
	\[ \rho(x, A) = \inf \{\rho(x, y) \mid y \in A \} \]
\end{Def}

\begin{Def}
	$\delta$-окрестность множества $U_\delta(A)$ (иногда обозначается $A_\delta$):
	\[ U_\delta(A) = \{ x \mid \rho(x, A) < \delta\} = \{x \mid \exists y \in A\colon \rho(x, y) < \delta\} \]
\end{Def}
\begin{Rem}
	$\delta$-окрестность является объединением открытых шаров радиуса $\delta$ с центрами из $A$:
	\[ U_\delta(A) = \bigcup_{x \in A} B_\delta(x) \]
\end{Rem}

\begin{theorem}
	$E$ "--- измерима, тогда $\mu^*(U_\delta(E)) \ra \mu E$ при $\delta \ra 0$.
\end{theorem}
\begin{proof}
	\begin{description}
	\item[$E$ "--- ячейка:]
		\begin{gather*}
			E = (a_1, b_1] \times \dots \times (a_n, b_n] \\
			E \subset U_\delta(A) \subset (a_1 - \delta, b_1 + \delta] \times (a_n - \delta, a_n + \delta] \\
			\prod_{k=1}^n (b_k - a_k) = \mu E \le \mu^* U_\delta(E) \le \mu((a_1 - \delta, b_1 + \delta] \times (a_n - \delta, a_n + \delta])
			= \prod_{k=1}^n (b_k - a_k + 2\delta)
		\end{gather*}

	\item[$E$ "--- клеточное множество:]
		\begin{gather*}
			E = \bigcup_{k=1}^m P_k \\
			U_\delta(E) = \bigcup_{k=1}^m U_\delta(P_k) \\
			\mu E = \mu^* U_\delta(E) \le \sum_{k=1}^m \mu^*(U_\delta(P_k)) \ra \sum_{k=1}^m \mu P_k
		\end{gather*}

	\item[$E$ "--- измеримое:]
		\begin{gather*}
			\forall \epsilon > 0, \exists B_\epsilon \supset E\colon \mu B_\epsilon < \mu E + \epsilon \\
			U_\delta(B_\epsilon) \supset U_\delta(E) \\
			\mu^*(U_\delta(B_\epsilon)) \ra \mu B_\epsilon \\
			\exists \delta_0 > 0\colon \forall \delta < \delta_0, \mu E \le \mu^* U_\delta(E) \le \mu^* U_\delta(B_\epsilon) < \mu B_\epsilon + \epsilon < \mu E + 2 \epsilon
		\end{gather*}
	\end{description}
\end{proof}

\begin{conseq}
	$E$ "--- множество нулевой меры.
	Тогда существует клеточное множество $C_\epsilon$, что $\mu C_\epsilon < \epsilon$ и $E \subset \Int C_\epsilon$.
\end{conseq}
\begin{proof}
	\begin{gather*}
		\mu^*(U_\delta(E)) \ra \mu E = 0 \\
		\exists \delta > 0\colon \mu^*(U_\delta(E)) < \frac\epsilon2
			\Ra \exists C_\epsilon \supset U_\delta(E)\colon \mu C_\epsilon < \mu^* U_\delta(E) + \frac\epsilon2 < \epsilon \\
		\Int C_\epsilon \supset U_\delta(E) \text{ "--- открытое}
	\end{gather*}
\end{proof}

\begin{Def}
	$\partial E$ "--- граница множества:
	\[ \partial E = \{ x \mid \forall r > 0, B_r(x) \cap E \ne \emptyset \land B_r(x) \setminus E \ne \emptyset \} \]
\end{Def}
\begin{Rem}
	Можно показать, что $\partial E = \cl E - \Int E$.
\end{Rem}

\begin{lemma}
	$x \in E$, $y \notin E$.
	Тогда на $[x, y]$ есть точка из границы $E$.
\end{lemma}
\begin{proof}
	$x_0 = x$, $y_0 = y$. Возьмём $m = \frac{x+y}2$. Если она в $E$, то перейдём к $x_1 = m$, $y_1 = y_0$, иначе к $x_1 = x_0$, $y_1 = m$.
	Получим сходящуюся последовательность вложенных отрезков.
	В их пересечении лежит одна точка $c$.
	Тогда для каждой окрестности этой точки есть отрезок $[x_i, y_i]$, начало которого лежит в $E$, а конец "--- не лежит.
\end{proof}

\begin{lemma}
	Пусть $D$ "--- клетчатое множество, а ограниченное множество $E$ такое, что $\partial E \subset D$.
	Тогда $E \cup D$ тоже клетчатое.
\end{lemma}
\begin{proof}
	$Q$ "--- ячейка, $Q \supset E \cup D$.
	\[ Q \setminus D = \bigcup_{i=1}^m P_k \]
	\begin{description}
	\item[$P_k \cap E \ne \emptyset$:]
		Проверим, что $E \supset P_k$.
		Пусть $P_k \setminus E \ne \emptyset$.
		$x \in P_k \cap E$, $y \in P_k \setminus E$, значит на отрезке $[x, y]$ есть граничная точка $E$.
		Этот отрезок, а значит и точка, лежат в ячейке.
		Тогда $P_k \cap \partial E \ne \emptyset$, что неверно, так как $P_k \cap D = \emptyset$.

	\item[$P_k \cap E = \emptyset$:]
		Ничего интересного.
	\end{description}

	Получили, что
	\[ E \cup D = D \cup \bigcup_{P_k \subset E} P_k \]
	<<$\supset$>> очевидно, <<$\subset$>>: взяли точку $x \in E \setminus D$, тогда $x \in Q \setminus D$, значит $\exists k\colon x \in P_k$.
\end{proof}

\begin{theorem}[Критерий измеримости]
	$E$ "--- ограниченное множество.
	Тогда $E$ измеримо тогда и только тогда, когда граница имеет нулевую меру.
\end{theorem}
\begin{proof}
	\begin{description}
	\item[$\Ra$:]
		$E$ "--- измеримо, значит
		\begin{gather*}
			\forall \epsilon > 0, \exists A_\epsilon, B_\epsilon \text{ "--- клеточные}\colon A_\epsilon \subset E \subset B_\epsilon
				\land\mu B_\epsilon - \mu A_\epsilon < \epsilon \\
			\partial E = \cl E \setminus \Int E \subset \cl B_\epsilon \setminus \Int A_\epsilon \\
			\tilde A_\epsilon \text{ "--- клеточное, вложенное в } \Int A_\epsilon \quad \mu \tilde A_\epsilon > \mu A_\epsilon - \epsilon \\
			\tilde B_\epsilon \text{ "--- клеточное, содержащее } \cl B_\epsilon \quad \mu \tilde B_\epsilon < \mu B_\epsilon + \epsilon\\
			\partial E \subset \tilde B_\epsilon \setminus \tilde A_\epsilon \\
			\mu (\tilde B_\epsilon \setminus \tilde A_\epsilon) \le \mu \tilde B_\epsilon - \mu \tilde A_\epsilon
				< \mu B_\epsilon + \epsilon - \mu A_\epsilon + \epsilon < 3\epsilon \\
			\Ra \mu^* \partial E = 0 \Ra \mu (\partial E) = 0
		\end{gather*}
	\item[$\La$:]
		\begin{gather*}
			\mu (\partial E) = 0 \Ra \exists C_\epsilon \text{"--- клеточное}\colon
				\mu C_\epsilon < \epsilon \land C_\epsilon \supset \partial E \\
			\Ra \text{$E \cup C_\epsilon$ "--- клеточное} \\
			B_\epsilon = E \cup C_\epsilon \quad A_\epsilon = B_\epsilon \setminus C_\epsilon = E \setminus C_\epsilon \\
			B_\epsilon \supset E \supset A_\epsilon \\
			\mu B_\epsilon = \mu((B_\epsilon \setminus C_\epsilon) \sqcup C_\epsilon)
				= \mu (A_\epsilon \cup C_\epsilon) \le \mu A_\epsilon + \mu C_\epsilon < \mu A_\epsilon + \epsilon
		\end{gather*}
	\end{description}
\end{proof}

\begin{lemma}
	$E, F \subset \R^n$.
	Тогда
	\begin{align*}
		\partial (E \cup F) &\subset \partial E \cup \partial F \\
		\partial (E \cap F) &\subset \partial E \cup \partial F \\
		\partial (E \setminus F) &\subset \partial E \cup \partial F
	\end{align*}
\end{lemma}
\begin{proof}
	Возьмём $x \in \partial(E \cup F)$, но $x \notin \partial F$.
	Покажем, что она в $\partial E$.
	Хотим увидеть это:
	\begin{align*}
		B_r(x) \cap (E \cup F) &\ne \emptyset & (B_r(x) \cap E) \cup (B_r(x) \cap F) &\ne \emptyset \\
		B_r(x) \setminus (E \cup F) &\ne \emptyset & (B_r(x) \setminus E) \cap (B_r(x) \setminus F) &\ne \emptyset
	\end{align*}
	Покажем:
	\begin{gather*}
		B_r(x) \setminus F \ne \emptyset \quad B_r(x) \setminus E \ne \emptyset \\
		x \notin \partial F \Ra B_r(x) \cap F = \emptyset \quad \text{для некоторого $r$} \Ra B_r(X) \cap E \ne \emptyset
	\end{gather*}

	Остальное аналогично можно поупражняться.
\end{proof}

\begin{conseq}
	$E$, $F$ измеримы.
	Тогда измеримы $\Int E$, $\cl F$, $E \cup F$, $E \cap F$, $E \setminus F$.
\end{conseq}
\begin{proof}
	Хотим понять, что
	\begin{align*}
		\partial (\Int E) &\subset \partial E \\
		\partial (\cl  E) &\subset \partial E
	\end{align*}
	Тогда победим из того, что $\mu (\partial E) = 0$.
	\begin{alignat*}{8}
		\partial (\Int E) &= \cl \Int E \setminus \Int \Int E &=& \cl \Int E \setminus \Int E &\subset& \cl E \setminus \Int E &=& \partial E \\
		\partial (\cl  E) &= \cl \cl  E \setminus \Int \cl  E &=& \cl  E \setminus \Int \cl E &\subset& \cl E \setminus \Int E &=& \partial E \\
	\end{alignat*}

	Если $E$ и $F$ имзеримы, то мера объединения их границ нулевая, а тогда по лемме всё остальное тоже имеет нулевую меру границы.
\end{proof}

\begin{theorem}
	$E$, $F$ измеримы.
	Тогда
	\begin{enumerate}
		\item (монотонность меры) $E \subset F \Ra 0 \le \mu E \le \mu F$
		\item (полуаддитивность меры) $\mu (E \cup F) \le \mu E + \mu F$
		\item (аддитивность меры) $E \cap F = \emptyset \Ra \mu(E \sqcup F) = \mu E + \mu F$
	\end{enumerate}
\end{theorem}
\begin{proof}
	\begin{enumerate}
	\item
		Знаем для $\mu^*$ и $\mu = \mu^*$.

	\item
		Знаем для $\mu^*$ и $\mu = \mu^*$.

	\item
		$A \subset E$, $B \subset F$ "--- клеточные, $A \cap F \ne \emptyset$.
		Тогда $A \sqcup B \subset E \sqcup F$.
		\begin{gather*}
			\mu E + \mu F \ge \mu(E \cup F) \ge \mu(A \cup B) = \mu A + \mu B \\
			\mu A + \mu B \le \mu (E \cup F) \le \mu E + \mu F
		\end{gather*}
		Перешли к супремумам по $A$ и $B$.
		\[
			\mu E + \mu F = \mu_* E + \mu_* F \le \mu (E \cup F) \le \mu E + \mu F
		\]
	\end{enumerate}
\end{proof}

\begin{Rem}
	Аддитивности верхней и нижней меры нет.
	\begin{gather*}
		E = [0, 1] \cap \Q \quad [0, 1] \setminus Q \\
		\mu^* E = \mu^* F = 1 \quad \mu_* E = \mu_* F = 0 \\
		\mu_*(E \cup F) = \mu^* (E \cup F) = 1
	\end{gather*}
\end{Rem}
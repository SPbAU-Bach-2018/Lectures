\chapter{Интегральное исчисление (продолжение)}
\section{Определенный интеграл}
\setauthor{Анастасия Гайдашенко}

\begin{Def}
	$\mathcal{F}$ "--- множество ограниченных подмножеств $\R^2$
\end{Def}

\begin{Def}
	Площадь:
	\[ S\colon \mathcal{F} \ra [0; +\infty) \]
	Должны выполняться следующие свойства:

	\begin{enumerate}
	\item
		$E_1 \cap E_2 = \emptyset \Ra S(E_1 \cup E_2) = S(E_1) + S(E_2)$
		\begin{conseq*}
			$E_1 \subset E_2 \Ra S(E_1) \le S(E_2)$
		\end{conseq*}
		\begin{proof}
			$E_2 = E_1 \cup (E_2 \setminus E_1); S(E_2) = S(E_1) + S(E_2 \setminus E_1); S(E_2 \setminus E_1) \ge 0$
		\end{proof}

	\item
		Если $E_1$ переводится движением в $E_2$, то $S(E_1) = S(E_2)$.

	\item
		Площадь прямоугольника есть произведение длин сторон.
	\end{enumerate}
\end{Def}

\begin{theorem}[Теорема Банаха]
	Объект, обладающий такими свойствами существует, но не единственнен.
\end{theorem}

\begin{theorem}[Хаусдорфа]
	Для любого $n \ge 3$ в $\R^n$ аналогичным образом определенного объекта не существует.
\end{theorem}

Мы будем пользоваться упрощением понятия площадь:
\begin{Def}
	$\sigma\colon \mathcal{F} \ra [0; +\infty)$ со свойствами:
	\begin{enumerate}
	\item
		Площадь прямоугольника со сторонами, \textit{параллельными осям координат}, есть произведение его сторон.

	\item
		$E_1 \subset E_2 \Ra \sigma(E_1) \le \sigma(E_2)$

	\item
		Разрешаем разбивать множество на части только горизонтальными/вертикальными прямыми. Обозначаем: $E_{-}, E_{+}$
		\begin{enumerate}
			\item $E_{-} \cap E_{+} = \emptyset$
			\item $E_{-} \cup E_{+} = E$
		\end{enumerate}
	\end{enumerate}
\end{Def}

\begin{Def}
	$P$ "--- прямоугольник.
	\[ |P| = \sigma(\left<a, b\right> \times \left<c, d\right>) = (b - a)(d - c) \]
\end{Def}

\begin{Def}
	\[ \sigma(E) = \inf{\sum_{k = 1}^n |P_k|: E \subset \bigcup_{k = 1}^n P_k} \]
\end{Def}

\begin{theorem}[$\sigma$ есть площадь]
	\begin{enumerate}
		\item $\sigma$ "--- площадь
		\item $\sigma$ не меняется при параллельном переносе
	\end{enumerate}
\end{theorem}
\begin{proof}
	\begin{enumerate}
	\item
		Площадь прямоугольника:
		\[ \sum_{k = 1}^n |P_k| \ge P_0 \Ra \inf{\sum|P_k|} \ge (b - a)(d - c) \]
		Мы можем взять покрытие из одного прямоугольника:
		\[ \inf \le (b - a)(d - c) \]

	\item
		$E_1 \subset E_2 \Ra \sigma(E_1) \le \sigma(E_2)$:
		\[ E_2 \subset \bigcup P_k \Ra E_1 \subset \bigcup P_k  \]
		Таким образом, класс покрытий $E_1$ шире класса покрытий $E_2$, и его инфимум не больше, то есть $\sigma(E_1) \le \sigma(E_2)$.

	\item
		$E = E_{-} \cup E_{+}$. Докажем, что $\colon \sigma(E) = \sigma(E_{-}) + \sigma(E_{+})$.
		Рассмотрим $\bigcup P_k^{\pm}$:
		\[ \sum \left|P_k^{\pm}\right| \le \sigma(E_{\pm}) + \epsilon \Ra \sigma(E) \le \sigma(E_{-}) + \sigma(E_{+}) + 2\epsilon \]
		Обратим $\epsilon$ в ноль, получим
		\[ \sigma(E) \le \sigma(E_{-}) + \sigma(E_{+}) \]

		Рассмотрим $\bigcup P_k\colon E \subset \bigcup P_k$:
		Разделим прямой: прямоугольники, которые пересеклись с ней, образуют два новых прямоугольника. Мы получили покрытия $E_{\pm}$.
		Инфинум суммы не меньше суммы инфинумов, поэтому
		\[ \sigma(E) \ge \sigma(E_{-}) + \sigma(E_{+}) \]
	\end{enumerate}
\end{proof}

\begin{Def}
	$f: E \ra \R$.
	\[ f_\pm\colon E \ra [0; +\infty) \quad f_+ = \max\{f, 0\} \quad f_- = \max \{-f; 0\} \]
\end{Def}
Свойства:
\begin{enumerate}
	\item $f = f_+ - f_- $
	\item $|f| = f_+ + f_- $
	\item $2f_\pm = |f| \pm f$
	\item $f$ "--- непрерывна $ \Ra f_\pm $ непрерывна
\end{enumerate}

\begin{Def}
	Подграфик функции $f \ge 0$:
	\[ P_f = \left\{(x, y) \mid x \in E, y \in [0; f(x)]\right\} \]
\end{Def}

\begin{Rem}
	\[ f \in C[a, b] \Ra P_f\text{ "--- ограниченное множество} \]
\end{Rem}

\begin{Def}
	$f \in C[a, b]$.
	$\int_a^b f$ "--- определенный интеграл функции $f$ на $[a; b]$:
	\[ \int\limits_a^b f = \sigma (P_{f_+}) - \sigma (P_{f_-}) \]
\end{Def}

Свойства:
\begin{enumerate}
\item
	\[ f \ge 0 \Ra \int\limits_a^b f = \sigma(P_f) \ge 0 \]

\item
	\[ \int\limits_a^b f  = \int\limits_a^b f_+ - \int\limits_a^b f_- \]

\item
	$f \ge 0$:
	\[ \int\limits_a^b f = 0 \Ra f \equiv 0 \]
	\begin{proof}
		От противного: пусть есть точка, значение в которой не ноль, тогда и в её окрестности не ноль (по теореме о стабилизации знака).
		Выбираем в этой окрестности минимум по значениям и получаем прямоугольник с площадью, отличной от нуля.
		Площадь графика больше этой площади, но равна нулю "--- противоречие.
	\end{proof}

\item
	\[ \int\limits_a^b -f = - \int\limits_a^b f \]
	\begin{proof}
		\[ (-f)_+ = f_- ; (-f)_- = f_+ \]
	\end{proof}

\item
	\[ \int\limits_a^b C = C(b - a) \]

\end{enumerate}

\section{Свойства интеграла}

\begin{Rem}
	Свойство $\sigma (E) = \sigma (E_-) + \sigma (E_+) $ не испортится, если убрать требование про непересечение, то есть могут быть общие точки на разделяющей прямой.
\end{Rem}
\begin{proof}
	Прямая "--- вырожденный прямоугольник со стороной, равной нулю, поэтому его $\sigma$ равна нулю, значит он не влияет на сумму.
\end{proof}

\begin{theorem}[Аддитивность интеграла]
	$f \in C[a, b]$, $c \in [a, b]$.
	Тогда
	\[ \int\limits_a^b f = \int\limits_a^c f + \int\limits_c^b f \]
\end{theorem}
\begin{proof}
	\begin{gather*}
		\int\limits_a^b f = \sigma\left(P_{f_+}\right) - \sigma\left(P_{f_-}\right) = \\
		= \sigma\left(P_{f_+}([a, c])\right) - \sigma \left(P_{f_-}([a, c])\right)
			+ \sigma \left(P_{f_+}([c, b])\right) - \sigma \left(P_{f_-}([c, b])\right) =
			\int\limits_a^c f + \int\limits_c^b f
	\end{gather*}
\end{proof}

\begin{theorem}[Монотонность интеграла]
	$f, g \in C[a, b]$, $f \ge g$.
	Тогда
	\[ \int\limits_a^b f \ge \int\limits_a^b g \]
\end{theorem}
\begin{proof}
	\begin{gather*}
		\int\limits_a^b f = \sigma\left(P_{f_+}\right) - \sigma\left(P_{f_-}\right) \quad
		\int\limits_a^b g = \sigma\left(P_{g_+}\right) - \sigma\left(P_{g_-}\right) \\
		\begin{array}{rlcrlcrlcrl}
			 f &\ge  g &\Ra& f_+ &\ge g_+ &\Ra& P_{g_+} &\subset P_{f_+} &\Ra& \sigma\left(P_{f_+}\right) &\ge \sigma\left(P_{g_+}\right) \\
			-g &\ge -f &\Ra& g_- &\ge f_- &\Ra& P_{f_-} &\subset P_{g_-} &\Ra& \sigma\left(P_{f_-}\right) &\le \sigma\left(P_{g_-}\right)
		\end{array}
	\end{gather*}
\end{proof}

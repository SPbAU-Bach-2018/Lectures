\chapter{Интегральное исчисление (продолжение)}
\section{Определенный интеграл}

\begin{Def}
	$\mathcal{F}$ --- множество ограниченных подмножеств $\R^2$
\end{Def}

\begin{Def}
	Площадь:
	\[ S\colon \mathcal{F} \ra [0; +\inf) \]
	Должны выполняться следующие свойства:

	\begin{enumerate}
	\item 
		$E_1 \cap E_2 = \emptyset \Ra S(E_1 \cup E_2) = S(E_1) + S(E_2)$
		\begin{conseq*}
			$E_1 \subset E_2 \Ra S(E_1) \leqslant S(E_2)$
		\end{conseq*}
		\begin{proof}
			$E_2 = E_1 \cup (E_2 \setminus E_1); S(E_2) = S(E_1) + S(E_2 \setminus E_1); S(E_2 \setminus E_1) \geqslant 0$
		\end{proof}
	
	\item 
		Если $E_1$ переводится движением в $E_2$, то $S(E_1) = S(E_2)$.

	\item 
		Площадь прямоугольника есть произведение длин сторон.
	\end{enumerate} 
\end{Def}

\begin{theorem}{Теорема Банаха}
	Объект, обладающий такими свойствами существует, но не единственнен.
\end{theorem}

\begin{theorem}{Хаусдорфа}
	Для любого $n \ge 3$ в $\R^n$ аналогичным образом определенного объекта не существует.
\end{theorem}

\begin{Def}
	$\sigma\colon \mathcal{F} \ra [0; +\inf)$ со свойствами:
	\begin{enumerate}
	\item 
		$\sigma$(прямоугольника) $=$ произведение длин сторон

	\item 
		$E_1 \subset E_2 \Ra \sigma(E_1) \leqslant \sigma(E_2)$
	
	\item 
		Разрешаем разбивать множество на части только горизонтальными/вертикальными прямыми. Обозначаем: $E_{-}, E_{+}$
		\begin{enumerate}
			\item $E_{-} \cap E_{+} = \emptyset$
			\item $E_{-} \cup E_{+} = E$
		\end{enumerate}
	\end{enumerate} 
\end{Def}

\begin{Def}
	$P$ "--- прямоугольник. $|P| = \sigma(\left<a, b\right> \times \left<c, d\right>) = (b - a)(d - c)$
\end{Def}

\begin{Def}
	$\sigma(E) = \inf{\sum_{k = 1}^n |P_k|: E \subset \bigcup_{k = 1}^n P_k}$
\end{Def}

\begin{theorem}{$\sigma$ есть площадь}
	\begin{enumerate}
		\item $\sigma$ --- площадь
		\item $\sigma$ не меняется при параллельном переносе
	\end{enumerate}
\end{theorem}
\begin{proof}
	\begin{enumerate}
	\item
		$\sigma$(прямоугольника).
		$\sum_{k = 1}^n |P_k| \geqslant P_0 \Ra \inf{\sum|P_k|} \geqslant (b - a)(d - c); $
		берем покрытие из одного прямоугольника $\Ra \inf \leqslant (b - a)(d - c)$
	
	\item
		$E_1 \subset E_2 \Ra \sigma(E_1) \leqslant \sigma(E_2); $
		Пусть $E_2 \subset \bigcup P_k \Ra E_1 \subset \bigcup P_k \Ra$ класс покрытий $E_1$ шире класса покрытий $E_2 \Ra \inf $ меньше $\Ra \sigma(E_1) \leqslant \sigma(E_2)$

	\item
		$E = E_{-} \cup E_{+}; ! \sigma(E) = \sigma(E_{-}) + \sigma(E_{+}) \measuredangle \bigcup P_k^{\pm}:
		\sum|P_k^{\pm}| \leqslant \sigma(E_{\pm}) + \varepsilon \Ra \sigma(E) \leqslant \sigma(E_{-}) + \sigma(E_{+}) + 2\varepsilon;
		\varepsilon \ra 0; \measuredangle \bigcup P_k: E \subset \bigcup P_k; $
		делим прямой, те, прямоугольники, которые пересеклись с ней, образуют два новых прямоугольника $\Ra$ получили покрытия $E_{\pm}$
		Инфинум суммы не меньше суммы инфинумов $\Ra$ $\sigma(E) \geqslant \sigma(E_{-}) + \sigma(E_{+})$

	\end{enumerate}
\end{proof}

\begin{Def}
	$f: E \ra \R ; f_+ = \max {f, 0}; f_- = \max {-f; 0}; f_\pm : E \ra [0; +\inf); $  Свойства:
	\begin{enumerate}
		\item $f = f_+ - f_- $
		\item $|f| = f_+ + f_- $
		\item $2f_\pm = |f| \pm f$
		\item $f$ --- непрерывна $ \Ra f_\pm $ непрерывна
	\end{enumerate}
\end{Def}

\begin{Def}
	Подграфик функции. $ f \geqslant 0; P_f = {(x, y): x \in E, y \in [0; f(x)]} $
\end{Def}

\begin{Rem}
	$f \in C[a, b] \Ra P_f $ --- ограниченное множество
\end{Rem}

\begin{Def}
	$f \in C[a, b]; \int_a^b f$ --- определенный интеграл функции $ f$ на $[a; b]$; $\int_a^b f = \sigma (P_{f_+}) - \sigma (P_{f_-}) $
\end{Def}
Свойства:
\begin{enumerate}
\item 
	$f \geqslant 0 \Ra \int_a^b f = \sigma(P_f) \geqslant 0$
	
\item 
	$\int_a^b f  = \int_a^b f_+ - \int_a^b f_- $

\item 
	$f \geqslant 0 \int_a^b f = 0 \Ra f \equiv 0$
	\begin{proof}
		от противного: пусть есть точка, значение в которой не ноль, тогда и в окрестности не ноль (по теореме о стабилизации знака.
		Выбираем в этой окрестности минимум по значениям и получаем прямоугольник с площадью, отличной от нуля.
		Площадь графика больше этой площади, но равна нулю --- противоречие.
	\end{proof} 

\item 
	$\int_a^b -f = - \int_a^b f$
	\begin{proof}
		$(-f)_+ = f_- ; (-f)_- = f_+ $
	\end{proof} 

\item 
	$\int_a^b C = C(b - a)$

\end{enumerate}

\section{Свойства интеграла}

\begin{Rem}
	$ \sigma (E) = \sigma (E_-) + \sigma (E_+) $ --- не испортится, если убрать требование про непересечение, т.е. могут быть общие точки на прямой
\end{Rem}
\begin{proof}
	Прямая --- вырожденный прямоугольник со стороной, равной нулю $ \Ra \sigma $ (прямоугольника) равна нулю $ \Ra $ не влияет на сумму.
\end{proof}

\begin{theorem}{Аддитивность интеграла}
	$f \in C[a, b]; c \in [a, b] \Ra \int_a^b f = \int_a^c f + \int_c^b f$
\end{theorem}
\begin{proof}
	$ \int_a^b f = \sigma (P_{f_+}) - \sigma (P_{f_-}) = \sigma ((P_{f_+})_+) - \sigma ((P_{f_-})_+) + \sigma ((P_{f_+})_-) - \sigma ((P_{f_-})_-) = \int_a^c f + \int_c^b $
\end{proof}

\begin{theorem}{Монотонность интеграла}
	$ f, g \in C[a, b]; f \geqslant g \Ra \int_a^b f \geqslant \int_a^b g $
\end{theorem}
\begin{proof}
	$ \int_a^b f = \sigma (P_{f_+}) - \sigma (P_{f_-}); \int_a^b g = \sigma (P_{g_+}) - \sigma (P_{g_-});
	f \geqslant g \Ra f_+ \geqslant g_+ \Ra P_{g_+} \subset P_{f_+} \Ra \sigma(P_{f_+}) \geqslant \sigma(P_{g_+});
	-g \geqslant -f \Ra  g_-  \geqslant f_- \Ra P_{f_-} \subset P_{g_-} \Ra \sigma(P_{f_-}) \leqslant \sigma(P_{g_-});$
\end{proof}
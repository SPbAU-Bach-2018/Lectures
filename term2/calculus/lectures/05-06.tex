\section {Длина кривой}
\setauthor{Никита Подгузов}

\begin{Def}
	Путь "--- непрерывное отображение $\gamma \colon [a,b] \ra \R^d$.
\end{Def}

\begin{Def}
	Начало пути "--- $\gamma(a)$.
\end{Def}

\begin{Def}
	Конец пути "--- $\gamma(b)$.
\end{Def}

\begin{Def}
	Носитель пути "--- $\gamma([a,b])$.
\end{Def}

\begin{Def}
	Замкнутый путь "--- путь, в котором $\gamma(a) = \gamma(b)$.
\end{Def}

\begin{Def}
	Несамопересекающийся (простой) путь "--- путь, в котором
	\[
		\gamma(t_1) = \gamma(t_2) \Ra
		\left[
			\begin{gathered}
				t_1 = t_2 \\
				t_1 = a, t_2 = b \\
				t_1 = b, t_2 = a \\
			\end{gathered}
		\right.
	\]
\end{Def}

\begin{Def}
	$C^r$-гладкий путь "--- путь
	\[
		\gamma =
		\begin{pmatrix}
			\gamma_1 \\
			\gamma_2 \\
			\vdots \\
			\gamma_d
		\end{pmatrix}
	\]
	в котором
	\[ \forall i=1..d, \gamma_i \in C^r[a, b] \]
\end{Def}

\begin{Def}
	Гладкий путь "--- $C^1$-гладкий путь.
\end{Def}

\begin{Def}
	Кусочно-гладкий путь "--- путь $\gamma \colon [a, b] \ra \R^d$ такой, что
	\[ \exists c_0=a \le c_1 \le c_2 \le \dots \le c_{n-1} \le c_n = b \colon \gamma\bigr|_{[c_i, c_{i+1}]} \text{ "--- гладкий путь} \]
\end{Def}

\begin{figure}[h]
	\begin{center}
		\begin{minipage}[h]{0.15\linewidth}
			\includegraphics[width=1\linewidth]{curve.png}
			\caption{Простой путь}
		\end{minipage}
		\begin{minipage}[h]{0.3\linewidth}
			\includegraphics[width=1\linewidth]{closedCurve.png}
			\caption{Замкнутый путь}
		\end{minipage}
		\begin{minipage}[h]{0.3\linewidth}
			\includegraphics[width=1\linewidth]{semicircle.png}
			\caption{Кусочно-гладкий путь}
		\end{minipage}
	\end{center}
\end{figure}

\begin{Def}
	$\gamma \colon [a, b] \ra \R^d$, $\tilde{\gamma} \colon [c, d] \ra \R^d$.
	Если
	\[ \exists \tau \colon [a, b] \ra [c, d]\colon \tau(a) = c \land \tau(b)=d \land \tau \uparrow \land \gamma = \tilde \gamma \circ \tau \]
	то $\gamma$ и $\tilde \gamma$ "--- эквивалентные пути.
\end{Def}

\begin{exmp}
	\begin{gather*}
		\gamma(t) = (\cos t, \sin t), 0 \le t \le \pi \\
		\tilde \gamma(x) = (-x, \sqrt{1 - x^2}), -1 \le x \le 1 \\
		\tau(t) = -\cos t
	\end{gather*}
\end{exmp}

Свойства:
\begin{enumerate}
	\item Носители эквивалентных путей одинаковы
	\item Это отношение эквивалентности
	\item Одинаковый порядок точек
\end{enumerate}

\begin{Def}
	Кривая "--- класс эквивалентности путей.
\end{Def}
\begin{Def}
	Параметризация кривой "--- конкретный представитель класса.
\end{Def}

\begin{Def}
	Носитель кривой "--- носитель пути из класса эквивалентности.
\end{Def}

\begin{Def}
	Гладкая ($C^r$-гладкая) кривая "--- кривая, у которой существует гладкая ($C^r$-гладкая) параметризация.
\end{Def}
\begin{Def}
	Кусочно-гладкая кривая "--- кривая, у которой существует кусочно-гладкая параметризация.
\end{Def}

\begin{exmp}
	$\gamma(t) = (\cos t, \sin t)$, $0 \le t \le 2\pi$ "--- гладкая параметризация.
	$\tilde \gamma(x) = (\cos \sqrt{t}, \sin\sqrt{t})$, $0 \le t \le (2\pi)^2$ "--- негладкая параметризация.
\end{exmp}

\begin{Def}
	$\gamma \colon [a, b] \ra \R^d$ "--- параметризация. Противоположная кривая "--- кривая, ориентированная в другую сторону:
	\begin{gather*}
		\tilde \gamma \colon [a, b] \ra \R^d \\
		\tilde \gamma(t) = \gamma(b + a - t) \\
		\tilde \gamma(a) = \gamma(b) \quad \tilde \gamma(b) = \gamma(a)
	\end{gather*}
\end{Def}

\begin{Rem}
	У разных кривых может быть один и тот же носитель.
	\begin{gather*}
		\gamma(t) = (\cos t, \sin t), 0 \le t \le 2\pi \\
		\tilde \gamma(t) = (\cos t, \sin t), 0 \le t \le 3\pi
	\end{gather*}
\end{Rem}

\begin{Def}
	$\gamma \colon [a, b] \ra \R^d$.
	Разобьем отрезок $[a, b]$ на $n$ кусков.
	\[ t_0 = a \le t_1 \le t_2 \le \dots \le t_{n-1} \le t_n = b \]
	Проведем ломаную через точки
	\[ \gamma(t_0), \gamma(t_1), \dots, \gamma(t_{n-1}), \gamma(t_n) \]
	Длина кривой $l(\gamma)$ "--- супремум длин таких вписанных ломаных.
\end{Def}

Свойства:
\begin{enumerate}
	\item Длина не зависит от параметризации.
	\item Длины противоположных путей (кривых) равны.
	\item $l(\gamma) \ge \|\gamma(b) - \gamma(a)\|$ "--- длина отрезка, соединяющего концы.
	\item $l(\gamma) \ge$ длины любой вписанной в $\gamma$ ломаной.
\end{enumerate}

\begin{theorem}
	$\gamma \colon [a, b] \ra \R^d$, $c \in (a,b)$, $\gamma_1 = \gamma\bigr|_{[a,c]}$, $\gamma_2 = \gamma\bigr|_{[c,b]}$.
	Тогда $l(\gamma) = l(\gamma_1) + l(\gamma_2)$.
\end{theorem}

\begin{proof}
	\begin{description}
	\item[$l(\gamma) \ge l(\gamma_1) + l(\gamma_2)$:]
		$l(\gamma_1)$ "--- супремум вписанных в $\gamma_1$ ломаных.
		$l(\gamma_2)$ "--- супремум вписанных в $\gamma_2$ ломаных.
		Объединение этих ломаных $l_1 \cup l_2$ "--- ломаная, вписанная в $\gamma_1 \cup \gamma_2 = \gamma$.
		\[ |l_1 \cup l_2| \le l(\gamma) \]
		Переходим к $\sup$ по $l_1$, а затем к $\sup$ по $l_2$.

	\item[$l(\gamma) \le l(\gamma_1) + l(\gamma_2)$:]
		Берем ломаную $l$, вписанную в $\gamma$.
		Делаем из нее ломаную $\tilde l$ = $l_1 \cup l_2$, добавляя точку разбиения $c$.
		\begin{gather*}
			|\tilde l| \ge |l| \\
			l(\gamma_1) + l(\gamma_2) \ge |l_1| |l_2| = |\tilde l| \ge |l|
		\end{gather*}
		Переходим к $\sup$ по $l$.
	\end{description}
\end{proof}

\begin{Def}
	Кривая спрямляемая, если её длина конечна.
\end{Def}

\begin{theorem}
	$\gamma \colon [a,b] \ra \R^d$ "--- гладкая кривая, $\|x\| = \sqrt{x_1^2 + \dots + x_d^2}$,
	\[
		\gamma' =
		\begin{pmatrix}
			\gamma_1' \\
			\gamma_2' \\
			\vdots \\
			\gamma_d'
		\end{pmatrix}
	\]
	Тогда
	\[ l(\gamma) = \int\limits_a^b \|\gamma'(t)\|\d t = \int\limits_a^b \sqrt{\gamma_1'(t)^2 + \gamma_2'(t)^2 + \dots + \gamma_d'(t)^2} \d t \]
\end{theorem}
\begin{proof}
	Рассмотрим какой-то подотрезок $\Delta$, лежащий в отрезке $[a,b]$, и сужение $\gamma\bigr|_\Delta\colon \Delta \ra \R^d$.
	\begin{gather*}
		m_{\Delta}^{(i)} = \min_{t \in \Delta} \left|\gamma_i'(t)\right|, \text{ достигается в точке $\zeta_\Delta^{(i)}$}, m_{\Delta}^2 = \sum_{i=1}^d \left(m_{\Delta}^{(i)}\right)^2 \\
		M_{\Delta}^{(i)} = \max_{t \in \Delta} \left|\gamma_i'(t)\right|, \text{ достигается в точке $ \eta_\Delta^{(i)}$}, M_{\Delta}^2 = \sum_{i=1}^d \left(M_{\Delta}^{(i)}\right)^2
	\end{gather*}
	\begin{lemma}
		\[ m_{\Delta} |\Delta| \le l\left(\gamma\bigr|_{\Delta}\right) \le M_{\Delta} |\Delta| \]
	\end{lemma}
	\begin{proof}
		Обозначим отрезок $\Delta$ за $[\alpha, \beta]$.
		Рассмотрим разбиение $\Delta$:
		\[ t_0 = \alpha, t_1, t_2, \dots, t_{n-1}, t_n = \beta \]
		и ломаную, построенную на точках $\gamma(t_0), \gamma(t_1), \dots, \gamma(t_n)$.

		Пусть длина $i$-го звена равна $a_i$.
		Длина ломаной тогда будет $\sum_{i=1}^n a_i$.
		\begin{gather*}
			a_i^2 = (\gamma_1(t_i) - \gamma_1(t_{i - 1}))^2 + \dots + (\gamma_d(t_i) - \gamma_d(t_{i - 1}))^2 \\
			\gamma_1(t_i) - \gamma_1(t_{i - 1}) = \gamma_1'(\mu_{i1})(t_i - t_{i-1}) \le M_{\Delta}^{(1)} (t_i - t_{i-1}) \\
			\vdots \\
			\gamma_d(t_i) - \gamma_d(t_{i - 1}) = \gamma_d'(\mu_{id})(t_i - t_{i-1}) \le M_{\Delta}^{(d)} (t_i - t_{i-1}) \\
			a_i^2 \le (M_{\Delta}^{(1)}(t_i - t_{i-1}))^2 + \dots + (M_{\Delta}^{(d)}(t_i - t_{i-1}))^2 = M_{\Delta}^{2}(t_i - t_{i-1})^2
		\end{gather*}
		Аналогично, $a_i^2 \ge m_{\Delta}^{2}(t_i - t_{i-1})^2$.
		\[ m_{\Delta}(t_i-t_{i-1}) \le a_i \le M_{\Delta}(t_i-t_{i-1}) \]

		Просуммируем по всем $i$:
		\begin{gather*}
			m_{\Delta}\sum_{i=1}^n (t_i-t_{i-1}) \le \sum_{i=1}^n  a_i \le M_{\Delta} \sum_{i=1}^n (t_i-t_{i-1}) \\
			m_{\Delta}(\beta - \alpha) \le \sum_{i=1}^n a_i \le M_{\Delta}(\beta - \alpha)
		\end{gather*}
		Переходим к супремумам.
		\[ m_{\Delta} |\Delta| \le l(\gamma|_{\Delta}) \le M_{\Delta} |\Delta| \]
	\end{proof}
	Разобьем отрезок $[a, b]$ на $n$ кусков.
	\[ t_0 = a, t_1, t_2, \dots, t_{n-1}, t_n = b \]
	Обозначим $m_i = m_{[t_{i-1}, t_i]}$, $M_i = M_{[t_{i-1}, t_i]}$.
	\begin{gather*}
		l(\gamma) = l\left(\gamma\bigr|_{[t_0, t_1]}\right) + l\left(\gamma\bigr|_{[t_1, t_2]}\right) + \dots + l\left(\gamma\bigr|_{[t_{n - 1}, t_n]}\right) \\
		m_i(t_i - t_{i - 1}) \le l(\gamma|_{[t_{i - 1}, t_i]}) \le M_i(t_i - t_{i - 1})
	\end{gather*}
	Просуммируем:
	\[ \sum\limits_{i=1}^n m_i(t_i - t_{i-1}) \le l(\gamma) \le \sum\limits_{i=1}^n M_i(t_i - t_{i - 1}) \]
	Посмотрим на сумму Римана для интеграла $\int\limits_a^b \|\gamma'(t)\|\d t$ в точках $\xi_i \in [t_{i - 1}, t_i]$:
	\begin{gather*}
		\sum\limits_{i = 1}^n \|\gamma'(\xi_i)\| (t_i - t_{i - 1}) \\
		\|\gamma'(\xi_i)\|^2 = |\gamma_1'(\xi_i)|^2 + |\gamma_2'(\xi_i)|^2 + \dots + |\gamma_d'(\xi_i)|^2 \\
		|\gamma_j'(\xi_i)|^2 \le (M_{[t_{i - 1}, t_i]}^{(j)})^2
	\end{gather*}
	Просуммируем по $j = 1..d$:
	\begin{gather*}
		\|\gamma'(\xi_i)\|^2 \le M_{[t_{i - 1}, t_i]}^2 = M_i^2 \\
		m_i \le \|\gamma'(\xi_i)\| \le M_i
	\end{gather*}
	Просуммируем по $i$:
	\[
		\sum\limits_{i = 1}^n m_i(t_i - t_{i - 1}) \le
			\sum\limits_{i = 1}^n \|\gamma'(\xi_i)\| (t_i - t_{i - 1}) \le
			\sum\limits_{i = 1}^n M_i(t_i - t_{i - 1})
	\]
	Хотим доказать, что $\sum\limits_{i = 1}^n (M_i - m_i)(t_i - t_{i - 1}) \ra 0$.
	\begin{gather*}
		M_i - m_i = \frac{M_i^2 - m_i^2}{M_i + m_i} =
			\frac{\sum\limits_{j = 1}^d ((M_{[t_{i - 1}, t_{i}]}^{(j)})^2 - (m_{[t_{i - 1}, t_{i}]}^{(j)})^2)}{M_i + m_i} =
			\sum\limits_{j = 1}^d (M_{[t_{i - 1}, t_{i}]}^{(j)} - m_{[t_{i - 1}, t_{i}]}^{(j)})
			\underbrace{\frac{M_{[t_{i - 1}, t_{i}]}^{(j)} + m_{[t_{i - 1}, t_{i}]}^{(j)}}{M_i + m_i}}_{\le 1} \\
		M_i - m_i \le \sum\limits_{j = 1}^d (M_{[t_{i - 1}, t_{i}]}^{(j)} - m_{[t_{i - 1}, t_{i}]}^{(j)})
	\end{gather*}
	Осталось доказать последнее неравенство: $\sum\limits_{j = 1}^d (M_{[t_{i - 1}, t_{i}]}^{(j)} - m_{[t_{i - 1}, t_{i}]}^{(j)}) \le \epsilon d$.
	\[ \left |M_{[t_{i - 1}, t_{i}]}^{(j)} - m_{[t_{i - 1}, t_{i}]}^{(j)} \right| = |\gamma_j'(\eta_i) - \gamma_j'(\zeta_i)| \]
	Знаем, что $\gamma_j'$ непрерывно на $[a, b]$, а значит равномерно непрерывно (по теореме Кантора):
	\[ \forall \epsilon > 0, \exists \delta > 0: \forall \eta, \zeta \colon |\eta - \zeta| < \delta \Ra |\gamma_j'(\eta) - \gamma_j'(\zeta)| < \epsilon \]
	Если наше разбиение ${t_i}$ имеет мелкость менее $\delta$, то верно, что $|\gamma_j'(\eta_i) - \gamma_j'(\zeta_i)| < \epsilon$.
	\begin{gather*}
		|M_i - m_i| = |\gamma_j'(\eta_i) - \gamma_j'(\zeta_i)| < \epsilon d \\
		\sum\limits_{i = 1}^n (M_i - m_i)(t_i - t_{i - 1}) < \epsilon d \sum\limits_{i = 1}^n (t_i - t_{i - 1}) = \epsilon d (b - a) \ra 0
	\end{gather*}
\end{proof}

Следствия:
\begin{enumerate}
	\item
		$f: [a,b] \ra \R$, $f \in C^1[a,b]$. Длина графика равна
		\[ \int\limits_a^b \sqrt{1 + f'(x)^2} \d x \]
		\begin{proof}
			Кривая с параметризацией $(x, f(x)), a \le x \le b$.
		\end{proof}

	\item
		Длина кривой в полярных координатах есть
		\[ \int\limits_\alpha^\beta \sqrt{r(\phi)^2 + r'(\phi)^2} \d\phi \]
		\begin{proof}
			\begin{gather*}
				\gamma(\phi) = (r(\phi) \cos \phi, r(\phi) \sin \phi) \\
				(r(\phi) \cos \phi)' = r' \cos \phi - r \sin \phi \\
				(r(\phi) \sin \phi)' = r' \sin \phi + r \cos \phi \\
				((r(\phi) \cos \phi)')^2 + ((r(\phi) \sin \phi)')^2 = r(\phi)^2 + r'(\phi)^2
			\end{gather*}
		\end{proof}

	\item
		$l(\gamma) \le (b-a) \max\limits_{t \in [a,b]} \|\gamma'(t)\|$
\end{enumerate}

\begin{exmp}
	Посчитаем длину эллипса:
	\begin{gather*}
		\frac{x^2}{a^2} + \frac{y^2}{b^2} = 1, a < b \\
		x(t) = a\cos t, x'(t) = -a\sin t \\
		y(t) = b\sin t, y'(t) = b\cos t \\
		0 \le t < 2\pi \\
		x'(t)^2 + y'(t)^2 = a^2\sin^2(t) + b^2\cos^2(t) = b^2 - (b^2 - a^2)\sin^2(t) \\
		l = \int\limits_0^{2\pi} \sqrt{b^2 - (b^2 - a^2)\sin^2t} \d t = b\int\limits_0^{2\pi}\sqrt{1 - \epsilon^2\sin^2t}\d t
	\end{gather*}
	где $\epsilon = \frac{\sqrt{b^2-a^2}}{b}$ и называется эксцентриситетом эллипса.
\end{exmp}

\begin{Def}
	$\gamma \colon [0, S] \ra \R^d$ "--- спрямляемая кривая.
	Параметризация называется натуральной, если
	\[ \forall s \in [0, S], l\left(\gamma\bigr|_{[0, s]}\right) = s \]
\end{Def}

\begin{theorem}
	У всякой гладкой спрямляемой кривой существует натуральная параметризация.
\end{theorem}
\begin{proof}
	$\tilde \gamma[a, b] \ra \R^d$ "--- произвольная параметризация.
	Введём отображение $\tau \colon [a, b] \ra [0, l(\tilde \gamma)]$, $\tau(t) = l(\tilde \gamma|_{[a, t]})$.
	Оно строго монотонно.
	$\tau$ непрерывна, так как
	\[ |l(\tilde \gamma(s_2)) - l(\tilde \gamma(s_1))| \le \left|\max\limits_{t \in [s_1,s_2]}\|\tilde \gamma'(t)\|\right|(s_2 - s_1) \ra 0 \]
	$\tau^{-1} \colon [0, l(\tilde \gamma)] \ra [a, b]$ непрерывна и строго монотонна.
	$\gamma = \tilde \gamma \circ \tau^{-1}$ "--- натуральная параметризация.
\end{proof}

\begin{Rem}
	Для кусочно-гладких кривых теоремы и следствия тоже верны.
\end{Rem}

\section {Формулы для вычисления площади}

\begin{Def}
	Ограниченное множество $E \subset \R^2$ квадрируемо, если площадь $E$ определена однозначно (то есть не зависит от $\sigma$ из определения площади).
\end{Def}

\begin{Rem}
	Подграфики непрерывных функций квадрируемы.
	\[ S = \int\limits_a^b f(x) \d x \]
\end{Rem}

\cimg{area_trapecia.png}{0.5}

Введем два дополнительных обозначения: точки $A_0 = (a, 0)$ и $B_0 = (b, 0)$.
\begin{assertion}
	$f, g \in C[a,b]$, $f \ge g$
	Рассмотрим криволинейную трапецию "--- множество
	\[ {(x,y) \mid a \le x \le b, g(x) \le y \le f(x)} \]
	Криволинейная трапеция квадрируема и
	\[ S = \int\limits_a^b (f(x)-g(x))\d x \]
\end{assertion}

\begin{proof}
	\[ S = S_f - S_g = \int\limits_a^b f(x) \d x - \int\limits_a^b g(x) \d x = \int\limits_a^b (f(x) - g(x)) \d x \]
	где $S_f$ "--- площадь фигуры $A_0A_2B_2B_0$, а $S_g$ "--- площадь фигуры $A_0A_1B_1B_0$.
\end{proof}

\begin{theorem}
	$E$ "--- множество, ограниченное замкнутой простой кусочно-гладкой кривой\\
	$\gamma \colon [a,b] \ra \R^2$, $\gamma(t) = (x(t), y(t))$.
	\[ S_E = \int\limits_a^b x(t)y'(t)\d t = -\int\limits_a^b y(t)x'(t) \d t = \frac{1}{2} \int\limits_a^b (x(t)y'(t) - y(t)x'(t))\d t \]
\end{theorem}
\begin{proof}
	\begin{enumerate}
	\item
		Все формулы дают один и тот же результат.
		\[ \int\limits_a^b x(t)y'(t) \d t + \int\limits_a^b y(t)x'(t) \d t = x(t)y(t) \bigg|_a^b = x(b)y(b) - x(a)y(a) = 0 \]
		так как $(x(a), y(a)) = (x(b), y(b))$.

	\item
		Формула верна для криволинейной трапеции. Обратимся к картинке (переобозначив $\alpha = a, \beta = b$).
		\begin{description}
			\item[$A_1B_1$:] $x(t) = t, y(t) = g(t)$
			\item[$B_1B_2$:] $x(t) = \beta$
			\item[$B_2A_2$:] $x(t) = -t + c, y(t) = f(-t + c)$
			\item[$A_2A_1$:] $x(t) = \alpha$
		\end{description}
		\[ -\int\limits_a^b y(t)x'(t) \d t = -\int\limits_a^b g(t) \d t + \int\limits_a^b f(t) \d t \]

	\item
		Теперь мы умеем делать вертикальные и горизонтальные разрезы, покажем это.
		Пусть было какое-то множество, мы его разделили вертикальным разрезом.
		Но тогда $\int y x' \d t$ даст 0 на этом вертикальном отрезке (две части сократятся).
		С горизонтальным аналогично, только рассматривать надо интеграл $\int x y' \d t$.

		На самом деле, это нам дает еще и способность объединять множества по общему вертикальному/горизонтальному отрезку.

	\item
		Имеется кусочно-гладкая кривая.
		Проведем через все точки, где меняется монотонность $x$, все точки, где меняется монотонность $y$, и все негладкие точки горизонтальные и вертикальные разрезы.
		Получили много криволинейных трапеций, на каждой из которых формула работает.
		После чего склеиваем все обратно.
	\end{enumerate}
\end{proof}

\begin{conseq}
	Площадь в полярных координатах.
	\[ S = \frac{1}{2}\int\limits_\alpha^\beta r^2(\phi) d\phi \]
	\cimg{area_for_polar.png}{0.5}
\end{conseq}
\begin{proof}
	$\gamma(\phi) = (r(\phi)\cos(\phi), r(\phi)\sin(\phi))$
	\[ \frac{1}{2} \int\limits_\alpha^\beta (y'(\phi)x(\phi) - x'(\phi)y(\phi)) d\phi = \frac{1}{2} \int\limits_\alpha^\beta r^2(\phi) d\phi \]
\end{proof}

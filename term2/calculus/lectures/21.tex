\setauthor{Дмитрий Лапшин}

Упражнение: $E \subset \cl \Int E$ и $f\colon E \to \R$ "--- интегрируема, то $f$ ограничена.

\begin{Def}
	$E$ "--- множество, $f\colon E \to \R$.
	$\omega(f, E)$ "--- колебание $f$ на $E$.
	\[ \omega(f, E) = \sup_{x \in E} f(x) - \inf_{x \in E} f(x) = \sup_{x, y \in E} (f(x) - f(y)) \]
\end{Def}

\begin{theorem}[критерий интегрируемости]
	$E$ "--- измерима, $f\colon E \to \R$.
	$f$ интегрируема на $E$ тогда и только тогда, когда
	\[ \forall \epsilon > 0, \exists \delta > 0\colon \forall \tau\colon |\tau| < \delta, \sum_{k=1}^{m(\tau)} \omega(f, E_k) \mu E_k < \epsilon\]
\end{theorem}
\begin{proof}
	\begin{description}
	\item[$\Ra$:]
		$f$ интегрируема, значит
		\[
			\forall \epsilon > 0, \exists \delta > 0\colon \forall \tau\colon |\tau| < \delta,
				\left| S(f, \tau, \xi) - \int\limits_E \right| < \epsilon
		\]
		Зафиксируем $\tau$.
		Посмотрим на $\omega(f, E_k) \le 2(f(\xi_k) - f(\xi_k'))$.
		\begin{description}
		\item[$f$ постоянна на $E_k$:] очевидно.
		\item[$f$ непостоянна на $E_k$:] $\gamma \lrh \omega(f, E_k) = \sup_{E_k} f - \inf_{E_k} f$.
			Приблизимся к супремуму с точностью $\gamma/4$ в точке $\xi_k$, к инфимуму в $\xi_k'$.
			\begin{gather*}
				\begin{aligned}
					f(\xi_k)  &> \sup_{E_k} f - \gamma / 2 \\
					f(\xi_k') &< \inf_{E_k} f + \gamma / 2
				\end{aligned} \\
				f(\xi_k) - f(\xi_k') > \sup f - \inf f - \gamma/2 = \gamma/2
			\end{gather*}
		\end{description}
		Тогда
		\begin{gather*}
			\begin{aligned}
				\left| S(f, \tau, \xi ) - \int\limits_E f \right| &< \epsilon \\
				\left| S(f, \tau, \xi') - \int\limits_E f \right| &< \epsilon
			\end{aligned} \\
			\left| S(f, \tau, \xi) - S(f, \tau, \xi') \right| < 2\epsilon \\
			\sum_{k=1}^{m(\tau)} \left( f(\xi_k) - f(\xi_k') \right) \mu E_k < 2\epsilon \\
			\frac12 \sum_{k=1}^{m(\tau)} \omega(f, E_k) \mu E_k \le \sum_{k=1}^{m(\tau)} \left( f(\xi_k) - f(\xi_k') \right) \\
			\sum_{k=1}^{m(\tau)} \omega(f, E_k) \mu E_k \le 2\sum_{k=1}^{m(\tau)} \left( f(\xi_k) - f(\xi_k') \right) < 4\epsilon
		\end{gather*}

	\item[$\La$:]
		Пусть
		\[
			\forall \epsilon > 0, \exists \delta > 0\colon \forall \tau\colon |\tau| < \delta,
			\sum_{k=1}^{m(\tau)} \omega(f, E_k) \mu E_k \le \sum_{k=1}^{m(\tau)} < \epsilon
		\]
		Рассмотрим $\tau' \succ \tau$. Тогда
		\[ E_k = \bigsqcup_{i=1}^{i_k} E_{k_i}' \]
		Далее оценим
		\[
			\left| f(\xi_k) \mu E_k - \sum_{i=1}^{i_k} f\left(\xi_{k_i}'\right) \mu E_{k_i}' \right| = \cdots
		\]
		$\xi_k \in E_k$, $\xi_{k_i}' \in E_{k_i}' \subset E_k$.
		\begin{gather*}
			|f(\xi_k) - f\left(\xi_{k_i}\right)| \le \omega(f, E_k) \\
			\dots = \left| \sum_{i=1}^{i_k} f(\xi_k) \mu E_{k_i}' - \sum_{i=1}^{i_k} f(\xi_k') \mu E_{k_i}' \right| \le \\
			\le \sum_{i=1}^{i_k} \left| f(\xi_k) - f\left(\xi_{k_i}'\right) \right| \mu E_{k_i}' \le \sum_{i=1}^{i_k}  \omega(f, E_k) \mu E_{k_i}' = \omega(f, E_k) \mu E_k
		\end{gather*}
		Суммируем по $k$
		\[
			|S(f, \tau, \xi) - S(f, \tau, \xi')| \le \sum_{k=1}^{m(\tau)} \omega(f, E_k) \mu E_k
		\]
		Если при этом $|\tau| < \delta$, то $|S(f, \tau, \xi) - S(f, \tau, \xi')| < \epsilon$.

		Пусть $\tau'$ и $\tau''$ "--- разбиения мелкости меньше $\delta$.
		Рассмотрим $\tau \succ \tau', \tau''$. Тогда
		\begin{gather*}
			\begin{aligned}
				|S(f, \tau, \xi) - S(f, \tau' , \xi' )| &< \epsilon \\
				|S(f, \tau, \xi) - S(f, \tau'', \xi'')| &< \epsilon
			\end{aligned} \\
			|S(f, \tau', \xi') - S(f, \tau'', \xi'')| < 2\epsilon
		\end{gather*}
		Рассмотирм последовательность разбиений $\tau_n$, что $|\tau_n| \to 0$.
		\[ \forall \epsilon > 0, \exists N\colon |\tau_n| < \delta \colon \forall n,m > N, |S(f, \tau_n, \xi_n) - S(f, \tau_m, \xi_m)| < 2\epsilon \]
		Это уже критерий Коши для последовательностей.
		Значит, $S_n$ имеет предел $I$.
		Тогда взяв ряд неравенств (начиная с некоторого $n$)
		\[ | S(f, \tau', \xi') - S(f, \tau_n, \xi_n) | < 2\epsilon \]
		в пределе получим
		\[ | S(f, \tau', \xi') - I| < 2\epsilon \]
	\end{description}
\end{proof}

\begin{Def}
	Верхняя и нижняя сумма Дарбу $S_\tau(f)$ и $s_\tau(f)$:
	\begin{align*}
		S_\tau(f) &= \sum_{k=1}^{m(\tau)} \sup_{x \in E_k} f(x) \cdot \mu E_k \\
		s_\tau(f) &= \sum_{k=1}^{m(\tau)} \inf_{x \in E_k} f(x) \cdot \mu E_k
	\end{align*}
\end{Def}
\begin{Rem}
	\[ S_\tau(f) \ge s_\tau(f) \]
\end{Rem}
\begin{Rem}
	\[ S_\tau(f) - s_\tau(f) = \sum_{k=1}^{m(\tau)} \omega(f, E_k) \mu E_k \]
\end{Rem}

\begin{theorem}
	$f\colon E \to \R$ "--- ограниченна.
	Тогда $f$ интегрируема на $E$ тогда и только тогда, когда
	\[ \forall \epsilon > 0, \exists \delta > 0\colon \forall \tau\colon |\tau| < \delta, S_\tau - s_\tau < \epsilon \]
\end{theorem}

\begin{conseq}
	$f\colon E \to \R$ "--- ограниченна и интегрируема.
	Тогда
	\[ s_\tau \le \int\limits_E f \le S_\tau \]
\end{conseq}
\begin{proof}
	\begin{gather*}
		s_\tau \le S(f, \tau, \xi) \le S_\tau \\
		\sum_{k=1}^{m(\tau)} \inf_{x \in E_k} f(x) \mu E_k
			\le \sum_{k=1}^{m(\tau)} f(\xi_k) \mu E_k
			\le \sum_{k=1}^{m(\tau)} \sup_{x \in E_k} f(x) \mu E_k
	\end{gather*}
	$\tau' \succ \tau$
	\begin{gather*}
		s_\tau \le S(f, \tau', \xi') \le S_\tau \\
		\inf_{x \in E_k} f(x) \mu E_k
			\le \sum_{i=1}^{i_k} f(\xi_{k_i}') \mu E_{k_i}'
			\le \sup_{x \in E_k} f(x) \mu E_k
	\end{gather*}
	$\tau_n \succ \tau$, $|\tau_n| \to 0$, $S(f, \tau_n, \xi_n) \to \int_E f$
	\begin{gather*}
		s_\tau \le S(f, \tau', \xi') \le S_\tau \\
		s_\tau \le \int\limits_E \le S_\tau
	\end{gather*}
\end{proof}

\begin{theorem}
	$f\colon K \ra \R$ непрерывна, $K$ "--- измеримый компакт.
	Тогда $f$ интегрируема на $K$.
\end{theorem}
\begin{proof}
	$f$ непрерына на компакте, значит она ограничена и равномерно непрерывна.
	\[ \forall \epsilon > 0, \exists \delta > 0\colon \forall x,y\in K, \|x - y\| < \delta \Ra |f(x) - f(y)| < \epsilon \]
	Если $\diam E < \delta$, то $\omega(f, E) < \epsilon$.
	$\tau$ "--- произвольное разбиение мелкости меньше $\delta$. Там все $\diam E_k < \delta$.
	\[ \sum_{k=1}^{m(\tau)} \omega(f, E_k) \mu E_k \le \sum_{k=1}^{m(\tau)} \epsilon \mu E_k = \epsilon \mu K \]
\end{proof}

Упражнение. $E$ "--- открытое измеримое множество.
$f\colon E \to \R$ непрерывно и ограничено.
Тогда $f$ интегрируема.

\section{Свойства кратного интеграла}

\begin{enumerate}
\item
	$E$ измеримо.
	Тогда
	\[ \int\limits_E 1 = \mu E \]
	\begin{Rem}
		$1$ "--- <<адын>>. \copyright Храбров.
	\end{Rem}

\item
	$E \supset E'$ измеримо, $f$ интегрируема на $E$.
	Тогда $f$ интегрируема на $E'$.
	\begin{proof}
		$\tau'$ "--- разбиение $E'$. Достроим, сохраняя мелкость, до разбиения $E$.
		\begin{gather*}
			\forall \epsilon > 0, \exists \delta > 0\colon \forall \tau\colon |\tau| < \delta,
				\sum_{k=1}^{m(\tau)} \omega(f, E_k) \mu E_k < \epsilon \\
			\sum_{k=1}^{m(\tau)} \omega(f, E_k') \mu E_k' < \sum_{k=1}^{m(\tau)} \omega(f, E_k) \mu E_k < \epsilon
		\end{gather*}
	\end{proof}

\item
	Аддитивность интеграла.
	$E = E' \sqcup E''$, $E'$ и $E''$ измеримы, $f$ ограничена.
	Тогда $f$ интегрируема на $E$ тогда и только тогда, когда $f$ интегрируема на $E'$ и $E''$, и в этом случае
	\[ \int\limits_E f = \int\limits_{E'} f + \int\limits_{E''} f \]
	\begin{proof}
		Достаточно доказать влево. $\tau$ "--- разбиение $E$.
		\begin{gather*}
			\tau' = \{E_k \cap E' \mid E_k \in \tau\} \\
			\tau'' = \{E_k \cap E'' \mid E_k \in \tau\} \\
			\tau_0 = \{E_k \mid E_k \cap E' \ne \emptyset \ne E_k \cap E''\} \\
			\sum_{E_k \notin \tau_0} \omega(f, E_k) \mu E_k
				\le \sum_{F \in \tau'} \omega(f, F) \mu F + \sum_{F \in \tau''} \omega(f, F) \mu F \\
			\sum_{E_k \in \tau_0} \omega(f, E_k) \mu E_k
				\le \sum_{E_k \in \tau_0} 2M \mu E_k = 2M \mu \left( \bigcup_{E_k \in \tau_0} E_k \right) \le \cdots \\
		\end{gather*}

		$E_k \cap E' \ne \emptyset \ne E_k \cap E''$.
		Мы можем брать точки $a_k$ и $b_k$ из $E_k$, что на $[a_k, b_k]$ есть точка из $\delta E'$.
		$\diam E_k < \delta$, значит длина $[a_k, b_k]$ меньше $\delta$.
		Тогда точка из границы $\delta E'$ лежит на рассоянии не более $\delta$ от концов.

		Так как мы можем в качестве $a$ перебрать любую точку, то $E_k \subset U_\delta (\partial E')$.
		\begin{gather*}
			\dots \le 2M \mu(U_\delta (\partial E')) \to 2M \mu(\delta E') = 0
		\end{gather*}
%%	\end{proof}
%%\end{enumerate}

\chapter{Исключения. Больше исключений}
\setauthor{Дмитрий Лапшин}

Мы сейчас ещё поговорим о исключениях.
Такая же лекция будет про шаблоны.
К STL мы вернёмся позже.

\section{Вспоминим пройденный материал}

Вспомним, за что люди любят исключениях. Пусть есть класс
\begin{cppcode}
class NetworkConnection {
	int* buffer;
	
	NetworkConnection() {
		buffer = new int[1024];
	}

	~NetworkConnection() {
		delete[] buffer;
	}

	void Connect(/*...*/);

	//...
};

void startConnection() {
	NetworkConnection nc();
	nc.Connect(/*...*/);
	//...
}

int main() {
	try {
		startConnection();
	}
	catch (NetworkException &e) {
		//...
	}
}
\end{cppcode}
Как видим, есть класс \cpp'NetworkConnection', обслуживающий общение с сетью и содержащий буффер.
Есть метод \cpp'startConnection()', создающий соединение.

У метода \cpp'NetworkConnection::Connect()' могут возникнуть ошибки: адрес компьютера указан неверно или нужный компьютер недоступен.
Он при ошибке кидает (\cpp'throw') \cpp'NetworkException'.
Мы бросаем объект, так как тогда можно передать подробную информацию об ошибке,
чтобы пользователь иди программа могла узнать, что произошло, и исправить ситуацию.

Почему это любят?
\begin{enumerate}
	\item Ошибка передаётся с подробными сведениями
	\item Ошибка может быть обработана не там, где произошла.
\end{enumerate}

При создании исключения начинается разкручивание стека (stack unwinding) до тех пор, пока не встретится обработчик (\cpp'try'):
\begin{center}\begin{tabular}{|c|}
	\hline \cpp'NetworkConnection::Connect()' \\
	\hline \cpp'startConnection()' \\
	\hline \cpp'main()' \\
	\hline
\end{tabular}\end{center}
Так \cpp'NetworkConnection' может передавать сообщения об ошибке, не зная, где, кто и как его использует и как обрабатывает ошибки.

Важно отметить, что при разкручивании стека вызываются деструкторы всех объектов на фреймах стека.
Это поможет нам, например, избежать утечки памяти в \cpp'NetworkConnection::buffer'.

Отметим, что блоки у \cpp'try catch finally' являются новыми областями видимости, и в них можно создавать локальные переменные.
Также, если исключение объект, его лучше в \cpp'catch' ловить по ссылке, чтобы не копировать объект исключения лишний раз.

\section{Разделение исключений}

Если у нас большой проект (например, игра), то его часто полезно разделять на отдельные подсистемы, например:
\begin{enumerate}
	\item Сеть (Network)
	\item Модель (Model)
	\item Отображение (View)
	\item Искусственный интеллект (AI)
\end{enumerate}

В каждой подсистеме логично создать свой тип исключения.
Так можно разделить ошибки и их обработку.
Сетевую ошибку можно попытаться разрешить, ошибку в модели чаще всего критическая.

Где-то высоко в \cpp'main()' есть главный блок:
\begin{cppcode}
try {
	connect();
	//...
	makeMove();
	//...
}
catch (NetworkException &ne) {
	//вывод сообщения об ошибке, переподключение, ...
}
catch (ModelException &me) {
	//Послать BugReport
	//Завершить работу
}
\end{cppcode}

Но иногда мы хотим получить любую ошибку.
Тогда можно сделать так: есть главный родитель \cpp'GameException', а остальных сделать его наследниками.
%\begin{cppcode*}{linenos,firstline=17}
\begin{cppcode}
catch (GameException &ge) {
	Log.Write(/* message */);
}
\end{cppcode}
%\end{cppcode*}

Важно заметить, что из всех блоков \cpp'catch' выбирается первый подходящий!
Если написать \cpp'catch (GameException &ge)', первым, он перехватит все исключения!

\section{Стандартная библиотека}

В C++ есть базовый класс \cpp'std::exception', являющийся предком всех исключений, бросаемых стандартной библиотекой.
У него всего один метод \cpp'virtual const char* std::exception::what();', который должен вернуть описание.
Собственно, сам \cpp'std::exception' не кидают никогда, а кидают его различных наследников 
(\cpp'std::ios::failure' у ввода-вывода, \cpp'std::runtime_exception' для критических ошибок, \cpp'std::logic_error' для логических и т. д.).

\begin{cppcode}
try {
	vector.push_back(/*...*/);
	file.open();
}
catch (failure &fe) {
	//...
}
catch (exception &e) {
	//...
}
\end{cppcode}

В некоторых случаях решают, что пользовательская система исключений тоже наследуется от \cpp'std::exception'.

\section{RAII}

RAII (Resource Acquisition Is Initialization) "--- идеология, при которой для использования каждого ресурса создаётся объект.
\begin{cppcode}
while (/*...*/)
	try {
		Animal *pA = read();
		pA->process();
		delete pa;
	}
	catch (ios::failure &fe) {
		//проблема с вводом
	}
\end{cppcode}

Пусть мы обрабатываем два класса животных "--- кошечки и собачки.
Виртуальная функция \cpp'Animal::process' решает, нужна ли вакцинация, и проводит её, если нужно.
У каждого вида животного понимает она по-своему.

Метод \cpp'read()' при чтении опознаёт, информацию о каком животном она сейчас читает, и возвращает указатель на соотвествующий новый объект.

Есть проблема "--- если ошибка при чтении произошла, а объект был создан, то мы его не удалим!

\begin{cppcode}
while (/*...*/)
	try {
		shared_ptr<Animal> pA;
		read(pA);
		pA->process();
	}
	catch (ios::failure &fe) {
		//проблема с вводом
	}
\end{cppcode}

В этом и идея RAII "--- если ресурс внезапно не нужен, будет вызван его деструктор.

\section{Исключения в деструкторах}

\begin{cppcode}
class NetworkConnection {
	int* buffer;
	NetworkConnection() {
		buffer = new /*...*/;
	}

	void Connect() {
		//...
		throw NetworkException(/*...*/);
		//...
	}

	~NetworkConnection() {
		delete[] Buffer;
		ofstream file;
		file.exceptions(ios::badbit | ios::failbit);
		file.open("log.txt");
		//запись в файл
	}
}

//...

try {
	NetworkConnection nc();
	nc.Connect();
}
\end{cppcode}

В чём скрывается проблема?
Пусть произошло исключение.
Нужно удалить объект \cpp'NetworkConnection', поскольку он на стеке.
Но его деструктор может создать исключение (из \cpp'ostream').
Тогда, если бы возникло два исклчения, не ясно, что делать.
В C++ просто запрещено наличие более одного исключения, и поэтому из деструктора запрещено создавать исключения.

\section{Исключения в конструкторе}

\begin{cppcode}
class BookEntry {
	string name;
	string phone;
	Image *image;
	Audio *track;
	BookEntry();
	~BookEntry();
};

BookEntry(/*...*/) {
	image = new Image(fname);
	image->Load();
	track = new Track(tname);
	track->Load();
}

~BookEntry() {
	delete image;
	delete track;
}
\end{cppcode}

Пусть в конструкторе произошло исключение (в \cpp'Image::Load', например).
Тогда деструктор не вызывается, так как объект не считается достроенным.
Вот так не прокатит:
\begin{cppcode}
try {
	BookEntry be("Vasya", "vasya.jpg", "vasya.mp3");
}
catch (/*...*/) {
	//...
}
\end{cppcode}

Необходимо в конструкторе освободить ресурсы обратно.
\begin{cppcode}
BookEntry(/*...*/) {
	try {
		image = new Image(fname);
		image->Load();
		track = new Track(tname);
		track->Load();
	}
	catch (/*...*/) {
		delete image;
		delete track;
	}
}
\end{cppcode}

\section{Объявление возможных исключений}

Были предприняты попытки указывать, какие исключения бросает метод
\begin{cppcode}
void C() throw (A, B) {
//...
}
\end{cppcode}

Оно просто проверяет, нужного ли типа исключения, и если нет, аварийно завершает работу. Имеенно из-за такого поведения эта практика не очень рекомендуется.

\section{Гарантии стандарта}

У стандарта есть три типа заявления о исключениях:
\begin{description}
\item[No throws:]
	Исключения никогда не бросаются.

\item[Strong:]
	Если исключение и будет создано, объект останется в корректном состояни, в каком был после последней транзакции (последнее цельное состояние объекта).
\begin{cppcode}
template <typename T>
T stack::pop() {
	if (count == 0)
		throw logic_error();
	else
		return data[--count];
}
\end{cppcode}
	У этого кода нет гарантии Strong: если копирование при возвращении не удалось, состояние стека будет испорчено.
	Для этого метод разделили на два: \cpp'stack<T>::top()' и \cpp'stack<T>::pop()'.

\item[Basic:]
	Если исключение и будет создано, объект останется в \textbf{некотором} корректном состояни.
\end{description}
\subsection{Различия \t{auto} и \t{decltype}}
Пример $1$: 
\begin{cppcode}
int main () {
  int x = 5; 
  int& rx = x;
  auto y1 = rx; // int
} 
\end{cppcode}
Будет непонятно, чем является \cpp"y1", ссылкой или обычным переменной типа \cpp"int". 
На самом деле \cpp"auto" снимает ссылку. 
Но если мы четко хотим указать, что хотим получить ссылку, то надо просто написать \cpp"auto& y1 = rx".

Пример $2$:
\begin{cppcode}
int main () {
	int x = 5; 
	const int& crx = x; 
	auto y1 = crx; // int
	auto &y2 = crx; // const int &
}
\end{cppcode}
Компилятор всегда пробует снять константность.



\subsection{rvalue references, move semantic}
\begin{enumerate}
	\item rvalue "--- может стоять только в правой части выражения
	\item lvalue "--- нечто с именем, нечто, от чего можно ввзять ссылку, может стоять и в левой и в правой части выражения
\end{enumerate}
\begin{cppcode}
int main () {
  int a = 3; // l = r 
  int b = 5; // l = r
  a = b; // l = l
  b = a; // l = l
  a = a * b; // l = r
}
\end{cppcode}
Так же \cpp"rvalue" "--- являются все временные значения
Но все же \cpp"const int" "--- \cpp"lvalue".

Рассмотрим некоторый класс \cpp"X", у него есть:
\begin{enumerate}
  \item конструктор копий
  \item оператор присваивания
  \item деструктор
  \item \cpp"X::pRes" --- некоторый ресурс, которым владет объект.
\end{enumerate}

Как работает следующее выражение \cpp"a = b":

\begin{cppcode}
operator=(const X& rhs) {
    // освободить this->pRes
    // выделить память
    // скопировать из rhs.pRes в this->pRes
}
\end{cppcode}

Как работает следующий код:
\begin{cppcode}
X foo () {
	//...
}

int main () {
	X a = foo (); // copy constructor
}
\end{cppcode}
Здесь неявно создаётяся временная переменная (назовём её \cpp"tmp"), в которую записывается 
результат \cpp"foo", а потом из неё копируется в \cpp"a". Но можно сделать быстрее, просто поменяв 
указатели на память у \cpp"a" и \cpp"tmp". Научимся так делать. 

Теперь у нас будет два конструктора копий и два оператора присваивания: первые когда в правой части выражения стоит \t{lvalue}, 
а вторые когда в правой части стоит \t{rvalue}. 

Чтобы ловить \cpp"rvalue" используем \cpp"rvalue reference", выглядит так: \cpp"X&&"

Синтаксис оператора присваивания для \cpp"rvalue":
\begin{cppcode}
X& operator= (X&& rhs) {
  T* tmp = this->pRes; 
  this->pRes = this->pRes; 
  rhs.pRes = tmp;
  /* или swap (this->pRes, rhs->pRes) */
  return *this; 
}
\end{cppcode}

Итак, мы написали сейчас следующие методы:
\begin{cppcode}
  X (const X&); 
  X (X&&);
  X& operator= (const X&);
  X& operator= (X&&);
\end{cppcode}

Теперь вспомним про обычный \cpp"swap". 
\begin{cppcode}
template<class T>
void swap (T& a, T& b) {
  T tmp (a);
  a = b;
  b = tmp;
}
\end{cppcode}

Но этот swap не будет использовать наши новые конструкторы (т.к. мы инициализируем именованной переменной --- \cpp"lvalue")
Чтобы превратить \cpp"lvalue" в \cpp"rvalue"  надо использовать \cpp"std::move", которое возвращает \cpp"rvalue". 

Перепишем \cpp"swap":
\begin{cppcode}
template<class T>
void swap (T& a, T& b) {
  T tmp (std::move (a));
  a = std::move (b);
  b = std::move (tmp);
}
\end{cppcode}

Пример:
\begin{cppcode}
	vector<string> v;
	v.push_back (string ("Hello"));
	string s ("World");
	v.push_back (std::move (v));
\end{cppcode}

Этот приём, когда мы \cpp"lvalue" превращаем в \cpp"rvalue" начинаем по сути перемещать объект, а не копировать и называется
\cpp"move semantic".
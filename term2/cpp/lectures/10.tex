\subsection{Различия \t{auto} и \t{decltype}}
Пример $1$: 
\begin{cppcode}
int main () {
  int x = 5; 
  int& rx = x;
  auto y1 = rx; // int
} 
\end{cppcode}
Будет непонятно, чем является \cpp"y1", ссылкой или обычным переменной типа \cpp"int". 
На самом деле \cpp"auto" снимает ссылку. 
Но если мы четко хотим указать, что хотим получить ссылку, то надо просто написать \cpp"auto& y1 = rx".

Пример $2$:
\begin{cppcode}
int main () {
	int x = 5; 
	const int& crx = x; 
	auto y1 = crx; // int
	auto &y2 = crx; // const int &
}
\end{cppcode}
Компилятор всегда пробует снять константность.



\subsection{rvalue references}
\begin{enumerate}
	\item rvalue "--- тольков  в правой части выражения. 
	\item lvalue "--- и в левой и в правой части выражения.
\end{enumerate}
\begin{cppcode}
int main () {
  int a = 3; // l = r 
  int b = 5; // l = r
  a = b; // l = l
  b = a; // l = l
  a = a * b; // l = r
}
\end{cppcode}
Так же \cpp"rvalue" "--- являются все временные значения
Но все же \cpp"const int" "--- \cpp"lvalue".

Объект \cpp"x", он имеет:
\begin{enumerate}
  \item constructor
  \item copy
  \item destructor
\end{enumerate}

Как работает конструктор копий \cpp"a = b":
\begin{enumerate}
  \item освободить
  \item выделить память
  \item скопировать
\end{enumerate}

Как работает следующий код:
\begin{cppcode}
X foo () {
	//...
}

int main () {
	X a = foo (); // copy constructor
}
\end{cppcode}
Внутри есть какая-то переменная \cpp"tmp", в которую присвоется результат \cpp"foo". 
А потом скопируется в \cpp"a".

Но можно в принципе сделать быстрее, просто поменяв указатели на память у \cpp"a" и \cpp"tmp". 
Научимся делать так. 
Теперь у нас будет два конструктора копий первый когда в правой части выражения стоит \t{lvalue}, а второй когда в правой части стоит \t{rvalue}.
\begin{cppcode}
X& operator= (X&& rhs) {
  T* tmp = this = -> pRes; 
  this-> pRes = this.pRes; 
  rhs.pRes = tmp;
  /* или swap (this.pRes, rhs.pRes) */
  return this; 
}
\end{cppcode}

Написали сейчас следующие методы:
\begin{cppcode}
  X (const X&); 
  X (X&&);
  X& operator= (const X&);
  X& operator= (X&&);
\end{cppcode}

Теперь вспомним про обычный \cpp"swap". 
\begin{cppcode}
template<class T>
void swap (T& a, T& b) {
  T tmp (a);
  a = b;
  b = tmp;
}
\end{cppcode}
Но это не будет использовать наши новые конструкторы. 
Поэтому надо использовать \cpp"std::move", которое возвращает \cpp"rvalue". 
Перепишем теперь наш \cpp"swap".
\begin{cppcode}
template<class T>
void swap (T& a, T& b) {
  T tmp (std::move (a));
  a = std::move (b);
  b = std::move (tmp);
}
\end{cppcode}

Пример:
\begin{cppcode}
	vector<string> v;
	v.push_back (string ("Hello"));
	string s ("World");
	v.push_back (std::move (v));
\end{cppcode}

\subsection{move semantic}
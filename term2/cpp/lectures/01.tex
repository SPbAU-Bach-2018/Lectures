\chapter{Шаблоны}
\setauthor{Ольга Черникова}

\section{Дальнейший план действий}

\subsection{Отчетность:}

Баллы ставятся за домашние задания и практики, домашних заданий будет больше.
Что бы быть допущенным к экзамену нужно сдать все домашние задания не на ноль. 

Чем позже язык был придуман, тем больше у него стандартная библиотека.

В С стандартная библиотека тоже есть, но в ней много чего не хватало.
Прежде всего, структур данных.

Но перед  тем как изучать STL, нужно изучить еще две темы: шаблоны и исключения. 

На следующем занятии (20.02.2015) большой тест на по темам прошлого семестра.
Надо пересмотреть конспекты.

\section{Зачем нужны шаблоны?}
Когда мы делали классы, мы сокрушались, что явно должны были указывать тип.

\begin{cppcode}
class Array{
     int *p;
};
\end{cppcode}

И чтобы хранить \cpp'int' и \cpp'double' нужно было сделать два различных класса: \verb'ArrayInt' и \verb'ArrayDouble'.
И нам бы пришлось два раза скопировать код.

Копировать код "--- это не очень хорошо, поскольку надо читать в два раза больше кода и если нашлась ошибка в первом классе,
то нужно исправлять ее и во втором.

\section{Как люди пытались это решить в языке С?}

Это делалось с помощью \cpp'#define'

\verb'array.h'
\begin{cppcode}
class Array {
    type *a;
    ...
};

inline Type Array::get(...) {  
    ...
}
\end{cppcode}

Реализация методов должна быть в заголовочном файле, соответственно нужен \verb'inline'. В отдельном файле не получится это реализовать.

\verb'main.c'
\begin{cppcode}
#define Type int;
#include "array.h"

int main() {
    Array a;
}
\end{cppcode}

В данном примере можно использовать \cpp'typedef' и здесь это даже лучше, поскольку \cpp'#define' обрабатывается в режиме препроцессора
и работает со строками, ничего не зная о типах и так далее.

Теперь пусть в одном и том же файле хотим иметь и массив элементов типа \cpp'int' и \cpp'double'.
\verb'main.c'
\begin{cppcode}
#define Type int
#include "array.h"
#undef Type
#define Type double
#include "array.h"
#undef Type 

int main() {
    ...
}
\end{cppcode}

В результате в нас получилось что-то вроде:
\begin{cppcode}
class Array{
    int *a;
};

inline int Array::get...

class Array{
    double *a;
};

inline double Array::get...

int main() {
    ...
}
\end{cppcode}
Теперь у нас два объявления одинаковых классов.
Такого быть не может.
\cpp'Array ar;' "--- компилятор не сможет понять, что выбрать. 

Нужно писать в \verb'array.h'
\begin{cppcode}
class Array#Type{
    Type *a;
}

inline Array#Type::get(...) {

}
\end{cppcode}

Тогда \verb'#Type' заменится на тип и получится \verb'Arrayint'. Синтаксис позволяет заменить подслово.
\verb'My#Type#Array', если хотим заменить слово посередине строки.

\begin{cppcode}
int main() {
    Arrayint a;
    Arraydouble b;
}
\end{cppcode}

В \verb'#define' проблема состоит еще в том, что мы видим не то, что видит компилятор, и ошибки выдаются уже в измененном коде.

\section{Еще одна попытка сделать шаблоны в С} 

\verb'array.h'
\begin{cppcode}
#define DefineArray(name, type) \
class name{ \
   type *date; \
};
\end{cppcode}
Многострочный \cpp'#define'. В многострочном макросе строчки разделяются \verb'\'. 

Дальше идет полный код класса. 
\verb'main.c'
\begin{cppcode}
#include "array.h"
DefineArray(ArrayInt, int);

main() {
    ArrayInt a;
}
\end{cppcode}
Подставится полный текст класса с заменой имени и типа. 

\section{Шаблоны в С++. Синтаксис}
         
\begin{cppcode}
template <typename T>
class A {
    T* data;
    T get(size_t index) {
        return data[index];
    }
};
\end{cppcode}

\begin{cppcode}
template <typename T>
class A {
    T* data;
    T get(size_t index);
};

template <typename T>
T Array<T>::get(size_t index) {

}
\end{cppcode}

\cpp'template' обозначает, что следующая единица зависит от параметров. Параметров может быть несколько. 

Это должен быть один файл.
Это важно.
В разных файлах это быть не может, поскольку это не может быть скомпилированно.
Этот код находится в заголовочном (\verb'*.h') файле, но не в \verb'*.cpp' файле.

Иногда, чтобы код не разбухал, пишут несколько \verb'*.h' файлов и в конце первого пишут \cpp'#include'.

\verb'main.cpp'
\begin{cppcode}
#include "Array.h"
Array <int> a;
Array <double> b;
Array < Array <int> > aa; 

main() {
    int c = a.get(3);
    b.get(5);
}
\end{cppcode}

\cpp'Array< Array <int> > aa;' "--- здесь компилятор начинает путать шаблоны и сдвиг вправо, поэтому здесь лучше ставить пробел. 

Препроцессор вставит весь кусок, потом компилятор создаст новое имя класса. Разница в том, что замена имени произойдет на другом этапе. 
Нам ничто не мешает написать на место \verb'T' любой тип. 

\section{Требования к типу}

\cpp'Array< Array <int> > aa;' "--- здесь у класса \verb'Array' есть неявное требование.
\begin{cppcode}
Array(size_t size) {
    data = new T[size];
} 
\end{cppcode}
У \verb'T' должен быть конструктор по умолчанию.
Такие требования описаны только в документации или надо смотреть исходники.

\section{Функции с template}

\begin{cppcode}
template <typename V>
void reverse(Array<V> &a) {
    for (size_t i = 0; i < a.size()/2; ++i) {
        V t = a.get(i);
        a.set(i, a.get(a.size() - i - 1));
        a.set(a.size() - i - 1, t);
    }
} 

Array <int> a;
reverse <int>(a);
\end{cppcode}

При вызове функции надо указать параметры шаблона.
Так делать муторно, если у функции много параметров, достаточно написать \verb'reverse(a)' "--- компилятор сам попытается вывести нужный тип. 
Задание на подумать к следующему тесту: Приведите пример, когда компилятор не может вывести тип и его нужно указывать явно.

\begin{cppcode}
template <typename V, typename T>
void copy(Array <V>& a, Array <T>& b);

Copy <int, double> (c,d);
Copy (c d);
\end{cppcode}
Компилятор может вывести тип, если о них можно догадаться по параметрам. 

\section{Нужно написать функцию сортировки, которая может сортировать массивы любых типов}

\begin{cppcode}
sort(/*?*/ array, int size)
\end{cppcode}
Раньше эта сортировка выглядела так:
\begin{cppcode}
sort(void *array, size_t elem_size, size_t size, int (*comp)(void *, void *));
\end{cppcode}

Теперь как сделать эту сортировку в стили ООП?

\begin{cppcode}
class compar{
    virtal int cmp(const compar &) const = 0;
};

class Complex: public compar{
    int cmp(...) {
        if (....
    }
}
\end{cppcode}

То есть мы можем сортировать только те объекты, которые от наследованы от compar.

\begin{cppcode}
compar* array;

class Complex:public compar{
    int re;
    int im;
    int cmp(const compar * c) {
        Complex * comp = (Complex)c;
        if (re < comp -> re)
    }
}

array[i] = new Cat();
array[i + 1] = new Dog();
\end{cppcode}

Классы \verb'dog' и \verb'cat' наследуются от \verb'compar'.
Теперь у нас в массиве лежат разные типы, для которых по-разному определен оператор сравнения.
Проблема.

\cpp'compar** array;'
Тип должен быть \verb'**'.
Мы сделали базовый класс compar (или интерфейс) и в итоге выяснили, что тип в \verb'sort' должен быть \verb'compar **', что бы в памяти объекты занимали 4 байта и 
можно было менять элементы местами.
Потенциальная ошибка: наличие в массиве разных типов. 

\begin{cppcode}
template <typename T>
void sort(T* array, size_t size) {
}
\end{cppcode}
Здесь можно писать указатель, а не \verb'**', поскольку она будет скомпилирована, только когда у нас будет уже конкретный массив собак.

\begin{cppcode}
Dog *d = new Dog[100];
sort <Dog> (d, 100);

int *i = new int[100];
sort <int> (i, 100);
\end{cppcode}

Неявное требование к типам: у типа должен быть \cpp'operator <'.

\section{Константы в шаблонах}

В шаблонную переменную можно передавать константу. 

\begin{cppcode}
template <typename T, size_t Size>
class Array{
    public: 
        T arr[Size];
};
\end{cppcode}

\cpp'Array<int, 100> a;' "--- значит \verb'а'' будет массив размера 100.

\section{Cпециализация}

Если мы хотим, например, сэкономить память и \cpp'bool' хранить в \cpp'char'.

\begin{cppcode}
template <typename T>
class Array{ 

}
\end{cppcode}

Отдельно выделяем класс \verb'bool' и его специализацию нужно написать целиком.
То есть, если другой тип, будет использован первый вариант, второй специально для \cpp'bool'.

\begin{cppcode}
template <>
class Array<bool>{ 
    set(){
    }
    char *data;
}
\end{cppcode}

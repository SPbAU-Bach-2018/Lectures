\section{Шаблоны}
\subsection{Дальнейший план действий}

{\bf Отчетность:}

Баллы ставятся за домашние задания и практики, Домашних заданий будет больше. Что бы быть допущенным к экзамену нужно сдать все домашние задания не на ноль. 


Чем позже язык был придуман, тем больше у него стандартная библиотека. 

В С стандартная библиотека тоже есть, но в ней много чего не хватало. Прежде всего, структур данных. 

Но перед  тем как изучать STL, нужно изучить еще две темы: шаблоны и исключения. 

На следующем занятие(20.02.2015) большой тест на по темам прошлого семестра. Надо пересмотреть конспекты. 


\subsection{Зачем нужны шаблоны?}
Когда мы делали классы, мы сокрушались, что явно должны были указывать тип. 

\begin{cppcode}
class Array{
     int *p;
};
\end{cppcode}

И что бы хранить int и double нужно было сделать два различных класса: ArrayInt и ArrayDouble. 
И нам бы пришлось два раза скопировать код. 

Копировать код ~--- это не очень хорошо, поскольку надо читать в два раза больше кода и если нашлась ошибка в первом классе, то нужно исправлять ее и во втором.

\subsection{Как люди пытались это решить в языке С?}

Это делалось с помощью \#define

{\bf array.h}

\begin{cppcode}
class Array{
Type *a;
...
};

inline Type Array::get(..){  
...
}
\end{cppcode}

Реализация методов должна быть в заголовочном файле, соответственно нужен inline. В отдельном файле не получится это реализовать.

{\bf В основной программе} 
\begin{cppcode}
#define Type int;
#include "array.h"

int main() {
    Array a;
}
\end{cppcode}
         

В данном примере можно использовать typedef и здесь это даже лучше, поскольку define обрабатывается в режиме препроцессора и работает со строками, ничего не зная о типах и так далее.

Теперь пусть в одном и том же файле хотим иметь и массив элементов типа int и double. 

{\bf main.c}
\begin{cppcode}
#define Type int
#include "array.h"
#undef Type
#define Type double
#include "array.h"
#undef Type 

int main() {
}
\end{cppcode}

В результате в нас получилось что-то вроде: 

\begin{cppcode}
class Array{
    int *a;
};

inline int Array::get...

class Array{
    double *a;
};

inline double Array::get...

int main() {
}
\end{cppcode}

Теперь у нас два объявления одинаковых классов. Такого быть не может. 

Array ar; ~--- компилятор не сможет понять, что выбрать. 

Нужно писать в {\bf array.h}
\begin{cppcode}
class Array#Type{
    Type *a;
}

inline Array#Type::get(...) {

}
\end{cppcode}

Тогда \#Type заменится на тип и получится Arrayint. Синтаксис позволяет, заменить подслово.

My\#Type\#Array. Если хотим заменить слово посередине строки.

\begin{cppcode}
int main() {
    Arrayint a;
    Arraydouble b;
}
\end{cppcode}

В define  проблема состоит еще в том, что мы видим не то, что видит компилятор, и ошибки выдаются уже в измененном коде.

\subsection{Еще одна попытка сделать шаблоны в С} 

{\bf array.h}

\begin{cppcode}
\#define DefineArray(name, type)\
class name{\
   type *date;\
};
\end{cppcode}

Многострочный define. В многострочном define строчки разделяются \\

Дальше идет полный код класса. 

{\bf main.c}

\begin{cppcode}
#include "array.h"
DefineArray(ArrayInt, int);

main() {
    ArrayInt a;
}
\end{cppcode}

Подставится полный текст класса с заменой имени и типа. 

\subsection{Шаблоны в С++. Синтаксис}
         
{\bf Array.h}
\begin{cppcode}
template <typename T>
class A {
    T* data;
    T get(size_t index) {
        return data[index];
    }
};
\end{cppcode}

{\bf Array.h}

\begin{cppcode}
template <typename T>
class A {
    T* data;
    T get(size_t index);
};

template <typename T>
T Array<T>::get(size_t index) {

}
\end{cppcode}

template обозначает, что следующая единица зависит от параметров. Параметров может быть несколько. 

Это должен быть один файл. Это важно. В разных файлах это быть не может, поскольку это не может быть скомпилированно. Этот код находится в h файле, но не в cpp файле.

Иногда, чтобы код не разбухал, пишут несколько h файлов и в конце первого пишут \#include

{\bf main.cpp}

\begin{cppcode}
#include "Array.h"
Array <int> a;
Array <double> b;
Array < Array <int> > aa; 

main() {
    int c = a.get(3);
    b.get(5);
}
\end{cppcode}

Array < Array <int> > aa; ~--- здесь компилятор начинает путать шаблоны и сдвиг вправо, поэтому здесь лучше ставить пробел. 

Препроцессор вставит весь кусок, потом компилятор создаст новое имя класса. Разница в том, что замена имени произойдет на другом этапе. 

Нам ничто не мешает написать на место T любой тип. 

\subsection{Требования к типу}

Array< Array <int> > aa; Здесь у класса Array есть неявное требование.
\begin{cppcode}
Array(size_t size) {
    data = new T[size];
} 
\end{cppcode}

У T должен быть конструктор по умолчанию.

Такие требования описаны только в документации или надо смотреть исходники.

\subsection{Функции с template}

\begin{cppcode}
template <typename V>
void reverse(Array <V> & a) {
    for (size_t i = 0; i < a.size()/2; ++i) {
        V t = a.get(i);
        a.set(i, a.get(a.size() - i - 1));
        a.set(a.size() - i - 1, t);
    }
} 

Array <int> a;
reverse <int> (a); 
\end{cppcode}

При вызове функции надо указать параметры шаблона. 

Так делать муторно, если у функции много параметров, достаточно написать reverse(a) компилятор сам выведет нужный тип. 

Задание на подумать к следующему тесту: "Приведите пример, когда компилятор не может вывести тип и его нужно указывать явно."

\begin{cppcode}
template <typename V, typename T>
void copy(Array <V>& a, Array <T>& b);

Copy <int, double> (c,d);
Copy (c d);
\end{cppcode}

Компилятор может вывести тип, если о них можно догадаться по параметрам. 

\subsection{Нужно написать функцию сортировки, которая может сортировать массивы любых типов.}

sort(? array, int size)

Раньше эта сортировка выглядела так:
sort(void *array, size\_t elem\_size, size\_t size, int (*comp)(void *, void *));

Теперь как сделать эту сортировку в стили ООП?

\begin{cppcode}
class compar{
    virtal int cmp(const compar &) const = 0;
};

class Complex: public compar{
    int cmp(...) {
        if (....
    }
}
\end{cppcode}

То есть мы можем сортировать только те объекты, которые от наследованы от compar.

\begin{cppcode}
compar* array;

class Complex:public compar{
    int re;
    int im;
    int cmp(const compar * c) {
        Complex * comp = (Complex)c;
        if (re < comp -> re)
    }
}

array[i] = new Cat();
array[i + 1] = new Dog();
\end{cppcode}

Классы dog и cat наследуются от compar.

Теперь у нас в массиве лежат разные типы, для которых по-разному определен оператор сравнения. Проблема. 

compar** array;  Тип должен быть **.

Мы сделали базовый класс compar(или интерфейс) и в итоге выяснили, что тип в sort должен быть compar **, что бы в памяти объекты занимали 4 байта и 
можно было менять элементы местами. Потенциальная ошибка: наличие в массиве разных типов. 

\begin{cppcode}
template <typename T>
void sort(T* array, size_t size) {
}
\end{cppcode}

Здесь можно писать указатель, а не **, поскольку она будет скомпилирована, только когда у нас будет уже конкретный массив собак.

\begin{cppcode}
Dog *d = new Dog[100];
sort <Dog> (d, 100);

int *i = new int[100];
sort <int> (i, 100);
\end{cppcode}

Неявное требование к типам, у типа должен быть оператор <.

\subsection{Константы в шаблонах}

В шаблонную переменную можно передавать константу. 

\begin{cppcode}
template <typename T, size_t Size>
class Array{
    public: 
        T arr[Size];
};
\end{cppcode}

Array <int, 100> a; ~--- значит а будет массив размера 100.

\subsection{Cпециализация}

Если мы хотим, например, сэкономить память и bool хранить в char.

\begin{cppcode}
template <typename T>
class Array{ 

}
\end{cppcode}

Отдельно выделяем класс bool  и его нужно написать целиком. То есть, если другой тип, будет использован первый вариант, второй специально для bool.

\begin{cppcode}
template <>
class Array<bool>{ 
    set(){
    }
    char *data
}
\end{cppcode}




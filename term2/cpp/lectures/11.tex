\subsection{placement new}
Это особоя форма \cpp"new", которая позволяет в выделенную память размещать объект с помощью конструктора.
Так же полезно, когда у класса не существует конструктора по умолчанию и при этом требуется создать массив объектов.
Синтаксис следующий:

\begin{cppcode}
#include <new>

class Packet {
private: 
    int info;
public:
    Packet (int info = 0) : info (info) {}
};

int main () {
    char* buffer = new char[sizeof (Packet)];
    Packet *p = new (buffer) Packet();

    p->~Packet ();
    delete [] buffer;
}
\end{cppcode}

Теперь чтобы удалить выделенные объекты, сначала нужно от каждого выделенного объекта вызвать деструктор, а затем удалить выделенную для них память, то что и происходит в последних строках, приведенного выше примера.
На самом деле и без строки \cpp"p->~Packet ();" будет все работать, так как в внутри объекта не выделяется память, но так делать не рекомендуется.

\section{lambda calculus}

\subsection{for\_each}
\begin{cppcode}
#include <algorithm>

template<class InputIt, class Function>
Function for_each (InputIt first, InputIt last, Function fn) {
    while (first!=last) {
        fn (*first);
        ++first;
    }
    return std::move (fn);
}
\end{cppcode}

Как видно по коду \cpp"for_each" применяет функцию \cpp"fn" ко всем элементам, находящимся в диапазоне $[first, last)$.
Примерно такая реализация этой функции в стандартной библиотеке.

Пример использования:

\begin{cppcode}
#include <algorithm>
#include <vector>

void f (int &n) {
    n = n * n;
}

class Func {
public:
    void operator () (int &n) {
        n = n * n;
    }
};

int main () {
    std::vector<int> v = {1, 3, 5, 7};
    for_each (v.begin (), v.end (), f);
    for_each (v.begin (), v.end (), Func ());
    return 0;
}
\end{cppcode}

Два вызова \cpp"for_each" делают абсолютно одинаковые вещи, просто реализованы по-разному, за исключением того, что у функтора есть свои преимущества.

\subsection{transform}
\begin{cppcode}
template<class InputIt, class OutputIt, class UnaryOperation>
OutputIt transform (InputIt first1, InputIt last1, OutputIt d_first, 
                    UnaryOperation unary_op) {
    while (first1 != last1) {
        *d_first++ = unary_op(*first1++);
    }
    return d_first;
}
\end{cppcode}

\cpp"transform" применяет оператор \cpp"unarry_op" ко всем элементам, находящимися в диапазоне $[first1, last1)$, и соохраняет результат в диапазоне, начинающимся с $d\_first$.

Пример использования:

\begin{cppcode}
#include <algorithm>
#include <vector>

double f (const double &d) {
    if (d < 0.001)
        return 0;
    return d;
}

class Func {
    double eps; 
public:
    Func (double eps) : eps (eps) {}

    double operator () (const double &d) {
        if (d < eps)
            return 0;
        return d;
    }
};

int main () {
    std::vector<double> v = {1, 0.1, 0.01, 0.001, 0.0001};
    transform (v.begin (), v.end (), v.begin (), f);
    transform (v.begin (), v.end (), v.begin (), Func (0.01));
    return 0;
}
\end{cppcode}

В данном случае функтор получился функциональнее обычной функции. 
Так же константные ссылки не обязательны и можно спокойно менять значения, которые передаются в функтор или функцию. 

Бонус. 
\cpp"transform" так же определен и для \cpp"BinaryOperation".

Примерная релизация:
\begin{cppcode}
template<class InputIt1, class InputIt2,
         class OutputIt, class BinaryOperation>
OutputIt transform(InputIt1 first1, InputIt1 last1, InputIt2 first2,
                   OutputIt d_first, BinaryOperation binary_op) {
    while (first1 != last1) {
        *d_first++ = binary_op(*first1++, *first2++);
    }
    return d_first;
}
\end{cppcode}

\subsection{lambda function}
Лямбда функции призваны уменьшить количество символов, так как позволяют записывать функторы более кратко. 

Посмотрим синтаксис:

\begin{cppcode}
#include <vector>
#include <algorithm>

int main () {
    std::vector<double> v = {0.1, 0.01, 0.001, 0.0001};

    for_each (v.begin (), v.end (), 
        [] (double &d) {
            if (d < 0.001)
                d = 0;
        }
    );
    
    transform (v.begin (), v.end (), v.begin (), 
        [] (const double &d) -> double {
            if (d < 0.001)
                return 0;
            return d;
        }
    );
    return 0;
}
\end{cppcode}

Описанные лямбда функции фактически соответствуют тем функторам, которые были написаны в двух предыдущих пунктах про \cpp"for_each" и \cpp"transform".

Разберемся теперь в синтаксисе.
В квадратных скобках \cpp"[]" указан \cpp"capture_list". 
Он позволяет передавать какие-то переменные в функтор. 
Рассмотрим на примерах:

\begin{enumerate}
\item \cpp"[a, &b]" "--- в этом случае \cpp"a" захватывается по значению, а \cpp"b" по ссылке.
\item \cpp"[&]" "--- захватывает все переменные по ссылке.
\item \cpp"[=]" "--- захватывает все переменные по значению.
\item \cpp"[]" "--- ничего не захватывает.
\end{enumerate}

В круглых скобках, описываются такие же параметры как и в функторах. 
Между круглой закрывающейся скобкой и \cpp"->" могут быть ключевые слова.
После \cpp"->" указывается возвращаемое значение лямбда функции.

Важное замечание, если захватывать по значению, то переменные будут доступны как \cpp"const".
Чтобы это исправить следует использовать \cpp"mutable", такой синтаксис выглядит так:

\cpp"[=] (const double &d) mutable -> double". 

Еще пример:

\begin{cppcode}
#include <vector>
#include <algorithm>

int main () {
    std::vector<double> v = {0.1, 0.01, 0.001, 0.0001};
    
    int epsilon = 0.001; 
    for_each (v.begin (), v.end (), 
        [epsilon](double &d) -> void {
            if (d < epsilon)
                d = 0;
        }
    );

    return 0;
}
\end{cppcode}





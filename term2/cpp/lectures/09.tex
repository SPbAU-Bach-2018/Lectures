\chapter{c++11}
\section{Новые ключевые слова}

\subsection{Ключевое слово \t{default}}

\begin{cppcode}
class A {
public:
	A (int);
	A () = default;
};
\end{cppcode}

Разумное использование:
\begin{cppcode}
class B {
public:
	B (int);
protected:
	B (const &B) = default;
};
\end{cppcode}

\subsection{Ключевое слово \t{delete}}

Предназаначение удалять что-то из класса, чтобы это нельзя было использовать.
Следующий код запретит не все:
\begin{cppcode}
class A {
public:
	A (const &A);
	f ();

	friend g();
};
\end{cppcode}
В функциях \t{g} и \t{f} все еще можно будет использовать приведение типов.

Поэтому надо использовать \cpp"= delete":
\begin{cppcode}
class A {
public:
	A (const &A) = delete;
};
\end{cppcode}

\subsection{Ключевое слово override}

\cpp"override" строго говорит что функция будет перекрыта.

\begin{cppcode}
/* в Base.h*/
class Base {
public:
	virtual void f (int);
	virtual int g () const;
	void h () {
		g ();
	}
};

/* в Derived.h*/
class Derived: public Base {
public:
	void f (int) = override; // Ok
	virtual int g () = override; // Error
};
\end{cppcode}
Функции \t{f} "--- одинаковые и вторая перекрывает первую, функции \t{g} "--- разные, так как одна реализована для \cpp'const' класса.

\subsection{Ключевое слово \t{final}}

Вспомним пример про рабочих и их зарплату.
Слово \cpp'final' говорит что функцию уже нельзя будет перекрывать.
Можно писать где первая строка описания класса.
\begin{cppcode}

class Worker {
	virtual int getSalary () = 0;
};

class Developer: public Worker {
	virtual int getSalary () = override, final {
		return 100;
	}
};

int main () {
	Worker *w;
	/* считываем n */
	/* динамическое связывание */

	if (n > 1)
		w = Developer ();
	else
		w = Tester ();
	w->getSalary ();

	/* статическое связывание */
	Developer *d = new Developer;
	d->getSalary ();
}
\end{cppcode}

\subsection{Ключевое слово \t{nullptr}}

\begin{cppcode}
f(long);
f(char *);

int main () {
	f (NULL); //error
	f (0l); /* будет вызывать f(long) */
	f ((char*)NULL); /* будет вызывать f(char*) */
	f (static_cast<char*>(NULL)); /* будет вызывать f(char*) */
	f (nullptr); /* будет вызывать f(char*) */
}
\end{cppcode}

\subsection{Конструкторы}

\begin{cppcode}
class Board {
public:
	static const size_t BOAD_SIZE = 8;
	int board [BOARD_SIZE];
	int x = 10; // c++11
	int y = 10; // c++11
};
\end{cppcode}

\subsection{constructor chaining}

Позволяет использовать нормально в одном контрукторе другие.

Как могли бы написатью
\begin{cppcode}
class Complex {
private:
	int myRe;
	int myIm;
public:
	Complex (int re = 0, int im = 0);
	Complex (int* arr) {
		Complex (arr[0], arr[1]);
		/* будет создан новый Complex и удалится в той же строке*/
	}
}
\end{cppcode}

Как надо:
\begin{cppcode}
class Complex {
private:
	int myRe;
	int myIm;
public:
	Complex (int re = 0, int im = 0);
	Complex (int* arr): Complex (arr[0], arr[1]) {}
}
\end{cppcode}
Но конструктор может бросить исключение.
Поэтому перепишем следующим образом:

\begin{cppcode}
class Complex {
private:
	int myRe;
	int myIm;
public:
	Complex (int re = 0, int im = 0);
	Complex (int* arr) try: Complex (arr[0], arr[1]) {
		//...
	}
	catch (){

	}
}
\end{cppcode}

\subsection{Списки инициализации}
\begin{cppcode}
stuct Point {
	int x;
	int y;
};

int main () {
	int arr[] = {3, 6, 9};
	Point p {0, 1};
	//C++11
	vector<int> a {3, 5, 6};
}
\end{cppcode}
\cpp"vector<int> a {3, 5, 6};" по этой штуке компилятор генерирует \cpp"std::inirializer_list".

\begin{cppcode}
template<typename T>
class vector {
public:
	vector (std::initializer_list<T> l) {
		size_t size = l.size ();
		const T* it = l.begin ();
		for (int i = 0; i < size; ++i)
			cout << *(it++); //it[i]
	}
};
\end{cppcode}

\section{Еще нововведения}

\subsection{\t{auto}}

Во время комипляции.
\begin{cppcode}
std::vector<int>::iterator it = v.begin ();
auto it = v.begin ();
int x = 3;
auto y = x;
\end{cppcode}

\subsection{\t{decltype}}
\begin{cppcode}
template<typename T, typename V>
void multiply (const T& t, const V& v) {
	typedef decltype (t * v) res_t;
	res_t r = t * v;
}

//хотелось, но нельзя, так как t, v раньше объявления находятся
template<typename T, typename V>
decltype (t * v) multiply (const T& t, const V& v) {
	auto r = t * v;
	return r;
}

//поэтому так
auto multiply (const T& t, const V& v) -> decltype (t * v) {
	auto r = t * v;
	return r;
}
\end{cppcode}

\section{std::array}
\begin{cppcode}
std::array<int> a (10);
\end{cppcode}

\section{Функции begin и end}
\begin{cppcode}
int res = accamulate (begin (a), end (a), 0, sum);
\end{cppcode}

\section{foreach}
Синтаксический сахар.

\begin{cppcode}
int main () {
  vector <int> {1, 2, 3};
  for (auto x : v) {
    cout << x << "\n";
  }
  for (auto it = v.begin (); it != v.end (); ++it) {
   cout << x << "\n"; 
  }

}
\end{cppcode}

Первый цикл развернеться во второй.

\section{Новые контейнеры}

\subsection{forward\_list}
Односвязный список.

\subsection{unordered\_set и unordered\`map}
Как сделать хеш, для своего объекта. 
\begin{cppcode}

#include <unordered_set>

class A {
  int x;
  string s;
  bool operator=(const A& rhs) {
    return rhs.a = a;
  };
  friend size_t std::hash <A>;
};

unordered_set<class T, class hash_func = std::hash<T>, class pred = equal_to<T>>;

namespace std {
  template<>
  struct hash<A> {
    size_t operator (const A& rhs ) const {
      size_t val = 0;
      val += std::hash<int> () (rhs.x) * 31;
      val += std::hash<string> () rhs.s * 59;
    }
  };
};

int main () {


}
\end{cppcode}


\subsection{}






\chapter{Метапрограммирование}
\setauthor{Дмитрий Лапшин}

Вспомним traits.

Что такое метапрограммирование?
\begin{enumerate}
\item
	Программа при компиляции может создать свою новую часть.
\item
	Программа при работе меняет саму себя.
\end{enumerate}
В C++ реализуема только первая часть, вторая касается больше полиморфных вирусов и Java.

Как мы могли уже написать один код, чтобы был доступен разный функционал без дублирования кода?
\begin{description}
\item[Указатель на фунцию]
	Так, как в Си делали.
\item[Виртуальные функции]
	Наследники реализуют конкретный функционал.

	У обоих подходов есть проблема производительности: идёт вызов функции, идёт расчёт, что вызывать.

\item[Обобщённое программирование (шаблоны)]
	Тут всё быстрее, так как всё происходит в момент компиляции.
	Вспомним прошлые лекции: мы написали код, порождающий функции и структуры:
\begin{cppcode}
int sum();
int sum(int);
int sum(int, int);
// ...
\end{cppcode}
Или вот в момент компиляции считаем факториал:
\begin{cppcode}
template<int n>
struct fact {
	static const int value = n * fact<n - 1>::value;
};
template<>
struct fact<0> {
	static const int value = 1;
};
\end{cppcode}
\end{description}

Что такое trait? trait "--- объект, хранящий информацию о деталях реализации другого типа.
Это исползуется для различных реализации для различных типов.

Например, шаблон принимает параметр: численный тип.
Файл \t{limits.h} содержит макросы, объявляющие константы.
Пока нам нужно просто писать специализации, что грустно.

Но в C++ есть стандартная структура-шаблон \cpp'std::numeric_limits', которая принимает тип-число и содержит методы:
\begin{cppcode}
template<typename T>
struct std::numeric_limits<class T> {
	static T max();
	//...
	static const bool is_integer = false;
	//...
};

template<>
struct std::numeric_limits<char> {
	static T max() {
		return CHAR_MAX;
	};
};
\end{cppcode}
Чуда не произошло, эти специализации всё ещё нужно писать, но теперь не в каждом алгоритме:
\begin{cppcode}
T largest = std::numeric_limits<T>::max();
\end{cppcode}
Вопрос на подумать: а зачем \cpp'max()' функция?

\begin{cppcode}
template<typename T>
struct is_pointer {
	static const bool value = false;
};

template<typename T>
struct is_pointer<T*> {
	static const bool value = true;
};

//...

bool b = is_pointer<T>::value;
if (b) {
	//...
}
else {

}
\end{cppcode}
Мы хотим, чтобы тут даже не доходило до компиляции ненужной ветки \cpp'if'-а.
В Си это выглядело бы так:
\begin{cppcode}
void f(int a) {
	//...
}

void f_optimized(int a) {
	//...
}

void real_f(int a) {
	#ifdef _SUPPORT_MMX
		f_optimized(a);
	#else
		f(a);
	#endif
}
\end{cppcode}
Мы не хотим препроцессор.

Как это делает старый стандарт:
\begin{cppcode}
template<bool b> // включать ли быструю версию
struct alg_selector {
	template<typename T>
	static void implementation(T& obj) {
		// медленная реализация
	}
};

template<>
struct alg_selector<true> {
	template<typename T>
	static void implementation(T& obj) {
		// быстрая реализация
	}
};

template<>
void algorithm(T& obj) {
	alg_selector<is_optimized<T>::value>::implementation(obj);
}

// struct ObjectB оптимальная, struct ObjectA нет

template<typename T>
struct is_optimized {
	static const bool value = false;
}

template<>
struct is_optimized<ObjectB> {
	static const bool value = true;
}

int main() {
	ObjectA a;
	ObjectB b;
	algorithm(a);
	algorithm(b);
}
\end{cppcode}

На экзамене попросят написать реализацию выше с чем-то содержательным.
Кстати, в стандартной библиотеке \cpp'#include <type_traits>' уже есть много подобных фич.

Есть механизм SFINAE (Suptitiution Failure Is Not An Error).
\begin{cppcode}
template<typename T>
struct has_iterator {
	template <typename U>
	static char test(typename U::iterator x);

	template <typename U>
	static long test(U* x);

	static const bool value = sizeof(test<T>(0)) == 1;
}

int main() {
	int a = 128;

}
\end{cppcode}
Там выше есть бага, но мы позже разберёмся.
Суть в том, что можно на уровне языка проверять наличие каких-то типов, методов и полей у структур  и прочего.

Теперь новый стандарт. На принципе выше есть конструкция \cpp'std::enable_if':
\begin{cppcode}
template<typename T>
typename enable_if<!has_iterator<T>::value, void>::type show(T& x) {
	cout << x << "\n";
}

template<typename T>
typename enable_if<has_iterator<T>::value, void>::type show(T& x) {
	for (auto& i: x)
		cout << i;
}
\end{cppcode}
Что делает \cpp'std::enable_if'?
Оно принимает два параметра "--- \cpp'bool' и некий тип.
Если выражение итинно, то \cpp'std::enable_if::type' просто равен второму параметру.
Иначе, подстановка не удаётся через SFINAE (подстановка не удаласть, но нет ошибки компиляции), и метод не создаётся.

В любом случае, будет создана только одна интанация метода \cpp'show' "--- только та, у которой условие верно.
\setauthor{Дмитрий Лапшин}

Хотим перед каждым чётным числом вставить нолик.
\begin{cppcode}
vector<int> v;
v.push_back();

// ++it эффективнее it++
for (vector<int>::iterator it = v.begin(); it != v.end(); ++it) {
	if (*it % 2 == 0)
		v.insert(it, 0);
}
\end{cppcode}
Тут возникает вопрос про инвалидацию итераторов.
Итератор "--- некоторая абстракция над указателем, как мы помним.
\cpp'insert' вставляет элемент ровно перед тем элементом, на который передали итератор.
\begin{center}
\begin{tabular}{c|cccccc}
Было  & 0 & 2 & 1 & 3 & \\
Стало & 0 & 0 & 2 & 1 & 3
\end{tabular}
\end{center}
Как видно, после итерации цикла итератор всё ещё указывает на чётный элемент. Это не то, чего мы хотели.
\begin{cppcode}
for (vector<int>::iterator it = v.begin(); it != v.end(); ++it) {
	if (*it % 2 == 0) {
		v.insert(it, 0);
		++it;
	}
}
\end{cppcode}
Теперь работает, но инвалидация указателей всё ещё сломана.
Если было перевыделение памяти, то итератор стал сломан и указывает на некорректную память. Поэтому стоит делать так:
\begin{cppcode}
for (vector<int>::iterator it = v.begin(); it != v.end(); ++it) {
	if (*it % 2 == 0) {
		it = v.insert(it, 0);
		++it;
	}
}
\end{cppcode}

\subsection{\texttt{string}}

Есть строки. Есть строки с ASCII"---кодировкой
\begin{cppcode}
typedef basic_string<char> string;
\end{cppcode}
Можно объявить свою строку с Unicode через тип \cpp'wchar_t'.

Строки очень удобно обрабатывать благодаря перегруженным операторам.
\begin{cppcode}
string s1 = "Hello, ", s2 = "World!";
string s3 = s1 + s2;
\end{cppcode}

Строки работают по принципу Copy-on-Write (CoW).
Это означает, что при копировании из одного контейнера в другой будет скопирован только указатель,
и он будет так храниться до тех пор, пока контейнер не будет изменён.
Только тогда произойдёт выделение новой памяти.
\begin{cppcode}
string s1 = "Hello"; // в памяти один раз "Hello"
string s2 = s1; // в памяти один раз "Hello"
s2 += "!"; // в памяти "Hello" и "Hello!"
\end{cppcode}
Но, например, такое не сможет сэкономтиь память (хотя в Java так работает):
\begin{cppcode}
string s1 = "Hello!"; // в памяти один раз "Hello!"
string s2 = "Hello!"; // в памяти два раза "Hello!"
\end{cppcode}

\subsection{\texttt{deque}}
\cpp'deque' "--- по-сути, такой же \cpp'vector' (добавление в конец, доступ к элементу), только ещё добавление и удаление из начала
(\cpp'push_front' и \cpp'pop_front') за амортизированное $O(1)$.

Типовая реализация такая: есть массив указателей на области памяти одинакого размера, а также указатель на начало и конец данных.
Если нам нужно перевыделить память, то мы только копируем массив указателей и выделяем новые фрагменты, а старые не трогаем.

\section{Работа с итераторами в шаблонах}

\begin{cppcode}
template <typename T>
f(T* a, size_t n);

vector<int> a;
f(&(*T.begin()), v.size());
\end{cppcode}

Выглядит не очень. У \cpp'string' есть решение "--- метод \cpp'c_str()' вернёт чистую строчку \cpp'char *'.

\section{\texttt{pair}}
\cpp'pair' "--- шаблонный класс, позволяющий хранить два элеметна любого типа:
\begin{cppcode}
pair<int, string> a;
a.first = 20;
a.second = "Meaow!";
\end{cppcode}

\section{Ассоциативные контейнеры (\texttt{set}, \texttt{multiset}, \texttt{map})}

\subsection{\texttt{set}}
\cpp'set' представляет собой реализацию массива с поиком.
Внутри он реализован как сбалансированное дерево поиска.
Именно поэтому почти все операции работают за $O(\log n)$:
\begin{itemize}
	\item \verb'insert'
	\item \verb'erase'
	\item \verb'lower_bound'
	\item \verb'upper_bound'
	\item \verb'find'
\end{itemize}

Метод \cpp'insert' возвращает не просто итератор на добавленный элемент.
\verb'set' хранит каждый элемент не более одного раза.
Поэтому \cpp'insert' возвразщает \cpp'pair<set<T>::iterator, bool>', где второй элемент пары указывает, 
был ли добавлен элемент, а не найден уже находящийся в контейнере.

Итератор у \cpp'set' ассоциативен в том смысле, что идёт не в порядке добаления элементов, а в порядке их возрастания.
\begin{cppcode}
set<int> s;
s.insert(10);
s.insert(5);
s.insert(1);
// Здесь it++ гораздо менее эффективен
for (set<int>::iterator it = s.begin(); it != s.end(); ++it)
	cout << *it << " "; //1 5 10
\end{cppcode}

С точки зрения инвалидации, итераторы у \cpp'set' аналогичны \cpp'list': итератор на элемент, лежащий в структуре, всегда корректен.
Именно поэтому просто нельзя делать так (не компилируется, это бы нарушало свойства дерева):
\begin{cppcode}
for (set<int>::iterator it = s.begin(); it != s.end(); ++it)
	if (*it == 5)
		*it = 6;
\end{cppcode}
\cpp'set<T>::iterator::operator *' возвращает \cpp'T const &'.
Если всё же нужно поменять значение в множестве, делаем \cpp'erase()' и \cpp'insert()'.

Если нужно потроить \cpp'set()' от нашей структуры, в ней необходим оператор сравнения
\begin{cppcode}
bool operator <(T const &b) const
\end{cppcode}
Принято, что достаточно его одного. \cpp'set' сам реализует остальные сравнения через \cpp'<', хотя это не так эффективно:
\cpp'a = b' через \cpp'!(a < b) && !(b < a)'.

\subsection{\texttt{multiset}}

\cpp'multiset' аналогичен \cpp'set', но может хранить равные значения сколь угодно много раз.

Важное отличие: операция \cpp'multiset<T>::erase' может принимать как итератор, так и значение. Первый вариант удалит только одно значение, второй "--- все равные.

\subsection{\texttt{map}}

\cpp'map' реализует ассоциативный массив. На самом деле он является надстройкой над \cpp'set<key, value>'
"--- мы ищем значение по ключу и храним некоторое значение:
\begin{cppcode}
map<string, int> age;
age.insert(pair<string, int>("vaysa", 18));
\end{cppcode}
Чего-то грустно. В \verb'utility' есть крутая шаблонная функция, и компилятор попытается вывести типы:
\begin{cppcode}
age.insert(make_pair("vasya", 18));
\end{cppcode}

Как она устроена?
\begin{cppcode}
template <typename T1, typename T2>
pair<T1, T2> make_pair(T1 a, T2 b);
\end{cppcode}

Ещё немного пар:
\begin{cppcode}
map<string, int> age;
cout << age.begin()->first; // string const &
\end{cppcode}

Всё же, хочется работать удобнее. Есть такой оператор:
\begin{cppcode}
age["petya"] = 32;
\end{cppcode}
Он, если находит элемент, позволяет его изменить. При этом, если такого ключа не было, он его создаст:
\begin{cppcode}
int x = age["peter"]; // 0
\end{cppcode}
Поэтому \cpp'map::operator []' недоступен у \cpp'const map'.

\section{Comparator-ы}

У нас может возникнуть необходимость строить \cpp'set' или \cpp'map' от нашей струтуры, но с разными порядками.
В Си мы это делали бы, передавая указатель на оракул сравнения (компаратор).
В ООП мы бы создали наследника класса \verb'Comparator', и через передачу его экзмепляра и виртуальные методы сравнивать объекты.

C шаблонами придумали несколько свой подход. Мы передаём функтор (объект-функция с перегруженным \cpp'operator ()'):
\begin{cppcode}
template <typename T>
class less {
	bool operator ()(T const &a, T const &b) const {
		return a < b;
	}
}
\end{cppcode}
\setauthor{Елизавета Третьякова}
\section{VARIADIC TEMPLATES}

Хочется сделать функцию с произвольным числом параметров.

Хочется делать:
\begin{cppcode}
int main() {
        int r = sum(1, 2, 3);
        r = sum(1, 2);
}
\end{cppcode}

Так же, как у крутых парней типа printf, scanf.

Механизм будет напоминать рекурсию.

\begin{cppcode}
template<typename T, typename ... Args> //'...' before Args means that the number of arguments may be any/
T sum(T n, Args ... rest) {
	return n + sum(rest ...);
}//macros PRETTY_FUNCTION gives us the name of the function with arguments and its return value

template<typename T>
T sum(T n) {
	return n;
}
\end{cppcode}


Как сделано в printf:

\begin{cppcode}
void printf(const char* fmt, ...) {
	va_list args;
	va_start(args, fmt);//макрос, настраивает args на смещение, равное размеру fmt(fmt здесь в том смысле, что это
						//последний аргумент перед '...')
	while(...) {
		int i = va_arg(args, int);//тут вторым параметром может быть любой примитивный тип в зависимости от %_
								  //args кастуется к нужному типу и на сколько нужно смещается по стеку
	}
	va_end(args);
}
\end{cppcode}

Написать таким механизмом правильно работающий код достаточно сложно, конечно, особенно с учетом того, 
что пользователь умеет радостно ошибаться, и это надо как-то отслеживать.
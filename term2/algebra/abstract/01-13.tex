\section{} % 01
$a$ левый делитель (делит слева) $b$: $b=ac$.
В коммутативных "--- просто делитель.
Всё транзитивно.
Если есть единица "--- рефлексивно.
Коммутативное с единицей "--- область целостности, есть нет делителей нуля ($\ne 0$).

\section{} % 02
В любом кольце.
Левый идеал: складываем/вычитаем и $a \in I \Ra xa \in I$.
Двухсторонний идеал: вычитаем и умножаем с любой стороны.
Можно пересекать, складывать (суммы элементов), умножать (произведения элементов, если один был левый, другой правый $\Ra$ двухсторонний).

\section{} % 03
Порождается $X$, если наименьший по включению, содержащий $X$.
Обозначается $I=(a_1, a_2, \dots, a_n)$.
Или если равен множеству линейных комбинаций.
Главный, если порождается одним элементом.
$a \mid b \Ra$ один идеал вложен в другой.
Область главных идеалов (ОГИ), если все главные.
Контрпример: $\R[x, y]$

\section{} % 04
Мультипликативная группа $R*$ из обратимых элементов ($ab=ba=1$).
Ассоциированность: есть $x \in R*$ такой, что $a=bx$.
Отношение эквивалентности.
$a \sim b \Ra a \mid b \land b | a$, в обратную в области целостности.

\section{} % 05
НОД, если делитель всех и делится на любой с таким же свойством.
В области целостности определён с точностью до ассоциированности.
Взаимная простота, если НОД "--- единица.
Если НОД, то разделим, получим взаимно простые (нужна единица).
Если идеал, порождённый $X$ равен $(d)$, то это НОД и есть линейное представление.
В ОГИ для любых есть НОД и линейное представление, а также свойство про идеалы.
В $\R[x, y]$ линейное представление не всегда есть.
В ОГИ если $a \mid bc$, то при взаимной простоте $\Ra a \mid c$ (выразили $c$ по линейности через $a$, $b$).

\section{} % 06
$R$ "--- область целостности.
Евклидова норма из $R \setminus \{0\}$ в $\N$, если можно делить с остатком (у остатка маленькая норма).
Евклидово кольцо, если можно задать норму.
Пример: целые ($|x|$), гауссовы ($a^2+b^2$, поделили $x$ на $y$ просто так, домножив на сопряжённое, округлили, сказали, что это частное, оценили).

\section{} % 07
Строим последовательность $r=ax+by$, делим с остатком $r_{i-1}$ на $r_i$, получаем следующую строчку.
Норма $r_i$ уменьшается, она натуральна, когда-нибудь получим ноль (не имеющий нормы).
Раскрутили обратно, получили линейное представление.

\section{} % 08
Разобрали $(0)$ отдельно.
Теперь взяли элемент $b$ с наименьшей нормой (кроме нуля).
Очевидно, что $(b) \subseteq I$.
Взяли элемент из $I \setminus (b)$, поделили с остатком на $b$, противоречие.

\section{} % 09
Классификация всего, кроме нуля и $R*$ (коммутативная группа).
Составной, если $a=bc$, где $b, c \notin R*$.
Неприводимый, если $a=bc \Ra b \in R* \lor c \in R*$.
Простой, если $a \mid bc \Ra a \mid b \lor a \mid c$.
Если область целостности, то всякий простой неприводим (если $p=bc$, то либо $p\sim b$, либо $p\sim c$).

\section{} % 10
В ОГИ любой неприводимый прост.
Если $p \mid ab$, то либо $p \mid a$, либо нет.
Тогда рассмотрим $(p,a)$, в ОГИ это $(d)$.
Значит $p=dv$, отсюда $d \in R*$ (если $v \in R*$, то $p \sim d$ и $p \mid a$).
То есть $(p,a)=(d)=(1)$, то есть $p$ и $a$ взаимно просты.
Значит, $p \mid ab \Ra p \mid b$.

\section{} % 11
Факториально, если разложили на мультимножество простых с точностью до порядка и домножение на элемент $R*$.
Не факториально $\Z[\sqrt{-5}]$: $6=2\cdot 3 = (1+\sqrt 5 i)(1 - \sqrt 5 i)$ (ввели норму $a+5b^2$, мультипликативна, нашли $R*$. нашли все элементы маленькой нормы).

\section{} % 12
Нётерово кольцо "--- нет бесконечной строгой цепочки идеалов.
Равносильно тому, что все идеалы конечнопорождены.
Если $R$ "--- ОГИ (и тогда Нётерово), то оно факториально.
Сначала покажем, что всегда существует неприводимый делитель.
Будем $a$ и раскладывать на составные, получим бесконечную цепочку, упс.
Теперь разложим: будем делить; если до бесконечности, то возьмём произведения на суффиксах и получим бесконечную цепочку, упс.
Единственность: пусть есть два разложения, одно короче другого, индукция по длине кратчайшего.
Возьмём простое из первого, найдём, кого из второго разложения оно делит, сократим.

\section{} % 13
Если $K$ "--- поле, то в $K[x]$, линейный неприводим.
Если $K$ алгебраически замкнуто, то по Безу других неприводимых нет.
$\C$ "--- такое (без доказательства).
Замечание: если степень 2 или 3, то неприводимость $\iff$ отсутствие корней (посмотрели на степени делителей).
Теперь к $\R$, покажем, что можно разложить на степени не больше 2 (если степень 2, то отрицательный дискриминант).
Разложим над $\C$, если $z$ "--- корень, то $\bar z$ тоже корень, причём той же кратности (это через производные), объединим их по парам.

\section{} % 14
$a \equiv b (\mod I) \iff (a - b) \in I$.
Рефлексивно, симметрично, транзитивно.
Можно ввести классы эквивалентности, независимо от представителей всё будет ок.

\section{} % 15
$R/I$ "--- факторкольцо, надо еще раз проверить корректность операций и всякие свойства колец.
Если было коммутативным/с единицей, то фактор тоже коммутативно/с единицей.

\section{} % 16
Идеал максимальный, если любой его содержащий равен либо ему, либо кольцу (максимальный по включению, может быть много).
$I$ "--- максимальный $\iff$ $R/I$ "--- поле.
$\Ra$: единица в факторкольце лежит и не равна нулю (потому что $I \subsetneq R$),
докажем существование обратного: взяли элемент $a$, взяли идеал $I+(a)$ (получили $R$ по максимальности),
в нём нашли единицу вида $1=x+ab$ (где $x \in I$, $b \in R$), это есть обратный.
$\La$: от противного, пусть есть $I \subsetneq J$ (тоже идеал), покажем $J=R$.
Взяли $a \in J \setminus I$, взяли обратный в поле $b$, получили $1 \equiv ab (\mod I)$.
Так как $J$ "--- идеал, то $ab \in J$, $c \in I \subset J$ $\Ra$ $ab+c \in J \Ra 1 \in J \Ra J=R$.
Еще теорема в ОГИ: $(a)$ "--- максимальный $\iff$ $a$ неприводим, док-во: разложили на неприводимые (они же в ОГИ простые).
Следствия: $\Z/m\Z$ иногда поле, $K[x]/f$ иногда поле.

\section{} % 17
Вычеты по модулю $x^2 + 1$ в $\R[x]$.

\section{} % 18
Взяли неприводимый в $F[x]$ многочлен $f$ степени $n$, взяли факторкольцо $K=F[x]/(f)$, тогда оно поле.
Рассмотрим естественный гомоморфизм из кольца в поле (по модулю идеала; надо что-то сказать про инъективность констант),
возьмём $[x]$, он является корнем $f$, так как $f([x])=[0]$.
$K$ "--- это поле, полученное присоединением корня $f$.
Возьмём другой многочлен $g$ (хотя бы степени 1) над $F[x]$, где $F$ "--- поле.
Поле $K$ есть поле разложения $f$, если в $K$ он раскладывается на линейные, а в любом подполе "--- нет.
Теорема: для любого $f$ существует такое поле.
Для начала найдём какое нибудь (не обязательно минимальное).
Просто делаем индукцию по суммарной степени нелинейный множителей, берём какой-нибудь множитель, присоединяем корень, расширяем поле.
Дальше можно просто пересечь все подполя результата, в которых разложим $f$, потому что все корни в поле мы знаем "--- их ровно $n$,
новым взяться неоткуда.

\section{} % 19
Пусть $R$ "--- область целостности, хотим построить поле $F \supset R$.
Введём множество дробей (знаменатель не ноль), отношение эквивалентности на них, возьмём фактор.
Покажем, что операции корректны, что это поле, что $\frac a 1$ изоморфно $R$.

\section{} % 20
Поле частных $F[x]$ ($F$ "--- поле) есть поле рациональных функций.
Покажем, что всякая дробь единственным образом записывается в виде $\frac f g$, старший коэффициент в $g$ единица, НОД тоже единица
(почему-то это было определением \TODO).
Дробь правильная, если числитель меньше знаменателя.
Примарная, если $g=q^k$, $q$ неприводим, $\deg f < \deg g$.
Простейшая, если $g=q^k$, $\deg f < \deg q$.
Теорема: единственно разложение в сумму простейших и многочлена (он не простейшая).
План: сначала разложим в многочлен и правильную (поделили с остатком), потом правильную в примарные, потом примарные в простейшие.
Лемма: правильную $\frac{f}{gh}$ можно разложить в правильные $\frac fg + \frac fh$ (при НОД=1).
Для этого (так как $F$ "--- ОГИ) представили НОД линейно, поделим переменные с остатком на знаменатели, покажем, что получили хорошие степени.
Правильную в примарные: разложили знаменатель, индукция по числу множителей.
Примарную в простейшие: индукция по степени $q^k$, на каждом шаге делим с остатком на $q$.
Единственность: индукция по суммарному количеству простейших, на шаге домножаем на общие знаменатели и вычитаем.

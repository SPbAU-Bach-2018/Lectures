\section{Инвариантные подпространства}
\begin{Def}

$V$ "--- векторное пространства над K, $\dim V = n < \infty$,

$U < V$ инвариантно относительно f, если  $\forall u \in U \colon f(u) \in U$

\end{Def}

\begin{exmp}

\begin{enumerate}
\item 
$\lambda$ "--- собственное число f, $U_1(\lambda)$ "--- f-инвариант

\item 
$\{0\}, V$ "--- также f-инвариантны. 
\end{enumerate}
\end{exmp} 

\begin{theorem}{}
Пусть $V = U \oplus W$ и U "--- $f$-инвариантное, тогда
$\exists$ базис $u_1, \cdots, u_k \in U$, 

$w_1, \cdots, w_{n - k} \in W$, в котором матрица $f$
имеет вид 
$\begin{pmatrix}
*&*\\
0&*\\
\end{pmatrix}$
(Первый квадрат $k \times k$).

Если же оба $U$ и $W$ инвариантны, то матрица $f$ в этом базисе имеет вид:
$
\begin{pmatrix}
*&0\\
0&*\\
\end{pmatrix}
$
Более того, это верно для любого базиса $V$, где первая часть векторов из $U$, а вторая "--- из $W$.
\end{theorem}
\begin{proof}
\begin{enumerate}
    \item
     $$[f]_{u_1, \cdots,u_k, w_1, \cdots w_{n - k}}$$
     $$f(u_i) \in U$$
     $$f(u_i) = \sum_{j = 1}^{k} \alpha_{ji}u_j + 0w_1 + \cdots + 0w_{n - k}$$
     \item Если и W инвариантно
     $$f(w_i) = 0u_1 + \cdots + 0u_n + \sum_{j = 1}^{n - k}B_{ji}w_j$$
\end{enumerate}
\end{proof}
\begin{conseq}
Если пространство V раскладывается в прямую сумму подпространств

$f$"--- инвариант подпространств, то выбрав в каждом из
подпространств по базису, получаем, что матрица $f$ в этом
базисе блочно-диагональная.
\end{conseq}
\begin{proof}
индукция по число инвариантных подпространств.(упражнение)
\end{proof}

$$f \in End(V)$$
$$f^{k} = f\circ f \circ \cdots \circ f$$
$$H \in K[t], H = a_0 + a_1t + \cdots + a_kt^{k}$$
$$H(f) = a_0 id + a_1f + \cdots + a_kf^k \in End(v)$$

\setauthor{Дмитрий Лапшин}

\begin{theorem}
	$V$ "--- конечномерное векторное пространство над $K$, $f \in \End(V)$, $H \in K[t]$.
	Тогда $\ker H(f)$ и $\Im H(f)$ есть $f$ "--- инвариантные подпространства $V$.
\end{theorem}

\begin{proof}
	\begin{gather*}
		H(t) = a_0 + a_1 t + \dots + d_n t^n \\
		H(f) = a_0 id + a_1 f + \dots + d_n f^n \\
		f^k = \underbrace{f \circ f \circ \dots \circ f}_{k}
	\end{gather*}
	\begin{lemma}
		$H_1, H_2 \in K[x]$
		\[ H_1(f)H_2(f) = H_2(f)H_1(f) \]
	\end{lemma}
	\begin{proof}
		\begin{enumerate}
		\item
			\[ f^k \circ f^e = f^e \circ f^k = f^{k+e} \]

		\item
			\begin{gather*}
				H(f) \circ f^l = \left( \sum_{i=0}^n a_k f^k \right) f^l = \sum_{i=0}^n a_k f^k f^l = \\
				= \sum_{i=0}^n a_k f^l f^k = f^l \sum_{i=0}^n a_k f^k = f^l \circ H(f)
			\end{gather*}

		\item
			\begin{gather*}
				H_1(f) H_2(f) = H_1(f) \left( \sum_{i=0}^m b_k f^k \right) = \sum_{i=0}^m b_k H_1(f) f^k = \\
				= \sum_{i=0}^m b_k f^k H_1(f) = H_2(f) H_1(f)
			\end{gather*}
		\end{enumerate}
	\end{proof}
	\begin{gather*}
		U = \Im H(f) = \left\{ v \in V \mid \exists w \in V\colon H(f)w = v\right\} \\
		f(U) \stackrel{\subset}{?} U
	\end{gather*}
	$u \in U$, $w, \in V$, $H(f)w = u$
	\begin{gather*}
		f(u) = f(H(f)(w)) = (f \circ H(f))(w) = (H(f) \circ f)(w) = H(f) (f(w)) \in \Im H(f) = U
	\end{gather*}
	Про $U$ доказали.

	$f \in \End (V)$, $H(f) \in \End(V)$, $V \xlongrightarrow{H(f)} V$.
	\begin{gather*}
		U = \ker H(f) = \{v \in V \mid H(f)(v) = 0\} 
	\end{gather*}
	$u \in U$
	\begin{gather*}
		H(f)(f(u)) = (H(f) \circ f)(u) = (f \circ H(f))(u) = f (\underbrace{H(f)(u)}_{u \in U = \ker H(f)}) = f(0) = 0
	\end{gather*}
\end{proof}

\begin{Rem}
	$f$ "--- обратим, $\Im f = V$, $\ker f = \{0\}$.

	Если $H(f)$ "--- обратимый оператор, то $\Im H(f) = V$, $\ker H(f) = \{0\}$.
\end{Rem}

\section{Теорема Гамильтона"---Кэли}

Вопрос: найти $H$ такое, чтобы $\ker H(f)$ было <<как можно больше>>.
\[ \ker H(f) = V \Lra H(f) = 0 \]
\begin{theorem}[Гамильтона"---Кэли]
	$\dim V = n < \infty$, $f \in \End(V)$.
	Тогда
	\[ \chi_f(f) = 0_{\text{в кольце операторов}} \]
\end{theorem}
\begin{proof}
	Зафиксируем базис $V$.
	$A = [f]$ "--- матрица $f$ в этом базисе.
	\[ H[f] = H(A) \]
	\[ \chi_f = \chi_A \]
	Нулевому оператору отвечает нулевая матрица.
	
	Перешли к равносильному утверждению в матрицах:
	\[ \chi_A (A) = 0 \]
	Докажем его:
	\begin{gather*}
		\chi_A = \det (A - tE) \\
		A - tE \in M(n, n, K[t]) \\
		B = (A - tE)^{\text{взаимная}} \in M(n, n, K[t]) \\
		(A - tE) B = B(A - tE) = (\det (A - tE)) E
	\end{gather*}
	При вычислении алгебраического дополниния считаем определитель размера $n - 1$ и каждый элемент "--- многочлен от $t$ степени не более 1.
	Таким образом, каждое алгебраическое дополение есть многочлен от $t$ степени не более $n - 1$.
	Разложим $B$ как многочлен по степеням $t$:
	\begin{gather*}
		B(t) \equiv B_0 + t B_1 + \dots + t^{n-1} B_{n-1} \\
		B_i \in M(n, n, K) \\
		(A - tE)(B_0 + t B_1 + \dots + t^{n-1} B_{n-1}) = \det (A - tE) E = (a_0 + a_1 t + \dots + a_n t^n) E \\
		\begin{alignedat}{3}
			AB_0 &      &= a_0 E \\
			AB_1 &- B_0 &= a_1 E \\
			&\vdots \\
			AB_k &- B_{k-1} &= a_k E \\
			&\vdots \\
			     &- B_{n-1} &= a_n E
		\end{alignedat}
	\end{gather*}
	Сложив эти строки с коэффицентами $E, A^1, \dots, A^n$, получим
	\begin{gather*}
		AB_0 + (A^2 B_1 - AB_0) + (A^3 B_2 - A^2B_1) + \dots + (A^{k+1} B_k - A^kB_{k-1}) + \dots + (0 - A^nB_{n-1}) = \\
		= a_0 E + a_1 A + \dots + a_n A^n = \chi_A(A) \\
		0 = \chi_A(A)
	\end{gather*}
\end{proof}

\begin{Rem}
	$f \in \End(V)$.
	Аннулятор:
	\[ \Ann(f) = \{H \in K[t] \mid H(f) = 0\} \]
	Это множество замкнуто по сумме, разности и умножение на произвольный многочлен, то есть идеал в кольце $K[t]$.
	$K[t]$ "--- область главных идеалов.
	Тогда
	\[ \Ann(f) = \left( \mu_f(t) \right) \]
	$\mu_f$ "--- минимальный аннулятор.

	Как мы знаем, $\chi_f \in \Ann(f)$, тогда $\mu_f \mid \chi_f$.
\end{Rem}

\begin{exmp}
	$f = \lambda id$.
	Тогда
	\begin{gather*}
		\chi_f = (\lambda - t)^n \\
		\mu_f = t - \lambda
	\end{gather*}
\end{exmp}

\section{Разложение в прямую сумму корневых}

\begin{theorem}
	$H_1, H_2 \in K[t]$, $f \in \End(V)$, $H_1$ и $H_2$ взаимно просты.
	Тогда 
	\[  \ker ((H_1H_2)(f)) = \ker H_1(f) \oplus \ker H_2(f) \]
\end{theorem}
\begin{proof}
	\begin{enumerate}
	\item
		$v \in \ker H_i(f)$, $H_i(f)(v) = 0$.
		\[ (H_1H_2(f))(v) = (H_1(f) \circ H_2(f))(v) = (H_2(f) \circ H_1(f))(v) \]
		Если $H_i(f)(v) = 0$, то и $(H_1H_2(f))(v) = 0$.
		Получили, что $\ker H_i(f) \subset \ker (H_1H_2)(f)$

	\item
		$v \in \ker H_1(f) \cap \ker H_2(f)$.
		\begin{gather*}
			1 = G_1 H_1 + G_2 H_2 \\
			id = G_1(f) \circ H_1(f) + G_2(f) \circ H_2(f) \\
			v = id(v) = G_1(f) \circ H_1(f)(v) + G_2(f) \circ H_2(f)(v) = G_1(f)(0) + G_2(f)(0) = 0
		\end{gather*}
		Таким образом, $\ker H_1(f) \cap \ker H_2(f) = \{0\}$ и сумма "--- прямая.

	\item
		$v \in \ker(H_1H_2)(f)$.
		\begin{gather*}
			1 = H_1 G_1 + H_2 G_2 \\
			id = H_1(f) \circ G_1(f) + H_2(f) \circ G_2(f) \\
			v = id(v) = \underbrace{(G_1(f) \circ H_1(f))(v)}_{v_2} + \underbrace{(G_2(f) \circ H_2(f))(v)}_{v_1} \\
			v = v_2 + v_1
		\end{gather*}
		Проверим, что $v_1 \in \ker H_1(f)$
		\[ H_1(f)(v_1) = (H_1H_2G_2)(f)(v) = (G_2(f) \circ (H_1H_2)(f))(v) = G_2(f)(\underbrace{(H_1H_2)(f)(v)}_{=0}) = 0 \]
		Аналогично $v_2 \in \ker H_2(f)$.
	\end{enumerate}
\end{proof}

\begin{conseq}
	$H(t) = H_i(t) \circ \dots \circ H_k(t)$, $H_i$ попарно взаимно просты.
	Тогда
	\[ \ker H(f) = \bigoplus_{i=1}^k \ker H_1(f) \]
\end{conseq}

$\dim V = n < \infty$, $K$ алгебраически замкнуто.
Тогда
\begin{gather*}
	V = \ker \mathbb{0} = \ker \chi_f(f) \\
	\chi_f(f) = \prod_{i=1}^k (t-\lambda_i)^{a_i} \quad \lambda_i \ne \lambda_j \\
	V = \bigoplus_{i=1}^k \ker (f - \lambda id)^{a_i}
\end{gather*}

\begin{Def}
	$k$-корневое пространство, отвечающее $\lambda$
	\[ U_k(\lambda) = \ker (f - \lambda id)^k \]
	Соотвественно, $U_1(\lambda)$ есть собственные вектора $f$.
\end{Def}

Как можно заметить,
\[ U_1(\lambda) \subset U_2(\lambda) \subset \dots \]
Причём оно стабилизируется.
\begin{Def}
	Корневое пространство, отвечающее $\lambda$
	\[ U(\lambda) = \bigcup_{k=1}^\infty \ker (f - \lambda id)^k \]
\end{Def}

\begin{Def}
	Корневой вектор "--- вектор корневого пространства.
	Если он при этом лежит в $U_{k}(\lambda)$, но не лежит в $U_{k-1}(\lambda)$, его называют корневым вектором высоты $k$.
\end{Def}

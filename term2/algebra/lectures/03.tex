\setauthor{Ольга Черникова}
$$p|ab \Ra p|a \vee p|b$$
$R$ "--- область целостности

$\{0\}\cup R^{*} \cup \{$ неприводимые $\} \cup \{$ составные $\}$

\begin{theorem}{}
\begin{enumerate}
    \item $R$ ~--- область целостности

    Всякий простой элемент неприводим.
    \item $R$ ~--- ОГИ, всякий неприводимый простой.
\end{enumerate}
\end{theorem}

\begin{proof}
\begin{enumerate}
    \item $p\in R, p$ "--- простой.
    $$p = ab \Ra a|p, b|p$$
    $$p|ab \xLongrightarrow[\text{p "--- простой}]{} p|a \cup p|b$$
    
    $$
    \begin{matrix}
    p|a \wedge a|p \Ra a \sim p, b \in R^{*}\\
    p|b \wedge b|p \Ra b \sim p, b \in R^{*}
    \end{matrix}
     \Ra p \text{"--- неприводим}\\
    $$
    
    \item $R$ "--- ОГИ

    $p$ "--- неприводимый.
    $$p|ab$$
    $p|a (ok)$

    $p \nmid a:$
    $$(p, a) = (d)$$
    $$p = dv$$
    $$d \nsim p (\text{Если $d \sim p$, то $p|d|a !?$})$$
    
    $$
    \begin{matrix}
    p = dv\\
    p \nsim d\\
    p \text{ "--- неприводим}
    \end{matrix}
    \Ra d \in R^{*}
    $$
    
    $$(p, a) = (d) = (1)$$
    $p, a$ "--- взаимно просты.
    
    $$
    \begin{matrix}
    p|ab\\
    p \text{ взаимно прост с } a
    \end{matrix}
    \Ra p|b
    $$
    
    $$\Ra p \text{"--- простой}$$
\end{enumerate}
\end{proof}

prume "--- простой.

irreducible "--- неприводимый

composite "--- составной

\section{Факториальные кольца}
\begin{Def}
В кольце $R$ выполнена теорема об однозначном разложение на множители.
\begin{enumerate}
\item $$a \in R^{*} \setminus \{0\}$$
$$a = \epsilon \prod_{i = 1}^n p_i $$
$$\epsilon \in R^{*}, n \ge 0, p_i \text{"--- неприводимы}$$
Или эквивалентное условие
$$a \in R, a\notin \{0\} \cup R^{*}$$
$$a = \prod_{i = 1}^{n} p_i, n \ge 1$$

\item $$0 \ne a = \epsilon, \prod_{i = 1}^{n}p_i = \eta \prod_{j = 1}^{n} q_{j}$$
$$\epsilon, \eta \in R^{*}, p_i,q_i \text{ "--- неприводимы} \Ra n = m$$
$$\text{и } \exists \sigma \in S_n \colon \forall i = 1 \ldots n \colon p_i \sim q_{\sigma(i)}$$
\end{enumerate}
Такое кольцо называется факториальным (unique factorization domain UFD)
\end{Def}

\textbf{Пример не факториального кольца}
$$R = \Z[\sqrt{-5}] = \{a + b\sqrt{-5}\mid a, b \in \Z\}$$

\textbf{Упражнение}

$$6 = 2 \cdot 3 = (1 + \sqrt{5}i)(1 - \sqrt{5}i)$$
\begin{proof}
$2, 3, 1 + \sqrt{5}i, 1 - \sqrt{5}i$ "--- неприводимые, попарно не ассоциированные. 
$$N \colon \Z[\sqrt{-5}] \to \N \cup \{0\}$$
$$N(a + b \sqrt{5}i) = |a + b\sqrt{5}i|^2 = a^2 + 5b^2$$
$$N(z_1z_2) = N(z_1)N(z_2)$$
$$z_{1} \in R^{*}$$
$$z_{1}z_{2} = 1$$
$$N(z_1)N(z_2) = N(1) = 1$$
$$x_1 = a + b\sqrt{5}i$$
$$a^2 + 5b^2 \in \Z$$
$$a^2 + 5b^2 = 1, b = 0, a = \pm 1$$
$2, 3, 1 \pm \sqrt5i$ "--- неприводимые, попарно не ассоциированы.
$$\Z[\sqrt5]^* = \{\pm1\} \Ra 2, 3, 1 \pm \sqrt{-5} \text{ "--- попарно не ассоциированы}$$

$$z_1z_2 = 1$$
$$N(z_1) = 1, z_1 = \pm 1$$
$$N(z_2) = 1, z_2 = \pm 1$$


$$2 = z_1z_2$$
$$4 = N(2) = N(z_1)N(z_2)$$
$$N(z_1) = N(z_2) = 2 \text{"--- невозможно}$$
$$z_1 = a + b\sqrt{-5}$$
$$a^2 + 5b^2 = 2 \text{ нет решений в $\Z$}$$

$$N(1 + \sqrt{5}i) = 6$$
$$1 \pm \sqrt{5}i = z_1z_2$$

6 1

1 6

2 3

3 2

Нет элементов нормы 2 $\Ra$ нет решений.

$$9 = N(3) = N(z_1)N(z_2)$$

9 1

1 9

3 3

$a^2 + 5b^2 = 3$ "--- нет решений в целых $\Ra$ нет элементов нормы 3.

\end{proof}

\textbf{Замечание:} В этом же кольце не для всех пар определен gcd.

\textbf{Докажем:} ОГИ "--- факториальны.

$K[x_1, \cdots, x_n]$ "--- факториально(но не ОГИ $n \ge 2$)

$\Z[x_1, \cdots, x_n]$ "--- факториально(но не ОГИ, если $n \ge 1$)

R "--- факториальная о.ц. $\Ra R[x]$ "--- факториальна.


\section{Нётеровы кольца и условия обрыва возрастающих цепей идеалов}

$R$ "--- коммутативное кольцо

Выполнено условие обрыва возрастающей цепей идеалов, если $I_1 \subset I_2 \cdots (I_j)$ "--- идеалы.

$\exists n \colon I_n = I_{n + 1} = \cdots$

(нет бесконечных строго возрастающих цепочек)

$I_1 \subsetneq I_2 \subsetneq I_3 \cdots$ "--- такие цепочки невозможны. 

\begin{Def}
Кольцо, в котором выполняется условие обрыва возрастающей цепочки идеалов, называется Нётеровым.
\end{Def}

\begin{exmp}
$\Z$ "--- нётерово.

$(m)$

$$(m_1) \subset (m_2) \subset \cdots $$

$$m_1 \vdots m_2 \vdots m_3 \cdots m_i$$
\end{exmp}

\begin{theorem}{}
Следующие условия равносильны
\begin{enumerate}
\item $R$ "--- нётерово.
\item Всякий идеал в $R$ конечно порожден.
\end{enumerate}
\end{theorem}

\begin{conseq}
R "--- ОГИ $\Ra$ R нётерово $\Ra$ в R выполнено условие обрыва возрастающих цепей идеалов. 
\end{conseq}

\begin{proof}
\textbf{$1) \Ra 2):$}

$I$ "--- идеал в R

$$a_1 \in I$$

$$I = (a_1)\colon ok$$
$$I \ne (a_1) \Ra \exists a_2 \in I \setminus (a_1)$$
$$(a_1, a_2)$$
$$I = (a_1, a_2)\colon ok$$
$$I \ne (a_1, a_2) \Ra \exists a_3 \in I \setminus (a_1, a_2)$$
$$(a_1) \subsetneq (a_1, a_2) \subsetneq (a_1, a_2, a_3) \cdots$$

так как $R$ Нётерово, то цепочка обрывается. 

$$\exists n \colon I = (a_1, \cdots a_n)$$

\textbf{$2) \Ra 1):$}

$$I_{1} \subset I_2 \subset \cdots \subset I_{k} \subset \cdots$$
$$I = \cup_{j = 1}^{\infty} I_{j} \text{"--- идеал в} R$$
$$a, b \in I$$
$$\exists i, j a \in I_{i}, b \in I_{j}$$
$$m = max(i, j)$$
$$I_j, I_i \subset I_{m} \Ra a, b \in I_{m}$$
$$a\pm b \in I_m \Ra a \pm b \in \cup I_{j} = I$$
$$a \in I, r \in R$$
$$\exists j \colon a \in I_j, r \in R, ra \in I_{j} \Ra ra \in \cup_{j}I_j = I$$

$I$ "--- идеал. 

$I$ "--- конечно порожден $\exists a_1, \cdots, a_n$

$$I = (a_1, \cdots, a_n)$$

$$a_1 \cdots a_n \in \cup_{j} I_j$$
$$\exists j_k\colon a_k \in I_{j_k} k = 1 \cdots n$$
$$m =max(j_1, \cdots, j_k)$$
$$I_{j_{k}} \subset I_m$$
$$a_1, \cdots, a_k \subset I_{m}$$
$$(a_1, \cdots, a_k) \subset I_m \subset I_{m + 1} \subset \cdots$$
$$\subset \cup_{j} I_{j} = I = (a_1, \cdots, a_n)$$
$$\Ra I_m = I_{m + 1} = \cdots$$
\end{proof}
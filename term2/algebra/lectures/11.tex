\section[Коор-ты вектора. Матрица перехода. Замена коор-т]{Координаты вектора. Матрица перехода. Замена координат.}

\begin{Def}
	$K$ "---  поле, $V$ "--- вектороное пространство над $K$.
	$\{V_{\alpha}\}_{\alpha \in I}$ "--- базис $V$.
	\[ v \in V \Ra v = \sum_{\alpha \in I}a_{\alpha}v_{\alpha} \]
	почти все $a_{\alpha} = 0$ и такое разложение единственно.
	$\{a_{\alpha}\}_{\alpha \in I}$ "--- набор координат вектора $v$ в базисе $\{v_{\alpha}\}$.
\end{Def}

\begin{Def}
	$\dim V = n < +\infty$, $v_1, \dots, v_n$ "--- базис $V$.
	$
		\begin{pmatrix}
			a_1\\
			\vdots\\
		a_n\\
		\end{pmatrix}
	$ "--- столбец координат $v$ в базисе $v_1, \dots, v_n$.
\end{Def}

\begin{exmp}
	$\C$ над $\R$. Базис "--- $\{1, i\}$
	\[
		z \mapsto
		\begin{pmatrix}
			\Re z\\
			\Im z\\
		\end{pmatrix}
	\]
\end{exmp}

\begin{Def}
	$\{v_1, \dots, v_n\}$, $\{v_1', \dots, v_n'\}$ "--- базисы $V$.
	\begin{gather*}
		v'_j = \sum_{i = 1}^{n}a_{ij}v_i, a_{ij}\in K \\
		A = (a_{ij})_{i, j = 1, \dots n}
	\end{gather*}
	$j$-й столбец $A$ "--- столбец координат $v_j'$ в базисе $v_i$.

	$A$ "--- матрица перехода от $v_1, \dots, v_n$ к базису $v_1', \dots, v_n'$
	\[ A = [A]_{\{v_i\} \to \{v_i'\}} \]
\end{Def}

\begin{exmp}
	$\R^2$, рассмотрим стандартный базис и базис повернутый на угол $\phi$.
	\begin{gather*}
		\begin{aligned}
			v_1' &\to
				\begin{pmatrix}
					\cos \phi\\
					\sin \phi\\
				\end{pmatrix} \\
			v_2' &\to
				\begin{pmatrix}
					-\sin \phi\\
					\cos \phi\\
				\end{pmatrix}
		\end{aligned} \\
		A =
		\begin{pmatrix}
			\cos \phi & -\sin \phi\\
			\sin \phi & \cos \phi\\
		\end{pmatrix}
	\end{gather*}
\end{exmp}

\begin{theorem}
	$V$ "--- векторное пространство над $K$, $\dim V = n$,\\
	$\{v_1, \dots, v_n\}$, $\{v_1', \dots, v_n'\}$, $\{v_1'', \dots, v_n''\}$  "--- базисы $V$.
	$A$ "--- матрица перехода $\{v_i\} \to \{v_i'\}$,
	$B$ "--- матрица перехода $\{v'_i\} \to \{v_i''\}$,
	$C$ "--- матрица перехода от $\{v_i\} \to \{v_i''\}$.
	Тогда $C = A \cdot B$.
\end{theorem}
\begin{proof}
	\begin{gather*}
		\begin{alignedat}{3}
			v_j'' &= \sum_{i = 1}^{n}c_{ij}v_i &\\
			v_j'' &= \sum_{r = 1}^{n}b_{rj}v_r' &= \sum_{r = 1}^{n}b_{rj}\left(\sum_{i = 1}^{n}a_{ir}v_i\right) = \\
			      &= \sum_{r = 1}^{n}\sum_{i = 1}^{n}(b_{rj}a_{ir})v_i &= \sum_{r = 1}^{n}\sum_{i = 1}^{n}a_{ir}b_{rj}v_i = \\
			      &= \sum_{i = 1}^{n}\underbrace{\left(\sum_{r = 1}^{n}a_{ir} b_{rj}\right)}_{c_{ij}} v_i
			        &\Ra c_{ij} = \sum_{r = 1}^{n}a_{ir}b_{rj}
		\end{alignedat} \\
		C = AB
	\end{gather*}
\end{proof}

\begin{conseq}
	Матрица перехода от базиса к базису является обратимой и обратная к ней
	"--- это матрица перехода от $\{v_i'\}$ к $\{v_i\}$.
\end{conseq}
\begin{proof}
	\[ \{v_i\} \underbrace{\to}_{A} \{v_i'\} \underbrace {\to}_{B} \{v_i\} \]
	$A \cdot B$ "--- матрица перехода от $\{v_i\}$ к $\{v_i\}$.
	\[ v_i = 0\cdot v_1 + \dots + 1\cdot v_i + \dots + 0\cdot v_n \]
	т.е. единичная матрица.
	$AB = E \land BA = E \Ra B = A^{-1}$
\end{proof}

\begin{exmp}
	$\R^2$, стандартный базис и базис повернутый на $\phi$.
	Матрица перехода от $\{v_1', v_2'\}$ к $\{v_1, v_2\}$:
	\[
		\begin{pmatrix}
			\cos \phi & -\sin \phi\\
			\sin \phi & \cos \phi\\
		\end{pmatrix}^{-1}
		=
		\begin{pmatrix}
			\cos \phi & \sin \phi\\
			-\sin \phi & \cos \phi\\
		\end{pmatrix}
	\]
	Получили поворот на $-\phi$.
\end{exmp}

\begin{theorem}
	$V$ "--- векторное пространства над $K$.
	$v_1, \dots, v_n$, $v_1', \dots, v_n'$ "--- базисы $V$. \\
	$
		\begin{pmatrix}
			b_1\\
			\vdots\\
			b_n\\
		\end{pmatrix}
	$ "--- координаты $v$ в базисе $\{v_1, \cdots, v_n\}$,
	$
		\begin{pmatrix}
			b_1'\\
			\vdots\\
			b_n'\\
		\end{pmatrix}
	$ "--- координаты $v$ в базисе $\{v_1', \cdots, v_n'\}$.
	Тогда
	\[
		\begin{pmatrix}
			b_1\\
			\vdots\\
			b_n\\
		\end{pmatrix}
		= A \cdot
		\begin{pmatrix}
			b_1'\\
			\vdots\\
			b_n'\\
		\end{pmatrix}
	\]
\end{theorem}

\begin{proof}
	\begin{gather*}
		v = \sum_{i = 1}^{n}b_i v_i \\
		v = \sum_{j = 1}^{n}b_{j}'v_{j}' = \sum_{j = 1}^{n}b_j'\left(\sum_{i = 1}^{n}a_{ij}v_i\right)
			=\sum_{i = 1}^{n}\underbrace{\left(\sum_{j = 1}^{n}a_{ij}b_j'\right)}_{b_i}v_i \\
		b_i = \sum_{j = 1}^{n}a_{ij}b_{j}'
	\end{gather*}
\end{proof}

\section{Сумма и пересечение подпространств}

\begin{theorem}
	$V$ "--- векторное пространство над $K$,
	$U, W \subset V$.
	Тогда
	\begin{enumerate}
	\item
		$U \cap W \subset V$ (упражнение).

	\item
		\begin{gather*}
			U + W = \{u + w \mid u \in U, w \in W\} \\
			U + W \subset V
		\end{gather*}
	\end{enumerate}
	Объединение подпространством быть не обязано.
\end{theorem}

\begin{proof}
	Достаточно проверить замкнутость относительно операций.
	\begin{gather*}
		(\underbrace{u}_{\in U} + \underbrace{w}_{\in W}) + (\underbrace{u'}_{\in U} + \underbrace{w'}_{\in W})
			= (\underbrace{u+u'}_{\in W}) + (\underbrace{w+w'}_{\in U}) \in U + W \\
		c(\underbrace{u}_{\in U} + \underbrace{w}_{\in W}) = \underbrace{cu}_{\in U} + \underbrace{cw}_{\in W} \in U + W \\
		U + W \ne \emptyset \\
		0_v = 0_{u} + 0_{w} \in U + W
	\end{gather*}
\end{proof}

\begin{theorem}
	Пусть в обозначениях выше $\dim U = n < \infty$, $\dim W = m < \infty$
	Тогда $U + W$ и $U \cap W$ конечномерны и
	\[ \dim(U + W) + \dim(U \cap W) = \dim U + \dim W \]
\end{theorem}
\begin{proof}
	\[ U \cap W \subset U \quad U \cap W \subset W \]
	Так как $\dim U < \infty$, то и $\dim (U \cap W) < \infty$.

	$v_1, \dots, v_r$ "--- базис  $U \cap W$.
	$\{v_1, \dots, v_r\} \subset U$, дополним до базиса $U$:
	\[ \{v_1, \dots, v_r, u_{r + 1}, \dots, u_{n}\} \quad n = \dim U \]
	Аналогично дополним до базиса $W$:
	\[ \{v_1, \dots, v_r, w_{r + 1}, \dots, w_{m}\} \quad m = \dim W \]
	Утверждаем, что
	\[ v_1, \dots, v_r, u_{r + 1}, \dots, u_n, w_{r + 1}, \dots, w_{m} \]
	"--- базис $U + W$.
	\begin{gather*}
		v \in U + W \\
		v = u + w, u \in U, w \in W \\
		u = \sum_{i = 1}^{r}a_iv_i + \sum_{i = r + 1}^{n}b_iu_i \\
		w = \sum_{i = 1}^{r}c_iv_i + \sum_{i = r + 1}^{m}d_iw_i \\
		v = \sum_{i = 1}^{r}(a_i + c_i)v_i + \sum_{i = r + 1}^{n}b_iu_i + \sum_{i = r + 1}^{m}d_iw_i
	\end{gather*}
	Таким образом, это "--- система образующих $U + W$.
	Проверим линейную независимость.
	\begin{gather*}
		\sum_{i = 1}^{r}a_iv_i + \sum_{i = r + 1}^{n}b_iu_i + \sum_{i = r + 1}^{m}c_iw_i = 0 \\
		v = \underbrace{\sum_{i = 1}^{r}a_iv_i + \sum_{i = r + 1}^{n}b_iu_i}_{\in U} = \underbrace{-\sum_{i = r + 1}^{m}c_iw_i}_{\in W} \\
		v = U \cap W \Ra v = \sum_{i = 1}^{r}d_iv_i = \sum_{i = 1}^{r}d_iv_i + \sum_{i = r + 1}^{n}0u_i
	\end{gather*}
	$\{v_i, u_i\}$ "--- базис $U$, следовательно разложение должно быть единственным, значит $b_i = 0$.
	\[ \sum_{i = 1}^{n}a_iv_i + \sum_{i = r + 1}^{m}c_iw_i = 0 \]
	$\{v_i, w_i\}$ "--- базис $W$, значит $a_i = 0$ и $c_i = 0$.

	Таким образом, $v_1, \dots, v_r, u_{r + 1}, \dots, u_n, w_{r + 1}, \dots, w_{m}$ "--- линейно независимы,
	а значит они "--- базис $U + W$.
	\[ \dim(U + W) + \dim(U \cap W) = r + (n - r) + (m  - r) + r = n + m = \dim(U) + \dim(W) \]
\end{proof}

\section{Координаты вектора. Матрица перехода. Замена координат.}

\begin{Def}
$K$ "---  поле, $V$ "--- вектороное пространство над $K$.
$\{V_{\alpha}\}_{\alpha \in I}$ "--- базис $V$.

$v \in V \Ra v = \sum_{\alpha \in I}a_{\alpha}v_{\alpha}$, почти все $a_{\alpha} = 0$
 и такое разложение единственно.

$\{a_{\alpha}\}_{\alpha \in I}$ "--- набор координат вектора $v$ в базисе $\{v_{\alpha}\}$.
\end{Def}

\begin{Def}
$\dim V = n < +\infty$

$v_1, \ldots, v_n$ "--- базис V.

$v \to 
\begin{pmatrix}
a_1\\
\vdots\\
a_n\\
\end{pmatrix}$ "--- столбец координат v в базисе $v_1, \cdots, v_n$.
\end{Def}

\begin{exmp}
$\C$ над $\R$. Базис "--- $\{1, i\}$

$z \to 
\begin{pmatrix}
Re z\\
Im z\\
\end{pmatrix}$ 
\end{exmp}

\begin{Def}
$\{v_1, \cdots, v_n\}$,$\{v_1', \cdots, v_n'\}$ "--- базисы V.

$$v'_j = \sum_{i = 1}^{n}a_{ij}v_i, a_{ij}\in K$$
$$A = (a_{ij})_{i, j = 1, \cdots n}$$

j-ый столбец $A$ "--- столбец координаты $v_j'$.

$A$ "--- матрица перехода от $v_1, \cdots, v_n$ к базису $v_1', \cdots, v_n'$

$A = [A]_{\{v_i\} \to \{v_i'\}}$                                          
\end{Def}

\begin{exmp}
$\R^2$, рассмотрим стандартный базис и базис повернутый на угол $\phi$.

$$v_1' \to
\begin{pmatrix}
\cos \phi\\
\sin \phi\\
\end{pmatrix}$$
$$v_2' \to
\begin{pmatrix}
-\sin \phi\\
\cos \phi\\
\end{pmatrix}$$
$$
A =
\begin{pmatrix}
\cos \phi & -\sin \phi\\
\sin \phi & \cos \phi\\
\end{pmatrix} 
$$
\end{exmp}

\begin{theorem}
$V$ "--- векторное пространство над $K$, $\dim V = n$

$\{v_1, \cdots, v_n\}, {v_1', \cdots, v_n'}, {v_1'', \cdots, v_n''}$  "---
базисы $V$

$A$ "--- матрица перехода от $\{v_i\} \to \{v_i'\}$

$B$ "--- матрица перехода от $\{v'_i\} \to \{v_i''\}$

$C$ "--- матрица перехода от $\{v_i\} \to \{v_i''\}$

Тогда $C = A \cdot B$ 
\end{theorem}

\begin{proof}
$$v_j'' = \sum_{i = 1}^{n}c_{ij}v_i$$
$$v_j'' = \sum_{r = 1}^{n}b_{rj}v_r' = \sum_{r = 1}^{n}b_{rj}\left(\sum_{i = 1}^{n}a_{ir}v_i\right) =$$
$$=\sum_{r = 1}^{n}\sum_{i = 1}^{n}(b_{rj}a_{ir})v_i = \sum_{r = 1}^{n}\sum_{i = 1}^{n}a_{ir}b_{ri}v_i =$$
$$= \sum_{i = 1}^{n}\underbrace{\left(\sum_{r = 1}^{n}a_{ir} b_{rj}\right)}_{c_{ij}} v_i \Ra c_{ij} = \sum_{r = 1}^{n}a_{ir}b_{rj} $$

$$C = AB$$
\end{proof}

\begin{conseq}
Матрица перехода от базиса к базису 
является обратимой и обратная к ней "--- это матрица перехода от 
$\{v_i'\}$ к $\{v_i\}$.
\end{conseq}
\begin{proof}
$$\{v_i\} \underbrace{\to}_{A} \{v_i'\} \underbrace {\to}_{B} \{v_i\}$$

$A \cdot B$ "--- матрица перехода от $\{v_i\}$ к $\{v_i\}$.
$$v_i = 0\cdot v_1 + \cdots + 1\cdot v_i + \cdots + 0\cdot v_n$$

т.е. единичная матрица. 

$AB = E$, $BA = E \Ra B = A^{-1}$
\end{proof}

\begin{exmp}
$\R^2$, стандартный базис и базис повернутый на $\phi$.

Матрица перехода от $\{v_1', v_2'\}$ к $\{v_1, v_2\}$

$$ 
\begin{pmatrix}
\cos \phi & -\sin \phi\\
\sin \phi & \cos \phi\\
\end{pmatrix}^{-1}
=
\begin{pmatrix}
\cos \phi & \sin \phi\\
-\sin \phi & \cos \phi\\
\end{pmatrix}
$$

Поворот на $-\phi$.
\end{exmp}

\begin{theorem}
V "--- векторное пространства над K.

$v_1, \cdots, v_n$, $v_1', \cdots, v_n'$ "--- базисы V

$\begin{pmatrix}
b_1\\
\vdots\\
b_n\\
\end{pmatrix}$ "--- координаты $v$ в базисе $\{v_1, \cdots, v_n\}$ 


$\begin{pmatrix}
b_1'\\
\vdots\\
b_n'\\
\end{pmatrix}$ "--- координаты $v$ в базисе $\{v_1', \cdots, v_n'\}$

Тогда $\begin{pmatrix}
b_1\\
\vdots\\
b_n\\
\end{pmatrix} = A \cdot 
\begin{pmatrix}
b_1'\\
\vdots\\
b_n'\\
\end{pmatrix}$
\end{theorem}

\begin{proof}
$$v = \sum_{i = 1}^{n}b_i v_i$$
$$v = \sum_{j = 1}^{n}b_{j}'v_{j}' = \sum_{j = 1}^{n}b_j'(\sum_{i = 1}^{n}a_{ij}v_i) = $$
$$= \sum_{i = 1}^{n}\underbrace{(\sum_{j = 1}^{n}a_{ij}b_j')}_{b_i}v_i $$

$$b_i = \sum_{j = 1}^{n}a_{ij}b_{j}'$$
\end{proof}

\section{Сумма и пересечение подпространств}
\begin{theorem}
$V$ "--- векторное пространство над $K$.
$U, W \subset V$
$U \cap W \subset V$(упр).

Объединение подпространством быть не обязано.

$$U + W = \{u + w| u \in U, w \in W\}$$
$$U + W \subset V$$
\end{theorem}

\begin{proof}
Достаточно проверить замкнутость относительно операций. 

\begin{tabular}{cccccc}
$(u+$&$w)+$&$(u'+$&$w')$&$=(u + u')$&$+(w + w')$\\
$\in u$&$\in w$&$\in u$&$\in w$ &$\in u$&$\in w$\\
\end{tabular}


\begin{tabular}{cccc}
$c(u+$&$w)$&$=cu$&$+cw$\\
$\in u$&$\in w$&$\in u$&$\in w$\\
\end{tabular}

$U + W \ne 0$

$0_v = 0_{u} + 0_{w} \in U + W$
\end{proof}

\begin{theorem}
Пусть в обозначениях выше
$\dim U = n < \infty, \dim W = m < \infty$

Тогда $U + W$, $U \cap W$ конечномерны и 
$\dim(U + W) + \dim(U \cap W) = \dim U + \dim W$.
\end{theorem}
\begin{proof}
$$U \cap W \subset U$$
$$U \cap W \subset W$$
Так как $\dim U < \infty$, то  и $\dim (U \cap W) < \infty$

$v_1, \cdots, v_r$ "--- базис  $U \cap W$

$\{v_1, \cdots, v_r\} \subset U$, дополним до базиса $U$.

$\{v_1, \cdots, v_r, u_{r + 1}, \cdots, u_{n}\}$ "--- базис U, $n = \dim U$.

Аналогично дополним до базиса W.
$\{v_1, \cdots, v_r, w_{r + 1}, \cdots, w_{m}\}$ "--- базис W, $m = \dim W$.

Утверждаем, что $v_1, \cdots, v_r, u_{r + 1}, \cdots, u_n, w_{r + 1}, \cdots, w_{m}$ "--- базис $U + W$.

$$v \in U + W$$
$$v = u + w, u \in U, w \in W$$
$$u = \sum_{i = 1}^{r}a_iv_i + \sum_{i = r + 1}^{n}b_iu_i$$
$$w = \sum_{i = 1}^{r}c_iv_i + \sum_{i = r + 1}^{m}d_iw_i$$
$$v = \sum_{i = 1}^{r}(a_i + c_i)v_i + \sum_{i = r + 1}^{n}b_iu_i + \sum_{i = r + 1}^{m}d_iw_i \Ra$$
$\Ra$ это система образующих $U + W$.

Проверим линейную независимость. 

$$\sum_{i = 1}^{r}a_iv_i + \sum_{i = r + 1}^{n}b_iu_i + \sum_{i = r + 1}^{m}c_iw_i = 0$$
$$v = \underbrace{\sum_{i = 1}^{r}a_iv_i + \sum_{i = r + 1}^{n}b_iu_i}_{\in U} = \underbrace{-\sum_{i = r + 1}^{m}c_iw_i}_{\in W}$$
$$v = U \cap W \Ra v = \sum_{i = 1}^{r}d_iv_i = \sum_{i = 1}^{r}d_iv_i + \sum_{i = r + 1}^{n}0u_i$$
$\{v_i, u_i\}$ "--- базис U $\Ra$ разложение должно быть единственным $\Ra$ $b_i = 0$
$\Ra \sum_{i = 1}^{n}a_iv_i + \sum_{i = r + 1}^{m}c_iw_i = 0$

$\{v_i, w_i\}$ "--- базис W $\Ra a_i = 0, c_i = 0$
$a_i = 0, c_i = 0$
$\Ra v_1, \cdots, v_r, u_{r + 1}, \cdots, u_n, w_{r + 1}, \cdots, w_{m}$ "--- линейно независимы $\Ra$ базис $U + W$

$$\dim(U + W) + \dim(U \cap W) = r + (n - r) + (m  - r) + r = n + m = \dim(U) + \dim(W)$$
\end{proof}    
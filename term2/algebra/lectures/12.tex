\section{Линейные отображения}
\begin{Def}
$U, V$ "--- векторное пространство над $K$.

$f\colon U \to V$ называется линейным отображением, если $\forall u_1, u_2 \in U, c_1, c_2 \in K$

$f(c_1u_1 + c_2u_2) = c_1f(u_1) + c_2f(u_2)$ "--- линейность.
\end{Def}
\begin{Rem}
$\forall u_1, u_2 \in U$

$f(u_1 + u_2) = f(u_1) + f(u_2)$ "--- адитивность. 

$\forall c \in K \forall u \in U, f(cU) = cf(U)$ "--- однородность степени 1.
\end{Rem}

\begin{Def}
$f \colon U \to V, f$ "--- линейное отображение. 

$f(U) = Im(F)$ "--- образ f.

$f^{-1}(0_v) = \{u \in U\colon f(u) = 0_v\} = ker(f)$ "--- ядро f.
\end{Def}

\textbf{Упражнение:} 
$Im(f) \subset V$

$ker(f) \subset U$

\begin{conseq}{Свойства линейного отображения}
$$f(0_u) = 0_v$$

$$f(0_k u) = 0_kf(u) = 0_v$$

Ядро не пусто (всегда содержит ноль).
\end{conseq}

\begin{lemma}
$f\colon u \to v, f$ "--- линейное отображение.

$f$ "--- инъективно $\Lra ker(f) = \{0_u\}$ 
\end{lemma}

\begin{proof}

$\Ra$

$f$ "--- инъективно: $\forall x \in V |f^{-1}(V)| \le 1 \Ra 
|f^{-1}(0)| \le 1$, но $0 \subset f^{-1}(0) \Ra ker(f) = \{0\}$

$\La$
$$u_1, u_2 \in U$$
$$f(u_1) = f(u_2)$$
$$f(u_1 - u_2) = f(u_1) - f(u_2) = 0 $$
$$u_1 - u_2 \in ker(f) = \{0\} \Ra u_1 = u_2$$

В силу произвольности $u_1$ и $u_2$, f "--- инъективно. 
\end{proof}

\begin{Def}
U, V "--- векторные пространства над K.
$f \colon U \to V$, f "--- изоморфизм, если
\begin{enumerate}
\item f "--- линейное отображение.
\item f "--- биекция. 

Говорим, что U и V изоморфны $U \cong V$.
\end{enumerate}
\end{Def}

\begin{Rem}
Линейное отображение $f:U \to V$ "--- изоморфизм, если 
\begin{enumerate}
\item $ker(f) = \{0\}$
\item $Im(f) = f(U) = V$
\end{enumerate}
\end{Rem}

\begin{Def}
$U, V$ "--- векторное пространство.

$Hom(U, V), \mathcal{L}(U, V)$ "--- пространство линейных отображений из U в V.
\end{Def}

\begin{theorem}
$\mathcal{L}(U, V)$ "--- векторное пространство над K/

$f, g \in \mathcal{L}(U, V)$

$u \in U, (f + g)(u) = f(u) + g(u)$

$(\alpha f)(u) = \alpha f(u)$

$f + g, \alpha f$ "--- линейные отображения.
\end{theorem}

\begin{proof}
$$(f + g)(\beta_{1}u_{1} + \beta_2u_{2}) = f(\beta_1 u_1 + \beta_{2}u_{2})+
g(\beta_1u_1 + \beta_2u_2) =$$
$$=\beta_1f(u_1) + \beta_2f(u_2) + \beta_1g(u_1) + \beta_2g(u_2) = $$
$$=\beta_1(f(u_1) + g(u_1)) + \beta_{2}(f(u_2) + g(u_2)) =$$
$$=\beta_1(f + g)(u_1) + \beta_2(f + g)(u_2) \Ra f + g \in \mathcal{L}(U, V)$$
\end{proof}

Когда два конечномерных векторных пространства изоморфны?
\begin{lemma}
$U \cong V, f \colon U \to V$ "--- изоморфизм.
$\dim U < \infty$
$u_1, \cdots, u_n$ "--- базис U, 
тогда $f(u_1), \cdots, f(u_n)$ "--- базис V.
\end{lemma}
\begin{proof}
$$\alpha_1 f(U_1) + \cdots + \alpha_nf(U_n) = 0$$
$$f(\underbrace{\alpha_1 u_1 + \cdots + \alpha_nU_n}_{\in ker(f) = \{0\}}) = 0$$

$\alpha_1u_1 + \cdots + \alpha_nU_n = 0, u_1 + \cdots + u_n$ "--- базис U $\Ra \alpha_1 = \cdots = \alpha_n = 0$

$\Ra f(u_1), \cdots, f(u_n)$ "--- линейно независимы.

$v \in V$, так как f "--- сюръекция, то $\exists u \in U, f(u) = v$

$$\exists \alpha_1, \cdots, \alpha_n \in  K, u = \alpha_1 u_1 + \cdots + \alpha_nu_n$$ 
$$v = f(u) = \alpha_1f(u_1) + \cdots + \alpha_nf(u_n)$$
Верно $\forall v \in V, f(u_1), \cdots, f(u_n)$ "--- семейство порождающих.

$\Ra f(u_1), \cdots, f(u_n)$ "--- базис V.
\end{proof}

\begin{conseq}
U, V "--- векторное пространство над K, конечномерное. 

$U \cong V \Lra \dim U = \dim V$
\end{conseq}

\begin{proof}
$\Ra$ базис в базис $\Ra \dim U = \dim V$

$\La$

$\dim U = \dim V$

$f: U \to V$ 

$u_1, \cdots, u_n$ "--- базис U.

$v_1, \cdots, v_n$ "--- базис V.

$u \in U$

$$u = \alpha_1u_1 + \cdots + \alpha_nu_n$$
$$f(u) = \alpha_1 v_1 + \cdots + \alpha_n v_n$$

f "--- линейно(упражнение)

f "--- сюръективно.

$$v \in V$$
$$\exists \alpha_1, \cdots, \alpha_n, v = \alpha_1 v_1 + \cdots + \alpha_n v_n$$
$$v = f(\alpha_1 u_1 + \cdots + \alpha_n u_n)$$
f "--- сюръективно.
f "--- инъективно.

$$u \in ker(f) = \alpha_1 u_1 + \cdots + \alpha_nu_n, f(u) = 0, 
\alpha v_1 + \cdots + \alpha_nv_n = 0 $$
$v_1, \cdots, v_n$ "--- базис V $\Ra \alpha_1 = \cdots = \alpha_n = 0$
$\Ra ker(f) = \{0\} \Ra$ f "--- инъективно.

\end{proof}

\begin{conseq}
$\dim U = n \Ra U \cong K^n$

Фиксируем базис в $U\colon u_1 \cdots u_n$
$$ u = \alpha_1 u_1 + \cdots + \alpha_n u_n \to 
\begin{pmatrix}
\alpha_1\\
\vdots\\
\alpha_n\\
\end{pmatrix}$$

Зависит от выбора базиса.
\end{conseq}

\begin{Rem}
В общем случае два векторных пространства изоморфны 
$\Ra$ мощность их базисов совпадает. 
\end{Rem}

\begin{Rem}
$U, V, \{u_{i}\}_{i \in I}$ "--- базис U.

$f \in \mathcal{L}(U, V)$

Тогда f полностью определяется своими значениями на базисных векторах.

$f(u_i) i \in I$

$u = \sum_{i \in I}\alpha_i u_i$, почти все $\alpha_i = 0$

$f(u) = \sum_{i \in I}\alpha_if(u_i)$
\end{Rem}

\begin{exmp}
$$U = K^n, V = K^m$$
$$A \in M(n, m, K)$$
$$f \colon U \to V$$
$$ u \to Au$$

f "--- линейное отображение $A(\alpha_1 u_1 + \alpha_2 u_2) = \alpha_1Au_1 + \alpha_2 A u_2$

Общая ситуация $\dim U = m, \dim V = n$.

Фиксируем $u_1, \cdots, u_m$ "--- базисы u, $v_1, \cdots, v_n$ "--- базисы V/

\begin{Rem}
Если U = V, то базисы могут быть различными.
\end{Rem}

$f \in \mathcal{L}(U,V)$
\end{exmp}

\begin{Def}
Матрицей f в базисах $\{u_1 \cdots, u_m\}, \{v_1, \cdots, v_n\}$
называется следующая матрица $A \in M(n, m, K)$

$$u_j, f(u_j) = \sum\alpha_{ij}v_i$$
$$A = (\alpha_{ij})_{i = 1 \cdots n, j = 1 \cdots m}$$
\end{Def}


\begin{exmp}
$$\R^2$$
Поворот на угол $\Phi$ вокруг начала координат "--- линейное отображение.

$$u_1 = 
\begin{pmatrix}
1\\
0\\
\end{pmatrix}, u_2 = 
\begin{pmatrix}
0\\
1\\
\end{pmatrix}$$


$$v_1 = 
\begin{pmatrix}
1\\
0\\
\end{pmatrix}, 
v_2 = 
\begin{pmatrix}
0\\
1\\
\end{pmatrix}$$

$$ 
f(u_1) = 
\begin{pmatrix}
\cos \Phi\\
\sin \Phi\\
\end{pmatrix}
$$

$$ 
f(u_2) = 
\begin{pmatrix}
-\sin \Phi\\
\cos \Phi\\
\end{pmatrix}
$$

Матрица поворота на $\Phi$ в этих базисах 
$$ 
\begin{pmatrix}
\cos \Phi & -\sin \Phi\\
\sin \Phi & \cos \Phi\\
\end{pmatrix}
$$
\end{exmp}
\begin{exmp}
$$U = \{f \in \R[x], deg f \le n\}$$
$$\dim_{\R}U = n + 1$$
$$\frac{d}{dx}\colon U \to V = \{f \in \R[x]\colon \deg f \le n - 1\}$$
$$u_0 = 1, \cdots, u_n = x^n$$
$$v_0 = 1, \cdots, v_{n - 1} = x^{n - 1}$$
$$
\begin{pmatrix}
0&1&0&\cdots&0\\
0&0&2&\cdots&0\\
\cdots&\cdots&\cdots&\cdots&\cdots\\
0&0&0&\cdots&n\\
\end{pmatrix} 
$$
$$A = [f]_{\{u_1, \cdots, u_m\}, \{v_1, \cdots, v_n\}}$$
\end{exmp}

\begin{Rem}
$$u \in U, u \to 
\begin{pmatrix}
\beta_1\\
\vdots\\
\beta_m\\
\end{pmatrix}, u = \sum_{j = 1}^{m}\beta_{j}u_{j}$$

$$f(u) = \sum_{i = 1}^{n}\gamma_{i}v_{i}, f(u) \to 
\begin{pmatrix}
\gamma_1\\
\vdots\\
\gamma_n\\
\end{pmatrix}$$

$$
\begin{pmatrix}
\gamma_1\\
\vdots\\
\gamma_n\\
\end{pmatrix}
= A
\begin{pmatrix}
\beta_1\\
\vdots\\
\beta_m\\
\end{pmatrix} 
$$

$$f(u) = f(\sum_{j = 1}^{n}\beta_ju_i) = \sum_{j = 1}^{n}\beta_{j}f(u_j) = $$
$$= \sum_{j = 1}^{m}\beta_j \sum_{i = 1}^{n}\alpha_{ij}v_i = \sum_{i = 1}^{n}(\sum_{j = 1}^{m}\alpha_{ij}\beta_{j}) v_{i}$$

$$\gamma_{i} = \sum_{j = 1}^{m}\alpha_{ij}\beta_{j}$$
\end{Rem}
$$U, V \dim U = m < \infty, \dim V = n < \infty$$

Фиксируем базис 
$\mathcal{L}(U, V) \to M(n, m, K)$

$f \to [f]_{\{u_{j}\}, \{v_{i}\}}$ "--- линейное отображение.

$$[f_1 + f_2] = [f_1] + [f_2]$$
$$[\alpha f] = \alpha[f]$$
 
$j$-ый столбец $[f_1 + f_2]$ координаты разложения $(f_1 + f_2)(u_j)$ в базисе $v_1, \cdots, v_n$

$$f_1(u_j) + f_2(u_j)$$
$$A = (\alpha_{ij}) $$
                       
Определим $f \colon f(u_j) = \sum_{i = 1}^{n}\alpha_{ij}v_i$

$f \to A$

Нулевой матрице соответствует нулевое отображение $f \colon U \to V$, $u \to 0_v$

\begin{conseq}
$$\mathcal{L}(U, V) \cong M(n, m, K)$$
\end{conseq}
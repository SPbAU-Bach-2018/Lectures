\section{Линейные отображения}

\begin{Def}
	$U$, $V$ "--- векторные пространства над $K$.
	$f\colon U \to V$ называется линейным отображением, если
	\[ \forall u_1, u_2 \in U, c_1, c_2 \in K, f(c_1u_1 + c_2u_2) = c_1f(u_1) + c_2f(u_2) \]
	"--- линейность.
\end{Def}

\begin{Rem}
	\begin{itemize}
		\item $\forall u_1, u_2 \in U, f(u_1 + u_2) = f(u_1) + f(u_2)$ "--- адитивность.
		\item $\forall c \in K, \forall u \in U, f(cU) = cf(U)$ "--- однородность степени 1.
	\end{itemize}
\end{Rem}

\begin{Def}
	$f \colon U \to V$ "--- линейное отображение.
	$f(U) = \Im(F)$ "--- образ $f$.
	\[ f^{-1}(0_v) = \{u \in U \mid f(u) = 0_v\} = \ker f \]
	"--- ядро $f$.
\end{Def}

\textbf{Упражнение:}
\[ \Im f \subset V \quad \ker f \subset U \]

\begin{conseq}{Свойства линейного отображения}
	\begin{enumerate}
		\item $f(0_u) = 0_v$
		\item $f(0_k u) = 0_kf(u) = 0_v$: ядро не пусто (всегда содержит ноль).
	\end{enumerate}
\end{conseq}

\begin{lemma}
	$f\colon u \to v$ "--- линейное отображение.
	\[ \text{$f$ "--- инъективно} \Lra \ker f = \{0_u\} \]
\end{lemma}

\begin{proof}
	\begin{description}
	\item[$\Ra$:]
		$f$ "--- инъективно.
		\[ \forall x \in V \left|f^{-1}(V)\right| \le 1 \Ra |f^{-1}(0)| \le 1 \]
		но $0 \subset f^{-1}(0) \Ra \ker f = \{0\}$

	\item[$\La$:]
		\begin{gather*}
			u_1, u_2 \in U: f(u_1) = f(u_2) \\
			f(u_1 - u_2) = f(u_1) - f(u_2) = 0 \\
			u_1 - u_2 \in \ker f = \{0\} \Ra u_1 = u_2
		\end{gather*}
		В силу произвольности $u_1$ и $u_2$, $f$ инъективно.
	\end{description}
\end{proof}

\begin{Def}
	$U$, $V$ "--- векторные пространства над $K$.
	$f \colon U \to V$ "--- изоморфизм, если
	\begin{enumerate}
		\item $f$ "--- линейное отображение.
		\item $f$ "--- биекция.
	\end{enumerate}
	Говорим, что $U$ и $V$ изоморфны: $U \cong V$.
\end{Def}

\begin{Rem}
	Линейное отображение $f\colon U \to V$ "--- изоморфизм, если
	\begin{enumerate}
		\item $\ker f = \{0\}$
		\item $\Im f = f(U) = V$
	\end{enumerate}
\end{Rem}

\begin{Def}
	$U, V$ "--- векторное пространство.
	$Hom(U, V)$, $\mathcal{L}(U, V)$ "--- пространство линейных отображений из $U$ в $V$.
\end{Def}

\begin{theorem}
	$\mathcal{L}(U, V)$ "--- векторное пространство над $K$: \\
	$f, g \in \mathcal{L}(U, V)$
	\begin{gather*}
		\forall u \in U, (f + g)(u) = f(u) + g(u) \\
		(\alpha f)(u) = \alpha f(u)
	\end{gather*}
	$f + g, \alpha f$ "--- линейные отображения.
\end{theorem}

\begin{proof}
	\begin{alignat*}{2}
		(f + g)(\beta_1u_1 + \beta_2u_2) &= f(\beta_1 u_1 + \beta_2u_2)+
			g(\beta_1u_1 + \beta_2u_2) = \\
		&=\beta_1f(u_1) + \beta_2f(u_2) + \beta_1g(u_1) + \beta_2g(u_2) = \\
		&=\beta_1(f(u_1) + g(u_1)) + \beta_2(f(u_2) + g(u_2)) = \\
		&=\beta_1(f + g)(u_1) + \beta_2(f + g)(u_2) \Ra f + g \in \mathcal{L}(U, V)
	\end{alignat*}
\end{proof}

Когда два конечномерных векторных пространства изоморфны?
\begin{lemma}
	$U \cong V, f \colon U \to V$ "--- изоморфизм, $\dim U < \infty$, $u_1, \cdots, u_n$ "--- базис $U$.\\
	Тогда $f(u_1), \dots, f(u_n)$ "--- базис $V$.
\end{lemma}
\begin{proof}
	\begin{gather*}
		\alpha_1 f(u_1) + \dots + \alpha_nf(u_n) = 0 \\
		f(\underbrace{\alpha_1 u_1 + \dots + \alpha_nu_n}_{\in\ker f = \{0\}}) = 0 \\
	\end{gather*}
	$\alpha_1u_1 + \dots + \alpha_nU_n = 0$ и $u_1, \dots, u_n$ "--- базис $U$, поэтому
	\[ \alpha_1 = \dots = \alpha_n = 0 \]
	откуда $f(u_1), \dots, f(u_n)$ "--- линейно независимы.

	Рассмотрим $v \in V$: так как $f$ "--- сюръекция, то $\exists u \in U, f(u) = v$.
	Говорим, что $U$ и $V$ изоморфны $U \cong V$.
	\begin{gather*}
		\exists \alpha_1, \dots, \alpha_n \in  K, u = \alpha_1 u_1 + \dots + \alpha_nu_n \\
		v = f(u) = \alpha_1f(u_1) + \dots + \alpha_nf(u_n) \\
	\end{gather*}
	Верно, что $\forall v \in V, f(u_1), \dots, f(u_n)$ "--- семейство порождающих.
	Значит, $f(u_1), \dots, f(u_n)$ "--- базис V.
\end{proof}

\begin{conseq}
	$U$, $V$ "--- векторное пространство над $K$, конечномерное.
	\[ U \cong V \Lra \dim U = \dim V \]
\end{conseq}
\begin{proof}
	\begin{description}
	\item[$\Ra$:]
		Переводим базис в базис, получаем $\dim U = \dim V$.

	\item[$\La$:]
		$\dim U = \dim V$.
		Рассмотрим функцию $f: U \to V$:
		$u_1, \dots, u_n$ "--- базис $U$, $v_1, \dots, v_n$ "--- базис $V$.
		\begin{gather*}
			u = \alpha_1u_1 + \dots + \alpha_nu_n \\
			f(u) = \alpha_1 v_1 + \cdots + \alpha_n v_n
		\end{gather*}
		$f$ "--- линейно (упражнение).

		Покажем, что $f$ сюръективно:
		\begin{gather*}
			\exists \alpha_1, \dots, \alpha_n\colon v = \alpha_1 v_1 + \dots + \alpha_n v_n \\
			v = f(\alpha_1 u_1 + \dots + \alpha_n u_n) \\
		\end{gather*}
		$f$ "--- сюръективно.

		Покажем, что $f$ инъективно:
		\[
			u \in \ker f = \alpha_1 u_1 + \cdots + \alpha_nu_n, f(u) = 0,
				\alpha v_1 + \cdots + \alpha_nv_n = 0 \\
		\]
		$v_1, \dots, v_n$ "--- базис $V$, значит $\alpha_1 = \dots = \alpha_n = 0$, откуда
		$\ker f = \{0\}$, и $f$ "--- инъективно.
	\end{description}
\end{proof}

\begin{conseq}
	$\dim U = n \Ra U \cong K^n$
\end{conseq}
\begin{proof}
	Фиксируем базис в $U$: $u_1, \dots, u_n$:
	\[
		u = \alpha_1 u_1 + \dots + \alpha_n u_n \to
		\begin{pmatrix}
			\alpha_1 \\
			\vdots \\
			\alpha_n \\
		\end{pmatrix}
	\]
	Зависит от выбора базиса.
\end{proof}

\begin{Rem}
	В общем случае, если два векторных пространства изоморфны, то мощность их базисов совпадает.
\end{Rem}

\begin{Rem}
	$U$, $V$, $\{u_i\}_{i \in I}$ "--- базис $U$, $f \in \mathcal{L}(U, V)$.
	Тогда $f$ полностью определяется своими значениями на базисных векторах:
	\begin{align*}
		u &= \sum_{i \in I} \alpha_i u_i \quad \text{почти все $\alpha_i = 0$} \\
		f(u) &= \sum_{i \in I}\alpha_if(u_i)
	\end{align*}
\end{Rem}

\begin{exmp}
	$U = K^n$, $V = K^m$, $A \in M(n, m, K)$.
	$f \colon U \to V; u \mapsto Au$.
	$f$ "--- линейное отображение:
	\[ A(\alpha_1 u_1 + \alpha_2 u_2) = \alpha_1Au_1 + \alpha_2 A u_2 \]

	Общая ситуация $\dim U = m$, $\dim V = n$.
	Фиксируем $u_1, \dots, u_m$ "--- базисы $U$, $v_1, \dots, v_n$ "--- базисы $V$
	\begin{Rem}
		Если $U = V$, то базисы могут быть различными.
	\end{Rem}
	\[ f \in \mathcal{L}(U,V) \]
\end{exmp}

\begin{Def}
	Матрицей $f$ в базисах $\{u_1 \dots, u_m\}$, $\{v_1, \dots, v_n\}$ называется следующая матрица $A \in M(n, m, K)$:
	\begin{gather*}
		f(u_j) = \sum\alpha_{ij}v_i \\
		A = (\alpha_{ij})_{i = 1..n, j = 1..m}
	\end{gather*}
\end{Def}


\begin{exmp}
	$\R^2$.
	Поворот на угол $\phi$ вокруг начала координат "--- линейное отображение.
	\begin{gather*}
		\begin{array}{cc}
			u_1 = \begin{pmatrix}
				1 \\
				0 \\
			\end{pmatrix}
			&
			u_2 = \begin{pmatrix}
				0 \\
				1 \\
			\end{pmatrix}
			\\
			v_1 =
			\begin{pmatrix}
				1 \\
				0 \\
			\end{pmatrix}
			&
			v_2 =
			\begin{pmatrix}
				0 \\
				1 \\
			\end{pmatrix}
		\end{array} \\
		\begin{array}{cc}
			f(u_1) = \begin{pmatrix}
				\cos \phi \\
				\sin \phi \\
			\end{pmatrix}
			&
			f(u_2) = \begin{pmatrix}
				-\sin \Phi \\
				\cos \Phi \\
			\end{pmatrix}
		\end{array}
	\end{gather*}
	Матрица поворота на $\Phi$ в этих базисах
	\[
		\begin{pmatrix}
			\cos \Phi & -\sin \Phi \\
			\sin \Phi & \cos \Phi \\
		\end{pmatrix}
	\]
\end{exmp}

\begin{exmp}
	$U = \{f \in \R[x] \mid \deg f \le n\}$, $\dim_{\R}U = n + 1$.
	\begin{gather*}
		\frac{\d}{\d x}\colon U \to V = \{f \in \R[x] \mid \deg f \le n - 1\} \\
		u_0 = 1, \dots, u_n = x^n \\
		v_0 = 1, \dots, v_{n - 1} = x^{n - 1} \\
		\begin{pmatrix}
			0 & 1 & 0 & \cdots & 0 \\
			0 & 0 & 2 & \cdots & 0 \\
			\vdots&\vdots&\vdots&\ddots&\vdots \\
			0 & 0 & 0 & \cdots & n \\
		\end{pmatrix} \\
		A = [f]_{\{u_1, \cdots, u_m\}, \{v_1, \cdots, v_n\}}
	\end{gather*}
\end{exmp}

\begin{Rem}
	\begin{gather*}
		u \in U, u \to \begin{pmatrix}
			\beta_1 \\
			\vdots \\
			\beta_m \\
		\end{pmatrix}, u = \sum_{j = 1}^m\beta_ju_j \\
		f(u) = \sum_{i = 1}^n\gamma_iv_i, f(u) \to \begin{pmatrix}
			\gamma_1 \\
			\vdots \\
			\gamma_n \\
		\end{pmatrix} \\
		\begin{pmatrix}
			\gamma_1 \\
			\vdots \\
			\gamma_n \\
		\end{pmatrix}
		= A
		\begin{pmatrix}
			\beta_1 \\
			\vdots \\
			\beta_m \\
		\end{pmatrix} \\
		f(u) = f\left(\sum_{j = 1}^n\beta_ju_i\right) = \sum_{j = 1}^n\beta_jf(u_j)
		= \sum_{j = 1}^m\beta_j \sum_{i = 1}^n\alpha_{ij}v_i = \sum_{i = 1}^n\left(\sum_{j = 1}^m\alpha_{ij}\beta_j\right) v_i \\
		\gamma_i = \sum_{j = 1}^m\alpha_{ij}\beta_j \\
	\end{gather*}
\end{Rem}

$U$, $V$, $\dim U = m < \infty$, $\dim V = n < \infty$.
Фиксируем базис. Хотим рассмотреть переход $\mathcal{L}(U, V) \to M(n, m, K)$.
$f \to [f]_{\{u_j\}, \{v_i\}}$ "--- линейное отображение.
\begin{gather*}
	[f_1 + f_2] = [f_1] + [f_2] \\
	[\alpha f] = \alpha[f] \\
\end{gather*}
$j$-ый столбец $[f_1 + f_2]$ "--- координаты разложения $(f_1 + f_2)(u_j)$ в базисе $v_1, \dots, v_n$.
\begin{gather*}
	f_1(u_j) + f_2(u_j) \\
	A = (\alpha_{ij}) \\
	f(u_j) = \sum_{i = 1}^n\alpha_{ij}v_i
\end{gather*}
Получили отображение $f \to A$.
Нулевой матрице соответствует нулевое отображение $f\colon U \to V\colon u \mapsto 0_v$.
\begin{conseq}
	\[ \mathcal{L}(U, V) \cong M(n, m, K) \]
\end{conseq}

\subsection{Матрица композиции линейных отображений}

\begin{theorem}
	$U \stackrel{g}{\to} V \stackrel{f}{\to}W$, $g$, $f$ "--- линейные отображения,
	$U$, $V$, $W$ "--- конечномерные.
	$\{u_i\}_{i=1..m}$, $\{v_i\}_{i=1..n}, \{w_i\}_{i = 1..k}$ "--- базисы $U$, $V$, $W$.
	\[ [f \circ g]_{\{u_i\}, \{w_i\}} = [f]_{\{v_i\}, \{w_i\}}[g]_{\{u_i\}, \{v_i\}} \]
\end{theorem}

\begin{proof}
	\begin{gather*}
		A = (a_{ij}) = [f], B = (b_{js}) = [g] \\
		C = (c_{is}) = [f \circ g] \\
		C = m \times k \\
		(f \circ g)(u_s) = f(g(u_s)) = f\left(\sum_{j = 1}^nb_{js}v_j\right) = \sum_{j = 1}^nb_{js}f(v_j) = \\
		= \sum_{j = 1}^nb_{js}\sum_{i = 1}^ka_{ij}w_i = \sum_{i = 1}^k\left(\sum_{j = 1}^nb_{js}a_{ij}\right) w_i
			= \sum_{i = 1}^k\underbrace{\left(\sum_{j = 1}^na_{ij}b_{js}\right)}_{c_{is}}w_i \\
		\Ra c_{is} = \sum_{j = 1}^na_{ij}b_{js} \Ra C = AB
	\end{gather*}
\end{proof}

\subsection{Преобразование матрицы линейного отображения при замене базисов}

\begin{theorem}
	$U$, $V$ "--- конечномерные векторные пространства, $f\colon U \to V$ "--- линейное отображение.
	$\{u_1, \dots, u_m\}$, $\{u'_1, \dots, u'_m\}$  "--- базисы $U$,
	$\{v_1, \dots, v_m\}$, $\{v'_1, \dots, v'_m\}$  "--- базисы $V$.
	$C$ "--- матрица перехода от $\{u_j\}$ к $\{u'_j\}$,
	$D$ "--- матрица перехода от $\{v_i\}$ к $\{v'_i\}$,
	$A$ "--- матрица $f$ в базисах $\{u_j\}$, $\{v_i\}$,
	$A'$ "--- матрица $f$ в базисах $\{u'_j\}$, $\{v'_i\}$.
	Тогда
	\[ A' = D^{-1}AC \]
\end{theorem}
\begin{Rem}
	$\dim U = m$, $\dim V = n$.
	$A, A' \in M(n, m, K)$,
	$C \in M(m, m, K)$,
	$D \in M(n, n, K)$.
\end{Rem}
\begin{proof}
	$A = (a_{ij})$, $A'=(a'_{ij})$, $C = (c_{js})$.
	$D^{-1}$ "--- матрица перехода от $\{v_r'\}$ к $\{v_r\}$, $D^{-1}=(\check d_{ri})$.
	\begin{gather*}
		f(u'_s) = \sum_ra'_{rs}v'_r \\
		f(u'_s) = f(\sum_{j=1}^m c_{js}u_j) = \sum_{j=1}^m c_{js}f(u_j)
			= \sum_{j=1}^m c_{js} \sum_{i=1}^n a_{ij}v_i = \sum_{j=1}^m c_{js}\sum_{i=1}^n a_{ij} \sum_{r=1}^n\check d_{ri} v_r' = \\
		= \sum_{r=1}^n \left(\sum_{j=1}^m \sum_{i=1}^n c_{js}a_{ij}\check d_{ri}\right) v_r'
			= \sum_{r=1}^n \left(\sum_{j=1}^m \sum_{i=1}^n \check d_{ri}a_{ij}c_{js}\right)v_r' \\
		\sum_{j=1}^m \sum_{i=1}^n \check d_{ri}a_{ij} c_{js} = a'_{rs} \\
		(A')_{rs} = a'_{rs} = \left(\sum_{j=1}^m \sum_{i=1}^n (\check d_{ri}a_{ij})c_{js}\right)
			= \sum_{j = 1}^m(D^{-1}A)_{rj}c_{js} = (D^{-1}AC)_{rs} \Ra A' = D^{-1}AC
	\end{gather*}
\end{proof}

\begin{theorem}
	$U, V$ "--- конечномерные векторные пространства над $K$, $\dim U = m$, $\dim V = n$,
	$f\colon U \to V$ "--- линейное отображение.
	Тогда существуют базисы $U$ и $V$, что
	\[
		[f] = \begin{pmatrix}
			E_r & 0 \\
			0   & 0
		\end{pmatrix}
	\]
	$E_r$ "--- единичная матрица размера r.
\end{theorem}

\begin{proof}
	Пусть $\dim \ker f = m - r$ для некоторого $0 \le r \le m$.
	$u_{r+1}, \dots, u_m$ "--- базис $\ker f$.
	Дополним до базиса $U$:
	\[ u_1, \dots, u_r, \underbrace{u_{r+1}, \dots, u_m}_{\text{базис $\ker f$}} \]
	Для $i = 1..r$ возьмём $v_i = f(u_i)$.
	Утверждаем, что $v_i$ "--- линейно независимые.
	\begin{gather*}
		0 = \sum_{i=1}^r \alpha_iv_i = \sum_{i=1}^r \alpha_if(u_i)= f\left(\sum_{i=1}^r \alpha_iu_i\right) \\
		\sum_{i=1}^r \alpha_iu_i \in \ker f \Ra \sum_{i=1}^r \alpha_iu_i = \sum_{i=r+1}^m \beta_i u_i \\
		\sum_{i=1}^r \alpha_iu_i - \sum_{i=r+1}^m \beta_iu_i = 0
	\end{gather*}
	$u_1, \dots, u_m$ "--- базис $U$, значит $\alpha_i - \beta_i = 0$ и $f(u_i)$ линейно независимы.
	Значит $v_1, \dots, v_r$ линейно независимы.

	Дополним до базиса $V$:
	\begin{gather*}
		v_1, \dots, v_r, v_{r+1}, \dots, v_n \\
		\forall j=1..r, f(u_j) = v_j
	\end{gather*}
	Покажем, что для $j=1..r$ $j$-ый столбец $[f]$ есть $\begin{pmatrix} 0 \\0 \\1 \\0 \\0 \\ \end{pmatrix}$,
	а для прочих $j$ "--- нулевой столбец:
	\begin{gather*}
		v_j = 0u_1 + \dots + 0u_{j-1} + 1u_j + 0u_{j+1} + \dots \\
		\forall j\colon r < j \le m, f(u_j) = 0 = 0 v_1 + \dots + 0v_n
	\end{gather*}
	Откуда
	\[
		[f] = \begin{pmatrix}
			E_r & 0 \\
			0   & 0
		\end{pmatrix}
	\]
\end{proof}

\begin{conseq}
	\begin{enumerate}
	\item
		$A \in M(n, m, K)$.
		Тогда существуют обратимые матрицы $C \in M(m, m, K)$ и  $D \in M(n, n, K)$, что
		\begin{gather*}
			D^{-1}AC = \begin{pmatrix}
				E_r & 0 \\
				0   & 0
			\end{pmatrix} \\
			U = K^m, V = K^n \\
			f\colon U \to V\colon x \mapsto Ax
		\end{gather*}

	\item
		$U, V$ "--- векторные конечномерные пространства над $K$.
		$f\colon U \to V$ "--- линейное отображение.
		\[ \dim \Im f + \dim \ker f = \dim U \]
	\end{enumerate}
\end{conseq}
\begin{proof}
	В обозначениях из доказательства теоремы $\dim \ker f = m - r$.
	\begin{gather*}
		\Im f = \{f(u) \mid u \in U\} = \left\{f\left(\sum_{j=1}^m \alpha_i u_j\right) \mid \alpha_i \in K\right\} = \\
		= \left\{\sum_{j=1}^m\ \alpha_i f(u_j) \mid \alpha_j \in K\right\} = \left\{\sum_{j=1}^r \alpha_j f(u_j) \mid \alpha_j \in K\right\} \\
		\Ra \Im f = \left<f(u_1), \dots, f(u_r)\right> = \left<v_1, \dots, v_r\right>
	\end{gather*}
	Доказываем линейную независимость $v_1, \dots, v_r$:
	$v_1, \dots, v_r$ "--- базис $\Im f$, $\dim \Im f = r$.
	\[ \dim \Im f + \dim \ker f = r + m - r = m = \dim U \]
\end{proof}

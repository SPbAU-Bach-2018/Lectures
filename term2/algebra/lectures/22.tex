\section{Диагонализуемый оператор}
\setauthor{Черникова Ольга}
\begin{Def}
$f \in End(v), \dim(V) < \infty$

$f$ "--- диагонализуем, если существует базис $V$, в котором
матрица $f$ имеет диагональный вид. 
\end{Def}

\begin{theorem}{}
$f$ "--- диагонализуем $\Lra$ существует базис $V$, состоящий из собственных
векторов $f$.
\end{theorem}
\begin{proof}
$\Ra$ 

$f$ "--- диагонализуем.

Существует базис $v_1, \cdots, v_n \colon [f]_{v_1, \cdots, v_n} = 
\begin{pmatrix}
\lambda_1&0&\cdots&0\\
\vdots&\ddots&\vdots&\vdots\\
0&0&\cdots&\lambda_n\\
\end{pmatrix}$

($\lambda$ не обязательно различные).

$f(v_i) = 0\cdot v_1 + \cdots + 0\cdot v_{n - 1} + \lambda_i v_i + 0\cdots v_{i + 1} +  \cdots + 0v_{n} = \lambda_i v_i \Ra
v_i$ собственный вектор отвечающий $\lambda_i \Ra$ базис $v_1, \cdots, v_n$ состоит из собственных векторов.

$\La$

$\exists v_1, \cdots, v_n \colon f(v_i) = \lambda_i v_i$

$[f]_{v_1, \cdots, v_n} = 
\begin{pmatrix}
\lambda_1&0&\cdots&0\\
\vdots&\ddots&\vdots&\vdots\\
0&0&\cdots&\lambda_n\\
\end{pmatrix}$

То есть в базисе $v_1, \cdots, v_n$ матрица $[f]$ диагоналная, то есть $f$ "--- диагонализуемая.
\end{proof}

\begin{Rem}
Не всякий оператор диагонализуем.

Если у $f$ лишь одно собственное число($K$ "--- алгебраически замкнуто), то геометрическая кратность меньше алгебраической кратности

$$\dim U_1(\lambda) < \dim V$$
$$U_1 \lneqq V$$

$\Ra f$ не диагонализуем.

$$End(V)$$
$$\dim_{\C}(V) = n < \infty$$
$$M(n, n, \C)$$
$$\dim_{\C}(M(n, n, \C)) = n^2$$
$$\dim_{\R}(M(n, n, \C)) = 2n^2$$

Множество недиагонализуемых операторов есть множество меры 0.
\end{Rem}
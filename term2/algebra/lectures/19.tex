\section{Формулировка теоремы о каноническом виде матрицы линейного оператора}
\begin{Def}
$\mathcal{J}_k(\lambda) = \begin{pmatrix}
\lambda&1&0&\cdots&0\\
\vdots&\vdots&\vdots&\cdots&\vdots\\
0&0&0&\cdots&1\\
0&0&0&\cdots&\lambda\\
\end{pmatrix}$

$\mathcal{J})_k(\lambda)$ "--- Жорданова клетка, размера k, отвечает $\lambda$.

$\mathcal{J}_1(\lambda) = (\lambda)$
\end{Def}

\begin{theorem}{}
$K$ "--- алгебраически замкнутое поле, $V$ "--- векторное пространство над $K$, 
$f \in End(V)$.

Тогда существует базис $V$, в котором матрица $f$ имеет блочно диагональный вид.

Причем, диагональные блоки "--- это какие-то клетки Жордана, отвечающие собственным 
числам $f$.

Причем такое представление единственное с точностью до перестановки блоков.
Кроме того, число клеток, отвечающих одному и тому же $\lambda$, равно геометрической
кратности $\lambda$, а суммарный размер клеток, отвечающих $\lambda$, равен
алгебраической кратности $\lambda$. 
\end{theorem}

\begin{Def}
Матрица из предыдущей теоремы называется канонической жордановой формой
матрицы линейного оператора, а соответствующий базис "--- жорданов базис.
\end{Def}
\begin{conseq}
Геометрическая кратность $\le$ алгебраической кратности.
\end{conseq}

\begin{conseq}
$A \in M(n, n, K)$, $K$ "--- алгебраически замкнуто.

$\exists C \det C \ne 0$

$C^{-1}AC$ "--- жорданова форма.

$V = K^{n}$

$f(v) = Av$, $C$ "--- матрица перехода от стандартного базиса к жорданову базису.
\end{conseq}

\begin{exmp}
Пусть $\lambda$ "--- единственное собственное число $f$ $\chi_f(t) = (-1)^{n}(t - \lambda)^n$
\begin{enumerate}
\item $n = 1$, клетка размера 1, алгебраическая кратность  = геометрической кратности.
\item $n = 2$, 
\begin{enumerate}
\item $1 + 1$, алгебраическая кратность  = геометрической кратности = 2
\item 2, алгебраическая кратность = 2, геометрическая кратность = 1.
$$
\begin{pmatrix}
\lambda & 1\\
0&\lambda\\
\end{pmatrix} \begin{pmatrix}
1\\
0\\
\end{pmatrix}
= \begin{pmatrix}
\lambda\\
0\\
\end{pmatrix}
$$

Если $v_1, v_2$ "--- жорданов базис, то $v_1$ "--- собственный вектор.

$v_2 \to v_1 + \lambda v_2$

$$(f - \lambda E) = \begin{pmatrix}
0&1\\
0&0\\
\end{pmatrix}$$
$$
\begin{pmatrix}
0&1\\
0&0\\
\end{pmatrix} X = \begin{pmatrix}
0\\
0\\
\end{pmatrix}
$$
\end{enumerate}

\item $n = 3$
\begin{enumerate}
\item $1 + 1 + 1$, геометрическая кратность = 3
\item $2 + 1$, геометрическая кратность = 2
\item 3, геометрическая кратность = 1
\end{enumerate}

\item $n = 7$
$3 + 2 + 2$

$3 + 3 + 1$

Геометрическая кратность совпадает, максимальная размерность совпадает.
\end{enumerate}
\end{exmp}

\section{Применение жордановой формы}
\begin{enumerate}
\item Вычисление $A^{m}$

$C^{-1}AC = B$ "--- жорданова форма.

$$A = CBC^{-1}$$
$$A^{m} = (CBC^{-1}) = CB^{m}C^{-1}$$

Достаточно возвести жордановы клетки в степень $m$.
$$\mathcal{J}_k(\lambda)^m = (\lambda E + 
\begin{pmatrix}
0&1&\cdots&0\\
\vdots&\vdots&\cdots&\vdots\\
0&0&\cdots&0\\
\end{pmatrix}) = (\lambda E + \mathcal{J})^m = $$
$$= \sum_{r = 0}^{m}C_m^{r}\lambda^{m - r}\mathcal{J}^r$$

$$ 
\mathcal{J}^2 = \begin{pmatrix}
0&0&1&0\\
0&0&0&1\\
0&0&0&0\\
0&0&0&0\\
\end{pmatrix}
$$

$$\mathcal{J}^k = 0$$

$$(\mathcal{J}_k(\lambda))^m = \sum_{r = 0}^{k - 1}C_m^r \lambda^{m - r}\mathcal{J}^r$$
$$
\begin{pmatrix}
\lambda^m& C_m^1 \lambda^{m - 1} &\cdots& C_m^{k - 1}\lambda^{m - k + 1}\\
0&\ddots&\ddots&\ddots\\
0&\ddots&\ddots&C_m^2\lambda^{m - 2}\\
0&\ddots&\ddots&C_m^1\lambda^{m - 1}\\
0&\ddots&\ddots&\lambda^{m}\\
\end{pmatrix}
$$
\item $A^m, g(A), g \in K[t]$
$C^{-1}AC = B$ "--- жорданова форма.
$g(A) = Cg(B)C^{-1}$
\item Вычисление некоторых специальных функций
$$
A = \begin{pmatrix}
2&-1\\
1&0\\
\end{pmatrix}
$$
$$
v_2 = \begin{pmatrix}
1\\
0\\
\end{pmatrix}   
$$
$$
v_1 = \begin{pmatrix}
1\\
1\\
\end{pmatrix} 
$$
$$\begin{pmatrix}
0&1\\
1&-1\\
\end{pmatrix} = 
\begin{pmatrix}
2&-1\\
1&0\\
\end{pmatrix} 
\begin{pmatrix}
1&1\\
1&0\\
\end{pmatrix} = 
\begin{pmatrix}
1&0\\
1&-1\\
\end{pmatrix} \begin{pmatrix}
1&1\\
1&0\\
\end{pmatrix} =
\begin{pmatrix}
1&1\\
0&1\\
\end{pmatrix} =  
$$
$$
\begin{pmatrix}
1&1\\
0&1\\
\end{pmatrix}^m =
\begin{pmatrix}
1&m\\
0&1\\
\end{pmatrix} 
$$

$$
\begin{pmatrix}
2&-1\\
1&0\\
\end{pmatrix} = 
\begin{pmatrix}
1&1\\
1&0\\
\end{pmatrix} \begin{pmatrix}
1&m\\
0&1\\
\end{pmatrix} \begin{pmatrix}
0&1\\
1&-1\\
\end{pmatrix} = 
\begin{pmatrix}
1&m + 1\\
1&m\\
\end{pmatrix} \begin{pmatrix}
0&1\\
1&-1\\
\end{pmatrix} = 
\begin{pmatrix}
m + 1& -m\\
m& -m + 1\\
\end{pmatrix}
$$

$$exp(t) = \sum_{k = 0}^{\infty}\frac{t^k}{k!}$$ 
$$A \in M(n, n, \C)$$
$$exp(A) = \sum_{k = 0}^{\infty}\frac{A^k}{k!}$$
$$A = a_{ij} $$

$$M = max|a_{ij}|$$
$$|(A^k)_{ij}| \le n^{k - 1}M^{k} \le (nM)^k$$
$$(\sum_{k = 0}^{N}\frac{A^k}{k!}) \le \sum_{k = 0}^{N}\frac{(nM)^k}{k!} \le
\sum_{k = 0}^{\infty}\frac{(nM)^k}{k!} < e^{nM}$$

$$C^{-1}AC = B(\text{"--- жорданова форма})$$

$$\exp(A) = C \exp(B)C^{-1}$$

\end{enumerate}
                  
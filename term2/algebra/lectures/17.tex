\section{Двойственное пространство}

$V$ "--- векторное пространство над K.

$\{\Phi| \Phi\colon V \to K, \Phi$ "--- линейное отображение $\}$ "--- множество 
линейных отображений из V в K(пространство над K).

$V^{*}$ "--- двойственное к V пространство линейных отображений со значением в K = $K^{1}$
называется линейным функционалами.

$V^*$ "--- пространство линейных функционалов на V.

Всякий линейный функционал полностью определяется своими значениями на базисных элементах V.

$\{v_i\}_{i \in I}$ "--- базис V.

$\Phi \in V^{*}$
$$v \in V, v = \sum_{i \in I}\alpha_i v_{i}(\text(почти все \alpha_i = 0))$$
$$\Phi(v) = \Phi(\sum_{i \in I}\alpha_i v_{i}) = \sum_{i \in I}\alpha_{i}\Phi_{i}(v_i)$$

Пусть $dim(V) = n < \infty$

Зафиксируем базис $v_1, \cdots, v_n$

$v_i^{*} \in v^{*}$
$$v_{i}^{*} = \begin{cases}1, & i = j\\ 0, & i \ne j\\ \end{cases} $$
$$v = \sum_{i = 1}^{n}\alpha_i v_i$$
$$v_i^{*}(v) = \alpha_i$$

\begin{theorem}{}
$dim(V) = n < \infty$

В указанных выше обозначениях $v_1^*, \cdots, v_n^*$ "--- базисы $V^{*}$
\end{theorem}

\begin{proof}
$\Phi \in V^{*}$

$\Phi \colon V \to K, \Phi$ "--- линейное отображение. 

$\beta_{i} = \Phi(v_i)$

Нужно проверить, что $\Phi = \sum_{i = 1}^{w} \beta_{i}v_i^{*}$
$$\forall v \in V, \Phi(v) \ne (\sum_{i = 1}^{n} \beta_{i}v_i^{*})(v)$$

Достаточно проверить это равенство для $v = v_1, \cdots, v_n$

$\Phi(v_j) = \beta_j$
$$(\sum_{i = 1}^{n} \beta_i v_i^{*})(v_j) = \sum_{i = 1}^{n}\beta_iv_i^{*}(v_j) = \beta_j$$

$$\forall j = 1 \cdots n, \Phi(v_j) = (\sum_{i = 1}^{n}\beta_{i}v_{i}^{*})(v_j) \Ra$$
$$\Ra \Phi = \sum_{i = 1}^{n}\beta_i v_i^{*}$$

$v_1^{*}, \cdots, v_n^{*}$ "--- семейство образующих.

Необходимо проверить линейную независимость. 
 
$$\sum_{i = 1}^{n}\beta_{i}v_{i}^{*} = 0 \in V^{*}$$
$$\forall j,  (\sum_{i = 1}^{n} \beta_i v_{i}^{*})(v_j) = 0$$
$$\sum_{i = 1}^{n} \beta_{i}v_{i}^*(v_j) = \beta_j$$
$\Ra \forall j, \beta_j = 0 \Ra v_1^{*} \cdots v_n^{*}$ "--- линейно независимые. 
\end{proof}

\begin{conseq}
$$dim(V) < \infty \Ra V \simeq v^{*}$$

Изоморфизм может быть построен следующим образом:

$\Psi\colon v \to v^*$

$v_1, \cdots, c_n$ "--- базис V.

$\alpha_1 v_1 + \cdots + \alpha_n v_n \to \alpha_1 v_1^* + \cdots \alpha_n v_n^*$

является изоморфизмом. 
\end{conseq}

\begin{proof}
$\dim(V^*) = n = \dim V \Ra V^* \cong V$

$\Psi$ "--- линейно

$\Phi \in V^*, \Phi$ "--- линейная комбинация $v_1^*, \cdots, v_n^* \Ra \Psi$ сюръективно. 

$\Psi(\alpha_1 v_1 + \cdots + \alpha_n v_n) = 0 \Lra \alpha_1 v_1^* + \cdots + \alpha_n v_n^* = 0 \Lra \alpha_1 = \cdots = \alpha_n = 0$

$ker(\Psi) = \{0\} \Ra \Psi$ инъективно.
$\Psi$ "--- биекция.
\end{proof}
\begin{Rem}
Построенный изоморфизм зависит от выбора базиса в V.
\end{Rem}
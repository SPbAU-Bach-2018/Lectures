\section{Двойственное пространство}

\begin{Def}
	$V$ "--- векторное пространство над $K$. Двойственное пространство
	\[ V^* = \{\Phi \mid \text{$\Phi\colon V \to K$ "--- линейное отображение}\} \]
	"--- множество линейных отображений из $V$ в $K$ (пространство над $K$).
\end{Def}
\begin{Def}
	Линейное отображение $f\colon V \ra K = K^1$ называется линейным функционалом.
	$V^*$ "--- пространство линейных функционалов на $V$.
\end{Def}

\begin{Rem}
	Всякий линейный функционал полностью определяется своими значениями на базисных элементах $V$.
	Пусть $\{v_i\}_{i \in I}$ "--- базис $V$, $\Phi \in V^*$:
	\begin{gather*}
		v \in V, v = \sum_{i \in I}\alpha_i v_i \quad \text{почти все $\alpha_i = 0$} \\
		\Phi(v) = \Phi \left(\sum_{i \in I}\alpha_i v_i\right) = \sum_{i \in I}\alpha_i\Phi_i(v_i)
	\end{gather*}
	Пусть $\dim V = n < \infty$. Зафиксируем базис $v_1, \dots, v_n$
\end{Rem}

$v_i^* \in V^*$
\begin{gather*}
	v = \sum_{i=1}^n \alpha_i v_i \\
	v_i^*(v) \lrh \alpha_i \\
	v_i^*(v_j) = \begin{cases}
		1 & i = j \\
		0 & i \ne j
	\end{cases}
\end{gather*}

\begin{theorem}
	$\dim V = n < \infty$.
	В указанных выше обозначениях $v_1^*, \dots, v_n^*$ "--- базис $V^*$.
\end{theorem}

\begin{proof}
	$\Phi \in V^*\colon V \to K$ "--- линейное отображение, $\beta_i = \Phi(v_i)$.
	Нужно проверить, что $\Phi = \sum_{i=1}^n \beta_iv_i^*$:
	\[ \forall v \in V, \Phi(v) = \left(\sum_{i = 1}^n \beta_iv_i^*\right)(v) \]
	Достаточно проверить это равенство для $v = v_1, \dots, v_n$. $\Phi(v_j) = \beta_j$:
	\begin{gather*}
		\left(\sum_{i=1}^n \beta_i v_i^*\right)(v_j) = \sum_{i=1}^n\beta_i v_i^*(v_j) = \beta_j \\
		\forall j=1..n, \Phi(v_j) = \left(\sum_{i=1}^n \beta_i v_i^*\right)(v_j) \Ra \\
		\Ra \Phi = \sum_{i = 1}^n\beta_i v_i^*
	\end{gather*}
	Получили, что $v_1^*, \cdots, v_n^*$ "--- семейство образующих.
	Необходимо проверить линейную независимость.
 	\begin{gather*}
		\sum_{i=1}^n \beta_iv_i^* = 0 \in V^* \\
		\forall j, \left(\sum_{i=1}^n \beta_i v_i^*\right)(v_j) = 0 \\
		\sum_{i=1}^n \beta_iv_i^*(v_j) = \beta_j
	\end{gather*}
	Откуда $\forall j, \beta_j = 0$ и $v_1^* \cdots v_n^*$ "--- линейно независимые.
\end{proof}

\begin{conseq}
	\[ \dim V < \infty \Ra V \simeq V^* \]
	Изоморфизм может быть построен следующим образом:
	$\Psi\colon V \to V^*$.
	$v_1, \dots, c_n$ "--- базис $V$.
	Отображение
	\[ \alpha_1 v_1 + \cdots + \alpha_n v_n \mapsto \alpha_1 v_1^* + \cdots \alpha_n v_n^* \]
	является изоморфизмом.
\end{conseq}

\begin{proof}
	$\dim (V^*) = n = \dim V \Ra V^* \cong V$.
	$\Psi$ "--- линейно.

	$\Phi \in V^*$, $\Phi$ "--- линейная комбинация $v_1^*, \dots, v_n^*$. Значит, $\Psi$ сюръективно.
	\[
		\Psi(\alpha_1 v_1 + \dots + \alpha_n v_n) = 0 \iff \alpha_1 v_1^* + \dots + \alpha_n v_n^* = 0 \Lra \alpha_1 = \dots = \alpha_n = 0
	\]
	$\ker(\Psi) = \{0\} \Ra \text{$\Psi$ инъективно}$.
	Таким образом, $\Psi$ "--- биекция.
\end{proof}
\begin{Rem}
	Построенный изоморфизм зависит от выбора базиса в $V$.
\end{Rem}

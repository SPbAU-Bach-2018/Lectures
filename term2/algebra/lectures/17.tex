\section{Двойственное пространство}

\begin{Def}
	$V$ "--- векторное пространство над $K$. Двойственное пространство
	\[ V^* = \{\Phi \mid \text{$\Phi\colon V \to K$ "--- линейное отображение}\} \]
	"--- множество линейных отображений из $V$ в $K$ (пространство над $K$).
\end{Def}
\begin{Def}
	Линейное отображение $f\colon V \ra K = K^1$ называется линейным функционалом.
	$V^*$ "--- пространство линейных функционалов на $V$.
\end{Def}

\begin{Rem}
	Всякий линейный функционал полностью определяется своими значениями на базисных элементах $V$.
	Пусть $\{v_i\}_{i \in I}$ "--- базис $V$, $\Phi \in V^*$:
	\begin{gather*}
		v \in V, v = \sum_{i \in I}\alpha_i v_i \quad \text{почти все $\alpha_i = 0$} \\
		\Phi(v) = \Phi \left(\sum_{i \in I}\alpha_i v_i\right) = \sum_{i \in I}\alpha_i\Phi_i(v_i)
	\end{gather*}
	Пусть $\dim V = n < \infty$. Зафиксируем базис $v_1, \dots, v_n$
\end{Rem}

$v_i^* \in V^*$
\begin{gather*}
	v = \sum_{i=1}^n \alpha_i v_i \\
	v_i^*(v) \lrh \alpha_i \\
	v_i^*(v_j) = \begin{cases}
		1 & i = j \\
		0 & i \ne j
	\end{cases}
\end{gather*}

\begin{theorem}
	$\dim V = n < \infty$.
	В указанных выше обозначениях $v_1^*, \dots, v_n^*$ "--- базис $V^*$.
\end{theorem}

\begin{proof}
	$\Phi \in V^*\colon V \to K$ "--- линейное отображение, $\beta_i = \Phi(v_i)$.
	Нужно проверить, что $\Phi = \sum_{i=1}^n \beta_iv_i^*$:
	\[ \forall v \in V, \Phi(v) = \left(\sum_{i = 1}^n \beta_iv_i^*\right)(v) \]
	Достаточно проверить это равенство для $v = v_1, \dots, v_n$. $\Phi(v_j) = \beta_j$:
	\begin{gather*}
		\left(\sum_{i=1}^n \beta_i v_i^*\right)(v_j) = \sum_{i=1}^n\beta_i v_i^*(v_j) = \beta_j \\
		\forall j=1..n, \Phi(v_j) = \left(\sum_{i=1}^n \beta_i v_i^*\right)(v_j) \Ra \\
		\Ra \Phi = \sum_{i = 1}^n\beta_i v_i^*
	\end{gather*}
	Получили, что $v_1^*, \cdots, v_n^*$ "--- семейство образующих.
	Необходимо проверить линейную независимость.
 	\begin{gather*}
		\sum_{i=1}^n \beta_iv_i^* = 0 \in V^* \\
		\forall j, \left(\sum_{i=1}^n \beta_i v_i^*\right)(v_j) = 0 \\
		\sum_{i=1}^n \beta_iv_i^*(v_j) = \beta_j
	\end{gather*}
	Откуда $\forall j, \beta_j = 0$ и $v_1^* \cdots v_n^*$ "--- линейно независимые.
\end{proof}

\begin{conseq}
	\[ \dim V < \infty \Ra V \simeq V^* \]
	Изоморфизм может быть построен следующим образом:
	$\Psi\colon V \to V^*$.
	$v_1, \dots, c_n$ "--- базис $V$.
	Отображение
	\[ \alpha_1 v_1 + \cdots + \alpha_n v_n \mapsto \alpha_1 v_1^* + \cdots \alpha_n v_n^* \]
	является изоморфизмом.
\end{conseq}

\begin{proof}
	$\dim (V^*) = n = \dim V \Ra V^* \cong V$.
	$\Psi$ "--- линейно.

	$\Phi \in V^*$, $\Phi$ "--- линейная комбинация $v_1^*, \dots, v_n^*$. Значит, $\Psi$ сюръективно.
	\[
		\Psi(\alpha_1 v_1 + \dots + \alpha_n v_n) = 0 \iff \alpha_1 v_1^* + \dots + \alpha_n v_n^* = 0 \Lra \alpha_1 = \dots = \alpha_n = 0
	\]
	$\ker(\Psi) = \{0\} \Ra \text{$\Psi$ инъективно}$.
	Таким образом, $\Psi$ "--- биекция.
\end{proof}
\begin{Rem}
	Построенный изоморфизм зависит от выбора базиса в $V$.
\end{Rem}

\begin{Rem}
$V = K^n$, тогда всякий $\phi \in V^{*}$ 
может быть представлен в виде $\phi(v) = uv$, где 
$u$ "--- фиксированная строка $u = (u_1, \cdots, u_n)$.

$$
\Phi\begin{pmatrix}
v_1\\
v_2\\
\vdots\\
v_n\\
\end{pmatrix} = u_1v_1 + \cdots + u_nv_n
$$

Базис V = $\{e_1, \cdots, e_n\}$ "--- стандартный базис.

$\phi = \sum_{i = 1}^{n}u_ie_i^{*}, u = (u_1, \cdots, u_n)$.
\end{Rem}

\subsection{Второе двойственное пространство}
$V, V^*, V^{**}$

$$V \to V^{**}, u \to \phi_u (\phi_u \in V^{**})$$

Определим $\phi_u$ на произвольном элементе $\phi_u(f) = f(u), (f \in V^*)$

\begin{enumerate}
\item $\phi_u$ "--- линейный функционал на $V^*$.
$$\phi_u(\alpha_1f_1 + \alpha_2f_2) = (\alpha_1f_1 + \alpha_2f_2)(u) = $$
$$= \alpha_1f_1(u) + \alpha_2f_2(u) = \alpha_1\phi_u(f_1) + \alpha_2\phi_u(f_2)$$
\item $u \to \phi_u$ линейное отображение
$$\beta_1u_1 + \beta_2u_2 \to \phi_{\beta_1u_1 + \beta_2u_2}$$

$$\phi_{\beta_1u_1 + \beta_2u_2}(f) = f(\beta_1u_1 + \beta_2u_2) = $$
$$= \beta_1f(u_1) + \beta_2f(u_2) = \beta_1 \phi_{u_1}(f) + \beta_2\phi_{u_2}(f)$$

Построили гомоморфизм из $V$  в $V^{**}$, $u \to \phi_u$
\item Инъективность. Достаточно проверить, что ядро тривиально. 

Для каких $u$ $\phi_u = 0 \in V^{**}$?
$$\forall f \in V^* \phi_u(f) = 0 \Ra f(u) = 0 $$

Если $u \ne 0$, то можно дополнить $\{u\}$ до базиса $V$.

Рассмотрим $f$: $f(u) = 1$, $f(u_j) = 0$ для остальных
векторов из этого базиса и доопределим на всем $V$ по линейности.

$\phi_u(f) = f(u) = 1 \ne 0$

Если $u \ne 0$, то и $\phi_u \ne 0$.

Значит, ядро отображения $u \to \phi_u$ состоит только из 0, 
то есть имеется инъективный гомоморфизм из $V$ в $V^{**}$.
\item Пусть $\dim(V) = n < \infty$

$\dim(V^{*}) = n, \dim(V^{**}) = n$

$V \to V^{**}(u \to \phi_u)$ "--- инъективный гомоморфизм.

$\Ra \dim$ образа есть n,  то есть это отображение на все $V^{**}$,
значит это сюръекция, то есть это изоморфизм.  
\end{enumerate}

\begin{exmp}
$V$ "--- пространство финитных(начиная с какого-то места только 0) над
полем K.
$$V = \{(a_1, a_2, \cdots) \colon \text{почти все } a_i = 0\}$$

$B = (b_1, b_2, \cdots)$ "--- бесконечная последовательность.
$$f_B\colon V \to K$$
$$(a_1, a_2, a_3, \cdots) \to f_B(a_1, a_2, \cdot) = \sum_{i = 1}^{\infty}a_ib_i \in K$$

$f_B$ корректно определено на множестве финитных последовательностей(так как любая сумма превращается
в конечную).

$f_B$ линейно:
$$f_B(\alpha(a_1, a_2, \cdots) + \alpha'(a_1', a_2', \cdots)) = 
\alpha f_{B}(a_1, a_2, \cdots) + \alpha' f_B(a_1', a_2', \cdots)$$

$f_B \in V^{*}$ "--- линейное отображение со значением в поле K.

Если $B \ne B'$, то $f_B \ne f_{B'}$

Зафиксируем координату, по которой они отличаются. Пусть $b_i \ne b'_i$

$$f_B(0, \cdots, 1, \cdots) = b_i \ne b_i' = f_B'(0, \cdots, 1, \cdots)$$
\end{exmp}

\textbf{Рассмотрим случай, когда $K$ "--- конечное или счетное}

$V^*$ гораздо больше $V$.

$K$ "--- $K$ конечно или счетно, то $V$ "--- счетно.  (Представим
$V$ в виде счетного объединения конечной степини $K$).

Мощность $V^{*}$ не меньше, чем мощность всех
бесконечных последовательностей из 0 и 1(из $K$)
$\Ra$ не меньше, чем мощность множества подмножеств в $\N$ 
$\Ra$ не меньше мощности континуума.

То есть $V^*$ не счетно(даже нет счетного базиса).

\textbf{Итог:} в случае $K$ "--- конечное или счтное

$V$ "--- счетное.

$V^*$ "--- не счетное(даже нет счетного базиса).


Линейные функционалы могут иметь достаточно хитрую природу
об этом следующи примеры.

\begin{exmp}
$V = C([a, b] \to \R)$

\begin{enumerate}
\item Интегральные функционалы:
$f \to \int_a^b f(x) dx (f \in V)$
\item Весовая функция $\rho$
$f \to \int_{a}^{b}f(x)\rho(x)dx$

Мера определения весовой функции.
\item $f \to \int_a^{b}f(x) d \mu$ 

$\mu$ "--- мера на отрезке $[a, b]$.

\item  Мера сконцентрирована в 1 точке.

$\mu(с) = 1, \mu(x) = 0$ "--- во всех остальных точках.

$f \to f(c) = \delta_c(f)$

$v = C'(\R \to \R)$
\item $f \to f(a) + f'(b)$
\end{enumerate}
\end{exmp}
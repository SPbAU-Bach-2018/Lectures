\section{Прямая сумма векторных подпространств}

\begin{Def}
	$V$ "--- векторное пространство над $K$, $U, W \subset V$.
	Сумма $U+W$ называется прямой суммой, если $U \cap W = \{0\}$
	\[ U \oplus W \]
\end{Def}

\begin{theorem}
	Если пространство конечномерно, то $U + W$ "--- прямая сумма тогда и только тогда, когда $\dim(U + W) = \dim(U) + \dim(W)$.

	Сумма $U + W$ прямая тогда и только тогда, когда всякий $v \in U + W$ единственным образом представляется в виде $v = u + w$.
\end{theorem}

\begin{proof}
	\begin{description}
	\item[$\Ra$:]
		$v \in U + W$, $v = u + w = u' + w'$ ($u, u' \in U$, $w, w' \in W$).
		\begin{gather*}
			u - u' = w - w'\\
			\left\{\begin{aligned}
				u - u' &\in U \\
				w - w' &\in W
			\end{aligned}\right. \Ra \left\{\begin{aligned}
				(u - u') &\in (U \cap W) \\
				(w - w') &\in (U \cap W)
			\end{aligned}\right. \Ra \left\{\begin{aligned}
				u &= u' \\
				w &= w'
			\end{aligned}\right.
		\end{gather*}

	\item[$\La$:]
		$v \in U \cap W$.
		\[ v = (v \in U) + (0 \in W) = (0 \in U) + (v \in W) \]
		В силу единственности $v = 0$, значит пересечение тривиально.
	\end{description}
\end{proof}

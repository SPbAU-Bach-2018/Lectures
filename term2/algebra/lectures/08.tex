\setauthor{Маркелов Саша}
\section{Системы линейных уравнений}
\begin{Def}
$n$ уравнений с $m$ неизвестными. $K$ - поле.
\[ \left\{\begin{array}{ll}
\sum_{j=1}^{m}a_{i_j}x_j=b_i, &    i = 1 \dots n \\
\dots
\end{array} \right. \]
$$a_{i_j}, b_i \in K$$
\end{Def}
\begin{Def}
Более общая концепция есть в коммутативном кольце с 1
\end{Def}
\begin{Def}
Система "--- однородная , если все $b_i=0$.
\end{Def}
\begin{Def}
Eсли у системы есть решения, то она "--- совместна, иначе "--- несовместна
\end{Def}
\begin{Def}
Матричный вид
\[\begin{aligned}
A&=a_{i_j} (i=1 \dots n, j=1 \dots m) \text{  матрица коэффициентов}\\
X&= \begin{pmatrix}
				x_1\\
				\vdots\\
				x_m\\
			\end{pmatrix} \\
B&= \begin{pmatrix}
				b_1\\
				\vdots\\
				b_n\\
			\end{pmatrix} \\
A&X=B\\
(A&|B) \text{ расширенная матрица системы}
\end{aligned}\]
\end{Def}
\begin{Def}
Элементарные преобразования

\begin{enumerate}
	\item $i \text{ уравнение += }  j \text{ уравнение} \cdot c (\in K)$
	\item $i \text{ уравнение = }  i \text{ уравнение} \cdot c \neq 0(\in K)$
	\item Поменять местами $i$ уравнение и $j$ уравнение
\end{enumerate}
\end{Def}
\begin{Rem}
Во 2 пункте в общем случае $c \in R^*$
\end{Rem}
\begin{Rem}
Преобразование 3 типа может быть выражено через первые два

\begin{enumerate}
	\item $i \text{ += }  j$
	\item $j \text{ += }  (-i)$
	\item $i \text{ += }  j$
	\item $j = -j$
\end{enumerate}
\end{Rem}
\begin{Rem}
Система не менеят множества решений, так как новая система "--- следствие старой и есть обратное преобразование
\end{Rem}
\begin{Def}
Обратное преобразование

\begin{enumerate}
	\item $i \text{ уравнение += }  j \text{ уравнение} \cdot (-c) (\in K)$
	\item $i \text{ уравнение = }  i \text{ уравнение} \cdot c^{-1} (\in K)$
	\item Поменять местами $i$ уравнение и $j$ уравнение
\end{enumerate}
\end{Def}
\begin{Def}
Элементарные преобразования в матричной форме

\begin{enumerate}
	\item умножить слева на \[
			\begin{pmatrix}
				1     & \cdots & 0 & \cdots & 0      \\
				\vdots & \ddots & \vdots & c_{ij} & \vdots \\
				0      & \cdots & 1 & \cdots & 0      \\
				\vdots & \ddots & \vdots & \ddots & \vdots \\
				0      & \cdots & 0      & \cdots & 1
			\end{pmatrix}
		\]
	\item умножить слева на \[
			\begin{pmatrix}
				1     & \cdots & 0 & \cdots & 0      \\
				\vdots & \ddots & \vdots & \ddots & \vdots \\
				0      & \cdots & c_{ii} & \cdots & 0      \\
				\vdots & \ddots & \vdots & \ddots & \vdots \\
				0      & \cdots & 0      & \cdots & 1
			\end{pmatrix}
		\]
	\item умножить слева на \[
			\begin{pmatrix}
				1      & \cdots & 0     & \cdots & 0      & \cdots  & 0 \\
				\vdots & \ddots & \vdots & \ddots & \vdots & \ddots  & \vdots \\
				0    & \cdots & 0_{ii} & \cdots & 1_{ij}     & \cdots  & 0 \\
				\vdots & \ddots & \vdots & \ddots & \vdots & \ddots  & \vdots\\
				0      & \cdots & 1_{ji} & \cdots & 0_{jj} & \cdots  & 0 \\
				\vdots & \ddots & \vdots & \ddots & \vdots & \ddots  & \vdots\\
				0      & \cdots & 0      & \cdots & 0 	   & \cdots  & 1
			\end{pmatrix}
		\]
\end{enumerate}
\end{Def}
\begin{Def}
Ступенчатый вид матрицы

\begin{enumerate}
\item все ненулевые строки (имеющие по крайней мере один ненулевой элемент) располагаются над всеми чисто нулевыми строками;
\item   ведущий элемент (первый ненулевой элемент строки при отсчёте слева направо) каждой ненулевой строки располагается строго правее ведущего элемента в строке, расположенной выше данной.
\end{enumerate}
\end{Def}
\begin{Def}
Приведенный ступенчатый вид матрицы "--- такой ступенчатый вид, корый удовлетворяет дополнительному условию, что каждый ведущий элемент ненулевой строки - это единица, и он является единственным ненулевым элементом в своём столбце.
\end{Def}
\begin{Def}
Приведение матрицы к ступенчатому виду методом Гаусса 
\begin{enumerate}
\item Прямой ход
\begin{enumerate}
\item Идём по столбцам слева направо. 
\item Находим в текущем столбце ненулевой коэффициент( пусть он находится в строчке $i$ ), если такого не находится, то переходим к следующему столбцу и итерации
\item Меняем местами $i$ строчку и 1 строчку
\item Пусть мы находимся сейчас на $j$ столбце, тогда из каждой $k$ строки вычитаем первую строку с коэффициентом $\frac{a_{kj}}{a_{1j}}$. 
\item В точке $(1;j)$ получили начало новой ступеньки
\item Мысленно отбрасываем первую строку и столбец, переходим к подматрице и следующей итерации
\end{enumerate}
\item Проверка на совместность системы : пусть после $i$ строчки ступеньки заканчиваются. Тогда в строчках $j>i$ имеем уравнения вида $0=b_j$, если хотя бы одно $b_j$ не равно 0, то система несовместна. 
\item Обратный ход: для каждой $(i;j)$ позиции начала ступеньки из каждой $k < i$ строки вычитаем $i$ строку с коэффициентом $\frac{a_{kj}}{a_{ij}}$
\item Получение приведенного вида : для каждой $(i;j)$ позиции начала ступеньки делим строку $i$ на коэффициент в $(i;j)$ позиции в матрице.
\item Получение решений: пусть после $i$ строчки ступеньки заканчиваются. Тогда имеем $i$ уравнений вида $x_j=b_k$ ( для зависимых переменных) и $m-i$ свободных переменных - им мы имеем право придавать любое значение. 
\end{enumerate}
\begin{Rem}
Если работаем над эвклидовым кольцом, то при выборе ненулевого коэффициента в очередном столбце выбираем такой с наименьшим значением эвклидовой функции среди подходящих.
\end{Rem}
\end{Def}




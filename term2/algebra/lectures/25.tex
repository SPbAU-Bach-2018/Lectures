\section{Процесс ортогонализации}
\begin{Def}
$B$ "--- полуторалинейная(билинейная) форма на V.

$v_1, \cdots, v_n$ пространства $V$ называется ортогональным базисом, если 
$B(v_i, v_j) = 0$ для всех $i \ne j$.

Базис ортонормирован, если он ортогонален и $B(v_i, v_i) = 1$ для всех i.
\end{Def}
\begin{Def}
 $$x \bot y $$
 $x, y \in V$ ортогональны, если $B(x, y) = 0$ 
 
\end{Def}
\begin{Rem}
 B "--- эрмитово симметрична(симметрична, кососимметрична).
 $$x \bot y \Ra y \bot x$$
\end{Rem}
\begin{Rem}
 Не для всякой билинейной формы(даже невырожденной) существует 
 ортогональный базис.
\end{Rem}
\begin{exmp}
B "--- кососимметрична, невырожденная. 

$v_1, \cdots, v_n$ "--- ортогональный базис?
$$v_1 \bot v_2, \cdots v_1 \bot v_n$$
Но и $v_1 \bot v_1 \Ra v_2$ ортогонален $\forall$ вектору
из V $\forall i B(v_1, v_i) = 0$

$\forall y \in V, B(v_1, y) = 0$ противоречие с невырожденностью.
\end{exmp}
\begin{theorem}{}
$K = \C$(или $\R$) $V, \dim V < \infty$.
B "--- эрмитова симметричная положительно определенная.

Тогда существует ортогональный базис.
\end{theorem}
\begin{proof}
 \begin{description}
  \item[Шаг 1]  Строим ортогоналный базис.

  Пусть $v_1, \cdots, v_n$ "--- базис V.

  Найдем новый базис $u_1, \cdots, u_n$
  \begin{enumerate}
  \item $<u_1, \cdots, u_i> = <v_1,\cdots, v_i>, i = 1, \cdots, n$
  \item $u_i \bot u_j, i \ne j$

  $u_1 = v_1$

  Предположим, что построили $u_1, \cdots, u_i$
  $<u_1, \cdots, u_i> = <v_1, \vdots, v_i>$ и
  добавляем $u_{i + 1}$ в виде линейной комбинации $v_{i + 1}$ и $u_1, \cdots, u_i$
  $$u_{i + 1} = v_{i + 1} + \sum_{j = 1}^{i} \alpha_{j}u_j$$
  $$<u_1, \cdots, u_{i + 1}> = <v_1, \cdots, v_{i + 1})> $$

  Условие ортогональности: $u_{i + 1} \bot u_1 \cdots u_i$
  $$0 = B(u_{i + 1}, \cdots, u_k) = B(v_{i + 1}, u_k) + \sum_{j = 1} \alpha_j B(u_j, u_k) = $$
  $$= B(v_{i + 1}, u_k)  + \alpha_{k}B(u_k, u_k)$$
  $B(u_k, u_k) > 0$ в силу положительной определенности $u_k \ne 0$(так как $u_1, \cdots u_k$ "--- линейно независимые).
  
  Нашли $u_{i + 1}$
  \item[Шаг 2] Нормируем 
  $u_1, \cdots, u_n$ "--- ортогональный базис.

  $$w_i = \lambda_i u_i$$
  
  $$1 = B(w_i, w_i) = \lambda_i \overline{\lambda_i}B(u_i, u_i) $$
  $|\lambda_i|^2 = \frac{1}{B(u_i, u_i)} > 0$ В силу положительной определенности.
  $$\lambda_i = \frac{1}{\sqrt{B(u_i, u_i)}}$$

  Процесс ортогонализации Грама-Шмидта.
  \end{enumerate}
 \end{description}
\end{proof}
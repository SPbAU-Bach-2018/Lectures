\setauthor{Ольга Черникова}

\section{Базис векторного пространства}

\begin{Def}
	Базис векторного пространства "--- линейно независимая система образующих.
\end{Def}

\begin{theorem}
	$V$ "--- векторное пространство над полем $K$, $\{v_{\alpha}\}_{\alpha \in I} \subset V$.
	Следующие условия равносильны:
	\begin{enumerate}
	\item
		$\{v_{\alpha}\}$ "--- базис $V$.

	\item
		$\{v_{\alpha}\}$ "--- максимальное по включению линейно независимое семейство.

	\item
		$\{v_{\alpha}\}$ "--- минимальное по включению семейство образующих

	\item
		Всякий вектор $v \in V$ единственным образом представляется в виде конечной
		линейной комбинации векторов из $\{v_{\alpha}\}$

	\item
		Всякий $v \in V$ раскладывается в виде конечной линейной комбинации векторов из
		$\{v_{\alpha}\}$ и нулевой вектор раскладывается единственным образом (то есть с нулевыми коэффициентами).
	\end{enumerate}
\end{theorem}
\begin{proof}
	\begin{description}
	\item [1 $\Ra$ 2:]
		Так как $\{v_{\alpha}\}_{\alpha \in I}$ "--- базис, то это линейно независимое семейство.
		Рассмотрим $v \in V$ и $\{v_{\alpha}\}_{\alpha \in I} \cup \{v\}$.
		Надо проверить, что это семейство линейно зависимо $\forall v$.
		\[ v = \sum_{\alpha \in I} a_{\alpha}v_{\alpha} \]
		и почти все коэффициенты $a_{\alpha} = 0$ (так как $\{v_{\alpha}\}_{\alpha \in I}$ "--- базис).
		\[ 0 = -v + \sum_{\alpha \in I} a_{\alpha}v_{\alpha} \]
		Нетривиальная линейная комбинация и почти все коэффициенты равны нулю, поэтому вектора $\{v\} \cup \{v_{\alpha}\}_{\alpha \in I}$ "--- линейно зависимы.

	\item [2 $\Ra$ 1:]
		$\{v_{\alpha}\}$ "--- максимальное линейно независимое семейство. \\
		Необходимо проверить, что $\{v_{\alpha}\}$ "--- семейство образующих.
		$v \in V$, $\{v\} \cup \{v_{\alpha}\}$ "--- линейно зависимы, поэтому существует линейная комбинация, равная нулю,
		и $v$ входит в нее с ненулевым коэффициентом.
		\[ av + \sum_{\alpha \in I}a_{\alpha}v_{\alpha} = 0 \]
		и почти все $a_{\alpha} = 0$, $a \ne 0$.
		\[ v = \sum_{\alpha \in I}(-a^{-1}a_{\alpha})v_{\alpha} \]
		В силу произвольности $v$ $\{v_{\alpha}\}$ "--- семейство образующих, следовательно, базис.

	\item [1 $\Ra$ 3:]
		${v_{\alpha}}$ "--- базис, следовательно, семейство образующих.
		Необходимо проверить, что ономинимальное по включению.

		$v \in \{v_{\alpha}\}, v = v_{\alpha_0}, \alpha_0 \in I$.
		Пусть
		\[ \{v_{\alpha}\}_{\alpha \in I \setminus \{\alpha_0\}} \]
		"--- семейство образующих. Тогда
		\[ v_{\alpha_0} = \sum_{\alpha \ne \alpha_0} a_{\alpha}v_{\alpha} \]
		Получили
		\[ 0 = -v_{\alpha_0} + \sum_{\alpha \ne \alpha_0}a_{\alpha} v_{\alpha} \]
		и почти все $a_{\alpha} = 0$.
		Это противоречие с линейной независимостью ${v_\alpha}_{{\alpha} \in I}$.
		Тогда $\{v_{\alpha}\}$ "--- минимальное по включению.

	\item [3 $\Ra$ 1:]
		$\{v_{\alpha}\}$ "--- минимальное по включению семейство образующих.
		Хотим проверить, что $\{v_{\alpha}\}$ "--- линейно независимые.

		Пусть линейно зависимые. Тогда $\sum_{\alpha \in I} a_{\alpha}v_{\alpha} = 0$, почти все, но не все $a_{\alpha} = 0$.
		$\exists \alpha_0: a_{\alpha} \ne 0$. Тогда
		$v_{\alpha_0}$ "--- (конечная) линейная комбинация остальных векторов, поэтому
		\[ \{v_{\alpha}\}_{\alpha \in I \setminus {\alpha_0}} \]
		"--- тоже семейство образующих.
		\[ v_{\alpha_0} = \sum_{\alpha \in I \ne \alpha_0} a_{\alpha}v_{\alpha} \]
		почти все $a_{\alpha} = 0$.
		$u \in V$,
		\[ u = \sum_{\alpha \in I}b_{\alpha}v_{\alpha} \]
		почти все $b_{\alpha} = 0$.
		\begin{alignat*}{2}
			u &= \sum_{\alpha \ne \alpha_{0}}b_{\alpha}v_{\alpha} + b_{\alpha_0}v_{\alpha_0} = \\
			  &= \sum_{\alpha \ne \alpha_{0}}b_{\alpha}v_{\alpha} + \sum_{\alpha \ne \alpha_{0}}b_{\alpha_0}a_{\alpha}v_{\alpha} = \\
			  &= \sum _{\alpha \ne \alpha_0}(b_{\alpha + b_{\alpha_0}a_{\alpha}})v_{\alpha}
		\end{alignat*}
		Почти все коэффициенты равны 0, противоречие с минимальностью. Значит $\{v_{\alpha}\}$ "--- линейно независимы, значит базис.

	\item [5 "--- переформулировка пункта 1.]
		Рассмотрим $\{v_{\alpha}\}$.
		Утверждение, что всякий $v \in V$ "--- (конечная) линейная комбинация $\{v_{\alpha}\}_{\alpha \in I}$, равносильно
		утверждению, что $\{v_{\alpha}\}$ "--- семейство образующих.

		Утыерждение, что $0_v$ представим только в виде тривиальной линейной комбинации $\{v_{\alpha}\}$ равносильно тому,
		что $\{v_{\alpha}\}$ "--- линейно независимы.

	\item [4 $\Ra$ 5:]
		4 "--- более сильное условие.

	\item [5 $\Ra$ 4:]
		$v \in V$
		Пусть
		\[ v = \sum_{\alpha \in I}a_{\alpha}v_{\alpha} = \sum_{\alpha \in I} b_{\alpha}v_{\alpha} \]
		почти все $a_{\alpha}$, $b_{\alpha}$ равны 0. Тогда:
		\[ \sum_{\alpha \in I}(a_{\alpha} - b_{\alpha})v_{\alpha} = 0 \]
		почти все $a_{\alpha} - b_{\alpha} = 0$.
		Тогда $\forall \alpha, a_\alpha = b_\alpha$.
	\end{description}
\end{proof}

\begin{exmp}
	\begin{description}
	\item[Базис в $K^n$:]
		\[
			e_i =
			\begin{pmatrix}
				0\\
				0\\
				\vdots\\
				1\\
				0\\
			\end{pmatrix}
		\]
		Единичный вектор "--- на $i$-й строчке стоит 1, все остальные 0.
		\[
			\sum_{i = 1}^{n}a_ie_i =
			\begin{pmatrix}
				a_1\\
				\vdots\\
				a_n\\
			\end{pmatrix}
		\]
		$e_1, e_2, \dots, e_n$ "--- стандартный базис.

	\item[Базис в $K^3$:]
		\[
			\begin{pmatrix}
				1\\1\\1\\
			\end{pmatrix}
			,
			\begin{pmatrix}
				1\\1\\0\\
			\end{pmatrix}
			,
			\begin{pmatrix}
				1\\0\\0\\
			\end{pmatrix}
		\]
		Так же базис в $K^3$.

	\item[$M(n \times m, K)$:]
		Стандартные матричные единицы.
		\[
			e_{ij} =
			\begin{pmatrix}
				0      & \cdots & 0      & \cdots & 0      \\
				\vdots & \ddots & \vdots & \ddots & \vdots \\
				0      & \cdots & 1_{ij} & \cdots & 0      \\
				\vdots & \ddots & \vdots & \ddots & \vdots \\
				0      & \cdots & 0      & \cdots & 0
			\end{pmatrix}
		\]

	\item[В {$K[x]$}:]
		$\{1, x, x^2, \cdots \}$ "--- базис.

		\textbf{Упражнение:}
			\[ P_i(x) \in K[x]; i = 0, 1, \dots; \deg P_i = i \]
			Докажите, что $\{P_i\}$ "--- базис $K[x]$ над $K$.
	\end{description}
\end{exmp}

\begin{theorem}
	$V$ "--- векторное пространство над $K$.
	Тогда в $V$ есть базис.
\end{theorem}

\begin{Def}
	$V$ "--- конечнопорожденное(конечномерное) если в нем есть
	конечная система образующих.
\end{Def}
\begin{proof}
	Докажем теорему для конечнопорожденных пространств.

	\begin{lemma}
		$V$ "--- конечнопорожденное векторное пространство.
		Всякое линейно независимое семейство может быть дополнено до базиса.
	\end{lemma}
	\begin{proof}
		Пусть $V = \left<u_1, u_2, \dots, u_m\right>$, $V$ порожден данными векторами, то есть это система образующих.
		$\{v_{\alpha}\}$ ~--- линейно независимо.
		Попробуем дополнить $\{v_{\alpha}\}$ до базиса.

		Пусть существует $v \in V \setminus \{v_{\alpha} \mid \alpha \in I\}$, то есть существует вектор,
		который нельзя получить с помощью нашего множества.
		Добавим его в множество и оно останется линейно независимым.
		$\{v\} \cup \{v_{\alpha}\}_{\alpha \in I}$.

		Число векторов в линейно независимом семействе не более $m$.
		Все $v$ являются линейными комбинациями $u_1, \dots, u_m$.
		Если число векторов больше $m$, то они линейно зависимы (теорема о линейной зависимости линейной комбинации).
		Таким образом процесс оборвется за конечное число шагов.
	\end{proof}

	Первое доказательство.
	Возьмем любое линейно независимое множество и дополним его до базиса.
\end{proof}

\begin{proof}
	\begin{lemma}
		$V$ "--- конечнопорожденное векторное пространство.
		Из всякой конечной системы образующих можно выбрать базис.
	\end{lemma}

	\begin{proof}
		$u_1, \dots, u_m$ "--- семейство образующих.

		Если они линейно независимые, то базис найден.
		Если линейно зависимы, то один из них есть линейная комбинация остальных.
		Значит $\{u_1, \dots, u_m\} \setminus \{u_j\}$ "--- тоже семейство образующих.
		Так как исходное семейство конечно, то процесс оборвется. Значит найдем базис.
	\end{proof}

	Второе доказательство.
	Есть семейство образующих, применим лемму.
\end{proof}

\begin{exmp}
	$\R$ "--- векторное пространство над $\Q$.
	Базис $\R$ над $\Q$ есть, но уже несчетный.
\end{exmp}

\begin{theorem}
	Любые два базиса конечнопорожденного векторного пространства содержат одно и то же
	число элементов.
\end{theorem}

\begin{proof}
	Из теоремы о линейной зависимости линейных комбинаций следует, что всякий базис конечен.
	$V$ "--- векторное пространство.
	$u_1, \dots, u_m$ и $v_1, \dots, v_n$ "--- базисы $V$.

	$u_1, \dots, u_m$ "--- образующие $V$, $v_1, \dots, v_n$ "--- линейно независимые и линейные комбинации векторов $u_1, \dots, u_m$.
	По теореме о линейной зависимости линейных комбинаций $n \le m$, аналогично $m \le n$.
	Значит $n = m$.
\end{proof}

\begin{Def}
	Если $V$ "--- конечнопорожденное векторное пространство, то число векторов в базисе
	называется размерностью $V$ над $K$.
	\[\dim  \quad \dim_K V \]
	Если векторное пространство не является конечнопорожденным, то $\dim V = \infty$.
\end{Def}

\begin{exmp}
	\begin{enumerate}
		\item $\dim_K K^n = n$
		\item $\dim_K M(n \times m, K) = nm$
		\item $\dim_K K[x] = \infty$
		\item $\dim_{K}\{g \in K[x] \mid \deg g \le n \in \N_{0}\} = n + 1$
		\item $\dim_{\R}\C = 2$ ($1, i$)
		\item $\dim_{\R}\H = 4$ ($1, \mathbb{i}, \mathbb{j}, \mathbb{k}$)
		\item $\dim_{\C}\H = 2$ $(1, \mathbb{j})$
	\end{enumerate}
\end{exmp}

$AX = 0$, множество решений "--- векторное пространство.
Размерность пространства решений равно числу свободных переменных.

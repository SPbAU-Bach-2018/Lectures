\setauthor{Черникова Ольга}

\section{Базис векторного пространства}

\begin{Def}                          
    Базис векторного пространства "--- линейно независимая система образующих.
\end{Def}

\begin{theorem}
Теорема V "--- векторное пространство над  полем K:
$\{v_{\alpha}\}_{\alpha \in I} \subset V$

Следующие условия равносильны: 

\begin{enumerate}
\item $\{v_{\alpha}\}$ "--- базис  V.
\item $\{v_{\alpha}\}$ "--- максимальное по включению линейно независимое семейство.
\item $\{v_{\alpha}\}$ "--- минимальное по включению семейство образующих
\item Всякий вектор $v \in V$  единственным образом представляется в виде конечной
линейной комбинации векторов из $\{v_{\alpha}\}$
\item Всякий $v \in V$ раскладывается в виде конечной линейной комбинации векторов из 
$\{v_{\alpha}\}$ и нулевой вектор 
раскладывается единственным образом (то есть с нулевыми коэффициентами).
\end{enumerate}
\end{theorem}
\begin{proof}
\begin{description}
\item [1 $\Ra$ 2:]
    
    Базис $\Ra$ линейно независимое семейство. 

    $v \in V$

    $\{v_{\alpha}\}_{\alpha \in I} \cup \{v\}$. Надо проверить, что это
    семейство линейно зависимо $\forall v$. 
    
    $v = \sum_{\alpha \in I} a_{\alpha}v_{\alpha}$ и почти все 
    коэффициенты $a_{\alpha} = 0$(так как $\{v_{\alpha}\}_{\alpha \in I}$ "--- базис).
    
    $0 = -v + \sum_{\alpha \in I} a_{\alpha}v_{\alpha}$
    Нетривиальная линейная комбинация и почти все коэффициенты = 0 $\Ra$ вектора $\{v\} \cup \{v_{\alpha}\}_{\alpha \in I}$ "--- линейно зависимы.

\item [2 $\Ra$  1:] 
$\{v_{\alpha}\}$ "--- максимальное линейно независимое семейство.

Необходимо проверить, что $\{v_{\alpha}\}$ "--- семейство образующих.

$v \in V$, $\{v\} \cup \{v_{\alpha}\}$ --- линейно зависимы
$\Ra \exists$ линейная комбинация = 0 и v входит в нее с ненулевым коэффициентом.

$av + \sum_{\alpha \in I}a_{\alpha}v_{\alpha} = 0$  и почти все $a_{\alpha} = 0$, $a \ne 0$.

$v = \sum_{\alpha \in I}(-a^{-1}a_{\alpha})v_{\alpha}$

В силу произвольности v ${v_{\alpha}}$ "--- семейство образующих $\Ra$ базис.

\item [1 $\Ra$ 3:]

${v_{\alpha}}$ "--- базис $\Ra$ семейство образующих.
Необходимо проверить, что минимальное по включению.

$v \in \{v_{\alpha}\}, v = v_{\alpha_0}, \alpha_0 \in I$

Пусть $\{v_{\alpha}\}_{\alpha \in I \setminus \{\alpha_0\}}$ "--- семейство образующих
$\Ra v_{\alpha_0} = \sum_{\alpha \ne \alpha_0} a_{\alpha}v_{\alpha}$

$0 = -v_{\alpha_0} + \sum_{\alpha \ne \alpha_0}a_{\alpha} v_{\alpha}$ 
и почти все $a_{\alpha} = 0$.  
Это противоречие с линейной независимостью ${v_\alpha}_{{\alpha} \in I}$
$\Ra \{v_{\alpha}\}$ "--- минимальное по включению. 

\item [3 $\Ra$ 1:] 

$\{v_{\alpha}\}$  "--- минимальное по включению семейство образующих. 

Хотим проверить, что $\{v_{\alpha}\}$ "--- линейно независимые.

Пусть линейно зависимые. $\Ra$
$\sum_{\alpha \in I} a_{\alpha}v_{\alpha} = 0$, почти все
$a_{\alpha} = 0$, не все $a_{\alpha} = 0$

$\exists \alpha_0: a_{\alpha} \ne 0$

$v_{\alpha_0}$ "--- (конечная) линейная комбинация остальных векторов 
$\Ra \{v_{\alpha}\}_{\alpha \in I \setminus {\alpha_0}}$ 
"--- тоже семейство образующих.

$v_{\alpha_0} = \sum_{\alpha \in I \ne \alpha_0} a_{\alpha}v_{\alpha}$,
почти все $a_{\alpha} = 0$ 

$u \in V, u = \sum_{\alpha \in I}b_{\alpha}v_{\alpha}$, 
почти все $b_{\alpha} = 0$

$= \sum_{\alpha \ne \alpha_{0}}b_{\alpha}v_{\alpha} + b_{\alpha_0}v_{\alpha_0} =$
$= \sum_{\alpha \ne \alpha_{0}}b_{\alpha}v_{\alpha} + \sum_{\alpha \ne \alpha_{0}}b_{\alpha_0}a_{\alpha}v_{\alpha}$ 
$= \sum _{\alpha \ne \alpha_0}(b_{\alpha + b_{\alpha_0}a_{\alpha}})v_{\alpha}$
Почти все коэффициенты равны 0, Противоречие с минимальностью. $\Ra {v_{\alpha}}$ - линейно независимы $\Ra$ базис.

\item [5 "--- переформулировка пункта 1.]

${v_{\alpha}}$

Всякий $v \in V$ "--- (конечная) линейная комбинация $\{v_{\alpha}\}_{\alpha \in I} \Lra$ 
$\{v_{\alpha}\}$ "--- семейство образующих.

$0_v$ представим только в виде тривиальной линейной комбинации $\{v_{\alpha}\}  \Lra \{v_{\alpha}\}$ "--- линейно независимы.

\item [4 $\Ra$ 5:]
4 "--- более сильное условие.

\item [5 $\Ra$ 4:]

$v \in V$

Пусть $v = \sum_{\alpha \in I}a_{\alpha}v_{\alpha} = \sum_{\alpha \in I} b_{\alpha}v_{\alpha}$
почти все $a_{\alpha}$, $b_{\alpha}$ равны 0.

Тогда:
                                                            
$\sum_{\alpha \in I}(a_{\alpha} - b_{\alpha})v_{\alpha} = 0$,  почти все $a_{\alpha} - b_{\alpha} = 0$
$\Ra \forall \alpha \Ra a_\alpha = b_\alpha$
\end{description}
\end{proof}

\begin{exmp}
\begin{description}
\item [Базис в $K^n$:]

$$e_i = 
\begin{pmatrix}
0\\
0\\
\vdots\\
1\\
0\\
\end{pmatrix}$$ 
Единичный вектор "--- на i-ой строчке стоит 1, все остальные 0.

$$
\sum_{i = 1}^{n}a_ie_i = 
\begin{pmatrix}
a_1\\
\vdots\\
a_n\\
\end{pmatrix}
$$

$e_1, e_2, \ldots, e_n$ "--- стандартный базис.

\item[Базис в $K^3$:]
$$
\begin{pmatrix}
1\\1\\1\\
\end{pmatrix}
,
\begin{pmatrix}
1\\1\\0\\
\end{pmatrix}
,
\begin{pmatrix}
1\\0\\0\\
\end{pmatrix}
$$

Так же базис в $K^3$.

\item [$M(n \times m, K)$:]

Стандартные матричные единицы.
$$e_{ij} = 
\begin{pmatrix} 
0&0&\ldots&0\\
0&0&\ldots&0\\
0&1&\ldots&0\\
\vdots&\vdots&\ddots&\vdots\\
0&0&\ldots&0\\
\end{pmatrix}$$

\item[В]K[x]:

$\{1, x, x^2, \cdots \}$ "--- базис.

\textbf{Упражнение:}
$$P_i(x) \in K[x], i = 0, 1, \cdots$$
$$deg P_i = i$$
Докажите, что $\{P_i\}$ "--- базис K[x] над K. 
\end{description}                            	
\end{exmp}

\begin{theorem}{}
V "--- векторное пространство над K, тогда в V есть базис.
\end{theorem}

\begin{Def}
    V ~--- конечнопорожденное(конечномерное) если в нем есть
    конечная система образующих.
\end{Def}
\begin{proof}
    Докажем теорему для конечнопорожденных пространств.
    
    \begin{lemma}
        V "--- конечнопорожденное векторное пространство.
        Всякое линейно независимое семейство может быть дополнено до базиса. 
    \end{lemma}
    \begin{proof}
        Пусть $V = < u_1, u_2, \ldots, u_m >$, V порожден данными векторами, то есть это система образующих.

        $\{v_{\alpha}\}$ ~--- линейно независимо. 

        Попробуем дополнить $\{v_{\alpha}\}$  до базиса. 

        Пусть существует $v \in V \setminus (v_{\alpha}|\alpha \in I)$, то есть существует вектор, который нельзя получить с помощью нашего множества. 

        Добавим его в множество и оно останется линейно независимым. $\{v\} \cup \{v_{\alpha}\}_{\alpha \in I}$.

        Число векторов в линейно независимом семействе $\le m$.

        Все $v$ являются линейными комбинациями $u_1, \ldots, u_m$.
        Если число векторов >m, то они линейно зависимы(теорема о линейной зависимости линейной комбинации). 
        $\Ra$ Процесс оборвется за конечное число шагов. 
    \end{proof}
    
    Первое доказательство. Возьмем любое линейно независимое множество и дополним его до базиса. 
\end{proof}

\begin{proof}
    \begin{lemma}
    V "--- конечнопорожденное векторное пространство. 
    Из всякой конечной системы образующих можно выбрать базис. 
    \end{lemma}

    \begin{proof}
        $u_1, \cdots, u_m$ "--- семейство образующих. 

        Если линейно независимые, то базис найден.

        Если линейно зависимы, то один линейная комбинация остальных. 

        Значит $\{u_1, \cdots, u_m\} \setminus \{u_j\}$ "--- тоже семейство образующих.

        Так как исходное семейство конечно, то процесс оборвется $\Ra$ найдем базис. 
    \end{proof}

    Второе доказательство. 

    Есть семейство образующих, применим лемму. 
\end{proof}

\begin{exmp}
$\R$ ~--- векторное пространство над $\Q$. 
Базис $\R$ над $\Q$ есть, но уже несчетный. 
\end{exmp}

\begin{theorem}
 Любые два базиса конечнопорожденного векторного пространства содержат одно и то же
 число элементов.  
\end{theorem}

\begin{proof}
(Из теоремы о линейной зависимости линейных комбинаций следует, что всякий базис конечен).

$V$ "--- векторное пространство. 

$u_1, \cdots, u_m$ "--- базис V

$v_1, \cdots, v_n$ "--- базис V

$u_1, \cdots, u_m$ "--- образующие V

$v_1, \cdots, v_n$ "--- линейно независимые и линейная комбинация $u_1, \cdots, u_m$.

По теореме о линейной зависимости линейных комбинаций $n \le m$, аналогично $m \le n$.

Значит $n = m$.

\end{proof}

\begin{Def}
   Если $V$ "--- конечнопорожденное векторное пространство, то число векторов в базисе 
   называется размерностью V над K.

   $dim V$, $dim_K V$.

   Если векторное пространство не является конечнопорожденным, то $dim V = \infty$.
\end{Def}

\begin{exmp}
\begin{enumerate}
\item $dim_K K^n = n$
\item $dim_K M(n \times m, K) = nm$
\item $dim_K K[x] = \infty$
\item $dim_{K}\{g \in K[x]: \deg g \le n \in \N_{0}\} = n + 1$
\item $dim_{\R}\C = 2 (1, i)$
\item $dim_{\R}\H = 4 (1, i, j, k)$
\item $dim_{\C}\H = 2 (1, j)$
\end{enumerate}
\end{exmp}

AX = 0, множество решений "--- векторное пространство. 

Размерность пространства решений = число свободных переменных.

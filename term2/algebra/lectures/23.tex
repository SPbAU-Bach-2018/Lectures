\chapter{Пространства со скалярным произведением}
\section{Билинейные и полуторалинейные формы}
\begin{Def}
$V$ "--- векторное пространство над $K$. 

$B \colon V \times V \to K, B$ "--- билинейная форма, 
если $B$ линейно по каждому аргументу.

$$B(\alpha_1 x_1 + \alpha_2 x_2, y) = \alpha_1 B(x_1, y) + \alpha_2 B(x_2, y)$$
$$B(x, \alpha_1y_1 + \alpha_2 y_2) = \alpha_1 B(x, y_1) + \alpha_2(x, y_2)$$
\end{Def}
\begin{Def}
$B$ "--- билинейная форма.
    \begin{enumerate}
    \item $B$ "--- симметрическая, если $\forall x, y \in V, B(x, y) = B(y, x)$.
    \item $B$ "--- кососимметрическая, если 
        \begin{enumerate}
        \item $\forall x, y \in V B(x, y) = -B(y, x)$
        \item $\forall x \in V, B(x, x) = 0$
        \end{enumerate}
    \end{enumerate}
\end{Def}
\begin{Rem}
Если $char K \ne 2$, то  $a \Ra b$.

$$B(x, x) = -B(x, x) \Ra 2B(x, x) = 0 \Ra B(x, x) = 0$$
\end{Rem}
\begin{Rem}
Для произвольного поля $b \to a$.
$$0 = B(x + y, x + y) = B(x + y, x) + B(x + y, y) = B(x, x) + B(y, x) + B(x, y) + B(y, y)$$
$$B(y, x) + B(x, y) = 0$$
$$B(x, y) = -B(y, x)$$
\end{Rem}

$$B(0, y) = B(0 + 0, y) = B(0, y) + B(0, y)$$
$$\forall y \in V \colon 0 = B(0, y)$$
$$\forall x \in V \colon 0 = B(x, 0)$$

\begin{Def}
$B$ "--- невырожденная, если 
$$\forall x \ne 0 \exists y \colon B(x, y) \ne 0$$
$$\forall y \ne 0 \exists x \colon B(x, y) \ne 0$$
\end{Def}
\begin{Def}
Пусть $K = \R, V$ "--- векторное пространство над $\R$.

$B \colon V \times V \to \R, B$ "--- билинейная форма.

$B$ "--- положительно определенна, если 
$\forall x\colon B(x, x) \ge 0$ и $B(x, x) = 0 \Lra x = 0$
\end{Def}

\begin{Def}
$B$ "--- неотрицательно определена, если $\forall x \colon B(x, x) \ge 0$
\end{Def}
\begin{Rem}
Положительная определнность $\Ra$ невырожденность.
\end{Rem}
\begin{exmp}
\begin{enumerate}
\item $\R^n$
$$x = 
\begin{pmatrix}
x_1\\
\vdots\\
x_n\\
\end{pmatrix}, 
y = 
\begin{pmatrix}
y_1\\
\vdots\\
y_n\\
\end{pmatrix}$$
$$B(x, y) = x^{T}y = \sum_{i = 1}^{n}x_iy_i$$
$B$ "--- симметричный положительный оператор.
\item $a_1, \cdots, a_n \in R$
$B(x, y) = \sum_{i = 1}^{n}a_ix_iy_i$
$B$ "--- симметричная

$B$ "--- положительно определенная $\Lra$ все $a_i > 0$

\item K "--- произвольное 
$$A \in M(n, n, K), V = K^{n}$$
$$B \colon V \times V \to K$$
$$B(x, y) = x^{T}Ay$$

Если $A = A^{T}$, то $B$ "--- симметричная.
$$B(x, y) = x^TAy = (x^tAy)^T = y^TA^T(x^T)^T = y^TA^Tx = y^TAx = B(y, x)$$

Верно и обратное.
Если $A \ne A^{T}$
$\exists i, j a_{ij} \ne a_{ji}$

$$x = 
\begin{pmatrix}
0\\
0\\
1(-i)\\
0\\
\end{pmatrix}$$
$$y = 
\begin{pmatrix}
0\\
0\\
1(-j)\\
0\\
\end{pmatrix}$$

$x^{T}Ay = a_{ij}, y^TAx = a_{ji} \Ra B$ не симметричная.

$B$ "--- кососимметричная $\Lra$
\begin{enumerate}
\item $A = -A^{T}$
\item на диагонали $A$ нули. 
\end{enumerate}
\item
 $$V = C([a, b], \R)$$
 $$B(f, g) = \int_{a}^{b}f(t)g(t)dt$$
 $B$ "--- симметричная
 $$\int_{a}^{b}f^2(t)dt \ge 0$$
 $$\int_{a}^{b}f^2(t)dt = 0 \Lra f = 0$$
 Положительно определен.
\item $V = C([a, b], \R)$
$\rho$ "--- неприрывная функция на $[a, b]$.
$B(f, g) = \int_{a}^{b}\rho(t)f(t)g(t)dt$

$B$ "--- симметрична.

Если $\rho > 0$, на $[a, b]$, то $B$ "--- положительно определено. 

(Упражнение. Когда $B$ "--- невыраждена?)
\item $V = \R[t]$
$$B(f, g) = \int_{0}^{\infty}e^{-t}f(t)g(t)dt$$
$$B(f, g) = \int_{-\infty}^{\infty}e^{-t^2}f(t)g(t)dt$$
\end{enumerate}     
\end{exmp}

\begin{Def}
$V$ "--- векторное пространство над $R, \dim_{\R}V < \infty$

$B \colon V \times V \to \R, B$ "--- симмитрична, положительно определенная.

Тогда $(V, B)$ называется евклидовым пространством. 
\end{Def}

\begin{Def}
$K$ "--- поле.

$-\colon K \to K$ "--- изоморфизм поля $K$.

$- \ne id$

$-^2 = id$

$\forall a \in K\colon \overline{\overline{a}} = a$ "--- инвлюция на поле $K$.
\end{Def}

\begin{exmp}
$\C$ и комплексное сопряжение. 
\end{exmp}

\begin{Def}
$V$ "--- векторное пространство над $K$. 

$B\colon V \times V \to K$  $B$ "--- полуторолиненая форма если:

\begin{enumerate}
\item $B(x_1 + x_2, y) = B(x_1, y) + B(x_2, y)$
\item $B(x, y_1 + y_2) = B(x, y_1) + B(x, y_2)$
\item $B(\alpha x, y) = \alpha B(x, y)$
\item $B(x, \alpha y) = \overline{\alpha}B(x, y)$
\end{enumerate}
\end{Def}

\begin{Def}
$B$ "--- называется эрмитово симметричная если 
$B(x, y) = \overline{B(y, x)}$
\end{Def}

\begin{Def}
Если $B$ "--- эрмитова симмитричная.
$B(x, x) \in \R$

и если $\forall x \in V, B(x, x) \ge 0, B(x, x) = 0 \Lra x = 0$, то $B$ "--- 
положительно определенная. 
\end{Def}



\begin{Def}
$B$ "--- невырожденная, если 
$$\forall x \ne 0 \exists y B(x, y) \ne 0$$
$$\forall y \ne 0 \exists x B(x, y) \ne 0$$
\end{Def}

Положительно определенная $\Ra$ невыражденная.
\begin{Def}
$K = \C, \dim_{\C}V = n < \infty$

$B \colon V \times V \to K, B$ "--- эрмитово симмитрично, положительно определенное,
тогда $(V, B)$ "--- называется унитарным пространством. 
\end{Def}

\begin{exmp}
\item $C^n$
 $$B(x, y) = \sum_{i = 1}^{n}x_i\overline{y_i}$$
 $$B(x, \lambda y) = \overline{\lambda}B(x, y)$$
 эрмитово симмитрично, положительно определено.
\item $B(x, y) = x^{T}A\overline{y}$
(Упражнение $B$ "--- эрмитово  симмитрично $\Lra A  = \overline{A}^{T}$)
\item $V = C([a, b] \to \C)$
$$B(f, g) = \int_{a}^{b}f(t)\overline{g(t)}dt$$  
$$B(f, f) = \int_{a}^{b}|f|^2dt$$
$B$ "--- положительно определенный. 
\end{exmp}

$B(u, v)$ "--- полуторолинейно эрмитово симметрично.

$K$ "--- поле с инволюцией.

Если $K = \C$
$$B(u, v) = \overline{B(u, v)} \Ra B(u, v) \in \R$$ 

Выше было: $-$ "--- инволюция на $K$(то есть, в частности, - $\ne$ $id$)
$$B(u_1 + u_2, v) = B(u_1, v) + B(u_2, v)$$
$$B(u, v_1 + v_2) = B(u, v_1) + B(u, v_2)$$
$$B(\lambda u, v) = \lambda B(u, v)$$
$$B(u, \lambda v) = \overline{\lambda} B(u, v)$$

Далее - "--- либо инволюция на $K$, либо $id$.

$B$, соответственно, полуторолинейная или билинейная форма.
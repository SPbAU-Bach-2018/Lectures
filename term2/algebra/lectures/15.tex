\section{Ещё раз об элементарных преобразованиях и элементарных матрицах}
\textbf{Элементарные матрицы:}
\begin{enumerate}
\item 
Элементарная диагональная 
$$
\begin{pmatrix}
1&0&\cdots&0\\
.&.&.&.\\
0&\cdots&c&0\\
0&0&\cdots&1\\
\end{pmatrix}
$$
\item Элементарная трансвекция.
$$ E + \lambda e_{ij}, i \ne j$$
\item
$$ 
\begin{pmatrix}
1&0&0&0&0\\
0&0&0&1&0\\
0&0&1&0&0\\
0&1&0&0&0\\
0&0&0&0&1\\
\end{pmatrix}
$$
\end{enumerate}

Элементарное преобразование над строками матриц "--- 
умножение слева на элементарную матрицу нужного размера.

Элементарное преобразование над столбцами матриц "--- 
умножение справа на элементарную матрицу.

\begin{theorem}{}
$A \in M(n, n, K), det(A) \ne 0$
A "--- есть призведение элементарных диагональных матриц и элементарных трансвекций. 
\end{theorem}
\begin{proof}
$$
A \to 
\begin{pmatrix}
1&0&*&\cdots&0\\
0&1&*&\cdots&0\\
0&0&0&\cdots&0\\
.&.&.&.&.\\
0&0&0&\cdots&1\\
\end{pmatrix}
= B
$$

Привели к ступенчатому виду(верхней треугольной матрице)

$det(A) \ne 0 \Ra det(B) \ne 0 \Ra $ на диагонале B ненулевые элементы $\Ra B = E$.

Существует последовательность элементарных матриц $C_1 \cdots C_k$

$$C_kC_{k - 1}\cdots C_{1} A = E$$
$$A = (C_k \cdots C_1)^{-1} = C_1^{-1} \cdots C_{k}^{-1}$$

Обратные к элементарным матрицам "--- элементарные. 

$$
\begin{pmatrix}
1&0&0\\
0&c&0\\
0&0&1\\
\end{pmatrix}^{-1} 
=
\begin{pmatrix}
1&0&0\\
0&c^{-1}&0\\
0&0&1\\
\end{pmatrix} 
$$

$$
(E + \lambda e_{ij})^{-1} = E - \lambda e_{ij}
$$
\end{proof}

Напоминание:
$$A \in M(n, m, K), C \in M(m, m, K), D \in M(n, n, K)$$
невыражденные матрицы. 

$$D^{-1}AC = 
\begin{pmatrix}
E_{r}&0\\
0&0\\
\end{pmatrix}$$
                 
D, С "--- обратимые матрицы.

$C, D^{-1}$ можно представить в виде произведения элементарных метриц. 
\begin{conseq}
$\exists C_1, \cdots, C_k, D_1, \cdots, D_l$ "--- элементарные матрицы. 

$D_l \cdots D_1 A C_1 \cdots C_k =  
\begin{pmatrix}
E_{r}&0\\
0&0\\
\end{pmatrix}$
\end{conseq}
\begin{conseq}
$A \in M(n, m, K)$ "--- элементарными преобразованиями над строками и столбцами 
может быть приведена к виду $ 
\begin{pmatrix}
E_{r}&0\\
0&0\\
\end{pmatrix}$

r = rk(A)
\end{conseq}

Что изменится, если рассмотреть $A \in M(n, m, R)$, R "--- евклидово кольцо?

В диагональных элементарных матрицах требуем, чтобы $c \in R^{\times}$. То есть
можно домножать строки(столбцы) только на обратимые элементы $c \in R^{\times}$

\begin{theorem}{}
$A \in M(n, m, R) \Ra \exists d_1, \cdots, d_r \in R\setminus \{0\}, d_i|d_{i + 1}$

Матрица A может быть приведина элементарными преобразованиями над строками и столбцами $
\begin{pmatrix}
d_1&0&\cdots&0\\
0&.&0&\cdots&0\\
0&0&d_r&\cdots&0\\
0&0&0&\cdots&0\\
.&.&.&.&.\\
0&0&0&\cdots&0\\
\end{pmatrix}
$

Причем $d_i$ определены однозначно с точность до ассоциативности. 
\end{theorem}

Без доказательства.
\subsection{Системы линейных уравнений над евклидовыми кольцами}

$ax + by = c$ разрешима $\Lra gcd(a, b)|c$

$A \in M(n, m, R), R$ "--- евклидово кольцо.
$B$ "--- столбец высоты n(над R).

$AX = B$ когда разрешима над R?

$\delta_{i}(A) = $ НОД всех миноров порядка i.
$\delta_i(A|B) =$ НОД всех миноров порядка i расширенной матрицы.
\begin{theorem}{}
В введеных обозначениях $AX = B$ разрешимо в R $\Lra$ 
\begin{enumerate}
\item rk(A) = rk(A|B)
\item $\forall i \delta_i(A) = \delta_i(A|B)$
\end{enumerate}
\end{theorem}

Без доказательство. Указание: $\delta_i$ не меняется при элементарных преобразованиях.

$$A = (a, b), \delta_1(A) = gcd(a, b)$$
$$(A|B) = (a, b, c), \delta_1(A|B) = gcd(a, b, c)$$ 

Пусть привели A к почти единичному виду. $d_1|\cdots|d_r$

$$\delta_1 = d_1$$
$$\delta_2 = d_1d_2$$
$$\cdots$$
$$\delta_r = d_1d_2 \cdots d_r \Ra d_i = \frac{\delta_i}{\delta_{i - 1}}$$
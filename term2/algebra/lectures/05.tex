\section{Сравнимость по модулю двусторонних идеалов}
\setauthor{Дима Розплохас}

\begin{Def}
$R$ "--- кольцо,	
$I$ "--- двусторонний идеал $R$.

\[a \equiv b (\mod I)\], если
\[a - b \in I\]
Также обозначается
\[a \equiv b (I)\]
\[a \equiv_I b\]
\end{Def}

\begin{Rem}
	$\equiv$ "--- отношение эквивалентности.
\end{Rem}
	
\begin{proof}
	$ \forall a, b \in R$
	\begin{enumerate}
		\item Рефлексивность.
			\[a - a = 0 \in I \Ra a \equiv a\]
		\item Симметричность. 
			\[a \equiv b \Ra (a - b) \in I \Ra -(a - b) \in I \Ra b \equiv a\]
		\item Транзитивность: 
			\[a \equiv b, b \equiv c \Ra (a - b) \in I, (b - c) \in I 
			\Ra ((a - b) - (b - c)) \in I \Ra (a - c) \in I \Ra a \equiv c\]
	\end{enumerate}
\end{proof}

\begin{Def}
	$a \in R$,
	$[a]$ "--- класс эквивалентности $R$
	\[ [a] = \{b \in R \colon a \equiv b (I) \} \]
	\begin{center}
		$R/ \equiv$ или $R/I$ "--- множество классов эквивалентости
	\end{center}
\end{Def}

\textbf{Свойства сравнения}

Пусть $a \equiv b (I)$, $c \equiv d (I)$. Тогда
\begin{enumerate}
	\item $a \pm c \equiv b \pm d$
	\item $ac \equiv bd$
\end{enumerate}
	
\begin{proof}
	\[a - b \in I\]
	\[c - d \in I\]
	\begin{enumerate}
		\item $(a - b) \pm (c - d) \in I \Ra
			(a \pm c) - (b \pm d) \in I \Ra a \pm c \equiv b \pm d$
		\item $(a - b) c \in I,$
		$b (c - d) \in I$
		\[(a - b) c + b (c - d) = ac - bd \in I \Ra ac \equiv bd (I)\]
	\end{enumerate}
\end{proof}
	
\begin{conseq}	
	$a \equiv b (I)$, тогда
	$ac \equiv bc (I)$, $ca \equiv cb (I)$	.
	Так как $c \equiv c (I)$.
\end{conseq}

\section{Факторкольцо}

$R$ "--- кольцо, $I$ "--- двусторонний идеал $R$.

$R / I$ хотим превратить в кольцо.

\textbf{Определим операции:}
\begin{enumerate}
	\item $[a] + [b] = [a + b]$
	\item $[a] \cdot [b] = [a \cdot b]$
\end{enumerate}	
\[+, \cdot \colon R/I \times R/I \to R/I\]
	
\textbf{Проблема:}
результат может зависеть от выбора представителей класса
\[[a] = [a']\]
\[[b] = [b']\]
\[[a + b] \overset{?}{=} [a' + b']\]
\[[ab] \overset{?}{=} [a'b']\]
	
\textbf{Проверка корректности операций}
\[a \equiv a' (I)\]
\[b \equiv b' (I)\]
\[a + b \equiv a' + b' (I) \Ra [a + b] = [a' + b']\]
\[ab \equiv a'b' (I) \Ra [ab] = [a'b']\]
	
\begin{theorem}
	$(R/I, +, \times)$ "--- кольцо.
	Если $R$ "--- коммутативное, то $R/I$ "--- коммутативное.
	Если $R$ содержит $1$,  то и $R/I$ содержит $1$.
\end{theorem}

\begin{proof}
	 \begin{enumerate}
	 	\item $([a] + [b]) + [c] = [a + b] + [c] = [(a + b) + c] =
	 		[a + (b + c)] = [a] + [b + c] = [a] + ([b] + [c])$
	 	\item $[a] + [b] = [a + b] = [b + a] = [b] + [a]$
		\item $0_{R/I} = [0]$
			$[0] + [a] = [0 + a] = [a]$
	 		$[a] + [0] = [a + 0] = [a]$
	 		$b \in [0] \Lra b - 0 \in I \Lra b \in I$
	 		$[0] = I$
	 	\item $-[a] = [-a]$
	 		$[a] + [-a] = [a + (-a)] = [0]$
	 		$[-a] + [a] = [(-a) + a] = [0]$
	 	\item $([a] \cdot [b]) \cdot [c] = [a \cdot b] \cdot [c] =
	 		[(a \cdot b) \cdot c] = [a \cdot (b \cdot c)] =
	 		[a] \cdot [b \cdot c] = [a] \cdot ([b] \cdot [c])$
	 	\item $[a] \cdot ([b] + [c]) = [a] \cdot [b + c] = 
	 		[a \cdot (b + c)] = [(a \cdot b + a \cdot c)] = 
	 		[a \cdot b] + [a \cdot c] = [a] \cdot [c] + [a] \cdot [b]$
	 		$([b] + [c]) \cdot [a]  = [b + c] \cdot [a] = 
	 		[(b + c)  \cdot a] = [(b \cdot a + c \cdot a)] = 
	 		[b \cdot a] + [c \cdot a] = [b] \cdot [a] + [c] \cdot [a]$
	 	\item Если $R$ "--- коммутативное
	 		$[a] \cdot [b] = [a \cdot b] = [b \cdot a] = [b] \cdot [a]$
	 	\item Если $R$ содержит $1$
	 		$1_{R/I} = [1_R]$
			$[a] \cdot [1_R] = [a \cdot 1_R] = [a]$
	 		$[1_R] \cdot [a] = [1_R \cdot a] = [a]$
	\end{enumerate}
\end{proof}
	 
\begin{Def}
	$(R, +, \cdot)$ "--- факторкольцо кольца $R$ по двустороннему
	идеалу $I$ (факторкольцо $R$ по $I$)
\end{Def}	 

\begin{exmp}
	 \[R = \mathbb{Z}\]
	 \[I = m \mathbb{Z}, m > 1\]
	 \[\mathbb{Z} /m \mathbb{Z} \overset{\text{def}}{=} R/I\]
	 \[x \in \mathbb{Z}/m\mathbb{Z}\]
	 \[x = [a] = {b \in \mathbb{Z} \colon b = a (m)}\]
	 В один класс попадают числа, 
	 дающие один остаток при делении на $m$.	 
\end{exmp}
	 
\[\phi \colon R \to R/I\]
\[a \mapsto [a]\]
\[\phi(a + b) = [a + b] = [a] + [b] = \phi(a) + \phi(b)\]
\[\phi(a \cdot b) = [a \cdot b] = [a] \cdot [b] = \phi(a) \cdot \phi(b)\]
\[\phi \text{ "--- гомоморфизм}\]
\[\ker \phi = \phi^{-1}(0_{R/I}) = \phi^{-1}([0]) = I\]
	
Значит каждый идеал "--- ядро некоторого гомоморфизма.
Вместе с доказанным ранее, ядра гомоморфизмов "--- в точности двусторонние идеалы.
	 
\begin{theorem} (о гомоморфизме)
	$f \colon R \to R'$ "--- гомоморфизм.
	Тогда  $f(R)$ изоморфно $R / \ker f$.	
\end{theorem}

\begin{proof}
	Построим $\phi \colon f(R) \to R / \ker f$, следующим образом: 
	\[\forall x \in f(R),  \exists a \colon f(a) = x\]
	\[\phi(x) = [a]\]
	Докажем, что это изоморфизм.
	\begin{enumerate}
		\item Проверка корректности
			\[\text{Доказать:  }f(a) = f(a') \Ra [a] = [a']\]
			\[f(a - a') = f(a) - f(a') = x - x = 0\]
			\[(a - a') \in \ker f\]
			\[a \equiv a' (\ker f)\]
			\[[a] = [a']\]
		\item $\phi$ - гомоморфизм
			\[x, y \in f(R)\]
			\[a \in f^{-1}(x)\]
			\[b \in f^{-1}(y)\]
			\[f(a + b) = f(a) + f(b) = x + y\]
			\[(a + b) \in f^{-1}(\{x + y\})\]
			\[\phi(x + y) = [a + b] = [a] + [b] = \phi(x) + \phi(y)\]
			\[f(ab) = f(a)f(b) = xy\]
			\[(ab) \in f^{-1}(\{xy\})\]
			\[\phi(xy) = [ab] = [a] \cdot [b] = \phi(x) \cdot \phi(y)\]
		\item $\phi$ "--- биекция
			\[ \text{Пусть } [a] \in R / \ker f\]
			\[a \in R\]
			\[f(a) \in f(R)\]
			\[a \in f^{-1}(\{f(a)\})\]
			\[\phi(f(a)) = [a] \Ra \phi\text{ "--- сюрьекция}\]
			\[x, y \in f(R)\]
			\[a \in f^{-1}(\{x\})\]
			\[b \in f^{-1}(\{y\})\]
			\[\text{Пусть } \phi(x) = \phi(y)\]
	 		\[[a] = [b]\]
			\[a \equiv b (\ker f)\]
			\[(b - a) \in \ker f\]
	 		\[x = f(a) = f(a) + 0 = f(a) + f(b - a) = f(a + b - a) = f(b) = y\]
	 		\[\text{Значит }\phi\text{ "--- инъекция.}\]
	\end{enumerate}
\end{proof}

\section{Максимальные идеалы}

\begin{Def}
	$R$ "--- кольцо.
	Двусторонний идеал $I$ называется максимальным, если
	\begin{enumerate}
		\item $R \neq I$
		\item Если $I \subseteq J \subseteq R$, где $J$ "---  двусторонний идеал, то 
			\[J = I \text{ или }J = R\]
	\end{enumerate}
\end{Def}

Иначе говоря, возьмем множество всех идеалов $R$,
кроме самого $R$ и упорядочим их по включению.
Максимальный идеал "--- максимальный элемент множества.

\begin{theorem}
	$R$ "--- коммутативное кольцо с $1$.
	$I$ "--- идеал $R$.
	\[R / I \text{  "--- поле } \Lra I \text{ "--- максимальный идеал}\]
\end{theorem}

\begin{proof}
Влево.
\[I \text{ "--- максимальный идеал}\]
\[1_R \notin I\]
\[(1) = R \supsetneq I\]    
\[[1] \neq [0]\]
\[x \in R/I \colon x \neq 0_{R/I}\]
\[x = [a], a \notin I\]
\[I \subsetneq I + (a) \subset R\]
\[\Ra^{I - max} I + (a) = R\]
\[I + (a) = \{b + a \cdot c \mid  b \in I, c \in R \}\]
\[\exists b \in I, c \in R \colon  1 = b + a \cdot c\]
\[[1] = [b + ac] = [b] + [a] \cdot [c] = x \cdot [c] = [c] \cdot x\]
\[c \text{ "--- обратный к } x\]

Вправо
\[R/I \text{ "---  поле}\]
\[\text{Пусть } I \subset J \subset R\]
\[\text{Пусть } I \neq J\]
\[\text{Докажем, что }J = R\]
\[J \neq I \Ra \exists a \in J \colon a \notin I\]
\[[a]_I \neq [0]_I\]
\[R/I \text{ "---  поле } \Ra \exists [b]_I \colon [b]_I \cdot [a]_I = [1]_I\]
\[1 = ba (I)\]
\[1 = ba + c, c \in J\]
\[ba \in J, c \in J \Ra 1 \in J \Ra 
r = r \dot 1 \in J \forall r \in R \Ra J = R\]
\end{proof}

Пусть $R$ "--- ОГИ

$a \not \sim 1$

$(a)$ "---  максимальный $\Lra$ $a$ "---  неприводим

$a = \epsilon p_1 p_2 \dots p_k$, $p_i$ "---  неприводим

$(a) \subsetneq (p_i) \subsetneq R$

\begin{conseq}
	$R = Z$
	\[Z/mZ \text{ "--- поле} \Lra m \text{ "--- простое}\]
\end{conseq}

\begin{conseq}
	$R = K[x]$, $K$ "--- поле.
	\[K[x]/f \text{  "--- поле } \Lra f \text{ "--- неприводим над } K\]
\end{conseq}
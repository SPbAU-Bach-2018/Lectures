\setauthor{Беляев Станислав}

\section{Векторные пространства}

\begin{Def}
	$K$ "--- поле, $V \ne 0$
	\[ + \colon V \times V \to V \]
	\[ \cdot \colon K \times V \to V \]
	$V$ "--- векторное пространство над полем $K$, если:
	\begin{enumerate}
	\item
		$\forall v_1, v_2, v_3 \in V \colon (v_1 + v_2) + v_3 = v_1 + (v_2 + v_3)$
	\item
		$\exists 0 \in V ~ \forall v \in V \colon 0 + v = v + 0 = v$
	\item
		$\forall v \in V ~ \exists -v \in V \colon v + (-v) = (-v) + v = 0$
	\item
		$\forall v_1, v_2 \in V \colon v_1 + v_2 = v_2 + v_1$
	\item
		$\forall v \in V \colon 1_{K} \cdot v = v$
	\item
		$\forall a,b \in K ~ \forall v \in V \colon (ab)v = a(bv)$
	\item
		$\forall a \in K ~ \forall v_1, v_2 \in V \colon a(v_1 + v_2) = av_1 + av_2$
	\item
		$\forall a,b \in K ~ \forall v \in V \colon (a + b)v = av + bv$
	\end{enumerate}
\end{Def}

\begin{exmp}
	\begin{enumerate}
	\item
		$V = K$, внешнее умножение "--- умножение в поле $K$
	\item
		$K^n$ "--- пространство векторов столбцов
		\[ K^n = \left\{
			\begin{pmatrix}
				a_1\\
				\vdots\\
				a_n\\
			\end{pmatrix}
			\mid a_i \in K
		\right\} \]
	\item
		$\tensor[^n]{K}{}$ "--- пространство векторов строк
		\[ \tensor[^n]{K}{} = \left\{
			( a_1 \dots a_n) \mid a_i \in K
		\right\} \]
	\item
		$K \subseteq L$, $K$ "--- поле, $L$ "--- тело
		\[ \cdot \colon K \times L \to L ~ \text{(сужение умножения в $L$)} \]
		$\C$ "--- в.п. над $\R$
	\item
		$\H$ "--- в.п. над $\R$
	\item
		$\H$ "--- в.п. над $\C$
	\item
		$Q(i) = \{ a + bi \mid a,b \in \Q \}$ "--- в.п. над $\Q$ \\
		$\R$ "--- в.п. над $\Q$
	\item
		$M(n,m,k)$ "--- в.п. над $K$
	\item
		\[ \cdot \colon K \times K[x] \to K[x] ~ \text{(сужение обычного умножения)} \]
		$K[x]$ "--- в.п. над $K$
	\item
		$C([a,b] \to \R)$ "--- в.п. над $\R$
	\end{enumerate}
\end{exmp}

\textbf{Свойства векторного пространства}

\begin{enumerate}
	\item
		$0_v$ "--- единственный
		\begin{proof}
			$v + 0_1 = v = v + 0_2$
		\end{proof}
	\item
		$\forall a \in K \colon a \cdot 0_v = 0_v$
		\begin{proof}
			$a \cdot 0_v = a \cdot (0_v + 0_v) = a \cdot 0_v + a \cdot 0_v$ \\
			$0_v = a \cdot 0_v$
		\end{proof}
	\item
		$\forall v \in V \colon 0_k \cdot v = 0_v$
		\begin{proof}
			$0_k \cdot v = (0_k + 0_k) \cdot v = 0_k \cdot v + 0_k \cdot v$
		\end{proof}
	\item
		$\forall a \in K ~ \forall v \in V \colon (-a) \cdot v = a \cdot (-v) = -(a \cdot v)$
		\begin{proof}
			$av + (-a)v = (a + (-a))v = 0_k \cdot v = 0_v$ \\
			$-(av) = (-a)v$\\
			$2$-e равенство аналогично
		\end{proof}
\end{enumerate}

\section{Подпространства, линейные комбинации}

\begin{Def}
	$v_1, \dots, v_n \in V, a_1, \dots, a_n \in K$
	\[ a_1v_1 + \dots + a_nv_n \text{"--- линейная комбинация $v_i$ с кооф. $a_i$} \]
\end{Def}

\begin{Def}
	$i \in I, v_i \in V, a_i \in K$, почти все ($=$ все, кроме конечного числа) $a_i$ равны $0$
	\[ \sum_{i \in I} a_iv_i \text{"--- линейная комбинация (сумма конечна, т.к. п.в. $a_i = 0$)} \]
\end{Def}

\begin{Def}
	$V$ "--- в.п. над $K$, $0 \neq U \subseteq V$ \\
	\[U \text{ называют подпространством в } V, \text{ если } U \text{"--- в.п. над } K \text{ относительно } + \mid_{U \times U}, \cdot \mid_{K \times U}\]
\end{Def}

\begin{exmp}
	\begin{enumerate}
		\item
			$\C \subseteq \H$, в.п. над $\R$
		\item
			$K[x^2] \subseteq K[x]$
		\item
			$\R[x]$ "--- можно рассматривать, как подпространство в $C([a,b] \to \R)$
	\end{enumerate}
\end{exmp}

\begin{assertion}
	$V$ "--- в.п. над $K$, $0 \neq U \subseteq V$ \\
	\[U \text{"--- подпространство } V \Lra \left\{ \begin{array}{ll}
		U \text{ замкнуто относительно сложения} \\
		U \text{ замкнуто относительно умножения на скаляр}
	\end{array} \right.\]
\end{assertion}
\begin{proof}
	Посмотрим на пункты из определения векторного пространства. $1, 4, 5, 6, 7, 8$ "--- очевидны. Рассмотрим остальные:
	\begin{enumerate}
	\item[2.]
		$\exists 0 \in K ~ \forall u \in U \colon 0_k \cdot u \in U$ \\
		$0_v = 0_k \cdot u = 0_u$
	\item[3.]
		$\forall u \in U \colon (-1_k) \cdot u \in U$ \\
		$(-1_k) \cdot u = -u \in V$, а значит обратный в $U$
	\end{enumerate}
\end{proof}

\begin{lemma}
	(Без доказательства) Пересечение подпространств - подпространство.
\end{lemma}

\begin{assertion}
	$\{ v_i \}_{i \in I} \subseteq V$, какое наименьшее подпространство, содержащее это семейство векторов? Ответ: \\
	\[\bigcap\limits_{\mathclap{\substack{\text{\tiny{$U-$ под-во. $V$}} \\ \text{\tiny{$\{ v_i \}_{i \in I} \subseteq U$}}}}}{U} = \left\{ \sum\limits_{i \in I} a_iv_i \mid \text{п.в. } a_i = 0 \right\} \text{"--- мн-во всех конечных лин. комбинаций векторов из } \{ v_i \}_{i \in I}\]
\end{assertion}

\begin{Def}
	$\{ v_i \}_{i \in I}$ "--- семейство образующих 
\end{Def}

\begin{Def}
	$\left< v_i \mid i \in I \right>$ "--- подпространство, порожденное семейством $\{ v_i \}_{i \in I}$ (линейная оболочка векторов $v_i$)
\end{Def}

\begin{Def}
	Если $V = \left< v_i \mid i \in I \right>$, то говорят, что cемейство образующих порождает пространство $V$
\end{Def}

\begin{exmp}
	\begin{enumerate}
	\item
		$\C$ "--- в.п. над $\R, \C = \left< 1, i \right>$
	\item
		$K[x] = \left< 1, x, x^2, \dots \right>$
	\end{enumerate}
\end{exmp}

\section{Линейная зависимость и линейная независимость}

\begin{Def}
	$\{ v_i \}_{i \in I}$ "--- линейно зависимые, если $\exists a_i$ "--- п.в. (но не все) $a_i = 0$, такие что:
	\[\sum_{i \in I} a_iv_i = 0\]
\end{Def}

\begin{Def}
	Если $\{ v_i \}_{i \in I}$ не являются лин. зависимыми, то они называются линейно независимыми. \\
	\textbf{Эквивалентные определения}
	\begin{enumerate}
	\item
		$\forall a_i$ п.в. (но не все) $a_i = 0 \colon \sum\limits_{i \in I} a_iv_i \neq 0$
	\item
		$\forall a_i$ п.в. $a_i = 0 \colon \sum\limits_{i \in I} a_iv_i = 0 \Rightarrow$ все $a_i = 0$
	\end{enumerate}
\end{Def}

\begin{exmp}
	$\C$ "--- в.п. над $\R$
	\begin{enumerate}
	\item
		$\{1, i\}$ "--- лин. независимые
	\item
		$\{1, 1 + i, i\}$ "--- лин. зависимые \\
		$1 \cdot 1 + (-1) \cdot (1 + i) + 1 \cdot i = 0$
	\item
		$\{1, x, x^2, \dots\}$ "--- лин. независимые ($K[x]$)
	\end{enumerate}
\end{exmp}

\textbf{Свойства линейной зависимости и линейной независимости}

\begin{enumerate}
	\item
		Любая подсистема лин. независимой системы "--- лин. независима
	\item
		Любая подсистема лин. зависимой системы "--- лин. зависима
	\item
		Семейство, содержащее $0$, лин. зависима
	\item
		Семейство, содержащее $2$ одинаковых вектора "--- лин. зависима
	\item
		Семейство лин. независимо $\Lra$ всякое его конечное подсемейство "--- лин. независимо
\end{enumerate}

\begin{lemma}
	$\{ v_i \}_{i \in I}$ лин. зависимые $\Rightarrow$ один из $v_i$ есть лин. комбинация остальных
\end{lemma}
\begin{proof}
	$\sum\limits_{i \in I} a_iv_i = 0$, п.в. (но не все) $a_i = 0, j \colon a_j \neq 0$ \\
	$a_jv_j + \sum\limits_{\substack{\text{\tiny{$i \in I $}} \\ \text{\tiny{$i \neq j $}}}} a_iv_i = 0 \Rightarrow v_j = \sum\limits_{\substack{\text{\tiny{$i \in I $}} \\ \text{\tiny{$i \neq j $}}}} (\frac{-a_i}{a_j})v_i$
\end{proof}

\begin{theorem}
	(О лин. зависимости лин. комбинаций) \\
	$V$ "--- в.п. над $K, ~ v_1, \dots, v_n \in V, ~ u_1, \dots, u_m \in \left< v_1, \dots, v_n \right>$ \\
	\[\text{Если } m > n, \text{ то } u_1, \dots, u_m \text{ лин. зависимые}\]
\end{theorem}
\begin{proof}
	$u_i = \sum\limits_{j = 1}^{n} a_{ij}v_j, ~ a_{ij} \in K$ \\
	$x_1, \dots, x_m = ?$ \\
	$0 = x_1 \cdot u_1 + \dots + x_m \cdot u_m = \sum\limits_{i = 1}^{m} x_iu_i = \sum\limits_{i = 1}^{m} x_i \sum\limits_{j = 1}^{n} a_{ij}v_j = \sum\limits_{i = 1}^{m} \sum\limits_{j = 1}^{n} (x_ia_{ij})v_j = \sum\limits_{j = 1}^{n} \sum\limits_{i = 1}^{m} (x_ia_{ij})v_j$ \\
	Потребуем, чтобы для $j = 1, \dots, n \colon \sum\limits_{i = 1}^{m} x_ia_{ij} = 0$ \\
	Это система $n$ линейных уравнений с $m$ неизвестными, $m > n \Rightarrow \exists x_i$ (не все $0$), являющиеся решением этой системы \\
	$\sum\limits_{i = 1}^{m} x_iu_i = \sum\limits_{j = 1}^{n} 0 \cdot v_j = 0$ \\
	Так как не все $x_i = 0$, то $\{u_i\}_{i = 1}^{m}$ - лин. зависимые
\end{proof}
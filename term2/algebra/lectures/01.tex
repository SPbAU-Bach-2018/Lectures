\chapter{Теория делимости в кольцах}

\section{Делимость}
$R$ --- кольцо

\begin{Def}
	$a$ делит $b$ слева, если \[  \exists c \in R \colon b = ac \]
	
	$a$ --- левый делитель $b$
\end{Def}

\begin{Def}
	$a$ делит $b$ справа, если \[ \exists d \in R \colon b = da \]
	
	$a$ --- левый делитель $b$
\end{Def}

Если $R$ коммутативно, то говорят просто о делителе:
$$ a | b \Lra  b \vdots a $$

Если R не коммутативно:

$a|_R b$ --- $a$ делит $b$ слева

$a_R | b$ --- $a$ делит $b$ справа

\textbf{Свойства:}
\begin{enumerate}
	\item $a | b \wedge b | c \Ra a | c$
	\item $a |_R b \wedge b |_R c \Ra a |_R c$
	\item $a_R | b \wedge  b_R | c \Ra a_R | c$
	\begin{proof}
		\begin{gather*}
			 b = da \\
			 c = fb = (fd)a
		\end{gather*}
		\textit{для делимости слева аналогично, для двусторонней --- показать обе выкладки}
	\end{proof}
	\item 	если $R$ - кольцо с единицей, то $a | a, a |_R a, a_R | a$
	\begin{proof}
		\[ a = a \cdot 1 = 1 \cdot a \]
	\end{proof}
\end{enumerate}

\begin{Def}
	\[ ab = 0, a \neq 0 ,b \neq 0 \]
	
	$a$ --- левый нетривиальный делитель нуля

	$b$ --- правый нетривиальный делитель нуля
\end{Def}

\begin{Def}
	Коммутативное кольцо с единицей --- область целостности, если в нём нет делителей нуля.
	\[ \forall a, b \in R \colon ab = 0 \Ra a = 0 \vee b = 0 \]
\end{Def}

\begin{Def}
	$R$ --- кольцо с единицей
	
	$R^*$ --- мультипликативная группа кольца $R$
	
	\[ R^* = \left\lbrace a \in R \mid \exists b \in R \colon ab = ba = 1 \right\rbrace \]
\end{Def}

\section{Ассоциированность}

\begin{Def}
	$R$ --- кольцо с  единицей
	
	$\sim$ --- отношение ассоциированности

	\[ a \sim b \Lra \exists c \in R^* \colon a = bc \]
\end{Def}

$R^*$ --- группа, док-во см. 1 семестр

\begin{Rem}
	Ассоциированность --- отношение эквивалентности
\end{Rem}

\begin{proof}
	\begin{enumerate}
		\item \[ a \sim a, a = a \cdot 1, 1 \in R^* \]
		\item 	\begin{gather*}
			 a \sim b \Ra b \sim a \\
			 a = bc, c \in R^* \\
			 \exists d \colon cd = 1 \\
			 ad = bcd = b \Ra b \sim a			
		\end{gather*}
		\item \begin{gather*}
			 a \sim b, b \sim c \Ra a \sim c \\
			 \exists u \in R^* \colon a = bu \\
			 \exists v \in R^* \colon b = cv	\\
			 a = c(uv) \Ra a \sim c, uv \in R^* \text{(т.к. R группа)} \\
			 \end{gather*}
	\end{enumerate}
\end{proof}

\textbf{Свойства:}
\begin{enumerate}
	\item Если $a \sim \Ra a | b \wedge b | a $
	\item Если  $R$ область целостности и $a | b \wedge b | a \Ra a \sim b$
\end{enumerate}

\begin{proof}
	\begin{enumerate}
		\item \begin{gather*}
			a = bc, c \in R^* \Ra b | a \\
			a \sim b \Ra b \sim a \Ra a | b
		\end{gather*}
		\item \begin{itemize}
			\item \[ a = 0 \Ra b = 0, 0  = 0 \cdot 1 \Ra a \sim b \]
			\item \begin{gather*}
				a \neq 0, a = auv \\
				a(1 - uv) = 0 \stackrel{\text{R - область целостности}}{\Ra} 1 - uv = 0 \\
				\Ra u \in R^*, v \in R^* \Ra a \sim b
			\end{gather*}
		\end{itemize}
	\end{enumerate}
\end{proof}

\section{Идеалы в кольце}
$R$ --- произвольное кольцо

\begin{Def} 
	$\oslash \neq I \subseteq R$ --- идеал в $R$, если:
	\begin{enumerate}
		\item $ \forall a, b \in I \colon a \pm b \in I $
		\item $ \forall a \in I, \forall r \in R \colon ra \in I $
	\end{enumerate}
\end{Def}

\begin{Rem}
$I + I \subseteq I$

$RI \subseteq I$

$I$ - подкольцо  $I \cdot I \subseteq I$
\end{Rem}

\begin{Def}
	$I$ --- правый идеал, если:
	\begin{enumerate}
		\item $ \forall a, b \in I \colon a \pm b \in I $
		\item $ \forall a \in I \forall r \in R \colon ar \in I, (IR \subseteq I)) $
	\end{enumerate}
\end{Def}

\begin{Def}
	Двусторонний идеал, если $I$ --- левый $(IR \subseteq I)$ и правый $(RI \subseteq I)$
\end{Def}

\begin{Rem}
	в первом условии достаточно требовать: $\forall a, b \in I \colon a - b \in I$
\end{Rem}

\begin{proof}
	\begin{enumerate}
		\item $\exists a \in I \Ra 0 = a - a \in I, 0 \in I$
		\item $a + I \Ra 0 - a \in I \Ra -a \in I$
		\item  \begin{gather*}
			a, b \in I -b \in I \\
			a + b = a - (-b) \in I \\
			\end{gather*}
	\end{enumerate}
\end{proof}

\begin{Rem}
	$R$ --- кольцо
	
	$I = \left\lbrace 0 \right\rbrace, I = R$ --- двусторонние идеалы
\end{Rem}

\textbf{Пример:}

Рассмотрим кольцо $\Z$

$I = m\Z = \left\lbrace a \mid m \text{ делит } a \right\rbrace$

$R = M(n, K)$, где $M$ --- кольцо матриц размера $n$ над полем $K$

$S \subseteq \left\lbrace 1, ... , n \right\lbrace $

${}_SI$ --- множество матриц, у которых: $ \forall j \in S$, $j$-ый столбец заполнен нулями

$I_S$ --- множество матриц, у которых: $ \forall j \in S$, $j$-ая строка заполнена нулями

${}_SI$ --- левые идеалы не являются правыми

$I_S$ --- правые идеалы не являются левыми

\begin{Rem}
	Далее 'идеал' - двусторонний идеал
\end{Rem}

\subsection{операции над идеалами}

\begin{enumerate}
	\item $I_\alpha (\alpha \in A)$ --- идеал в кольце $R$ (левый, правый, двустор.)
	$I = \cap_{\alpha \in A}$ --- идеал (левый, правый, двустор.)
	\item $I_1, I_2$ - идеалы в $R$ (левый, правый, двустор.)
	$I_1 + I_2 = \left\lbrace  a + b \mid a \in I_1, b \in I_2 \right\rbrace$ (левый, правый, двустор.)
	\item $I_1, I_2$ - идеалы в $R$
	$I_1 \cdot I_2 = \left\lbrace  \sum_{i=1}^k a_i b_i \mid a_i \in I_1, b_i \in I_2, k \in \N \right\rbrace$
	
	$I_1$, $I_2$ --- левый, правый, двустор.  $\Ra$ $I_1 \cdot I_2$ --- левый, правый, двустор.
	
	$I_1$ --- левый, $I_2$ --- правый $\Ra$ $I_1 \cdot I_2$ --- двустор.
\end{enumerate}

\begin{proof}
	\textit{все док-ва для левых, для правых аналогично, для двусторонних --- обе выкладки}
	\begin{enumerate}
	\item 	\begin{gather*}
			 I = \cap_{\alpha \in A} I_\alpha, \forall a, b \in I \\
			 \forall \alpha \in A \colon a, b \in I_\alpha \wedge \forall \alpha \in A \colon a \pm b \in I_\alpha \Ra a \pm b \in \cap_{\alpha \in A}I_\alpha \\
			 \forall r \in R, a \in I \colon \forall \alpha \in A \colon a \in I_\alpha, r_a \in I_\alpha \Ra r_a \in \cap_{\alpha in A} I_\alpha
		 \end{gather*}
	\item	\begin{gather*}
			 I_1 + I_2 = \left\lbrace a + b \mid a \in I_1, b \in I_2 \right\rbrace \\
			a, c \in I_1 \wedge b, d \in I_2 \\
			a + b = I_1 + I_2 \wedge c + d \in I_1 + I_2 \\
			(a + b) \pm (c + d) = (a \pm c) + (b \pm d) \in I_1 + I_2 \\
			r \in R, a + b \in I_1 + I_2,  a \in I-1, b \in I_2 \\
			r(a + b) = ra + rb \in I_1 + I_2
		\end{gather*}
	\item \begin{itemize}
		\item \begin{gather*}
			\sum_{i = 1}^k a_i b_i, a_i \in I_1, b_i \in I_2 \\
			\sum_{i = 1}^s c_i d_i, c_i \in I_1, d_i \in I_2 \\
			a_1 b_1 + \dots + a_k b_k + (\pm c_1)d_1 + \dots + (\pm c_s)d_s = a_1 b_1 + \dots a_k b_k + (\pm \dots \pm c_s d_s) \in I_1 I_2
		\end{gather*}
		\item \begin{gather*}
			r \in R \\
			r(\sum^k a_i b_i) = \sum^k (ra_i)b \in I_1 \cdot I_2 \\
			(\sum^K a_i b_i)r = \sum^k a_i(b_ir) = I_1 \cdot I_2
		\end{gather*}
		$I_1$ --- левый, $I_2$ --- правый, $I_1 \cdot I_2$ --- двусторонний
		\end{itemize}
	\end{enumerate}
\end{proof}

\subsection{Идеалы порождённые семейством}

$R$ --- кольцо

$\left\lbrace a_\alpha \right\rbrace_{\alpha \in A}, a_\alpha \in R$

\begin{Def}
	Идеал (левый, правый, двустор.) порождённый $\left\lbrace a_\alpha \right\rbrace_{\alpha \in A}$ - наименьший по включению идеал (левый, правый, двустор.) содержащий в себе это семейство.
	
	$ J = \cap I $
	
	$ \forall \alpha \in A \colon a_\alpha \in I, \exists$ (левый, правый, двустор.) идеал
\end{Def}

\begin{Def}
	\begin{gather*}
		J' = \left\lbrace r_1 a_{\alpha_1} + ... + r_k a_{\alpha_k} + n_1 a_{\beta_1} + ... + n_s a_{\beta_s} \right\rbrace \\
		k, s \in \N, \cup \left\lbrace 0 \right\rbrace, r_i \in R, \alpha_i, \beta_i \in A, n_i \in \Z
	\end{gather*}
\end{Def}

\begin{proof}
	\begin{gather*}
		J' = J = ? \\
		J' \text{ --- идеал}, a_\alpha \in J' \Ra J \subseteq J' \\
		a_{\alpha} \in J \Ra J' \subseteq J \\
		r(r_1 a_{\alpha_1} + \dots + r_k a_{\alpha_k} + n_1 a_{\beta_1} + \dots + n_s a_{\beta_s}) = r r_1 a_{\alpha_1} + \dots + r r_k a_{\alpha_k} + r n_1 a_{\beta_1} + \dots + r n_s a_{\beta_s} \\
		\sum_{i = 1}^k r_i a_{\alpha_i} \in J \\
		\sum_{i = 1}^s n_i a_{\beta_i} \in J
	\end{gather*}
	\begin{Rem}
		Если $1 \in R$, то слагаемые $n_i a_{\beta_i}$ можно опустить
		\[n_i > 0, n_i a_{\beta_i} = \underbrace{a_{\beta_i} + \dots + a_{\beta_i}}{n \text{ раз}} = \underbrace{(1 + \dots + 1)}{\in R} a_{\beta_i} \]
	\end{Rem}
\end{proof}

\textbf{Примеры:}

$R = 2\Z$ - идеал порождённый $2$

$r \cdot 2 \in 4 \Z$

$r \in 2\Z$

\textbf{Обозначения:}

${}_R \{(a_\alpha \mid \alpha \in A)\}$ --- левый идеал, порождённый $\left\lbrace a_\alpha \right\rbrace_{\alpha \in A}$

$(a_\alpha \mid \alpha \in A)_R$ --- правый идеал, порождённый $\left\lbrace a_\alpha \right\rbrace_{\alpha \in A}$

$(a_\alpha \mid \alpha \in A)$ --- двусторонний идеал, порождённый $\left\lbrace a_\alpha \right\rbrace_{\alpha \in A}$

\begin{Def}
	Главный идеал --- идеал, порождённый одним элементом
	
\textbf{Обозначения:} 
	\begin{gather*}
		Ra, R(a), (a)R, aR, (a) \\
		R^{(a_1, \dots, a_n)} = Ra_1 + \dots + Ra_n \\
		{}^{(a_1, \dots, a_n)}R = a_1R + \dots + a_nR
	\end{gather*}
\end{Def}


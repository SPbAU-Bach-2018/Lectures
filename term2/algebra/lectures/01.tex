\chapter{Теория делимости в кольцах}

\section{Делимость}

$R$ "--- кольцо.

\begin{Def}
	$a$ делит $b$ слева, если 
	\[  \exists c \in R \colon b = ac \]
	$a$ "--- левый делитель $b$.
\end{Def}

\begin{Def}
	$a$ делит $b$ справа, если 
	\[ \exists d \in R \colon b = da \]
	$a$ "--- правый делитель $b$.
\end{Def}

Если $R$ коммутативно, то говорят просто о делителе:
\[ a \mid b \Lra  b \vdots a \]

Если $R$ не коммутативно:
$a \mid_R b$ "--- $a$ делит $b$ слева, $a {}_R\mid b$ "--- $a$ делит $b$ справа.

\textbf{Свойства:}
\begin{enumerate}
	\item $a \mid b \land b \mid c \Ra a \mid c$
	\item $a \mid_R b \land b \mid_R c \Ra a \mid_R c$
	\item $a_R \mid b \land  b_R \mid c \Ra a_R \mid c$
	\begin{proof}
		\begin{gather*}
			 b = da \\
			 c = fb = (fd)a
		\end{gather*}
		Для делимости слева аналогично, для двусторонней "--- показать обе выкладки
	\end{proof}
	\item Если $R$ "--- кольцо с единицей, то $a \mid a$, $a \mid_R a$, $a_R \mid a$.
	\begin{proof}
		\[ a = a \cdot 1 = 1 \cdot a \]
	\end{proof}
\end{enumerate}

\begin{Def}
	\[ ab = 0, a \ne 0 ,b \ne 0 \]
	$a$ "--- левый нетривиальный делитель нуля, $b$ "--- правый нетривиальный делитель нуля.
\end{Def}

\begin{Def}
	Коммутативное кольцо с единицей "--- область целостности, если в нём нет делителей нуля.
	\[ \forall a, b \in R \colon ab = 0 \Ra a = 0 \vee b = 0 \]
\end{Def}

\begin{Def}
	$R$ "--- кольцо с единицей. $R^*$ "--- мультипликативная группа кольца $R$
	\[ R^* = \left\{ a \in R \mid \exists b \in R \colon ab = ba = 1 \right\} \]
\end{Def}

\section{Ассоциированность}

\begin{Def}
	$R$ "--- кольцо с  единицей. Введём $\sim$ "--- отношение ассоциированности:
	\[ a \sim b \Lra \exists c \in R^* \colon a = bc \]
\end{Def}

$R^*$ "--- группа, док-во было в 1 семестре.

\begin{Rem}
	Ассоциированность "--- отношение эквивалентности.
\end{Rem}

\begin{proof}
	\begin{enumerate}
	\item 
			\[ a \sim a \La a = a \cdot 1 \La 1 \in R^* \]
	\item
		\begin{gather*}
			a \sim b \Ra b \sim a \\
			a = bc, c \in R^* \\
			\exists d \colon cd = 1 \\
			ad = bcd = b \Ra b \sim a			
		\end{gather*}
	\item
		\begin{gather*}
			a \sim b, b \sim c \Ra a \sim c \\
			\exists u \in R^* \colon a = bu \\
			\exists v \in R^* \colon b = cv	\\
			a = c(uv) \Ra a \sim c, uv \in R^* \text{ (так как $R$ "--- группа)}
		\end{gather*}
	\end{enumerate}
\end{proof}

\textbf{Свойства:}
\begin{enumerate}
	\item Если $a \sim b$, то $a \mid b \land b \mid a$.
	\item Если $R$ "--- область целостности и $a \mid b$ и $b \mid a$, то $a \sim b$.
\end{enumerate}

\begin{proof}
	\begin{enumerate}
		\item \begin{gather*}
			a = bc, c \in R^* \Ra b \mid a \\
			a \sim b \Ra b \sim a \Ra a \mid b
		\end{gather*}
		\item \begin{itemize}
			\item \[ a = 0 \Ra b = 0, 0  = 0 \cdot 1 \Ra a \sim b \]
			\item \begin{gather*}
				a \ne 0, a = auv \\
				a(1 - uv) = 0 \xLongrightarrow{\text{$R$ "--- область целостности}} 1 - uv = 0 \\
				\Ra u \in R^*, v \in R^* \Ra a \sim b
			\end{gather*}
		\end{itemize}
	\end{enumerate}
\end{proof}

\section{Идеалы в кольце}
$R$ "--- произвольное кольцо

\begin{Def} 
	$\emptyset \ne I \subseteq R$ "--- левый идеал в $R$, если:
	\begin{enumerate}
		\item $ \forall a, b \in I \colon a \pm b \in I $
		\item $ \forall a \in I, \forall r \in R \colon ra \in I $
	\end{enumerate}
\end{Def}

\begin{Rem}
$I + I \subseteq I$, $RI \subseteq I$, $I$ "--- подкольцо $I \cdot I \subseteq I$
\end{Rem}

\begin{Def}
	$I$ "--- правый идеал, если:
	\begin{enumerate}
		\item $ \forall a, b \in I \colon a \pm b \in I $
		\item $ \forall a \in I \forall r \in R \colon ar \in I, (IR \subseteq I)) $
	\end{enumerate}
\end{Def}

\begin{Def}
	Двусторонний идеал, если $I$ "--- левый $(IR \subseteq I)$ и правый $(RI \subseteq I)$
\end{Def}

\begin{Rem}
	в первом условии достаточно требовать: $\forall a, b \in I \colon a - b \in I$
\end{Rem}

\begin{proof}
	\begin{enumerate}
		\item $\exists a \in I \Ra 0 = a - a \in I, 0 \in I$
		\item $a + I \Ra 0 - a \in I \Ra -a \in I$
		\item $a, b, -b \in I$. $a + b = a - (-b) \in I$
	\end{enumerate}
\end{proof}

\begin{Rem}
	$R$ "--- кольцо. $I = \left\{ 0 \right\}$, $I = R$ "--- двусторонние идеалы.
\end{Rem}

\begin{exmp}
	Рассмотрим кольцо $\Z$.
	\[ I = m\Z = \left\{ a \mid m \text{ делит } a \right\} \]
	$R = M(n, K)$, где $M$ "--- кольцо матриц размера $n$ над полем $K$.
	
	$S \subseteq \left\{ 1, ... , n \right\}$

${}_SI$ "--- множество матриц, у которых: $ \forall j \in S$, $j$-ый столбец заполнен нулями

$I_S$ "--- множество матриц, у которых: $ \forall j \in S$, $j$-ая строка заполнена нулями

${}_SI$ "--- левые идеалы не являются правыми

$I_S$ "--- правые идеалы не являются левыми
\end{exmp}

\begin{Rem}
	Далее <<идеал>> "--- двусторонний идеал
\end{Rem}

\subsection{Операции над идеалами}

\begin{enumerate}
	\item $I_\alpha (\alpha \in A)$ "--- идеал в кольце $R$ (левый, правый, двусторонний)
	$I = \cap_{\alpha \in A}$ "--- идеал (левый, правый, двусторонний)
	\item $I_1, I_2$ - идеалы в $R$ (левый, правый, двусторонний)
	$I_1 + I_2 = \left\{  a + b \mid a \in I_1, b \in I_2 \right\}$ (левый, правый, двусторонний)
	\item $I_1, I_2$ - идеалы в $R$
	$I_1 \cdot I_2 = \left\{  \sum_{i=1}^k a_i b_i \mid a_i \in I_1, b_i \in I_2, k \in \N \right\}$
	
	$I_1$, $I_2$ "--- левый, правый, двусторонний  $\Ra$ $I_1 \cdot I_2$ "--- левый, правый, двусторонний
	
	$I_1$ "--- левый, $I_2$ "--- правый $\Ra$ $I_1 \cdot I_2$ "--- двусторонний
\end{enumerate}

\begin{proof}
	Все док-ва для левых, для правых аналогично, для двусторонних "--- обе выкладки.
	\begin{enumerate}
	\item 	\begin{gather*}
			 I = \cap_{\alpha \in A} I_\alpha, \forall a, b \in I \\
			 \forall \alpha \in A \colon a, b \in I_\alpha \land \forall \alpha \in A \colon a \pm b \in I_\alpha \Ra a \pm b \in \cap_{\alpha \in A}I_\alpha \\
			 \forall r \in R, a \in I \colon \forall \alpha \in A \colon a \in I_\alpha, r_a \in I_\alpha \Ra r_a \in \cap_{\alpha in A} I_\alpha
		 \end{gather*}
	\item	\begin{gather*}
			 I_1 + I_2 = \left\{ a + b \mid a \in I_1, b \in I_2 \right\} \\
			a, c \in I_1 \land b, d \in I_2 \\
			a + b = I_1 + I_2 \land c + d \in I_1 + I_2 \\
			(a + b) \pm (c + d) = (a \pm c) + (b \pm d) \in I_1 + I_2 \\
			r \in R, a + b \in I_1 + I_2,  a \in I-1, b \in I_2 \\
			r(a + b) = ra + rb \in I_1 + I_2
		\end{gather*}
	\item \begin{itemize}
		\item \begin{gather*}
			\sum_{i = 1}^k a_i b_i, a_i \in I_1, b_i \in I_2 \\
			\sum_{i = 1}^s c_i d_i, c_i \in I_1, d_i \in I_2 \\
			a_1 b_1 + \dots + a_k b_k + (\pm c_1)d_1 + \dots + (\pm c_s)d_s = a_1 b_1 + \dots a_k b_k + (\pm \dots \pm c_s d_s) \in I_1 I_2
		\end{gather*}
		\item \begin{gather*}
			r \in R \\
			r\left(\sum^k a_i b_i\right) = \sum^k (ra_i)b \in I_1 \cdot I_2 \\
			\left(\sum^K a_i b_i\right)r = \sum^k a_i(b_ir) = I_1 \cdot I_2
		\end{gather*}
		$I_1$ "--- левый, $I_2$ "--- правый, $I_1 \cdot I_2$ "--- двусторонний
		\end{itemize}
	\end{enumerate}
\end{proof}

\subsection{Идеалы, порождённые семейством}

$R$ "--- кольцо, $\left\{ a_\alpha \right\}_{\alpha \in A}, a_\alpha \in R$.

\begin{Def}
	Идеал (левый, правый, двусторонний), порождённый $\left\{ a_\alpha \right\}_{\alpha \in A}$ "--- наименьший по включению идеал (левый, правый, двусторонний) содержащий в себе это семейство.
	
	$ J = \cap I $
	
	$ \forall \alpha \in A \colon a_\alpha \in I, \exists$ (левый, правый, двусторонний) идеал
\end{Def}

\begin{Def}
	\begin{gather*}
		J' = \left\{ r_1 a_{\alpha_1} + ... + r_k a_{\alpha_k} + n_1 a_{\beta_1} + ... + n_s a_{\beta_s} \right\} \\
		k, s \in \N, \cup \left\{ 0 \right\}, r_i \in R, \alpha_i, \beta_i \in A, n_i \in \Z
	\end{gather*}
\end{Def}

\begin{proof}
	\begin{gather*}
		J' = J = ? \\
		J' \text{ "--- идеал}, a_\alpha \in J' \Ra J \subseteq J' \\
		a_{\alpha} \in J \Ra J' \subseteq J \\
		r(r_1 a_{\alpha_1} + \dots + r_k a_{\alpha_k} + n_1 a_{\beta_1} + \dots + n_s a_{\beta_s}) = r r_1 a_{\alpha_1} + \dots + r r_k a_{\alpha_k} + r n_1 a_{\beta_1} + \dots + r n_s a_{\beta_s} \\
		\sum_{i = 1}^k r_i a_{\alpha_i} \in J \\
		\sum_{i = 1}^s n_i a_{\beta_i} \in J
	\end{gather*}
	\begin{Rem}
		Если $1 \in R$, то слагаемые $n_i a_{\beta_i}$ можно опустить
		\[n_i > 0, n_i a_{\beta_i} = \underbrace{a_{\beta_i} + \dots + a_{\beta_i}}_{n \text{ раз}} = \underbrace{(1 + \dots + 1)}_{\in R} a_{\beta_i} \]
	\end{Rem}
\end{proof}

\begin{exmp}
$R = 2\Z$ - идеал порождённый $2$

$r \cdot 2 \in 4 \Z$

$r \in 2\Z$
\end{exmp}

\textbf{Обозначения:}

${}_R \{(a_\alpha \mid \alpha \in A)\}$ "--- левый идеал, порождённый $\left\{ a_\alpha \right\}_{\alpha \in A}$

$(a_\alpha \mid \alpha \in A)_R$ "--- правый идеал, порождённый $\left\{ a_\alpha \right\}_{\alpha \in A}$

$(a_\alpha \mid \alpha \in A)$ "--- двусторонний идеал, порождённый $\left\{ a_\alpha \right\}_{\alpha \in A}$

\begin{Def}
	Главный идеал "--- идеал, порождённый одним элементом.

	\textbf{Обозначения:} 
	\begin{gather*}
		Ra, R(a), (a)R, aR, (a) \\
		R^{(a_1, \dots, a_n)} = Ra_1 + \dots + Ra_n \\
		{}^{(a_1, \dots, a_n)}R = a_1R + \dots + a_nR
	\end{gather*}
\end{Def}

\begin{exmp}
	$R$ "--- кольцо с единицей.
	\begin{enumerate}
	\item
		\begin{itemize}
			\item $a \mid _{R} b \iff \left(a\right)_{R} \supseteq \left(b\right)_{R}$
			\item $a {}_{R} \mid b \iff {}_{R}\left(a\right) \supseteq {}_{R}\left(b\right)$
		\end{itemize}
	\item
		$R$ "--- область целостности. \\
			$\left(a\right) = \left(b\right) \iff a \sim b$
	\end{enumerate}
	\begin{proof}
		\begin{enumerate}
			\item 
			\begin{description}
				\item[$\Rightarrow$] 			
				\begin{gather*}
					a \mid_{R} b \Rightarrow \exists c \colon b=ac \\
					\left(b\right)_{R} = \{br \mid r \in R\} = \{acr \mid r \in R\} \subseteq \left(a\right)_{R} 
				\end{gather*}
				\item[$\Leftarrow$] 
				\begin{gather*}
					\left(b\right)_{R} \subseteq \left(a\right)_{R} \Rightarrow b \in \left(a\right)_{R} \\
					\exists c \colon b = ac \\
					c \in R \Rightarrow a \mid_{R} b
				\end{gather*}
			\end{description}
			\item
				\[
				\begin{array}{rcl}
					\begin{array}{l}
						\left(a\right) \subseteq \left(b\right) \iff b \mid a\\
						\left(a\right) \supseteq \left(b\right) \iff a \mid b
					\end{array}
					& \iff & a \sim b
				\end{array}
				\]
		\end{enumerate}
	\end{proof}
\end{exmp}

\begin{Def}
	$R$ "--- коммутативное кольцо с единицей.\\
	Всякий идеал порожденный одним элементом "--- главный.
\end{Def}

\begin{Def}
	$R$ "--- область целостности.\\
	$R$ "--- область главных идеалов (ОГИ), если каждый идеал в нем главный.
\end{Def}

\begin{exmp}
	$R = K\left[x, y\right]$ $R$ "--- не ОГИ
	\begin{proof}
		\begin{gather*}
			I = \{f \in K\left[x,y\right] \mid f\left(0,0\right) = 0\} \\
			\text{Если } I = \left(d\right), \text{то}
			\begin{array}{rcccl}
				\begin{array}{c} x \in d \\ y \in d \end{array}
				& \Rightarrow & 
				\begin{array}{c} d \mid x \\ d \mid y \end{array}
				& \Rightarrow & 
				d - const \neq 0
			\end{array} \\
			\text{но } d \notin I  \text{ "--- противоречие } \Rightarrow I \text{ "--- не главный идеал.}
		\end{gather*}
	\end{proof}
\end{exmp}



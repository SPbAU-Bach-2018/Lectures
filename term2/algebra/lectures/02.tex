\section{НОД}

$R$ "--- коммутативное кольцо, $a_{1}, \dots, a_{n} \in R$
\begin{Def}
	$d \in R$ $d = GCD\left(a_{1}, \dots, a_{n}\right)$, если 
	\begin{enumerate}
		\item $d \mid a_{1}, \dots, d \mid a_{n}$
		\item Если $\delta \mid a_{1}, \dots, \delta \mid a_{n}$, то $\delta \mid d$
	\end{enumerate}
\end{Def}

\begin{Rem}
	$R$ "--- область целостности $\Rightarrow GCD$ определен с точностью до ассоциированности. 
	\begin{gather*}
		\begin{array}{c}
			\delta \mid d \\
			d \mid \delta
		\end{array}
		\Rightarrow d \sim \delta
	\end{gather*}
\end{Rem}

\begin{Def}
	$a_{1}, \dots, a_{n} \in R, 1 \in R$ \\
	$\left(a_{1}, \dots, a_{n}\right)$ "--- взаимно просты, если $GCD\left(a_{1}, \dots, a_{n}\right) = 1$
\end{Def}

\begin{conseq}
	$GCD\left(a, b\right) = GCD\left(a - bq, b\right), q \in R$
\end{conseq}
\begin{conseq}
	$GCD\left(a, 0\right) = a$
\end{conseq}
\begin{conseq}
	$d = GCD\left(a_{1}, \dots, a_{n}\right), 1 \in R $ \\
	$a_{1} = db_{1}, \dots, a_{n} = db_{n} \Rightarrow b_{1}, \dots, b_{n}$ "--- взаимно просты.
\end{conseq}

\begin{proof}
	\begin{enumerate}
	\item
	\begin{gather*}
		d = GCD\left(a, b\right) \\
		\begin{array}{c} d \mid a\\ d \mid b \end{array} \Rightarrow d \mid a - bq \Rightarrow d 
		\text{ "--- общий делитель } a-bq \text{ и } b \\
		\text{Пусть существует } \delta \colon \begin{aligned} \delta \mid& a - bq \\ \delta \mid& b \end{aligned}
		\text{, тогда } a = a - bq + bq \Rightarrow \delta \text{ "--- общий делитель } a \text{ и } b \Rightarrow \\
		\Rightarrow \delta \mid d \Rightarrow d = GCD\left(a - bq, b\right)
	\end{gather*}
	\item
	\begin{gather*}
		\begin{array}{c} a \mid a \\ a \mid 0 \end{array} \Rightarrow a \text{ "--- общий делитель} \\
		\begin{array}{c} \delta \mid a \\ \delta \mid 0 \end{array} \Rightarrow \delta \mid a \Rightarrow a = GCD\left(a, 0\right)
	\end{gather*}
	\item
	\begin{gather*}
		d = GCD\left(a_{1}, \dots, a_{n}\right), a_{i} = db_{i}, a_{i} \neq 0 \\
		1 \mid b_{1}, \dots, b_{n} \\
		\delta \mid b_{1}, \dots, b_{n} \\
		d\delta \mid a_{1}, \dots, a_{n} \\
		d\delta \mid d \Rightarrow \exists u \colon d\delta u = d, d \neq 0  \Rightarrow \delta u = 1 \Rightarrow \delta \mid 1
	\end{gather*}
	\end{enumerate}
\end{proof}

\begin{theorem}{}
	$R$ "--- коммутативное кольцо с единицей. $a_{1}, \dots, a_{n} \in R$
	\begin{enumerate}
		\item Если $\left(a_{1}, \dots, a_{n}\right) = \left(d\right)$, то $d = GCD\left(a_{1}, \dots, a_{n}\right)$ и 
		$\exists x_{1}, \dots, x_{n} \in R \colon d = a_{1}x_{1} + \dots + a_{n}x_{n}$
		\item Если $R$ "--- ОГИ, то для $\forall a_{1}, \dots, a_{n} \exists GCD\left(a_{1}, \dots, a_{n}\right)$ и
		допускается линейное представление
		\item $R$ "--- ОГИ, $d = GCD\left(a_{1}, \dots, a_{n}\right) \Rightarrow \left(d\right) = \left(a_{1}, \dots, a_{n}\right)$
	\end{enumerate}
\end{theorem}

\begin{Rem}
	Наибольший общий делитель, если и существует, то не всегда допускает линейное представление. 
	\begin{gather*}
		K\left[X, Y\right], GCD\left(x, y\right) = 1\\
		1 = xf\left(x, y\right) + yg\left(x, y\right)\\
		\text{поставим } x = 0, y = 0 \text{ "--- противоречие}
	\end{gather*}
\end{Rem}

\begin{proof}
	\begin{enumerate}
		\item 
		\begin{gather*}
			\left(a_{1}, \dots, a_{n}\right) = \left(d\right) \\
			a_{1}, \dots, a_{n} \in \left(d\right) \\
			a_{i} = db_{i} \Rightarrow d \mid a_{i} \Rightarrow d \text{ "--- общий делитель} \\
			\text{С другой стороны } d \in \left(a_{1}, \dots, \right) \\
			\exists x_{1}, \dots, x_{n} \in R \colon d = a_{1}x_{1} + \dots + a_{n}x_{n} = S \\
			\delta \text{ "--- общй делитель } a_{1}, \dots, a_{n} \Rightarrow \delta \mid S \Rightarrow \delta \mid d
		\end{gather*}
		\item
		\begin{gather*}
			\exists d \colon \left(d\right) = \left(a_{1}, \dots, a_{n}\right) \text{ так как } R \text{ "--- ОГИ} \\
			\text{значит по предыдущему пункту: } d = GCD \text{ и существует линейное представление}
		\end{gather*}
		\item
		\begin{gather*}
			d = GCD\left(a_{1}, \dots, a_{n}\right) \\
			\left(a_{1}, \dots, a_{n}\right) = \left(\delta\right) \text{ так как ОГИ}\\
			\text{Значит } \delta = GCD\left(a_{1}, \dots, a_{n}\right) \Rightarrow d \sim \delta \Rightarrow 
			\left(d\right) = \left(a_{1}, \dots, a_{n}\right)
		\end{gather*}
	\end{enumerate}
\end{proof}

\begin{Rem}
	Существуют кольца, в которых наибольший общий делитель не всегда определен.
\end{Rem}

\begin{theorem}{}
	$R$ "--- ОГИ $a$ и $b$ "--- взаимно просты и $a \mid bc$, тогда $a \mid c$.
\end{theorem}
\begin{proof}
	\begin{gather*}
		GCD\left(a, b\right) = 1 \\
		\exists u, v \colon au+bv = 1 \mid c \Rightarrow auc + bcv = c \\
		a \mid bc \Rightarrow a \mid c
	\end{gather*}
\end{proof}


\section{Евклидовы кольца}

\begin{Def}
$R$ "--- область целостности, $\lambda\colon R\smallsetminus\{0\} \to \mathbb{N}$ 
$\lambda$ "--- Евклидова функция (евклидова норма), елси $\forall a, \forall b \neq 0, \exists q, r \in R \colon a = bq + r$ и 
$\left(r = 0 \text{ или } \lambda\left(r\right) < \lambda\left(b\right)\right)$
\end{Def}

\begin{Def}
	$R$ "--- область целостности, $R$ "--- евклидово кольцо, если на нем можно задать евклидову функцию.
\end{Def}

\begin{exmp}
	\begin{enumerate}
		\item $R = \mathbb{Z}, \lambda\left(r\right) = |x|$
		\item $R = K\left[x\right], \lambda = deg\left(g\right), K$ "--- поле
		\item $\mathbb{Z}\left[i\right]$
		\item $\mathbb{Z}\left[\omega\right]$
	\end{enumerate}
\end{exmp}

\begin{proof}
	\begin{gather*}
		\mathbb{Z}\left[i\right] = \{u + vi \mid u, v \in \mathbb{Z}\} \\
		\lambda\left(u + vi\right) = |u + vi|^{2} = u^{2} + v^{2}, \lambda \text{ "--- евклидова функция, что и будем доказывать.} \\
		\text{Пусть } a = u + vi, b = s + ti, \; u,v,s,t \in \mathbb{Z}, \left(s, t\right) \neq \left(0, 0\right) \\
		\frac{a}{b} = \frac{u + vi}{s + ti} = \frac{\left(u + vi\right)\left(s - ti\right)}{s^{2} + t^{2}} = \gamma + \delta i, 
		\gamma, \delta \in \mathbb{Q} \\
		\left<j\right> \text{ "--- ближайшее целое, } |j - \left<j\right>| \leq \frac{1}{2} \\
		q = \left<\gamma\right> + \left<\delta\right>i, a = bq + r\\
		r = a - vq = u + vi - \left(s + ti\right)\left(\left<\gamma\right> + \left<\delta\right>i\right) 
		\Rightarrow r \in \mathbb{Z}\left[i\right] \\
		\frac{\lambda\left(r\right)}{\lambda\left(b\right)} = \frac{|r|^{2}}{|b|^{2}} = \left|\frac{a - bq}{b}\right|^{2} = 
		\left|\frac{a}{b} - q\right|^{2} = \left|\gamma + \delta i - \left<\gamma\right> - \left<\delta\right>i \right|^{2} = 
		\left| \gamma - \left<\gamma\right>\right|^{2} + \left|\delta - \left<\delta\right>\right|^{2} \leq \frac{1}{4} + \frac{1}{4} < 1\\
		\lambda\left(r\right) < \lambda\left(b\right)
		\end{gather*}
\end{proof}

\begin{theorem}
	Если $R$ евклидово, то $R$ "--- ОГИ.
\end{theorem}

\begin{proof}
	\begin{gather*}
		I \text{ "--- идеал в } R\\
		I = \{0\} = \left(0\right) \text{ "--- главный. Пусть } I \neq \{0\}\\
		\mathbb{N} \cup \{0\} \supseteq \Lambda = \{\lambda\left(a\right) \mid a \in I \smallsetminus \{0\} \} \neq \emptyset \Rightarrow 
		\exists m \text{ "--- наименьший элемент в } \Lambda \\
		\text{значит } \exists b \neq 0, b \in I \colon \lambda\left(b\right) = m\\
		\text{Докажем, что } I = \left(b\right) \\
		I \supseteq \left(b\right) \text{ очевидно} \\
		a \in I, a = bq + r, r = 0 \text{ или } \lambda\left(r\right) < \lambda\left(b\right) = m\\
		r = a - bq \in I, \text{ если } r \neq 0 \text{ то получаем противоречие с минимальностю } m \Rightarrow r = 0\\
		a = bq \in\left(b\right) \\
		I \in \left(b\right) \\
		I = \left(b\right)
	\end{gather*}
\end{proof}

\begin{conseq}
	$\mathbb{Z}, K\left[x\right], \mathbb{Z}\left[i\right] $ "--- ОГИ ($K$ "--- поле)
\end{conseq}

\subsubsection{Алгоритм Евклида поиска НОД в Евклидовых кольцах}
$R$ "--- евклидово кольцо. $a, b \in R$

$b = 0\colon gcd(a, 0) = a = a \cdot 1 + 0 \cdot 0$

$b \ne 0:$

$$r_i, x_i, y_i$$
$$r_i = ax_i + by_i$$
$$r_0 = a, x_0 = 1, y_0 = 0$$
$$r_1 = b, x_1 = 0, y_1 = 1$$
$$r_{i - 1} = r_iq_i + r_{i + 1}$$
$$r_{i + 1} = 0 \vee \lambda(r_{i + 1}) < \lambda(r_{i})$$
Продолжаем до тех пор, пока не получим нулевой остаток. 
$$\lambda(r_1) > \lambda(r_2) > \cdots > \lambda(r_i) > \cdots$$
$$\exists n  r_{n + 1} = 0$$
$$r_{i + 1} = r_{i - 1} - r_{i}q_{i} = $$
$$= (ax_{i - 1} + by_{i - 1}) - (ax_{i} + by_{i})q_{i} = $$
$$=a(x_{i - 1} - x_iq_i) + b(y_{i - 1} - y_iq_i)$$
$$x_{i + 1} = x_{i - 1} - x_iq_i$$
$$y_{i + 1} = y_{i - 1} - y_iq_i$$
$$r_{n + 1} = 0, r_n \ne 0$$
$$gcd(a, b) = r_n = ax_n + by_n$$
$$gcd(r_{i + 1}, r_i) = gcd(r_iq_i + r_{i + 1}, r_i) = gcd(r_{i - 1}, r_i) = gcd(r_i, r_{i + 1})$$
$$gcd(a, b) = gcd(r_0, r_1) = gcd(r_1, r_2) = \cdots = gcd(r_n, r_{n + 1}) = gcd(r_n, 0) = r_n$$

\section{Неприводимый, состовные и простые элементы}
$R$ "--- область целостности 
\begin{Def}
	$a \in R \smallsetminus \left(\{0\} \cup R^{*}\right)$, $a$ "--- \textbf{составной}, если $a = bc; \; b, c \notin R^{*}$
\end{Def}

\begin{Def}
	$a \in R \smallsetminus \left(\{0\} \cup R^{*}\right)$, $a$ "--- \textbf{неприводимый}, если из $a = bc$ следует, что 
	$b \in R^{*}$ или $c \in R^{*}$
\end{Def}

\begin{Def}
	$a \in R \smallsetminus \left(\{0\} \cup R^{*}\right)$, $a$ "--- \textbf{простой}, если $\forall b, c \in R 
	\left(a \mid bc \Rightarrow a \mid b \text{ или } a \mid c \right)$ 
\end{Def}

$$p|ab \Ra p|a \vee p|b$$
$R$ "--- область целостности

$\{0\}\cup R^{*} \cup \{$ неприводимые $\} \cup \{$ составные $\}$

\begin{theorem}{}
\begin{enumerate}
    \item $R$ ~--- область целостности

    Всякий простой элемент неприводим.
    \item $R$ ~--- ОГИ, всякий неприводимый простой. 
\end{enumerate}
\end{theorem}

\begin{proof}
\begin{enumerate}
    \item $p\in R, p$ "--- простой.
    $$p = ab \Ra a|p, b|p$$
    $$p|ab \xLongrightarrow[\text{p "--- простой}]{} p|a \cup p|b$$
    $$
    \begin{matrix}
    p|a \wedge a|p \Ra a \sim p, b \in R^{*}\\
    p|b \wedge b|p \Ra b \sim p, b \in R^{*}\\
    \end{matrix} 
     \Ra p \text{"--- неприводим}\\
    $$
    \item $R$ "--- ОГИ

    $p$ "--- неприводимый.
    $$p|ab$$
    $p|a (ok)$
   
    $p \nmid a:$
    $$(p, a) = (d)$$
    $$p = dv$$
    $$d \nsim p (\text{Если $d \sim p$, то $p|d|a !?$})$$
    $$
    \begin{matrix}
    p = dv\\
    p \nsim d\\
    p \text{ "--- неприводим}\\
    \end{matrix} 
    \Ra d \in R^{*}
    $$
    $$(p, a) = (d) = (1)$$
    $p, a$ "--- взаимно просты.
    $$
    \begin{matrix}
    p|ab\\
    p \text{ взаимно прост с } a\\
    \end{matrix} 
    \Ra p|b
    $$
    $$\Ra p \text{"--- простой}$$   
\end{enumerate}
\end{proof}

prume "--- простой.

irreducible "--- неприводимый

composite "--- составной

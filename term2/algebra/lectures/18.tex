\section{Линейные операторы, собственные числа, собственные векторы и 
характерестический многочлен}

$V$ "--- векторное пространство над K.

$End(v)$ "--- пространство линейных отображений из $V$ в $V$, 
то есть пространство линейных операторов на V.

(enclomorphisms)

\textbf{Цель:} найти наиболее простой вид матрицы линейного оператора.

Найти один базис в V, в котором матрица линейного оператора имеет наиболее 
простой вид.

$\dim (V) = n < \infty$
$f \in End(V)$

$v_1, \cdots, v_n$, $f \to [f]_{\{v_1, \cdots, v_n\}, \{v_1, \cdots, v_n\}}$

Если C "--- матрица перехода к другому базису. 

$C^{-1}[f]_{\{v\}}C = [f]_{\{v'\}}$

\begin{Def}
(работает и при бесконечномерном случае)

$f \in End(v)$

$\lambda$ "--- собственное число оператора f, если 
$\exists v \ne 0, f(v) = \lambda v, \lambda \in K$.
\end{Def}

\begin{Def}
   Если $\lambda$ "--- собственное число f, то всякий v, такой что $f(v) = \lambda v$, называется собственным
   вектором.
\end{Def}
\begin{Rem}
$v_1, v_2$ "--- собственные векторы f, отвечающие $\lambda$, тогда 
$\alpha_1 v_1 + \alpha_2 v_2$ "--- собственный вектор f, отвечающий $\lambda$
(их линейная комбинация).

$$f(\alpha_1 v_1 + \alpha_2 v_2) = \alpha_1 f(v_1) + \alpha_2 f(v_2) = $$
$$= \alpha_1 \lambda v_1 + \alpha_2 \lambda v_2 = \lambda(\alpha_1 v_1 + \alpha_2 v_2) $$

$[f]_{v_i}$
\end{Rem}

Множество собственных векторов f, отвечающих $\lambda$ "--- подпространство в V.

$$U_1(\lambda) = \{\delta\ \colon f(\delta) = \lambda \delta\} $$
$$U_1(\lambda) \le V$$

$\lambda$ "--- собственное число $\Lra \dim(U_1(\lambda)) > 0$

В этом случае говорим, что $U_1(\lambda)$ "--- пространство собственных
векторов, отвечающих $\lambda$.

$V = K^{n}$ 

$End(V) \cong M(n, n, K)$

$A \in M(n, n, K)$ 

$\lambda$ "--- собственное число A, если $\exists v \ne 0,  Av = \lambda v$
                                                     
Если $\lambda$ "--- собственное число A, то $v\colon Av = \lambda v$ называется
собственным вектором матрицы A.

Пусть $dim V = n < \infty$
зафиксируем базис $v_1, \cdots, v_n$

$$f \in End(v)$$
$$f(v) = \lambda v$$
$$A = [f]_{\{v_i\}}$$
$$
X = \begin{pmatrix}
x_1\\
\vdots\\
x_n\\
\end{pmatrix}
$$
$$v = x_1v_1 + \cdots + x_nv_n $$

$$[f]_{v_i}\cdot[V]_{\{v_i\}} = \lambda[V]_{\{v_i\}}$$
$$AX = \lambda X$$


\begin{Def}
Теперь перейдем к характеристическому многочлену:

$A \in M(n, n, K)$
$\chi_{A}(t) = det(A - tE) \in K[t]$ "--- характеристический многочлен матрицы A.

$\dim V = n < \infty$

$f \in end(v)$

$\chi_{f}(t)$ "--- характеристический многочлен матрицы f в некотором базисе.
\end{Def}

\begin{Rem}
$\chi_f$ не зависит от выбора базиса в пространстве V.
\end{Rem}
\begin{proof}
$A$ "--- матрица f в каком-то базисе.

В другом базисе: $C^{-1}AC$, где С "--- матрица перехода от базиса к базису.

$$\chi_{C^{-1}AC}(t) = \det(C^{-1}AC - tE) = \det(C^{-1}AC - C^{-1}(tE)C) = $$
$$ = \det(C^{-1}(A - tE)C) = \det(C^{-1})\det(A - tE)\det(C) = \det(A - tE)\cdot\det(C^{-1})\det C =  $$
$$= \det(A - tE)\det(C^{-1}C) = \det(A - tE) = \chi_{A}(t) $$

\end{proof}

$$\chi_A(t) = \det(A - tE) = \det(\begin{pmatrix}
a_{11} - t& a_{12}& \cdots & a_{1n}\\
\vdots& \vdots&\cdots&\vdots\\
a_{n1}& a_{n2} &\cdots&a_{nn} - t\\
\end{pmatrix}) = $$
$$= (-1)^{n}t^{n} + (-1)^{n - 1}(a_{11} + a_{nn})t^{n - 1} + \cdots + \det(A)$$

$a_{11} + a_{22} + \cdots + a_{nn}$ "--- след A(trace)

Tr(A).

\begin{theorem}{}
$\dim V = n < \infty, f \in End(V)$

Тогда собственные числа f "--- это, в точности, корни $\chi_{f}(t)$
\end{theorem}

\begin{proof}
Пусть A "--- матрица $f$ в некотором базисе.

Собсвенные числа f совпадают с собственными числами A.

$\exists x \ne 0, AX = \lambda X$
$$AX = \lambda E X $$
$$(A - \lambda E)X = 0$$

$\lambda$ "--- собственное число $\Lra (A - \lambda E)X = 0$  имеет нетривиальное
решение $\Lra \det(A - \lambda E) = 0 \Lra \chi_A(\lambda) = 0 \Lra \lambda$ "--- корень
$\chi_A = \chi_f$. 
\end{proof}

Наиболее простое описание следует ожидать, 
когда K "--- алгебраически замкнутое поле
(тогда $\chi_A$ полностью раскладывается на множетели)

Далее предполагаем, что K "--- алгебраически замкнуто.

$$\chi_f(t) = \chi_A(t) = (-1)^{n}\prod(t - \lambda_i)^{a_i}$$ 
$$ \sum a_i = n$$
$\lambda_i$ "--- попарно различны.

$a_i$  "--- алгебраическая кратность собственного числа $\lambda_i$

$\dim U_1(\lambda_i) = b_i$ "--- геометрическая кратность собственного числа $\lambda_i$.

\begin{exmp}
$$A = \begin{pmatrix}
2 & -1\\
1 & 0\\
\end{pmatrix}$$

$$\chi_A(t) = (2 - t)(-t) - 1(-1) = t^2 - 2t + 1 = (t - 1)^2$$

алгебраическая кратность = 2

$$A - E = \begin{pmatrix} 
1&-1\\
1&-1\\
\end{pmatrix} $$
$$
(A - E)X =
\begin{pmatrix}
1&-1\\
1&-1\\
\end{pmatrix}X = 0 
$$
$$
X = \begin{pmatrix}
C\\
C\\
\end{pmatrix}
$$

$$U_1(1) = \{\begin{pmatrix} C\\ C\\ \end{pmatrix}, c \in K \} $$
$$b = 1 = \dim(U_1(1)) $$

Позднее увидем, что геометрическая кратность всегда 
не больше арифметической.

\end{exmp}
 

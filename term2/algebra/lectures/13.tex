\section{Ранг матрицы}
\begin{Def}
$A \in M(n, m, K)$, наибольшее r такое, что в $A$ есть ненулевой минор размера r, называется рангом матрицы. 
\end{Def}

\begin{Def}
Строковый ранг A "--- это размерность пространства, 
порожденного строками A.
\end{Def}

\begin{Def}
Столбцовый ранг "--- это размерность пространства, порожденного столбцами A.
\end{Def}

\begin{theorem}{}
Ранг матрицы, ее строковой ранг и столбцовый ранг совпадают.
\end{theorem}

\begin{Rem}
\begin{enumerate}
\item Ранг "--- число ненулевых строк после приведения к ступенчатому виду. 
\item Если существуют обратимые C, D, такие что $D^{-1}AC = 
\begin{pmatrix}
E_r&0\\
0&0\\
\end{pmatrix}$, r "--- ранг.

rk A "--- ранг A, rank.
\item Критерий разрешимости:
После приведения к ступенчатому виду число ступенек равно числу ступенек расширенной матрицы
(это утвержденние равносильно теореме Кронера-Каппели)
\end{enumerate}
\end{Rem}

\begin{proof}
Докажем, что строковый ранг и обычный совпадают. 
\begin{enumerate}
\item Не меняется при элементарном преобразование над строками.
\item Сравним оба ранга для ступенчатых

Столбцовый ранг A = строковый ранг $A^{T}$.

У обычного ранга $A \sim A^{T}$.

\end{enumerate}
\end{proof}
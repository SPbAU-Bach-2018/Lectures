\section{Ранг матрицы}
\begin{Def}
$A \in M(n, m, K)$, наибольшее r такое, что в $A$ есть ненулевой минор размера r, называется рангом матрицы. 
\end{Def}

\begin{Def}
Строковый ранг A "--- это размерность пространства, 
порожденного строками A.
\end{Def}

\begin{Def}
Столбцовый ранг "--- это размерность пространства, порожденного столбцами A.
\end{Def}

\begin{theorem}{}
Ранг матрицы, ее строковой ранг и столбцовый ранг совпадают.
\end{theorem}

\begin{Rem}
\begin{enumerate}
\item Ранг "--- число ненулевых строк после приведения к ступенчатому виду. 
\item Если существуют обратимые C, D, такие что $D^{-1}AC = 
\begin{pmatrix}
E_r&0\\
0&0\\
\end{pmatrix}$, r "--- ранг.

rk A "--- ранг A, rank.
\item Критерий разрешимости:
После приведения к ступенчатому виду число ступенек равно числу ступенек расширенной матрицы
(это утвержденние равносильно теореме Кронера-Каппели)
\end{enumerate}
\end{Rem}

\begin{proof}
Докажем, что строковый ранг и обычный совпадают. 
\begin{enumerate}
\item Не меняется при элементарном преобразование над строками.
\item Сравним оба ранга для ступенчатых

Столбцовый ранг A = строковый ранг $A^{T}$.

У обычного ранга $A \sim A^{T}$.

\end{enumerate}

\begin{enumerate}
\item Ранг по строкам не меняется при элементарных преобразованиях.
$$A = \begin{pmatrix}
A_1\\                                         
A_2\\
\cdots\\
A_n\\
\end{pmatrix}$$

$A_1,\cdots, A_n$ "--- строки.

$$dim<A_1, \cdots, A_n>$$ 
Пространство $<A_1, \cdots, A_n>$ не меняется при элементарных преобразованиях.

\begin{enumerate}
\item $A_i \lra A_j$ (Поменять две строки местами).
\item $A_i \ra cA_i$ (Домножить строчку на константу).
\item $A_1 \ra A_i + \lambda A_j$  (Прибвить к строке другую строчку).
      \begin{proof}
      $$v = \sum_{k = 1}^{n} c_k A_k = \sum_{k \ne i, j}c_kA_k + c_iA_i + c_jA_j = $$  
      $$= \sum_{k \ne i, j}c_kA_k + c_i(A_i + \lambda A_j) + (c_j + c_i\lambda)A_j$$
      \end{proof}
\end{enumerate}
\item Ранг не меняестя при элементарных преобразованиях. 
Подробно было доказано в книге Боревича "определители и матрицы" страница 67.
\item  
Строковый ранг и обычный совпадают. 

Любую матрицу можно привести к степенчитаму виду.
$$ 
A \ra
\begin{pmatrix}
1&0&x&x&0&x&0&\cdots&x\\
0&1&x&x&0&x&0&\cdots&x\\
0&0&0&0&1&x&0&\cdots&x\\
\cdot&\cdot&\cdot&\cdot&\cdot&\cdot&\cdot&\cdot\\
0&0&0&0&0&0&0&\cdots&0\\
\end{pmatrix} = B
$$

Достаточно показать, что строковый ранг B и ранг B совпадают.

В B r ненулевых строк. Строковый ранг $\le r$.

$$0 = \lambda_1 B_1 + \cdots + \lambda_rB_r = (x, \lambda_1, x, x, \lambda_2, \cdots, \lambda_r) \Ra \lambda_1 = \cdots = \lambda_r = 0$$
$$\Ra dim<B_1, \cdots, B_r, 0, \cdots, 0> = \dim <B_1, \cdots, B_r> = r$$

$$rk B \ge r$$

Выбираем строки $1 \cdots r$, и соответствующие им столбцы. 

$$
\begin{bmatrix}
1&\cdots&0\\
\cdot&\cdot&\cdot\\
0&\cdots&1\\
\end{bmatrix} = 1 \ne 0
$$

Если $min\{$число строк B, число столбцов B$\} = r$, тогда ранг, очевидно, равен r.

Если $min\{$число строк B, число столбцов B$\} > r \Ra$ любая квадратная 
подматрица B размера r + 1 содержит 0 строку $\Ra$ все миноры размера $r + 1$ равны 0.
$\Ra rk B = r \Ra$ строковой и обычный ранги совпадают. 

\item столбцовый ранг A = строковой ранг $А^T$ = ранг $A^T$ = ранг A = строковой ранг A.
\end{enumerate}
\end{proof}

\begin{conseq}
Строковой ранг не меняется при элементарных преобразованиях столбцов. 
\end{conseq}

\begin{Rem}
Из доказательства следует, что rk A "--- число ненулевых  строк после приведения
к ступенчатому виду. 
\end{Rem}

\begin{Rem}
$A \in M(n, m, k)$

\begin{alignat*}{2}
f\colon& K^m \to K^n\\
       & x \to Ax\\
\end{alignat*}


$rk A = \dim \Im f$
$$A = (A_1, \cdots, A_m), A_i \text{--- столбцы}$$

$$Im f = \{Ax| x \in K^{m} \} = \{A \begin{pmatrix}x_1\\ \cdots\\ x_m\\ \end{pmatrix} \colon x_i \in K \} = $$
$$= \{A_1 x_1 + \cdots + A_m x_m| x_i \in K\} = $$
$$= \{x_1 A_1 + \cdots + x_m A_m| x_i \in K\} = <A_1, \cdots, A_m> $$
$\dim \Im f = $ столбцовый ранг A = rk A.
\end{Rem}

\begin{Def}
U, V "--- конечномерные векторные пространства над K.

$f \colon U \to V$ f "--- линейное преобразование.

$rk f = \dim Im f$

$rk f = rk[f]_{\{u_1, \cdots, u_m\}, \{v_1, \cdots, v_n\}}$ 
\end{Def}

\begin{conseq}
$$
rk \begin{pmatrix} 1 & *\\ 0 & C\\ \end{pmatrix} = 1 + rk C
$$

$$
rk \begin{pmatrix} 1 & 0\\ * & C\\ \end{pmatrix} = 1 + rk C
$$
\end{conseq}

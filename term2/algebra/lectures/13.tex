\section{Ранг матрицы}
\begin{Def}
	$A \in M(n, m, K)$.
	Наибольшее $r$ такое, что в $A$ есть ненулевой минор размера $r$, называется рангом матрицы.
	\[ \rang A \]
\end{Def}

\begin{Def}
	Строковый ранг $A$ "--- это размерность пространства, порожденного строками $A$.
\end{Def}

\begin{Def}
	Столбцовый ранг "--- это размерность пространства, порожденного столбцами $A$.
\end{Def}

\begin{theorem}
	Ранг матрицы, ее строковой ранг и столбцовый ранг совпадают.
\end{theorem}

\begin{Rem}
	\begin{enumerate}
	\item
		Ранг "--- число ненулевых строк после приведения к ступенчатому виду.
	\item
		Если существуют обратимые C, D, такие что
		\[
			D^{-1}AC = \begin{pmatrix}
				E_r & 0 \\
				0   & 0
			\end{pmatrix}
		\]
		то $r$ "--- ранг.

	\item
		Критерий разрешимости: после приведения к ступенчатому виду число ступенек равно числу ступенек расширенной матрицы
		(это утвержденние равносильно теореме Кронера-Каппели).
	\end{enumerate}
\end{Rem}

\begin{proof}
	Докажем, что строковый ранг и обычный совпадают.
	План такой:
	\begin{enumerate}
	\item
		Не меняется при элементарном преобразование над строками.

	\item
		Сравним оба ранга для ступенчатых.
		Столбцовый ранг $A$ равен строковому рангу $A^T$.
		У обычного ранга $A \sim A^T$.
	\end{enumerate}
	Докажем:
	\begin{enumerate}
	\item
		Ранг по строкам не меняется при элементарных преобразованиях.
		\[
			A = \begin{pmatrix}
				A_1\\
				A_2\\
				\cdots\\
				A_n\\
			\end{pmatrix}
		\]
		$A_1,\cdots, A_n$ "--- строки.
		\[ \dim \left<A_1, \dots, A_n \right> \]
		Пространство $\left<A_1, \dots, A_n\right>$ не меняется при элементарных преобразованиях:
		\begin{enumerate}
		\item
			$A_i \lra A_j$ (Поменять две строки местами).

		\item
			$A_i \ra cA_i$ (Домножить строчку на константу).

		\item
			$A_1 \ra A_i + \lambda A_j$  (Прибвить к строке другую строчку).
			\begin{proof}
				\[ v = \sum_{k = 1}^n c_k A_k = \sum_{k \ne i, j}c_kA_k + c_iA_i + c_jA_j =
				\sum_{k \ne i, j}c_kA_k + c_i(A_i + \lambda A_j) + (c_j - c_i\lambda)A_j \]
			\end{proof}
		\end{enumerate}

	\item
		Ранг не меняестя при элементарных преобразованиях.
		Подробно было доказано в книге Боревича <<Определители и Матрицы>>, страница 67.

	\item
		Строковый ранг и обычный совпадают.
		Любую матрицу можно привести к степенчитаму виду.
		\[
			A \ra \begin{pmatrix}
				1 & 0 & x & x & 0 & x & 0 & \cdots & x \\
				0 & 1 & x & x & 0 & x & 0 & \cdots & x \\
				0 & 0 & 0 & 0 & 1 & x & 0 & \cdots & x \\
				\vdots&\vdots&\vdots&\vdots&\vdots&\vdots&\vdots&\ddots&\vdots\\
				0 & 0 & 0 & 0 & 0 & 0 & 0 & \cdots & 0
			\end{pmatrix} = B
		\]
		Достаточно показать, что ранг и строковый ранг $B$ совпадают.
		В $B$ есть $r$ ненулевых строк, поэтому строковый ранг не больше $r$:
		\begin{gather*}
			0 = \lambda_1 B_1 + \dots + \lambda_rB_r = (x, \lambda_1, x, x, \lambda_2, \dots, \lambda_r)
				\Ra \lambda_1 = \dots = \lambda_r = 0 \\
			\Ra \dim \left<B_1, \dots, B_r, 0, \dots, 0\right> = \dim \left<B_1, \dots, B_r\right> = r
		\end{gather*}
		Значит, строковый ранг есть ровно $r$.

		$\rang B \ge r$: выбираем строки $1..r$, и соответствующие им столбцы.
		\[
			\begin{vmatrix}
				1 & \cdots & 0 \\
				\vdots&\ddots&\vdots \\
				0 & \cdots & 1
			\end{vmatrix} = 1 \ne 0
		\]
		Если $\min\{\text{число строк $B$}, \text{число столбцов $B$}\} = r$, тогда ранг, очевидно, равен $r$.
		Иначе любая квадратная подматрица $B$ размера $r + 1$ содержит нулевую строку, значит все миноры размера $r + 1$ равны 0.
		Тогда $\rang B = r$, значит строковой и обычный ранги совпадают.

	\item
		Cтолбцовый ранг $A$ равен строковому рангу $А^T$ равен рангу $A^T$ равен рангу $A$ равен строковому рангу $A$.
	\end{enumerate}
\end{proof}

\begin{conseq}
	Строковой ранг не меняется при элементарных преобразованиях столбцов.
\end{conseq}

\begin{Rem}
	Из доказательства следует, что $\rang A$ "--- число ненулевых строк после приведения к ступенчатому виду.
\end{Rem}

\begin{Rem}
	$A \in M(n, m, k)$, $f\colon K^m \to K^n\colon x \mapsto Ax$. Тогда
	$\rang A = \dim \Im f$, $A = (A_1, \dots, A_m)$, $A_i$ "--- столбцы.
	\begin{gather*}
		\Im f = \{Ax \mid x \in K^m \} = \left\{A \begin{pmatrix}x_1\\ \vdots\\ x_m \end{pmatrix} \mid x_i \in K \right\} = \\
		= \{A_1 x_1 + \dots + A_m x_m \mid x_i \in K\} = \\
		= \{x_1 A_1 + \dots + x_m A_m \mid x_i \in K\} = \left<A_1, \dots, A_m\right>
	\end{gather*}
	Получили, что $\dim \Im f = \text{столбцовый ранг $A$} = \rang A$.
\end{Rem}

\begin{Def}
	U, V "--- конечномерные векторные пространства над $K$,
	$f\colon U \to V$ "--- линейное преобразование.
	\begin{align*}
		\rang f &= \dim \Im f \\
		\rang f &= \rang [f]_{\{u_1, \cdots, u_m\}, \{v_1, \cdots, v_n\}}
	\end{align*}
\end{Def}

\begin{conseq}
	\begin{align*}
		\rang \begin{pmatrix} 1 & * \\ 0 & C \end{pmatrix} &= 1 + \rang C \\
		\rang \begin{pmatrix} 1 & 0 \\ * & C \end{pmatrix} &= 1 + \rang C
	\end{align*}
\end{conseq}

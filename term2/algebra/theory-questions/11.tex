\section{Факториальные кольца}
\begin{Def}
В кольце $R$ выполнена теорема об однозначном разложение на множители.
\begin{enumerate}
\item $$a \in R \setminus \{0\}$$
$$a = \epsilon \prod_{i = 1}^n p_i $$
$$\epsilon \in R^{*}, n \ge 0, p_i \text{"--- неприводимы}$$
Или эквивалентное условие
$$a \in R, a\notin \{0\} \cup R^{*}$$
$$a = \prod_{i = 1}^{n} p_i, n \ge 1$$

\item $$0 \ne a = \epsilon, \prod_{i = 1}^{n}p_i = \eta \prod_{j = 1}^{n} q_{j}$$
$$\epsilon, \eta \in R^{*}, p_i,q_i \text{ "--- неприводимы} \Ra n = m$$
$$\text{и } \exists \sigma \in S_n \colon \forall i = 1 \ldots n \colon p_i \sim q_{\sigma(i)}$$
\end{enumerate}
Такое кольцо называется факториальным (unique factorization domain UFD)
\end{Def}

\textbf{Пример не факториального кольца}
$$R = \Z[\sqrt{-5}] = \{a + b\sqrt{-5}\mid a, b \in \Z\}$$

\textbf{Упражнение}

$$6 = 2 \cdot 3 = (1 + \sqrt{5}i)(1 - \sqrt{5}i)$$
\begin{proof}
$2, 3, 1 + \sqrt{5}i, 1 - \sqrt{5}i$ "--- неприводимые, попарно не ассоциированные. 
$$N \colon \Z[\sqrt{-5}] \to \N \cup \{0\}$$
$$N(a + b \sqrt{5}i) = |a + b\sqrt{5}i|^2 = a^2 + 5b^2$$
$$N(z_1z_2) = N(z_1)N(z_2)$$
$$z_{1} \in R^{*}$$
$$z_{1}z_{2} = 1$$
$$N(z_1)N(z_2) = N(1) = 1$$
$$x_1 = a + b\sqrt{5}i$$
$$a^2 + 5b^2 \in \Z$$
$$a^2 + 5b^2 = 1, b = 0, a = \pm 1$$
$2, 3, 1 \pm \sqrt5i$ "--- неприводимые, попарно не ассоциированы.
$$\Z[\sqrt5]^* = \{\pm1\} \Ra 2, 3, 1 \pm \sqrt{-5} \text{ "--- попарно не ассоциированы}$$

$$z_1z_2 = 1$$
$$N(z_1) = 1, z_1 = \pm 1$$
$$N(z_2) = 1, z_2 = \pm 1$$


$$2 = z_1z_2$$
$$4 = N(2) = N(z_1)N(z_2)$$
$$N(z_1) = N(z_2) = 2 \text{"--- невозможно}$$
$$z_1 = a + b\sqrt{-5}$$
$$a^2 + 5b^2 = 2 \text{ нет решений в $\Z$}$$

$$N(1 + \sqrt{5}i) = 6$$
$$1 \pm \sqrt{5}i = z_1z_2$$

6 1

1 6

2 3

3 2

Нет элементов нормы 2 $\Ra$ нет решений.

$$9 = N(3) = N(z_1)N(z_2)$$

9 1

1 9

3 3

$a^2 + 5b^2 = 3$ "--- нет решений в целых $\Ra$ нет элементов нормы 3.

\end{proof}

\textbf{Замечание:} В этом же кольце не для всех пар определен gcd.

\textbf{Докажем:} ОГИ "--- факториальны.

$K[x_1, \cdots, x_n]$ "--- факториально(но не ОГИ $n \ge 2$)

$\Z[x_1, \cdots, x_n]$ "--- факториально(но не ОГИ, если $n \ge 1$)

R "--- факториальная о.ц. $\Ra R[x]$ "--- факториальна.


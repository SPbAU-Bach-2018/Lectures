\section{Алгоритм Евклида поиска НОД в Евклидовых кольцах}
$R$ "--- евклидово кольцо. $a, b \in R$

$b = 0\colon gcd(a, 0) = a = a \cdot 1 + 0 \cdot 0$

$b \ne 0:$

$$r_i, x_i, y_i$$
$$r_i = ax_i + by_i$$
$$r_0 = a, x_0 = 1, y_0 = 0$$
$$r_1 = b, x_1 = 0, y_1 = 1$$
$$r_{i - 1} = r_iq_i + r_{i + 1}$$
$$r_{i + 1} = 0 \vee \lambda(r_{i + 1}) < \lambda(r_{i})$$
Продолжаем до тех пор, пока не получим нулевой остаток.
$$\lambda(r_1) > \lambda(r_2) > \cdots > \lambda(r_i) > \cdots$$
$$\exists n \colon r_{n + 1} = 0$$
$$r_{i + 1} = r_{i - 1} - r_{i}q_{i} = $$
$$= (ax_{i - 1} + by_{i - 1}) - (ax_{i} + by_{i})q_{i} = $$
$$=a(x_{i - 1} - x_iq_i) + b(y_{i - 1} - y_iq_i)$$
$$x_{i + 1} = x_{i - 1} - x_iq_i$$
$$y_{i + 1} = y_{i - 1} - y_iq_i$$
$$r_{n + 1} = 0, r_n \ne 0$$
$$gcd(a, b) = r_n = ax_n + by_n$$
$$gcd(r_{i + 1}, r_i) = gcd(r_iq_i + r_{i + 1}, r_i) = gcd(r_{i - 1}, r_i) = gcd(r_i, r_{i + 1})$$
$$gcd(a, b) = gcd(r_0, r_1) = gcd(r_1, r_2) = \cdots = gcd(r_n, r_{n + 1}) = gcd(r_n, 0) = r_n$$

\section{Идеалы, их свойства}
$R$ "--- произвольное кольцо

\begin{Def}
	$\emptyset \ne I \subseteq R$ "--- левый идеал в $R$, если:
	\begin{enumerate}
		\item $ \forall a, b \in I \colon a \pm b \in I $
		\item $ \forall a \in I, \forall r \in R \colon ra \in I $
	\end{enumerate}
\end{Def}

\begin{Rem}
$I + I \subseteq I$, $RI \subseteq I$, $I$ "--- подкольцо $I \cdot I \subseteq I$
\end{Rem}

\begin{Def}
	$I$ "--- правый идеал, если:
	\begin{enumerate}
		\item $ \forall a, b \in I \colon a \pm b \in I $
		\item $ \forall a \in I \forall r \in R \colon ar \in I, (IR \subseteq I)) $
	\end{enumerate}
\end{Def}

\begin{Def}
	Двусторонний идеал, если $I$ "--- левый $(IR \subseteq I)$ и правый $(RI \subseteq I)$
\end{Def}

\begin{Rem}
	в первом условии достаточно требовать: $\forall a, b \in I \colon a - b \in I$
\end{Rem}

\begin{proof}
	\begin{enumerate}
		\item $\exists a \in I \Ra 0 = a - a \in I, 0 \in I$
		\item $a + I \Ra 0 - a \in I \Ra -a \in I$
		\item $a, b, -b \in I$. $a + b = a - (-b) \in I$
	\end{enumerate}
\end{proof}

\begin{Rem}
	$R$ "--- кольцо. $I = \left\{ 0 \right\}$, $I = R$ "--- двусторонние идеалы.
\end{Rem}

\begin{exmp}
	Рассмотрим кольцо $\Z$.
	\[ I = m\Z = \left\{ a \mid m \text{ делит } a \right\} \]
	$R = M(n, K)$, где $M$ "--- кольцо матриц размера $n$ над полем $K$.
	
	$S \subseteq \left\{ 1, \dots , n \right\}$

$\tensor[_S]{I}{}$ "--- множество матриц, у которых: $ \forall j \in S$, $j$-ый столбец заполнен нулями

$I_S$ "--- множество матриц, у которых: $ \forall j \in S$, $j$-ая строка заполнена нулями

$\tensor[_S]{I}{}$ "--- левые идеалы не являются правыми

$I_S$ "--- правые идеалы не являются левыми
\end{exmp}

\begin{Rem}
	Далее <<идеал>> "--- двусторонний идеал.
\end{Rem}

\subsection{Операции над идеалами}

\begin{enumerate}
\item
	$I_\alpha (\alpha \in A)$ "--- идеал в кольце $R$ (левый, правый, двусторонний).
	$I = \bigcap_{\alpha \in A} I_\alpha$ "--- идеал (левый, правый, двусторонний).

\item
	$I_1, I_2$ "--- идеалы в $R$ (левые, правые, двусторонние).
	$I_1 + I_2 = \left\{  a + b \mid a \in I_1, b \in I_2 \right\}$ "--- идеал (левый, правый, двусторонний).

\item
	$I_1, I_2$ "--- идеалы в $R$.
	$I_1 \cdot I_2 = \left\{ \sum_{i=1}^k a_i b_i \mid a_i \in I_1, b_i \in I_2, k \in \N \right\}$
	Если $I_1$, $I_2$ "--- левые, правые, двусторонние, то $I_1 \cdot I_2$ "--- левый, правый, двусторонний.
	Если $I_1$ "--- левый, а $I_2$ "--- правый, то $I_1 \cdot I_2$ "--- двусторонний.
\end{enumerate}

\begin{proof}
	Все док-ва для левых, для правых аналогично, для двусторонних "--- обе выкладки.
	\begin{enumerate}
	\item 	\begin{gather*}
			 I = \cap_{\alpha \in A} I_\alpha, \forall a, b \in I \\
			 \forall \alpha \in A \colon a, b \in I_\alpha \land \forall \alpha \in A \colon a \pm b \in I_\alpha \Ra a \pm b \in \cap_{\alpha \in A}I_\alpha \\
			 \forall r \in R, a \in I \colon \forall \alpha \in A \colon a \in I_\alpha, ra \in I_\alpha \Ra ra \in \cap_{\alpha \in A} I_\alpha
		 \end{gather*}
	\item	\begin{gather*}
			 I_1 + I_2 = \left\{ a + b \mid a \in I_1, b \in I_2 \right\} \\
			a, c \in I_1 \land b, d \in I_2 \\
			a + b = I_1 + I_2 \land c + d \in I_1 + I_2 \\
			(a + b) \pm (c + d) = (a \pm c) + (b \pm d) \in I_1 + I_2 \\
			r \in R, a + b \in I_1 + I_2,  a \in I_1, b \in I_2 \\
			r(a + b) = ra + rb \in I_1 + I_2
		\end{gather*}
	\item \begin{itemize}
		\item \begin{gather*}
			\sum_{i = 1}^k a_i b_i, a_i \in I_1, b_i \in I_2 \\
			\sum_{i = 1}^s c_i d_i, c_i \in I_1, d_i \in I_2 \\
			a_1 b_1 + \dots + a_k b_k + (\pm c_1)d_1 + \dots + (\pm c_s)d_s = a_1 b_1 + \dots a_k b_k + (\pm \dots \pm c_s d_s) \in I_1 I_2
		\end{gather*}
		\item \begin{gather*}
			r \in R \\
			r\left(\sum^k a_i b_i\right) = \sum^k (ra_i)b_i \in I_1 \cdot I_2 \\
			\left(\sum^k a_i b_i\right)r = \sum^k a_i(b_ir) = I_1 \cdot I_2
		\end{gather*}
		$I_1$ "--- левый, $I_2$ "--- правый, $I_1 \cdot I_2$ "--- двусторонний
		\end{itemize}
	\end{enumerate}
\end{proof}


\section{Максимальные идеалы}

\begin{Def}
	$R$ "--- кольцо.
	Двусторонний идеал $I$ называется максимальным, если
	\begin{enumerate}
		\item $R \neq I$
		\item Если $I \subseteq J \subseteq R$, где $J$ "---  двусторонний идеал, то 
			\[J = I \text{ или }J = R\]
	\end{enumerate}
\end{Def}

Иначе говоря, возьмем множество всех идеалов $R$,
кроме самого $R$ и упорядочим их по включению.
Максимальный идеал "--- максимальный элемент множества.

\begin{theorem}
	$R$ "--- коммутативное кольцо с $1$.
	$I$ "--- идеал $R$.
	\[R / I \text{  "--- поле } \Lra I \text{ "--- максимальный идеал}\]
\end{theorem}

\begin{proof}
Влево.
\[I \text{ "--- максимальный идеал}\]
\[1_R \notin I\]
\[(1) = R \supsetneq I\]    
\[[1] \neq [0]\]
\[x \in R/I \colon x \neq 0_{R/I}\]
\[x = [a], a \notin I\]
\[I \subsetneq I + (a) \subset R\]
\[\Ra^{I - max} I + (a) = R\]
\[I + (a) = \{b + a \cdot c \mid  b \in I, c \in R \}\]
\[\exists b \in I, c \in R \colon  1 = b + a \cdot c\]
\[[1] = [b + ac] = [b] + [a] \cdot [c] = x \cdot [c] = [c] \cdot x\]
\[c \text{ "--- обратный к } x\]

Вправо
\[R/I \text{ "---  поле}\]
\[\text{Пусть } I \subset J \subset R\]
\[\text{Пусть } I \neq J\]
\[\text{Докажем, что }J = R\]
\[J \neq I \Ra \exists a \in J \colon a \notin I\]
\[[a]_I \neq [0]_I\]
\[R/I \text{ "---  поле } \Ra \exists [b]_I \colon [b]_I \cdot [a]_I = [1]_I\]
\[1 = ba (I)\]
\[1 = ba + c, c \in J\]
\[ba \in J, c \in J \Ra 1 \in J \Ra 
r = r \dot 1 \in J \forall r \in R \Ra J = R\]
\end{proof}

Пусть $R$ "--- ОГИ

$a \not \sim 1$

$(a)$ "---  максимальный $\Lra$ $a$ "---  неприводим

$a = \epsilon p_1 p_2 \dots p_k$, $p_i$ "---  неприводим

$(a) \subsetneq (p_i) \subsetneq R$

\begin{conseq}
	$R = Z$
	\[Z/mZ \text{ "--- поле} \Lra m \text{ "--- простое}\]
\end{conseq}

\begin{conseq}
	$R = K[x]$, $K$ "--- поле.
	\[K[x]/f \text{  "--- поле } \Lra f \text{ "--- неприводим над } K\]
\end{conseq}
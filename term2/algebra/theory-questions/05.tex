\section{Наибольший общий делитель}

\begin{Def}
	$R$ "--- коммутативное кольцо, $a_{1}, \dots, a_{n} \in R$, $d \in R$. $d$ называется наибольшим общим делителем, если
	\begin{enumerate}
		\item $d \mid a_{1}, \dots, d \mid a_{n}$
		\item Если $\delta \mid a_{1}, \dots, \delta \mid a_{n}$, то $\delta \mid d$
	\end{enumerate}
	\[ d = \gcd \left(a_{1}, \dots, a_{n}\right) = НОД(a_1, \dots, a_n) \]
\end{Def}

\begin{Rem}
	Если $R$ "--- область целостности, то $\gcd$ определен с точностью до ассоциированности.
\end{Rem}
\begin{proof}
	\begin{gather*}
		\left\{\begin{aligned}
			\delta &\mid d \\
			d &\mid \delta
		\end{aligned}\right.
		\Ra d \sim \delta
	\end{gather*}
\end{proof}

\begin{Def}
	$a_{1}, \dots, a_{n} \in R$, $1 \in R$.
	$\left(a_{1}, \dots, a_{n}\right)$ "--- взаимно просты, если
	\[ \gcd \left(a_{1}, \dots, a_{n}\right) = 1 \]
\end{Def}

\begin{conseq}
	$\gcd \left(a, b\right) = \gcd \left(a - bq, b\right), q \in R$
\end{conseq}
\begin{conseq}
	$\gcd \left(a, 0\right) = a$
\end{conseq}
\begin{conseq}
	$d = \gcd \left(a_{1}, \dots, a_{n}\right)$, $1 \in R$, $a_{1} = db_{1}, \dots, a_{n} = db_{n}$.
	Тогда $b_{1}, \dots, b_{n}$ "--- взаимно просты.
\end{conseq}

\begin{proof}
	\begin{enumerate}
	\item
		\begin{gather*}
			d = \gcd \left(a, b\right) \\
			\left\{\begin{aligned}
				d \mid a \\ d \mid b
			\end{aligned}\right. \Ra d \mid a - bq \Ra \text{$d$ "--- общий делитель $a-bq$ и $b$}
		\end{gather*}
		Пусть $\exists \delta \colon \delta \mid a - bq \land \delta \mid b$. Тогда
		\begin{gather*}
			a = a - bq + bq \Ra \text{$\delta$ "--- общий делитель $a$ и $b$ }\Ra \\
			\Ra \delta \mid d \Ra d = \gcd \left(a - bq, b\right)
		\end{gather*}

	\item
		\begin{gather*}
			\left\{\begin{aligned} a &\mid a \\ a &\mid 0 \end{aligned}\right. \Ra \text{$a$ "--- общий делитель} \\
			\left\{\begin{aligned} \delta &\mid a \\ \delta &\mid 0 \end{aligned}\right. \Ra \delta \mid a \Ra a = \gcd \left(a, 0\right)
		\end{gather*}

	\item
		\begin{gather*}
			d = \gcd \left(a_{1}, \dots, a_{n}\right), a_{i} = db_{i}, a_{i} \neq 0 \\
			\begin{aligned}
				1 &\mid b_{1}, \dots, b_{n} \\
				\delta &\mid b_{1}, \dots, b_{n} \\
				d\delta &\mid a_{1}, \dots, a_{n}
			\end{aligned} \\
			d\delta \mid d \Ra \exists u \colon d\delta u = d, d \neq 0  \Ra \delta u = 1 \Ra \delta \mid 1
		\end{gather*}
	\end{enumerate}
\end{proof}

\begin{theorem}{}
	$R$ "--- коммутативное кольцо с единицей. $a_{1}, \dots, a_{n} \in R$.
	\begin{enumerate}
	\item
		Если $\left(a_{1}, \dots, a_{n}\right) = \left(d\right)$, то $d = \gcd \left(a_{1}, \dots, a_{n}\right)$ и
		\[ \exists x_{1}, \dots, x_{n} \in R \colon d = a_{1}x_{1} + \dots + a_{n}x_{n} \]

	\item
		Если $R$ "--- ОГИ, то $\forall a_{1}, \dots, a_{n} \exists \gcd \left(a_{1}, \dots, a_{n}\right)$ и
		допускается линейное представление.

	\item
		$R$ "--- ОГИ, $d = \gcd \left(a_{1}, \dots, a_{n}\right)$. Тогда
		\[ \left(d\right) = \left(a_{1}, \dots, a_{n}\right) \]
	\end{enumerate}
\end{theorem}

\begin{Rem}
	Наибольший общий делитель, если и существует, то не всегда допускает линейное представление:
	\begin{gather*}
		K\left[X, Y\right]; \gcd \left(x, y\right) = 1\\
		1 = xf\left(x, y\right) + yg\left(x, y\right)
	\end{gather*}
	Подставим $x = y = 0$ "--- противоречие.
\end{Rem}

\begin{proof}
	\begin{enumerate}
		\item
			\begin{gather*}
				\left(a_{1}, \dots, a_{n}\right) = \left(d\right) \\
				a_{1}, \dots, a_{n} \in \left(d\right) \\
				a_{i} = db_{i} \Ra d \mid a_{i} \Ra \text{$d$ "--- общий делитель}
			\end{gather*}
			С другой стороны $d \in \left(a_{1}, \dots, a_n\right)$
			\begin{gather*}
				\exists x_{1}, \dots, x_{n} \in R \colon d = a_{1}x_{1} + \dots + a_{n}x_{n} = S \\
				\delta \text{ "--- общй делитель } a_{1}, \dots, a_{n} \Ra \delta \mid S \Ra \delta \mid d
			\end{gather*}

		\item
			Так как $d$ "--- ОГИ, то $\exists d \colon \left(d\right) = \left(a_{1}, \dots, a_{n}\right)$.
			Значит по предыдущему пункту: $d = \gcd$ и существует линейное представление.

		\item
		\begin{gather*}
			d = \gcd \left(a_{1}, \dots, a_{n}\right) \\
			\left(a_{1}, \dots, a_{n}\right) = \left(\delta\right) \text{ так как ОГИ}\\
			\text{Значит } \delta = \gcd \left(a_{1}, \dots, a_{n}\right) \Ra d \sim \delta \Ra
			\left(d\right) = \left(a_{1}, \dots, a_{n}\right)
		\end{gather*}
	\end{enumerate}
\end{proof}

\begin{Rem}
	Существуют кольца, в которых наибольший общий делитель не всегда определен.
\end{Rem}

\begin{theorem}{}
	$R$ "--- ОГИ, $a$ и $b$ "--- взаимно просты и $a \mid bc$, тогда $a \mid c$.
\end{theorem}
\begin{proof}
	\begin{gather*}
		\gcd \left(a, b\right) = 1 \\
		\exists u, v \colon au+bv = 1 \mid c \Ra auc + bcv = c \\
		a \mid bc \Ra a \mid c
	\end{gather*}
\end{proof}


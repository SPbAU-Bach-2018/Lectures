\section{Расширение полей и присоеденение корней}
\begin{Def}
$F$ "--- подполе $K$, если $F$, $K$ "--- поля, $F \subseteq K$.
\end{Def}
\begin{Def}
$K$ "--- расширение $F$, если $F$, $K$ "--- поля, $F \subseteq K$.
\end{Def}
\begin{theorem}
$F$ "--- поле, $f \in F[x]$, $f$ "--- неприводимый многочлен, $K = F[x]_{/(f)}$. Тогда: 
$$K \text{ "--- поле}, f \text{ имеет корень в} K$$
\end{theorem}
\begin{proof}
$$f \text{ "--- неприводимый} \Rightarrow (f) \text{ "--- максимальный идеал} \Rightarrow$$
$$F[x]_{/(f)} \text{ "--- поле}$$
Введем гомоморфизм $\phi \colon F[x] \rightarrow F[x]_{/(f)}$, редукция по модулю идеала.
По теореме о делении с остатком, каждый класс вычитов в $F[x]_{/(f)}$ содержит единственный многочлен степени $\leq n - 1$, где $n = \deg f$ $\Rightarrow$ сужение $\phi|_{F}$ на множество констант "--- инъективно.
$$F_K \cong \phi(F) \subseteq K \text{ отождевствляем $F$ и $\phi(F)$, $[const] = const$}$$
$$x \xrightarrow{\phi} [x] = \alpha$$
$$f = a_0 + a_1x + a_2x^2 + \dots + a_nx^n, a_n \neq 0$$
$$[f(x)] = 0_K, 0_K \text{ "--- класс нулевого многочлена в $K$}$$
$$0_K = [a_0] + [a_1][x] + \dots + [a_n][x]^n = f(\alpha) \Rightarrow$$
$$x \text{ "--- корень $f$ в поле $K$}$$
\end{proof}
\begin{Rem}
В этом доказательстве сужение $\phi$ на множество многочленов степени $\leq n - 1$ инъективно $\Rightarrow$ всякий элемент поля $K$ единственным образом представляется в виде $$c_0 + c_1\alpha + \dots + c_{n - 1}\alpha^{n - 1} = [c_0 + c_1x + \dots + c_{n - 1}x^{n - 1}]$$
\end{Rem}
\begin{Def}
$K$ "--- поле, полученное присоединением корня $f$.
\end{Def}

\begin{Def}
$F$ "--- поле, $f \in F[x], \deg f \geq 1$. Поле $K \supseteq F$ называется полем разложения $f$, если $f$ раскладывается в $K$ на линейные множители, но не разложим в любом подполе $K$.
\end{Def}
\begin{theorem}
$\exists K$ "--- поле разложения $f$.
\end{theorem}
\begin{proof}
\begin{enumerate}
\item $\exists L$, в котором $f$ раскладывается на линейные множетели.
Докажем по индукции по суммарной степени нелинейных множителей $f$:
\begin{description}
\item[База:] Суммарной степени нелинейных множителей ноль. 
Значит, $f$ разложим на линейные множители.
\item[Переход:] 
$$\exists f_1 | f, \deg f_1 \geq 2, f_1 \text{ "--- неприводим}$$
$$F[x]_{(f_1)} = L_1, \text{ в } L_1, f_1 \text{имеет корень}$$

$$F[x] \subseteq L_1[x]$$
$$f_1 = (x - \alpha)g \text{ в } L_1[x]$$
По индукционному предположению, $\exists L$ "--- расширение $L_1$, в котором $f$ раскладывается на линейные множители.
$$F \subseteq L_1 \subseteq L$$
\end{description}
\item $K = \bigcap M$, $M \subset L$, $M$ "--- подполе $L$, $f$ "--- раскладывается на линейный члены.
\end{enumerate}
\end{proof}

\section{} % 37
Есть векторное пространство $V$ над $K$, если билинейная форма $B\colon V^2 \to K$ (линейна по каждому аргументу).
Бывает симметрическая, кососимметрическая (меняет знак, а $B(x,x)=0$) ($2 \Ra 1$, в поле с $\mathrm{char} K \neq 2$ есть $1 \Ra 2$).
Невырождена, если $\forall x \neq 0\colon \exists y\colon B(x,y)\neq 0$ (и наоборот).
Неотрицательно определена, если $\ge 0$.
Положительно определена, если $\ge 0$ и ноль только в $(x, x)$ (отсюда $\Ra$ невырожденность).
Пространство Евклидово, если есть симметричная положительная $B$.
Инволюция: автоморфизм $\bar{}$ поля $K$, $\bar{} \neq id$, $\bar{}^2 = id$ (комплексное сопряжение).

Полуторная форма: по первому аргументу линейно, а вот по второму "--- только аддитивно и однородно.
Эрмитова симметричность: $B(x,y)=\overline{B(y,x)}$.
Невырожденность та же.
Унитарное пространство: если есть Эрмитова симметричная положительная форма (пример: $\C^n$, скалярное произведение со сопряжением второго вектора).

\section{} % 38
Есть базис $v_i$, матрица Грама задаёт значения для пар: $\Gamma_{ij}=B(v_i, v_j)$.
Она задаёт всю форму (расписали аргументы по базисам, дальше по аддитивности; для бесконечномерного у нас почти все нули).
Матрица Эрмитово симметрична ($\Gamma^T=\bar\Gamma$) $\iff$ форма Эрмитово симметрична.
\TODO[лемма 3.2.2 на странице 77]
Матрица невырождена $\iff$ форма невырождена.
\TODO[доказательство]
По определению симметричная матрица над $\R$ положительно определена, если форма положительно определена ($x^T\Gamma x > 0$ для ненулевых).
Эрмитово симметричная матрица над $\C$ положительно определена тогда же ($x^T\Gamma\bar x > 0$).

\section{} % 39
\TODO

\section{} % 40
\TODO

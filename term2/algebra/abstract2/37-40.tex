\section{} % 37
Есть векторное пространство $V$ над $K$, если билинейная форма $B\colon V^2 \to K$ (линейна по каждому аргументу).
Бывает симметрическая, кососимметрическая (меняет знак, а $B(x,x)=0$) ($2 \Ra 1$, в поле с $\mathrm{char} K \neq 2$ есть $1 \Ra 2$).
Невырождена, если $\forall x \neq 0\colon \exists y\colon B(x,y)\neq 0$ (и наоборот).
Неотрицательно определена, если $\ge 0$.
Положительно определена, если $\ge 0$ и ноль только в $(x, x)$ (отсюда $\Ra$ невырожденность).
Пространство Евклидово, если есть симметричная положительная $B$.
Инволюция: автоморфизм $\bar{}$ поля $K$, $\bar{} \neq id$, $\bar{}^2 = id$ (комплексное сопряжение).

Полуторная форма: по первому аргументу линейно, а вот по второму "--- только аддитивно, зато $B(x,\alpha y)=\bar\alpha B(x, y)$.
Эрмитова симметричность: $B(x,y)=\overline{B(y,x)}$.
Невырожденность та же.
Унитарное пространство: если есть Эрмитова симметричная положительная форма (пример: $\C^n$, скалярное произведение со сопряжением второго вектора).

\section{} % 38
Есть базис $v_i$, матрица Грама задаёт значения для пар: $\Gamma_{ij}=B(v_i, v_j)$.
Она задаёт всю форму (расписали аргументы по базисам, дальше по аддитивности; для бесконечномерного у нас почти все нули), $B(x,y)=x^T\Gamma \bar y$.
Матрица Эрмитово симметрична ($\Gamma^T=\bar\Gamma$) $\iff$ форма Эрмитово симметрична.
Аналогично для обычной симметричности и билинейных форм.
Билинейная форма кососимметрична $\iff$ матрица кососимметрична и на диагонали нули.

Матрица невырождена $\iff$ форма невырождена.
Для любого столбца $x$ есть такой $\bar y$, что $x^T\Gamma \bar y \neq 0$
$\iff$
любая линейная комбинация столбцов (кроме тривиальной) даёт строчку, которую можно скалярно домножить на некий $\bar y$ и получить не ноль
(то есть даёт ненулевую строку)
$\iff$
посмотрим на $\Gamma$ как на отображение $x \to (x^T\Gamma)^T=\Gamma^Tx$
$\iff$
у $\Gamma^T$ тривиальное ядро
$\iff$
$\det \Gamma \neq 0$.
Аналогично для $y$, но с другой стороны.

По определению симметричная матрица над $\R$ положительно определена, если форма положительно определена ($x^T\Gamma x > 0$ для ненулевых).
Эрмитово симметричная матрица над $\C$ положительно определена тогда же ($x^T\Gamma\bar x > 0$).

\section{} % 39
Пусть есть базисы $V$: $v_i$ и $v_i'$, есть матрицы Грама в них ($\Gamma$, $\Gamma'$), есть матрица $C$ перехода от $v_i$ к $v_i'$,
тогда $\Gamma'=C^T\Gamma\bar C$.
Доказательство: возьмём столбцы координат в старом/новом базисе, приравняли как вектора, проверили <<в лоб>>.

\section{} % 40
Пусть $B$  "--- полуторалинейная/билинейная форма на $V$.
Базис ортогонален, если $B(v_i, v_j)=0$ для всех $i \neq j$ (перпендикулярны: $v_i \bot v_j$)
Базис ортонормирован, если ортогонален и $B(v_i, v_i)=1$ для всех $i$.
Если $B$ эрмитово симметрична/симметрична/кососимметрична, то $x \bot y$ симметрично.
Не для всех $B$ есть ортогональный: если $B$ кососимметрична и невырожденна (например), то $x \bot x$.
Тогда $B(v_1, a_1v_1+a_2v_2+\dots)=0$, что противоречит невырожденности.

Теорема: если пространство конечномерно, $B$ "--- эрмитово симметричная положительноопределённая форма, то существует ортогональный базис.
В $\C$ или $\R$ можно даже ортонормированный найти (хотя на паре мы даже ортогонализировали только в них, а не в полях).
Строим по шагам из какого-то базиса $v_i$: на $k$-м шаге первые $k$ векторов (назовём их $u_i$) уже попарно ортогональны (и порождают то же пространство, что и раньше).
Добавляем $u_{k+1}$ в виде линейной комбинации $\alpha_1u_1 + \dots + \alpha_ku_k + v_{k+1}$ (уже сохранили оболочку).
Надо ортогональность: приравняли для каждой пары $(u_i, u_{k+1})$, раскрыли по какой-то линейности, получили $\alpha_i$ однозначно (так как $B(u_i, u_i)>0$ в силу положительноопределённости).
Дальше надо отнормировать, поделив каждый вектор из базиса независимо на $\sqrt{B(u_i,u_i)}$ (правда, тогда нам надо жить в $\C$ или $\R$).

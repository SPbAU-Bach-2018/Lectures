\section{} % 01
Система из уравнений вида $a_{i1}x_1+a{i2}x_2+\dots+a_{im}x_m = b_i$ ($i = 1 \dots n$) (всё в поле, можно смотреть и в коммутативном кольце с единицей).
Однородна, если $b_i=0$, совместна при наличии каких-нибудь решений, матрица системы $A$, вектор-столбец-решение $X$, вектор-столбец $B$ ($AX=B$), расширенная матрица $(A|B)$.
Можно домножать уравнение на обратимый (из $R^*$), прибавлять к $i$-му $j$-е, умноженное на произвольный.
Из этого выражается обмен местами.
Множество решений остаётся.
В матричной форме "--- умножение на матрицу.
Ступенчатый вид: все ненулевые над нулевыми, ведущие элементы (самый левый не ноль) образуют лесенку.
Приведённый ступенчатый: ведущие "--- единицы, кроме них не-нулей нет.
Метод Гаусса: идём по столбцам, приводя к ступеньчатому виду (прямой ход), потом проверяем на совместность, потом обратный ход (почти приведённый вид).

Если работаем над Евклидовым кольцом, то домножать строчки умеем только на обратимые.
Поэтому надо по-другому: в столбце выбирать с наименьшим значением нормы, остальные брать по его модулю, минимум нормы уменьшился (а она натуральная),
так в конце концов все окажутся нули, кроме одного.

\section{} % 02
Над полем $K$, можно складывать и умножать на элемент поля.
$V$ "--- абелева группа по сложению, умножение на элемент поля коммутативно, ассоциативно, дистрибутивно, единица адекватна.
Есть пространство строк ($\tensor[^n]{K}{}$)/столбцов ($K^n$), $\C$ "--- векторное пространство над $\R$ (с добавленным умножением).
Нулевой вектор единственен, $v \cdot 0_K = 0_V$, можно выносить минус из поля за скобки.

\section{} % 03
Линейные комбинации конечные и бесконечные (тогда почти все коэффициенты ноль).
Линейно зависимы, если есть линейная комбинация, дающая ноль (при этом хотя бы один коэффициент не ноль).
Независимость "--- отрицание.
Эквивалентно: любая сумма из ненулевых не ноль; если сумма ноль, то все ноль.
Зависимость подсистем/надсистем, содержание нуля, двух одинаковых.
Семейство линейно независимо $\iff$ всякое конечное подмножество линейно независимо.
Если линейно зависимы, то один есть комбинация остальных (надо делить, да).

\section{} % 04
Если выбрать $m$ векторов из пространства, порождённого $n$ векторами ($m>n$), то они линейно зависимы.
Доказательство без базисов: выразили каждый из $m$ через исходные, составили систему, у неё есть тривиальное решение $\Ra$ есть нетривиальное ($m>n$).

\section{} % 05
Подпространство, если само по себе является векторным пространством над $K$ ($\iff$ замкнуто относительно операций).
Пересечение двух есть подпространство.

\section{} % 06
Наименьшее, содержащее все вектора "--- множество линейных комбинаций.
Семейство образующих для подпространства (определение).

\section{} % 07
Базис "--- линейно независимая система образующих
$\iff$ максимальное по включению линейно независимое
$\iff$ минимальное по включению семейство образующих
$\iff$ всякий вектор единственным образом разложим
$\iff$ всякий вектор раскладывается, а нулевой единственным образом.
Доказываем $1 \iff 2, 3, 5$, $4 \iff 5$.

\section{} % 08
Есть везде, для конечнопорождённых: всякое линейно независимое можно дополнить до базиса (добавляем любой, процесс остановится по билету 4).
Взяли любое линейно независимое (например, пустое) и дополнили.
Второй способ: из конечной системы образующих можно выбрать базис (один есть линейная комбинация, выкинули), взяли семейство образующих, нашли базис.

\section{} % 09
Все базисы конечны.
Если два содержат разное число элементов, то выразили вектора большего через меньший, применили теорему о линейной зависимости линейных комбинаций (билет 4), успех.
Размерность пространства "--- либо $\infty$, либо число.

\section{} % 10
По определению базису разложили вектор (в любом пространстве) в линейную комбинацию базиса, это координаты вектора.
Записываем в столбец.
Смена базиса: взяли два разных, разложили один по другому, записали в матрицу $A$ ($j$-й столбец матрицы "--- координаты вектора нового базиса в старом).
Пока что не наделяем матрицу смыслом.
Если есть три базиса, то матрица перехода $x \to y \to z$ есть произведение промежуточных матриц.
Тогда обратная к матрице перехода всегда есть и она является переходом в другую сторону (потому что переход в другую сторону есть и удовлетворяет $AB=BA=E$).

\section{} % 11
Наделяем матрицу смыслом: $Av'=v$ (если матрица перехода от $v$ к $v'$) "--- матрица должна быть рядом с новыми координатами.
Если действительно хотим переход из старых в новые, надо брать обратную.

\section{} % 12
Пересечение всегда подпространство (а объединение не всегда).
Сумма "--- взяли все суммы векторов из $A$ и $B$.
Если исходные конечномерны, то $\dim (U+W) + \dim(U\cap W) = \dim U + \dim W$.
Доказательство: взяли базис пересечения, дополнили до каждого, объединили эти два базиса, проверили, что образует базис для $U+W$ (независимость и система образующих).

\section{} % 13
Линейность: $f(au+bv)=af(u)+bf(v)$ (тогда линейное отображение).
Аддитивность: $f(u+v)=f(u)+f(v)$ (следует оттуда).
Однородность степени один: $f(au)=af(u)$.
Свойства: $f(0)=0$ (так как $f(0u)=0f(u)=0$).

\section{} % 14
Образ $\Im$ "--- те, кто получается как результат.
Ядро $\ker$ "--- те, кто переходят в ноль.
Тривиальное ядро "--- только из нуля.
$\Im f$ и $\ker f$ "--- подпространства.

\section{} % 15
Инъективность $\iff \ker f = \{0\}$.
$\Ra$ так как прообраз у нуля ровно один.
$\La$ от противного: взяли два разных прообраза, вычли, получили нетривиальное ядро.

\section{} % 16
$f \colon U \to V$ изоморфизм, если биекция и $f$ линейно (обозначается $U \cong V$).
Линейное отображение изоморфизм $\iff$ $\ker f = \{0\}$ и $\Im f = V$.
Если $U \cong V$ и $\dim U < \infty$ (и есть базис), то образ базиса есть базис $V$.
Тогда два конечномерных изоморфны $\iff$ размерности равны ($\Ra$: см. чуть раньше; $\La$: перевели один базис в другой, проверили свойства).
Тогда $\dim U = n$ $\Ra$ $U \cong K^n$.
В общем случае (без доказательства) совпадать должны мощности базисов.

\section{} % 17
И для бесконечномерных тоже:
$Hom(U, V)$, $\mathcal{L}(U, V)$ "--- пространство линейных отображений $U \to V$.
Это векторное пространство над $K$ (надо проверить два свойства).
Гомоморфизм определяется значениями на базисных векторах.
Для конечномерных можно написать матрицу $a_{ij}$ в каких-то базисах $u_i$ b $v_j$: $f(u_i) = \sum a_{ji} v_j$ ($i$-й столбец "--- значение на соответствующем базисном), $Au' = v'$.
Можно ввести очевидный изоморфизм между матрицами и линейными отображениями, надо проверить, что он изоморфизм.

\section{} % 18
Есть $U \stackrel{g}{\to} V \stackrel{f}{\to}W$, $g$, $f$ "--- линейные отображения, всё конечномерное, тогда если матрицы $A$ и $B$ соответствуют $f$ и $g$,
то $f \circ g$ соответствует матрица $AB$ (подставили, проверили).

\section{} % 19
Есть линейное $f \colon U \to V$, есть переход от базисов $u$ к $u'$ (матрица $C$), от $v$ к $v'$ (матрица $D$),
матрица $A$ (отображение $f$ в базисах $u$ и $v$), матрица $A'$ (в базисах $u'$, $v'$.
Тогда $A'=D^{-1}AC$ (расписали честно и получили, что надо).

\section{} % 20
Пусть $\dim U = m$, $\dim V = n$, есть $f \colon U \to V$, тогда в каких-то базисах матрица $f$ (размера $m \times n$) выглядит как нулевая, а в левом верхнем углу "--- единичная некого размера $r$.
Пусть $\dim \ker f = m - r$.
Возьмём базис $\ker f$: $u_{r+1}, \dots, u_m$ , дополнили до базиса $U$.
Потом возьмём $v_i=f(u_i)$ (для $1 \le i \le r$), они линейно независимые, дополнили до базиса $V$.
Теперь покажем, что матрица какая надо (посчитаем координаты образов $u_i$ в новом базисе).
Следствие: для любой матрицы $A\in M(n,m,K)$ можно найти обратимые $C \in M(m,m,K)$ и $D\in M(n,n,K)$ такие, что $D^{-1}AC$ почти диагональна (посмотрели на соответствующее линейное отображение).

\section{} % 21
$\dim \Im f + \dim \ker f = \dim U$ (если $f \colon U \to V$, где $\dim U = m$).
Как в предыдущем: взяли базис $u_i$, разложили все вектора из образа по нему, потом выкинули $u_{r+1}, \dots, u_m$ (они в ядре и превращаются в ноль),
получили порождающее множество $\Im f$ размера $r$ ($f(u_i)$).
Покажем линейную независимость, так как пространство $(u_1, \dots, u_r)$ пересекается с ядром лишь тривиально.
$r + m - r = m$, что и требовалось.

\section{} % 22
Ранг: наибольшее $r$, что есть ненулевой минор размера $r$ (минор "--- определитель квадратной матрицы, составленной из пересечения некоторых $k$ строк и $k$ столбцов).
Строковый ранг: размерность пространства, порождаемого строками.
Столбцовый аналогично.

Докажем, что строковый и обычный совпадают: при элементарных над строками не меняются (нужные миноры всё еще можно найти, невырожденность не меняется от элементарных; пространство не меняется).
Привели к ступенчатому (столбцы трогать нам не надо), пусть есть $r$ ненулевых строк, тогда строковой ранг ровно $r$.
А обычный ранг хотя бы $r$ (выбрали первые $r$ строк и их начальные столбцы), и не больше $r$ (либо у нас всего $r$ строк, либо любая группа из хотя бы $r+1$ строки содержит нулевую).
То есть обычный и строковый совпадают.
Столбцовый ранг $A$ равен строковому $A^T$, обычный от транспозиции неменяется.
Следствие: строковой не меняется даже при преобразованиях столбцов.
Из доказательства есть метод поиска ранга: привели к ступенчатому, посчитали строки.
Следствие: $\rang \begin{pmatrix}1&*\\0&C\end{pmatrix} = 1 + \rang C$ (и симметрично).

Замечание: $\dim \Im f = \rang A$ (независимо от базиса), так образ "--- это пространство, натянутое на столбцы $A$, т.е. столбцовый ранг.
Теперь есть ранг линейного отображения.
Еще замечания: ранг равен порядку $E$ в почти диагональном виде, критерий разрешимости систему уравнений: ранг расширенной матрицы равен рангу матрицы системы.

\section{} % 23
Пусть $A \in M(n, m, K)$ ($n$ строк, $m$ столбцов) и $B \in M(m, l, K)$, тогда
$\rang A + \rang B - m \le rang AB \le \{ \rang A, \rang B \}$.
Строки произведения "--- это линейные комбинации строк $B$, то есть пространство, ими порождаемое, не больше пространства, порожденного строками $B$ ($\rang AB \le \rang B$).
Столбцы аналогично: $\rang AB \le \rang A$.
Возьмём $f\colon K^l \to K^m$: $f(x)=Bx$, и $g(y)=Ay \colon K^m \to K^n$.
Тогда $[g \circ f] = A \cdot B$, пусть $u = \Im f$, положим $h = g|_U$.
Имеем линейное преобразование $h \colon U \to V$, знаем, что $\dim \Im h + \dim \ker h = \dim U$.
Дальше балуемся с рангом матрицы и размерностями образов/ядер (для $f_0 \colon X \to Y$: $\dim \Im f_0 + \dim \ker f_0 = \dim X$):
$\rang AB
= \dim \Im (g \circ f)
= \dim \Im h
= \dim U - \dim \ker h
\ge \dim \Im f - \dim \ker g
= \dim \Im f - (m - \dim \Im g)
= \dim \Im f + \dim \Im g - m
= \rang A + \rang B - m
$

\section{} % 24
Элементарная диагональная ($E$, на диагонали в одном месте $c \neq 0$),
трансвекция ($E+\lambda e_{ij}$, $i \neq j$),
транспозиция (поменяли в $E$ две строки местами).
Преобразование $X$ над строками $A$ даёт в результате $XA$.
А $AX$ даёт преобразование над столбцами.
Теорема: если $\det A \neq 0$, то $A$ есть произведение элементарных диагональных и трансвекций.
Доказательство: привели к приведённому ступенчатому виду, на диагонали ненулевые, почти успех, обратные к элементарным "--- элементарные, успех.
Следствие: можно элементарными над строками и столбцами привести к почти диагональному виду.

Для Евклидовых колец (а не полей): на диагональных элементарных матрицах надо, чтобы $c \in R^*$.
А матрицу $A$ тогда можно привести к почти диагональному, а на диагонали стоят элементы $d_i$, причём $d_i \mid d_{i+1}$ (например: 1, 2, 4, 12, 24, \dots).
Без доказательства (было на практике: надо искать минимальный с Евклидовой нормой в подматрице, вытаскивать, что-то занулять по Гауссу и Евклиду, потом индукцию).

Системы линейных уравнений над евклидовыми кольцами (не надо и было без доказательства):
$AX=B$ разрешимо $\iff$ ранг обычной и расширенной совпадают, а также совпадают их $\delta_i$ для всех $i$ ($\delta_i$ "--- НОД всех миноров порядка $i$).
Указание к доказательству: $\delta_i$ не меняется при элементарных, можно привести $A$ к почти единичному виду (как в предыдущем абзаце).

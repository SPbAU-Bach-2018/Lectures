\section{} % 21
Взяли перестановки, ввели знак перестановки (через инверсии, есть геометрический смысл <<наклон отрезка между выбранными элементами>>), ввели определитель.

\section{} % 22
Транспонирование не меняет ничего (нарисовали, что такое <<наклон отрезка>>), значит, строки и столбцы равноправны.
Меняем соседние строки местами: геометрически поменялись только отрезки между этими строчками, сменился знак.
Транспозиция строк: нечётное число транспозиций соседних (туда и обратно).
Если две строки одинаковы, то переставим, знак поменялся $\Ra$ ноль.

\section{} % 23
Миноры: вычеркнули строку и столбец, посчитали определитель.
Алгебраическое дополнение: минор на знак (сумма номеров строки и столбца).
Если транспонируем, алг.доп. тоже транспонируются.
Если в $i$-й строчке все нули (кроме $a_{ij}$), то можно разложить по этому элементу через алгебраическое дополнение.

\section{} % 24
Можно разложить по строке: сумма элементов на их алг.доп.
Если в разложении по строке взять алг.доп. из другой строки, будет ноль.
Если строчку умножить на $c$, определитель умножится на $c$.
Есть две пропорциональные строки $\Ra$ ноль.
Если строчку разложим на сумму двух строк, можно разложить так же определитель (из предыдущих очевидно).

\section{} % 25
\[
\begin{vmatrix}
1 & x_1 & x_1^2 & \dots x_1^{n-1} \\
1 & x_2 & x_2^2 & \dots x_2^{n-1} \\
1 & x_3 & x_3^2 & \dots x_3^{n-1} \\
\vdots & \vdots & \vdots & \ddots & \vdots \\
1 & x_{n-1} & x_{n-1}^2 & \dots x_{n-1}^{n-1} \\
\end{vmatrix}
=\prod_{i > j} (x_i - x_j)
\]
Занунили последнюю строчку, разложили по ней, вынесли множители $\prod (x_i - x_n)$, получили рекурсию.

\section{} % 26
\[
\begin{vmatrix}
A_1 & 0 \\
B   & A_2
\end{vmatrix}
= |A_1| \cdot |A_2|
\]
Индукция по порядку матрицы $A_1$.
Можно обобщить на матрицу из нескольких диагональных клеток.

\section{} % 27
Крамер: заменили $i$-й столбец в матрице системы столбцом свободных, посчитали определитель, поделили на определитель системы.
Теорема: если определитель системы не ноль, есть ровно одно решение, по формулам.
Сначала докажем единственность: домножим строчки на алгебраические дополнения первого столбца, получим формулу для $x_1$, аналогично для $x_k$.
Корректность <<в лоб>>.

\section{} % 28
Вводим минор порядка $k$ для прямоугольной матрицы (выбрали какие-то $k$ строк/столбцов, считаем определитель).
Если все миноры порядка $k$ нули, то и большего порядка тоже нули.
Ранг: максимальный порядок минора такой, что есть ненулевой.
Если везде нули, то ранг ноль.

\section{} % 29
Можно умножать строчку на число, прибавлять другую строчку с коэффициентом, переставлять строки местами
(и со столбцами то же).
Это обратимо и перестановка выражается через первые два.
В лоб можно показать, что при преобразованиях ранг не меняется (так как занулённость миноров получается равносильной).
Можно привести к трапецевидной (сначала на диагонали единицы, а ниже нули).
Если приписываем нулевую строчку, ранг сохраняется.
Если приписываем какую-то строчку, ранг увеличивается не больше, чем на единицу.

\section{} % 30
Записали прямоугольную матрицу системы и расширенную матрицу (приписали справа свободные члены).
Пусть ранги $r_A$ и $r_B$.
Теорема: совместность системы (существование решения) $\iff r_A=r_B$.
$\Ra$: сделали в матрице $B$ новый столбец нулём элементарными преобразованиями, ой, отличаются только столбцом нулей.
$\La$: возьмём $r_A$ уравнений, покажем равносильность существования их решения чему надо.
Если существует решение большей, то оно же и для меньшей.
В другую сторону: занулим в расширенной матрице первые $r_A$ строк последнего столбца, ранг сохранился.
Значит, любой минор порядка $r_A+1$ нулевой, возьмём такой: первые $r_A$ строк и любая, разложим по последнему столбцу.
Теперь покажем, что у меньшей системы есть решение.
Если $r_A=n$, то квадратная матрица и всё ок.
Если $r_A<n$, то придадим последним любые значения, остальные выразятся.
Доказали.
Отсюда следствие про однородные системы: чтобы было нетривиальное решение, надо маленький ранг (т.е. нулевой определитель)

\section{} % 31
\[
\begin{vmatrix}
A & 0 \\
-E & B \\
\end{vmatrix}
= |A| \cdot |B|
\]
Преобразуем левую часть так, чтобы кусок $B$ занулился, над ним получим матрицу $C$, у неё определитель равен определителю $C$.
Теперь посмотрим на $C$ и увидим, что оно в точности по формуле перемножения.

\section{} % 32
Обратна матрица обратна с двух сторон (потом увидим, что односторонних не бывает).
Взаимная матрица "--- заменили элементы на алг.доп. и потом транспонировали.
В лоб покажем, что $A \tilde A = \tilde A A = |A| \cdot E$.
Поделим, видимо, что если определитель не ноль (матрица неособенная), то есть решение, причём двухстороннее.
Отсюда следует единственность.

\section{} % 33
$AX=\lambda X$ "--- собственное число и вектор-столбец.
Сделали соответствующую систему матрицы однородной, обозначили $\lambda$ переменной, приравняли определитель нулю (хотим нетривиальное решение).
Получили характеристический многочлен степени $n$ (надо показать, что старший коэффициент не ноль).

\section{} % 34
По построению из предыдущего.
Для поиска векторов надо решать систему.

\section{} % 35
Взяли характеристический многочлен $\phi(t)$, подставили вместо $t$ матрицу $A$ (умножать-то умеем), посчитали, получили, внезапно, ноль.
Док-во: давайте разрешим класть в ячейки не только числа, но и многочлены.
Потом рассмотрим матрицу $B(t)=A-Et$, по определению $|B|=\phi$.
Пусть $\phi(t)=a_0+a_1t+\dots+a_nt^n$.
Составим $\tilde B = B_0+B_1t+B_2t^2+\dots+B_{n-1}t^{n-1}$ (степени не больше $n-1$, так как исходно всё линейно, а миноры есть произведения $n-1$ члена).
Теперь помним, что $B \tilde B = \Delta E = \phi(t) E$.
Получили систему $AB_0 = a_0E; AB_1-B_0=a_1E; AB_2-B_1=a_2E; \dots$.
Домножили на степени $A$, сложили.
Слева занулится, справа получим $\phi(A)$.

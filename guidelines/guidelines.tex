\ifdefined\GeometryForAbstract%
\documentclass[10pt,a4paper,notitlepage]{report}
\else%
\documentclass[12pt,a4paper,notitlepage]{report}
\fi
\usepackage{polyglossia}
\usepackage{fontspec}
\usepackage{amsmath, amssymb, amsthm, extarrows, esint, tensor}
\usepackage{xcolor}
\usepackage[russian]{hyperref}
\usepackage{indentfirst}
\usepackage{ifthen}
\usepackage{mathtools}
\usepackage{datetime}
\usepackage{enumitem}
\usepackage{changepage}
\usepackage{calc}
\usepackage{fancyhdr}
\usepackage{lastpage}
\usepackage{wrapfig}
\usepackage{array}
\ifdefined\GeometryForAbstract%
\usepackage[top=0.5cm,bottom=0.5cm,left=1cm,right=1cm,marginparwidth=0.5cm,marginparsep=0cm,headsep=0cm,includehead,includefoot]{geometry}%
\else%
\usepackage[left=1cm,right=1cm,top=2cm,bottom=2cm]{geometry}%
\fi
\usepackage[math-style=ISO,vargreek-shape=unicode]{unicode-math}
\usepackage{gnuplottex}
\usepackage{tikz}


\usepackage{metalogo}
\usepackage{verbatim}
\usepackage{minted}


\setdefaultlanguage[spelling=modern,babelshorthands=true]{russian}
\setotherlanguage{english}

\defaultfontfeatures{Ligatures={TeX}}
\setmainfont{CMU Serif}
\setsansfont{CMU Sans Serif}
\setmonofont{CMU Typewriter Text}  
\setmathfont{Latin Modern Math}
\newcommand{\Setminus}{\mathbin{\backslash}}

\DeclareSymbolFont{cyrletters}{\encodingdefault}{\familydefault}{m}{it}
\newcommand{\makecyrmathletter}[1]{%
  \begingroup\lccode`a=#1\lowercase{\endgroup
  \Umathcode`a}="0 \csname symcyrletters\endcsname\space #1
}
\count255="409
\loop\ifnum\count255<"44F
  \advance\count255 by 1
  \makecyrmathletter{\count255}
\repeat
%% Simpy adds cyrillic to maths!

\frenchspacing

\def\la{\leftarrow}
\def\ra{\rightarrow}
\def\lra{\leftrightarrow}
\def\La{\Leftarrow}
\def\Ra{\Rightarrow}
\def\Lra{\Leftrightarrow}
\def\lrh{\leftrightharpoons}
\def\btu{\bigtriangleup}

\def\N{\mathbb{N}}
\def\Z{\mathbb{Z}}
\def\Q{\mathbb{Q}}
\def\R{\mathbb{R}}
\def\C{\mathbb{C}}
\def\d{\mathup{d}}

\def\LraDef{\stackrel{\mathrm{Def}}{\Lra}}
\def\eqDef{\stackrel{\mathrm{Def}}{=}}

\renewcommand{\le}{\leqslant}
\renewcommand{\ge}{\geqslant}
\renewcommand{\leq}{\leqslant}
\renewcommand{\geq}{\geqslant}

\DeclareMathOperator{\Int}{int}
\DeclareMathOperator{\cl}{cl}
\DeclareMathOperator{\diam}{diam}
\DeclareMathOperator{\Dom}{Dom}
\DeclareMathOperator{\coDom}{coDom}
\DeclareMathOperator{\Char}{char}
\DeclareMathOperator{\Arg}{Arg}
\AtBeginDocument{\let\Re\relax}
\AtBeginDocument{\newcommand{\Re}{\mathop{\mathrm{Re}}\nolimits}}
\AtBeginDocument{\let\Im\relax}
\AtBeginDocument{\newcommand{\Im}{\mathop{\mathrm{Im}}\nolimits}}
\newcommand{\emod}[1]{\mathop{\equiv}\limits_{#1}}
\newcommand{\Choose}[2]{{\left(#1 \atop #2\right)}}

\renewcommand{\thechapter}{\Roman{chapter}}
\renewcommand{\thesection}{\thechapter.\arabic{section}}

\theoremstyle{definition}
\newtheorem{theorem}{Теорема}[section]
\newtheorem{assertion}{Утверждение}[section]
\newtheorem{lemma}{Лемма}[section]

\newtheorem{Def}{$\mathfrak{Def}$}[section]

\theoremstyle{remark}
\newtheorem{conseq}{Следствие}[theorem]
\newtheorem{conseq*}{Следствие}
\newtheorem{Rem}{Замечание}[section]
\newtheorem{exmp}{Пример}[section]

\newcommand{\proofbegin}{$\blacktriangleright$}
\newcommand{\proofend}{$\blacktriangleleft$}

% http://tex.stackexchange.com/a/67740
\renewenvironment{proof}{%
	\begin{adjustwidth}{0pt}{\widthof{\proofend}}%
	\begin{description}[labelwidth=\widthof{\proofbegin},leftmargin=!]%
	\item[\proofbegin]%
}{%
    \hfill\makebox[0pt][l]{\proofend}
	\end{description}%
	\end{adjustwidth}%
}

\setcounter{MaxMatrixCols}{40}
\makeatletter
\def\tableofcontents{\section*{\contentsname}\@starttoc{toc}}
\makeatother

\gdef\LectureName{}

\AtBeginDocument{\newcommand{\BigHeader}[3]{%
	\thispagestyle{plain}
	\begin{center}
	\textbf{\Huge #1} \\
	\vspace{1.5em}
	\textbf{\large #2} \\
	\vspace{0.5em}
	\textbf{\large #3} \\
	\vspace{0.5em}
	{Собрано: \today~\currenttime} \\
	\end{center}

	\gdef\LectureName{#1}

    \hrulefill

    {
    \let\clearpage\relax
    \tableofcontents
    }
    \clearpage
}}

\pagestyle{fancyplain}
\fancyhf{}
\renewcommand{\headrulewidth}{0.4pt}
\renewcommand{\footrulewidth}{0.4pt}
\renewcommand{\plainheadrulewidth}{0pt}
\renewcommand{\plainfootrulewidth}{0.4pt}

\lhead{\fancyplain{}{\LectureName}}
\rhead{\fancyplain{}{\rightmark}}

\cfoot{\thepage~из~\totalpages}

\newmintinline[cinl]{c}{} %\c is defined :(
\newmintinline[cpp]{cpp}{}
\newmintinline[python]{python}{}
\newmintinline[bash]{bash}{}
\newmintinline[make]{make}{}
\newmintinline[java]{java}{}
\newmintinline[tex]{tex}{}
\setminted{obeytabs,tabsize=4,linenos}
\newminted{c}{}
\newminted{cpp}{}
\newminted{python}{}
\newminted{bash}{}
\newminted{make}{}
\newminted{java}{}
\newminted{tex}{}

\renewcommand{\thesection}{\arabic{section}}
\newcommand{\ok}{& \textcolor{green!60!black}{Правильно}}
\newcommand{\bad}{& \textcolor{red}{Неправильно}}

\begin{document}
\BigHeader
	{Guidelines}{по вёрстке конспектов в \XeLaTeX}
	{Авторы: Лапшин Дмитрий, Егор Суворов}

\section{Введение}

Это руководство "--- наша попытка унифицировать вид электронных конспектов.
Мы написали рекомендации, которые кажутся нам достаточно разумными.
Конечно, в некоторых местах мы можем ошибаться и другие варианты оформления могут оказаться логичнее.
В целях единообразия мы просим вас в таком случае сообращать об этой идее всем остальным,
обсуждать, добавлять её в этот список рекомендаций и только потом использовать.
С любыми вопросами и предложениями по поводу конспектов и улучшению организации
процесса вы можете обращаться к Егору и Диме.
Если у вас возникло какое-то нетривиальное в оформлении место "--- тоже обращайтесь,
сверстаем вместе и добавим в эти рекомендации.
Обратите внимание, что рекомендации даны с учётом нашего шаблона для конспектов, при
других настройках \XeLaTeX\,результаты вёрстки наших примеров могут отличаться.

\section{Оформление}

\begin{enumerate}
\item
	Не стоит как-то руками выставлять пробелы (в том числе неразрывные \tex'~') и
	переводы строк (не используем \tex'\\' вне таблиц, где эта комбинация является частью синтаксиса).
	Просто аккуратно делите на абзацы.
	Также не стоит использовать \tex`\vspace` и похожие команды, которые выставляют отступы.

\item
	Математические символы, записи следует оборачивать в \tex'$...$'.
	Также следует оборачивать запятые и прочие символы пунктуации, если они
	идут в одном <<математическом предложении>>.
	Если пунктуация скорее относится к русскому языку, чем к математическому выражению, то её стоит вынести за доллары.
	Примеры:
	\begin{center}\begin{tabular}{|c|c|c|}
		\hline Это... & ...верстается так & \\
		\hline \tex!Имеется n слагаемых! & Имеется n слагаемых \bad \\
		\hline \tex!Имеется $n$ слагаемых! & Имеется $n$ слагаемых \ok \\
		\hline \tex!с графиком $\Gamma_f,$ обозначим! & с графиком $\Gamma_f,$ обозначим \bad \\
		\hline \tex!с графиком $\Gamma_f$, обозначим! & с графиком $\Gamma_f$, обозначим \ok \\
		\hline \tex!пусть f, g непрерывны! & пусть f, g непрерывны \bad \\
		\hline \tex!пусть $f, g$ непрерывны! & пусть $f, g$ непрерывны \bad \\
		\hline \tex!пусть $f$, $g$ непрерывны! & пусть $f$, $g$ непрерывны \ok \\
		\hline \tex!где a, b $\in \R$! & где a, b $\in \R$ \bad \\
		\hline \tex!где $a$, $b \in \R$! & где $a$, $b \in \R$ \bad \\
		\hline \tex!где $a$, $b$ $\in \R$! & где $a$, $b$ $\in \R$ \bad \\
		\hline \tex!где $a, b \in \R$! & где $a, b \in \R$ \ok \\
		\hline
	\end{tabular}\end{center}

	В примере ниже это критично (хотя таких ситуаций лучше вообще избегать):
	\begin{center}\begin{tabular}{|c|c|l|}
		\hline \tex!$a < b, c < d$! & $a < b, c < d$ & Двойное неравенство на $b$, $c$ \\
		\hline \tex!$a < b$, $c < d$! & $a < b$, $c < d$ & Два независимых неравенства \\
	\hline
	\end{tabular}\end{center}

\item
	В обычном тексте ставьте пробелы после знаков препинания, но не ставьте перед, тогда переносы будут производиться
	в адекватных местах.
	\begin{center}\begin{tabular}{|c|c|c|}
		\hline \tex!Если погода будет хорошая, мы пойдем в лес Да.! \bad \\
		\hline \tex!Если погода будет хорошая, мы пойдем в лес .Да.! \bad \\
		\hline \tex!Если погода будет хорошая, мы пойдем в лес . Да.! \bad \\
		\hline \tex!Если погода будет хорошая , мы пойдем в лес. Да.! \bad \\
		\hline \tex!Если погода будет хорошая, мы пойдем в лес.Да.! \bad \\
		\hline \tex!Если погода будет хорошая,мы пойдем в лес. Да.! \bad \\
		\hline \tex!Если погода будет хорошая,мы пойдем в лес.Да.! \bad \\
		\hline \tex!Если погода будет хорошая, мы пойдем в лес. Да.! \ok \\
		\hline
    \end{tabular}\end{center}

\item
	Выключные формулы (которые отдельно от текста), если их несколько подряд, набирайте в \tex'\begin{gather*}':
	\begin{center}\begin{tabular}{|c|c|}
	\hline
		\begin{minipage}{8cm}
			\begin{texcode}
				\begin{gather*}
				\int x \d x = \frac{x^2}{2} + c \\
				\int x^2 \d x = \frac{x^3}{3} + c
				\end{gather*}
			\end{texcode}
		\end{minipage}
		&
		\begin{minipage}{8cm}
			\begin{gather*}
				\int x \d x = \frac{x^2}{2} + c \\
				\int x^2 \d x = \frac{x^3}{3} + c
			\end{gather*}
		\end{minipage} \\
		\hline
	\end{tabular}\end{center}

	Перед ним не стоит разрывать параграф пустой строкой!
	Аналогично не следует разрывать параграф пустой строкой между несколькими подряд идущими
	\text{gather*} и другими окружениями "--- будет слишком много пустого места.

\item
	Если хотим поставить одну выключную формулу, то вместо \tex!$$...$$! используем \tex!\[...\]!.

\item
	Математику в условии теоремы, утверждения пишем в тексте (\tex'$...$'), а вывод "--- уже выделенно (\tex'\begin{gather*}'):
	\begin{center}\begin{tabular}{|c|c|}
		\hline
		Это... \\\hline
		\begin{minipage}{11.5cm}
			\begin{texcode}
				\begin{theorem}[Теорема Больцано "--- Коши]
					$f\colon [a, b] \ra \R$, $f$ "--- непрерывна
					на $[a, b]$. Тогда:
					\begin{equation*}
						\forall C\in [f(a), f(b)], \exists
						c \in (a, b)\colon f(c) = C
					\end{equation*}
				\end{theorem}
			\end{texcode}
		\end{minipage}
		\\\hline
		... верстается так \\\hline
		\begin{minipage}{11.5cm}
			\begin{theorem}[Теорема Больцано "--- Коши]
				$f\colon [a, b] \ra \R$, $f$ "--- непрерывна
				на $[a, b]$. Тогда:
				\begin{equation*}
					\forall C\in [f(a), f(b)], \exists
					c \in (a, b)\colon f(c) = C
				\end{equation*}
			\end{theorem}
		\end{minipage}
		\\\hline
	\end{tabular}\end{center}

\item
	Последовательности кванторов отделяем запятыми и \tex'\colon' (в зависимости от квантора):
	\begin{center}\begin{tabular}{|c|c|c|}
		\hline Это... & ...верстается так & \\
		\hline \begin{minipage}{0.5\linewidth}
			\begin{texcode}
				$\forall a \exists b
				\forall c P(A, B, C)$
			\end{texcode}
		\end{minipage} & $\forall a \exists b \forall c: P(A, B, C)$ \bad \\
		\hline \begin{minipage}{0.5\linewidth}
			\begin{texcode}
				$\forall a: \exists b:
				\forall c: P(A, B, C)$
			\end{texcode}
		\end{minipage} & $\forall a: \exists b: \forall c: P(A, B, C)$ \bad \\
		\hline \begin{minipage}{0.5\linewidth}
			\begin{texcode}
				$\forall a\colon \exists b\colon
				\forall c\colon P(A, B, C)$
			\end{texcode}
		\end{minipage} & $\forall a\colon \exists b\colon \forall c\colon P(A, B, C)$ \bad \\
		\hline \begin{minipage}{0.5\linewidth}
			\begin{texcode}
				$\forall a, \exists b\colon
				\forall c, P(A, B, C)$
			\end{texcode}
		\end{minipage} & $\forall a, \exists b\colon \forall c, P(A, B, C)$ \ok \\
		\hline \begin{minipage}{0.5\linewidth}
			\begin{texcode}
				$\exists a\colon \forall b,
				\exists c\colon P(A, B, C)$
			\end{texcode}
		\end{minipage} & $\exists a\colon \forall b, \exists c\colon P(A, B, C)$ \ok \\
		\hline
	\end{tabular}\end{center}

\item
	Текст посреди математики всегда включаем в \tex'\text'.
	Иначе вы его даже не увидите, или увидите сильно не то!

\item
	Многоточия.
	\tex!\ldots! "--- это многоточия снизу, \tex!\cdots! "--- посередине, \tex!\dots! автоматически выбирает нужное.
	При этом \tex!...! распознаётся как многоточие \textit{только} в text-mode,
	поэтому наша рекомендация "--- везде использовать \tex!\dots!.
	\begin{center}\begin{tabular}{|c|c|c|}
		\hline Это... & ...верстается так & \\
		\hline \tex!Ого...! & Ого... \ok \\
		\hline \tex!Ого\dots! & Ого\dots \ok \\
		\hline \tex!Ого\ldots! & Ого\ldots \ok \\
		\hline \tex!$a_1+...+a_n$! & $a_1+...+a_n$ \bad \\
		\hline \tex!$a_1+\dots+a_n$! & $a_1+\dots+a_n$ \ok \\
		\hline \tex!$a_1+\ldots+a_n$! & $a_1+\ldots+a_n$ \bad \\
		\hline \tex!$a_1+\cdots+a_n$! & $a_1+\cdots+a_n$ \ok \\
		\hline \tex!$a_1\cdot...\cdot a_n$! & $a_1\cdot ... \cdot a_n$ \bad \\
		\hline \tex!$a_1\cdot\dots\cdot a_n$! & $a_1\cdot\dots\cdot a_n$ \ok \\
		\hline \tex!$a_1\cdot\ldots\cdot a_n$! & $a_1\cdot\ldots\cdot a_n$ \ok \\
		\hline \tex!$a_1\cdot\cdots\cdot a_n$! & $a_1\cdot\cdots\cdot a_n$ \bad \\
		\hline
	\end{tabular}\end{center}

\item
	Не используйте псевдографику, не заменяйте одни символы другими.
	Не забывайте, что одни и те же команды имеют разный смысл в math-mode и в text-mode (и могут иметь разные пробелы, соответственно):
	\begin{center}\begin{tabular}{|c|c|c|}
		\hline Это... & ...верстается так & \\
		\hline \tex!A --> B! & A --> B \bad \\
		\hline \tex!A $-->$ B! & A $-->$ B \bad \\
		\hline \tex!A \rightarrow B! & A \rightarrow B \bad \\
		\hline \tex!A $\rightarrow$ B! & A $\rightarrow$ B \ok \\
		\hline \tex!A ==> B! & A ==> B \bad \\
		\hline \tex!A $==>$ B! & A $==>$ B \bad \\
		\hline \tex!A \Rightarrow B! & \bad \\
		\hline \tex!A $\Rightarrow$ B! & A $\Rightarrow$ B \ok \\
		\hline \tex!$\Sigma_l^r x_n$! & $\Sigma_l^r x_n$ \bad \\
		\hline \tex!$\sum_l^r x_n$! & $\sum_l^r x_n$ \ok \\
		\hline \tex!$\sqcap_l^r x_n$! & $\sqcap_l^r x_n$ \bad \\
		\hline \tex!$\prod_l^r x_n$! & $\prod_l^r x_n$ \ok \\
		\hline \tex!$o(||h||^2)$! & $o(||h||^2)$ \bad \\
		\hline \tex!$o(\|h\|^2)$! & $o(\|h\|^2)$ \ok \\
		\hline
	\end{tabular}\end{center}

\item
	Не используйте программистские обозначения в формулах, используйте математические:
	\tex!\cdot! для умножения чисел, \tex!\times! для декартова произведения.
	\begin{center}\begin{tabular}{|c|c|c|}
		\hline Это... & ...верстается так & \\
		\hline \tex!$2*n$! & $2*n$ \bad \\
		\hline \tex!$2n$! & $2n$ \ok \\
		\hline \tex!$2\cdot n$! & $2\cdot n$ \ok \\
		\hline \tex!$2\times n$! & $2\times n$ \bad \\
		\hline \tex!$[a,b]\cdot [c,d]$! & $[a,b]\cdot [c,d]$ \bad \\
		\hline \tex!$[a,b]\times [c,d]$! & $[a,b]\times [c,d]$ \ok \\
		\hline
	\end{tabular}\end{center}

\item
	Используйте окружения для единого стиля:
	\begin{enumerate}
		\item \tex'theorem' для тела теорем, у которых может быть название (не для доказательств!).
		\item \tex'conseq' для следствий. Они сами нумеруются, не вписывайте все следствия в одно.
		\item \tex'lemma' для тел лемм.
		\item \tex'proof' для доказательств чего угодно.
		\item \tex'assertion' для утверждений.
		\item \tex'Def' для определений.
		\item \tex'Rem' для замечаний.
		\item \tex'exmp' для примеров.
	\end{enumerate}

\item
	Части доказательства и шаги индукции мы набираем окружением \tex'\begin{description}' вот так:
	\begin{center}\begin{tabular}{|c|c|}
		\hline
		\begin{minipage}{8cm}
			\begin{texcode}
				Докажем по индукции:
				\begin{description}
					\item[База:]
						Для $n = 1$ всё очевидно

					\item[Переход:]
						Пусть для $k$ всё доказано...
				\end{description}
			\end{texcode}
		\end{minipage}
		&
		\begin{minipage}{7cm}
			Докажем по индукции:
			\begin{description}
				\item[База:] Для $n = 1$ всё очевидно
				\item[Переход:] Пусть для $k$ всё доказано...
			\end{description}
		\end{minipage} \\
		\hline
	\end{tabular}\end{center}

\item
	Некоторые правила набора символов
	\begin{enumerate}
	\item
		Черта в множествах "--- \tex'\mid', а не \tex'|'.
		\begin{center}\begin{tabular}{|c|c|c|}
			\hline Это... & ...верстается так & \\
			\hline \tex'$\{n | 2 \div n\}$' & $\{n | nx > y\}$ \bad \\
			\hline \tex'$\{n \mid 2 \div n\}$' & $\{n \mid nx > y\}$ \ok \\
			\hline
		\end{tabular}\end{center}

	\item
		Двоеточие в определении отношения и функции "--- \tex'\colon', а не \tex':'.
		\begin{center}\begin{tabular}{|c|c|c|}
			\hline Это... & ...верстается так & \\
			\hline \tex'$f: \R \ra \R$' & $f:\R\ra\R$ \bad \\
			\hline \tex'$f\colon \R \ra \R$' & $f\colon\R\ra\R$ \ok \\
			\hline
		\end{tabular}\end{center}

	\item
		Нестрогие равенства "--- \tex'\le' и \tex'\ge', а не \tex'\leq' и \tex'\geq'.
		Вообще говоря, мы переопределили \tex'\le', чтобы она делала \tex'\leqslant'.
		Используем первую как короткую.
		\begin{center}\begin{tabular}{|c|c|c|}
			\hline Это... & ...верстается так & \\
			\hline \tex'$a \le b$' & $a \le b$ \ok \\
			\hline \tex'$a \leq b$' & $a \leq b$ \bad \\
			\hline \tex'$a \leqslant b$' & $a \leqslant b$ \bad \\
			\hline
		\end{tabular}\end{center}

	\item
		Черта над одной буквой "--- \tex'\bar', а не \tex'\overline'.
		При этом если буква имеет индексы, черта всё равно ставится только над буквой.
		Над выражением "--- \tex'\overline'.
		\begin{center}\begin{tabular}{|c|c|c|}
			\hline Это... & ...верстается так & \\
			\hline
			\begin{minipage}{8cm}
				\begin{texcode}
					$\overline{z_1z_2} =
					\overline{z_1}\overline{z_2}$
				\end{texcode}
			\end{minipage}
			& $\overline{z_1z_2} = \overline{z_1}\overline{z_2}$ \bad \\
			\hline \tex'$\overline{z_1z_2}=\bar{z_1} \bar{z_2}$' & $\overline{z_1z_2} = \bar{z_1} \bar{z_2}$ \bad \\
			\hline \tex'$\overline{z_1z_2}=\bar z_1 \bar z_2$' & $\overline{z_1z_2} = \bar z_1 \bar z_2$ \ok \\
			\hline \tex'$\overline{z_1z_2}=\bar{z}_1 \bar{z}_2$' & $\overline{z_1z_2} = \bar z_1 \bar z_2$ \ok \\
			\hline
		\end{tabular}\end{center}

	\item
		Треугольные скобки "--- \tex'\left< \right>' или хотя бы \tex'\langle \rangle', но никак не \tex'< >'!
		\begin{center}\begin{tabular}{|c|c|c|}
			\hline Это... & ...верстается так & \\
			\hline \tex'$<G, \cdot>$' & $<G, \cdot>$ \bad \\
			\hline \tex'$\langle G, \cdot \rangle$' & $\langle G, \cdot \rangle$ \ok \\
			\hline \tex'$\left<G, \cdot \right>$' & $\left<G, \cdot \right>$ \ok \\
			\hline \tex'$<a, \cfrac23>$' & $<a, \cfrac23>$ \bad \\
			\hline \tex'$\langle a, \cfrac23\rangle$' & $\langle a, \cfrac23\rangle$ \bad \\
			\hline \tex'$\left<a, \cfrac23\right>$' & $\left<a, \cfrac23\right>$ \ok \\
			\hline
		\end{tabular}\end{center}

	\item
		Тире ставим так: \tex'"---', окружая пробелами (не минусиками "--- это дефисы!!!).
		\begin{center}\begin{tabular}{|c|c|c|}
			\hline Это... & ...верстается так & \\
			\hline \tex'Слово - не воробей' & Слово - не воробей \bad \\
			\hline \tex'Слово"---не воробей' & Слово"---не воробей \bad \\
			\hline \tex'Слово "--- не воробей' & Слово "--- не воробей \ok \\
			\hline
		\end{tabular}\end{center}

	\item
		Используемые математические символы (обратите внимание, обычно они задаются наоборот, мы упростили жизнь):
		\begin{center}\begin{tabular}{|c|c|c|}
			\hline Это... & ...верстается так & \\
			\hline \tex'$\varepsilon$' & $\varepsilon$ \bad \\
			\hline \tex'$\epsilon$' & $\epsilon$ \ok \\
			\hline \tex'$\varphi' & $\varphi$ \bad \\
			\hline \tex'$\phi$' & $\phi$ \ok \\
			\hline
		\end{tabular}\end{center}

	\item
		Набор длинных стрелок с подписями идёт с помощью комманд \tex|\xlongrightarrow| и аналогичных:
		\begin{center}\begin{tabular}{|c|c|c|}
			\hline Это... & ...верстается так & \\
			\hline \tex'$\stackrel{ab=1}{\Ra}$' & $\stackrel{ab=1}{\Ra}$ \bad \\
			\hline \tex'$\xLongrightarrow{ab=1}$' & $\xLongrightarrow{ab=1}$ \ok \\
			\hline \tex'$\xLongrightarrow[cd=3]{ab=1}$' & $\xLongrightarrow[cd=3]{ab=1}$ \ok \\
			\hline \tex'$\xrightarrow{f}$' & $\xrightarrow{f}$ \ok \\
			\hline
		\end{tabular}\end{center}

	\item
		Сложные индексы (перед выражением или если их много) набираем с помощью \tex'\tensor':
		\begin{center}\begin{tabular}{|c|c|c|}
			\hline Это... & ...верстается так & \\
			\hline \tex'$a {}_R \mid b$' & $a {}_R \mid b$ \bad \\
			\hline \tex'$a \tensor[_R]{\mid}{} b$' & $a \tensor[_R]{\mid}{} b$ \ok \\
			\hline \tex'${}_1^2_3^4 X_5^6_7^8$' & \bad \\
			\hline \tex'$\tensor[_1^2_3^4]{X}{_5^6_7^8}$' & $\tensor[_1^2_3^4]{X}{_5^6_7^8}$ \ok \\
			\hline \tex'$\tensor*{X}{*_5^{66}_{77}^8}$' & $\tensor*{X}{*_5^{66}_{77}^8}$ \ok \\
			\hline
		\end{tabular}\end{center}

	\end{enumerate}

\item
	Используем команды \tex'\textbf{}' и \tex'\textit{}' вместо \tex'{\bf ...}' и \tex'{\it ...}'.

\item
	Чтобы в больших формулах скобки подстраивались по размеру под внутренности
	используем \tex'\left(...\right)'
	(команды \tex'\left' и \tex'\right' должны образовывать правильную скобочную последовательность,
	они <<вытягивают>> соответствующие символы по вертикали):
	\begin{center}\begin{tabular}{|c|c|c|}
		\hline Это... & ...верстается так & \\
		\hline \tex'\[(a+b)^2\]'                        & \begin{minipage}{3cm}\[(a+b)^2\]\end{minipage} \ok \\
		\hline \tex'\[\left(a+b\right)^2\]'             & \begin{minipage}{3cm}\[\left(a+b\right)^2\]\end{minipage} \ok \\
		\hline \tex'\[(\frac{a+b}{c+d})^2\]'            & \begin{minipage}{3cm}\[(\frac{a+b}{c+d})^2\]\end{minipage} \bad \\
		\hline \tex'\[\left(\frac{a+b}{c+d}\right)^2\]' & \begin{minipage}{3cm}\[\left(\frac{a+b}{c+d}\right)^2\]\end{minipage} \ok \\
		\hline
	\end{tabular}\end{center}

\item
	Будьте осторожны с верхними и нижними индексами и фигурными скобками вокруг них:
	\begin{center}\begin{tabular}{|c|c|c|}
		\hline Это... & ...верстается так & \\
		\hline \tex'a_f(x)'      & $a_f(x)$ \bad \\
		\hline \tex'a_{f(x)}'    & $a_{f(x)}$ \ok \\
		\hline \tex'a_f^(n)'     & $a_f^(n)$ \bad \\
		\hline \tex'a_f^{(n)}'   & $a_f^{(n)}$ \bad \\
		\hline \tex'a_{f^{(n)}}' & $a_{f^{(n)}}$ \ok \\
		\hline \tex'a^5'         & $a^5$ \ok \\
		\hline \tex'a^{5}'       & $a^{5}$ \ok \\
		\hline \tex'a^100'       & $a^100$ \bad \\
		\hline \tex'a^{100}'     & $a^{100}$ \ok \\
		\hline \tex'F_g_+'       & \bad \\
		\hline \tex'F_{g_+}'     & $F_{g_+}$ \ok \\
		\hline
	\end{tabular}\end{center}

\item
	Для выравнивания нескольких формул в столбики вокруг (не)равенства
	мы используем окружения \t{align*}/\t{aligned}.
	Последний может быть использован только в math mode, а окружение \t{align*}
	"--- только в text mode, оно просто включает внутри себя math и делает \t{aligned}.
	Пример использования:

	\begin{center}\begin{tabular}{|c|c|}
		\hline Это... & ...верстается так \\
		\hline \begin{minipage}{0.5\linewidth}
			\begin{texcode}
				\[\begin{aligned}
				    a &= b & c &= 0 + d \\
				0 + a &= b & c &= d \\
				\end{aligned}\]
			\end{texcode}
		\end{minipage}
        &
		\begin{minipage}{0.35\linewidth}
			\[\begin{aligned}
			    a &= b & c &= 0 + d \\
			0 + a &= b & c &= d \\
			\end{aligned}\]
		\end{minipage} \\
		\hline \begin{minipage}{0.5\linewidth}
			\begin{texcode}
				\begin{align*}
				    a &= b + 0 & c &= 0 + d \\
				0 + a &= b & 0 + c &= d \\
				\end{align*}
			\end{texcode}
		\end{minipage}
        &
		\begin{minipage}{0.35\linewidth}
			\[\begin{aligned}
			    a &= b + 0 & c &= 0 + d \\
			0 + a &= b & 0 + c &= d \\
			\end{aligned}\]
		\end{minipage} \\
		\hline
	\end{tabular}\end{center}

	Обратите внимание, что это окружение ориентировано на выравнивание
	нескольких столбцов выражений "--- между соседними добавляется небольшое пространство.
	На более низком уровне эти окружения представляю собой таблицу, нечётные столбцы
	которой выравниваются вправо, а чётные "--- влево.
	Таким образом знак \tex'&' надо ставить \textit{перед} оператором (не)равенства.

\item
	Для более сложного выравнивания (например, процесса выкладок) можно использовать
	окружения \t{alignat*}/\t{alignedat} (суффикс \t{at} ведёт себя так же).
	В этом случае пространство между соседними <<уравнениями>> не добавляется и можно
	считать, что у нас просто есть механизм выравнивания неявно заданной таблицы с чётным
	числом столбцов (как и раньше, они выравниваются в разные стороны в зависимости от чётности).
	При этом требуется явно указать число таких пар столбцов как параметр.
	Например:
	\begin{texcode}
		\begin{alignat*}{2}
		y &= \sin^2 ax \cos bx = \\
		  &= (\sin ax \sin ax ) \cos bx =  \\
		  &= \frac 1 2 (\cos(ax-ax)-\cos(2ax)) \cos bx = \\
		  &= \frac {1-cos(2ax)} 2 \cos bx \neq \\
		  &\begin{alignedat}{4}
		   &\neq \frac 1 2 \cos bx &+& (-0.5) &\cdot& (0&\cdot&0+1) = \\
		   &=    \frac 1 2 \cos bx &+& (-0.25) &\cdot& [0.5 &+& 0.5] \neq \\
		   &\neq \frac 1{239} \cos bx &-& 0.25 &\cdot& 0.5 &-& 100 \\
		   \end{alignedat} \\
		\end{alignat*}
	\end{texcode}
	будет свёрстано в:
	\begin{alignat*}{2}
	y &= \sin^2 ax \cos bx = \\
	  &= (\sin ax \sin ax ) \cos bx =  \\
	  &= \frac 1 2 (\cos(ax-ax)-\cos(2ax)) \cos bx = \\
	  &= \frac {1-cos(2ax)} 2 \cos bx \neq \\
	  &\begin{alignedat}{4}
	   &\neq \frac 1 2 \cos bx &+& (-0.5) &\cdot& (0&\cdot&0+1) = \\
	   &=    \frac 1 2 \cos bx &+& (-0.25) &\cdot& [0.5 &+& 0.5] \neq \\
	   &\neq \frac 1{239} \cos bx &-& 0.25 &\cdot& 0.5 &-& 100 \\
	   \end{alignedat} \\
	\end{alignat*}
	Обратите внимание, что такие окружения можно вкладывать.
	Впрочем, иногда использование окружение \t{array} может оказаться проще.

\item
	Картинки мы пока что рисуем от руки, фотографируем и ссылаемся на них в тексте:
	<<как видно	на моей фотографии с именем \t{IMG\_2359}>>.
	В репозиторий не кладём, планируем завести Dropbox для этих целей, так как git
	хранит везде всю историю всех файлов, поэтому большие файлы сильно раздуют
	объём папочки (даже удалённые).

\item
	Для набора кода используем команды пакета \t{minted}:
	\begin{enumerate}
	\item
		Набор кода в строке (название классов, методов, операторы) используется аналог команды \tex'\verb':
		\tex'\cpp' для плюсов, \tex'\java' для Java и так далее.
		\begin{center}\begin{tabular}{|c|c|c|}
			\hline Это... & ...верстается так & \\
			\hline \tex'std::sort<int>'        & std::sort<int>        \bad \\
			\hline \tex"\verb'std::sort<int>'" & \tex'std::sort<int>' \bad \\
			\hline \tex"\cpp'std::sort<int>'"  & \cpp'std::sort<int>'  \ok  \\
			\hline
		\end{tabular}\end{center}

	\item
		Набор кусков кода уже делаем в аналоге окружения \tex'verbatim':
		\tex'cppcode' для плюсов, \tex'javacode' для Java и так далее:
		\begin{center}\begin{tabular}{|c|c|c|}
			\hline Это... & ...верстается так & \\
			\hline \begin{minipage}{0.4\linewidth}
				\begin{texcode}
					void main() {
					    cout
					        << "Hello, World!"
					        << endl;
					    return 1;
					}
				\end{texcode}
			\end{minipage} & \begin{minipage}{0.4\linewidth}
				void main() {
				    cout
				        << "Hello, World"
				        << endl;
				    return 1;
				}
			\end{minipage} \bad \\
			\hline \begin{minipage}{0.4\linewidth}
				\begin{texcode}
					\begin{verbatim}
					void main() {
					    cout
					        << "Hello, World!"
					        << endl;
					    return 1;
					}
					\end{verbatim}
				\end{texcode}
			\end{minipage} & \begin{minipage}{0.4\linewidth}
\begin{verbatim}
void main() {
    cout
        << "Hello, World!"
        << endl;
    return 1;
}
\end{verbatim}
			\end{minipage} \bad \\
			\hline \begin{minipage}{0.4\linewidth}
				\begin{texcode}
					\begin{cppcode}
					void main() {
					    cout
					        << "Hello, Wolrd!"
					        << endl;
					    return 1;
					}
					\end{cppcode}
				\end{texcode}
			\end{minipage} & \begin{minipage}{0.4\linewidth}
				\begin{cppcode}
					void main() {
					    cout
					        << "Hello, World!"
					        << endl;
					    return 1;
					}
				\end{cppcode}
			\end{minipage} \ok \\
			\hline
		\end{tabular}\end{center}
	\end{enumerate}
	\textbf{Важно:} в последнем случае будут корректно работать отстубы табами. Используйте их!

\item
	Чтобы избежать некрасивого расположения пунктов списка, который начинается сразу в окружении (например, в \t{exmp}), используйте \tex'\hfill'.
	При этом если между списком и началом окружения имеется какой-то текст, то \text'\hfill' не нужен.
	\begin{center}\begin{tabular}{|c|c|c|}
		\hline Это... & ...верстается так & \\
		\hline \begin{minipage}{0.4\linewidth}
			\begin{texcode}
				\begin{exmp}
					\begin{enumerate}
					\item Раз
					\item Два
					\end{enumerate}
				\end{exmp}
			\end{texcode}
		\end{minipage}
        &
		\begin{minipage}{0.35\linewidth}
				\begin{exmp}
					\begin{enumerate}
					\item Раз
					\item Два
					\end{enumerate}
				\end{exmp}
		\end{minipage} \bad \\
		\hline \begin{minipage}{0.4\linewidth}
			\begin{texcode}
				\begin{exmp} \hfill
					\begin{enumerate}
					\item Раз
					\item Два
					\end{enumerate}
				\end{exmp}
			\end{texcode}
		\end{minipage}
        &
		\begin{minipage}{0.35\linewidth}
			\begin{exmp} \hfill
				\begin{enumerate}
				\item Раз
				\item Два
				\end{enumerate}
			\end{exmp}
		\end{minipage} \ok \\
		\hline
	\end{tabular}\end{center}
\end{enumerate}

\section{Особенности правописания}

\begin{enumerate}
\item
	Формулы с комментариями могут быть понятнее, чем просто формулы.
	Не стесняйтесь комментировать (например, кратко указать смысл
	дальнейших действий) "--- потом будет проще читать.
	Кто быстрее набирает, чем пишет от руки "--- пользуйтесь!

\item
	Не допускаются сокращения, кроме <<и т. п.>> и <<и т. д.>> (а эти выражения не рекомендуется использовать).
	Следует писать полностью (<<то есть>>, <<так как>>).

\item Используем букву Ё!
	\begin{center}\begin{tabular}{|c|c|}
		\hline Винни-Пух и всё-всё-всё \bad \\
		\hline Винни-Пух и все-все-все \ok \\
		\hline Ежики \bad \\
		\hline Ёжики \ok \\
		\hline Они все знают \ok \\
		\hline Они всё знают \ok \\
		\hline
	\end{tabular}\end{center}

\item
	Избегайте написания цифрами коротких числительных.
	Сокращения "--- через дефис (\tex!-!, не тире!), максимально краткие (нельзя начинать с гласных!):
	\begin{center}\begin{tabular}{|c|c|c|}
		\hline Рассмотрим 2 случая: \bad \\
		\hline Рассмотрим два случая: \ok \\
		\hline 1-й и 2-й \bad \\
		\hline первый и второй \ok\\
		\hline 123-ый и 239-ый \bad \\
		\hline 123"---й и 239"---й \bad \\
		\hline 123-ый и 239-ый \bad \\
		\hline 123-ого и 239-ого \bad \\
		\hline 123-й и 239-й \ok \\
		\hline 123-го и 239-го \ok \\
		\hline
	\end{tabular}\end{center}

\item
	Заголовки: без точек в конце, с большой буквы. Например: <<Интегралы. Их свойства>>
\end{enumerate}

\section{Исходный код}

\begin{enumerate}
\item
	В исходном файле каждое предложение начинаем в новой строке.
	Можно одно бить на несколько строк, если одно длинное.
	Удобно с точки зрения редактирования и отслеживания изменений.
	Также просим соблюдать отступы и делать их \textbf{табами}, смотрите в качестве примера исходный код этого файла.

	\begin{center}\begin{tabular}{|c|}
	\hline
	До \\
	\hline
	\begin{minipage}{0.9\textwidth}
		\begin{texcode}
			Архимед родился в Сиракузах, греческой колонии на
			острове Сицилия. Отцом Архимеда был математик и
			астроном Фидий, состоявший, как утверждает Плутарх,
			в близком родстве с Гиероном II, тираном Сиракуз.
			Отец привил сыну с детства любовь к математике,
			механике и астрономии. Для обучения Архимед отправился в
			Александрию Египетскую "--- научный и культурный
			центр того времени.
		\end{texcode}
	\end{minipage}
	\\
	\hline После \\
	\hline
	\begin{minipage}{0.9\textwidth}
		\begin{texcode}
			Архимед родился в Сиракузах, греческой колонии на острове Сицилия.
			Отцом Архимеда был математик и астроном Фидий, состоявший, как
			утверждает Плутарх, в близком родстве с Гиероном II, тираном Сиракуз.
			Отец привил сыну с детства любовь к математике,
			механике и астрономии.
			Для обучения Архимед отправился в Александрию Египетскую "---
			научный и культурный центр того времени.
		\end{texcode}
	\end{minipage} \\
	\hline
	\end{tabular}\end{center}

\item
	Отступы и пустые строки делаются так:
	\begin{enumerate}
	\item
		Разделители, как \tex'\section', \tex'\chapter' и прочие не порождают отступа,
		ограждаются одной пустой строкой с каждой стороны.

	\item
		Все окружения, кроме \tex'\begin{document}', порождают отступ внутри себя.
		Если вы используете цепочку вложеных окружений, то можно сделать отступ один раз.
		\begin{texcode}
			\begin{center}\begin{tabular}
			    %и поехали)
			\end{tabular}\end{center}
		\end{texcode}

	\item
		Если вы используете короткие строки в \tex'\item',
		то пишите в одну строку и отступите перед \tex'\item', пустых строк не нужно.
		Иначе \tex'\item' не сдвигается относительно \tex'enumerate', \tex'itemize' и \tex'description',
		содержимое пишется со следующей строки со сдвигом, а перед следующим \tex'item'-ом ставится пустая строка:
		\begin{texcode}
			\begin{itemize}
			    \item 1
			    \item 2
			    \item 3
			\end{itemize}
			\begin{itemize}
			\item
			    1
			    1

			\item
			    2
			    2
			\end{itemize}
		\end{texcode}
	\item
		Если внутри многострочных окружений приходится переноситсь строчку в исходном файле, то желательно добавить отступ:
		\begin{texcode}
			\begin{gather*}
			    2 + 2 = 4 \\
			    2 + 3
			        = 6
			\end{gather*}
		\end{texcode}

	\end{enumerate}

\item
	Не делаем строки существенно длиннее сотни символов.
	Длинные следует разбивать на несколько подряд идущих, места разбиения выбираем по смыслу, а не абы как (это не влияет на вёрстку).
	Это, опять же, упрощает отслеживание изменений.

\item
	Если надо сделать таблицу с преимущественно текстом, используем окружение \t{tabular}.
	Если надо сделать таблицу с математикой, то можно использовать окружение \t{array}
	внутри math mode "--- оно аналогично \t{tabular} по использованию.

\item
	Любой файл должен заканчиваться переводом строки, чтобы было удобно выделять файл
	методом <<поднялись наверх, зажали Shift, спустились вниз>>

\item
	Если у символа есть несколько имён, старайтесь использовать более точное в данном контексте,
	используйте наши сокращения.
	Это не только позволит избежать ошибок с похожими символами, но и сделает код более читаемым.
	\begin{center}\begin{tabular}{|c|c|c|}
		\hline Это... & ...верстается так & \\
		\hline \tex'$f\colon A \xrightarrow{} B$' & $f\colon A \xrightarrow{} B$ \bad \\
		\hline \tex'$f\colon A \rightarrow B$' & $f\colon A \rightarrow B$ \bad \\
		\hline \tex'$f\colon A \ra B$' & $f\colon A \ra B$ \bad \\
		\hline \tex'$f\colon A \to B$' & $f\colon A \to B$ \ok \\
		\hline \tex'$a>0 \Leftarrow a \ge 0$' & $a>0 \Leftarrow a \ge 0$ \bad \\
		\hline \tex'$a>0 \La a \ge 0$' & $a>0 \La a \ge 0$ \ok \\
		\hline \tex'$a<b \Lra b>a$' & $a<b \Lra b>a$ \bad \\
		\hline \tex'$a<b \iff b>a$' & $a<b \iff b>a$ \ok \\
		\hline
	\end{tabular}\end{center}
	Если из имени символа понятно, что он предназначен для другой цели "--- не надо его использовать.
	Например, не используйте \tex'\measuredangle' ($\measuredangle$) в качестве значка <<рассмотрим>>,
	пишите это словами "--- лучше читается.

\end{enumerate}
\end{document}

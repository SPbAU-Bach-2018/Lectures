\documentclass[12pt,a4paper]{report}
\usepackage{polyglossia}
\usepackage{fontspec}
\usepackage{amsmath, amssymb, extarrows}
\usepackage{xcolor}
\usepackage[russian]{hyperref}
\usepackage{indentfirst}
\usepackage{ifthen}
\usepackage[left=1cm,right=1cm,top=2cm,bottom=2cm]{geometry}
\usepackage{minted}
\usepackage[math-style=ISO,vargreek-shape=unicode]{unicode-math}

\setdefaultlanguage[spelling=modern,babelshorthands=true]{russian}
\setotherlanguage{english}

\defaultfontfeatures{Ligatures={TeX}}
\setmainfont{CMU Serif}
\setsansfont{CMU Sans Serif}
\setmonofont{CMU Typewriter Text}  
\setmathfont{Latin Modern Math}
\newcommand{\Setminus}{\mathbin{\backslash}}

\DeclareSymbolFont{cyrletters}{\encodingdefault}{\familydefault}{m}{it}
\newcommand{\makecyrmathletter}[1]{%
  \begingroup\lccode`a=#1\lowercase{\endgroup
  \Umathcode`a}="0 \csname symcyrletters\endcsname\space #1
}
\count255="409
\loop\ifnum\count255<"44F
  \advance\count255 by 1
  \makecyrmathletter{\count255}
\repeat
%% Simpy adds cyrillic to maths!

\frenchspacing

\def\la{\leftarrow}
\def\ra{\rightarrow}
\def\lra{\leftrightarrow}
\def\La{\Leftarrow}
\def\Ra{\Rightarrow}
\def\Lra{\Leftrightarrow}
\def\lrh{\leftrightharpoons}
\def\btu{\bigtriangleup}

\def\N{\mathbb{N}}
\def\Z{\mathbb{Z}}
\def\Q{\mathbb{Q}}
\def\R{\mathbb{R}}
\def\C{\mathbb{C}}
\def\d{\mathup{d}}

\def\LraDef{\stackrel{\mathrm{Def}}{\Lra}}
\def\eqDef{\stackrel{\mathrm{Def}}{=}}

\renewcommand{\le}{\leqslant}
\renewcommand{\ge}{\geqslant}

\DeclareMathOperator{\Int}{int}
\DeclareMathOperator{\cl}{cl}
\DeclareMathOperator{\diam}{diam}
\DeclareMathOperator{\Dom}{Dom}
\DeclareMathOperator{\coDom}{coDom}
\DeclareMathOperator{\Char}{char}
\DeclareMathOperator{\Arg}{Arg}
\AtBeginDocument{\let\Re\relax}
\AtBeginDocument{\newcommand{\Re}{\mathop{\mathrm{Re}}\nolimits}}
\AtBeginDocument{\let\Im\relax}
\AtBeginDocument{\newcommand{\Im}{\mathop{\mathrm{Im}}\nolimits}}
\newcommand{\emod}[1]{\mathop{\equiv}\limits_{#1}}
\newcommand{\Choose}[2]{{\left(#1 \atop #2\right)}}

\renewcommand{\thechapter}{\Roman{chapter}}
\renewcommand{\thesection}{\thechapter.\arabic{section}}

\newcounter{theorem}[section]
\renewcommand{\thetheorem}{\thesection.\arabic{theorem}}
\newcommand*{\theoremheader}[1]{%
	\par\refstepcounter{theorem}%
	\textbf{Теорема \thetheorem.\ifthenelse{\equal{#1}{}}{}{ #1.}}%
}
\newenvironment*{theorem}[1]{
	\theoremheader{#1}%
}{%
	\par%
}

\newcounter{conseq}[theorem]
\renewcommand{\theconseq}{\arabic{conseq}}
\newcommand*{\conseqheader}{%
	\par\refstepcounter{conseq}%
	\textit{Следствие \thetheorem.}%
}
\newenvironment*{conseq}{
	\conseqheader%
}{%
	\par
}

\newenvironment{assertion}{%
	\par%
	\textbf{Утверждение.}%
}{%
	\par%
}

\newenvironment{proof}{%
	\par%
	$\blacktriangleright$%
}{%
	\hfill$\blacktriangleleft$%
	\par%
}

\newenvironment{Def}{%
	\par%
	$\mathfrak{Def\colon}$%
}{%
	\par%
}

\newenvironment{Rem}{%
	\par%
	\textit{REM:}%
}{%
	\par%
}

\newenvironment{exmp}{%
	\par%
	\textbf{Пример:}%
}{%
	\par%
}

\setcounter{MaxMatrixCols}{40}

\newmintinline[cinl]{c}{} %\c is defined :(
\newmintinline[cpp]{cpp}{}
\newmintinline[python]{python}{}
\newmintinline[bash]{bash}{}
\newmintinline[make]{make}{}
\setminted{obeytabs,tabsize=4,linenos,texcomments}
\newminted{c}{}
\newminted{cpp}{}
\newminted{python}{}
\newminted{bash}{}
\newminted{make}{}
\usepackage{metalogo}
\usepackage{verbatim}

\renewcommand{\thesection}{\arabic{section}}
\newcommand{\ok}{& \textcolor{green!60!black}{Правильно}}
\newcommand{\bad}{& \textcolor{red}{Неправильно}}

\begin{document}
\BigHeader
	{Guidelines}{по вёрстке конспектов в \XeLaTeX}
	{Авторы: Лапшин Дмитрий, Егор Суворов}

\section{Введение}
Это руководство "--- наша попытка унифицировать вид электронных конспектов.
Мы написали рекомендации, которые кажутся нам достаточно разумными.
Конечно, в некоторых местах мы можем ошибаться и другие варианты оформления могут оказаться логичнее.
В целях единообразия мы просим вас в таком случае сообращать об этой идее всем остальным,
обсуждать, добавлять её в этот список рекомендаций и только потом использовать.
С любыми вопросами и предложениями по поводу конспектов и улучшению организации
процесса вы можете обращаться к Егору и Диме.
Если у вас возникло какое-то нетривиальное в оформлении место "--- тоже обращайтесь,
сверстаем вместе и добавим в эти рекомендации.
Обратите внимание, что рекомендации даны с учётом нашего шаблона для конспектов, при
других настройках \XeLaTeX\,результаты вёрстки наших примеров могут отличаться.

\section{Оформление}

\begin{enumerate}
\item
	Не стоит как-то руками выставлять пробелы (в том числе неразрывные \verb'~') и
	переводы строк (не используем \verb'\\' вне таблиц, где эта комбинация является частью синтаксиса).
	Просто аккуратно делите на абзацы.
	Также не стоит использовать \verb`\vspace` и похожие команды, которые выставляют отступы.
	
\item
	Математические символы, записи следует оборачивать в \verb'$...$'.
	Также следует оборачивать запятые и прочие символы пунктуации, если они
	идут в одном <<математическом предложении>>.
	Если пунктуация скорее относится к русскому языку, чем к математическому выражению, то её стоит вынести за доллары.
	Примеры:
	\begin{center}\begin{tabular}{|c|c|c|}
		\hline Это... & ...верстается так & \\
		\hline \verb!Имеется n слагаемых! & Имеется n слагаемых \bad \\
		\hline \verb!Имеется $n$ слагаемых! & Имеется $n$ слагаемых \ok \\
		\hline \verb!с графиком $\Gamma_f,$ обозначим! & с графиком $\Gamma_f,$ обозначим \bad \\
		\hline \verb!с графиком $\Gamma_f$, обозначим! & с графиком $\Gamma_f$, обозначим \ok \\
		\hline \verb!пусть f, g непрерывны! & пусть f, g непрерывны \bad \\
		\hline \verb!пусть $f, g$ непрерывны! & пусть $f, g$ непрерывны \bad \\
		\hline \verb!пусть $f$, $g$ непрерывны! & пусть $f$, $g$ непрерывны \ok \\
		\hline \verb!где a, b $\in \R$! & где a, b $\in \R$ \bad \\
		\hline \verb!где $a$, $b \in \R$! & где $a$, $b \in \R$ \bad \\
		\hline \verb!где $a$, $b$ $\in \R$! & где $a$, $b$ $\in \R$ \bad \\
		\hline \verb!где $a, b \in \R$! & где $a, b \in \R$ \ok \\
		\hline 
	\end{tabular}\end{center}

	В примере ниже это критично (хотя таких ситуаций лучше вообще избегать):
	\begin{center}\begin{tabular}{|c|c|l|}
		\hline \verb!$a < b, c < d$! & $a < b, c < d$ & Двойное неравенство на $b$, $c$ \\
		\hline \verb!$a < b$, $c < d$! & $a < b$, $c < d$ & Два независимых неравенства \\
	\hline
	\end{tabular}\end{center}

\item
	В обычном тексте ставьте пробелы после знаков препинания, но не ставьте перед, тогда переносы будут производиться
	в адекватных местах.
	\begin{center}\begin{tabular}{|c|c|c|}
		\hline \verb!Если погода будет хорошая, мы пойдем в лес Да.! \bad \\
		\hline \verb!Если погода будет хорошая, мы пойдем в лес .Да.! \bad \\
		\hline \verb!Если погода будет хорошая, мы пойдем в лес . Да.! \bad \\
		\hline \verb!Если погода будет хорошая , мы пойдем в лес. Да.! \bad \\
		\hline \verb!Если погода будет хорошая, мы пойдем в лес.Да.! \bad \\
		\hline \verb!Если погода будет хорошая,мы пойдем в лес. Да.! \bad \\
		\hline \verb!Если погода будет хорошая,мы пойдем в лес.Да.! \bad \\
		\hline \verb!Если погода будет хорошая, мы пойдем в лес. Да.! \ok \\
		\hline
    \end{tabular}\end{center}

\item
	Выключные формулы (которые отдельно от текста), если их несколько подряд, набирайте в \verb'\begin{gather*}':
	\begin{center}\begin{tabular}{|c|c|}
	\hline
		\begin{minipage}{8cm}
\begin{verbatim}
\begin{gather*}
\int x \d x = \frac{x^2}{2} + c \\
\int x^2 \d x = \frac{x^3}{3} + c
\end{gather*}
\end{verbatim}
		\end{minipage}
		&
		\begin{minipage}{8cm}
			\begin{gather*}
				\int x \d x = \frac{x^2}{2} + c \\
				\int x^2 \d x = \frac{x^3}{3} + c
			\end{gather*}
		\end{minipage} \\
		\hline
	\end{tabular}\end{center}

	Перед ним не стоит разрывать параграф пустой строкой!
	Аналогично не следует разрывать параграф пустой строкой между несколькими подряд идущими
	\text{gather*} и другими окружениями "--- будет слишком много пустого места.

\item 
	Если хотим поставить одну выключную формулу, то вместо \verb!$$...$$! используем \verb!\[...\]!.

\item 
	Математику в условии теоремы, утверждения пишем в тексте (\verb'$...$'), а вывод "--- уже выделенно (\verb'\begin{gather*}'):
	\begin{center}\begin{tabular}{|c|c|}
		\hline
		Это... \\\hline
		\begin{minipage}{11.5cm}
\begin{verbatim}
\begin{theorem}{Теорема Больцано "--- Коши}
$f\colon [a, b] \ra \R$, $f$ "--- непрерывна
на $[a, b]$. Тогда:
\[\forall C\in [f(a), f(b)], \exists
c \in (a, b)\colon f(c) = C\]
\end{theorem}
\end{verbatim}
		\end{minipage} 
		\\\hline
		... верстается так \\\hline
		\begin{minipage}{11.5cm}
			\begin{theorem}{Теорема Больцано "--- Коши}
			$f\colon [a, b] \ra \R$, $f$ "--- непрерывна на $[a, b]$. Тогда:
			\[\forall C\in [f(a), f(b)], \exists c \in (a, b)\colon f(c) = C\]
			\end{theorem}
		\end{minipage}
		\\\hline
	\end{tabular}\end{center}

\item
	Последовательности кванторов отделяем запятыми и \verb'\colon' (в зависимости от квантора):
	\begin{center}\begin{tabular}{|c|c|c|}
		\hline Это... & ...верстается так & \\
		\hline \begin{minipage}{0.5\linewidth}
\begin{verbatim}
$\forall a \exists b 
\forall c P(A, B, C)$
\end{verbatim}
		\end{minipage} & $\forall a \exists b \forall c: P(A, B, C)$ \bad \\
		\hline \begin{minipage}{0.5\linewidth}
\begin{verbatim}
$\forall a: \exists b: 
\forall c: P(A, B, C)$
\end{verbatim}
		\end{minipage} & $\forall a: \exists b: \forall c: P(A, B, C)$ \bad \\
		\hline \begin{minipage}{0.5\linewidth}
\begin{verbatim}
$\forall a\colon \exists b\colon 
\forall c\colon P(A, B, C)$
\end{verbatim}
		\end{minipage} & $\forall a\colon \exists b\colon \forall c\colon P(A, B, C)$ \bad \\
		\hline \begin{minipage}{0.5\linewidth}
\begin{verbatim}
$\forall a, \exists b\colon
\forall c, P(A, B, C)$
\end{verbatim}
		\end{minipage} & $\forall a, \exists b\colon \forall c, P(A, B, C)$ \ok \\
		\hline \begin{minipage}{0.5\linewidth}
\begin{verbatim}
$\exists a\colon \forall b,
\exists c\colon P(A, B, C)$
\end{verbatim}
		\end{minipage} & $\exists a\colon \forall b, \exists c\colon P(A, B, C)$ \ok \\
		\hline
	\end{tabular}\end{center}

\item
	Текст посреди математики всегда включаем в \verb'\text'.
	Иначе вы его даже не увидите, или увидите сильно не то!

\item
	Многоточия.
	\verb!\ldots! "--- это многоточия снизу, \verb!\cdots! "--- посередине, \verb!\dots! автоматически выбирает нужное.
	При этом \verb!...! распознаётся как многоточие \textit{только} в text-mode, 
	поэтому наша рекомендация "--- везде использовать \verb!\dots!.
	\begin{center}\begin{tabular}{|c|c|c|}
		\hline Это... & ...верстается так & \\
		\hline \verb!Ого...! & Ого... \ok \\
		\hline \verb!Ого\dots! & Ого\dots \ok \\
		\hline \verb!Ого\ldots! & Ого\ldots \ok \\
		\hline \verb!$a_1+...+a_n$! & $a_1+...+a_n$ \bad \\
		\hline \verb!$a_1+\dots+a_n$! & $a_1+\dots+a_n$ \ok \\
		\hline \verb!$a_1+\ldots+a_n$! & $a_1+\ldots+a_n$ \bad \\
		\hline \verb!$a_1+\cdots+a_n$! & $a_1+\cdots+a_n$ \ok \\
		\hline \verb!$a_1\cdot...\cdot a_n$! & $a_1\cdot ... \cdot a_n$ \bad \\
		\hline \verb!$a_1\cdot\dots\cdot a_n$! & $a_1\cdot\dots\cdot a_n$ \ok \\
		\hline \verb!$a_1\cdot\ldots\cdot a_n$! & $a_1\cdot\ldots\cdot a_n$ \ok \\
		\hline \verb!$a_1\cdot\cdots\cdot a_n$! & $a_1\cdot\cdots\cdot a_n$ \bad \\
		\hline
	\end{tabular}\end{center}

\item 
	Не используйте псевдографику, не заменяйте одни символы другими. 
	Не забывайте, что одни и те же команды имеют разный смысл в math-mode и в text-mode (и могут иметь разные пробелы, соответственно):
	\begin{center}\begin{tabular}{|c|c|c|}
		\hline Это... & ...верстается так & \\
		\hline \verb!A --> B! & A --> B \bad \\
		\hline \verb!A $-->$ B! & A $-->$ B \bad \\
		\hline \verb!A \rightarrow B! & A \rightarrow B \bad \\
		\hline \verb!A $\rightarrow$ B! & A $\rightarrow$ B \ok \\
		\hline \verb!A ==> B! & A ==> B \bad \\
		\hline \verb!A $==>$ B! & A $==>$ B \bad \\
		\hline \verb!A \Rightarrow B! & \bad \\
		\hline \verb!A $\Rightarrow$ B! & A $\Rightarrow$ B \ok \\
		\hline \verb!$\Sigma_l^r x_n$! & $\Sigma_l^r x_n$ \bad \\
		\hline \verb!$\sum_l^r x_n$! & $\sum_l^r x_n$ \ok \\
		\hline \verb!$\sqcap_l^r x_n$! & $\sqcap_l^r x_n$ \bad \\
		\hline \verb!$\prod_l^r x_n$! & $\prod_l^r x_n$ \ok \\
		\hline
	\end{tabular}\end{center}

\item
	Не используйте программистские обозначения в формулах, используйте математические:
	\verb!\cdot! для умножения чисел, \verb!\times! для декартова произведения.
	\begin{center}\begin{tabular}{|c|c|c|}
		\hline Это... & ...верстается так & \\
		\hline \verb!$2*n$! & $2*n$ \bad \\
		\hline \verb!$2n$! & $2n$ \ok \\
		\hline \verb!$2\cdot n$! & $2\cdot n$ \ok \\
		\hline \verb!$2\times n$! & $2\times n$ \bad \\
		\hline \verb!$[a,b]\cdot [c,d]$! & $[a,b]\cdot [c,d]$ \bad \\
		\hline \verb!$[a,b]\times [c,d]$! & $[a,b]\times [c,d]$ \ok \\
		\hline
	\end{tabular}\end{center}

\item 
	Используйте окружения для единого стиля:
	\begin{enumerate}
		\item \verb'theorem' для тела теорем, у которых может быть название (не для доказательств!).
		\item \verb'conseq' для следствий. Они сами нумеруются, не вписывайте все следствия в одно.
		\item \verb'lemma' для тел лемм.
		\item \verb'proof' для доказательств чего угодно.
		\item \verb'assertion' для утверждений.
		\item \verb'Def' для определений.
		\item \verb'Rem' для замечаний.
		\item \verb'exmp' для примеров.
	\end{enumerate}

\item 
	Части доказательства и шаги индукции мы набираем окружением \verb'\begin{description}' вот так:
	\begin{center}\begin{tabular}{|c|c|}
		\hline
		\begin{minipage}{8cm}
\begin{verbatim}
Докажем по индукции:
\begin{description}
\item[База:] Для $n = 1$ всё 
очевидно
\item[Переход:] Пусть для $k$ всё 
доказано...
\end{description}
\end{verbatim}
		\end{minipage}
		&
		\begin{minipage}{7cm}
			Докажем по индукции:
			\begin{description}
				\item[База:] Для $n = 1$ всё очевидно
				\item[Переход:] Пусть для $k$ всё доказано...
			\end{description}
		\end{minipage} \\
		\hline
	\end{tabular}\end{center}

\item
	Некоторые правила набора символов
	\begin{itemize}
	\item 
		Черта в множествах "--- \verb'\mid', а не \verb'|'.
		\begin{center}\begin{tabular}{|c|c|c|}
			\hline Это... & ...верстается так & \\
			\hline \verb'$\{n | 2 \div n\}$' & $\{n | nx > y\}$ \bad \\
			\hline \verb'$\{n \mid 2 \div n\}$' & $\{n \mid nx > y\}$ \ok \\
			\hline
		\end{tabular}\end{center}
	
	\item 
		Двоеточие в определении отношения и функции "--- \verb'\colon', а не \verb':'.
		\begin{center}\begin{tabular}{|c|c|c|}
			\hline Это... & ...верстается так & \\
			\hline \verb'$f: \R \ra \R$' & $f:\R\ra\R$ \bad \\
			\hline \verb'$f\colon \R \ra \R$' & $f\colon\R\ra\R$ \ok \\
			\hline
		\end{tabular}\end{center}

	\item
		Нестрогие равенства "--- \verb'\le' и \verb'\ge', а не \verb'\leq' и \verb'\geq'.
		Вообще говоря, мы переопределили \verb'\le', чтобы она делала \verb'\leqslant'.
		Используем первую как короткую.
		\begin{center}\begin{tabular}{|c|c|c|}
			\hline Это... & ...верстается так & \\
			\hline \verb'$a \le b$' & $a \le b$ \ok \\
			\hline \verb'$a \leq b$' & $a \leq b$ \bad \\
			\hline \verb'$a \leqslant b$' & $a \leqslant b$ \bad \\
			\hline
		\end{tabular}\end{center}
	
	\item 
		Черта над одной буквой "--- \verb'\bar', а не \verb'\overline'.
		При этом если буква имеет индексы, черта всё равно ставится только над буквой.
		Над выражением "--- \verb'\overline'.
		\begin{center}\begin{tabular}{|c|c|c|}
			\hline Это... & ...верстается так & \\
			\hline
			\begin{minipage}{10cm}
\begin{verbatim}
$\overline{z_1z_2} = 
\overline{z_1}\overline{z_2}$
\end{verbatim}
			\end{minipage}
			& $\overline{z_1z_2} = \overline{z_1}\overline{z_2}$ \bad \\
			\hline \verb'$\overline{z_1z_2}=\bar{z_1} \bar{z_2}$' & $\overline{z_1z_2} = \bar{z_1} \bar{z_2}$ \bad \\
			\hline \verb'$\overline{z_1z_2}=\bar z_1 \bar z_2$' & $\overline{z_1z_2} = \bar z_1 \bar z_2$ \ok \\
			\hline \verb'$\overline{z_1z_2}=\bar{z}_1 \bar{z}_2$' & $\overline{z_1z_2} = \bar z_1 \bar z_2$ \ok \\
			\hline
		\end{tabular}\end{center}

	\item 
		Треугольные скобки "--- \verb'\left< \right>' или хотя бы \verb'\langle \rangle', но никак не \verb'< >'!
		\begin{center}\begin{tabular}{|c|c|c|}
			\hline Это... & ...верстается так & \\
			\hline \verb'$<G, \cdot>$' & $<G, \cdot>$ \bad \\
			\hline \verb'\langle G, \cdot \rangle' & $\langle G, \cdot \rangle$ \ok \\
			\hline \verb'\left<G, \cdot \right>' & $\left<G, \cdot \right>$ \ok \\
			\hline \verb'$<a, \cfrac23>$' & $<a, \cfrac23>$ \bad \\
			\hline \verb'\langle a, \cfrac23\rangle' & $\langle a, \cfrac23\rangle$ \bad \\
			\hline \verb'\left<a, \cfrac23\right>' & $\left<a, \cfrac23\right>$ \ok \\
			\hline
		\end{tabular}\end{center}
	
	\item 
		Тире ставим так: \verb'"---', окружая пробелами (не минусиками "--- это дефисы!!!).
		\begin{center}\begin{tabular}{|c|c|c|}
			\hline Это... & ...верстается так & \\
			\hline \verb'Слово - не воробей' & Слово - не воробей \bad \\
			\hline \verb'Слово"---не воробей' & Слово"---не воробей \bad \\
			\hline \verb'Слово "--- не воробей' & Слово "--- не воробей \ok \\
			\hline
		\end{tabular}\end{center}

	\item
		Используемые математические символы:
		\begin{center}\begin{tabular}{|c|c|c|}
			\hline Это... & ...верстается так & \\
			\hline \verb'$\varepsilon$' & $\varepsilon$ \bad \\
			\hline \verb'$\epsilon$' & $\epsilon$ \ok \\
			\hline \verb'$\varphi' & $\varphi$ \bad \\
			\hline \verb'$\phi$' & $\phi$ \ok \\
			\hline
		\end{tabular}\end{center}
	\end{itemize}

\item
	Используем команды \verb'\textbf{}' и \verb'\textit{}' вместо \verb'{\bf ...}' и \verb'{\it ...}'.

\item
	Чтобы в больших формулах скобки подстраивались по размеру под внутренности
	используем \verb'\left(...\right)'
	(команды \verb'\left' и \verb'\right' должны образовывать правильную скобочную последовательность,
	они <<вытягивают>> соответствующие символы по вертикали):
	\begin{center}\begin{tabular}{|c|c|c|}
		\hline Это... & ...верстается так & \\
		\hline \verb'\[(a+b)^2\]'                        & \begin{minipage}{3cm}\[(a+b)^2\]\end{minipage} \ok \\
		\hline \verb'\[\left(a+b\right)^2\]'             & \begin{minipage}{3cm}\[\left(a+b\right)^2\]\end{minipage} \ok \\
		\hline \verb'\[(\frac{a+b}{c+d})^2\]'            & \begin{minipage}{3cm}\[(\frac{a+b}{c+d})^2\]\end{minipage} \bad \\
		\hline \verb'\[\left(\frac{a+b}{c+d}\right)^2\]' & \begin{minipage}{3cm}\[\left(\frac{a+b}{c+d}\right)^2\]\end{minipage} \ok \\
		\hline
	\end{tabular}\end{center}

\item
	Будьте осторожны с верхними и нижними индексами и фигурными скобками вокруг них:
	\begin{center}\begin{tabular}{|c|c|c|}
		\hline Это... & ...верстается так & \\
		\hline \verb'a_f(x)'      & $a_f(x)$ \bad \\
		\hline \verb'a_{f(x)}'    & $a_{f(x)}$ \ok \\
		\hline \verb'a_f^(n)'     & $a_f^(n)$ \bad \\
		\hline \verb'a_f^{(n)}'   & $a_f^{(n)}$ \bad \\
		\hline \verb'a_{f^{(n)}}' & $a_{f^{(n)}}$ \ok \\
		\hline \verb'a^5'         & $a^5$ \ok \\
		\hline \verb'a^{5}'       & $a^{5}$ \ok \\
		\hline \verb'a^100'       & $a^100$ \bad \\
		\hline \verb'a^{100}'     & $a^{100}$ \ok \\
		\hline \verb'F_g_+'       & \bad \\
		\hline \verb'F_{g_+}'     & $F_{g_+}$ \ok \\
		\hline
	\end{tabular}\end{center}

\item
	Для выравнивания нескольких формул в столбики вокруг (не)равенства
	мы используем окружения \texttt{align*}/\texttt{aligned}.
	Последний может быть использован только в math mode, а окружение \texttt{align*}
	"--- только в text mode, оно просто включает внутри себя math и делает \texttt{aligned}.
	Пример использования:

	\begin{center}\begin{tabular}{|c|c|}
		\hline Это... & ...верстается так \\
		\hline \begin{minipage}{0.5\linewidth}
\begin{verbatim}
\[\begin{aligned}
    a &= b & c &= 0 + d \\
0 + a &= b & c &= d \\
\end{aligned}\]
\end{verbatim}
		\end{minipage}
        &
		\begin{minipage}{0.35\linewidth}
\[\begin{aligned}
    a &= b & c &= 0 + d \\
0 + a &= b & c &= d \\
\end{aligned}\]\end{minipage} \\
		\hline \begin{minipage}{0.5\linewidth}
\begin{verbatim}
\begin{align*}
    a &= b + 0 & c &= 0 + d \\
0 + a &= b & 0 + c &= d \\
\end{align*}
\end{verbatim}
		\end{minipage}
        &
		\begin{minipage}{0.35\linewidth}
\[\begin{aligned}
    a &= b + 0 & c &= 0 + d \\
0 + a &= b & 0 + c &= d \\
\end{aligned}\]\end{minipage} \\
		\hline
	\end{tabular}\end{center}

	Обратите внимание, что это окружение ориентировано на выравнивание
	нескольких столбцов выражений "--- между соседними добавляется небольшое пространство.
	На более низком уровне эти окружения представляю собой таблицу, нечётные столбцы
	которой выравниваются вправо, а чётные "--- влево.
	Таким образом знак \verb'&' надо ставить \textit{перед} оператором (не)равенства.

\item
	Для более сложного выравнивания (например, процесса выкладок) можно использовать
	окружения \texttt{alignat*}/\texttt{alignedat} (суффикс \texttt{at} ведёт себя так же).
	В этом случае пространство между соседними <<уравнениями>> не добавляется и можно
	считать, что у нас просто есть механизм выравнивания неявно заданной таблицы с чётным
	числом столбцов (как и раньше, они выравниваются в разные стороны в зависимости от чётности).
	При этом требуется явно указать число таких пар столбцов как параметр.
	Например:
\begin{verbatim}
\begin{alignat*}{2}
y &= \sin^2 ax \cos bx = \\
  &= (\sin ax \sin ax ) \cos bx =  \\
  &= \frac 1 2 (\cos(ax-ax)-\cos(2ax)) \cos bx = \\
  &= \frac {1-cos(2ax)} 2 \cos bx \neq \\
  &\begin{alignedat}{4}
   &\neq \frac 1 2 \cos bx &+& (-0.5) &\cdot& (0&\cdot&0+1) = \\
   &=    \frac 1 2 \cos bx &+& (-0.25) &\cdot& [0.5 &+& 0.5] \neq \\
   &\neq \frac 1{239} \cos bx &-& 0.25 &\cdot& 0.5 &-& 100 \\
   \end{alignedat} \\
\end{alignat*}
\end{verbatim}
будет свёрстано в:
\begin{alignat*}{2}
y &= \sin^2 ax \cos bx = \\
  &= (\sin ax \sin ax ) \cos bx =  \\
  &= \frac 1 2 (\cos(ax-ax)-\cos(2ax)) \cos bx = \\
  &= \frac {1-cos(2ax)} 2 \cos bx \neq \\
  &\begin{alignedat}{4}
   &\neq \frac 1 2 \cos bx &+& (-0.5) &\cdot& (0&\cdot&0+1) = \\
   &=    \frac 1 2 \cos bx &+& (-0.25) &\cdot& [0.5 &+& 0.5] \neq \\
   &\neq \frac 1{239} \cos bx &-& 0.25 &\cdot& 0.5 &-& 100 \\
   \end{alignedat} \\
\end{alignat*}
	Обратите внимание, что такие окружения можно вкладывать.
	Впрочем, иногда использование окружение \texttt{array} может оказаться проще.

\item
	Картинки мы пока что рисуем от руки, фотографируем и ссылаемся на них в тексте:
	<<как видно	на моей фотографии с именем \texttt{IMG\_2359}>>.
	В репозиторий не кладём, планируем завести Dropbox для этих целей, так как git
	хранит везде всю историю всех файлов, поэтому большие файлы сильно раздуют
	объём папочки (даже удалённые).

\end{enumerate}

\section{Особенности правописания}

\begin{enumerate}
\item
	Формулы с комментариями могут быть понятнее, чем просто формулы.
	Не стесняйтесь комментировать (например, кратко указать смысл
	дальнейших действий) "--- потом будет проще читать.
	Кто быстрее набирает, чем пишет от руки "--- пользуйтесь!

\item
	Не допускаются сокращения, кроме <<и т. п.>> и <<и т. д.>> (а эти выражения не рекомендуется использовать).
	Следует писать полностью (<<то есть>>, <<так как>>).
	
\item Используем букву Ё!
	\begin{center}\begin{tabular}{|c|c|}
		\hline Винни-Пух и всё-всё-всё \bad \\
		\hline Винни-Пух и все-все-все \ok \\
		\hline Ежики \bad \\
		\hline Ёжики \ok \\
		\hline Они все знают \ok \\
		\hline Они всё знают \ok \\
		\hline
	\end{tabular}\end{center}

\item
	Избегайте написания цифрами коротких числительных.
	Сокращения "--- через дефис (\verb!-!, не тире!), максимально краткие (нельзя начинать с гласных!):
	\begin{center}\begin{tabular}{|c|c|c|}
		\hline Рассмотрим 2 случая: \bad \\
		\hline Рассмотрим два случая: \ok \\
		\hline 1-й и 2-й \bad \\
		\hline первый и второй \ok\\
		\hline 123-ый и 239-ый \bad \\
		\hline 123"---й и 239"---й \bad \\
		\hline 123-ый и 239-ый \bad \\
		\hline 123-ого и 239-ого \bad \\
		\hline 123-й и 239-й \ok \\
		\hline 123-го и 239-го \ok \\
		\hline
	\end{tabular}\end{center}

\item
	Заголовки: без точек в конце, с большой буквы. Например: <<Интегралы. Их свойства>>
\end{enumerate}

\section{Исходный код}

\begin{enumerate}
\item
	В исходном файле каждое предложение начинаем в новой строке.
	Можно одно бить на несколько строк, если одно длинное.
	Удобно с точки зрения редактирования и отслеживания изменений.
	Также просим соблюдать отступы и делать их \textbf{табами}, смотрите в качестве примера исходный код этого файла.

	\begin{center}\begin{tabular}{|c|}
	\hline
	До \\
	\hline
	\begin{minipage}{0.9\textwidth}
\begin{verbatim}
Архимед родился в Сиракузах, греческой колонии на
острове Сицилия. Отцом Архимеда был математик и
астроном Фидий, состоявший, как утверждает Плутарх,
в близком родстве с Гиероном II, тираном Сиракуз.
Отец привил сыну с детства любовь к математике,
механике и астрономии. Для обучения Архимед отправился в
Александрию Египетскую "--- научный и культурный
центр того времени.
\end{verbatim}
	\end{minipage}
	\\
	\hline После \\
	\hline
	\begin{minipage}{0.9\textwidth}
\begin{verbatim}
Архимед родился в Сиракузах, греческой колонии на острове Сицилия.
Отцом Архимеда был математик и астроном Фидий, состоявший, как
утверждает Плутарх, в близком родстве с Гиероном II, тираном Сиракуз.
Отец привил сыну с детства любовь к математике,
механике и астрономии.
Для обучения Архимед отправился в Александрию Египетскую "---
научный и культурный центр того времени.
\end{verbatim}
	\end{minipage} \\
	\hline
	\end{tabular}\end{center}

\item
	Не делаем строки существенно длиннее сотни символов.
	Длинные следует разбивать на несколько подряд идущих, места разбиения выбираем по смыслу, а не абы как (это не влияет на вёрстку).
	Это, опять же, упрощает отслеживание изменений.

\item
	Если надо сделать таблицу с преимущественно текстом, используем окружение \texttt{tabular}.
	Если надо сделать таблицу с математикой, то можно использовать окружение \texttt{array}
	внутри math mode "--- оно аналогично \texttt{tabular} по использованию.

\item
	Любой файл должен заканчиваться переводом строки, чтобы было удобно выделять файл
	методом <<поднялись наверх, зажали Shift, спустились вниз>>

\end{enumerate}
\end{document}

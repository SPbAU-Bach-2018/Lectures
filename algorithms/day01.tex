\Section{Деревья поиска (начало)}{10 февраля}{Сергей Копелиович}

\Subsection{Не сбалансированное дерево}
              
\begin{Def}{Search tree (дерево поиска)}
 -- бинарное дерево, для каждого поддерева $t$ которого выполняется:
\up
\begin{formula}
  $$(\forall z \in t.left \colon z < t.x) \vee (\forall z \in t.right \colon z > t.x)$$
\end{formula}
\end{Def}

\up % Уменьшить вертикальный отступ
\THE{Хранение дерева}

\begin{code}
  struct node {
    int x;
    node *l, *r;
    node( int x, node *l, node *r ) : x(x), l(l), r(r) { }
  };
\end{code}

\THE{Вставка ключа}

  Первая версия:
\begin{code}
  bool add( node* &t, int x ) { // возвращает true, если элемент добавился 
    if (!t) {
      t = new node(x, 0, 0);
      return 1;
    }
    if (t->x == x)
      return 0;
    return add(x < t->x ? t->l : t->r, x);
  }
\end{code}

  Вторая версия, персистентная:
\begin{code}
  node* add( node* t, int x ) { // t без ссылки
    if (!t) 
      return new node(x, 0, 0);
    if (t->x == x)
      return t;
    if (x < t->x)
      return new node(t->x, add(t->l, x), t->r);
    else
      return new node(t->x, t->l, add(t->r, x));
  }
\end{code}

%\pagebreak\up
\THE{Поиск вершины по ключу}

\begin{code}
  node* & lower_bound( node* &t, int x ) { // min y : y >= x
    if (!t)
      return t;
    if (t->x < x)
      return lower_bound(t->r, x);
    node* &res = lower_bound(t->l, x);
    return res ? res : t;
  }
\end{code}
\begin{code}
  node* & find( node* &t, int x ) { // находит элемент равный $x$
    if (!t || t->x == x)
      return t;
    return find(x < t->x ? t->l : t->r, x);
  }
\end{code}
  Заметим, что обе функции возвращают, ссылку, поэтому
  следующий код находит $x$ и удаляет его вместе со всем поддеревом
\begin{code}
  find(root, x) = 0; // утечка памяти, возможно разименование нулевого указателя
\end{code}

\THE{Правый сосед}

  Спускаемся один раз вправо, затем ``влево до упора''.
\begin{code}
  node*& down( node* &t ) { // возвращает ссылку на вершину
    return t->l ? down(t->l) : t;
  }
  node*& right( node* &t ) { // возвращает ссылку на вершину
    return down(t->r);
  }
\end{code}

\THE{Удаление ключа}

  Если у $v$ нет левого или правого поддерева, удалить вершину $v$ просто.
  Если оба ребёнка присутствуют, находим ближайшего справа и делаем swap ключей с ним.
\begin{code}
  bool delete( node* &v ) { // удаляет конкретную вершину из дерева
    if (!v) return 0;
    if (!v->r) {
      v = v->r; // утечка памяти
    } else {
      node* &neighbour = right(v);
      swap(v->x, neighbour->x);
      neighbour = neighbour->l; // утечка памяти
    }
    return 1;
  }
  bool delete( node* &t, int x ) { // удаляет по ключу
    return delete(find(t, x));
  }
     
\end{code}

\begin{Thm}{Средняя глубина вершины дерева после добавления случайной перестановки из $n$ элементов равна $\O(\log n)$}.
\label{Thm:DepthOfRandomTree}
\begin{proof}
  ?
\end{proof}
\end{Thm}

\THE{Вывод дерева}

Можно из дерева поиска получить за $\O(n)$ упорядоченный массив. Если обходить дерево
слева направо, массив упорядоченный по возрастанию. Если справа налево, упорядоченный по убыванию.

Обход дерева слева направо:
  
\begin{code}
  void output( node* t, vector<int> &result ) { 
    if (!t)
      return;
    output(t->l, result);
    result.push_back(t->x);
    output(t->r, result);
  }
\end{code}

Можно дерево вывести в двухмерном, удобном для чтения виде:

\begin{code}
  void output( node* t, int dep = 0 ) { 
    if (!t)
      return;
    output(t->l, dep + 1);
    printf("%*s%d\n", 2 * dep, "", t->x);
    output(t->r, dep + 1);
  }
\end{code}

\Subsection{AVL-дерево}

\Subsection{Дерево по неявному ключу}

\Subsection{Персистентное дерево}

\Subsection{Декартово дерево}

\begin{Thm}{Если все $x_i$ различны, все $y_i$ различны, то декартово дерево единственно}
\begin{proof}
  По построению
\end{proof}
\end{Thm}

\begin{Thm}{При выборе случайных $y_i$ средняя глубина вершины декартова дерева $\O(\log n)$}
\begin{proof}
  Воспользуемся \autoref{Thm:DepthOfRandomTree}
\end{proof}
\end{Thm}

\begin{Thm}{При выборе случайных $y_i$ матожидание максимума глубин вершин декартова дерева равно $\O(\log n)$}

Пока без доказательства.
\end{Thm}

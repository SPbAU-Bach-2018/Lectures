\setauthor{Егор~Суворов}

\section{27.09}
\subsection{Замечания про ДЗ}
В задаче про пьяницу и обрыв можно заметить $p_2=p_1^2$.

В задаче про слова: разумнее всего разбить нашу бесконечную последовательность
на блоки длины $n$ и считать вероятность встретить наше слово в одном из таких блоков.
Она не больше/меньше(?) вероятности встретить слово хотя бы где-то.
Потом удобно подключить лемму Барреля-Контелли, получаем, что эта вероятность равна единице,
значит, вероятность найти хоть где-то тоже единица.

Задача три: в явном виде вроде вероятности не выписываются.
Напоминание: задача была <<посчитать вероятность того, что в перестановке будет цикл длины хотя бы $k$>>.
Для произвольного $k$ надо писать какие-то рекуррентные соотношения (и не забыть стартовые условия),
и это решением является.
Зато если $k > \sfrac n2$, то мы можем всё явно выписать: сначала выбираем, кто войдёт в цикл длины ровно $k$
и в каком порядке, а потом на остальных элементах идёт произвольная перестановка:
\[ \frac{\binom{n}{k}(k-1)! \cdot (n-k)!}{n!} = \frac{1}{k} \]
Теперь суммируем по всем числам $\ge k$:
\[ P(\text{успех}) = 1 - \frac1n = \frac1{n-1} - \dots - \frac{1}{k+1} \approx 1 - m\frac{n}{k} \]

\subsection{Распределение Коши}
Обсудим следующую задачу: берём прямую, вешаем над ней лампочку на высоте $a$.
Она светит во все стороны равномерно (в нижний полукруг), будем считать, что суммарное излучение лампочки равно единице.
Вопрос: как прямая освещена?

Матмодель будет такая: нарисуем вокруг лампочки нижний полукруг, на нём берётся случайная точка (равномерно),
проводится прямая через лампочку и эту точку, потом пересекается с исходной горизонтальной прямой.
Вопрос: как у нас будет распределена случайная величина <<положение точки на окружности>>?
Нам надо научиться считать для произвольной $x$ массу точек, оказавшихся так на прямой левее $x$,
это будет функция распределения.

Решение: взяли точку $x$.
Вероятность того, что другая точка будет левее, пропорциональна углу дуги к точке $x$.
А этот угол равен углу между горизонтом и направлением от $x$ на лампочку, его можно найти из прямоугольного
треугольника с катетами $(-x, a)$.
Получаем при отрицательных $x$ следующую функцию:
\[ F_{\xi}(x) = \frac{1}{\pi} \arctg \frac{a}{-x} \]
Если продифференцировать, то получаем т.н. распределение Коши:
\begin{gather*}
	p_{\xi}(x) = \frac{1}{\pi} \cdot \frac{a}{a^2+(x-b)^2} \\
	\xi \sim \text{Коши}(b, a)
\end{gather*}
Здесь $a$ "--- \textit{параметр масштаба}, $b$ "--- \textit{параметр сдвига}.
Картинка похожа на нормальное распределение, но они на самом деле сильно отличаются.
В нормальном была экспонента с квадратом, а тут просто квадрат.
На жаргоне распределения вроде Коши зовутся <<распределения с тяжёлыми хвостами>>>.


\subsection{Показательное распределение}
Также зовётся экспоненциальным.
Задаётся через плотность (она не ноль только при $x \ge 0$):
\begin{align*}
	\xi &\sim \text{Показ}(\lambda), \lambda > 0 \\
	p_{\xi}(x) &= \lambda \cdot e^{-\lambda x} \cdot \mathrm{1}_{x \ge 0} \\
	F_{\xi}(x) &= (1 - e^{-\lambda x}) \cdot \mathrm{1}_{x \ge 0} \\
\end{align*}

Чем оно хорошо?
Давайте решим такую задачу: пусть $x, y > 0$, хотим найти $P(\xi > x + y \mid \xi > x)$.
\begin{gather*}
	P(\xi > x + y \mid \xi > x) =
	\frac{P(\xi > x + y)}{P(\xi > x)} =
	\frac{1 - F(x+y)}{1-F(x)} =
	\frac{e^{-\lambda(x+y)}}{e^{-\lambda x}}  =
	e^{-\lambda y} =
	1 - F(y) =
	P(\xi > y)
\end{gather*}
Неплохо, не правда ли?

Физическая интерпретация: пусть у нас есть устройство, которое через $x$ времени ломается.
Тогда если у нас вероятность поломки описывается показательным распределением, то есть
такое свойство: если устройство дожило до $x$, то вероятность поломки к моменту $x+y$ такая же,
как если бы новое устройство сломалось через $y$.
То есть устройство не стареет.
Вот везде, где есть нестареющие процессы, показательное распределение и возникает.
Например, радиоактивный распад.
Кстати, лампочки тоже довольно хорошо описываются показательным распределением.

Дома мы докажем, что если у нас есть нестареющий объект, то мы обязательно придём к показательному распределению.

\subsubsection{Задачка}
Пусть есть величина $\xi$ с показательным распределением.
Рассмотрим величину $\eta$:
\[ \eta = 1 - e^{-\lambda \xi} \]
Вопрос: какое у неё распределение?
Сначала надо расписать руками по определению, а потом догадаться, как можно было без рассчётов.
По определению "--- это написать неравенство на $\eta$, выразить из него неравенство $\xi$, подставить 
$F_{\xi}$, получаем $F_\eta(x) = x$.
Но это верно только при $0 < x < 1$ (это условие мы формально продолбали по ходу дела, но его легко восстановить).
Итого имеем равномерное распределение на отрезке $[0, 1]$.

А на самом деле можно было заметить, что $\eta$ задано просто формулой для $F_\xi$.
И получили мы такой факт:
\[ F_{\xi}(\xi) \sim U[0, 1] \]
На самом деле это верно для любого непрерывного распределения, не только показательного.
Это даже было на лекции, когда мы рассуждали про метод обратной функции, только чуть в другой обёртке.

\subsection{Преобразования и плотность}
Задача: пусть у нас есть функция плотности распределения, а мы захотели найти функцию
плотности распределения $\eta=f(\xi)$.
Для функции распределения у нас всё уже было в предыдущем пункте: расписываем по определению.

Будем решать сразу в многомерном случае.
Пусть есть случайный вектор $\vec\xi$ с плотностью $p_1$ и есть диффеоморфизм (гладкая дифференцируемая биекция) $g$.
Тогда можно рассмотреть величину $\eta=g(\xi)$ и мы хотим понять, какая получится плотность $p_2$.

Решение: мы знаем, что вероятность попасть в какую-нибудь область равна интегралу по плотности в этой области:
\begin{gather*}
	P(q \in B) = \int_B p_{\eta} p_{\eta}(y) \d y \\
	P(\xi \in g^{-1}(B)) = \int_{g^{-1}(B)} p_{\xi}(x) \d x \\
	\text{делаем замену в первом интеграле:} \\
	\int_B p_{\eta} p_{\eta}(y) \d y = \int_{g^{-1}(B)} p_{\eta}(g(x)) |\underbrace{\det g'(x)}_{J}| \d x
\end{gather*}
Мы знаем, что для любого $B$ интегралы равны, тогда получаем совпадение подынтегральных функций почти везде:
\begin{equation}\label{density_change}
	p_{\eta}(y) = \frac{p_{\xi}(g^{-1}(y))}{|\det g'(g^{-1}(y))|}
\end{equation}
\begin{Rem}
	Вероятность события "--- $P$ большое, функция плотности "--- $p$ малое.
\end{Rem}

\begin{exmp}
	$\eta = a\xi \cdot b$, получаем
	\[ p_{\eta}(y) = \frac{p_{\xi}\left(\frac{\eta-b}{a}\right)}{|a|} \]
\end{exmp}
\begin{exmp}
	$\eta = x^3$ (квадрат плохо, потому что не биекция).
	Получаем
	\[ p_{\eta}(y) = \frac{p_{\xi}(\sqrt[3]{y})}{3y^{\sfrac23}} \]
\end{exmp}

\subsubsection{Многомерный пример}
Пусть есть вектор $(\xi_1, \xi_2)$, равномерно распределённый в $[0,1]^2$.
И есть преобразование:
\[
	\begin{cases}
		\eta_1 = \sqrt{-2\ln \xi_1} \cdot \cos (2 \pi \xi_2) \\
		\eta_2 = \sqrt{-2\ln \xi_1} \cdot \sin (2 \pi \xi_2)
	\end{cases}
\]
Подсказка: надо взять два новых распределения $\alpha=2\pi \xi_2$ (равномерное на отрезке $[0, 2\pi]$)
и $\rho = \sqrt{-2\ln \xi_1}$ (у него плотность надо найти).
Они будут полезны.
Плотность для $\rho$ получится будет такая:
\[ p_{\rho}(r) = r^\cdot e^{-\sfrac{r^2}{2}} \cdot \mathrm{1}_{r \ge 0} \]

Домашняя задача: добить в соответствие с формулой \ref{density_change} эту задачу.
Мы сначала перешли $(\xi_1, \xi_2) \to (\alpha, \rho)$, а теперь осталась полярная замена,
Якобиан которой мы знаем (просто $\rho$).
Должно получиться хорошее распределение, которое мы знаем.

\subsection{Ещё домашнее задание}
\begin{enumerate}
	\item Добить предыдущий пункт.
	\item Пусть $\xi$ "--- непрерывная, неотрицательная случайная величина, для которой верно свойство нестарения (из показательного распределения), то она имеет показательное распределение.
	\item Пусть $\xi$ "--- показательное распределение с параметром $\lambda$.
		Надо проверить зависимость целой и дробной части $\xi$: зависят они друг от друга или нет?
		В частности, придётся вычислить их распределение.
\end{enumerate}

\section{13.09}

\begin{itemize}
	\item У нас 2 кубика. Мы их кидаем. Найти безусловные и условные вероятности следующих событий: 
	
	A = "есть 5"
	
	B = "сумма чисел на кубиках делится на 3"
	
	\[ P(A) = \frac{1}{6} + \frac{5}{6}\cdot\frac{1}{6} = \frac{11}{36}\] - либо сразу на первом кубике выпадет 5, либо нет и тогда нас спасёт только второй.
	
	\[P(B) = \frac{1}{3}\] - для любого числа с первого кубика, подходит ровно 2 числа, чтобы сумма делилась на 3(а 2 числа - это 1/3).
	
	\[P(A\cup B) = 2 \frac{1}{6}\cdot\frac{1}{3} = \frac{1}{9}\] - либо первый, либо второй кубик с 5(оба не может быть, так кк сумма на 5 не делится), и остаётся 2 варианта на второй.
	
	\[P(A|B) = \frac{P(AB)}{P(B)} = \frac{1}{3} > P(A)\] - по формуле условной вероятности.
	
	\[P(B|A) = \frac{\frac{1}{9}}{\frac{11}{36}} = \frac{4}{11} > P(B) \]
	
	(рисунок)
	
	\item Теперь кубиков 3. На первом - 5 красных и 1 белая грань, на втором - 3 и 3, на третьем - 1 и 5. Мы выбираем случайный кубик и кидаем его. Какова вероятность выбора каждого кубика при условии, что выпала красная грань?
	
	\begin{tabular}{cc}
		\romannumeral 1 & 5к, 1б \\
		\romannumeral 2 & 3к, 3б \\
		\romannumeral 3 & 5к, 1б \\
	\end{tabular}
	
	\begin{tabular}{c|ccc}
	& \romannumeral 1 & \romannumeral 2 & \romannumeral 3 \\
	выбор кубика без условия & 1/3 & 1/3 & 1/3 \\
	|к & 5/9 & 3/9 & 1/9 \\
	\end{tabular}
	
	Считаем мы это через формулы условной и полной вероятности:
	
	\[\boxed{ P(H_m|A) = \frac{P(H_m)P(A|H_m}{\sum\limits_k P(H_k)P(A|H_k)} }\]
	
	P(\romannumeral 1 | к) = $\frac{1/3 \cdot 5/6}{1/3(5/6 + 3/6 + 1/6)}$.
	
	Теперь усложним задачу. Какая вероятность, если кинули два раза, и выпало кк? А если кб? Мы выбираем кубик в самом начале и навсегда.
	
	\begin{tabular}{c|ccc}
	& \romannumeral 1 & \romannumeral 2 & \romannumeral 3 \\
	без условия & 1/3 & 1/3 & 1/3 \\
	|к & 5/9 & 3/9 & 1/9 \\
	|кк & 5/7 & 9/35 & 1/35 \\
	|кб & 5/19 & 9/19 & 5/19 \\
	|ккб & 25/57 & 27/57 & 5/57 \\
	\end{tabular}	
	
	\[\frac{1/3 \cdot 25/36}{1/3(25/36 + 9/36 + 1/36)} = \frac{5}{7}\]
	
	\item Есть n + 1 урна с шариками с номерами от 0 до n. В k-ой урне есть k белых и n - k чёрных шариков. Сначала выбираем случайно урну, с которой будем работать. А затем последовательно достаём по шарику и кладём обратно. Найти вероятность того, что мы вытащили белый шарик при условии, что до этого было r белых шариков подряд.
	\[P(\text{Б}|\text{было r Б подряд}) =(def) \frac{P(B_{r + 1})}{P(B_r)} = \]
	\[= \frac{\frac{1}{n + 1}(0 + (\frac{1}{n})^{r + 1} + \dots + (\frac{n}{n})^{r + 1})}{\frac{1}{n + 1}(0 + (\frac{1}{n})^r + (\frac{2}{n})^r + \dots + (\frac{n}{n})^r)} \xrightarrow{n \rightarrow \infty} \frac{\int\limits_0^1 x^{r + 1} \d x}{\int\limits_0^1 x^r \d x} = \frac{r + 1}{r + 2}\]
	\item Кидаем монетку. Она честная. A выигрывает, если в какой-то момент подряд будет два герба. B выигрывает, если будет подряд РГР. Посчитает вероятность того, что A выигрывает.
	
	\begin{equation*}
		\begin{cases}
			P(A) = P(Герб)P(A\, выигрывает\, |\, первый\, герб) + P(Решка)P(A\, выигрывает\, |\, первая\, решка) = \\ = 1/2P_\Gamma + 1/2P_P \\
			P_{\Gamma} = 1/2P_{\Gamma\Gamma} + 1/2P_{\Gamma P} = 1/2\cdot 1 + 1/2 \cdot P_P \\
			P_P = 1/2P_{P\Gamma} + 1/2P_{PP} = 1/2P_{P\Gamma} + 1/2P_P \\
			P_{P\Gamma} = 1/2P_{P\Gamma\Gamma} + 1/2P_{P\Gamma P} = 1/2 \\
		\end{cases}
	\end{equation*} - это система уравнений. 4 уравнения - 4 переменные. Дальше - алгебра.
	
	\begin{equation*}
		\begin{cases}
			P(A) = 5/8 \\
			P_{\Gamma} = 3/4 \\
			P_P = 1/2 \\
			P_{P\Gamma} = 1/2 \\
		\end{cases}
	\end{equation*}
	
	Также это можно решить с помощью марковских цепей(по нашему - конечных автоматов): (рисунок).
	
	\item A, B, C - три дуэлянта стреляют друг в друга по очереди - сначала стреляет A, потом B, потом C. Каждый из них может выбрать любого дуэлянта и стрелять в него. Убитые выбывают. Вероятность того, что A попадёт - 0.3, что B попадёт - 1, что C попадёт - 0.5. Какая наивыгодная стратегия для A, если считать, что все стрелки действуют оптимально(чтобы увеличить свои шансы на выживание).
	
	Для A есть несколько вариантов. Понятно, что A выгодно стрелять или в B, или в воздух. C ему убивать вообще невыгодно, потому что в этом случае B будет стрелять в него и точно убьёт. Таким образом есть несколько вариантов:
	
	\begin{enumerate}
		\item A стреляет в B. Пусть попал. Тогда дальше идёт перестрелка с C: 
		\begin{equation*}
			\begin{cases}
				P_A = 0.3 \cdot 1 + 0.7 P_C = 0.3 + 0.35 P_A = \frac{6}{13} \\
				P_C = 0.5 \cdot 0 + 0.5 P_A \\
			\end{cases}
		\end{equation*}
		
		Где буква внизу - чей сейчас ход, но вероятность мы считаем выигрыша A. Поэтому, например, если A попадёт в C, то он выигрывает с вероятностью 1,а если ход C и он попадёт в A, то A выигрывает с вероятностью 0. (картинка)
		\item Не попал в B. Тогда вероятность выигрыша у A равно 0.3. Так как следующим ходом, очевидно, B убьёт C и у A останется один ход, чтобы убить B.
	\end{enumerate}
	Но заметим, что если мы вообще не будем стрелять, то мы получим ту же вероятность 0.3, что мы выиграем. Но $0.3 > \frac{3}{13} \Rightarrow 
	0.3 \frac{3}{13} + 0.7 \cdot 0.3 < 0.3$. Поэтому A выгодно стрелять просто в воздух.
	\item А что теперь, если A применил секретную технику тренировок и стал попадать с вероятностью 0.6? А потом пошёл и всем похвастался? То есть у нас теперь вероятность попадания A возрасла до 0.6, и все это знают. Тогда B выгодно будет стрелять в A(так как у него наибольшая вероятность убить B). А значит и A выгодно теперь стрелять в B, и если он не попал, то гарантированно умрёт. Если он попал то дальше пойдёт перестелка с C. Запишем уравнения:
	\begin{equation*}
			\begin{cases}
				P_A = 0.6 \cdot 1 + 0.4 P_C = \frac{6}{8} \\
				P_C = 0.5 \cdot 0 + 0.5 P_A \\
			\end{cases}
		\end{equation*}			
		Таким образом, A выживает с вероятностью $0.6 \frac{3}{8} = 0.225$. Видно, что несмотря на то, что стрелять A стал лучше, вероятность победы уменьшилась. Можно подсчитать, что вероятность выигрыша B стала равна - 0.2, а выигрыша C - 0.575.
		
		\item Есть n ящиков с номерами от 1 до n. В них разложены случайным образом бумажки от 1 до n. Далее, есть n звключённых, пронумерованных от 1 до n. Каждый человек по очереди заходит в комнату с ящиками и открывает k любых из них. Заключённые не общаются. Цель каждого - открыть ящик, внутри которого бумажка с его номером. Если хотя бы 1 заключённый не достиг цели, то всех казнят. Какая наилучшая стратегия для заключённых?
		
		Ответ такой: сначала человек открывает ящик с его собственным номером, затем с номером, который был написан на бумажке в этом ящике, затем с номером, который лежит в последнем открытом ящике и так далее. Утверждается, что если заключённый не открыл ящика. где лежит его номер - он наткнулся на цикл длины больше k. Утверждается также, что их мало. Найти вероятность успеха заключённых - часть домашней работы.
\end{itemize}

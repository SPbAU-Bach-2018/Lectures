\TODO еще пара процессов случайного блуждания (ничего важного вроде) \TODO

\section{Марковские цепи с непрерывным временем}
У нас по-прежнему есть матрица перехода, только она теперь не от шагов зависит, а от времени: $P_{ij}(t), t \in [0, +\infty)$.

Свойства матрицы перехода:
\begin{enumerate}
\item $\sum\limits_j P_{ij}(t) = 1$
\item $P(t+s) = P(t)P(s)$ -- формула полной вероятности, уравнение Чэмпена-Колмогорова
\item Дополнительное соглашение (ниоткуда не выводится, но принимают за истину): $P_{ij}(t) \xlongrightarrow[t \to +0]{} \mathbb{1}_{i=j}$
\end{enumerate}
 
\begin{theorem}
$P_{ij}(t)$ равномерно непрерывно на $[0, +\infty)$
\end{theorem}
\begin{proof}
\begin{gather*}
P_{ij}(t+h) = P_{ii}(h) P_{ij}(t) + \sum\limits_{k \neq i} P_{ik}(h) P_{kj}(t) \\
1 - P_{ii}(h) = \sum\limits_{k \neq i} P_{ik}(h) \geq P_{ij}(t+h) - P_{ij}(t) = P_{ij}(t)(P_{ii}(h) - 1) + \sum\limits_{k \neq i} \dots \geq \\
\geq P_{ii}(h) - 1 \Ra |P_{ij}(t+h) - P_{ij}(t)| \leq 1 - P_{ii}(h) \to 0
\end{gather*}
\end{proof}

\begin{theorem}
Существуют производные в нуле: 
\begin{gather*}
a_{jj} = \lim\limits_{h \to +0} \frac{P_{jj}(h) - 1}{h} \in (-\infty, 0] \\
a_{ij} = \lim\limits_{h \to +0} \frac{P_{ij}(h)}{h} \in [0, +\infty)
\end{gather*}
\end{theorem}
\begin{proof}
Без доказательства.
\end{proof}

\subsection{Случай конечного числа состояний}

\begin{Def}
$A = (a_{ij})_{i,j=0}^N$ -- инфинитезимальная матрица (генератор процесса), если $\sum\limits_{j=0}^N a_{ij} = 0$ (консервативный процесс)
\end{Def}

$A = \lim\limits_{h \to +0} \frac{P(h) - E}{h} = P'(0)$

$P'(t) = \frac{P(t+h) - P(t)}{h} = P(t) \frac{P(h) - E}{h} \xlongrightarrow[h \to +0]{} P(t) A$ -- прямые ДУ Колмогорова.

Или же, можно вынести $P(t)$ с другой стороны, тогда получится $AP(t)$ -- обратные ДУ Колмогорова.

Если состояний конечное число, то не очень важно, какие уравнения брать, результат будет одинаковым: 
$$P(t) = e^{At} = \sum\limits_{k = 0}^\infty \frac{A^k}{k!}$$

Таким образом, $A$ и начальное распределение $\xi_0$ задают Марковскую цепь: $P(\xi_t = j) = \sum\limits_{k=0}^N P(\xi_0 = k) P_{kj}(t)$

\subsection{Бесконечное число состояний}
В случае консервативного процесса справедливы обратные ДУ Колмогорова (а прямые нет).

Если процесс не консервативен, то ни те ни другие неверны.

\begin{exmp}
Опять численность популяции: $\xi_t$ -- численность в момент времени $t$. $P_{j,j+1}(h) = \lambda_jh + o(h)$.

$A = \begin{matrix}
-\lambda_0 & \lambda_0 & 0 & 0 & \dots \\
\mu_1 & (-\mu_1 - \lambda_1) & \lambda_1 & 0 & \dots \\
0 & \mu_2 & (-\mu_2 - -\lambda_2) & 0 & \dots \\
\dots
\end{matrix}$

Если $\lambda_n = \lambda n, \mu_n = \mu n$ -- линейный рост.

Если $\lambda_n = \lambda n + a$ -- иммиграция.
\end{exmp}
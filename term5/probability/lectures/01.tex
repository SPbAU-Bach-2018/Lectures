\chapter{Теория вероятностей}
\section{Введение, аксиоматика}
Принятые обозначения: 
$\Omega$ -- множество исходов

$\omega \in \Omega$ -- исход (обозначаем маленькой буквой)

$A \subset \Omega$ -- событие (обозначаем большими буквами)

$\mathcal{F}$ -- множество всех событий

Часто мы будем сокращать и вместо $A \cap B$ будем писать просто $AB$.

\begin{Rem}
Не все множества исходов являются событиями, например, плохо, если нельзя измерить его объем.
Может получиться парадокс Банаха-Тарского, когда мы берем, разрезаем один шар на несколько кусков и получаем два точно таких же шара.
\end{Rem}

$P: \mathcal{F} \to [0, 1]$ -- вероятность событий.

\begin{Def}
    $\Omega$ -- произвольное множество. $\mathcal{F}$ замкнуто относительно $\cap, \cup, \setminus$. Тогда $\mathcal{F}$ -- алгебра событий.
\end{Def}
\begin{Def}
    $\mathcal{F}$ -- $\sigma$-алгебра, если $A_k \in \mathcal{F} \Ra \bigcup\limits_{k=1}^{\infty}, \bigcap\limits_{k=1}^{\infty} \in \mathcal{F}$.
\end{Def}
Упражнение: привести пример алгебры, не являющейся $\sigma$-алгеброй.
\begin{exmp}
$\{\varnothing, \Omega \}$ -- тривиальная $\sigma$-алгебра.
\end{exmp}

Пересечение $\sigma$-алгебр тоже $\sigma$-алгебра, а вот объединение -- нет 
(например, в их объединении нет множеств, которые частично лежат в одном, а частично в другом, хотя они должны быть из-за замкнутости).

\begin{Def}
	$\sigma$-оболочка $\mathcal A$ "--- это пересечение всех $\sigma$-алгебр, содержащих $\mathcal A$ (тоже $\sigma$-алгебра):
	\[
		\sigma(\mathcal A) = \bigcap\limits_{{\mathcal F}_x \supseteq \mathcal A} \mathcal F_x
	\]
\end{Def}

Множество $\sigma$-алгебр является решеткой: наибольшей нижней границей $A, B$ будет $A \cap B$, а наименьшей верхней будет $\sigma(A \cap B)$

\begin{Def}
    Тройка $(\Omega, \mathcal{F}, P)$ называется вероятностным пространством (или экспериментом), если $\mathcal{F}$ -- $\sigma$-алгебра и у $P$ выполняются две аксиомы вероятности:
    \begin{enumerate}
        \item $AB = \varnothing \Ra P(A \cup B) = P(A) + P(B)$ (аддитивность)
        \item $P(\Omega) = 1$ (нормировка)
    \end{enumerate}
\end{Def}
Так же есть более сильная аксиома $1^*$ для случая счетных или бесконечных множеств:

$A_kA_j = \varnothing \Ra P(\bigcup\limits_k A_k) = \sum\limits_k P(A_k)$ -- $\sigma$-аддитивность.
\begin{Rem}
Функция $P$, удовлетворяющая аксиомам $1^*, 2$ называется $\sigma$-аддитивной.
\end{Rem}

Свойства:
\begin{enumerate}
    \item $P(\bar A) = 1 - P(A)$. В частности, $P(\varnothing) = 0$
    \item Монотонность: $A \subset B \Ra P(A) \leq P(B)$
    \begin{proof}
        \begin{gather*} 
            B = A \cup B\bar A \\
            P(B) = P(A) + P(B\bar A) \geq P(A)
        \end{gather*}
    \end{proof}
    \item $P(A \cup B) = P(A) + P(B) - p(AB)$. 
    
    Общий случай: $P(\bigcup\limits_k^n A_k) = \sum P(A_i) - \sum\limits_{i<j} P(A_iA_j) + \sum\limits_{i < j < k}P(A_iA_jA_k) - \dots$ -- просто формула включений-исключений.
    \begin{proof}
        Это у нас когда-то там было, когда говорили про множества.
    \end{proof}
    \begin{conseq}
    $P(\bigcup\limits_1^n A_k) \leq \sum P(A_k)$
    \end{conseq}

    \item
    $P(\lim A_n) = \lim P(A_n)$ (непрерывность $P$).

    Под пределом в данном случае имеется следующее: 
    
    Если $A_i \subset A_{i+1}$, то $\lim A_n = \bigcup\limits^{\infty} A_k$. 
    
    Если $A_{i+1} \subset A_{i}$, то $\lim A_n = \bigcap\limits^{\infty} A_k$. 

    Доказательство будет следовать из

    \begin{theorem}
        $P$ удовлетворяет аксиомам 1 и 2. Тогда равносильны:
        \begin{enumerate}
            \item Аксиома $1^*$
            \item Непрерывность $P$
            \item Непрерывность $P$ в нуле: $B_k \searrow \varnothing \Ra P(B_k) \to 0$        
        \end{enumerate}
    \end{theorem}                                                                    
    \begin{proof}
    \begin{description}        
        \item[$1 \Ra 2$]
        Пусть $A_i \subset A_{i+1}$ (иначе перейдем к дополнениям).
        Рассмотрим следующее объединение: $A_n \cup A_{n+1} \bar A_n \cup A_{n+2} \bar A_{n+1} \cup \dots = \lim(A_n) \eqcolon A$.
        Это объединение несовместных событий.
        Тогда $P(A) = P(A_n) + \sum\limits_{k=n}^{\infty} P(A_{k+1} \bar A_k)$.
        Устремим $n$ к бесконечности, второе слагаемое стремится к 0 как хвост сходящегося ряда
        \item[$2 \Ra 3$] -- очевидно
        \item[$3 \Ra 1$] 
        Рассмотрим события $A_i$, $A_iA_j = \varnothing$.
        Возьмем $B_n = \bigcup\limits_{k=n}^{\infty} A_k$, $B_1 = A = \bigcup\limits^{\infty} A_k$

        $B_n$ убывают, у них есть предел $B$. Хотим доказать, что $B = \varnothing$. 
        Пусть это не так, тогда в $B$ есть какой-то элемент $b \in B$, тогда $b \in B_1$, значит, $b \in A_l$ для какого-то $l$.

        Тогда $b \notin A_k, k \neq l \Ra b \notin \bigcup\limits_{l+1}^{\infty} A_k = B_{l+1} \Ra b \notin \bigcap B_k = B$, противоречие.

        Тогда $P(A) = \sum\limits^k P(A_i) + P(B_{k+1})$ -- применили 1-ю аксиому. 
        Устремим $k$ к бесконечности, тогда $P(B_{k+1}) \to 0$ по непрерывности в нуле, получим $\sigma$-аддитивность.

    \end{description}
    \end{proof}

    \item
    $P(\bigcup\limits^{\infty}) \leq \sum\limits^{\infty} P(A_k)$
    \begin{proof}
        Знаем для конечного случая, перейдем к пределу. 
        
        Получим $\lim P(\bigcup\limits_{i=0}^n A_i) \leq \lim \sum\limits_{i=0}^{n} A_i = \sum\limits^{\infty} A_i$.
        
        По 4 свойству, левая часть равна $P(\lim \bigcup\limits_{i=0}^n A_i) = P(\bigcup\limits^{\infty} A_i)$, что и требовалось.
    \end{proof}
\end{enumerate}

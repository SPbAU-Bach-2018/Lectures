\begin{exmp}
Блуждание на прямой с вероятностью $p$ движения на 1 вправо и вероятностью $q = 1 - p$ влево.

Тогда вероятность за $2n$ шагов вернуться в точку, откуда начали, будет равна $P_{jj}(2n) = {2n \choose n} p^n q^n$, что можно оценить по формуле Стирлинга:
$\frac{(4pq)^n}{\sqrt{\pi n}}$. 

Откуда видно, что $j$ возвратно тогда и только тогда, когда $p = q = \frac12$, иначе ряд, очевидно, сходится.
\end{exmp}
\begin{exmp}
Теперь ходим по плоскости с вероятностью $p = \frac14$ по каждому из направлений.

$P_{jj}(2n) = \frac{1}{4^{2n}} {2n \choose n} \underbrace{\sum\limits_k {n \choose k}^2}_{{2n \choose n}} = \frac{{2n \choose n}}{2^{2n}}^2 \sim \frac{1}{\pi n}$.

Этот ряд тоже очевидно расходится.
\end{exmp}
\begin{exmp}
В общем случае, в $\R^d, d \geq 3$

\begin{gather*} 
P_{jj}(2n) = \frac{1}{(2d)^{2n}} {2n \choose n} \sum\limits_{k_1 + \dots + k_d = n} (C_n^{k_1, \dots, k_d})^2 \leq \\
\leq \frac{{2n \choose n}}{(2d)^{2n}} \max C_n^{k_1, \dots, k_d} \underbrace{\sum C_n^{k_1, \dots, k_d}}_{d^n} \leq \\
\max C_n^{k_1, \dots, k_d} \leq A \max \frac{n^n \sqrt{n}}{k_1^{k_1} \dots k_d^{k_d} \sqrt{k_1 k_2 \dots k_d}} \stackrel{k_j = \frac{n}{d}}{=}  A d^n n^{-\frac{d-1}{2}} \\
\leq \frac{A}{n^{\frac{d}{2}}}
\end{gather*}

Все блуждания для $d \geq 3$ являются невозвратными.
Для $d=3$ вероятность вернуться примерно $0.35$
\end{exmp}

Общий случай случайного блуждания: $S_n = \xi_1 + \dots + \xi_n$, $\xi_j$ -- ``шаги'' $\in \Z^d$, $\xi_j$ -- iid.

\begin{theorem}

Случайное блуждание возврано тогда и только тогда, когда $\lim\limits_{\lambda \to 1} \int\limits_{[-\pi, \pi]} Re \frac{\d t}{1 - \lambda f(t)} = \infty$,
где $f(t)$ -- характеристическая функция $\overrightarrow{\xi_j}$
\end{theorem}
\begin{proof}
$$f(t) = \sum\limits_{\overrightarrow{k}} e^{i (t, k)} P(\overrightarrow{\xi} = \overrightarrow{k})$$
$$f^n(t) = f_{S_n}(t) = \sum e^{i (t, k)} P(S_n = t)$$

$$\int\limits_{[-\pi; \pi]^d} e^{i t_1 k_1} = 0 \Ra \int f^n(t) \d t = (2\pi)^d P(S_n = 0)$$

Умножаем на $\lambda^n$ и суммируем по $n$: $\sum \lambda^n P(S_n = 0) = \frac{1}{(2\pi)^d} \int \frac{\d t}{1 - \lambda f(t)} = \frac{1}{(2\pi)^d} \int Re \frac{\d t}{1 - \lambda f(t)}$

Слева стоит вероятность вернуться в точку старта за $n$ шагов в виде степенного ряда. Если подставить $\lambda = 1$, то получится в точности сумма из определения возвратности.

Устремляем $\lambda \to 1 - 0$ и получаем что надо.
\end{proof}
\begin{Rem}
С помощью этой теоремяы можно доказать, что в $\R^1$ и в $\R^2$ при $E |\xi_j|^2 < \infty$ блуждание возвратно тогда и только тогда, когда $E |\xi_j| = 0$.

При $d=1$ надо расписать $f(t) = u(t) + iv(t)$, $u(t) = \int \cos tx \d F_\xi(x)$, разложить косинус по Тейлору: $= 1 - \frac{t^2}{2} E\xi_1^2 + o(t^2)$,
$v(t) = \int \sin tx \d F_\xi(x) = t E\xi_1 + o(t^2)$. Потом надо честно расписать вещественную часть, подставить в интеграл.

В двумерном случае так же, просто чуть больше возни.

А при $d \geq 3$ любое блуждание невозвратно.
\end{Rem}

\begin{theorem}

$j$ --возвратное апериодическое состояние, тогда $P_{jj}(n) \to \pi_j = \frac{1}{\mu_j}$, $\mu_j$ -- среднее время возвращения в 0
\end{theorem}
\begin{conseq}
Если $i \leftrightarrow j$, то $P_{i,j}(n) \to \pi_j$.
\end{conseq}
\begin{proof}
$P_{ij}(n) = \sum\limits_{k=1}^n f_{ij}(k) P_{jj}(n - k)$. Устремим $n \to \infty$, $\sum f_{ij}(k) = f_{ij} = 1$
\end{proof}
\begin{conseq}
Если период $d > 1$, то $C = C_0 \cup C_1 \cup \dots \cup C_{d-1}$.

Пусть $i \in C_0, j \in C_a$. Тогда $P_{i,j}(nd+a) \xrightarrow[n \to \infty]{} \frac{d}{\mu_j}$.

\end{conseq}
\begin{proof}
Рассмотрим цепь с матрицей перехода $\widetilde{P} = P^d$. Пусть $a = 0$

%$k$ --число циклов, через которое ...

$P_{i,j}(nd) = \widetilde{P_{ij}}(n) \to \frac{1}{\sum k f_{ij}(kd)} = \frac{d}{\sum\limits_{k-1}^\infty kd f_{ij}(kd)} = \frac{1}{\mu_j}$

$a = 1$: $P_{i,j}(nd + 1) = \sum\limits_{k \in C_1} P_{i,k} P_{kj}(nd) \xrightarrow[n \to \infty]{} \frac{d}{\mu_j} \underbrace{\sum\limits_k P_{i, k}}_{=1}$. 
А дальше по индукции.
\end{proof}
\begin{Rem}
$\pi_j$ можно находить, составив следующую систему:
$P_{ij}(n + 1) = \sum\limits_k P_{i, k}(n) P_{k, j}$, устремляем $n \to \infty$ и получаем уравнения $\pi_j = \sum P_{k, j} \pi_k$

В матричной форме: $(P^T - I) \overrightarrow{\pi} = 0$
\end{Rem}

\begin{theorem}
Неприводимая цепь $j \in \{0, 1, 2, \dots \}$ -- возвратные положительные апериодичные состояния.

Тогда $\pi_j = \sum\limits_k P_{kj} \pi_k, \sum\limits_k \pi_k = 1$ и $\overrightarrow{\pi}$ однозначно определяется из этой системы.
\end{theorem}
\begin{proof}
$1 = \sum\limits_{j=0}^\infty P_{ij}(n) \geq \sum\limits_{j=0}^M \pi_j$ (устремили $n \to \infty$). 
А теперь $M \to \infty$, тогда $\sum\limits_0^\infty \pi_j \leq 1$.

$P_{ij} (n + !) \geq \sum\limits_0^M P_{i,k}(n)P^{kj}$ $n \to \infty$, $\pi_j \geq \sum\limits_0^M \pi_k P^{k, j}$. Теперь $M \to \infty, \pi_j \geq \sum\limits_0^\infty \pi_k P_{kj}$

Умножим теперь на $P_{j,i}$ и суммируем по $j$

$\pi_j \geq \sum\limits_{j=0}^\infty \pi_j P_{ji} \geq \sum\limits_j P_{ji} \sum\limits_{k=0}^\infty \pi_k P_{kj} = \sum\limits_k \pi_k P_{ki}(2)$

По индукции, $\pi_j \geq \sum\limits_k \pi_k P_{kj}(n)$

Если $\exists j_0 \colon \pi_{j_0} > \sum$, то суммируем по $j$ и получаем противоречие, откуда $\pi_j = \sum$.

$\sum\limits_j \pi_j > \sum\limits_k  \underbrace{\sum\limits_j P_{kj}(n)}_{=1}$

Устремляем $n \to \infty$, получаем $\pi_j = \sum\limits_k \pi_j \pi_k \Ra \sum \pi_k = 1$.


Осталось понять, что решение единственное. 
Пусть $(P^T - I) \overrightarrow{\alpha} = 0, \sum \alpha_k = 1$.
$\alpha_k = \sum\alpha_j P_{jk}(n)$, устремляем $n \to \infty$, получаем $\alpha_k = \sum_j \alpha_j \pi_k = \pi_k$.
Значит, наборы совпадают.
\end{proof}

\begin{conseq}
В условиях этой теореым, при любом начальном распределении $P(\xi_n = j) \to \pi_j$.

Набор $\{\pi_j\}$ называется стационарным распределениям процесса. Пусть $P(\xi_0 = j) = \pi_j$. Тогда $P(\xi_1 = j) = \pi_j$, и так далее.
\end{conseq}
\begin{proof}
$P(\xi_n = j) = \sum\limits_k P(\xi_0 = k) P_{kj}(n) \xrightarrow[n \to \infty]{} \pi_j \underbrace{\sum P(\xi_0 = k)}_{=1}$.
\end{proof}


Если несколько Эргодических классов, и мы в один уже попали, то для него справедлива эта теорема и \TODO


\begin{exmp}[Задача про пьяницу]

Пьяница стоит в точке 1, в точке 0 пропасть, он ходит вправо с вероятностью $p$ и влево с вероятностью $q = 1 - p$

$P = \begin{matrix} 
1 & 0 & 0 & 0 & \dots \\
q & 0 & p & 0 & \dots \\
0 & q & 0 & p & \dots \\
\dots
\end{matrix}
$

0 -- существенное состояние, все остальные -- нет.

А если в состоянии $N$ поместить вытрезвитель, в котором пьяница тоже заканчивает блуждание, тогда (если $q < p$)
$f_{k,0} = \frac{(\frac{q}{p})^k - (\frac{q}{p})^N}{1 - (\frac{q}{p})^N} \xrightarrow[N \to \infty]{} (\frac{q}{p})^k$ -- вероятность упасть из точки $k$.
\end{exmp}

\begin{theorem}[Муавра-Лапласа, интегральная]
$p \in (0, 1)$. Тогда равномерно по $-\infty \leq a < b \leq \infty$:

$$\sum P_{n, p}(k) \xrightarrow[n \to \infty]{} \int\limits_a^b \underbrace{\frac{1}{\sqrt{2 \pi}} e^{\frac{-t^2}{2}}}_{\phi(t)} \d t$$

Погрешность при этом порядка $\frac{1}{\sqrt{npq}}$.
\end{theorem}

\begin{theorem}[Пуассон]
$\pi_k = e^{-\lambda} \frac{\lambda^k}{k!}$, $\sum\limits_0^{\infty} \pi_k = 1, \lambda > 0$

Если $p \xrightarrow[n \to \infty]{} \to 0$, $\lambda = np$, то $$P_{n, p}(k) - \pi_k \xrightarrow[n \to \infty]{} 0$$

Погрешность: $|\sum\limits_{k \in A} (P_{n, p}(k) - \pi_k)| \leq np^2 \forall A \subset \{0, 1, \dots, n\}$

\end{theorem}
\begin{Rem}
Все без доказательства. 
Теорему Муавра докажем в будущем для всех распределений, а теорему Пуассона в чуть более слабой форме.
\end{Rem}
\begin{Rem}
Можно либо честно считать погрешности, либо пользоваться эвристическим правилом: если $np$ меньше нескольких десятков, то используем Пуассона, иначе Лапласа.
\end{Rem}

\chapter{Случайные величины, их распределения}
\begin{Def}
Возьмем все открытые лучи в $\R$, которые идут из $-\infty$. 

Тогда множество $\mathcal{B} = \sigma(\{(-\infty, x), x \in \R \})$ называется Борелевской $\sigma$-алгеброй.
\end{Def}                    
\begin{Rem}
Чтобы привести пример не Борелевской алгебры, нужно сильно постараться. 
В частности, для этого надо как-то использовать аксиому выбора. 
На практике считаем, что нам такие множества не встретятся.
\end{Rem}
Упражнение: доказать, что $\mathcal{B}$ -- $\sigma$-алгебра, $\mathcal{B} = \sigma(\{ (x, y) \}) = \sigma(\{[x, y]\}) = \dots$

\begin{Def}
Случайная величина (с.в.) $\xi$ -- функция $\Omega \to \R$, если она $\mathcal{F}-\mathcal{B}$-измерима, 
то есть, $\forall B \in \mathcal{B} \xi^{-1}(B) \in \mathcal{F}$.
\end{Def}
\begin{Rem}
Смысл $\mathcal{F}-\mathcal{B}$-измеримости: хотим, чтобы высказывания вида ``$\xi$ принимает значение от 3 до 5'' и ему подобные были событиями.
\end{Rem}
\begin{exmp}
Пусть $\Omega \subset \R$
\begin{enumerate}
\item $\mathcal{F} = \{\varnothing, \Omega\}$, тогда $\xi$ может быть только константой.
\item $\mathcal{F} = \sigma(\{[k, k + 1), k \in \Z\})$, тогда $\xi$ -- любая ступенчатая функция (длина ступеньки 1).
\item $\mathcal{F} = \{A\colon A = -A = \{-x, x \in A\}\}$, тогда $\xi$ -- любая четная функция.
\end{enumerate}
\end{exmp}

\begin{Def}
$P_{\xi}\colon \mathcal{B} \to [0, 1]$ -- распределение случайной величины, $P_{\xi}(B) \eqDef P(\xi^{-1}(B))$.
\end{Def}

От тройки $(\Omega, \mathcal{F}, P)$ мы сейчас перешли перешли к тройке $(\R, \mathcal{B}, P_{\xi}$), это тоже вероятностное пространство.

\begin{Def}
$\mathcal{F}_{\xi} = \{ \xi^{-1}(B), B \in \mathcal{B} \} \subset \mathcal{F}$ -- $\sigma$-алгебра, порожденная $\xi$. 
\end{Def}
Упражнение: проверить, что это действительно $\sigma$-алгебра (просто по определению).
\begin{Rem}
В нее входят те и только те события, про которые можно сказать, зная только значение $\xi$.
\end{Rem}
\begin{exmp}
Пусть у нас есть колода карт, из которой мы вытягиваем одну карту, $\xi$ -- достоинство вытянутой карты.

Тогда $\mathcal{F}_{\xi}$ будет состоять только из событий вида (вытянута двойка / вытянута либо тройка, либо четверка, и так далее).
При этом, событий вида (вытянута пиковая дама) в этом множестве лежать не будут, так как они не следуют только из значения $\xi$.
\end{exmp}

\begin{Def}
Функция распределения (ф.р.) случайной величины $\xi$: $F_{\xi}(x) = P((-\infty, x)) = P(\xi < x)$
\end{Def}
\begin{Rem}
Часто индекс $\xi$ у $F$ опускают.
\end{Rem}

Несложно заметить, что $F$ однозначно опреляет $P_{\xi}$ по теореме Каратеодори для набора полуоткрытых интервалов.

\begin{theorem}[Свойства функции распределения]

\begin{enumerate}

\item $F(x) \nearrow$
\item $F(x) \xrightarrow[x \to -\infty]{} 0$, $F(x) \xrightarrow[x \to \infty]{} 1$
\item $F$ непрерывна слева
\end{enumerate}
\end{theorem}
\begin{proof}
1 -- из монотонности $P$.

2, 3 -- из непрерывности $P$: рассмотрим $x_n = n$, последовательность лучей $(-\infty, n)$ возрастает и стремится к $\R$, тогда $P((-\infty, x_n) \xrightarrow[n \to \infty]{} P(\R) = 1$
\end{proof}

Если $x$ -- точка разрыва, то $P(\xi = x) = F(x + 0) - F(x)$.
\begin{gather*}
P_{\xi}([x, y)) = F(y) - F(x)\\
P_{\xi}([x, y]) = F(y + 0) - F(x)\\
P_{\xi}((x, y)) = F(y) - F(x + 0)\\
P_{\xi}((x, y]) = F(y + 0) - F(x + 0)
\end{gather*}

\section{Классификация и примеры распределений}
\begin{enumerate}[label=\Roman*.]
    \item $\xi$ и ее распределение называются дискретным, если $\exists B\colon P_{\xi}(B) = 1$, $B$ не более чем счетное.
    \begin{enumerate}[label=\arabic*.]
    \item $\xi = C$ -- константа, тогда $F_{\xi}(x) = \begin{cases} 0, x \leq c \\ 1, x > c\end{cases}$
    \begin{Rem}
        Удобно записывать $F_{\xi}(x) = \mathbb{1}_(x \geq c)$, $\mathbb{1}_A (x)$ выдает 1, если $x \in A$, 0 иначе
    \end{Rem}   
    \item $\xi = \begin{cases} 1, \text{с вероятностью } p \\ 0, \text{с вероятностью } q = 1 - p\end{cases}$.

    $F(x) = q \mathbb{1}_{(x \geq 0)} + p \mathbb{1}_{(x \geq 1)}$

    В общем случае: $\{x_k\}, \{p_k\}$ -- ряд распределения, $p_k = P(\xi = x_k)$, $F_{\xi}(x) = \sum\limits_{k\colon x_k \in x} p_k$.

    \item
    $\xi \sim Bin(n, p)$, если $x_k = 0 \dots n, p_k = P_{n,p}(k)$

    \item
    $\xi \sim Poiss(x), x_k = 0, 1, 2, \dots, p_k = e^{-\lambda} \frac{\lambda^k}{k!}$
    \end{enumerate}
    \item $\xi$ и ее распределене называют непрерывным, если $F_{\xi}(x) \in C(\R)$; 
    
    Называют абсолютно непрерывной, если $F_{\xi}(x) =\int\limits_{-\infty}^{x} p_{\xi}(x) \d x$, $p_{\xi}(x)$ называют плотностью распределения $\xi$.

    \begin{enumerate}[label=\arabic*.]
    \item $\xi \sim U[a, b]$, если $F_{\xi} = \begin{cases} 0, x \leq a; \\ \frac{x - a}{b - a}, a < x \leq b \\ 1, x > b\end{cases}$. 
        $P_{\xi}(x) = \frac{\mathbb{1}_{[a, b]}(x)}{b - a}$

    \item $\xi \sim N(a, b)$ - нормальное (Гауссово), если $P_{\xi}(x)= \frac{1}{\sqrt{2 \pi \sigma^2}} e^{-\frac{-(x-a)^2}{2\sigma^2}}$.
    График распределения -- ``колокол'', $\sigma$ отвечает за его ширину, $a$ -- за его вытянутость.

    \end{enumerate}
    
    \item $\xi$ и ее распределение называют сингулярным, если $\xi$ непрерывна и $\exists B \in \mathcal{B}\colon P_{\xi}(B) = 1$, при этом $\mu(B) = 0$.
    Пример: Канторова лестница
\end{enumerate}
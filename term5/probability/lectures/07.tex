\begin{theorem}
Свойства условного матожидания:
\begin{enumerate}
\item $E(\xi \mid \mathbb{F}_\xi) = \xi$
\item Пусть $\xi$ не зависит от $\sigma$-алгебры $\mathbb{A}$ (то есть $\xi^{-1}(B \in \mathbb{B})$ и $A \in \mathbb{A}$ независимы).
Тогда $E(\xi \mid \mathbb{A}) = E\xi$ (потому что в силу независимости $\xi$ от $\mathbb{A}$ мы не получаем никакой информации от того, произошло ли какое-то событие $A$, или нет).
\item Формула полного матожидания: $E(E(\xi \mid \mathbb{A})) = E\xi$
\item $E(E(\xi \mid \mathbb{A}_1) \mid \mathbb{A}_2) = E(\xi \mid \mathbb{A}_1 \cap \mathbb{A}_2)$
\item $\eta$ -- $\mathbb{A}$-измерима ($\mathbb{F}_\eta \subset \mathbb{A}$). Тогда $E(\xi\eta \mid \mathbb{A}) = \eta E(\xi \mid \mathbb{A})$.
\end{enumerate}
\end{theorem}
\begin{proof}
\begin{enumerate}
\item  Просто по определению.
\item  На пальцах: можно рассмотреть $E (\xi \mid A_1)$ и $E (\xi \mid A_2)$. 
Если предположить, что они разные, то $A_1$ и $A_2$ оказывают влияние на $\xi$. 
Это же противоречит независимости.

Эти рассуждения при желании, видимо, можно формализовать.

\item Частный случай 4 свойства.

Остальные два -- без доказательства.
\end{enumerate}
\end{proof}

Теперь докажем, что $g^*$ -- условное матожидание $\xi$ будет оптимальной функцией.
Для этого надо доказать, что расстояние до произвольной функции хотя бы расстояние до $g^*$:

$$E(\eta - g(\xi))^2 = E(\eta - g^* + g^* - g)^2 =E(\eta - g^*)^2 + \underbrace{E(g - g^*)^2}_{\geq 0} + 2E(\eta - g^*)(g^* - g) \stackrel{?}{\geq} E(\eta - g^*)^2$$


По 3 свойству:
$$E(\eta - g^*)(g^* - g) = E(E(\eta - g^*)(g^* - g) \mid \mathbb{F}_\xi) = $$

По свойству 5 (потому что вторая скобка $\mathbb{F}_\xi$-измерима, так как обе функции зависят от $\xi$)
$$= E(g^* - g) E(\eta - g^*(\xi) \mid \mathbb{F}_\xi)$$

Правый множитель равен (по 1 свойству) 
$$E(\eta \mid \mathbb{F}_\xi) - E(g^*(\xi) \mid \mathbb{F}_\xi) = g^*(\xi) - g^*(\xi) = 0$$

\section{Характеристические производящие функции}
\begin{Def}
Характеристическая функция $f_\xi(t) = E e^{it\xi} = \int\limits_{-\infty}^\infty e^{itx} \d F_\xi(x)$.

Для дискретных: $ = \sum\limits_k e^{itx_k} p_k$

Для абсолютно непрерывных: $\int\limits_{-\infty}^\infty e^{itx}p_\xi(x) \d x$. (А еще это называется преобразованием Фурье для функции $p_\xi$)

$f_\xi(t) \colon \R \to \C$.
\end{Def}
\begin{Rem}
Взяли плотность, растянули ее в $t$ раз, после чего свернули в единичную окружность. Потом усреднили = взяли центр масс
\end{Rem}

\begin{Def}
Если $\xi$ -- неотрицательная дискретная целочисленная случайная величина, то у нее есть производящая функция $\phi_\xi(t) = Et^\xi = \sum p_k t^k$
\end{Def}
\begin{Rem}
$f_\xi(t) = \phi_\xi(e^{it})$. Свойства почти одинаковые, производящая функция удобна только тем, что она вещественная. В общем случае же пользуемся характеристической.
\end{Rem}

Основное свойство: если $\xi, \eta$ независимы, то $f_{\xi + \eta}(t) = Ee^{it(\xi + \eta)} = f_\xi(t)f_\eta(t)$. 
Аналогично для производящих.

$f_{a\xi + b}(t) = E e^{it(a\xi + b)} = e^{itb} Ee^{ita\xi} = e^{itb} f_\xi(ta)$

\begin{exmp}
$\xi = C$, $f_\xi(t) = e^{itc}$
\end{exmp}
\begin{exmp}
$\xi = \begin{cases}1, p \\0, q\end{cases}$, $f_\xi(t) = pe^{it1} + q^{eit0} = p^{it} + q$; $\phi(t) = pt + q$
\end{exmp}
\begin{exmp}
$\xi \sim Bin(n, p)$. $\xi = \sum\limits_{k=1}^n \mathbb{1}_{A_k}$, $f_\xi(t) = (pe^{it} + q)^n$
\end{exmp}
$\xi \sim N(0, 1)$ -- нормальное распределение. Тогда (упражнение): $f_\xi(t) = e^{-\frac{t^2}{2}}$.

$\xi \sim N(a, \sigma^2)$, $\xi = \sigma \widetilde{\xi} + a$, где $\widetilde{\xi} \sim N(0, 1)$. Откуда $f_\xi(t) = e^{ita - \frac{t^2\sigma^2}{2}}$.

\begin{theorem} (Формулы обращения)
                                  
\begin{enumerate}
\item $a < b, F_\xi$ непрерывно в точках $a, b$. Тогда $F(b) - F(a) = \frac{1}{2\pi} v.p. \int\limits_{-\infty}^\infty \frac{e^{-ita} - e^{-itb}}{it}f_\xi(t) \d t$, ($v.p.$ -- главное значение интеграла)

\item Если $\int\limits_\R |f_\xi(t)| \d t < +\infty$, то $\xi$ -- абсолютно непрерывна и $p_\xi(x) = \frac{1}{2\pi} \int\limits_{-\infty}^\infty e^{-itx}f_\xi(t) \d t$
\end{enumerate}
\end{theorem}
\begin{proof}
Без доказательства. Вообще -- должно доказываться в матанализе как существование обратного преобразования Фурье
\end{proof}
Поэтому характеристической функции достаточно, чтобы описать распределение.

\begin{theorem}[Свойства характеристической функции]

Везде ниже $f = f_\xi$.

\begin{enumerate}
\item $|f(t)| \leq f(0) = 1$
\item $f$ равномерно непрерывна на $\R$
\item $f(-t) = \overline{f(t)}$
\item $\forall t f(t) \in \R \Leftrightarrow \forall B \in \mathbb{B} P_\xi(B) = P_{-\xi}(B)$
\item Если $E|\xi|^n < \infty$, то $\forall k \leq n \exists f^{(k)}(t) = \int (ix)^ke^{itx} dF_\xi(x)$. $f^{(k)}(0) = i^k E \xi^k$
\item Если $\exists f^{(2n)}(0)$, то $E|\xi|^{2n} < \infty$
\item $f(t) = \sum\limits_0^n \frac{i^k E\xi^k}{k!}t^k + o(t^n)$. Если радиус сходимости положителен ($\varlimsup \frac{(E|\xi|^n)^\frac1n}{n} = \frac1R < \infty$, то
$f(t) = \sum\limits_{k=0}^\infty \dots$ при $|t| < R$. 

Отсюда, в частности, следует теорема про моменты, когда они определяют распределение.
\end{enumerate}
\begin{proof}
1 и 3 свойства очевидно следуют из определения.

2:   
\begin{gather*}
|f(t + h) - f(t)| = |\int (e^{i(t + h)}x - e^{itx}) \d F| \leq \int \underbrace{|e^{itx}|}_{=1} |e^{ixh} - 1| \d F = \\
\int\limits_{-T}^T + \int\limits_{\R \setminus [-T, T]}|e^{ixh} - 1| \d F(x)\\
\text{Возьмем T такое, что } P(R \setminus [-T, T]) < \epsilon, \text{и } h\colon \max (|e^{ihT}  -1|, |e^{ih-T}  -1|) < \epsilon \\
|e^{ihx} - 1| \leq 2 \Ra \int\limits_{\R \setminus [-T, T]} + \int\limits_{-T}^T \leq 2\epsilon + \epsilon = 3\epsilon \\
\end{gather*}
Мы взяли настолько маленькое $h$, что получили какую-то маленькую окрестность $1$ на окружности, на обоих концах у нас модуль разности мелкий, тогда интеграл тоже будет мелким
                                          

4: $f(t) = \int \cos tx \d F(t) + i \int \sin tx \d F(t)$. Второе слагаемое равно 0 для симметричных распределений (синус нечетный).

В обратную сторону: $f_\xi(-t) = \overline{f(t)} = f(t)$. С другой стороны, минус можно от $t$ перекинуть к $\xi$ и получить $f_\xi(-t) = f_{-\xi}(t)$
\end{proof}
\end{theorem}

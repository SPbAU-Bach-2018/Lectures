

\TODO Броуновский процесс \TODO


\section{Марковские цепи}
\begin{Def}
Марковская цепь (м.ц.) -- марковский процесс с дискретным временем ($T = \{0, 1, 2, \dots \}$) и не более чем счетным числом состояний.
\end{Def}

$$P(\xi_{n+k} = j \mid \xi_k = i) = P_{i, j}(n, k)$$ зависит от 4 величин. Но мы будем рассматривать только такие процессы, которые не зависят от $k$ (однородные по времени).

\begin{Def}
$P_{i,j} = P_{i,j}(1), P = (P_{i, j})_{i, j}$ -- матрица перехода.
\end{Def}
\begin{theorem}
Конечномерное распределение $\xi_n$ задается распределением $\xi_0$ и матрицей $P$.
\end{theorem}
\begin{proof}
$P_{i,j}(2) = \sum\limits_k P_{i,k} P_{k, j}$ (по формуле полной вероятности). А это, как несложно увидеть, квадрат матрицы.
А дальше по индукции, $P(n) = P^n$

И, например, $P(\xi_n = j) = \sum\limits_i P(\xi_0 = i) P_{i,j}(n)$
\end{proof}

\date{September 22, 2016}
\author{Chernikova Olga}

\section{Простейшие свойства}
Мы научились строить автомат, доказали, что он быстро работает, осталось 
научиться его использовать.% Пока единственное, что вы умеете делать, 
%это дана строка и автомат.

У нас был текст, построили от него суффиксный автомат за линейное время ($text \to S.A(text)$). 
%И теперь по строке мы умеем говорить входит ли она в текст. 

\begin{description}
\item[Провекра, входит ли строка в текст]
    Есть строчка S, хотим проверить, входит ли S в текст за $O(|S|)$.

    Подстрока текста, это префикс какого-то ее суффикса. То есть, нужно просто 
    съесть строчку S автоматом, и если были все переходы по дорогие и в конце мы 
    оказались в какой-то, не обязательно терминальной, вершине, значит входит.

\item[Подсчет количества вхождений строки S]
    %Это динамика. У вершины должна быть насчитана динамика, которая  равна сумме 
    
    Обозничим v(S) "--- вершину, в которую мы попали, когда съели строчку S.

    Теперь хотим посчитать, размер правого контекста вершины v(S). То есть, 
    если мы прочитали строчку S, сколько способов дописать символы, что бы 
    получить суффикс text. 

    Для этого посчитаем динамику. Пусть $x_i$ все возможные вершины в которые 
    можно попасть за один шаг из вершины $v$. $f[v]$ "--- размер правого контекста вершины. 
    Тогда $f[v] = \sum_{x_i} + is_term[v]$.
\item[Самое левое вхождение строки S]
    При добавление вершины при построение суффиксного автомата,
    запомним в какой позиции мы добавили эту вершину. При расщипление
    вершины, позиция самого левого вхождения не меняется.

    \begin{cppcode}
    for (int i = 0; i < n; ++i) {
        add(text[i]);
        left[last] = i;
    }
    \end{cppcode}

    При расщипление \cpp"left[r] = left[q]"
\item[Самое правое вхождение]
    Это считается динамикой по автомату. Самое правое вхождение у
    последней вершины мы знаем. У остальных вершин надо взять максимум по соседям и вычесть 1. 
\end{description}

\section{Примеры задач}
\subsection{Наибольшая общая подстрока k строк}
    \begin{description}
    \item[Наивный алгоритм:] Можно строить автомат от, например, конкатинации. Взять 
     строчки слить, еще через разные разделительные символы. Но так не 
     делают, нужно строить автомат от минимльной строки, он будет есть 
     куда меньше памяти и как следствие - быстрее работать. 
    
    \item [Описание алгоритма:\\]
    \begin{enumerate}
    \item Строим суффиксный автомат от минимальной из k строк. S.A(минимальная $s_i$)
    
    \item  %Нужно через этот автомат пропустить все остальные строки. И для каждой 
    %вершины что-то помнить. Например, длину наиьбольшей общей подстроки. 
    
    Съедаем последующие строчки атоматом, переходя по суффиксным ссылкам, если не 
    можем пройти по символу и в каждой вершине помним длину наибольшей общей строчки. 
    \end{enumerate}

    \item [Обработка крайних случаев:]
    Что бы не было неудобство с крайними случаями и всегда можно было откатиться 
    до состояния, когда есть переход, у нас будет фиктивная вершин, из начала мы откатываемся 
    именно в нее, а потом по любому символу переходим вперед. 
    
    \item [Как обновлять длину проходя по суфссылкам:]
    Если два раза прошлись по одной и той же вершине нужно запомнить первое или последнее вхождение?
    Нужно брать максимаьную длину из всех вхождений. 
    
    Есть текущая длина x, мы точно занаем, что эта длина x не больше
    чем len[v], но она может быть меньше. Мы же в вершину могли
    не по самой длинной строке прийти, а по любой. Когда мы
    откатываемся по суффиксной ссылке мы делаем $x = min(len[suf[v]], x)$. 
    
    \item [Обощение, когда строчек все-таки не две, а k:]
    Но это будет верно, только если мы пропускаем первую строчку через автомат. 
    Теперь не первую. 

    Тогда будем хранить, ограничение с предыдущих строк $max[v]$ "--- максимальная длина 
    подстроки, которая являлвсь общей для предыдущих вершин. 

    Изначально вершина принимает строки от некоторой 
    минимальной длины, до максимальной, но общими являются только подстроки, 
    которые не больше, чем $max[v]$.

    Изначально $max[v] = len[v]$. Потому что строка была всего одна и 
    все строки являются общими. Когда я прохожу через вершины автомата, я пересчитываю x.     
    
    \item[Реализация:] 
    
    w[v] "--- максимальная общая подстрока текущей и нулевой строки которая заканчивается в вершине v. 

    Надо не забыть в конце пробросить по суффиксным ссылкам в обе стороны. 

    \begin{cppcode}
    max[v] = len[v]
    for (int g = 1; g < k; ++g) {
        v = root, x = 0, w = {0}
        for (int j = 0; j < s[g].size(); ++j) {
            while (go[v][s[g][j]] == -1) {
                v = suf[v];
                x = min(x, max[v]);
            }
            v = go[v][s[g][j]];
            ++x;
            w[v] = max(w[v], x);
        }

        vector<int> vs[len]; //список вершин с данной длиной.
        for (int i = len[last]; i >= 0; --i) {
            for (v : vs[i]) {
                max[v] = min(max[v], w[v]);
                max[suf[v]] = min(max[suf[v]], max[v]);
            }
        }
    }
    \end{cppcode}
    
    \item[Время работы и память:] 
    Заметим, что первая часть деалется за $\Theta(s_k)$, 
    а вторая часть за $\Theta(N)$(количество вершин в автомате, которое 
    пропорционально длине наименьшей строки). 

    Время работы $O(\sum_{k = 1}^{m}s_k + Nm)$ \\
    
    Память $O(N)$.
    \end{description}

\subsection{LZSS за O(n)}
    \begin{description}
    
    \item[Определение:] 
    
    Есть строчка s. Для каждого i находм такое j < i,
    что k = LCP(s[j:], s[i:]) максимально. И
    выписываем j k, если k = 0, то выписываем симовол s[i].
   
    \item[Пример:]
    Это метод сжатия. 
    
    Например для следующей строчки.
    s = aaaaabbbaaabb

    Сначала я напишу букву a, потом повторю с 0 символа 4 буквы, 
    потом зпишу символ b, потом начиная с позиции 5 повторю 2
    буквы, потом начиная с позиции 2 повторю 5 букв. 

    lzss:\\
    a\\
    0 4\\
    b\\
    5 2\\
    2 5\\
    

    \item[Алгоритм]
    От автомата нам нужно, что когда мы уже выписали i букв находить 
    такое j для которого k максимально. 
    
    В чем сложность, когда выписали первых i букв нужно 
    максимум взять только по j, которые меньше чем i.
    
    Заметим, что для i имеем право найти ответ за $O(k)$. 
    Потому что после этого мы сделаем i += k. Значит 
    суммарное время будет линейное. 

    Мы будет от начала автомата идти вперед. Пропуская
    i-ый суффикс. Пуская мы уже выписали k первых символов 
    i-ого суффикса, и нам нужно проверить, что 
    у строки, которую мы написали - самое
    левое вхождение левее чем i. 

    Строим автомт от строки и считаем left[v]. 

    Что бы найти ответ для i идем по автомату до тех 
    пор, пока left[v] < i + k - 1. То есть проверяем, 
    что левое вхождение строки левее позиции i.
   
    \end{description}
\section{Суффиксное дерево по автомату}
    Хотим сказать, что дерево может быть в чем-то удобнее автомата, 
    но автомат нам дает дерево. 
    
    \begin{theorem}
    Ребра из v в suf[v] "--- это ребра суффиксного дерева перевернутой 
    строки S.
    \end{theorem}

    Если хотим дерево от строки S, то строим автомат 
    от обратной строки, берем ребра из v в suf[v] и 
    вот суффиксное дерево. 

    Заметим, что суффдерево можно не строить, если мы не собираемся его обходит DFS вниз. 

    После того, как мы построили автомат у нас уже есть ссылки. Уже есть
    суффиксное дерево, которое просто другим способом задано.    
    
    Например, что хорошего есть в суффиксном дереве, чего нет 
    в автомате. Там есть LCA = LCP. LCA в суффиксном дереве - это
    LCP двух суффиксов. Через автомат мы не можем
    взять два суффикса и как-то быстро найти LCP. Если мы хотим
    быстро код, то у нас есть массив отцов, и мы на нем насчитываем
    двоичные подъемы.

    %-Почему тогда все Укконена пишут а не вот это?   
    %-Не все пишут Уконнена. Это уже достаточно популярная тема. 
    %И на сколько я знаю, поскольку Уконнен штука достаточно сложная и 
    %нужно мастерство, что бы его написать, то людей, 
    %которые пишут это больше, чем людей, которые пишут Укконена. 
    %Но меньше, чем те, которые знают Уконнена. 
    
    \begin{description}
    \item[Пример:]

    строим автомат от \#ababb\\ 
    и получаем суф дерево для строки: bbaba\#

    \includegraphics[width=0.4\linewidth]{suf_im8.png}


    И рисуем суффиксные ссылки и
    по ним дерево.

    \includegraphics[width=0.4\linewidth]{suf_im9.png}

    Что написано на этом ребре. На этой вершине написано 
    ababb, а на этой вершине написано b, значит на 
    самом ребре написано baba. И именно она была отрезана.

    Суффиксная ссылка которая переходит в строчку a отрезает ab,
    занчит написать надо ba.

    Смотрите, как у нас интересно получилось.
    У нас есть суффикс a, ba, aba, baba, bbaba.
    Что в этом суфдереве может показаться странным. У нас
    есть вершина по середине ребра. Что бы такого
    не было, добавим в начало решетку.  

    Эта вершина по середине ребра означает, что у нас 
    есть суффикс a и суффикс ba  и они заканчиваются по середине ребер. 

    Строим от этой строчки автомат. 
    $\#$ нужно, что бы листья не были по 
    середине ребра. 

    Давай-те напишем код, как
    мы получали, какая строка написана на ребере. 

    У нас есть вершина v и вершина suf[v].  У них 
    есть len[v] и len[suf[v]]. То есть 
    мы знаем длину того, что нужно 
    написать. Осталось только понять, 
    где эта штука находится в строке.

    v - это какая-та страка, в suf[v] это 
    какой-то ее суффикс. А написать
    нужно строчку, которая их различает.

    То есть нужно взять место, в котором начинается 
    строка, которая написана на вершине v и выписать 
    первые len[v] - len[suf[v]] символов. Мы
    умеем считать уже и первое и последнее вхождение, то
    есть какое-то мы можем взять.

    И когда мы считали и первое и последнее вхождение - мы
    считали позицию конца. Что бы получить позицию
    начала - нужно вычесть длину.   

    В суффиксном дереве удобно знать, где заканчиваются суффиксы. 
    
    Получается что в вершинах. Это просто last по 
    ходу построения. Действительно. Автомат у нас 
    от обратной строки. То есть в автомате нам 
    нужно смотреть на префиксы. Нужно
    знать, где префиксы заканчиваются - это last.
    После добавления каждого символа - запоминаем 
    где был last, и множество last - это терминальные 
    состояния. 

    И ни один last не является суффиксом другого last из-за \#.
     
    \item[Доказательство теоремы:]
    Давай-те теорему докажем, мы ее сформулировали, надо 
    доказать.

    \begin{lemma}
    Последнее вхождение в любой строки - это первое вхождение в 
    перевернутой.  
    $Last_{S}(u) = First_{S^T}(rev(u)) = ST.path(rev(u))$
    \end{lemma}

    Тут Сережа запутался в переворачивание строк и пошло 
    доказательсвтво на пальцах..

   
    У нас была строчка u и у нее была суффиксная ссылка. 
    Теперь предствим перевернутый мир. Есть 
    подстрочка rev(u) и есть suf[rev(u)]. А теперь
    рисуем суффиксное дерево, теперь
    суффиксная ссылка - это ссылка на префикс.
    Где-то в суффиксном дереве есть
    место, которое соответствует suf[rev(u)].
    А из этого места мы
    можем еще дописать сколько-то символов и попасть в
    строчку rev(u).  Значит,
    как минимум, они лежат на одном пути от корня в суффиксном дереве.
    Осталось сказать, что это вершины.

    Каким образом у нас получаются
    вершины, есть другая строчка x. Так что суффиксная
    ссылка от x совпадает с suf[u].

    Вот была строчка x и u и у них есть суффиксная ссылка, 
    в перевернутом мире - это префикс. У них есть 
    общий префикс.

    Самый важный факт, что suf лежит на пути, а про вершины - это
    просто понимание.  
    
    У suf[v] более большой правый контекст. Нарисуем тот правый контекст, 
    которого не было. 
    Появилось новое вхождение. А теперь это новое вхождение продолжим 
    на один символ. Либо отличие в eps, тогда появится новая
    терминальная вершина, либо мы можем продолжить
    на один символ. Хотим сказать, что 
    вершина, полученная дописыванием 
    одного символа имеет такую же суффиксную 
    ссылку. 

    Взяли вхождение v перешли по суф ссылки, 
    у суфссылки больше правый контест, значит 
    есть какое-то новое вхождение, взяли это новое вхождение. 

    Мы пытаемся доказать, что на suf[v] кто-то еще ссылается, 
    это обязательно развилка. 

    Возбмем правый контекст suf[v] возьмем новое вхождение 
    и расширем его влево. Суфиксная ссылка полученной строки 
    совпадает с suf[v].

    Невсегда можно расширить слева. Слева нельзя расширить, 
    если слева ничего нет. Но тогда мы как раз терминал. 
    \end{description}

\section{Перебор всех вхождений строки в текст}

У нас есть тект, мы как всегда построили автомат от текста, 
нам подсовывают строчку s, а мы  хотим найти все 
вхождения в текст за O(|S| + |Answer|). 

Пока мы умеем делать одна операцию, взять из строки s пройти 
вперед и оказаться в некоторой вершине v. От вершины v мы
могли что-то предподсчитать, но от
вершины v есть очень много разных путей в терминалы. Каждый
путь от вершины v в терминал задает новое вхождение строки s.
И так как их много мы не можем просто так их
хранить, нам нужно как-то их быстро получить.   

В решение этой задачи как раз помогает неявное построение суффиксного дерева.

Представим, что у нас есть суфдерево. Тогда что бы вывести 
все вхождения строки s нужно просто обойти поддерево строки s.

Обойдем поддерево мы просто за размер поддерева.А вхождений
хотябы размер поддерева/2. Потому что каждый лист задает новое
вхождение.

Мы знаем связь между автоматом и деревом.    

Берем суфиксное дерево. Которое мы построили методом взяли автомат
перевернутой строки и запомнили \cpp"list[suf[v]].push(v)".  

Теперь нужно обойти поддерево дерева суфиксных ссылок. 

Но при этом, кстати, не обязательно переворачивать явно, 
стока s входит тогда и только тогда, когда 
входяит в перевернутую строчку s перевернутое. Поэтому 
можно спокойно жить в перевернутом мире.

 
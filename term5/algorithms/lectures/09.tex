\setauthor{Дима Лапшин}
\chapter{Планарные графы}

\begin{Def}
	Планарный, плоский граф.
\end{Def}

\begin{theorem}
	Всякую грань можно сделать внешней.
\end{theorem}
Укладка на сферу.

\begin{theorem}[Эйлера]
	\[ E + C + 1 = V + G \]
\end{theorem}
\begin{conseq}
	Если у каждой грани степень хотя бы 3, в каждой компоненте хотя бы 3 вершины, то $E \leq 3V - G$.
\end{conseq}

\begin{theorem}[Куратовский, 1930]
	Граф планарен тогда и только тогда, когда несуществует подграфа, гомеоморфного\footnote{удаление вершины степени 2 на ребро}
	$K_5$ или $K_{3,3}$.
\end{theorem}
\begin{theorem}[Вагнер, 1937]
	То же самое, но через стягивание\footnote{замена ребра на вершину, смежную со всеми их соседями}.
\end{theorem}

Разные алгоритмы стягивания видят разные графы.
Тут Куратовский видит $K_{3,3}$, а Вагнер "--- $K_5$.

\begin{theorem}[Форш, 1948]
	Существует укладка плоского прямыми отрезками.
\end{theorem}
\begin{theorem}[Шнаудер, 1989]
	Тем более, можно уложить вершины на сетку $(V - 2)^2$.
\end{theorem}

\section{Алгоритмы}

\begin{enumerate}
	\item Демукрон, 1964, $O(N^2)$. Planarity Testing.
		В контесте будет задачка, можно пользоваться и лекцией, и исходной статьёй.
	\item Тарьян, Хопкрофт, 1974. $O(N)$. Ужасен по всему, но первый)
	\item Boyer, 2004; Brandes 2011. $O(N)$, пишется и шустрое.
	\item Теорема Татта: любой вершинно трёхсвязный можно уложить так, чтобы все рёбра "--- отрезки, следующим методом:
		каждую внешнуюю и сопостовляем выпуклые многоугольники, и каждая не внешняя вершина "--- центр масс соседей, система линейных уравнений.
\end{enumerate}
Побочный эффект всех алгоритмов: грани дают.

Пусть есть разбиение на грани двусвязного графа, хотим сделать временно трёхсвязным.
Сделаем триангуляцию: взяли, и или провели диагональные рёбра, или добавили новую вершину в центр.

Как уложить несвязный? Покомпонентно.
Как уложить связный? Или укладывать по точкам сочленения, или взять все компоненты вокруг, выбрали внешние циклы в каждой, соединили их по кругу.

Осталось решить систему линейных уравнений.
Метод Гаусса "--- $O(V^3)$, жирно.
Есть метод итераций, может помочь.

Пусть есть несколько вершин, лежат вперемешку, внутри внешней грани. Посчитали, где она должна лежать, передвинули.
Сойдётся быстро, получается что-то вида $O(kE)$, нам ведь не честное решение нужно, а уложить граф.

После этого всего удаляем лишнее и радуемся жизни.

\section{Демукрон}

Есть две версии, из статьи и эта:

\begin{enumerate}
\item
	Выбрали цикл, <<нарисовали>>.
\item
	Теперь взяли рёбра и разбили их на компоненты относительно цикла.
	Их можно укладывать независимо друг от друга, лишь бы выбрать, снаружи или внутри.
\item
	По одну сторону две компоненты можно, если концы, касающиеся цикла, одной их компонент лежат целиком между парой соседних концов другой компоненты. \TODO. Это делаем так: взяли одну компоненту, расположили по циклу, перебрали другие компоненты вторым указателем, умещается ли она в текущей паре. Это делается за $O(K^2 + V)$.
\item
	Раскрасим этот граф совместимости в два цвета. $O(K^2)$. Если не получилось... Ну очень жаль. Дальше в каждой независимо.
\item
	Теперь, взяли компоненту, хотим её уложить.
	Взяли путь между двумя вершинами цикла, нарисовали его, разбили по двум цветам относительно него рёбра компоненты (какие можно слева, какие можно справа), продолжаем рекурсивно на всём слева и справа.
\end{enumerate}

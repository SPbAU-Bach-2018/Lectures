\date{Sentember 12, 2016}
\author{Bugakova Nadezhda}

\section{FFT}

$\omega_n = e^{2\pi i/n}$

$\omega_n^j = e^{2 \pi i j/n} $

2 задачи:
\begin{itemize}
	\item Экстерполяция(по многочлену - значения в точках).
	
	$P(x) \xrightarrow{FFT} P(\omega_n^0), P(\omega_n^1), \dots$
	
	$P \cdot Q (\omega_n^j) = P(\omega_n^j) \cdot Q(\omega_n^j)$ // $\O(n)$
	
	\item Интерполяция(по точкам - многочлен).
	
	$P(\omega_n^0), P(\omega_n^1), \dots \xrightarrow{FFT^{-1}} P(x)$
\end{itemize}

$FFT^{-1}$ =  \begin{enumerate}
	\item FFT
	\item reverse(a + 1, a + n). $a[0]$ остаётся на месте
	\item $a[i] /= n$
\end{enumerate}

Псевдокод:
\begin{cppcode}
FFT(n, p) { // $n = 2^k, \omega_n^j - we want to count in this roots$
	if (n == 1) { return p[0] }
	// $P(x) = P_0(x^2) + xP_1(x^2) - divide degrees on even and odd
	for i = 0..n - 1
		A[i % 2].push_back(p[i])
	F_0 = FFT(n/2, A[0])
	F_1 = FFT(n/2, A[1])
	for i = 0..n - 1
		res[i] = F_0[i % (n/2)] + w_n^i \cdot F_1[i % (n/2)] // P_0(w_n^{2i}) = P_0(w_{n/2}^i)
\end{cppcode}

$FFT = \theta(n \log n) \Ra Mul = \theta(n \log n)$

\section{Разделяй и властвуй}
\subsection{Перевод из одной системы счисления в другую}

Например, их десятичной в двойчную(10 $\rightarrow$ 2).

$A = A_0 \cdot 10^{n/2} + A_1$

$B_0 = Get_2(A_0)$

$B_1 = Get_2(A_1)$

$C = Get_2(10^n/2)$

$B = B_0 \cdot C (FFT - n \log n) + B_1$ - всё в двоичной системе счисления.

Так мы сможем решить задачу, но тут три рекурсивных вызова, а мы хотим 2! $\Ra$ предподсчёт!

$n = 2^k$

$10^1, 10^2, 10^3, \dots, 10^{2^k}$ - просто считаем втупую.

\[ \sum\limits_{\text{По степеням двойки}} i \log i = \theta (n \log n) \]

Итого, Работа = $T(n) = 2T(n/2) + MUL(n \log n for FFT)$

Предподсчёт = $\sum i \log i = \theta(n \log n)$

\underline{Мастер-теорема:}

Пусть $T(n) = aT(n/b) + n^c\log^d n = aT(n/b) + f(n)$.
\begin{itemize}
	\item $a < b^c \rightarrow T(n) = f(n)$
	\item $a = b^c \rightarrow T(n) = f(n)\cdot\log n$
	\item $a > b^c \rightarrow T(n) = a^{\log_b n} = n^{\log_a b}$
\end{itemize}

Подсчёт времени на перевод:
\begin{itemize}
	\item Если MUL = FFT, то T(n) = ?. По теореме $T(n) = n\log^2 n$(a = 2, b = 2, c = d = 1)
	\item Если MUL = Карацуба, то умножение занимает: $T(n) = 3T(n/2) + n = n^{1.58}$
	
	Таким образом, $T(n)$ для перевода: $T(n) = n^{1.58} = f(n)$ (a = 2, b = 2, c = 1.58)	
\end{itemize}

На практике непонятно, что лучше. Чем больше n, тем FFT лучше. Например, при n = 100 000, FFT уже лучше.

// Java 1.7 $\rightarrow$ Java 1.8 быстрый BigInteger! Не хуже Карацубы.

Python MUL = Карацуба, но 2 $\rightarrow$ 10 работает за $n^2$, поэтому вывод такой долгий. //

\subsection{Деление}

A(x) = B(x)Q(x) + R(x) $\in \R[x]$, $\deg R < \deg B$.

По A, B хотим найти Q и R.

$T(n) = 2T(n/2) + n \log n$ - хотим применить метод разделяй и властвуй, чтобы получилось такое уравнение.

Если получим Q, R потом посчитать несложно.

Утверждается, что чтобы найти Q, достаточно знать несколько первых коэффициентов A и B. К примеру, если $deg(A) = 100, deg(B) = 100$, достаточно всего одного старшего коэффициента у A и у B. А если $deg(A) = 100, deg(B) = 98$, то достаточно всего три старших коэффициента. Отсюда:

\underline{Th:}

x - количество старших коэффициентов у A и B, которые нам необходимо знать. Тогда $x = k_1 - k_2 + 1$, где $k_1 = deg(A), k_2 = deg(B)$.

\underline{Док-во:}

Деление в столбик.

Пусть Div(n, A, B) - находит n старших коэффициентов частного A и B. $n = 2^k$. Если частное имеет степень меньшую n, то мы считаем также несколько коэффициентов при отрицательных степенях частного - ничего страшного. Изначально n - первая степень двойки, большая или равная $k_1 - k_2 + 1$.

Псевдокод:

\begin{cppcode}
Div(n, A, B) {
	if (n == 1): return A[0]/B[0]
	C1 = Div(n/2, A[:n/2], B[:n/2])
	A -= B*C_1 // Now first n/2 coefficients A is null. Here's FFT.
	C2 = Div(n/2, A[n/2:], B[:n/2])
	return [C1, C2] //concatenation 
}
\end{cppcode}

Посчитаем время. T(n) = 2T(n/2) + $\left[ \begin{gathered} 
										n \log n -- FFT \\
										n^{1.58} -- \text{Карацуба}
										\end{gathered}
								\right.$
								
\subsection{Экстраполяция(значения в точках)}

$x_1, x_2, \dots, x_n$ - произвольные точки. Хотим посчитать значения в них.

План действий:
\begin{enumerate}
	\item $P(x)$ mod $(x - x_1)(x - x_2)(x - x_3)\dots(x - x_n) =: Q(x)$
	\item $Q(x_1, \dots, x_{n/2})$
	\item $Q(x_{n/2 + 1}, \dots, x_n)$ 
\end{enumerate}							

Псевдокод:
\begin{cppcode}
Calc(P, n, x1, x2, x3, .., xn) {
	if (n == 1): return P(x1)
	P %= (x - x1)(x - x2)..(x - xn) //after this, deg P < n
	A1 = Calc(P, n/2, x1, x2, .., xn/2)
	A2 = Calc(P, n/2, x(n/2 + 1), .., xn)
	return A1 concat A2
}
\end{cppcode}	

Остался единственный вопрос - как мы быстро находим многочлен $(x - x_1)(x - x_2)\dots(x - x_n)$?

Дерево отрезков! (Картинка) 

Считаем время. Mod = Div + Mul

T(n) = 2T(n/2) + Mod + Mul(<- на каждом уровне ДО FFT, мы его включаем в уравнение тоже на каждом уровне)

Mod = $n \log^2 n \Ra n \log^3 n = T(n)$ - FFT

Mod = $n^{1.58} \Ra n^{1.58} = T(n)$ - Карацуба

Здесь разница стала уже не такой большой. Поэтому, 

\underline{Правильный разделяй и властвуй}:

$\left[
	\begin{gathered}
	FFT -- n \ge 32 \\
	\text{Карацуба} -- n \le 16
	\end{gathered}
\right.$

\subsection{Интерполяция}
Задача аналогична предыдущей, только наоборот. Есть произвольные точки, а нам надо восстановить многочлен. Для этого мы вспоминаем интерполяционный многочлен Ньютона и его построение.

$(x_1, y_1), (x_2, y_2), \dots, (x_n, y_n) \rightarrow P(x)$

Псевдокод:

\begin{cppcode}
Calc(n, x1, y1, x2, y2, ..)
	P = Calc(n/2, x1, y1, x2, y2, ..)
	z(x) = (x - x1)(x - x2)..(x - xn/2) // we can count this with help segment tree
	yi` = (yi - P(yi))/z(yi) // for all i = n/2 + 1..n
	Q = Calc(n/2, x(n/2 + 1), y`(n/2 + 1), .., xn, yn`)
	Ans = P + z(x)Q
\end{cppcode}

$T(n) = 2T(n/2) + Mul + \text{Экстраполяция(когда считали P(yi))}$, если считать всё в терминах FFT, то $T(n) = n \log^4 n$.

\underline{Упражнение:}

Посчитать $\sqrt{P(x)}$ методом разделяй и властвуй. Корень многочлена - это такой многочлен Q, степени не меньше n/2, что deg($Q^2 - P$) $\rightarrow$ min.

\section{Деление за \texorpdfstring{$n\log n$}{n log n}}

\begin{itemize}
	\item Деление чисел = метод Ньютона.
	
	$x_{i + 1} = x_i - \frac{f(x_i)}{f'(x_i)}$ Тогда, если в $x_i$ было k верных знаков, то в $x_{i + 1}$ примерно 2k верных знаков.
	
	Из матана мы знаем: $f(x) = f(0) + xf'(0) + 1/2x^2f''(\xi)$, для какой то $\xi \in [0, x]$. Если x - корень f, то уравнение получается таким: $0 = f(0) + xf'(0) + 1/2x^2f''(\xi)$, таким образом, $x = \frac{1/2x^2f''(\xi) + f(0)}{f'(0)}$.
	
	Таким образом, если изначальная разница была $x - 0 = x$, то после того, как мы сделали переход по правилу и получили новый $x'$, $x - x' = 1/2x^2\frac{f''(\xi)}{f'(0)}$. Если $M = \frac{max f''}{min f'}\frac{1}{2}$, то разница в $x$ переходит в разницу $Mx^2$. А если в терминах точных знаков, то $k$ точных знаков становятся $2k - \log M$ точных знаков. 
	
	Чтобы M ничего не портило, надо начинать с достаточно большого приближения.
	
	Но всё, что наверху относимо в общем к принципу Ньютона. Мы хотим научиться делить числа. Деление чисел = взятие обратного + умножение, так что научимся брать обратный. То есть $a \rightarrow a^{-1}$.
	
	Так как Фурье умеет работать только с целыми, то вещественные числа храним так: Integer number + Integer pos. При перемножении number перемножаются, а pos складываются.
	
	В методе Ньютона для наших целей $f(x) = \frac{1}{x} - a$. $x \rightarrow x - \frac{f(x)}{f'(x)} = x + \frac{1/x - a}{\frac{1}{x^2}} = 2x - ax^2$.
	
	Откуда начинаем? Начинаем со значения $x^{-1}$ в double. Там будут 15 точных знаков. Можно считать, что каждый раз количество точных знаков увеличивается в полтора раза, чтобы не оценивать M. Каждый раз отрубаем по количеству точных знаков.
	
	$T(n) = \sum\limits_{k} 1.5^k\log(1.5^k) = \theta(n \log n)$, где n - количество точных знаков, которые мы хотим получить.
	
	\item Многочлены.
	
	A(x) $\in \R[[x]]$ - формальный степенной ряд. Чтобы у него был обратный, $a_o \ne 0$.
	
	Обратим бесконечный ряд. $A(x) = a_0 + a_1x + \dots$, deg A = n. $\frac{1}{A(x)}$ может не иметь степени, поэтому надо заранее понимать, какое количество коэффициентов нам надо.
	
	Тупо, коэффициенты можно посчитать по формуле из Дискретки:
	
	$b_0 = \frac{1}{a_0}$
	
	$b_1 = -b_0a_1/a_0$  
	
	$b_2 = -(b_0a_2 + b_1a_1)/a_0$
	
	\dots
	
	$b_i = -\dots$
	
	Длина i-го уравнения - это i $\Ra$ Втупую работает за $n^2$.
	
	А теперь метод \underline{разделяй и властвуй}. Хотим $n\log n$.
	
	\begin{cppcode}
Inv(A) {
	if (n == 1): return {1/a[0]}
	R = Inv(n/2, A)
	Q = A*R(FFT) - 1 and shift
	Ans = -Q*R
}	
	\end{cppcode}
	
	Объяснение:
	
	$A*R = 1 + x^{n/2}Q$
	
	$(R + C*x^{n/2})A = 1 + x^nT$ - нам надо найти ещё С.
	
	$1 + x^{n/2}Q + AC\cdot x^{n/2} = 1 + x^nT$
	
	$Q + AC = x^{n/2}T$
	
	$AC \approx -Q$
	
	$C = (-QR)$
	
	Время работы:
	
	$T(n) = T(n/2) + Mul = T(n/2) + n \log n = n \log n$
\end{itemize}


\end{document}

\chapter{Молекулярная биология}
План на первую половину: 
\begin{enumerate}
\item организация генома, что такое геном. 
\item центральная догма молекулярной биологии.
\item клеточный цикл
\item починка ДНК
\item ДНК рекомбинация. 
\end{enumerate}

\section{Что такое биология? И зачем ее учить?}

\includegraphics[width=0.25\linewidth]{im1.png}

Биология это наука о жизни и учить ее можно, потому
что это любопытно.

Что такое живое?

- Давай-те последовательно. Сосед живой? Парта живая?
- А какой есть пограничный вариант? 
- Вирус. 
- Почему? 
- У него нет клеток и вне клеток он ведет себя как 
неживой. 
- Но вас если в космос выкинуть вы тоже вести себя будет как 
неживое существо. 
- Но я потом уже не ставну живой. А его если потом
выкинуть в клетку, то он станет.
- Ну всмысле это у него обратимое состояние. Но я
мог бы поспорить, сказав, что есть вирусы,
которые никогда не выходят за пределы клетки. И
если попробовать его вытащить за пределы клетки - он умрет.
- Ну в общем, все организмы у нас в той или иной степени живые. 
И восприятие вирусов как неживых связано скорее с некоторым артифактом
в биологие, о котором мы поговорим чуть дальше. 

\section{Рекомендуемое чтение}
\includegraphics[width=0.25\linewidth]{im2.png}

Очень простая тоненькая книжка. Практически для детей. Начинается 
с химии. Даже химии знать не нужно. Чтение исключительно для 
удовольствия. 

\includegraphics[width=0.25\linewidth]{im3.png}

Книжка для задротов. Уровень биологии для поступления в аспирантуру.
Плюс учебника, что тут все с нуля. Там есть все. Там 1400 страниц
и это можно долго читать, очень хорошо написано с хорошими примерами.
Книжку можно скачать(этого не говорили).   

\subsection{Клеточная теория}

\includegraphics[width=0.25\linewidth]{im4.png}

Теперь про артифакты. Суть в том, что вирусы считаются 
как бы не совсем живыми, потому что у них как бы нет 
клетки. 

Что такое клетка?
Клетка была открыта аж очень давно. И была открыта Робертом
Гуком на микроскопе, который он сконструировал сам. Это не тот
микроскоп. Тот микроскоп который он использвал он вообще дикий.
Поскольку таких глаз больше нет.

Мощность линзы определяется кривизной. То есть,
самая клевая линза будет шарик. 

Такой линзой здоровый глаз пользоваться не может.
Просто точка фокуса не будет где-то рядом. Но
у людей есть всякие хитрые формы глаза. Там, с
близарукостью и еще чем-нибудь. И может получиться
так, что этой линзой все-таки пользоваться можно.

То есть, по сути, первый микроскоп это был
просто зажим для кругленького шарика стекла.
Который, если шарик практически упереть в глаз,
то можно рассматривать что-нибудь, что находится
с другой стороны шарика.

Собственно клетка. Почему обнаружили клетку?
почему не обнаружено что-то другое?

Клетки заметить легко. Почему? Потому что есть
оболочка. Потому что он смотрел срезу
коры пробкового дерева и там по сути самих живых
клеток нет. Там по сути просто
клеточные стенки. Которые состоят у ростений из цилюлозы.

То есть он заметил просто мертвые останки клеток и
оно обладало некоторой регулярной структурой. Именно просто как
ячейки.

Потом в микроскоп стали смотреть все, кому не лень и
сформировалась некоторая клеточная теория.

В приметивном виде состоит из трех позиций:
\begin{enumerate}
\item 
Все живые организмы состоят из одной или больше клеток. 
\item 
Клетки - самые базовые структурные единицы всех живых организмов. 
В ней происходит метоболизм, вней происходит хронение генетической 
информации. Передача и так далее. 
\item 
Все клетки произошли от предыдущих клеток. 
\end{enumerate} 

Почему возникла такая мысль? Потому что 
смотрели на все, что есть вокруг. 
Смотрели на срезу животных, растений, 
грибов и еще кого-нибудь и везде видели одно и тоже. 

Давай-те ограничения. Откуда взялась первая клетка? Никто 
не знает. Есть некоторые гипотизы происхождения живых органимов. 
Но как это происходило на самом деле, в общем-то, никто не знает. 

\subsection{Вирусы}
\includegraphics[width=0.25\linewidth]{im5.png}

Маленькие инфикционные агенты. Живут и реплицируются только 
внутри клеток. И поражают, в общем-то, все известные живые 
организмы. 


Все организмы, включая вирусы. Есть вирусы, поразитирующие на вирусах.
Чаще всего вирус, который заражает один организм, не заражает другой. 

Вирус он специально задизайнен, что бы преписать, взломать набор 
молекулярных механизмов, если этот набор отличается, даже если 
он попадет внутрь, то его система не сработает. Клетка не воспримит 
сигналы и просто проигнорирует вирус. 

Почему их назвали инфекционными. Заболевания растений были известны 
давно. Были потом открыты бактерии, микро организмы, все хорошо. Все происходило 
так, что используя супер тонкие фильтры, 
размером в сотню нанометров можно было это все отфльтровать. 
То есть береж зараженный куст, 
который видно, что болеет бактериями. Перемалывается в труху, 
прогоняют через фильтр получается жидкость и если 
эту жидкость нанести на другие расстния, то 
ничего не будет. Все боктерии окажутся на фильтре. 

И считалось, что этот фильтр пропускает только не органические вещи. А
потом нашли заболевание табачная мозайка. Прогоняют через фильтр, 
берут больной лист, наносят и он становится больным. И не могли понять, 
что за фигня, поскольку все хоть сколько-то известные крупные компоненты должны 
были остаться на фильтре. Поэтому появилась фигня называемая инфекционностью, 
из серии заражает не смотря на фильтр. Потом когда появились 
электоронные микроскопы, то стали брать срезы расстений и 
смотреть прям отдельно клетку. И в ней видно, что 
происходит сборка некоторых частиц. Их не сразы связали с инфекционностью. 

\subsection{ДНК}
\includegraphics[width=0.25\linewidth]{im6.png}

Что такое ДНК? Расшифровывается как
дизоксирибо-нуклеиновая кислота. Или еще 
их исходно называли нуклииновыми кислотами. Что означает, что 
это кислота? Это значит, что это отрицательно заряжено. У кислот есть 
OH группа, если есть OH группа, значит заряд будет отрицательный.

Это важно для ДНК.

Как это было открыто? Человек достаточно упоротый тогда был, это было очень очень давно.
До понимания процесса генетики и так далее. Он брал клетки и доставал из
них ядра, потом эти ядра он перемалывал в кучу и потом
смотрел из каких молекул они состоят. То есть пытался выяснить какие-то
базывае характеристики. Вот он и выяснил, что это кислота. То есть понятно,
что это полимер, то есть состоит из многих маленьких компонент.

Есть два слова мономер и полимер. Полимер состоит из мономеров. 
Нуклиотид - это мономер, а все вместе - это полимер.

У ДНК есть сахорофосфадный остов и есть сами буквы.

Что мы еще знаем о днк? Двойная спираль. 

У нас в алфавите есть 4 буквы. Что мы о них знаем. Мы о них знаем одну важную вещь,
что они комплиментарны друг другу. A - T, C - G.

Аденин будет стараться оказаться рядом с Тимином образовав две водородные
связи. Гуганин будет образовывать с Цитазином три водородные связи.

Почему это важно? Потому что вторая связь крепче чем первая.

В двухцепочечной молекуле ДНК на против A всегда будет T и
напротив C всегда будет G.

У ДНК есть направленность. 5' 3' Свойство, что
мы можем из одной цепочки востановить вторую - называется - комплементарность.
Точнее обратно-комплементарная цепь. 

Название дизоксирибонуклеиновая кисота. По названию понятно, что там 
должна быть риба - рибоза. Нуклииновая, то есть там должны быть нуклиатиды. 

Там есть сахорофосвадный остов. Там есть радикалы, к которым крепятся 
нуклеотиды. И мономер он с куском рибозы. 

Почему 5' 3', потому что эта свзяь не симитрична. И фосфоры присоединяются в разных 
местах, поэтому цепь может удлиняться только в одну сторону. То есть новая буква 
может присоединяться только к 3' концу.

Из этих буковок состоит весь геном. И как мы недавно узнали в них
закодированна некоторая генетическая информация. Слово кодирование искользуется
не просто так. ДНК состоит из нуклеиновых кислот. Из чего состоят живые организмы?
Из клеток, клетки состоят из беков, жирков, углеводов и нуклеиновых кислот. Теперь вопрос,
каким образом из последовательности
нуклеиновых кислот можно собрать организм не только из них?

В ДНК информация хранится в линейной форме и последовательность сигналов
нуклеиновых кислот можно перевести в последовательность сигналов
аминокислот. После этого аминокислоты собираются уже не в линейную структуру,
делают они это уже достаточно хитро. И после этого белки собирают
уже все остальное, в том числе новые копии нуклеиновых кислот.

  
\includegraphics[width=0.25\linewidth]{im7.png}

Сколько существует аминокислот?
20?
Клевый вопрос, что бы всегда оказаться правым, потому что
аминокислот бесконечно много. Потому что аминокилоты - это
класс химических соединений. Их выдумывать сколько угодно.
А вот в живых организмах их действительно 20-21, смотря как 
считать. Никто же не мешает собрать свой живой организм, 
который будет состоять из аминокислот, которых больше нигде нет. 

То есть, у нас 64 кодона должны кодировать всего 20 аминокислот. 
То есть у нас какие-то кодоны либо ничего не кодируют, 
либо кодируют одно и тоже. 

То, что они кодируют одно и тоже называется вырожденностью. 
То что мы можем однозначно перевести из последовательности 
нуклеотидов в последовательность аминокислот, 
а вот если я вам дам последовательность аминокислот, 
то вы можете соченить много 
последовательностей ДНК из которых это было получено. 

Есть понятие открытой рамки считывание. Не весь 
геном, который есть внутри нас он хоть что-то кодирует. 
Кодируют только его некоторый фрагмет. Который 
находится между сигналами. Есть сигнал начала, 
есть сигнал конца. 

Что являются сигналами начала и конца? 
Сигнал начала, это вполне конкретный кодон. ATG, он 
кодирует метианин. Ген всегда начинается с 
ATG. 

А как природа научилась кодировать сигнал конца? Кодоны, 
которые кодируют конец они никаких 
аминокислот не кодируют. 

Бежит молекула и собирает аминокислоту. Потом она получает 
сигнал, которого как бы не существует. И какое-то время сидит ждет, 
поскольку она какое-то время вообще не получает ответа, 
комплекс разваливается и все готово. 

Стоп кодона у нас обычно три и при этом некоторые животные, 
в том числе мы можем читать стоп кодон как кодирующий сигнал. 

То есть это дает некотрое разнобразие в интерпретации информации. 
То есть мы можем одну и ту же последовательность читать поразному. 

Генетический код не универсален, как это не странно. В метохондриях
у нас один генетический код, так как метохондрии давным давно были 
паразитами симбионтами в эукориатических клетках и с тех 
пор несут некоторое бакториальное наследие. Генетический код 
у нас разный. GAG в одном организме у нас кодирует 
одну аминокислоту, в другом другую, а в другом вообще кодирует 
стоп кодон, то есть ничего не кодирует. То есть 
генетический код меняется, эвалюционирует и 
так далее.

\includegraphics[width=0.25\linewidth]{im8.png}

Кто-то говорил про РНК и вот она. rRNA - рибосональная 
РНК. рРНК состоит из двух субъединиц большая и малая. 

То что здесь протаскивается - это матричная РНК. Это 
цепочка РНК, на которой как раз закодированно что-то из 
ДНК. ДНК у нас превращается не на прямую в белок, 
а ДНК сначала превращается в РНК, 
а потом РНК превращается в белок. 

Есть ДНК, есть РНК, есть белки, когда молекулярная 
биология достигла уровня, где мы собрали много информации 
как и что происходит, люди заметили, что из 
ДНК получается РНК, из РНК получается белки и из 
ДНК должно получаться ДНК, а то не имеет смысла. 

И это считается классикой, это у нас 
происходит в организме постоянно. Стандартная схема, 
при деление клетки копируется ДНК, при работе клеток 
из ДНК читается РНК, которая потом превращается в белок. 

Какой функционал у РНК? Почему не синтезировать 
белки сразу на основе ДНК? Разница между ними минимальна. 
Считается, что жизнь возникла как РНК. 

ДНК хранить удобнее, дизоксирибонуклиновая кислота, из 
названия понятно, что там не хватет окси группы. А это 
просто рибонуклеиновая кислота. И если рисовать РНК, то 
там еще торчит окси группа, которая уменьшает стабильность 
двойной цепочки. 

РНК, сама по себе, как молекула менее стабильна. Она быстрее 
разрушается и плохо хранится. Возникновение ДНК, что бы 
было удобно и много хранить. 

Ну это не объясняет, нафига нам использовать после этого РНК. В чем 
приимущества? 

Можно представить, что у нас есть большой карабль и надо его 
иногда чинить. У нас есть огромная книга со всем чертяжами, 
но мы не будем ее каждый раз таскать, нам только нужны опредеенные странички. 
То есть РНК это просто кусочки ДНК. 

Это все было бы хорошо, если бы у нас 
еще не было вариантов из РНК в ДНК и из РНК в РНК. Можем
еще заставить происходить из ДНК в белки, но в природе такого нет.

Есть вирусы, которые существуют только в форме РНК. Опять же,
когда мы говорим об РНК нужно понимать, что там есть еще + и -. РНК
обычно существует в форме одной цеопчки, а не двух.

Рибосональная РНК кодируется в ДНК. У РНК есть
еще одна приятная штука, в отличие от ДНК. Они могут
принимать всякие разные вторичные/третичные постранственные структуры.
И ДНК не может ничего, кроме как хранить информацию, а РНК может
мало того, что хранить, так еще и самовоспроизволиться, например.

Рибосомальная РНК кроме того, как кодирует свою структуру, эта структура 
сворачивается в такие клубки, которые после этого умеют 
читать mRNA и сентезировать белки.

mRNA просто чертеж, а rRNA эта штука, которая умеет читать этот чертеж и соберать
на его основе белок.

tRNA - от слова транспортная, потому что она подтаскивает к rRNA аминокислоты.

На каждый тип кодона есть своя аминокислота(?tRNA), которая тоскает аминокислоты. 

\includegraphics[width=0.25\linewidth]{im9.png}
Внутри РНК процесс называется трансляция, из ДНК-ДНК - репликация, 
вот это транскрипция, а это трансляция.

Мы говорим о переходе из РНК в белок. Вот у нас рибосома. Две суб единицы.
Котрая сидит на mRNA. Внутри рибосомы можно условно выделить
три области. голубые кусочки - это транспортная РНК. Снаружи она плавает
как транспортная рнк и к ней пришита аминокислта. Рибосома узнает
некоторый триплет. На mRNA есть три буквы. и Трансопртная РНК тоже имеет
три буквы, комплементарные этим трем. Поэтому происходит узнование.

У РНК есть хвост, который просто тащит аминокислоты. Это просто транспорт. Этот
хвост принципиально узнает только одну конкретную аминокислоту. Сколько типов
транспортных РНК вообще бывает? 64. А сколько тоскать разных аминокислот - 20.
tRNA 64 разны вариантов, а с другой стороны только 20 аминокислот.

tRNA бывает в двух формах, активированная и нет. tRNA создает такую пространственную
структуру, что бы драгие белки ее узнавали и цепляли нужную аминокислоту.

mRNA не вся кодирует, у нее есть начало, которое ничего не кодирует. содится рибосома и
начинает двигаться по mRNA. Находит старт сигнал ATG. Всего в рибосоме есть три места.
Место узнование. Место объединания и выход. Она двигается линейно всегда в одну сторону.
Она не может двигаться в другую. Попала в это место, поняла, что 
это старт сигнал. После этого она будет ждать, пока 
сюда приплывет транспортное РНК, которое комплементарно этой последовательности. 
И tRNA принесет свою аминокислоту, шаг второй. Шаг равен у нее трем буквам. Теперь эта 
транспортная РНК переехало на вторую позицию и мы ждем, пока сюда придет следующая 
транспортная РНК. Теперь ситуация становится интереснее. Штука со второй позиции оказывается 
в последней, но сам процесс становится хитрее. Второй хвост пересаживается на аминокислоту. 
И при следующем шаге оказывается, что на второй позиции уже висят две бусинки, tRNA, которая 
передала свои аминокислты вышла, а на превой позиции ждем tRNA.

 
\includegraphics[width=0.25\linewidth]{im10.png}
Как живые организмы вообще бывают? 
Есть множества просто живые, какие бывают самые крупные подмножества 
этого множетсва. 

Бактерии, археи и эукориоты. 

Археи вообще не очень давно известны, бактреии известну уже 
достаточно давно и они все отличатся от эукориот. На уровне 
гентеического сходнстав 5\% сходства и 95 \% различий относительно каждой из этих 
групп. При этом бактерии и археи внешни очень похожи. И их 
долгое время ситали одним и тем же, до тех пор, 
пока не заглянули внутрь. А работают они совершенно подругому. 

И их вместе называют прокариоты. 

Начнем с них, поскольку ее условно можно назвать проще. Эукороты 
можно перевести как истиноядерные, а прокориоты - это доядерные. 

У них нету ядра, у них есть некотрый нуклиус, просто вся ДНК
клетки плавает более менее в одном месте более менее организовано.

Тут есть клеточная стенка, есть плазмотическая мембрана. Плазматическая мембрана
это просто то, что отделяет клетку, от окружающей среды. Живые организмы стараются
быть некоторым образом обособленны. И плазматическая мембрана это и делает.
                                                      
Из чего она состоит? Там белепидный слой... Какое основное
свойство должно быть у плазмотической мембраны, что бы выполнять роль обособления
от окружающего мира. Она не должна быть растворима в воде,
при условии, сколько вокруг воды и что все химические реакции идут в воде. Основное
свойство мембраны - не растворимсть в воде. То есть на нужно одну воду отделить от другой.
То есть должна быть гиброфобна. Это похорошему, хорошо структурированная, хорошо 
организованная жировая пленка. 

За прочность отвечает как раз слой под мемдранной, под названием клеточная стенка. Которая 
дает жесткую прочность. Бактерии как раз очень жесткая, ограничение объема из-за 
одной структуры, а другая дает жесткость каркаса. Зачем ей быть жесткой? Вокруг у нас
вода. Есть такая вещь, как астматическое давление. Берем мемдрану, которая 
частично проницайма для воды. И с одной стороны мы кладем много соли или 
сахора. Соль и сахар мембрана не пропускает. Вода будет стремиться попасть туда, 
где ее концентрация меньше. Так как переехать она не может, 
то соль будет оказывать давление на мембрану. Ей нужна крепкая стенка, 
потому что бактерия хочет много всего хранить. Ей нужно хранить много углеводов, 
много солей и всего остального. Если бактерию лишить этой стенки, то она лопается. 
Если кровь человека поместить в водно-солевой раствор с низкой концентрацией, кровенные 
клетки полоппуются. Потому что внутри концентрация всяких няштиков больше, чем снаружи и
они будут стремиться вылезти наружу. 

Спомощью мембраны бактериальная клетка вырабатывает энергию. Есть
специальные белковые комплексы под названием АТФ синтазы. АТФ - это
универсальный источник энергии для клетки. Есть белковый компелкс,
который умеет из подручных средств сентезировать молекулы АТФ.
Молекулы АТФ удобны тем, что их легко транспортировать в разные места.
Они универсальный источник энергии во всех клетках.

АТФ = аденазин три фосфат.

Родственница молекуле аденина. аденазин + фосфар + фосфар + фосфар. И именно
энергия связий этих трех фосфоров является запасанием энергии.

То есть штука работает следуюшим образом, она берет аденазинмонофосфат или 
аденазиндифосфат и пришивает к ним новые фосфары, которые можно будет оторвать
и получить новую энергию.

Но от куда эта штука берет энергию. А она создает разность потанцеалов. С одной
стороны она создает минус, снаружи положительный и мембрана сома по себе не
проницаема. Протоны начинают стримиться попасть внутрь клетки. Для них есть
для этого открытый канал и пока они движутся через него, они крутят что-то типа мельницы.

Двигаясь через АТФ синтазу они разгоняют ее и она синтезирует АТФ.

И как наш организм любит бороться с бактериями? В бактериях протыкается
дырка и тогда заряд двигается куда угодно. Но бактерии это умеют затыкаьть.

Есть круче, когда клетка знает, что надо бороться с бактериями, чаще всего клетка это
знает, если она попыталась бактерию съесть и у нее это не получилась, она сопративляется.
Она закисляет среду снаружи. Тогда снаружи получается тоже минус, заряд
уравновесился, бактерия не может синтизировать энергию.

Это происходит только в том случае, если клетка умеет реально быстро закислять. Иначе бактерия
начнет более активно создавать разность потянциалов.

Мы получаем энергию ровно таким же образом.

Метохондрии - это от всей бактариальной клетки осталась по сути
мембрана, напичканная АТФ синтазами. То есть метахондрии -
это такие заводы по производству АТФ, которые являются далекими
потомками бактерий.

Прикол в том, что эукориоты - это результат
симбиоза кого-то и бактерий.

Хлоропласты, например, это симбиатические водрасли. Тоже
результат симбиоза.   

\includegraphics[width=0.25\linewidth]{im11.png}

Прокариатический геном. Он кольцивой. Обычно 
делится на хромасому и плазмиды. Хромосомы бывает размером 
5миллионов букв, бывает сильно меньше, 
бывает сильно больше. 

Есть плазмиды, чем отличается от хромосомы - никто толком не знает, 
раньше отличались, потому что они просто считались разнми вещами.
Они отличаются по размеру, плазмиды меньше и плазмиды не 
обязательны. Бактерия может терять плазмиды, бактерия любит 
обмениваться плазмидами. Новые плазмиды делать. 

Обязательность и необязательность плазмиды сильно зависит от услловий.
Например плазмида, которая дает
устойчивость к антибиотикам. Она реаьна полезна только
если антибиотик есть и бесполезна если антибиотика нет,
так она только энергию тратить.

Поэтому нужно говорить не обящательна в каком случае.

В чем разница между ДНК и хромосомой?
Хромосома это ДНК связанная с кромотимой. ДНК
просто так не хранится. Например,
как это происходит у нас. Есть бусинки,
которые можно назвать гистонами и ДНК просто на них
наматывается. Потом бусинки можно перемешивать,
собирать вместе и так далее. ДНК поскольку
просто не существует в чистом виде, она
всегда связана с некоторыми белками.

Хромосом может быть несколько.   

\includegraphics[width=0.25\linewidth]{im12.png}

ГМО - генетически модифицированный организм. Изобрела природа. 
Первые ГМО стали делать бактерии, милионнов 60 лет назад. 

Есть бактерия, которая называется агробактерия. Живет внутри 
растений определнного вида. Тех же 
саммых бобовых. И они модифицируют им геном таким образом, 
то есть при заражение расстения они делают вокруг 
себя опухоль. Им там комфортно живется. Растене
не то что сильно против, так как пока оно
там живет оно снабжает
растение дополнительными азотистыми соединениями. Растение
предоставляет бактерии питани, то есть
у них в какой-то степени симбиоз. Но вопрос, как это
делает бактерия. У бактерии есть
такая штука под названием T-плазмида, на T-плазмиде есть
T регион. Это кусок плазмиды, которую бактерия
встраивает в геном растения. И люди научились это
достаточно давно и достаточно клево эксплатировать. То есть
если заменить этот регион на те гены, которые нам нужны, то бактерия просто возьмет
и воткнет этот кусок в растение. Таким образом,
сейчас есть методы, когда практически в домашних условиях
можно сделать растение, которое хочешь.

Всего лишь надо самому собрать себе плазмиду.

Сейчас есть технологии под названием криспы, дает
возможность обращаться с геномом на уровне текстового редактора.
Удалать букву, добавить букву, вырезать это,
вставить это.

Самое простое что делается - делается
светящиеся растение. Можно например сделать
какой-то белок, что бы он был связан со светящемся
компонентом. Например, хотим
ответить на вопрос, в каких местах
растения у нас работает такой-то белок.

    
\includegraphics[width=0.25\linewidth]{im13.png}

Зеленое, это то про что 
мы уже говорили. Это кодирующая часть. Здесь 
находится старт кодон и стоп кодон. 
Это ДНК. RBS - место посадки рибосомы. Рибосома 
ищет, где находится первый старт кодон. Это она здесь по 
сути и делает. Есть места под
названием 5' UTR, 3' UTR. Это не транслируемая часть. 
Теперь здесь появился кекс - называется промотер. Эта 
важная штука, поскольку так мы понимаем какой ген взять. 
У нас есть система сигналов. Есть много разных 
промотеров. Это как знак припенание, который говорит - вот 
здесь ген. Или указатель. Если нужно, что бы 
что-то заработало ищется конкретный промотер, так
как рядом с этим промотером есть гены. Промотер 
всегда находится на вполне фиксированном расстояние от начала гена. 

У промотеров есть определенный мотив. То есть определенная последовательность. 

Есть такая вещь как транскрипционные фаторы, которые как раз и запускают 
процесс транскрипции, то есть превращение ДНК в РНК. И эти транскрипционные 
факторы узнают эти мотивы. Если есть транскрипционный фактор и
есть сочетние, то они находят друг друга и это является сигналом для многих
разных других штук,
которые плавают вокруг. Что надо прийти и сделать на этом месте РНК.

И это достаточно простая структура гена.  

\includegraphics[width=0.25\linewidth]{im14.png}

Локтозный оперон. У бактерий гены организованны не просто, 
бывает конечно и так, что у нас есть промотер и на один ген есть
один промотер, но бактерии стремяться улучшить эффективность работы.

Есть штука под названием оперон, которая объеднияет в себе много
генов. Локтозный оперон занимается тем, что производит гены необходимые
для переваривания локтозы. Там несколько белков,
такие, что когда они есть бактерия
умеет усваивать локтозу.

У нас в геному есть регуляторный ген. Который в ситуации, когда у нас
локтозы нету... Бактерии нет смысла производить все это, если
она не может получить энергию. Поскольку белки живут не долго
имеет смысл, что бы система работала, только
когда вокруг есть локтоза. В норме есть регуляторный ген,
который работает всегда. Из него получается
РНК, из него получается белок. Который садится на оператор(находится рядом с промотером),
он туда садиться и эта зона становится как бы закоменченной. Она
не читается, потому что ничто не может связаться с промотером. Он
по суте выключает это место.

У нас появляется лактоза, Этот белок, который все комментирует,
он умеет узнавать либо последовательность букв, либо узнавать локтозу,
причем локтозу он узнает гораздо лучше, чем последовательность букв.
Если у него есть выбр, то он прицепится к локтозе. Поэтому, когда
появляется вокруглоктоза, она его оттаскиевает на себя. Снимается комментирование
приползает белок, который РНК полимираза. Читает всю это последовательность и
собираются все гены необходиммые для переваривания локтозы. и они начинают переваривать локтозу.  
    

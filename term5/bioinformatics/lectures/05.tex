\chapter{Молекулярная биология}
\section{Введение}
\subsection{Что такое биология? И зачем ее учить?}

\includegraphics[width=0.5\linewidth]{im1.png}

\begin{Def}
Биология "--- это наука о жизни и учить ее можно, потому что это любопытно.
\end{Def}

\textbf{Что такое живое?}\\

- Давайте последовательно. Сосед живой? Парта живая?\\

- А какой есть пограничный вариант? \\

- Вирус.\\ 

- Почему?\\ 

- У него нет клеток, и вне клеток он ведет себя как 
неживой. \\

- Но вас если в космос выкинуть вы тоже вести себя будет как 
неживое существо. \\

- Но я потом уже не стану живой. А его если потом
выкинуть в клетку, то он станет. \\

- В смысле это у него обратимое состояние. Но я
мог бы поспорить, сказав, что есть вирусы,
которые никогда не выходят за пределы клетки. И
если попробовать его вытащить за пределы клетки - он умрет.\\

- В общем, все организмы у нас в той или иной степени живые.\\ 
И восприятие вирусов как неживых связано, скорее, с некоторым артефактом
в биологии, о котором мы поговорим чуть дальше. \\

\subsection{Рекомендуемое чтение}
\includegraphics[width=0.25\linewidth]{im2.png}

Очень простая тоненькая книжка. Практически для детей. Начинается 
с химии. Даже химии знать не нужно. Чтение исключительно для 
удовольствия. 

\includegraphics[width=0.25\linewidth]{im3.png}

Книжка для задротов. Уровень биологии для поступления в аспирантуру.
Плюс учебника, что тут все с нуля. Там есть все. Там 1400 страниц
и это можно долго читать, очень хорошо написано с хорошими примерами.
Книжку можно скачать (этого не говорили).   

\subsection{Клеточная теория}

\includegraphics[width=0.25\linewidth]{im4.png}

\begin{description}
    \item[Артефакты:] Суть в том, что вирусы считаются якобы не совсем живыми, 
    потому что у них нет клетки. 

    \item[Открытие клетки, первый микроскоп]
    Клетка была открыта очень давно. И была открыта \textbf{Робертом
    Гуком} на микроскопе, который он сконструировал сам. Это не тот
    микроскоп, к которому мы привыкли. Тот микроскоп, который он использовал, вообще дикий.
    Поскольку таких глаз больше нет.
    
    Мощность линзы определяется кривизной. То есть, самая мощная линза будет шарик. 
    
    Такой линзой здоровый глаз пользоваться не может.
    Просто точка фокуса не будет где-то рядом. Но
    у людей есть всякие хитрые формы глаза. С
    близорукостью и еще чем-нибудь. И может получиться
    так, что этой линзой все-таки пользоваться можно.
    
    То есть, по сути, первый микроскоп это был
    просто зажим для кругленького шарика стекла.
    И если шарик практически упереть в глаз,
    то можно рассматривать что-нибудь, что находится
    с другой стороны шарика.

    \item[Собственно, почему обнаружили клетку:]
    
    Почему обнаружили клетку?
    Почему не обнаружено что-то другое?
    
    Клетки заметить легко. Почему? Потому что есть
    оболочка. Потому что он смотрел срезу
    коры пробкового дерева и там по сути самих живых
    клеток нет. Там просто
    клеточные стенки. Которые состоят у растений из целлюлозы.
    
    То есть он заметил просто мертвые останки клеток и
    оно обладало некоторой регулярной структурой. Именно просто как ячейки.
    
    Потом в микроскоп стали смотреть все, кому не лень и
    сформировалась некоторая клеточная теория.

    \item[Клеточная теория:]
    В примитивном виде состоит из трех позиций:
    \begin{enumerate}
    \item 
    Все живые организмы состоят из одной или больше клеток. 
    \item 
    Клетки - самые базовые структурные единицы всех живых организмов. 
    В ней происходит метаболизм, в ней происходит хранение генетической 
    информации. Передача и так далее. 
    \item 
    Все клетки произошли от предыдущих клеток. 
    \end{enumerate} 

    \item[Почему возникла теория?]
    Потому что смотрели на все, что есть вокруг. 
    Смотрели на срезы животных, растений, грибов и еще кого-нибудь и везде видели одно и то же. 

    \item[Ограничения] Откуда взялась первая клетка? Никто не знает. 
    Есть некоторые гипотезы происхождения живых организмов. 
    Но как это происходило на самом деле, в общем-то, никто не знает. 

\end{description}

\subsection{Вирусы}
\includegraphics[width=0.25\linewidth]{im5.png}

\begin{description}
\item[Описание:]
Маленькие инфекционные агенты. Живут и реплицируются только 
внутри клеток. И поражают, в общем-то, все известные живые 
организмы. Все организмы, включая вирусы. Есть вирусы, паразитирующие на вирусах.
Чаще всего вирус, который заражает один организм, не заражает другой. 

Вирус специально задизайнен так, что бы взломать некоторые молекулярные 
механизмы. Что бы клетка не восприняла сигналы и вирус был оставлен проигнорированным. 

%Вирус он специально задизайнен, что бы преписать, взломать набор 
%молекулярных механизмов, если этот набор отличается, даже если 
%он попадет внутрь, то его система не сработает. Клетка не воспримет 
%сигналы и просто проигнорирует вирус. 

\item[Почему их назвали инфекционными]
Заболевания растений были известны давно. 
Потом были открыты бактерии, микроорганизмы. И используя супер тонкие фильтры, 
размером в сотню нанометров, можно было это все отфильтровать. 

То есть берешь зараженный куст, который видно, что болеет бактериями. 
Перемалывается в труху, прогоняют через фильтр получается жидкость и если 
эту жидкость нанести на другие растения, то 
ничего не будет. Все бактерии окажутся на фильтре. 

И считалось, что этот фильтр пропускает только не органические вещи. А
потом нашли заболевание \textbf{табачная мозаика}. Прогоняют через фильтр, 
берут больной лист, наносят и он становится больным. И не могли понять, 
что за фигня, поскольку все хоть сколько-то известные крупные компоненты должны 
были остаться на фильтре. Поэтому появилась фигня называемая \textbf{инфекционностью}, 
из серии заражает, не смотря на фильтр. Потом когда появились 
электронные микроскопы, то стали брать срезы растений и 
смотреть прям отдельно клетку. И в ней видно, что 
происходит сборка некоторых частиц. Их не сразу связали с инфекционностью.
\end{description}

\section{Центральная догма}

\includegraphics[width=0.75\linewidth]{im20.png}

\subsection{ДНК}
\includegraphics[width=0.25\linewidth]{im6.png}
\includegraphics[width=0.35\linewidth]{im31.png}
\includegraphics[width=0.35\linewidth]{im32.png}

\begin{description}
\item[ДНК как кислота:]
Что такое ДНК? Расшифровывается как
\textbf{дезоксирибонуклеиновая кислота}. Или еще 
их исходно называли нуклеиновыми кислотами. Что означает, что 
это кислота? Это значит, что это отрицательно заряжено. У кислот есть 
OH группа, если есть OH группа, а это значит, что заряд будет отрицательный.

\item[Как это было открыто?] 
До понимания процесса генетики и так далее, человек брал клетки и доставал из
них ядра. Потом эти ядра он перемалывал в кучу и потом
смотрел из каких молекул они состоят. То есть пытался выяснить какие-то
базовые характеристики. Вот он и выяснил, что это кислота. То есть понятно,
что это \textbf{полимер}, то есть состоит из многих маленьких компонент.

\item[Что такое мономер и полимер:]
Есть два слова мономер и полимер. Полимер состоит из мономеров. 
Нуклеотид - это мономер, а все вместе - это полимер.

\item[Из чего состоит?]
Название дезоксирибонуклеиновая кисота. Риба - то есть там есть рибоза. 
Нуклеиновая - то есть там должны быть \textbf{нуклиатиды}. 

Там есть \textbf{сахарофосфатный остов}. Там есть радикалы, к которым крепятся 
нуклеотиды. Мономер включает в себя кусок рибозы. 

Нуклеотиды составляют наш алфавит из 4 букв, которые и кодируют информацию.

\item[Комплементарность]
ДНК является \textbf{двойной спиралью}. 

У нас в алфавите есть 4 буквы. Что мы о них знаем. Мы о них знаем одну важную вещь,
что они комплементарные друг другу. A - T, C - G.

\textbf{Аденин} будет стараться оказаться рядом с \textbf{Тимином} образовав две водородные
связи. \textbf{Гуанин} будет образовывать с \textbf{Цитозином} три водородные связи.

Почему это важно? Потому что вторая связь крепче чем первая.

В двухцепочечной молекуле ДНК на против A всегда будет T и
напротив C всегда будет G.


\item[Направление ДНК]

У ДНК есть направленность - \textbf{5' 3'}. Свойство, что
мы можем из одной цепочки восстановить вторую - называется - комплементарность.
Точнее обратно-комплементарная цепь. 

Почему 5' 3', потому что эта связь не симметричная. И фосфоры присоединяются в разных 
местах, поэтому цепь может удлиняться только в одну сторону. То есть новая буква 
может присоединяться только к 3' концу.

\item[Кодирование]
Из этих буковок состоит весь геном. И как мы недавно узнали в них
закодирована некоторая генетическая информация. Слово кодирование используется
не просто так. ДНК состоит из нуклеиновых кислот. Из чего состоят живые организмы?
Из клеток, клетки состоят из беков, жирков, углеводов и нуклеиновых кислот. Теперь вопрос,
каким образом из последовательности
нуклеиновых кислот можно собрать организм не только из них?

В ДНК информация хранится в линейной форме и последовательность сигналов
нуклеиновых кислот можно перевести в последовательность сигналов
аминокислот. После этого аминокислоты собираются уже не в линейную структуру,
делают они это уже достаточно хитро. И после этого белки собирают
уже все остальное, в том числе новые копии нуклеиновых кислот.

\end{description}  
\subsection{Генетический код}
\includegraphics[width=0.30\linewidth]{im7.png}
\includegraphics[width=0.50\linewidth]{im33.png}
\begin{description}
\item[Аминокислоты:]
Сколько существует аминокислот?
20?
Клевый вопрос, что бы всегда оказаться правым, потому что
аминокислот бесконечно много. Потому что аминокислоты - это
класс химических соединений. Их выдумывать сколько угодно.
А вот в живых организмах их действительно 20-21, смотря как 
считать. Никто же не мешает собрать свой живой организм, 
который будет состоять из аминокислот, которых больше нигде нет. 

\item[Вырожденность кода:]
То есть, у нас 64 кодона должны кодировать всего 20 аминокислот. 
То есть у нас какие-то кодоны либо ничего не кодируют, 
либо кодируют одно и то же. 

То, что они кодируют одно и то же называется вырожденностью. 
То что мы можем однозначно перевести из последовательности 
нуклеотидов в последовательность аминокислот, 
а вот если я вам дам последовательность аминокислот, 
то вы можете сочинить много 
последовательностей ДНК из которых это было получено. 

\item[Кодирующие части:]
Есть понятие \textbf{открытой рамки считывание}. Не весь 
геном, который есть внутри нас,  хоть что-то кодирует. 
Кодируют только его некоторый фрагмент, который 
находится между \textbf{сигналами}. Есть сигнал начала, 
есть сигнал конца. 

\item[Сигналы начала и конца:]
Что являются сигналами начала и конца? 
Сигнал начала, это вполне конкретный кодон. ATG, он 
кодирует метионин. Ген всегда начинается с 
ATG. 

А как природа научилась кодировать сигнал конца? Кодоны, 
которые кодируют конец они никаких 
аминокислот не кодируют. 

Бежит молекула и собирает аминокислоту. Потом она получает 
сигнал, которого как бы не существует. И какое-то время сидит ждет, 
поскольку она какое-то время вообще не получает ответа, 
комплекс разваливается и все готово. 

Стоп кодона у нас обычно три и при этом некоторые животные, 
в том числе мы можем читать стоп кодон как кодирующий сигнал. 

То есть это дает некоторое разнообразие в интерпретации информации. 
То есть мы можем одну и ту же последовательность читать по-разному. 

\item[Не универсаьность кода:]
Генетический код не универсален, как это не странно. В митохондриях
у нас другой генетический код, так как митохондрии давным-давно были 
паразитами симбионтами эукариотических клетках с тех 
пор несут некоторое бактериальное наследие. Генетический код 
у нас разный. GAG в одном организме у нас кодирует 
одну аминокислоту, в другом другую, а в другом вообще кодирует 
стоп кодон, то есть ничего не кодирует. То есть 
генетический код меняется, эволюционирует и 
так далее.

\end{description}

\subsection{РНК}
\includegraphics[width=0.6\linewidth]{im8.png}
\includegraphics[width=0.45\linewidth]{im34.png}

\begin{description}
\item[rRNA:]
Кто-то говорил про РНК и вот она. \textbf{rRNA - Рибосомальная РНК}. 
рРНК состоит из двух субъединиц большая и малая. 

\item[Классическая центральная догма:]
То что здесь протаскивается - это \textbf{матричная РНК}. Это 
цепочка РНК, на которой как раз закодировано что-то из 
ДНК. ДНК у нас превращается не напрямую в белок, 
а ДНК сначала превращается в РНК, 
а потом РНК превращается в белок. 

Есть ДНК, есть РНК, есть белки, когда молекулярная 
биология достигла уровня, где мы собрали много информации 
как и что происходит, люди заметили, что из 
ДНК получается РНК, из РНК получается белки и из 
ДНК должно получаться ДНК, а то не имеет смысла. 

И это считается классикой, это у нас 
происходит в организме постоянно. Стандартная схема, 
при деление клетки копируется ДНК, при работе клеток 
из ДНК читается РНК, которая потом превращается в белок. 

Какой функционал у РНК? Почему не синтезировать 
белки сразу на основе ДНК? Разница между ними минимальна. 
Считается, что жизнь возникла как РНК. 

\item[Разница между РНК и ДНК:]
ДНК хранить удобнее, дезоксирибонуклеиновая кислота, из 
названия понятно, что там не хватает окси группы. А это 
просто рибонуклеиновая кислота. И если рисовать РНК, то 
там еще торчит окси группа, которая уменьшает стабильность 
двойной цепочки. 

РНК, сама по себе, как молекула менее стабильна. Она быстрее 
разрушается и плохо хранится. Возникновение ДНК, что бы 
было удобно и много хранить. 

\item[Приемущества использования РНК перед ДНК:]
Ну это не объясняет, зачем нам использовать после этого РНК. В чем 
преимущества? 

Можно представить, что у нас есть большой корабль и надо его 
иногда чинить. У нас есть огромная книга со всем чертежами, 
но мы не будем ее каждый раз таскать, нам только нужны определённые странички. 
То есть РНК это просто кусочки ДНК. 

\item[Рассширенная центральная догма:]
Это все было бы хорошо, если бы у нас 
еще не было вариантов из РНК в ДНК и из РНК в РНК. Можем
еще заставить происходить из ДНК в белки, но в природе такого нет.

Есть вирусы, которые существуют только в форме РНК. Опять же,
когда мы говорим об РНК нужно понимать, что там есть еще + и -. РНК
обычно существует в форме одной цепочки, а не двух.

\item[Пространнственная структура rRNA и еще немного о rRNA]
Рибосомальная РНК кодируется в ДНК. У РНК есть
еще одна приятная штука, в отличие от ДНК. Они могут
принимать всякие разные вторичные/третичные пространственные структуры.
И ДНК не может ничего, кроме как хранить информацию, а РНК может
мало того, что хранить, так еще и самовоспроизводиться, например.

Рибосомальная РНК кроме того, как кодирует свою структуру, эта структура 
сворачивается в такие клубки, которые после этого умеют 
читать mRNA и синтезировать белки.

\item[mRNA]
mRNA просто чертеж, а rRNA эта штука, которая умеет читать этот чертеж и собирать
на его основе белок.

\item[tRNA]
tRNA - от слова \textbf{транспортная}, потому что она подтаскивает к rRNA аминокислоты.

На каждый тип кодона есть своя tRNA, которая приносит аминокислоты.
\end{description}

\TODO половину текста переместить в центральную догму? Переструктурировать часть про РНК.
\TODO нарисовать картинку для расширенной центральной догмы. 

\subsection{Трансляция РНК}
\includegraphics[width=0.5\linewidth]{im9.png}
\begin{description}
\item[Определение:]
    \textbf{Трансляция} - процесс преобразования из РНК в белок. Из 
    ДНК в ДНК - \textbf{репликация}, из ДНК в РНК - \textbf{транскрипция}.

\item[Рибосома:]
Мы говорим о переходе из РНК в белок. 
Вот у нас рибосома, две субъединицы, которая сидит на mRNA. 

Внутри рибосомы можно условно выделить три области.

Рибосома узнает некоторый триплет. На mRNA есть три буквы и транспортная РНК тоже имеет
три буквы, комплементарные этим трем. Поэтому происходит узнование.

\item[tRNA:]
Голубые кусочки - это транспортная РНК. 
Она плавает снаружи и к ней пришита аминокислота. 

У тРНК есть хвост, который просто тащит аминокислоты. Это просто транспорт. Этот
хвост принципиально узнает только одну конкретную аминокислоту. Сколько типов
транспортных РНК вообще бывает? 64. А сколько тоскать разных аминокислот - 20.
tRNA 64 разны вариантов, а с другой стороны только 20 аминокислот.

tRNA бывает в двух формах, активированная и нет. tRNA создает такую пространственную
структуру, что бы другие белки ее узнавали и цепляли нужную аминокислоту.

\item[Процесс склейки белка:]
mRNA не вся кодирует, у нее есть начало, которое ничего не кодирует. Садится рибосома и
начинает двигаться по mRNA. Находит старт сигнал ATG. 

Всего в рибосоме есть три места:
\begin{enumerate}
\item Место узнавание 
\item Место объединения 
\item выход
\end{enumerate} 

Она двигается линейно всегда в одну сторону.
Она не может двигаться в другую. Попала в это место, поняла, что 
это старт сигнал. После этого она будет ждать, пока 
сюда приплывет транспортное РНК, которое комплементарно этой последовательности. 
И tRNA принесет свою аминокислоту, шаг второй. Шаг равен у нее трем буквам. Теперь эта 
транспортная РНК переехало на вторую позицию и мы ждем, пока сюда придет следующая 
транспортная РНК. Теперь ситуация становится интереснее. Штука со второй позиции оказывается 
в последней, но сам процесс становится хитрее. Второй хвост пересаживается на аминокислоту. 
И при следующем шаге оказывается, что на второй позиции уже висят две бусинки, tRNA, которая 
передала свои аминокислоты вышла, а на первой позиции ждем tRNA.

\end{description}
\subsection{Ретровирусы}
\includegraphics[width=0.25\linewidth]{im21.png}
Про реторо вирусы мы тоже уже говорили, это вирусы, которые 
умеют встраиваться в геном. 

Которые поломали мозг немного людям, догма считала работающий и 
нет ничего возможного, что бы из РНК получалось ДНК. 
Почему это было важным? Для вас может казаться и причем тут это. 
Помните всю эту фигню о том,что наследуется, а что нет и так далее. 

Приобретенные свойства не наследуются. ДНК это генетика - РНК и белок - это 
реализация. Которое уже происходит после наследования. И в общем считалось, что 
это достается от родителей, это реализуется. В общем, осальное может как-то меняться, 
но в исходный геном это изменение не вносится. А вот оказывается нефига, 
мы можем переписывать РНК в ДНК. 

И получается, что организм накапливает какое-то количество свойств, 
которое потом берется и передается потомству. От части это так и происходят. 


Пример ретровируса - вирус имуннодиффецита, вич инфекция. Есть несколько 
случаев излечения, правда достаточно экзотических. У человека мало того что
ВИЧ инфекция, у него еще и лейкоз. Лейкоз лечится пересадкой костного мозга.
Для этого убивается весь костный мозг в человеке,
после чего ему пересаживается от донора. У некоторых
людей бывает частичный иммунитет. Врач нашел донара, у которого была такая
мутация. И посадил ему костный мозг устойчивый к ВИЧ инфекции. Новые клетки
оказались с имуннитетом к ВИЧ инфекции и вот он вылечился. Но процедура
пересадки костного мозга это не кислый риск для жизни, и это не лечение.

Больше шансов дожить до старости с ВИЧ инфекцией, чем пройти через подобную процедуру.
Не говоря уже о том, что на всех таких доноров не хватит.

Почему она считается неизличимой? Он встраивается в твои клетки, в том
числе в клетки костного мозга. Можно уничтажить вирусы, которые плавают в организме,
можно уничтожать вирусы, которые заразили клетки, но все клетки нельзя уничтожить,
если не уничтожить систему органов. Потому что вирус стал частью твоего генома и
клетки при деление будут нести его. Он может в ней лежать не работающим,
а потом заработать. Огромное количество иследований направлено на то,
как именно искать эти резервы, в которых он прячится.

У ВИЧ инфеированных рождаются здоровые дети. Если ретровирус не попал в яйцикетки и сперматозоиды.
если вирус встраивается в половые клетки, то ребенок сразу окажится носителем этих
вирусов.

Насколько это ситуация уникальна, с вирусом имнодифецита этого еще не случилось. Но
в процессе эволции это случилось очень много раз. У нас 10 \%  генома 
состоит из эндоретровирусов. Это когда наши очень далекие предки еще до млекопетающих 
сталкивались с эпидемией на подобе ВИЧ инфекции, которая по какой-то причине либо затухала, 
либо вырабатывался иммунитет, либо еще что-то, а у выживших 
оставались в геноме следы. 

Насоклько это плохо?
Опять же с точки зрения выживания конкретного вида. Конкртено 
наша эвалюция очень зависима от этих вирусов, например гены, которые 
отвечают за развитие мышечной системы и всего остального мы утащили у вирусов. 
Точнее вирусы сначала в наш геном встроились, а потом организм придумал как их можно 
использовать для себя. Гены отвечающие за формирвание плаценты тоже 
утащились у вирусов. Куча генов с развитием нервной системы, 
с развитием имуунитета. 

Для эндоретровирусов хорошо то, что им уже бесполезно искать внешнего хозяина. 
Паразиты распространняются вместе с основным видом. 

Для носителя хорошо, что он получает некоторый набор генов,
который он по другому получить не мог. 

Что такое сами по себе мышицы? Это образование многих 
клеток вместе. Мышицы это не одна клетка, мышечное 
волокно образовано путем слияния многих клеток поменьше 
и образуется здоровая клетка мышечного волокна. Синцитий. 
Синцитий - это изобретение вирусов, потому что они, зная, 
что выходить из клеток опасно, они придумали делать 
дырку в одной клетке и переходить в соседню. То есть путешествовать только 
по клеткам. Они так объединяют много клеток. 


\subsection{Метабалические пути}
\includegraphics[width=0.25\linewidth]{im22.png}

Дальше есть примеры метабалических путей. Общая структура, когда 
есть белок 1, белок 2, белок 3, при этом белок 3 он подавляет синтез белка 1. 
То есть есть обратная связь, то есть негативная. Есть система, причем работающая, 
где связь наоборот позитивная. Где есть один белок и 
тогда один белок начинает увеличивать работу этой системы. Куча петель, наложенных 
друг на друго, объединенных вместе. Образуют огромную метаболическую сеть. Отдельные клетки 
по всему организму. Если ее просто нарисовать, она наверное займет всю стену. Самое 
неприятное, что из нее понимаешь, что клетка на любой раздражитель она может выдать 
любой ответ в зависимости от настроек внутренних переключателей. И это не удобно. 

\subsection{Репликация ДНК}
\includegraphics[width=0.25\linewidth]{im23.png}

Подробнее про репликацию ДНК. Эта схема, которая имеет к реальности 
отдаленное отношение, каким образом происходит синтез второй цепочки ДНК на 
основе шаблона. 

Есть специальный белок под названием геликазы, который разрывает водородные связи. 
Делает так, что две цепочки расплетаются, на одинарные. Есть набор 
белков, которые называются ССБ, связываются с одноцепочесной последовательностью ДНК
не давая ей снова слипнуться. То есть просто ее держат в расплетенном состояние. 
Дальше есть штука под название полимераза.  Она раскручивают. Если лента изначально была 
скручина и ее разрезать на два куска, то она все еще будет скручена. Тут они снижают это 
напряжение, что бы не скручивалась, для этого они делают разрез, дают 
цепочке несколько раз прокрутиться и сшивают обратно. И они 
умеют наоборот накрутить. 

Есть штука под названием РНК праймер и РНК праймер синтезируется с помощью
ДНК проймазы. ДНК проймаза это класс ферментов близкородственный к ДНК 
полимиразам. ДНК полимираза занимается тем, что на базе 
цепочки синтизирует ей комплиментарную. Но у нее есть фундаментальные ограничения. 
Молекула которая умеет синтизировать цепь на основание первой умеет садиться 
только на двух цепочечную ДНК. То есть что бы ДНК полимираза начала работать 
ей нужна затравка. 

Исходно есть одна цепочка, есть фермент, который занимается тем, что выстраивает 
комплементарную цепь на основание этой цеопчки. Он не может 
сесть на одноцепочечную. Есть другой элемент ДНК проймазы, котрый 
синтезирует эту цепочку. Минус этого 
фермента в том, что он буквы скорее угадывает, а не ставит. Делается 
затравка, которая не имеет никакого отношение, что там реально находится. 
Далее туда садится клевый чувак, синтезирует дальше. А этот хвост потом вырезается и он 
будет достроенн потом. Другой чувак придет, заполнит дырку. 

И не только в этом беда. Одна идет в направление 5' 3' которая называется лидирующая цепь и 
в ней проблем нет. Синтизируется одна затравка и после этого 
фермент бежит. Но есть так называемая отстающая цепь, у нее направление в другую сторону. 

По ней просто так не побежишь, там еще не расплелось, куда бежать можно. Бжик построил, 
ждет пока расплилось - еще построил. Цепь называется отстающей и так и работает. 
Работает фермент легаза, она сшивает фрагменты. Есть много мелких фрагментов, они 
сшиваются вместе. Получается тоже самое, но медленнее. 

Когда цеопчка расплетенная у нее гораздо больше шансов мутации букв. Буквы могут время от времни 
менять на другие буквы из-за какой-то химии. 

У бактерии, места, в которых начинается репликация или удвоения ДНК - это 
фиксированные места. И мы можем лидирующую цепь отличить от отстающей, 
так как там больше ошибок. 

\subsection{Мутация}
\includegraphics[width=0.25\linewidth]{im24.png}

Мутации возникают постоянно и не смотря на то, что 
полимираза что-то умеет исправлять и все остально умеет исправлять
шанс ошибки полимиразы $1/10^9$. У если вспомнить 
эндроретровирусы, у них обратная полимираза, 
которая из РНК делает ДНК. Не смотря на то, что она 
это делает, она делает это не точно. Шанс ошибки $1/10^4$. 
Получается, что при размере вируса в 10 000, каждая вирусная 
частица отличается от другой в среднем на одну букву. При 
условии, что у нас таких частиц очень много, миллиарды и миллиарды. 

Насколько в среднем мы отличаемся от родителей. Сколько у
нас других букв в геноме относительно родителей. Мы сами по себе 
получаем не совсем то, из чего родители состоят, мы являемся уникаьной комбинацией. 
Не просто сумма. Есть еще набор дополнителььных изменений. 200 букв в общем, 
по всему геному. Мутации бывают разные. 
Большая часть из них совершенно безвредна. Потому что они скорее всего 
не попадут в гены, которые занимают небольшую часть генома. Есть мутации, 
которая поменяет цвет зрачков, вся генетика изучалась на мухаха. Здесь 
у нас два грудных сегмента. 

Есть мухи с нагами вместо глаз, 
есть мутанты, наполовину мужчины, наполовину женщины. 

Женский организм химеры. Все девушки полосатые или питнистые. Из-за того, 
что в женсоком организме две X хромосомы, организм одну из них выключает. Для 
работы нужна только одна. Выбирает случайным образом. Формируется от предков у которых 
выключилась либо одна либо другая. Получаются полосы или пятна. 
Есть некоторое генетическое состояние, когда можно заметить по пенгеманации кожи. 
Еще завбавно, что эти хромосомы передаются потомству и там 
надо эти блоки снять. Что бы обе могли работаь. И этот 
процесс не стль эффективный, то есть блоки снимаются, но не полностью. 
И если мальчику достается хромосома, которая не очень хорошо включилась, 
у него выше риск развития аутизма. 

Поэтому у девушки аутизм встречается редко. 

Те же самые мутации связаны с устойчивостью к вирусам. У нас в любой момент 
времени существует вирус устойчивый ко всему. Мы только надеямся, что мы никогда не встретимся. 

\section{Прокариоты} 
\includegraphics[width=0.25\linewidth]{im10.png}
\subsection{Домены}
Как живые организмы вообще бывают? 
Есть множества просто живые, какие бывают самые крупные подмножества 
этого множетсва. 

Бактерии, археи и эукориоты. 

Археи вообще не очень давно известны, бактреии известну уже 
достаточно давно и они все отличатся от эукориот. На уровне 
гентеического сходнстав 5\% сходства и 95 \% различий относительно каждой из этих 
групп. При этом бактерии и археи внешни очень похожи. И их 
долгое время ситали одним и тем же, до тех пор, 
пока не заглянули внутрь. А работают они совершенно подругому. 

И их вместе называют прокариоты. 

Начнем с них, поскольку ее условно можно назвать проще. Эукороты 
можно перевести как истиноядерные, а прокориоты - это доядерные. 

\subsection{Строение клетки}

\subsubsection{нуклиус}
У них нету ядра, у них есть некотрый нуклиус, просто вся ДНК
клетки плавает более менее в одном месте более менее организовано.

\subsubsection{мембрана}
Тут есть клеточная стенка, есть плазмотическая мембрана. Плазматическая мембрана
это просто то, что отделяет клетку, от окружающей среды. Живые организмы стараются
быть некоторым образом обособленны. И плазматическая мембрана это и делает.
                                                      
Из чего она состоит? Там белепидный слой... Какое основное
свойство должно быть у плазмотической мембраны, что бы выполнять роль обособления
от окружающего мира. Она не должна быть растворима в воде,
при условии, сколько вокруг воды и что все химические реакции идут в воде. Основное
свойство мембраны - не растворимсть в воде. То есть на нужно одну воду отделить от другой.
То есть должна быть гиброфобна. Это похорошему, хорошо структурированная, хорошо 
организованная жировая пленка. 

\subsubsection{клеточная стенка}
За прочность отвечает как раз слой под мемдранной, под названием клеточная стенка. Которая 
дает жесткую прочность. Бактерии как раз очень жесткая, ограничение объема из-за 
одной структуры, а другая дает жесткость каркаса. Зачем ей быть жесткой? Вокруг у нас
вода. Есть такая вещь, как астматическое давление. Берем мемдрану, которая 
частично проницайма для воды. И с одной стороны мы кладем много соли или 
сахора. Соль и сахар мембрана не пропускает. Вода будет стремиться попасть туда, 
где ее концентрация меньше. Так как переехать она не может, 
то соль будет оказывать давление на мембрану. Ей нужна крепкая стенка, 
потому что бактерия хочет много всего хранить. Ей нужно хранить много углеводов, 
много солей и всего остального. Если бактерию лишить этой стенки, то она лопается. 
Если кровь человека поместить в водно-солевой раствор с низкой концентрацией, кровенные 
клетки полоппуются. Потому что внутри концентрация всяких няштиков больше, чем снаружи и
они будут стремиться вылезти наружу. 

\subsubsection{способ получения энергии}
Спомощью мембраны бактериальная клетка вырабатывает энергию. Есть
специальные белковые комплексы под названием АТФ синтазы. АТФ - это
универсальный источник энергии для клетки. Есть белковый компелкс,
который умеет из подручных средств сентезировать молекулы АТФ.
Молекулы АТФ удобны тем, что их легко транспортировать в разные места.
Они универсальный источник энергии во всех клетках.

АТФ = аденазин три фосфат.

Родственница молекуле аденина. аденазин + фосфар + фосфар + фосфар. И именно
энергия связий этих трех фосфоров является запасанием энергии.

То есть штука работает следуюшим образом, она берет аденазинмонофосфат или 
аденазиндифосфат и пришивает к ним новые фосфары, которые можно будет оторвать
и получить новую энергию.

Но от куда эта штука берет энергию. А она создает разность потанцеалов. С одной
стороны она создает минус, снаружи положительный и мембрана сома по себе не
проницаема. Протоны начинают стримиться попасть внутрь клетки. Для них есть
для этого открытый канал и пока они движутся через него, они крутят что-то типа мельницы.

Двигаясь через АТФ синтазу они разгоняют ее и она синтезирует АТФ.

\subsubsection{Борьба организма с бактериями}
И как наш организм любит бороться с бактериями? В бактериях протыкается
дырка и тогда заряд двигается куда угодно. Но бактерии это умеют затыкаьть.

Есть круче, когда клетка знает, что надо бороться с бактериями, чаще всего клетка это
знает, если она попыталась бактерию съесть и у нее это не получилась, она сопративляется.
Она закисляет среду снаружи. Тогда снаружи получается тоже минус, заряд
уравновесился, бактерия не может синтизировать энергию.

Это происходит только в том случае, если клетка умеет реально быстро закислять. Иначе бактерия
начнет более активно создавать разность потянциалов.

\subsubsection{Метохондрия как бактерия}
Мы получаем энергию ровно таким же образом.

Метохондрии - это от всей бактариальной клетки осталась по сути
мембрана, напичканная АТФ синтазами. То есть метахондрии -
это такие заводы по производству АТФ, которые являются далекими
потомками бактерий.

Прикол в том, что эукориоты - это результат
симбиоза кого-то и бактерий.

Хлоропласты, например, это симбиатические водрасли. Тоже
результат симбиоза.   

\subsection{Прокариатический геном}
\includegraphics[width=0.25\linewidth]{im11.png}

Прокариатический геном. Он кольцивой. Обычно 
делится на хромасому и плазмиды. Хромосомы бывает размером 
5миллионов букв, бывает сильно меньше, 
бывает сильно больше. 

Есть плазмиды, чем отличается от хромосомы - никто толком не знает, 
раньше отличались, потому что они просто считались разнми вещами.
Они отличаются по размеру, плазмиды меньше и плазмиды не 
обязательны. Бактерия может терять плазмиды, бактерия любит 
обмениваться плазмидами. Новые плазмиды делать. 

Обязательность и необязательность плазмиды сильно зависит от услловий.
Например плазмида, которая дает
устойчивость к антибиотикам. Она реаьна полезна только
если антибиотик есть и бесполезна если антибиотика нет,
так она только энергию тратить.

Поэтому нужно говорить не обящательна в каком случае.

В чем разница между ДНК и хромосомой?
Хромосома это ДНК связанная с кромотимой. ДНК
просто так не хранится. Например,
как это происходит у нас. Есть бусинки,
которые можно назвать гистонами и ДНК просто на них
наматывается. Потом бусинки можно перемешивать,
собирать вместе и так далее. ДНК поскольку
просто не существует в чистом виде, она
всегда связана с некоторыми белками.

Хромосом может быть несколько.   

\subsection{ГМО}
\includegraphics[width=0.25\linewidth]{im12.png}

ГМО - генетически модифицированный организм. Изобрела природа. 
Первые ГМО стали делать бактерии, милионнов 60 лет назад. 

Есть бактерия, которая называется агробактерия. Живет внутри 
растений определнного вида. Тех же 
саммых бобовых. И они модифицируют им геном таким образом, 
то есть при заражение расстения они делают вокруг 
себя опухоль. Им там комфортно живется. Растене
не то что сильно против, так как пока оно
там живет оно снабжает
растение дополнительными азотистыми соединениями. Растение
предоставляет бактерии питани, то есть
у них в какой-то степени симбиоз. Но вопрос, как это
делает бактерия. У бактерии есть
такая штука под названием T-плазмида, на T-плазмиде есть
T регион. Это кусок плазмиды, которую бактерия
встраивает в геном растения. И люди научились это
достаточно давно и достаточно клево эксплатировать. То есть
если заменить этот регион на те гены, которые нам нужны, то бактерия просто возьмет
и воткнет этот кусок в растение. Таким образом,
сейчас есть методы, когда практически в домашних условиях
можно сделать растение, которое хочешь.

Всего лишь надо самому собрать себе плазмиду.

Сейчас есть технологии под названием криспы, дает
возможность обращаться с геномом на уровне текстового редактора.
Удалать букву, добавить букву, вырезать это,
вставить это.

Самое простое что делается - делается
светящиеся растение. Можно например сделать
какой-то белок, что бы он был связан со светящемся
компонентом. Например, хотим
ответить на вопрос, в каких местах
растения у нас работает такой-то белок.


\subsection{Структура прокариатического генома}   
\includegraphics[width=0.25\linewidth]{im13.png}

Зеленое, это то про что 
мы уже говорили. Это кодирующая часть. Здесь 
находится старт кодон и стоп кодон. 
Это ДНК. RBS - место посадки рибосомы. Рибосома 
ищет, где находится первый старт кодон. Это она здесь по 
сути и делает. Есть места под
названием 5' UTR, 3' UTR. Это не транслируемая часть. 
Теперь здесь появился кекс - называется промотер. Эта 
важная штука, поскольку так мы понимаем какой ген взять. 
У нас есть система сигналов. Есть много разных 
промотеров. Это как знак припенание, который говорит - вот 
здесь ген. Или указатель. Если нужно, что бы 
что-то заработало ищется конкретный промотер, так
как рядом с этим промотером есть гены. Промотер 
всегда находится на вполне фиксированном расстояние от начала гена. 

У промотеров есть определенный мотив. То есть определенная последовательность. 

Есть такая вещь как транскрипционные фаторы, которые как раз и запускают 
процесс транскрипции, то есть превращение ДНК в РНК. И эти транскрипционные 
факторы узнают эти мотивы. Если есть транскрипционный фактор и
есть сочетние, то они находят друг друга и это является сигналом для многих
разных других штук,
которые плавают вокруг. Что надо прийти и сделать на этом месте РНК.

И это достаточно простая структура гена.  

\subsection{Пример регуляции}
\includegraphics[width=0.25\linewidth]{im14.png}

Локтозный оперон. У бактерий гены организованны не просто, 
бывает конечно и так, что у нас есть промотер и на один ген есть
один промотер, но бактерии стремяться улучшить эффективность работы.

Есть штука под названием оперон, которая объеднияет в себе много
генов. Локтозный оперон занимается тем, что производит гены необходимые
для переваривания локтозы. Там несколько белков,
такие, что когда они есть бактерия
умеет усваивать локтозу.

У нас в геному есть регуляторный ген. Который в ситуации, когда у нас
локтозы нету... Бактерии нет смысла производить все это, если
она не может получить энергию. Поскольку белки живут не долго
имеет смысл, что бы система работала, только
когда вокруг есть локтоза. В норме есть регуляторный ген,
который работает всегда. Из него получается
РНК, из него получается белок. Который садится на оператор(находится рядом с промотером),
он туда садиться и эта зона становится как бы закоменченной. Она
не читается, потому что ничто не может связаться с промотером. Он
по суте выключает это место.

У нас появляется лактоза, Этот белок, который все комментирует,
он умеет узнавать либо последовательность букв, либо узнавать локтозу,
причем локтозу он узнает гораздо лучше, чем последовательность букв.
Если у него есть выбр, то он прицепится к локтозе. Поэтому, когда
появляется вокруглоктоза, она его оттаскиевает на себя. Снимается комментирование
приползает белок, который РНК полимираза. Читает всю это последовательность и
собираются все гены необходиммые для переваривания локтозы. и они начинают переваривать локтозу.  
    

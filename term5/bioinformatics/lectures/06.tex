\section{Эукариоты}

\subsection{Строение клетки}
\includegraphics[width=0.25\linewidth]{im16.png}
\subsubsection{Что вообще есть внутри прокариот}
Что у эукориотов есть такого, чего не было у прокориотов. 

У прокориотов была одна мембранная структура, которая отделает клетку от 
всего остального. 

Здесь есть внешняя мембрана, есть аппарат Гольджи, в котором много чего происходит, 
но в том числе синтизируется мембрана, от него отщипляются пузырики и присоединяются
к мембране. Эндоплазматическая сеть - тоже мембрана, Ядро отделено тоже мембраной, 
метохондрия тоже с мембраной, каждый из пузырьков тоже с мембранной. То есть 
у нас очень много мембранных структур. 

Они олицитворяют собой принцип комплитаризации, что в процессе эвалюции все 
стремится к некоторой специализации и так далее. Вместо того, 
что у нас одна клетка, которая делает все подряд, у нас клетка делится 
на части. 

\subsubsection{транспорт}
Например, что позволяют делать малкие пузырьки, они позволяют
заниматься мелким активным транспортом. То есть то, 
что сентезируется в эндоплазматической ретикуле может 
целеноправленно от него отделиться и не 
засчет действия теплотического движения, а оно перенесется 
в аппарат Гольджи, там это опять модифицируют и тот же самый 
компонент например возьмет и отправится, например, к поверхности клетки. 

То есть клетка может менять свою структуру ассиметрично. Она может 
с одной стороны увеличиваться, с другой уменьшаться. 

\subsubsection{разница размера у прокариот и эукориот}
Каким образом вообще одноклеточная клетка движится? 
Она делает части своей клетки более текучими, а
другие более твердыми. Тогда одно течет, а
другое стоит на месте, потом меняются местами. 

Эффективный транспорт позваляет снять некоторые ограничения на размер. 
Потому что у прокориотов. Почему мы не видим бактерии размером с арбуз? 

Даже самые крупные бактерии они очень маленькие. 

Эффективность внутренней работы ограничена, ограничена эффективностью 
транспорта засчет внутренней диффузии. 

Если радиус придет к увелечению на x, то объем на $O(x^3)$ небольшое 
увелечение линейного размера, 
приведет к большому увелечению объема. Это очень затруднит 
транспорт, поскольку вещества не просто как-то 
по клетке должны быть разбросаны, а попадать 
в конктертные места. 

Поэтому бактерий больших нет, зато есть большие эукориоты. 

\subsubsection{Пример больших одноклеточных}
Самые большие одноклеточные организмы какие бывают? 
Можно конечно сказать, что яйцо - это 
одна клетка. Гигантские аксоны разных 
кальмаров. Которые достигают размеров десятков 
сантиметров, метров. Ну это тоже некоторый чит, 
это не то что бы клетка, у нее просто огромные отростки. 
А бывают прямо клетки. Какого размера анебы? Сумое
большое яйцо, кстати, у акулы. Потому что они просто
здоровые, это конечно называется не яйцо,
а икринка, но по факту, в общем-то, тоже самое.
Есть организмы, которые называются, плазмодий, наверное.
Их долго не воспринимали как что-то свое, думали что
это какие-то варианты грибов. То есть думали, что
это какая-та своя безумная дурь. Они в одну из
стадий жизненного цикла очень напоминающая амебу фиговина,
которая ползает, ищет, что бы пожрать. И она в общем-то
не ползает, она растет. В лабораторных условиях
ее могут сделать квадратный метр или больше. Будет такая
ровная поверхность состоящая из одной живой клетки.
Почему ее любят? Потому что это одноклеточный механизм,
который способен решать задачу с лабиринтом. Если
ему дать лабиринт с несколькими путями, то
он из него выберет самый оптимальный путь.
Берут лабиринт, внутри есть много путей и есть
вход и выход. На вход и на выход кладут еду.
Изначально он занимает весь лабиринт, поскольку
ему пофиг, а когда у него
появляеются источники пищи, то он старается их объединить
наиболее эффективными магистралями. Потому что он будет пересылать
вещества с одного места на другое. Все остальное тело он
просто уничтажает и остается одна
дорожка, которая соответствует самому короткому пути.
Самый известный эксперемент, когда взяли карту соединенных 
щтатов с японеней и разложили еду по городам и чем крупнее 
город, тем больше кусок еды. И то, что построил слизнявик 
соответствовали дорогам. То есть железным дорогам, 
автомагистралям и так далее. 

То есть у него внутри сложный цитоскилет благодаря которому 
он организует пересылку вещества. 

Есть гипотизы, что он может просто маркировать дорожки, из 
сери он пускает груз по несколким путям и просто смотрит, 
где доставка заняла больше времени. То что оказалось медленее остальных, 
ту дорожку он разбирает. 

\subsubsection{Ядро}
Теперь давай-те поговорим немного о ядре. В ядре находится 
генетический матерьял - хромосомы. На что 
похожа его оболочка? Можно сказать, что 
на прокориота, но это будет только 
половина праильного ответа. Наиболее правильный ответ, 
что она похожа на архея. Организация оболка ядра напоминает 
архея, но есть предположение, 
что генетический матерьял нам достался от какого-то предка, 
который был именно археем и завел себе каких-то других симбионтов. 

Чем удобно отделение генетического матерьяла от отсальной клетки? 
Потому что удобнее регулировать Есть некоторые степени регуляции. 
Есть процессы репликаия, трансляция и транскрипция. Репликация происходит в ядре, 
транскрипция тоже в ядре, а вот трансляция уже происходит вне ядра. Когда 
из одной библиотеки все уже считывается на короткие фрагменты mRNA, эти 
короткие фрагменты отправляются наружу. Если на них есть сигнал. На самом 
деле когда он считывается на него вешается метка - куда отправлять. То есть 
что-то отправляется сразу, что-то заберживается. Снаружу сидят рибосомы, 
которые ждут, что бы сделать из 
mRNA белок. Так делается не только mRNA, у нас еще есть rRNA, tRNA и 
много других видов РНК, они все делаются тоже в ядре. Там есть 
специальная штука, называется, кажется, ядрышковой организатор, 
прикол в том, что это просто фабрика по сбору рибосом. Которые 
опять же собираются в ядре, потом отправляются наружу. Отправляются 
не просто, там есть ядерные поры, через которые проходит транспорт, 
причем проходит в обе стороны, как вы можете догадаться. Ресурсы нужно как и доставлять 
в ядро так и доставать от туда наружу. Принцип такой же, вешаются 
сигналы и если нужно доставить из ядра наружу, так и 
сигналы, что что-то нужно доставить в ядро. Другие механизмы уже доставляют их в ядро. 

\subsubsection{Как с этим борятся вирусы?}
Теперь вопрос как с этим выкручиваются вирусы. Потому что здесь 
есть момет такой, клетка разрабатывает механизм, как
отражать вирусную атаку, а вирус разрабатывает 
механизм, подражанию клеточны сигналов, ему нужно 
мало того, что проникнуть через клеточную оболочку, ему нужно 
еще проникнуть через ядерную оболочку. Есть вирусы, которые
живут в цитоплазме, любому вирусу надо рано или 
поздно сделать мРНК, что бы из этого 
сделать белки, а что бы мРНК пошло оборабатываться нужно подделать 
много подписей. И клекта умеет обрабатывать некоторые очевидные вещи. 
Например двух цепочесных РНК не бывает, и это означает присутствие 
вируса. Некоторые вирусы сузествуют в виде просто двух цепочек РНК. 

Если клетка встречает такую штуку внутри себя, то она 
либо пытается ее разобрать максимально быстро, либо она перегрвает 
реактор в виде метохондрий, устраивает кислородный взрыв и уничтажает себя и 
все что вокруг. Лучше уничтожить небольшую часть и не дать вирусу распространиться. 
НА самом деле вирусы и с этим умеют справляться, если 
вирусы успевают перехватить сигналы и успевает эти системы отключить, 
тогда клетка уже ничего сделать не сможет. 

Метохондрии есть во всех клетках, не бывает, что бы 
был хлоропласт и не было бы метохондрий. Бывают растения без хлоропластов, 
растения паразиты, которые живут внутри других растений. Растение одновременно и 
производит кислород и одновременно дышит. Причем, фотосинтезирует оно только на 
свету, а дышит оно все время. Растение производит кислорода больше, чем потребляет. 

Есть рибосомы приклееные к мембране, это нужно, что иногде надо, 
что белок синтезировался встроенный в мембрану. Потом уже этот 
кусок мембраны уже дальше ползет. В лизосомах 
у нас производится переваривание в аппарате Гольджи производится 
метаболизм жиров. 
  
\subsection{ДНК в хромосому}
\includegraphics[width=0.25\linewidth]{im17.png}

Интересный факт, так как размер эукориотического генома многократно 
превышает размер прокариот. У бактерий как правило 5 миллионов, у человека
3 миллиарда. И так как каждая хромосома в двух видах, то 
6 миллиардах. Размер генома может 
быть достаточно разным, мы хорошо знаем размер генома позвоночных. 
У некоторых рыб, с беспозвоноными труднее, с насекомыми совсем трудно. 
Но есть и генномы поменьше, у тех же саммых дрожжей геном всего 
16 миллионов. То есть как три средних бактерии. Но в среднем, 
геном эукориот намного больше. И именно поэтому считается, 
что генном эукориот обладает куда более сложной структурой в плане орагнизации. 

Вот есть наша стандратная цепочка, прямая цепь, обратная цепь и так далее. Из-за
того, что она обладает некоторыми физическими свойствами, она начинает, опять же 
это из-за того, что связь ассиметричная, она укладывается в виде спирали. 
А дальше эта спираль начинает закручиваться в еще более клевую спираль. Которая 
называется супер спираль. Можно представить, как если нитку или веревку скручивать, 
то она сначала просто закрутиться, после этого на ней начнут образовываться 
еще узелки скручивания. У нас образуется супер спираль, 
которая потом наматывается на гистоны. 

Гистоны, это что-то вроде таких бусин, 
которые состоят из белков, положительно заряжены, 
поэтому на них легко наматываться ДНК, которая 
отрицательно заряжена, потому что кислота. Наматываются 
в среднем на два с чем то оборота, 2.7 примерно. И дальше организовано 
в виде супер спирали намотанной на бусины. Эти бусины потом начинают складываться еще 
в более крупную спираль. Которая уже не супер спераль, а не важно. Потом эта огромная 
спираль, которая начинает образовывать эти 
петливые домены. Когда, пригрошни этой веровочки начинают сцепляться. И образуются 
такие типа звездочек. 

А дальше это упаковывается в нечто такое, что мы видим и что привычно называть 
хромосомой. 

Внимание, вопрос, всегда ли есть хромосомы? 
Нет. Потому что хромосомы нужны только при деление. 
Что бы было удобно переместить. Вообще, вся эта упаковка нужна 
для того, потому что в распетенном состояние ДНК занимает огромное 
место и работать с ней не удобно. Поэтому есть несклько 
этапов компатизации направленных на то, 
что бы ужать ее до очень маленького размера. И это 
опять механизм игнорирование участков. То что намотано на бусины - не читает. 
То что совсем хорошо упаковано, там совсем доступа нет. Дальше, как какое-то 
время назад выяснили, организм умеет регулировать заряд на 
этих бусинах. Эта бусина это набор белков, 
в которой торчат эти длинные хвосты и клетка может на эти хвосты вешать 
разные радикалы. Молекулы с определенными зарядами. Если она станет 
вешать на них плюсы, он тоже будет увеличиваться, 
а если заряд гистона становится еще более положительным, то она будет 
плотнее примотана туда и станет менее доступной для белков снаружи. 
А если клетка будет навешивать отрицательно 
заряженные группы, то заряд становится все 
менее положительным и начнет подрасплитаться. То есть 
клетка умеет регулировать плотность намотки. 
На самом деле она не только это умеет регулировать, 
потому что там есть группы, которые не несут заряда, но 
которые просто модифицируют пространнственную 
структуру, что бы туда никто подползти не мог. 


У прокориот ничего подобного вообще нет. У них есть какие-то варианты 
компатизации, но они даже близко к этому не подступаются, 
потому что у них в этом нет необходимости. 

Если говорить о сложности компатизации генома, то да, 
можно считать, что у эукориот больше. 

\subsection{Структура эукориатического гена}
\includegraphics[width=0.25\linewidth]{im18.png}
И тут начинается ужасная выборня. Потому что, 
что бы найти все гены в прокариоте нужно было 
найти только рамки считывания. Потому что 
рамки считывание не перекрываются. Потому что 
у бактерий не бывает что одна рамка находится внутри другой. 
Рамка считывания между старт кодоном и стоп кодоном и кратна трем. 
И если мы найдем все рамки, то мы  найдем все гены и найдем кучу всего, 
чего не существует. То есть так мы найдем все гены и 
часть не генов. То есть если мы не гены собираемя убрать, то 
на этом все закончится. 

Эукориоты и у них есть такая штука под названием интроны и экзоны. 
Инторны это типа спайсиров. И нафига нужны интроны? Когда 
секвенировали геном человека люди ставили деньги на то, 
сколько в итоге будет генов. Исходная оценка была что 
где-то окала 100 000. Оценка была сделана методом 
взять белки из клетки человека и посмотреть на их 
разнобразие. А выяснилось, что 
голых кодирующих генов всего 
20 000. Но при этом они производят 
огромное количество белков, намного больше чем 100 000. 
И как они это делают? 

\subsection{Сплайсинг}
\includegraphics[width=0.25\linewidth]{im19.png}

Есть такая вещь как сплайсинг и альтернативный сплайсинг. 
Сплайсинг - это процесс вырезания инторонв. И альтернативный 
сплайсинг это неменого другой. Слово происходит от слова сращивания. 
Это у моряков есть сращивания канатов. У нас есть последовательность, 
которая состоит из инторонов и экзонов. 

Формируется последовательность из [1][2][3][4][5] экзонов, которые вместе сшились. 

Алтернативный сплайсинг, это когда мы можем, например, получить 
последовательность [1][4][5]. А можем 
получить [1][2][5] или сколько-то из начала выкинуть. Это 
можно описать, как выреания кусочка. Чего никогда не происходит. 
Мы не можем перемешивать порядок экзонов. 

Мы привыкли, что открытая рамка считывания кратна трем, поэтому ее легко искать. 
ATG кратное трем количесто букв и стоп кодон. 

Тут рамка считывания состоящая из финальных экзонов тоже будет 
кратна трем. Но отдельные экзоны не обязательно кратны трем. 
Может быть экзон 2, может быть экзон 4. Вместе будут 
кратны 6, но поотдельности таковыми не являются. И 
интороны могут быть разной длины. Никакой рамки считывания 
там нету, длина может достигать десятков, в то и многих десятков. 
Инторны участвует в альтернативном сплайсинге, между ними 
находятся переключатели меняя их клетка определяет 
какой именно продукт будем от сюда делать. 

Инторны нужны для процесса сплайсинга, инторны нужны для регуляции. 
И это еще не все, в инторнах могут быть другие гены. 

То есть может быть структура, когда ген состоит из интонов и экзонов и внутри интронов 
могут быть другие гены. 

Понятие инторон оно относительно какого-то конкретного гена. В интороне 
могут находится кодирующие части но другого гена, не того же. 

Как вы думаете каким образом ищут ген? 

В бактериях можно было искать открытые рамки считывания. Есть 
методы основанные на похожих последовательностях. 

Марковские цепи описываются матрицей перехода, которые как раз описывают 
вероятность перехода с одного состояния в другое. В примитичном случае 
мы можем построить матрицу перехода, если возьмем все известные кодирующие гены и 
строим матрицу перехода. 
И как мы ее считаем? В скольких случаях у нас после A идет A. После A T и так
далее. Далее некоторые значения мы заполняем. Такую же 
матрицу мы можем построить для последовательностей, которые не кодируют ничего. 
И после этого мы можем взять последовательность 
букв, предположительно ген, и спросить какая 
вероятность того что он получин из первой 
матрицы или из второй. Почему это все не работает. 
Эта марковская цепь называется первого порядка. 
Следующее состояние зависит только от однйо буквы. Может быть 
марковская цепь более высоких порядков, 
когда у нас скажем буква A после 5-6 других букв и 
тогда матрица раздувается в размерах 
многократно. Не понятно, есть ли 
марковские цепи, которые включают в себя больше примерно 20 букв, 
потому что они просто не влазят в память. Хотя чем большего 
порядка марковская цепь, тем 
более точно она описывает все. 

Есть процесс созревания РНК и не зрелое РНК содержит и интроны и экзоны. 
В процессе созревания 
вырезаются интроны, навешиваются сигналы. У исходной mRNA нету ни 
сигналов, что она произведена самой клеткой, 
не сигналы, что ее можно пускать дальше в работу. Потому что 
после того, как от сюда вырезаются всякие экзоны, с
одной стороны вешается цепочка A хвост, это просто 
длинная цепочка A букв, с другой стороны вешается нуклеотид, 
который называется capирование от слова шапка. То есть 
там буква присоединяется бочком. Сигнал, что от 
сюда начинать читать. 

У нас тогда ищутся не какие-то гены, а некоторая 
усредненная модель гена. Модель конечно более сложная, 
там есть еще переходы из инторонов в экзон и из 
инторонов в экзон. Если в модели получилось, что 
в среднем 6-7 экзонов. То мы ровно такие гены находить и будем. 

Пусть вы программа, вы нашли последовательность, которая 
напоминает кодирующую часть. Она состоит из неокторого 
количества экзонов. А дельше вы знаете, что 
в среднем где-то в конкретном месте должен быть старт кодон и стоп
кодон, вы знаете, что размер интрона может быть 
любым, вам сместить немного рамку считывания что бы 
получить сигнал старта и сигнал стопа не проблема. 

В чем минус этих программ, они ищут ген в среднем, с длинными 
интроннами, он его поломает на несколько частей. Если гены 
стоят плотнинько к друг другу, причем разные гены, он 
их слипнит в один длинный ген. Потому что предпалагает, что они слишком близко. 

Задача поиска генов до сих пор не очень решенная. Есть смельчаки, 
которые пытаются прикрутить глубокое машинное обучение, но 
там свои трудности, так как там надо данные в очень специфической форме 
предоставлять. Которые еще не очень понятно как сделать. 

\section{Генетика}
\includegraphics[width=0.25\linewidth]{im25.png}

Что такое генетика. 

Когда люди стали заниматься гентикой? 
Что такое наследственность?

Когда человек задумался что можно что-то выводить? Почему 
это не было гентикой. Почему генетику связывают с эксперементами Грегоря Менделя? 

У него была статистика. Считается что появления классической науки, это
когда люди задумличь, что надо считать. 

Считается что он получил основу для генетики, как науки о наследственности, 
но к тому времени уже были одомашленные собаки, растения и все остальное. 

Генетики изучают наследственности. Механизм связанный с передачей некоторой информации
от себя к потомкам. Что передается, что нет.
 
\includegraphics[width=0.25\linewidth]{im26.png}

Мы знаем, что участвуют две клетки, сперматозоиды от мужской, 
яйциклетки от женсокой, которые вместе образуют зиготу, которая потом 
делится делится и получается чеовек. 


\subsection{Митоз и Мейоз}
\includegraphics[width=0.25\linewidth]{im27.png}

У нас есть митоз и мейоз. Митоз самое простое. Это 
деление клетки попалам и каждое клетка 
отражение родителя. У бактерий так происходит. 

С мейозом все труднее. Это два последовательных деления и для вас, 
что бы сделать жизнь проще. Задача из генома формы 2n 
сделать четыри клетки и получается генотип 1n. Это
важно, что бы в половых клетках было половина, что
бы при слияние было 2n.

В этом году было много арбузов без косточек.
Как они получается? Арбузы с генотипами 2n и 4n от
них получаются половые клетки n и 2n. А теперь
делают гибрид. У него есть 3n и это число не делися по палам.
Мейоз фейлится на этапе раздвоение клетки.   


\subsection{Рекомбинация}
\includegraphics[width=0.25\linewidth]{im28.png}
Но самое главное здесь это кроссенговер или рекомбинация. 

Организм получает две цепочки. Одна досталась от одного родителя,
другая от другого. Но они передаются не в неизменном виде. Если бы 
в неизменном, то одна бы цепочка досталась от папы, другая от мамы. 
Дело в том, что в потомков будут переданны премешанные фрагменты от 
этих. Передается потомству уникальная смесь того, что передалось от родителей. 
Рекомбинация увеличивает разнобразие. 

\includegraphics[width=0.25\linewidth]{im29.png}
Штуки связанные с рекомбинацией. Шанс того, что произойдет 
рекомбинация между двумя фрагментами увеличивается с увеличением расстояния между ними. 

Увеличивается вероятноть того, что они окажутся на разных хромосомах. 

Это позволило людям оценивать расстояние между генетическими маркерами до того, 
как узнали о ДНК, до того, как узнали, что такое гены.

Строились фенотипические карты. Увелечение шанса перестройки на один процент это около двух милиионов букв. 

\includegraphics[width=0.25\linewidth]{im30.png}

Если мы построим график координаты, и частота, сколько раз произошла рекомбинация, 
процесс не является равномерно распределнной, есть точки, в котрых рекомбинация происходит 
чаще, чем в других. В среднем размер блока 5-6000, гаплоблоки, блоки, которые не рекомбинируют. 

Из этого интересное следствие, что весь геном состоит из фрагментиков, 
которые нам достались, по сути от наших очень древних предков. 

Нам досточно знать маркеры, что бы различить фрагменты. И по этим маркерам 
можно реконструировать блоки. 


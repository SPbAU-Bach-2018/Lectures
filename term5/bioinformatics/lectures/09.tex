\section{Устройство хромосом}

Что такое геном? 

Набор хромосом. Место где хранится наследственная информация. 

Что такое хромосома? 

Можно сказать что хромосомы это ассоциация гистонов с ДНК. 
Гистоны это бусинки, на которые наматываются. Гистоны образуют комплексы. 
Там 4 разных белка слепляются вместе. И там не бусина, там скорее барабан. 

И еще есть гистоны, которые придерживают ДНК, что бы она там сидела. 

Такое определение - это хомосомы в широком смысле. 

В узком смысле хромосомы существуют крайне небольшое время только 
в метофазе при деление. Пенсионеры называют только то, что 
видно в микроском при деление. Все остально время хромосом нет, 
потому что они сильно размотаны и их увидеть нельзя. 

Какие хромосомы бывают? 

У кого хромосомы бывают? 
У эукориотов. А у прокариотов? Там кружочки. ТАм какие-то 
неправильные хромосомы. Плазмиды. 

Если отбрасываем понятие что они не связаны с белками. То есть
хромосомы это просто что-то более менее стабильное находящиеся в
геноме.

Для нас это достаточно. Но не дай бог сказать это на экамене по
генетике. На обычном экзамене по генетике нужно знать,
что есть опредление одно, есть определение второе. Что
хромосомы это только то что сплетено.

А это бактериальная хромосома.

Сколько у нас бывают хромосом? Какие есть границы.

Поставим ограничение, что число хромосом неотрицательное целое.

Может быть 1 хромосома не бактериальная?

Половые клетки говорят несут половину генома. Самое маленькое количетсво 
хромосом, которое известно на текущий момент 
это у каких-то никому неведомых муравей и у них 2 хромосомы. 1 хромосмоу
никто не встречал. 

Это теоретическое ограничение? 

Если размножение половое, то нужно всегда объединять. Поскольку у нас 
сливаются две половые клетки, то у нас (будет ли всегда? отсылко на гибридов, которые не умеют 
размножаться). Но тут идет речь просто про нормальную особь, которая размножается например делением 
и с одной хромосомой. Таких пока не встречали. 

А максимальное количество хромосом какое? 
Больше 100. Как таковых в природе хромосом не 
существует. Хромосомы придумали люди. 

Есть вроде бактериальные хромосомы, которые вроде не удовлетворяют определению 
хромосом эукориотических. Эукориатические есть микрохромосомы у птиц, 
например. Они не удовлетворяют 
определению ни тех, ни других. Кроме этого есть 
еще B-хромосмоы. Которые тоже выпадают из трех предыдущих 
множест и есть хромосомы, которые называются политенные хромосомы. 

Политенные хромосомы в микроскоп увидеть очень просто. Для этого 
нужно поймать личинку комара, достать из нее слюнные железы, 
раздавать под микроскопом. Станартная практика у биологов на белом море. 

Суть в том что в клетках слюнных желез. Из-за
того что им нужно производить много всяких компонент, 
Одна и таже хромосома получается в капийности 
сотни и тысячи, а дальше по какой-то причине они 
начинают сцепляться вместе, в стопку. От того, что 
их становится такой гигантский прут, можно увидеть 
хромосому в любую фазу клеточного цикла, 
а не только в мейоз.

Теперь есть еще и микрохромосомы. Микрохромосомы пока известны что встречаются 
у птиц и возможно у некоторых видов... Кто у нас ближе всего к птицам? 
Крокодилы. У них есть микрохромосомы. Это супер мелкие хромосмоы 
с очень большой плотность генов. На них нету центромер. И у них 
теломеры совсем. Другие. 

Если кто-то будет спрашивать, какие обязательные структуры у хромосом. Нужно 
сказать, что обязательно должна быть центромера и обязательно 
должна быть теломера. 

И то и другое состоит из тандемных повторов. В этом 
они похоже. Тандемные повторы разные. 

Мы сейчас говорили о деление. У нас фазы деления. 
0) у клетки есть набор чек поинтов,
которые она должна пройти, что бы выйти на деление.
Она должна подтвердить, что
генетический матерьял не поврежден,
что у нее достаточное количетсов ресурсов.
что бы успешно завершить деление. Что
нету никаких повреждений. И что бы она не застряла где-то на
пол пути.
1) Должно увеличится количество хромосом. То есть на
каждую хромосому нужно сделать пару. Реплекация ДНК.
2) Почему хромосомы упаковываются в такую супер компактную структуру?
Почему не может сидеть большой клубок, как обычно.
Нужно разделелить два набора. Два клубка
разделить сложно. А если каждую хромомосму свернуть
до уровня короткой ружинки, то
их разделить уже значительно проще.
В этом суть.
3) Как их разделить? Их нужно схватить и оттащить
как-то. Там специальный молекулярный механизм образуется.
Там будут такие тяжи, которые будут на себя тащить.

Суть в том, что эти микротрубочки крепятся к
центромерам. В хромосоме есть
специальная область, что бы за нее можно
было схавиться и оттащить в сторону.
Для этого и нужны центромеры.

У теломер функция значительно менее легко объяснимая.
По факту это счетчик количества делений.
При каждом раунде реплекации ДНК теломера становится
короче. От нее откучывается кусок. Когда
кусков не оставется ДНК не может поделиться с этой копии.
То есть каждый кусок можно
реплецировать не больше скольких-то раз.

Тогда почему мы не вымерли?
От куда этот кусок нашли? Помните овечку Долли, которая прожила как-то
сильно меньше, чем ожидалось. Суть в том, что взяли клетку,
которая уже была старая. Клетки на поверхности из
которых мы состоим, они проши уже много делений.

Если ее взять и заставить делить. Организм состоит из $10^14$ клеток.
А теперь мы говорим клетке сделать еще раз столько же
делений. Там мало останется дальше жить.

А почему она все-таки не кончается. Потому что есть специальный
фермент, называется теломераза, и он умеет удлинять теломеру и занимается
тем, что достраивает эти куски.

Если ты будешь очень часто делиться, то это плохо,
потому что будут накапливаться ошибки. Дальше вопрос,
а как же быть. А это необходимость.

Организ крайне таталитарная штука, любая попытка клтеки
выйти из под контроля карается, обычно,
уничтожением клетки. В том числе встроен механизм, что бы
отдельно взятая клетка не смогла ничего делать.

Теломеразы удлиняют теломеры и работают только в эмбриональных
тканях, крайне не большое время. На этапе развития зародыша.
В стволовых клетках немножко теломеры работают.
И это в общем то все. Еще теломеры офигенно круто работают в
раковых клетках. Потому что первое,
что учится делать раковая клетка, когда становится раковой -
ей нужно выйти из подконтроля ограничения теломер. Она
начинает гены теломеразные активно включать,
потому что ей нужно много делиться. В том числе
ограничение количества деления - это
защита от раковых клеток.

Если клетка станет раковый, но не снимет контроль, то она просто
поделится сколко-то раз и умрет.
После этого появилась идея ну все, мы бессмертные. В чем проблема.
Когда попытались сделать на мышах - это не очень получилось.
У них стал чаще рак возникать. Но попытки этого продолжаются.
Есть тетка, которая вколола препорат на удленение теломер и меостатин.
Кто-то видел супер мускулистых собак, или коров... У нас регуляция
функции роста мышочной ткани исходно подавлена. Рост мышиц это
крайне энергозатратно. Это куча ресурсов. Поэтому организм не будет
их тратить не в коем случае, если ты вот прям не умераешь. Если ты 
хочешь гору мышиц. То трудится надо будет. Если поламать меостатин, 
то мышицы будут рости так, что никакому качку и не снилось. 
Есть мутантные мыши. Есть породы коров, которые просто огромная куча мяса. 

Если будет много мышиц, то ты будешь медленнее двигаться и тратить еще и 
кучу энергии на поддержание. И она вколола спеуциальный
ген, который подавляет меостатин, что бы в старосте стемулировать
процесс развития мышиц. Все кроме некоторых инвесторов считают, что
это развод.

И таких препоратов еще нет. К вечной жизни мы еще не готовы.

Количестово микрохромосом бывает разным.

Есть еще неведомая фигня под названием B-хромосомы. С ними еще
все печальнее. Они встречаются радом с основными хромосомами.
Поэтому основные иногда называются A. Вот есть основная большая
хромосома и рядм с ней таким гроздями висят B хромосомы. Они очень мелкого
размера. Есть ли на них гены большой вопрос. Кто-то находит, кто-то нет. От куда
они беруться - тоже большой вопрос. Есть наблюдение что в
стрессовых ситуациях количество B хромосом у многих
видов увеличивается. Есть данные, что B хромосом не может
быть супер много, поскльку это все ломает. Почему это все происходит и
зачем они нужны сказать трудно. Есть мнение, что это просто ошибки репликации
нормальных хромосом. Попытка что-то сделать, когда в начали что-то
поломалось. но при этом она осталась висеть рядом. Но есть люди, которые
считают, что у B-хромосом есть какая-та функция.

Предстаьте, что мы сделали гибрид, но не арбуз
без семечек, а с каим-то другим свойством и нам пришлось сделать
арбуз с 5n хромосомами и после этого он не может делиться. Но такие
проблемы решают тем, что объединяют его с самими собой. И ему просто
удваивают количетсов хромосом, если он выживает после этого.

Пастидные геномы. Метохондриальные геном.
Метохондрии 0 производит энергию в форме АТФ. А кто
такие пластиды? Метохондрии - это в пршлом бактерии.
Хлоропласты - это в прошлом водросли. То есть
какие-то из клеток научились жить в симбиозе с
хлоропластом. Слово геном чаще всего подразумевает, что
это ядерный геном. Но сильно зависит от контекста.
Есть ядро. он делится при деление клетки. Плазмиды живут 
немного своей жизнью. Они живут внутри клеток, они тоже 
умеют делиться. Но плазмиды не самоятоятельные. Нельзя 
заставить их делиться. если рядом не будет здорового ядра. 
Внутри метохондрии порядка 20 000 букв, там 
находится порядка 15 генов. При этом большая часть из 
них - это транспортная РНК. Там немного белок кодирующих генов. 
Все гены которые находятся в метохондриальном геноме 
относятся к процессам связаным с синтезем энергии. 
То есть сама метохондрия не хранит ничего что бы контролировать 
свою струкуту, целостность и деление. 

В метохондрии есть кольцевая модекула ДНК. и Большая часть для поддержки 
метохондрии находится в ядре. И мы уже даже 
знаем от какой бактерии произошли метохондрии. 

Хлоропласты произошли много раз. Поскольку есть хлоропласты разных типов. 
Которые произошли от разных водрослей. Метохонлрии произошли один раз. У всех из 
них один предок. 

Почему гены исчезли из метохондрии? 
По логике, метохондрии было бы удобно быть автономность. Метохондрии 
в целом уже не надо быть автономной и в ядре все лежит более надежно. 
Производство АТФ связано с транспортом электронов. У нас 
образуется активные формы кислорода. Метахондрии
адски опасное место с точки зрения химического. Там 
повредается все очень быстро и часто. Есть гипотиза, что 
гены жизненно необходимые для функционирования метохондрий именно 
по этому были перенесены в ядро, что бы они не были 
поврждены. Точнее лучше сказать, что те, 
у кого они перенеслись выжили лучше чем те, 
у кого они остались. Собственно метохондрия это действительно 
реактор. 

Макрофаги, которые ползуют все и жрут. Это 
клетки именной системы. Эта огромная омеба. Она
поедает все подряд и бывает 
она съедает бактерию. И насколько бактерия рада? 
И она сопротивляется. Бактерия быстрее достраивает себя, 
чем ее переваривают. Что бы она перестала делать идут 
дополнительные системы вооружений. Есть белки 
дифенцыны. Они что-то типа штопора и он 
умеет вкручиваться в мембрану. В бактерии 
образуется дырка, 
сквозь которую начинают лететь неконтролируемый транспорт 
заряженных частиц. Мы используем метохондрии что бы синтизировать 
энергию, а бактарии используют свою поверхность. 

Клетка макрофаг пытается с этим справиться наделов в ней дырок. Но 
с этим не всегда получается справиться. Они иногда все рвно работают быстрее. 

Тогда макрофаги могут сцепиться вместе и попытаться переварить активнее. 
Если в этом случае не получается, то они совершают массвое самоубийство. 
Под  названием пентозномонофосфатный шунт. Метохондрии начинают работать 
так, что бы специально проиизвести огромное количество активного кислорода. 
И они унивтожают и себя и все что вокруг. Пример этого гной. Гной - это наши 
мертвые имунные клетки. 

Проблема в том, что это тоже не всегда помогает, но это 
пример того, как можно использовать метохондрию. Метохондрия 
это не только иногда источник энергии, но иногда и 
активное вооружение. 

Метохондрия живет своей жизнью так же как и хлоропласты. Метохондрии 
просто распределены более менее по всей клетке и при деление 
в среднем по полам делятся. 

Для разных типов клеток разное количество метохондрий является стандартом. 
Где думате будет много метохонлрий. Мышицы. 

Нейроны - это адский поразит. И вокруг него куча клеток, которые снабжаюют его едой. 

При пищиворение образуются питательные с одной стороны и надо их 
переехать в другую сторону. Трансопрт и мышочная активность. 

Можно ли заставить мозг потредлять больше энергии. 
На сколько процентов работает мозг. Как вы мериете работу мозгу? Что такое 
5\%. 

Измерять активность мозга придумали, когда придумали первые МРТ измеряют 
электрическую активность мозга. Смотрять, сколько работает нейронов в двнный момент. 
Известно какой заряд генерирует один нейрон. Сколько потанциална мозга в целом, сколько
нейронов суммарно. Это примерно тоже самое как считать, какой 
процент пианино работает. Пианино работающее на 100\% Мозг всегда потребляет константу. 
Если вы думаете, что вы потратели много времени решая задачу и потратили от этого 
больше энергии - это фигня. Это попытка самооправдать конфету. По 
факту ничего не изменилось. Он все время на определенном уровне нагрузки. 

Во сне может чуть ниже, но там у всего занижен уровень энергии. 

Геномика изучает геном.  

Генетика изучает наследственность.

Человеческий геном. Есть 22 аутосомы, которые есть у всех. Две половые хромосомы XY.
Наличие Y хромосомы определяет мужской пол. Потому что на ней находится
специальный ген. даже несколько генов. Половина генома - это 3 миллиарда букв.

У бактерии 5 миллионов и држжей 20 миллионов и табока 2 миллиарда.

У человека копии хромосом очень похожи. Есть животные у которых копии
могут очень отличаться и это дополнительные..

От 30 до 40 тысяч генов. Что они делают, где находятся?
Сколько исходно предпологалось количество генов. Их
предпологалось гораздо больше, потому что они количество белков
оказалось 100 тысяч...

Там то было 20 тысяч, почему тут 30-40. Там были белок-кодирующие
гены, а это попытка определить все гены. Разброс в 10 тысяч говорит
о размытости определения генов. Только 3\% генома отвечает
за белки. Она может называться еще КДС. Еще
есть от 40-50\% состоит из повторов мобильных
элементов генетических.

Есть ретровирусы. Они появились не сразу. Раньше 
был организм т него клетка, 
от него еще клетка и геном потихоньку меняется. Когда 
появились ретровирусы. Они заражают клетку и встраиваются в ее геном. 

Появилось петля из РНК в ДНК. Проблема того же ВИЧ, он встраивается 
в геном человека и от туда его невозможно достать. И поскольку это нормальный 
вирус,он начинает генерировать свои новые копии. Которые будут снова встраиваться. 
И они могут встроиться туда же. 

Есть несколько семействов мобильных повторв. Одни называются ретро транспозоны, 
другие просто транспозоны. Транспозоны произошли от 
ретро транспозонов. Они потеряли возможность . 

Одни ДНК -> РНК -> встроцка в ДНК. 
А другие просто из ДНК снова в ДНК. 

Первые это copy-past, а второе cut-past. 
Вторые умеют двигаться по геному, но он перед этим вырезается. И половина
нашего генома состоит вот из этого. Она может перемешиваться. 

Что делать оставшиеся 50\%? Наличие локтозного опероно в геноме
бактерии гарантирует, что бактерия способна переваривать 
локтозу. Наличие каких либо оперонов 
связанных с другими функциями гарантируют 
наличие других функций. 

Можно описать поведение бактерии просто 
зная из каких бактреий она состоит. 

Даже если эти гены немного отличаются друг от друга. 
У нас будет новая бактерия, котрую никто никогда не встречал, 
но можно на нее посотрет

Если сранить мышь и слона. Количетсво генов у них примерно одинаоковое и 
гены отвечают за примерно одинаоквые функции. Есть наборы генов, 
которые делают тоже самое. 

Что делают 50\%. Там наборы переключателей, причем 
сложные наборы переключателей. Переключатели переключателей. 
У эукориот есть промотреы есть инхансеры есть инсуляторы. 
Это просто регуляторы более высокого уровня. Промоторы регулируют один ген, 
который в свою очередь может регулировать других инхансеров промотеров и 
инсуляторов. Инхансеры регулируют целые блоки генома. А инсуляторы ограничивают регуляцию 
инхансеров. Что бы влияние инхансеров не распростронялось слишком далеко.

Есть огромное количество переключателей с помощью которых ты можешь как угодно
задействовать систему.

Гены позвоночных известны нам почти все, но мы не можем
с помощью них описать животное.

Ген может быть в состояниях как минимум работает или нет.

Эти 50\%. Мы о переключатялях мало чего знаем. Мы
плохо поимаем как они выглядят и как большая часть из них
работает.

Генетика - иследует наследуемые фенотипы.

Что такое фенотопиы? Генотип - набор генов. 
Почему так много терминов, которые говорят об одном и том же? 
Суть в том, что оно развивалось задом наперед. Мы о существование 
геномов узнали после существование генов. О сущестование генов 
узнали после того как о сущетсование наследуемых признаков. 

Функция хромомсом была неизвестно в момент открытия наследуемых 
признаков. 

Фентопи- результат взаимодействия генотипа с окружающей средой. 

Есть два близница. Если посадить их жить в разных 
условиях - они будут разные. Но при этом с одним генотипом, 
но фентоп разный. Это называется 
генетическая пластичность. 

Норма реакции - это диапозон распределения какого-то признака 
с заданым генотпом. Возьмем дерево, нарвать веток и 
растить новое дерево в разных услвиях. 

А потом мерить размер листьев. И будут листки как супер мелкие 
так и супер большие. Чаще всего распределение получается 
где-то околонормальное. Это и называется норма реакция. Гентоип один, 
а фентоип разный. 

Здесь не все так просто. Мы говорим, что фентоип - это что-то наблюдаемое. 
Но мы говорим, есть признаки летальные. Есть 
варианты генотипа, которые мы не наблюдаем, потому что 
такие варианты генотипа приводят к смерти и есть условно летальные. 

Условно леталльные - он летальные в определенной среде. Самый 
прикольный пример. Есть мухи, дрозофилы, у которых например 
при температуре 36 градусов, может чуть выше они умерают. 

Если температура ниже, то они живут нормально. 

Биохимия - занимается иследованием химии взаимодействием белков. 

Есть транскримптомика - 
эпигентека, об этом будем говорить дальше. 

Поиск функции генов. 

Что в остальных 50?  

Исходно гены скорее определялись по функциям мутантов.
Представте, что есть нормальные мухи. Потом мы их облучаем
радиацией и получаем мух без крыльев. Тогда
мы скажем окей, у этих мух
мутировал ген крыльев.

По факту мы не знеам это был один ген или несколько. Но это
стабильно из поколения в поколения. Мы их можем скрещивать.
Расщпление по мендалю.

Но мы говорим ни сколько о гене, сколько о признаке. И именно от сюда у нас была функция.
То есть лишняя лапа появилась, исчезли глаза и так далее.

От сюда было много заблуждений, что можно человеку посадить ген паука и станет человек паук.
На самом деле регуляция намного сложнее. Один ген здесь ничего не сдлеает тут
нужно встроить всю огромную систему и не один ген, а 10тки генов.

Поэтому со временем пришлось переопределить понятия гена. То есть ген кодирует
белок или какой либо другой продукт. Они могут регулировать экмпресию -
если это РНК и они могут использоваться в реплекации или еще чем-нибудь(тоже РНК).

У нас есть 3 миллиарда с лишним букв, часть из них
повторы, окей. Остальное не повторы. Из них у нас
есть два множества. Есть просто не повторы, а есть гены.

Повторы тоже делятся на некоторые семейства. В некодирующие
части входят интроны. Регуляторные элементы - это
промотеры, инхансеры. Есть псевдогены. Они тоже
входят в некодирующую часть ДНК. Межгенные пространства -
этим словом определено все то, что мы не знаем. И есть
интерс... - это распределенные повторы. Это мобиьные элементы.

Есть оценки что мы называем высоко, средни и однокопийными элементами.

У нас есть высоко повторенное ДНК, количество копий больше 100 тысяч. 
Есть среднеповторенные 10-10 000 и однокопийные 1-2. 

А что в интервалах 2-10 до 10 000 100 000. Это такие, 
выбранные из головы числа. Между 10 000 и 100 000 находится что-то, 
что является повторами, но между среднем и высокоповторенным. 

Вот внедрился ретровирус. Изночально у него копийность 
будет 1. Прошла 1000 лет его количество копий увеличилось. 
Некоторые расстения состоят из мобильных элементов, которые 
являются паразитарными состоят на 99.9 \%. В геноме архедеи 
уже почи ничего не осталось. 

Будем считать подстроки. Любой новый кмер будет унекален. Потом 
какой-то кусок взял и удвоился. В этом случае некоторые kмер стало 
в два раза больше. 

Можно с некоторым опасением говорить, что белок-кодирующие 
части повторены от 1 до 20 раз, но это не всегда так. 
РНК гены часто относится к среднеповторенным. 

Были древнее эксперементы  когда мы можем оценить количество 
повторов в геноме. И это еще до того, как
научились секвенировать. 

Можно ДНК нагреть, она же двух цепочечная. И она распадется на 
две цепочке. При температуре 95 \% то получается две цепочки. 

Потом мы понижаем температуру и ДНК начинает слепляться. 
И если повторов в ДНК нету, то она начинает слепляться очень быстро. 
Почему? Потому что достаточно найти любое количество совпадений. Она 
как застежка склеиться и противоречий у нее не возникнет. Хуже с повторами. 

Чем больше повторов, тем больше шанс, что она образует оптимальное сцепление. 
И ее будет долго колбасить. Чем больше повторов, 
тем медленнее будет происходить реассоциация. 

Можно ли сделать ДНК одноцепочечной? Ей на то что бы правильно 
склеиться - нужна энергия. Если ее мгновенно охладить. Например, 
поместить в житкий азот. Тогда ей не хватит 
энергии на то, что бы образовать связи. 

Цитогенеические карты. 
Этот рисунок может отличаться в зависимости от того, 
чем именно красили. Здесь ничем не красили. Есть области 
генома которые всегда супер упакованы. Это темные 
области, они почти всегда выключены. Светлые иногда включаются. 
Тут подкрашены тандемные повторы на каждой области. Центромеры и 
теломеры. Центромеры в центре, но не 
всегда. 

Перетяжка - это центрамера. Именно за счет структуры тандемных повторов хромосома в этом 
месте упаковывается плотнее. Есть штука под названием вторичные перетяжки. 

Вторичные перетяжки образуются в областе богатой рибосомных генов. Когда 
клубок расплетается последовательность, 
которые отвечают за рибосомные кластеры оказываются все вместе. 
И они образуют отдельную структуру, которая называется малый ядрышковый организатор. Это 
такой завод внутри ядра, который занимается сборкой рибосом. 

Здесь хромосомы отсортированы по размеру 
и так и пронумерованы. Есть другая классификация - 
по положению центромеры. Есть центромеры расположенные в центре - 
это называется метоцентрические хромосомы. Есть 
перетяжки по хвостам. Это называется акроцентрические хромосомы. 
Есть хромосомы ни туда ни туда - это называется субметоцентрические 
хромосомы. 

Откуда у нас беруться разные хромосомы? Почему положение центромеры такое 
разное? 
Центромеры это повторы. Причем повторы достаточно уневерсальные. У всех 
хромомсом примерно одни и теже хромосомы. Очень часто хромосомы ломаются 
по центромере. Два - слепаются по центромере. Самые частые перестройки которые будут это
метоцентрическая хромосома может разломиться на две акоцентрические. 

И будет нормальны. Различие между нами и шимпанзы, у нас на одну хромосому меньше
Потому что две хромосомы слепились вместе. Там есть 19a и 19b. Мелкие 
перестройки между хромосомами случаются постоянно. Это к трудности 
создания фиксированной карты. 

Мобильные элементы. 

Самое простое - это штука, которая называется 
ретровирусная подобная. Называется автономной. 
Средний размер от 6 до 11 тысяч букв. Средний 
размер ретровирусного генома 8 тыс. У ВИЧ 
инфекции 10 тыс. Это ровно последовательность 
вируса встроилась . 

Три гена, которые есть всегда. Endf - штука, 
которая отвечает за оболочку вируса. То что формирует ему структуру. 
Pol - это ген, который отвечает за копирование вируса. GAG - отвечает 
за регуляцию и созревание двух предыдущих. После того, 
как вирус встроился в геном. Если он потеряет Endf - он не 
сможет выйти, точнее не сможет сделать свою 
оболочку, что бы выйти из клетки. Что происходит если он 
ломает Pol, то он не копируется. Но все это не верно. 
Потому что большенство подобных элементов ломается постоянно. 

Постоянно перемещаются и ломаются. многие из них не рабочие и 
многие адские древние обломки от того, что когда-то перемещалось. 

Там есть сигнал, что ее нужно реплецировать, есть 
сигнал, который узнает белок что это конец. Пока есть 
сигнал - приди полимираза и перетащи меня он будет 
работать. Потому что есть другие ретровирусные элементы, 
у которых это все работает. Он будет просто брать их фермент. 

Также Endf. Вирус в процессе созревания. ТАм достаточно интересные штуки. 
У него белки, которые самооргниазуются в икосаидоры. Он собирает 
такую капсулу. А дальше у капсулы есть несколько входов и 
этот вход узнает определенный мотив РНК. Если он узнал, то 
он просто внутрь себя засасывает и он там остается. 

Если эта последовательность больше ничего не кодирует уже не будет никого волновать. 
Он просто возьмет и упакоет. 
Но он не сможет дальше ничего сделать. Он сможет встроиться в геном этой 
клетки. Просто сам он не сможет реплецировать. Но кто-то случайно может его прочитать. 

Есть ген кодирующий оболочку, а есть последовательность 
отвечающийся за засасывание. 

Какие есть гиппотиты. A-B-C. Гипотиты A - кишечные инфекции. B - днк вирус. C - РНК ретро вирус. 

Есть еще формы гиппотита D-E-F и так далее. И другие формы гиппотита, которые могут заразиться 
и болеть. Но они сами не умеют синтизировать ни оболочку ни все остальное. По факту если есть 
гиппотит C и какой-то гиппотит из этих паразитов, то тяжесть протикания гиппатита C падает. 
Поскольку есть другой вирус, который паразитирует на этом вирусе. 

По факту снижается вред организма. Если есть только D, то он ничего не может сделать. 

Ретро-вирус лайк может терять почти все. То есть там всего есть один ген и то 
под вопросом. Промежуточный - это автономный лайн элемент. 

Лайн элемент - это длинный и повтор
и короткий повтор. ААА это сигнал, который в начале
есть и есть сигнал узнования полимираза. И он упращен настолько,
что уже от него ничего не осалось. орфы - это просто какие-то
ген, которые даже могут быть не эти гены. И у вас работает эта фигня,
поскольку она автономная, поскольку один из генов может передвигать сам себя.

А есть штуки, в которых осталось три сигнала и это
реально просто сигналы. В ней ничего нет, он просто отвечает за то,
что бы его перемещали с места на место. Он размера от 100 до 300 букв.

Самый прикол,они составляют 13\% от нашего генома. Они все умеют передвигаться.

В половых клетках все происходит куда медленне. Оно медленнее делится, там медленее все переезжает.
Есть мнение, когда пытаются смотреть на наш мозг невозможно описать
то количество связий, которое возможно сформировать, тем количеством, например даже
генов и регуреторных элементах,которые есть в геноме. И каким образом мозг понимает,
что нужно собраться вот так, если этого как бы нет? И  есть мнение что это происходит
из-за того, что в мозговых клетках активнее происходит перемешивание.
Мало того что имеет значение состав генома, еще и важно что и в каком порядке где находится.

Для того, что бы разные клетки были разными и работали поразному. Клетку не напрягает, что 
все, что встраивается из вне в днк - это какие-то вирусы? По дифолту есть 
куча механизмов, которые выключают все это. В клетке около нулевая 
активность перемещения. Если сделать мутанта и отключить 
механизм сдерживания, то ДНК начинает адски перемешиваться 
и обычно клетка умирает. НО есть 
нюансы. У нас в организме есть 
пачка супер важных генов, которые мы утащили у вирусов. 

Это интерферин и интерферон. Это гены, связанные с имунитетом. Вирусы исходно 
предпологали противостояние организму и разрабатывали свою 
сигнальную систему. В итоге в определенный момент организм научился 
ее экспалтировать. Возникновение планцентарных животных. То есть 
для формирования плаценты. Клетки могут просто сливаться 
вместе образуя огромную клетку - это не всегда 
удобно. Те же самые мышичные клетки они вполне фиксированы. 
Они соеднияются в них образуются. 

Две клетки они разные, они могут объединиться так, так нам не нужно, 
это перебор. Иногда нам хочется, что бы 
они были типа вместе, но не до конца. Поэтому между  ними образуются 
что-то и серии ограниченных контактов. Мембрана стягивается 
вместе и образуюся дырки. 

Суть в том, что белки которые это умеют делать тоже сперты у вирусов. 
Потому что вирусу, что бы попасть в другую клетку. Он мало того, что 
упакован в свою оболочку. Он еще упакован в оболочку того, кого он хочет заразить. 
Клетка живая, вирусная частица, ему нужна что бы его оболочка и эта мембрана, 
которую он получил... Он сидит в клетке в цитоплазме, 
собранный, когда он выходит из клетки он из нее отпачковывается и 
в итоге получается пузырь и получается отдельный пузырь с вирусом. Мембрана несет отпечаток
хозяина и любая другая клетка воспринимает его как 
часть своего. 

Подходит к клетке хозяина. Если он просто об нее стукается, то он просто отскочит. Ему же
надо с ней объединиться. Что бы проникнуть в клетку, нужно что бы 
две мембраны объединились. У него тоже специальные механизмы в виде болтов, 
которые все это стягивают очень плотно. А дальше они начинают друг к другу притягиваться. 
Поскольку это два масленных пузыря. пелепидный мембранный слой... И если очень близко 
друг к другу сделать они в итоге сливаются в одну каплю. Потом включается 
другой механизм, икасаидр начинает 
разваливаться на части и из него выходит содержимое вируса, которое при помощи 
дополнительных вирусных 
белков, которые тоже там сидели навешивают спецефические сигналы, что 
бы отправить их в клеточное ядро и там поселиться. 

Некоторые вирусы адаптируются на столько, что они престают выходить во внешнюю
среду. Потому что снаружи ждет имунная система. Поэтому они делают дырку, переходят 
заражают соседнии клетки. Так имунная система даже не в курсе о существование инфекции. 
Дальше, по поводу того, как они работают. Грипп - это контакт со слизистой. Поэтому 
гриппа есть специальные ферменты, которые умеют растворять слизь. То есть, если 
он туда попадает, то он начинает ее растворять растоворят и в итоге цепляется к клетке. 
Варианты разные. 

Короткие повторы signы считается есть у всех позвоночных . Но есть несколько людей, 
которые утверждали, что они собирали геномы хитрых животных и саинов там нет. Но возможно это 
какая-та ошибка. 

Это только набор сигналов. Причем сигнал, который назван B это по структуре tRNA. Внутри саина нет
ничего кроме tRNA, размером он около 300 букв. В 300 букв можно вместить короткую
мобильную последовательность. Прикол в том, что они перемещаются по геному и могут вызывать мутации.

Есть набор заболеваний который вызван тем, что они перескакивают.
Эти забаливания детям не передаются и другого человека ты тоже не заразишь.
Но свой геном они ломают. Есть бактериальный геном и бактерия умеет делать что-то.
Например переваривать рыбьи кости. При этом мы о геноме ничего не знаем и не секвенировали его.
Есть вариант, что мы можем попробовать вытащить этот ген?
Мы можем сделать так, что в бактерии начнет перемещаться этот мобильный элемент саина.
Перемещается он случайно, но количество эксперементов мы можем
масштабировать. Их будет 100 1000.

300 букв перемещаясь в новое место есть шанс, что они встроются и
сломаются. Делаем кучу эксперементов с перемещением и выбираем тот,
в котором у нас сломалось. В этом случае мы знаем, что последовательность
нашего короткого повтора находится рядм с геном.

И из этого места мы можем попытаться иследовать в края и найти ген,
который это делает.

Можно сочинить зонды, которые будут смотреть рядом с известной. Последовательность
саина нам известна, а всего остального генома - нет. Неизвестную область иследовать нельзя,
а если там будет хоть какой-то островок изветсности, то зацеплясь за него можно
иследовать соседнии регионы.

Делаем так, что бы наш элемент был  уникальный.

Полимиразная цепная реакия. Была придумана Кирьем-Льюисом... За что ему дали
нобеливскую премию. По легенде
он в состояние расширенного сознания в дождь ехал на машине.
И втыкал в дворнике и думал о чем-то своем. А иследовал он, как
работает репликация ДНК. Что есть паймаза, которая
синтезирует короткий кусочек, на который садиться полимираза,
бежит, достраивает цепочку... Так это работает. Теперь что
он придумал. По большому счету. какие
у нас есть необходимые элементы.

У нас есть матрица, то есть нам нужна ДНК. Нам нужна полимираза,
без нее никуда не обойтись. Нам нужна затравка - праймер. Причем здесь
его синтизирует праймаза, но мы можем праймер синтизировать сами.
Мы можем соченить 20 букв. Которые сядут в какое-то конкретное место в
геноме. Когда мы знаем хотя бы кусочек генома - мы можем
соченить к нему праймер. ДНК исходное, добавляем праймер,
добавляем полимеразу, и у нас синтизировался такой фрагмет. 

Обычно используется не один праймер а несколько. Мы знаем последовательность 
какого-то гена. Мы знаем его начало и конец. А дальше, не смотря на то, что ген один и тот же, 
он может быть у разных людей немного разный. Представим еще курче ситуацию. У нас есть ген. 
CCR5 delta 32 , 32 - это мутация. Называется дилеция 32 букв. Ген может быть либо такой, 
либо укароченная его версия, в которой вырезан кусочек. 

Либо полный, либо на 32 буквы короче. Хотим узнать у отдельного человека - какой у него ген. 
Если мы возьмем чуть-чуть ДНК то там будет мало ДНК, что бы это увидеть и причем там будет весь геном. 
А нам нужно много и только этого гена. 
Что мы делаем, мы сочиняем праймер на начало сочиняем комплементарную, которая залипает сюда. 

Есть две хромосомы, на которых расположен этот ген, на одной может быть нормальный, на другой 
поломанный. Исходя из этого есть 3 варианты генотипа. Это может быть интресно, потому что 
такой генотип - это имунитет к ВИЧ инфекции. Такой генотип - это замедленное течение 
ВИЧ инфекции. А такой у всех остальных. 
                                             
В клетке двухцепочечная ДНК. Мы ее взяли, после этого 
мы ее нагреваем до 95\% градусов и она расплетается на две цепочки. 
После этого мы добавляем сюда наши праймеры как прямые так и обратные. 

Они где-то склеиваются. После этого садится палимираза. Температура отжига 
праймера меньше. Она 55 градусов. После этого получится 2 двухцепочечных. 
Причем она будет длинная только в одну сторону. 

Что бы склеилась ДНК температура должна стать ниже. 
Цепная реакция называется потому, что здесь много циклов. Замечаем что, 
количество удваивается. Теперь 
у нас часть цепочек покусаны. 

После этого плавим и отжигаем праймеры. 
И у нас опять получится короткий фрагмент. Количество 
фрагментов мы удвоили, причем количество коротких
фрагментов стало еще больше. 

В следующем раунде станет еще больше. Пока полимераза бежит по одной длинной, 
другая полимираза может срабоать с кучей коротких фрагментов. 

Дальше, как вы можете догадаться, увеличилось количество ДНК по 
сравнению с исходной пробой в $2^{35}$, 35 - примерное количество циклов. 

Изобретение этого метода позволило начать иследовать ДНК. Это 
первый шаг к секвенированию. До этого что бы 
иследовать ДНК - нужно было иследовать 
огромное количество клеток. Потому что из 
каждой клетки ты берешь по чуть-чуть ДНК. 
А клетки растить долго и дорого. 

В этом случае не нужны живые клетки, можно хоть 
со стены соскоблить кровь. Тесты на отцовство, проверка на 
заболевание что-то еще. ПЦР в клинической диагностини 
используется сейчас бесконечно часто. 

А дальше все просто. Получилась некоторая банка, в
которой плавает много ДНК. При чем часть фрагментов будет 
короткая, часть длинная. Как выяснить какие кто?

Хотим узнать там короткие или длинные или смесь коротких 
и длинных. 

Это ДНК - она отрицательно заряжена. Мы делаем гель. Как губка 
напитанная водой. Плотность геля может быть разной, не суть. 
Мы делаем гель, добавляем наши образцы. 

Прикладываем постоянный ток и поскольку ДНК
отрицательно заряжена она будет двигаться в сторону положительную. 

Скорость движения зависит от размера. Мелкие фрагменты будут двигаться 
заведомо быстрее больших. У первого человека был только крупный фрагмент. 
У него будет одна полоса. У второго был только мелкий фрагмент, 
у него будет другая полоса. А у третьего будут две полоски. 

Этот тест крайне простой, делается со студентами биологами в лаболатории. 
Гомозигота по этим алелям - процентов 10. 

Белки при большой температуре  денатурируют. При большой температуре они ломаются. 
Человеческая полимираза при какой температуре сломается? 42 градусов. А тут 
95. 

Раньше просто несколько обогревателей стояли в ряд и студенты всю ночь переносили. 
10 минут в одной баночке, 7 минут в той, снова добавить полимиразу и снова переносить. 

Но это было совсем давно. Теперь это все автоматизировали. 

И открыли бактерии, живущие в горячих источниках в 
Елучтонских заповедниках. И она живет при темеперутару 80 градусов и 
полимираза работает при 72. 

при этом температура в 95 ее не убивает. Побольшому счету получили белок, 
который может стабильно работать при очень больших температурах. 

И с этого жизнь стала легко и просто идти, потому что ты берешь пробирку, 
добавляешь свое ДНК, добавляешь праймеры, нуклиотиды, полимиразу, 
закрываешь, ставишь в машину, ставишь количество циклов и температуру и 
за несколько часов она это все прогонит. 

Потом ставишь форез. 

Весь геном так конечно никто не смотрит. Есть методы, когда 
можно 100 1000 локусов посмотреть, но это не делается. 

Сейчас это используется больше в клинике. Что бы поставить диагноз 
врачу не нужно весь геном смотреть. 

Есть конкретный точки, который нужно проверить поломаны они или нет. 

У него правда крыша в конце поехала, он стал стороником ВИЧ дисиденства, 
которые доказывают, что ВИЧ не существуют, это все заговор фарм компаний. 
Самый клевый пример ВИЧ дисидентов. Был журнал, который состоял из
больных СПИДом репартеров, ВИЧ дисидентов, который закрылся, потому что 
все умерли. 

Ретротранспазоны передвигаются только по средствам РНК. Там есть шаг превратиться
из ДНК в РНК а потом встроиться в геном. У ДНК-транспазонов нет этого шага. 

Лайны, размер 7 тысяч в среднем. 

В отличие от стандартных ретротранспазонов у них нет LTR длинные терминальные повторы. 
Когда работает фермент, который встраивает геном, он сентезирует некоторую 
техничесвую последовательность, которая остается просто как результат его работы. 
Она ни для чего не нужна вообще. Она необходима просто на момент встраивания. Строили дом, остались 
котлаваны. 

Для лайнов этих последовательностей нет. Они немного подругому работают. 

В момент страивание LTR у них последовательности индинтичны. Это можно 
использовать, что бы определять время, когда произошло встраивание. 
Они будут накапливать случайные мутации и зная скорость мутации. 
Для мобильных элементов можно 
говоить в какой момент кто из них 
переместился. 

Есть версия, что все эти элементы так или иначе произошли от ретровирусов.

Ндагенные ретровирусы - это те которые смогли встроиться в геном и передаются с поколениями. 

Человек заболев ВИЧ инфекцией не передает ее детям. Потому что она не встроена в половые клетки. 
Но у кого-то может случиться ситуация, что ВИЧ встроиться в половые клетки. Получится что будут 
люди, у которых ВИЧ инфекция является просто частью генома. 

В каком-то смысле ВИЧ инфекция сама определет клетки, которые заражать. Ей нужен определенный 
набор белков на поверхности клетки. Он выбирает клетки содержащий SSR5. Если он поломаный, то 
он не находит цели и поэтому он уничтожает. 

Этот имунитет условный, а не абсалютный, поскольку есть субтипы вируса, который используют 
немного другой рецептор. У тебя есть шанс заболеть другим. 

Поиск повторов мало чем отличается от поиска генов. Так же есть две фундоментальных ветки. 
Есть поиск по известной базе данных. Репит маскет. Что бы работать использует базу 
всех известных повторов на текущий момент. Реп базе. Даешь ему свой геном, даешь 
базу с повторами и он ищет в геноме где совпадения. Это рабоате просто и быстро. 

Как найти повторы, который не известны? С этим немного сложнее, но что хорошего? 
Мы можем построить модель эндоретровирусов. То есть есть LTR, есть еще что-то. Можно 
искать кандидатов потом сортировать по надежности совпадения с другими фичами и так далее. 
Повторы очень плохо собирают. И сборка генома обычно стопориться на повторах. Поэтому 
повторы либо собраны с ошибками либо совсем не собраны. Но в биологии они играют важную роль. 



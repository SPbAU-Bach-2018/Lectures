\section{Паразиты}
Паратизм это симбиозы. Симбиозы включают 
в себя много вариантов. 

Тело A и B
Актинии и рак отшельник ++
Симбиоз взаимновыгодное сосуществование
0+ архидеии они используют другое дерево, что бы 
забраться повыше к солнцу. Дереву при этом пофиг, 
они с деревом за ресурсы не конкурируют
-+ И вариант когда одному хорошо в ушерб другому. 

Паразитами обычно называют тех, кого видно глазом. 
В целом, вирусы это экстримальные паразиты и мы должны
их включать и бывают паразиты на уровне бактерий. Есть вирусы вирусов. 

Есть растения паразиты, есть не зеленые растения 

ленточный червь попадает к животным, их много, 
ленточные черви, глисты, сосатели. 

Пиявки не совсем паразит, она хищник, она не зависит от хозяина. 


Паразиты бывают назекомые. Не известна паразиты позвоночные или 
млекопитающие. 

Есть прилипалы, которые селятся в 


10 ногий рак заползает к рыбе в рот и съедает ее язык и рыба начинает использовать 
рак как ее язык. 

\
Эктопаразиты которые живут с наружи

Это клещ, второй тоже  самое, только наевшийся


Второе волос, на нас с нами есть супер мелкие насекомые, 
которые откладывают яйца в основание волоса. Питаются отщелушиваюшей кожей

Алергия на пыль - это алергия на самом деле на 
обломки насекомых.

Оцекнуи 

Эндопаразиты,
причем межклеточные. Их хорошо видно. У них 
сложный жизненый цикл. Есть те, которые живут просто в кишечнике, 
есть

Есть эндопаразиты, которые живут внатри клетки это вирусы и всякие бактерии. 

Как боротьься с внутриклеточнами паразитами это проблема, так как 
это сильно завязано на жизни самой клетки. 


Бывает социальный паразиизм. Это не только 
у животных, но и у людей. 

Например подбрасывать яйца - кукушка. 

Есть еще насекомые. Насекомые больше похожи на прогрраммы, которые можно хакнуть, 
на уровне ыерамонов. 

А кукушонок, если ему что-то давет на спину, то у него выпрямляются ноги и она выпихивает
другие яйца.

У акул например, еще внутри акулый акулята начинают друг друга есть и
рождается только самые сильные акулята.

Кукулки, и есть подброшенные личинки, они по фирамонам похожи и поэтому
они думают что это их личинки

есть жуки, которые по фирамоном проберается в муравейник, съедают королеву и
маскируются под нее и все остальные ее ходит.


Паразит может контролировать поведения хозяина.

Триматода, родстеник червика,
но жизне  птица корова, улитка, бывает рыба и
сам организм в разных фазах меняется и
в каждой фазе он может оставлять потомков. Переходят
только в одну сторону в определенном порядке.

Фаза, которая живет внутри улитки, готовиться перейти внутрь птицы. И
Нужно, что бы улитку съели. Паразит вползает в ус и начинает
мегать и превлекает птицу. Улитка начинает ползать на открытой местности
и что бы улитка не тратила время на размножения, она съедает половую систему улитку.


Муравьи умеют делать рабов. Есть муравьи, которые
и додецилацитылаты влияют на других муравьев и
обращенные муравьи начинают работать на
другую колонию.

Есть варианты, когда утаскивают еще не родившихся личинок и
их обрабатывают.

Паразиты под крышкой тоже челистоногое, родственик краба. Но у него нет
панцыря. Морские желудя. Эта фиговина проникает
между суставами панцыря и начинает выпирать из под
панцыря. Краб начинает воспринимать его как свою икру и
краб даже если он самец, начинает думать это самка.

Интересно, что произошло сильное изменение. 
Краба не смущяет, что это штука никогда 
не вылупляется. Исходно штука супер мелокое. 

Есть птица, которое есть насикомые, 
они съедают экстременты птиц. 


Червяк поселяется в кузнечика. Ему нужно кушать и 
он объедает его так, что бы он жил как можно дольше. 
Дальше ему нужно попасть в воду. И кузнечик начинает 
искать воду, хотя для кузнечика это опасно. 

Триматоды, которые заражают муравьев и 
Грибы которые живут в тропических лесах,
муравей идет и к нему прилетает спора гриба. 
гриб перевает его панцырь и проникает в нуторь. 
и берет под контроль нервной системы. Зомби-муравей, 
нервно двигается, ищет высокое место. Лист и вцепляется с
такой силой челустью, как сам вцепиться не может. Потом
разрастается по поверхности муравья из муравья выростает гриб 
и споры сыпятся с муровья. Можно эффективно уничтожать колонии. 

Дальше муравьи научились распозновать больных муравьев и оттаскивать его подальше. 

Таксоплазмос
Обычно на него тестируют беременных, обычно 
когда было несколько неудачных. 
60 миллионов человек на террритории штутов, населения примерн 300 миллионов. 
На мущинах меньше заметно, так как не береминеет. 

Паразитируют в мышах или птицах а потом в кошках. 
Таксоплозмосом иногда заражаются свиниь, так 
как жрут все подрят. Еще можно заразиться 
через фрукты или кошачии порезы. Он изличим. 

У мышей сильно меняется поведение. Мышке начинает 
нравится кошка. Нравится кошачий запах и так далее. 

Бывают люди, у которых очень много кошек. У тех, 
кто попал в аварии таксоплозмос встречается чаще.
Далает твое поведение более рискованным. 

Эндопомин паразит кодирует молекулы эндопомина и бъет 
по центрам удовольствия и слегка подстраивает их. 
С этим может быть связана с шизифринией или болезнью 
паркинсона. 
 
Сниижает эффективность работы. Если
очень много кошек снижается мозговая способность.

Голопередором, которым же лечат шизофрению. У
крыс востанавливает нормально поведение.

У нас в геноме таксоплазмы кадируются два гена
которые очень похожи
на человечиские нееромедиаторы, не те,
которые может использовать паразит,
а те, которые мы используем.

Нееромедиаторы 
Входящии сигналы
дендриты и аксоны
аксоны контактируют с другими дендритами.
Образуются много контакотов. Если
контакты увеличить внутри клеток есть
пузырьки и там есть нееромедиатора и
отправляют сигнал в клетку, если привышает планку,
то летит дальше. Аксон передает дендриту.

Осы могут откладывать яйца в крабу. 

Аксицатына и начиает копать яму, пока есть силы.

Аксицатын связан с социализация и 
заботой о потомстве,
импатия способность понять что чувствует
другой человек.
Поэтому мужщин

антигонист тестостерыно,
вырабатывается при обнимании.
Снимают социальную неловкость.

Мужщина умирает от рака легких,
отправляют мтостазы на анализ и
оказывается, что это не целовек,
а человек умер от рака червя. В нем
живет червяк и у него начался рак и оно
расползлось по человеку. Еще
у человека был имунодефицит.
В норме имунная система такое распознает.

У человека рак мозга,
делают тамограму, а опухуль в
другом полушарии. Оказалось,
это был червяк.

Стреляютс и голову и остается жив. 

Горных шахтах

Прбил череп и вылетил с другой стороны и
он остался жив.

Что относить к паразитам - это
филосовский вопрос.

Взаимодействие между разными организмами
может быть тонкорегулированным,
в том числе менять поведение. Например,
теже самые жуки - это
просто набор геннов, которые дает возможность
синтизировать эти вещества - ферамоны. С помощью
которых они захватывают муравейник.


Насколько паразиты вредны? Мы
являемся продуктами в очень длитильной эвалючие.
Все вокруг не очень гигиетично. Мир очень агрисивный вокруг.
Сейчас мы живем в экстримально стерильных условиях.
И это приводит к алергии. Имунитет расчитывая
что нужно искать постоянно врагов,
начинает искать врагов слишком активно,
а их нет. То есть что такое аллергии,
имунная система распознала чужиродный фрагмент, но он
не опасен, при этом мобилизуется все сила имунной системы, что
бы его уничтожить.

Системна красная волчанка. Аутоиммунные болезни. Начинают
атоковать собственную систему. Кроны, что бы
заразиться глистами. Имунная система начала выводить
глистов.

Препораты из белков. Займись делом.

То есть, сейчас мы зависим от паразитов, нельзя
просто так их выкинуть.
\section{Карты}
\subsection{Что такое карта?}
\includegraphics[width=0.5\linewidth]{im46.png}
\begin{description}
\item[Зачем нужны.]
Генетические и физические карты. В прошлый 
раз мы изучали структуру 
генома, там есть повторы, всякие хитрые штуки и 
задача карты все это как-то расположить, где что находится. 
Карты у нас бывают разные. 
Геном выглядит как-то так, но это мало информативно. Может 
показать всякие процессы - окей. 

\item[Географическая карта] - это 
поверхность содержащая сетку и некоторые условные знаки, 
здесь тоже самое. 

\item[Условные знаки] - гены, повторы, промотеры. Все разметка на 
карте. 

\item[Поверхность]- Геном. 

\item[Координаты] когда на карте указывается наличие какого-то места указывается 
широта, долгота. В нашем случае будут указываться начало конец. 
\end{description} 

\subsection{Как измерять расстояние:}
\includegraphics[width=0.5\linewidth]{im47.png}
\includegraphics[width=0.5\linewidth]{im43.png}

\begin{description}
\item[Описание картинок:]
Приблизительно так это выглядит. 
Это бактериальный геном. Точнее это видимо, митохондрия. 
Это названия генов. Здесь закодированная транспортная РНК.

\item[Проблема буквенного измерения:]
Расстояние как мерить? Количество буковок?
Это только если мы знаем буквы, а если буквы не знаем?

Проблема, количество букв мы знаем, еслимы все собрали.

А как нам быть с областями, например, центромерой. Мы их в принципе никогда не соберём.

\item[Обычное расстояние в условно метрах:]
Можно просто линейкой померить. Условно посмотреть в микроскоп
и посмотреть размер каждого участка. В целом,
в цито-генетических картах так и делают.

Берем, глазеем на хромосому, красим ее разными цветами.
Эта ее исчерченность сильно зависит от выбора красителя.
Причины, почему определённым красителем хромосома красится в определенный паттерн -
не известна. Есть предмет, который называется
генетические основы окраски хромосом и известно
что некоторые красители связываются, например с GC богатыми
областями, но почему именно в этом месте это не известно.

Можно линейкой. Но на уровне если так
хромосома нарисована. Ну вот можно представить,
что эта штука размером и миллионов 200,
а это 20 тысяч, то есть если взять
все эти элементы, то они схлопнутся
в точку.

А не должно ли быть количество букв на единицу длины какое-то конкретное?
Проблема в том, что количество букв сильно зависит от плотности упаковки в
конкретном месте.

А мы не говорим что оно вообще равномерно упаковано. То есть
погрешность будет где-то 10ки тысяч.

Если бы карта была в абсолютных значениях,
если бы мы знали последовательность букв,
то никакой проблемы. Это самый замечательный вариант.
Но этот вариант не возможно сделать.
\item[Что можно картировать:]
На эту карту можно картировать не только гены, но и
целые признаки. Можно сказать, что здесь находится диабет,
а здесь высокий рост или еще что-нибудь.
Я сейчас придумываю, что диабет и высокий рост
по одному куску, а так, в общем-то
для многих
вещей можно действительно определять конкретные позиции.
В общем, картируются не только гены, но и признаки. Цвет
волос, устойчивость к болезням. Что угодно. 
\end{description}

\subsection{Рекомбинации}
\includegraphics[width=0.5\linewidth]{im48.jpg}\\
\includegraphics[width=0.9\linewidth]{im30.png}

\begin{description}
\item[Рекомбинация:]
Дальше есть такая вещь, как группа сцепления.

Что такое рекомбинация вы помните? Которая
случается в метафазе мейоза. Когда
перекрестились и обменялись кусочками.

После рекомбинации получится ген, 
которого не было до этого 
\item[Цели рекомбинации:]
Изначально это в эволюции появилось 
как устранение двух цепочечных разрывов. 

А при деление клеток и производстве половых клеток и всем
остальном задача рекомбинации производить 
последовательности, которые до этого не встречались. 

Потомство всегда будет обладать последовательностью 
генома, которой не было даже у
родителя. То что передали родители 
вы перемешаете и передадите дальше. При этом 
образуются совершенно новые сочетания. Это очень 
клево. 

\item[Зависимость событий рекомбинации:]
Дальше вопрос. Cобытия рекомбинации могут случаться много раз. Есть 
ли связь, что события рекомбинации случались в одном и том же месте
или еще как. Они могут интерферировать. И как связано? 
Если где-то случилось случиться чаще? 

\begin{description}
\item[Места рекомбинации у популяции:]
Если взять геном, то у него будут некоторые точки, 
в которых это случается чаще, чем в других. 
НУ точнее если нарисовать график частоты, то 
он выглядит примерно так.

Рекомбинация может случаться где угодно, но в некоторых 
местах она действительно случается чаще. Но это 
место не конкретное, оно размазано. 

Но такую картину мы будем наблюдать, если 
мы возьмем несколько тысяч людей и посмотрим где 
у них происходила рекомбинация. 
\end{description}

Но если взять один отдельно взятый геном, то есть две 
взаимодействующих молекулы, то между ними не будут такого, что 
они здесь будут рекомбинировать, 
а потом редко дальше. У них интерференция 
будет выглядеть совсем иначе. 

Если здесь произошла интерференция, то 
следующая произойдет далеко от этого места. Почему? 

Событие рекомбинации в целом не очень простое. 
Две молекулы должны разорваться рядом, потом 
разорваться и соединиться вместе. Причем еще и без
ошибок. 
Там участвует много белков, там участвует \textbf{синаптонемный
комплекс}, который тут сидит, он большой, 
куча белков, 
которые всему этому помогают. И если он сидит, то 
рядом другому комплексу уже сесть не получится. 

Рекомбинация происходит вся одновременно. При мейозе 
гомологичные хромосомы в какой-то момент оказываются рядом. 
И одновременно по всей длине они перемешались. 

У нас есть мера длины основанная на том как часто происходит 
рекомбинация между двумя точками, называется. 

Из-за того что участвует белки у нас есть интерференция между 
местами рекомбинации, то есть не будут происходить две подряд. 

Прикол что это можно сломать или починить, если 
сделать мутантов, в которых 
эти белки не работают. То у нас исчезает 
интерференция, то есть можно 
искусственно сделать дрожжи мух кого угодно, 
у которых не будет интерференции. 

\item[Прячем повторы]
Белки скорее следят за правильностью.

То что сейчас описывается это рекомбинация - гомологичная 
рекомбинация, это частный случай гомологичной рекомбинации, 
которая называется кроссинговер. 

Рекомбинировать могут любые похожие участки. Если они 
похожи между собой они могут оказаться рядом и обменяться кусочками. 

Поэтому мы о повторах говорили, поэтому повторы очень сильно прячутся, 
метилируются, что бы они ни в коем случае небыли выставлены наружу, 
их упаковывают по максимуму. Потому что повторы, поскольку 
они всегда похожи, 
они приведут к нестабильности генома. Такие эксперименты тоже 
можно сделать. Если убрать метилирование с повторов, то 
весь геном начинает адски перемешиваться. 

\item[Измерение в шансах рекомбинации:]
Этот комплекс узнает какие-то места. Они называются 
\textbf{горячие точки} рекомбинации. 

Поскольку у нас есть горячие точки рекомбинации, смотрите что есть дальше. 

Шанс того, что два фрагмента окажутся вместе сильно зависит от расстояния между фрагментами. 
Если мы берем две точки, которые в геноме находятся рядом друг с другом. Если расстояние в одну букву 
это гарантирует, что участки не
будут рекомбинировать. Если мы начнем увеличивать это расстояние, то мы увеличиваем шанс того, что мы будем наблюдать это перемешивание. 

Но если измерять расстояние в шансах рекомбинации, то максимальное расстояние будет 50\%.
То есть мы наблюдаем рекомбенацию в половине случаев.

\end{description}

\subsection{Небольшое пояснение, почему шанс рекомбинации не может быть больше чем 50\%}

\begin{description}
\item[]
Есть один геном, который гетерозигота. У человека каждая 
позиция в геноме встречается дважды, потому что 
у него есть две хромосомы. 

Теперь нарисуем второго чувака. 

ABC это будут какие-то локусы(какая-та точка в геноме). 

Какие они могут сделать половые клетки? 
В половой клетке половина генома. 

Это разные аллели одного и того же гена. Низкий рост, высокий рост. 
Что такое диплоидный геном? 46 хромосом. всего. При этом 
можно сказать, что одна и та же хромосома встретилась два раза. 
Две первых хромосомы, две вторых, две третьих. В конце два 23. 

Если хромосомы одинаковые, то на них находится некоторый набор генов на первой 
хромосоме, 
1.1
1.2 
некоторый набор на второй. Это один и тот же ген,он выполняет почти одну и ту же функцию. 

Есть ген под названием A. Он является некоторым рецептором, который 
позволяет тебе воспринимать запах. При это есть 
ген a, это тот же самый ген, но в нем случилась дилиция, то есть 
в нем не хватает куска. Стал он размером 91 букву, теперь 
он этот запах не различает, он может быть как 
поломанным полностью, так и отличать 
запах но хуже. Один и тот же ген, выполняющий одну и ту же 
функцию. Но при этом он будет отличаться. 

Доменантный признок - это тот, который 
проявляется в гетерозиготном случае. 

Гометы у второй особе будут такие же. 

Некоторые гометы в целом оказоваться не жизнеспособными. 

\TODO %%выясняем варианты потомков без перемешивания. 

Есть некоторое 
расщипление. Это расщипление можно записать формулой 1:2:1.

Ровно так и думалось пока не было открыто, что 
эта штука может перемешиваться. Как 
это было открыто? 

Исходно предпологалось что мы можем наблюдать только такие генотипы, а
потом оказалось, что мы можем наблюдать занчительно 
более разный гентоип. Может слуиться генотип. 

Теперь теперь образуются еще много других вариантов. 

\TODO %%перебираем варианты.

Перемешивание происходит задолго до того, 
как мальчик или девочка вообще будут встречина. 

Перемешивание не всегда происходит. У нас 
так же 1000 наших наблюдений.

Те, где перемешивание не произошло убираем. 

То, по какие перемешивания происходят чаще 
можно востановить расстояния между локанами. 

\TODO %%пример  выписываем все варианты, смотрим в каких случаях 

В чем нюанс. В случае если признаки находятся на
разных хромосомах, то генотип будет передаваться такой же,
но при этом c на отдельной хромасоме. При этом в гамету
попадает одна из c или С. И в половине
случаев мы будем наблюдать рекамбинацию,
хотя ее там не было. И мы уже не можем
отличить от ситуации, когда локус находится на другой хромосоме.

Карту можно строить для триплетов, весь геном
бьется на тройки. Потом оставляют две и добавляют еще
один новый. Но все не так просто.

Если это разные концы одной хромосомы, то они точно так же
будут всегда рекомбинировать и это не отличимо от
того, что это совсем разные хромосомы.

Это одни из самых точных карт, которые сейчас есть. Но
они стоят очень дорого.

Мы можем картировать признаки. Можем выяснить расстояние, которое
отвечает за диабет относительно каких то других позиций в гене.
И мы будем более точно знать где он находится.

Это делали до того, как вообще появились сиквенаторы.

 
Группы сцепления. Признаки, которые рядом сидят 
они наследуются вместе. Есть одна группа сцепления. 
Каждый из призноков внути находится в ней, а есть 
границы. Сцепление определяется с помощью попарных сравнений. 

Обычно сравниваем, что маркеры оказываются вместе. 

Есть хромосома. Есть две точки. Если наблюдаем, что 
точки наследуются вместе. И мы тогда соединяем две точки. 

Это рисуется для популяции. 

Группа сцепления между горячими точками. 


То что сейчас было - это 
карты сцепления. Когда знаешь шансы рекомбинации между точками. 


Есть нюанс. Есть ABC. Первое расстояние 20, второе 10 и 30 в сумме 40. А дальше считаете и 
получается BCD и получается 15 и между AD расстояние 55. Больше 50. Но больше 50 не может 
быть для трех точек. А для общей карты может быть. 

Сантиморган - это единица измерение. И суммарное расстояние может быть 50 сантиморганов. 

Расстояние означает шанс сцепления. Когда считаем это шанс, а когда откладываем на карту - это 
расстояние. 50 означает это означает, что в половине случаев наблюдаем рекомбинацию. Если больше 50, то 
шанс все равно 50. 

Популяционный анализ - если возьмем одну особь, то мы не построим. Если мы возмем дофига особей. 
То можем найти блоки которые не рекомбинирующиеся. Дальше когда нашли эти кусочки. То 
мы можем просеквенировать эти кусочки по одному разу и дальше у человека есть 
фрагмет, у которого есть несколько характерных мутаций. Если паттерн этих 
мутаций бывает только в этом кусочке, то его потом не нужно весь секвенировать, 
то у него весь остальной блок, поскольку они наследуются все вместе. 

Но теперь эта идея не очень интересна, поскольку сейчас секвенировать стало дешево и 
проще все просеквенировать, чем микро чипы гнать отдельно. Когда мы ищем сцепление с нашей болезнью. 
Если мы находим что диабет сцеплен с этим меркером. Но мы  не нашли причину болезни. 
Мы нашли сцепленность со всем блоком. Блок 6-8 тысяч букв. Иногда воспринимают, что 
мы нашли сцепленность какой-то конкретной мутации с болезнью. Но это не так. 
Эта мотация не факт что вызывает болезнь. Болезнь может быть вызвана содержанием самого блока. 

\end{description}

\subsection{Наложения последовательностей нуклеотидов на хромосомы}

\includegraphics[width=0.8\linewidth]{im48.png}

\begin{description}
\item[Данные:] Есть карты, когда можно делать несколько по-другому. Есть геном, 
который имеет линейную форму. Это неизвестный геном. Например 
какой-нибудь странной обезьяны. Причем есть 
геном человека. Более менее известный.

\item[Что делаем:]
Берем из генома человека хромосому 1, ломаем ее на кусочки. 
Получается суп из кусочков хромосом 1. Мы еще метим их так, 
что бы они отсвечивали их каким-то цветом. А теперь 
мы наносим эти хромосомы на неизвестные хромосом. 

\item[Что получаем:]
А теперь куски, которые совпадают будут садиться на 
похожие места. И мы будем знать где она совпадает 
с первой хромосомой. 

\item[Что еще можно сделать:]
Можно взять не первую хромосому, а любую последовательность. 
Повтор центромеры, например, или эндоретро вирус. Его размножить, 
добавить к хромосомам, он сядет на места, которые с ним совпадают. 

Он садится на комплементарные. Сначала хромосомы расплетаем. 
\end{description}


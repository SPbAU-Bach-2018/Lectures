\subsection{Про тигра}
Простой вопрос на тему выбора. 
Пердставьте что вы тигр. На равном расстояние от 
вас находятся олень(то есть еда) и тигрица. Пойти в одну 
сторону - значит исключить шанс пойти в другую. То есть 
можно размножится и то не факт или можно поесть. Что 
выбрать тигру? 

Не все животные кстати, вообще едат, некоторые просто размножаются. 
Комары например, после личиночной стадии. 

И совсем не найти еду проблемы нет, она переодически встречается, 
но на это может не быть времени. 

Все животные всегда голодные в дикой природе. У человека многие 
проблемы со здоровьем из-за этого. Организм эвалюционировал так, 
что мы едим если можно есть и едим как можно больше как в 
последний раз, потому что не знаем, когда в следующий раз будет еда. 

Какую ошибку м совершаем? Мы слишком много думаем. Если долго 
сидеть и думать, то и тигрица уйдет и еда убежит. То есть 
правильное решение любое, которое сделано достаточно быстро. 

Теперь, когда подумать подольше себя оправдвает? 
Чем больше мы думаем, тем более точный ответ мы дадим. 
Так мы действительно лучше можем прикинуть расстояние, 
насколько реально поймать оленя и так далее... Думать 
дольше можно, если условия стабильны и ничего не меняются. 
Например, если олень и тигрица безношие и никуда не убегут. 

\subsection{Что такое геном?}
Что такое геном? 
В генетике ген - это наследуваемая единица. Можно 
дать такое определение, которое нас удовлетворит. Ген 
участок ДНК. То есть у нас есть последовательность 
на которой есть ген, как минимум один. 

Геномом можно назвать все последовательности ДНК, которые есть 
в ядре или в клетке. 

Дальше можем сделать достаточно точную класификацию 
всей ДНК даже не ззная, что такое ген. Можем разделить 
всю ДНК на гены и не гены. То есть про каждый учаток можем сказать, 
что это не ген, или пока что еще не ген. 

То есть у нас есть куча последовательностей и мы можем разделить их 
на части, что вот известные гены, которые кодируют белки, 
это известные части, которые кодируют РНК. 

Дальше остались еще какие-то последовательности, из 
них мы можем убрать повторы, о которых мы еще поговорим. Останутся и 
не гены и не повторы. Называть оставшуюся часть можно как угодно, 
например, пока не гены. 

В самом простом приближения, если я 
спрошу, все ли белки гены - то да. В понимание, что 
белки произошли от конкретной РНК, она от конкретного 
куска ДНК, который ген. Но это не совсем верно, 
поскольку есть белки, которые синтезируются белками. 

Есть ген, который является последовательностью нуклиотидов. 
Этот ген, без особо хитрых преобразований может быть прочитана 
в РНК из цепочки ДНК получается РНК. Строится комплиминатрная цепь, 
и да, там один нуклиотид меняется, но всем обычно лень писать 
вместо T U. Дальше у нас 
получается белок, как последовательность аминокислот. 
Там работает правила с триплетами. Всего 
триплетов 64 а аминокислот 20. Из этого 
следует избыточность кода. Из этого следует, что 
зная последовательность аминокислот, не факт, что правильно востановим 
последовательность РНК. Можно говорить, что ген - это участок ДНК,
а все остальное - это продукты этого гена. Может так даже будет точнее, 
но если остальные не хотят заморачиваться и скажут, что белок - это 
такой то ген, то я не обижусь. 

И еще есть разница между пептидом и белком, пептид - это просто 
последовательность нуклиатидов, а белок - это какая-та трехмерная 
более сложная свернутая структура. 

\subsection{Разница между эукориотическим и прокариатическим геномом}
Теперь какие варианты структуры генов мы знаем? 
Кто такие прокариоты? 
В чем отличие устройства генов? 
Что такое оперон? 
Если спрашивать про отличие структуры действитлеьно 
можно сказать, что у эукориот есть интроны и экзоны, 
а у прокариот есть просто 
открытая рамка считывания. 

Так что такое оперон? 

У бактерии ра 5 миллионов генов 4-5 тысяч. Локтозный оперон, 
который регулирует. И оперон нужен что бы включать и 
выключать сразу блоки в которых входят несколько генов. 
Так как у бактерий в целом все проще, там вариантов для регуляции немного 
меньше. Оперон кластер генов, объеденный вместе под одним переключателем. 

Ген, который рамка считывания. Есть начало есть конец. 5'-3'. Промотер 
находится рядом в начале. Он является сигналом. К нему присоединяется 
специальный фермент РНК полимираза.  Далее на основе ДНК будет прочитано 
РНК и тут может быть и как один ген, так и некоторый блок генов. После этого 
на РНК могут сесть рибосомы Они туда сядут, потому что они узнают, что 
там начало гена. В начале гена есть специальные факторы ATG, дальше 
получится два белка. В уонце терминал. Знак конца.

Пример зачем это нужно - это локтозный оперон. Что бы переваривать локтозу - нужны они
все.  

Так работает у прокариот. 

А сколько фрагментов белка может получиться с одного промотера? 
1-1-1 или 1-много-по одному 1- много-много? За один прогон получается 
всегда одна цепочка. Но рибосома может сесть новая, пока еще одна уже работает. 
Есть цепочка ДНК, с которой синтизируется цепочка РНК. Из РНК может 
быть много белков. Рибосомы могут начинать работу до того, 
как закончила работу полимираза. На самом деле может 
сесть еще полимираза. И начать синтизировать еще одну цепочку. 

У эукориот так быть не может, почему? 
Потому что в ядре происходит синтез РНК, но белки произвдятся в цитоплазме, 
то есть в другом месте. Поэтому такой структуры быть не может. 

Поскольку процессы разделены мы можем регулировать все более тонко. 
У прокориот мы можем запустить процесс и он будет клево работать, 
здесь мы можем синтизировать РНК и на этом остановиться и 
когда надо синтизировать быстро белок. Прокариоты работают быстрее, 
ну то есть быстрее делятся. Но у эукориот адаптивность выше. 

\subsubsection{Как устрои ген у эукорит?} 
Там есть промотер. Но промотер не такой, как у прокариот. Он 
куда сложнее последовательность букв больше. Почему? Промотер 
узнается специальной штукой, которая называется 
транскрипционный фактор. И у бактерии не очень много транскрипционных 
факторов. 

Перед тем как садиться белок РНК полимираза с промотером
должен связаться транскрипционный фактор. И он узнает последовательность
последовательность промотера. И полимираза узнает
транскрипционный фактр(это белок) вместе с ДНК.

Полимираза одна и та же, а транскрипционые факторы - разные.
Оперон может включать в себя гены, которые кодируют транскрипционные факторы другие и 
так регулируется их количество и что производится. И он запустит каскад других процессов. 
Еще может быть связь другая, то есть белок, который нейтрализует другой транскрипционный фактор. 
Самыми разными способами. 

И поскольку и промотеры и транскрипционные факторы - это 
последовательности ДНК, то вполне может быть ситуация, когда одно из 
них сломалось, а другое нет. То есть могут быть транскрипционные факторы, 
для которых нет промотера. Так и происходит эвалюция этой фигни. 
Еше транскрипционный фактор может меняться так, что начнет узнавать 
немного другие промотеры. 

Стандартная схема появления новых генов, это надо что бы старный 
ген дублицировался в процессе ошибки и начал далее немного изменяться. 

Если копии не будет, то организм скорее всего умрет. 

Есть промотер. Есть ген,то есть открытая рамка считывания. 
Какие у нее есть свойства? 
Она начинается с ATG. Заканчивается стоп кодоном. И длина
рамки считывания кратна 3. 

Работает точно так же, 
есть промотер, с которым связывается транскрипционный фактор. 
Потом садиться полимираза. Дальше синтизируется РНК. И до конца и 
в этом месте остановилось, потому что 
там сигнал конца. 
Там есть UTR. 

Это еще не созревшее РНК, можно 
назвать пре-mRNA. В ней есть 
все, кроме промотера. После этого 
с ней происходит парочка изменений. 
Выризаются вещи, которые называются интроны. 

В РНК образуется петля и в результате этого 
интроны вырезаются. Экзон может тоже случйно отрезаться, 
если петля образовалась подругому, 
одна большая. Интроны всегда вырезаются.

Этот процесс называется сплайсинг. Экзоны вырезаются если есть дополнительная регуляция. 
В том числе могут вырезаться экзоны и с начала или с конца. 

Но не может поменяться порядок экзонов в продукте. 

Сплайсинг - это из морского дела - сращивание канатов, сплести вместе два каната. 

Получился продукт сплайсинга. Не смотря на то, что там есть старт кадон и 
находится стоп, то расстояние вместе с интронами может быть не кратно трем, 
каждый из экзонов может быть не кратин трем, но все 
вместе кратно трем. 

UTR - можно считать тоже часть экзона. Определение экзона осново на 
определение сплайсинга. Экзон - это то, что остается после вырезания. 
UTR привязывают к первому и последнему экзону. 

В целом может начинаться два разных экзона ATG, просто 
будут два независимых продукта, смотря как вырежится. И Стопов 
может быть много. ATG в целом не является стопом, он кодирует 
конкретную аминокислоту - метианин. Только первое ATG после промотера имеет 
некоторое фундаментальное значение. В начале то что он кодирует потом вырезается. 

Стоп кодоны в метохондриях не совпадают 
со стоп кодоны в ДНК в ядре. У нас есть 
две разных версии генетического кода. 

Это уже можно назвать mRNA, но тут не зватает двух вещей. Но тут 
не хватает системы контроля версий. Не РНК нужно повесить две метки, что 
бы доказать, что это наше РНК. 

С конца пришивается полиA хвост, много букв A подряд идет. Пришивает 
специальный фермент. В начали пришивается CUP - шапка. Шапка - это 
вполне обычный нуклиатид, только это пришивается совсем с другого 
направляени. То есть он ребром втыкается. Кажется, это буква G. 

И после этого эта вся фигня называется mRNA. 

Дальше мы готовы это отдать рибсоме. Рибосома будет 
узнавать ATG и еще она узнает стоп. И в белок превратиться только 
часть между, а часть UTR- нетранслируемые регионы. 

Первый кусок не транслируется, потому что еще не встретилось ATG, 
а последний, потому что стоп кодон встретился раньше. 

Программы для которого ищут гены. опередление что и как 
устроенно - критически важно. Что после чего может идти, 
как часто оно так выглядит. 

Есть буквы в интронах и экзонах, которые весьма спецефические и 
называются слайты сплайсинга. Там есть специальные пары, не все возможные 
пары могут быть слайтами сплайсинга. Если произошла мутация и изменился 
слайт сплайсинга, то по сути ломается все система, потому что 
не может вырезаться интрон после этого. 

Эта вся фигня только у эукориотов. 

Какие преимущества - больше разнообразия, 
бывают стоп кодоны, которые могут менять свое значение в зависимости от контекста. 
Триплет, является стоп кодоном, но если есть 
рядом сицелотистыин, то можно читать как аминокислоту. 

Стопкодоны - это обычно просто пустота. Сицелотистыин сделает так, что бы
стоп кодон воспринимался как другой триплет. Он просто плавает рядом. 

Бывают интроны в сотни тысяч букв, в которых живут 100 тысяч букв, в которых 
живут свои гены и в гене количестов экзонов может быть десятки и сотни. Когда рассматриваем 
работу конкретного гена, то не очень важно, что внутри интрона, 
поскольку он вырежится. Есть гены, которые кодируются в разных частях 
генома, а потом на уровне РНК сшиваются вместе. 

Система позволяет большую регуляцию 1) за счет транспорта. 
сколько синтезируем РНК из ДНК, где мы храним РНК, можно в ядре, можно 
в клетке снаружи. Сколько рибосом доступно как мы регулируем сплайсинг. 

Альтернативный сплайсинг - это когда мы вырезаем экзоны. И образуются несколько 
вариантов. Разные варианты этих 
называются изоформами. Бывают транскрипционные изоформы, бывают 
изоформы белка - протеины. Один ген может давать несколько изоформ. 

ПреРНК всегда одна, а мРНК может быть разной. Интрон считается геном, 
он не считается кодирующей частью, но считается геном. 

Есть ДНК, с которой синтезируется премРНК Это всегда один к одному, 
а с одной премРНК может образоваться 
два варианта продукта. И это уже будет мРНК и с каждым из них 
может образоваться по одному белку, но при этом это разные белки. 

Нужно определить координат гена в ДНК. Есть несколько механизмоф сплайсинга. 
Он может сам происходить за счет структуры интрона, либо он может происходить за счет дополнительных 
белков. На границах интрона-экзонов должны быть слайты сплайсинга, иногда
этого достаточно, а иногда дополнительно будут привлекаться белки. 


Если мы хотим сделать список генов для прокариот то нет 
никакой трудности. 

Это просто набор интервалов. 
ch1 начинается позиция 100 заканчивается 10000
ch1 позиция 10500 заканчивается 100500. 

Мы предпологаем что позиция начала это ATG и заканчивается стоп кодоном. 
Тут мы просто рамки считывания нашли. 

В разных форматах начинают считать поразному с 0 или с 1. 

С экзонами сложнее. 
Говорим началось в позиции 1, закончилось в позиции 19,
это первый экзон, всего
19 букв, явно не кратно 3. Но теперь мы можем
посчитать, сколько нам не хватает до тройки.

Следующие 56 64. Здесь есть нюанс,
мы хотим понять, как жти последовательности превратить в белок.
19 букв. В кодоне есть три позиции и нцжно понять. какой сдвиг
по модулю 3 надо брать.

В первом случае с первой, а во втором случае с третий.
Параметр фаза, и она бывает 0 1 или 2.

Если дать строку в 30 букв, то сколько всего рамок считывания существует?
Всего 6, три в одну сторону, три в другую.

Аминокислота не потеряется, если мы обхединим
два куска вместе. Формат из трех слов называется ...
Но есть форматы с более сложной ирархической структурой,
в котором еще есть поле, которое говорит, что это экзон.
И еще одно поле говорит, экзон какого гена.

Правильно конец называть сигнал терминации репликации. Обычно
структура ДНК у нас линейная, а там будет
такая шпилька. Как образуется петля в ДНК? Она
сама себе комплементарна.

Бывает хитрее, что там не шпилька, а последовательность которая
узнается другим беком и сюда садиться другой белок.
На который полимираза натыкается и сваливается.

\subsection{Виды РНК}
На разных последовательностях разные гены могут быть. Там еще строчка,
на какой цепочке может быть.

Чаще всего на второй стороне ничего не будет, но бывает. И Ген будет
направлен в другую сторону и не смотря на то, что буквы будут
комплементарные, они будут другими.

есть штука, под названием Микроинтерфирирущие РНК.
siRNA используется для регуляции всей этой фигни. У вас есть
ген и он читается в ту сторону. При этом с другой стороны тоже
находится сигнал гена, но только не белок кодирующий, у него нету
открытой рамки считывания, но у него есть промотер и так далее.
Синтизирует зеркальное отражение гена. И если
они оба оказываются в клетке, то они склеються. А двухцепочечное
РНК бывает только у вирусов. И все клетки это знают и такое
сразу уничтожает клетка.

Если работает только этот ген, то его продукт есть, если
есть переключатель, который запустит siRNA, то они будут на пол пути
взаимоуничтожаться и будут типа выключен.

Какие еще виды РНК бывают?
mRNA - кодируют белок. 
tRNA - таскает аминокислоты для рибосомы. 
rRNA - рибосомы, которые будут синтизировать белок

lnRNA - длинная некодирующая РНК
siRNA - обратно комплиментарное. 
микроRNA  - тоже вариант переключателя, только не белок, а РНК. В чем-то транскрипционный фактор. 

si считается регулирует отдельный ген. 
микроРНК запускает каскады процессов. Они никуда не присоединяется и работает на другом 
уровне. Она регулирует целые блоки. 

У прокориот одна полимираза, которая делает все. А у эукориот их три. Все РНК проме mRNA
называются не кодирующеми. 

У эукориот три разных полимиразы с различным 
функционалом. Есть полимираза, которая делает mRNA и узнает 
специальные промоуторы для mRNA. Есть специальнаые полимиразы, 
которые делаеют rRNA. И есть тип полимиразы, которая работает со всеми остальными типами. 


Теперь нам нужно при сплайсинге, что бы два куска оказались рядом, они далеко. Это 
делается с посощью шпилек. Получаеются шпильки внутри шпилек - достаточно ожурная структура. 

 




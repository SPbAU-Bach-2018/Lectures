\subsection{Операции}

К ключам мы ещё вернёмся, давайте теперь об операциях с данными.
Есть несколько описывающих теорий, мы займёмся реляционной алгеброй.
Те операции, что мы рассмотрим, так или иначе реализованы в современных системах.

Операции:
\[ ∪ \quad ∩ \quad ∖ \quad × \quad \bowtie \quad σ \quad π \quad → \]

\subsubsection{Проекция \texorpdfstring{$π_A$}{π\_A}}

Проекция отображения $R$ по подмножеству атрибутов $A$ "--- это отношение $π_A(R)$ на атрибутах $A$, что
\[ t ∈ π_A(R) ⇔ ∃ t' ∈ R\colon V_A(t') = V_A(t) \]
Короче говоря, отобрали только нужное подмножество атрибутов. В SQL это соотвествует перечислению атрибутов после слова \sql'SELECT'.

\subsubsection{Селекция \texorpdfstring{$σ_θ$}{σ\_θ}}

Селекция отображения $R$ по предикату $θ$ "--- отношение, в котором из тела остались только те кортежи, которые принял предикат.
\[ σ_θ = \{ t ∈ R | θ(t) \} \]
В SQL это соотвествует \sql'WHERE'.

\subsubsection{Базовые операции над множествами}

Операции объединения, пересечения и разности отношений делают соотвествующие операции на телах отношений,
но есть соотвестующее требование "--- схемы отношений должны быть равны (как по именам атрибутов, так и по их доменов!).

На практике ограничения слабже, за счёт неявной следующей операции:

\def\xra{\xrightarrow}
\subsubsection{Переименование \texorpdfstring{$\xra{A \ra A'}$}{-- A-A' ->}}

Операция переименования $R \xra{A \ra A'} R'$ позволяет переименовать одно подмножество атрибутов, сменяя их имена, но не меняя домены.

\subsubsection{Декартого произведение \texorpdfstring{$×$}{×}}

Декартого произведение чем-то похоже на обычное на множествах.
Мы требуем, чтобы у отношений не было пересекающихся атрибутов.
Тогда результатом будет отношение, схема которого "--- объединение отношений, а кортежи "--- склееные пары кортежей обоих отношений.
\begin{gather*}
	Scheme(R × S) = Scheme(R) ⊔ Scheme(S) \\
	t ∈ R × S ⇔ ∃ r ∈ R, s ∈ S\colon V_R(r) = V_R(t) ∧ V_S(s) = V_S(t)
\end{gather*}
На практике можно, чтобы атрибуты пересекались, их просто переименовывают.
В SQL это написать через запятую несколько таблиц в \sql'FROM', например
\begin{sqlcode}
SELECT * FROM R, S;
\end{sqlcode}

\subsubsection{Соединение \texorpdfstring{$⋈ $}{$⋈ $}}

Пусть есть два отношения на пересекающихся атрибутах:
$R(A, B)$ и $S(B, C)$.
Далее, соединение "--- это отношение на объединении атрибутов,
а в тело войдут только те пары кортежей, у которых на пересекающимся множестве атрибутов значения равны.
\begin{gather*}
	Scheme(R ⋈ S) = Scheme(R) ∪ Scheme(S) \\
	t ∈ R ⋈ S ⇔ ∃ r ∈ R, s ∈ S\colon V_R(r) = V_R(t) ∧ V_S(s) = V_S(t)
		∧ V_B(r) = V_B(S)
\end{gather*}

В SQL это \sql'NATURAL JOIN'.
На практике накладывается $θ$-соединение, в котором последнее условие на равенство на полях $B$ заменяется на предикат.
По сути, мы построили выборку по декартову произведению $σ_θ(R × S)$, только не раздваивали атрибуты.
Например, вот:
\begin{sqlcode}
-- неявно
SELECT *
	FROM R, S
	WHERE R.id = S.id;
-- явно
SELECT *
	FROM R
	JOIN S ON R.id = S.id;
\end{sqlcode}
Тут в результат уйдут два атрибута равных значений из-за переименования.

Эта операция крайне распространённая на практике, вряд ли можно изучать базы данных дольше, чем два дня, и не написать это.

\subsection{Внешние ключи}

Пусть у нас есть таблица студентов (номер, ФИО), и таблица стипендий (номер студента, месяц, деньги).
\begin{center}\tt
\begin{tabular}{|c|c|}
	\hline
	id & name \\\hline
	1 & John \\
	9 & Jack \\
	2 & James \\\hline
\end{tabular} \quad \begin{tabular}{|c|c|c|}
	\hline
	id & month & money \\\hline
	1 & \$1000 & Sep \\
	2 & \$1000 & Sep \\
	8 & \$8000 & Sep \\\hline
\end{tabular}
\end{center}

Как видно, \$8000 достались никому.
А как так-то? Можно проверять, что выражение истинно:
\[ π_{StudentId}(Scholarship) \subset π_{StudentId}(Student) \]

\begin{Def}
	Внешним ключом (foreign key) в отношении $R$ на отношение $S$, имеющее первичный ключ $K$,
	называется множество атрибутов $R$, совпадающих по имени и домену с ключом $S$,
	что если в $R$ эти атрибуты принимают такие значения, то есть кортеж в $S$ с таким ключом:
	\[ π_K(S) ⊃ π_{FK}(R) \]
\end{Def}
СУБД, соотвесвтенно, может предоставлять гарантию, что заданное ограничение выполняется.
Программисту требуется задать ключ и внешний ключ.
Для гарантии СУБД может делать следующее:
\begin{itemize}
\item
	При добавлении строчки в $R$ с неверным значением внешнего ключа отвергнуть его.

\item
	При пропаже или изменении соотвествующей строчки в $S$ можно удалить/обновить всю зависимую информацию (каскадно),
	переписать значение в особое/сбросить в \sql'NULL' или отвергнуть изменение (по умолчанию).
\end{itemize}
До недавнего времени MySQL имел ещё один вариант действия: забить.
На практике совпадений имён не сильно требуют, нужно просто пары указывать или неявное переименование.

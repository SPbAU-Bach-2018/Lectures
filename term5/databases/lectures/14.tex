
Критрерий прост: если нет циклов, то сериализуемо.

Покажем индукцией по количеству вершин. Для одной вершины циклов нет.
Для нескольких вершин "--- для графа без циклов есть вершина-исток (нет входящих рёбер), её можно отсоеденить и применить предположение индукции.

По-сути: делаем topsort.

Как же теперь построить планировщик?
Нам помогут замки, или блокировки.
Хочешь что-то сделать с каким-нибудь объектом? "--- Возьми блокировку ($L_i(X)$).
Закончил что-то делать? Отпусти ($U_i(X)$).
Двух замков на одной записи быть не может.

Пока эти правила нам не очень помогают "--- мы можем каждую операцию с элементом обернуть в захват-освобождение, и никакой сериализации по конфликтам не получается. Нужны ещё правила.

Все захваты должны быть до всех освобождений.
Если нужны какие-то элементы, хватай блокировки заранее, и снимай в конце.

Такой набор правил называется двухфазовым протоколом (2PL).
Он уже гарантирует сериализуемость по конфликтам.
Почему? Возьмём первую транзакцию, которая будет снимать блокировки.
Вытянем её в начало, продолжим по индукции.
Получится ли это? Если возник конфликт, то есть нарушение блокировки, значит в оригинале был не 2PL.

Можно думать, что транзакция уже подтверждена, когда мы начинаем освобождать блокировки.

Ещё есть проблема с дедлоками. Хотелось бы гарантировать порядок взятий... Но так сложно.
Обычно поддерживают граф ожидания, и отслеживают возникновения циклов. Если возник "--- кто-то нехороший, и надо убить.

Ещё блокировки "--- иногда слишком строго.
Разрешим иногда разделяемые блокировки.
Их можно одновременно брать на один и тот же ресурс, но нельзя одновременно брать на этот же ресурс обычную блокировку.
Это позволяет оптимизировать чтения.

\section{Сбои}

\begin{description}
\item[Сбой носителя.]
	Такое всё-таки случается: на работающем диске головка царапает поверхность, данные повреждаются.
	Способы борьбы: дублировать!

\item[Системный сбой]
	Что-то пошло не так, программа упала или что-то такое, но данные на диске целые, хотя и не всё записанно.
	Перезапустились, восстановили.

\item[Катастрофический сбой.]
	Вот был ваш датацентр, иии... сгорел. Очень жаль.

\item[Сбой региона.]
	Вот были ваши датаценты в Америке, а там... взорвался Йеллоустоун.
	И нет ваших датацентров... И Америки нет.
	Тут можно много такого придумывать.
	Кажется, в этом сценарии может быть немножко не до ваших данных.
\end{description}

Системный сбой "--- программа записала данные в буфер, а потом мы перезагрузились... и данные пропали, мы запишем на диск ересь, или вообще ничего.
Можно делать сразу гарантируемую запись на диск, долго "--- случайный доступ очень уж медленный.

Можно пытаться устроить последовательную запись.
Построим опережающий журнал (WAL, write-ahead log): в него на диске будем сразу писать информацию о всех действиях записи,
потом только запускать запись на основной диск, а потом помечать в журнале об успешности.
Такой журнал имеет последовательную запись, это быстрее.

Зачем он нужен? Мы так можем после сбоя прочитать журнал и узнать, что успели сделать, что ещё надо попробовать записать.

\setauthor{Дима Лапшин}
\begin{Def}
	\textit{III нормальная форма} "--- любой неключевой атрибут неприводимо и \textbf{нетранзитивно} зависит от потенциального ключа.
\end{Def}
Например, если вспомнить нашу таблицу с планетами: дата, планета, капитан, корабль, капитан, рейтинг.
В прошлый раз, ради 2 нормальной формы мы вынесли строй на планете в справочник.
Тут мы наблюдаем, что рейтинг капитана зависит в том числе от капитана, что нарушает третью нормальную форму.
Починить есть два способа декомпозиции:
\begin{enumerate}
	\item Дата, планета, корабль, капитан; капитан, рейтинг.
	\item Дата, планета, рейтинг; дата, планета, корабль, капитан.
\end{enumerate}
Второй способ хуже, мы теряем функциональную зависимость между капитаном и его рейтингом.

\begin{Def}
	\textit{Нормальная форма Бойса-Кодда} "--- В любой нетривиальной зависимоти детерминантом является потенциальный ключ.
\end{Def}
Это надмножество 3 нормальной формы.
Рассмотрим отношение "--- дата, корабль, капитан.
Пусть капитан зависит от даты и корабля (это очевидно), а капитан всегда управляет только одним кораблём (и корабль зависит от капитана).
Итого, есть два потенциальных ключа: дата"---корабль, и дата"---капитан.
Каждый атрибут "--- часть потенциального ключа, 3 нормальная форма выполнена, но "--- не нормальная форма Бойса-Кодда,
так как в зависимости капитан$\ra$корабль капитан не является потенциальным ключом.

Как же решить такой пример?
Это довольно нетривиально.
При декомпозиции теряются какие-то зависимости.
Вообще, не все отношения можно привести в эту нормальную форму.

Есть ещё нормальные формы, но они довольно домороченные.
Если вы уже дойдёте до НФБК, у вас всё довольно хорошо.

\chapter{Физическая реализация}

А где вообще может храниться информация?
\begin{center}\begin{tabular}{l|lllll}
& Объём & RandAccess & SeqAccess & Зависит & Цена \\
Регистры           & <1 Кб      & 0               & N/A       & нет                                             & $∞$  \\
L1-L3 кеши         & 10 Кб-N Mб & 1 нс 5 нс 10 нс & N/A       & нет                                             & N/A \\
Оперативная память & до 1ТБ     & 100 нс          & N-NN Гб/c & нет\footnote{На грани да "--- Cold boot attack} & 100\$ / N Гб \\
HDD                & N Тб       & N мс            & NN Мб/c   & да                                              & 50\$ / 1 Тб \\
SSD                & NNN Гб     & 0.1мс           & 100Мб/c   & да                                              & 500\$ / Тб \\ 
Магнитная лента    & $∞$        & N/A             & быстро    & да                                              & ?
\end{tabular}\end{center}
Что такое жёсткий диск (Hard Disk Drive, HDD)?
Это набор блинов на единой вращающейся оси.
На этих блинах есть набор концентрических дорожек.
Также есть набор головок, парящих над этими блинами, находящихся на единой оси, и способными двигаться между этими дорожками.
Дорожки поделены на сектора (хотя правильно их называть сегментами).
Чтобы прочитать иди записать данные, надо передвинуть головку на дорожку, и прочитать-записать целиком сектор,
когда поверхность прокрутится и сектор окажется под головкой.

Что такое твердотельные накопители (Solid State Drive, SSD)?
Это целиком электронная схема, нет движущихся частей, гораздо быстрее работает, время записи при этом в несколько раз больше чтения.
Но: дорого, ограничена перезапись одной ячейки.

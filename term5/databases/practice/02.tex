\chapter{Занятие 2}

Домашнее задание выгладит примерно так.
Их будет три варианта, каждой команде достанется случайный (две комманды могут получить одно задание).

Белаем БД для управления полётами между планетами.
Есть флот кораблей (характеризуется названием корабля и рангом).
Рейсы ходят с Земли и обратно, на разные планеты. Планета характеризуется... \TODO есть в документе.

\section{SQL}

SQL описывается стандартами, нумеруемыми годами создания (последний SQL-2011).
В каждом стандарте есть базовая часть, обязательная для выполнения.
К сожалению, есть множество диалектов, по своему реализующих кучу деталей, практически не совместимы.

Типы данных? Их бывает много, бывает очень много:
\begin{itemize}
\item Целые: \sql'INTEGER', \sql'BIGINT', \sql'SMALLINT';
\item Фиксированная точка: \sql'NUMERIC', \sql'DECIMAL';
\item Плавущая точка: \sql'FLOAT', \sql'REAL', \sql'DOUBLE PRECISION';
\item Строковые: \sql'CHARACTER' (\sql'CHAR', фиксированная длина!!!), \sql'CHARACTER VARYING' (\sql'VARCHAR'), \sql'CLOB';
\item \sql'DATE', \sql'TIME', \sql'TIMESTAMP';
\item \sql'BOOLEAN'.
\end{itemize}

\subsection{Создание таблицы}
Все полные синтаксисы есть в документации.
SQL не чувствителен к регистру.
\begin{sqlcode}
CREATE TABLE Foo(); -- нет столбцов, но работает ;)
DROP TABLE Foo; -- удалили
CREATE TABLE Foo(INT); -- ошибка, нет названия столбца,
-- но ответ об ошибке вам не сильно это подскажет
DROP TABLE Foo;
CREATE TABLE Foo(id INT, value FLOAT); -- OK, нельзя запятую лишнюю в конце!
\end{sqlcode}
Если хотите добавить столбик, то надо пересоздать таблицу. Не хотите терять данные "--- есть средства...
Но традиционно пишут в скриптах пару \sql'DROP' - \sql'CREATE'.
\begin{sqlcode}
DROP TABLE Foo; CREATE TABLE Foo(id INT, value CHAR(8));
INSERT INFO Foo(id, value) VALUES (1, 'qwe');
-- дополнит пробелами. Когда-то считалось эффективнее!

DROP TABLE Foo; CREATE TABLE Foo(id INT, value VARCHAR(8));
INSERT INFO Foo(id, value) VALUES (1, 'qwe');
-- не дополнит. Слишком длинное -- ошибка!
\end{sqlcode}
PostgreSQL рекомендует тип \sql'TEXT' вместо этого цирка.

\subsection{Вставка}

\begin{sqlcode}
INSERT INTO Foo(id, value) VALUES (1, '100'), (3, 'qwe');
\end{sqlcode}

\subsection{Выборка}

\begin{sqlcode}
SELECT 5; -- 5
SELECT 6 * 5; -- 30
SELECT * FROM Foo; -- все столбцы и все значения
SELECT * FROM Foo WHERE id = 1;
SELECT * FROM Foo WHERE value='qwe';
\end{sqlcode}
В \sql'WHERE' кучу всего можно писать, важно: немножко Паскаля (одиночное равно, словами И-ИЛИ, хехехе!).

\subsection{\texorpdfstring{\sql'NULL'}{NULL}}

Особое униваерсальное значение, абсолютная пустота.
Она ничему ни равна, ни \textit{не} равна.
Правильно определять его "--- \sql'IS NULL' и \sql'NOT IS NULL' / \sql'IS NOT NULL'.

\begin{sqlcode}
SELECT NULL == NULL; -- f = false
SELECT NULL != NULL; -- f = false
SELECT NULL IS NULL; -- t = true! He-hey!
\end{sqlcode}

\section{Разработка БД}

Есть некоторые инструменты, где можно рисовать объектную модель со связями, по которым генерируются БД.
Но они дорогие... Без них живём!

Нам надо выписать сущности, понять их атрибуты, выделить ключи, задать контрольные вопросы.

Сущности у нас такие "--- корабль, планета, капитан, билет, пассажир. Если вам кажется, что экземпляр один (флот) "--- это не очень сущность.
\begin{sqlcode}
DROP TABLE Spacecraft;
-- можно добавить в середину IF EXISTS, если хочется меньше сообщений об ошибках.
CREATE TABLE Spacecraft (
	name TEXT NOT NULL,
	class INT CHECK(class BETWEEN 1 AND 3), -- весёлый оператор! Можно ещё IN (1, 2, 3)
	capacity INT CKECK(capacity >= 0) -- больше контроля!
);
INSERT INTO Spacecraft (capacity) VALUES (-1); -- Error!
\end{sqlcode}
Можно ещё отдельный домен объявить со своими значениями!
А вот если множество значений меняется, то надо по-другому.
\begin{sqlcode}
DROP TABLE Planet; CREATE TABLE Planet (
	name TEXT,
	distance REAL,
	goverment TEXT CHECK(goverment IN ('democracy', 'communism', 'dictatorship'))
);
\end{sqlcode}
Это немножко железобетонно! Если клерк захочет добавить \sql"'anarchy'", у него ни прав на удаление таблицы (а давать их, кхм, НЕ НАДО!),
ни навыков...

Для этого есть паттерн "--- \textit{справочник}.
Вспомогательная таблица со значениями:
\begin{sqlcode}
DROP TABLE Government(
	id INT,
	value TEXT
);
INSERT INTO Government(id, value) VALUES (1, 'comunism'); -- Очепятка!
\end{sqlcode}
Именно поэтому рекомендуется использовать айдишники)

Кстати, PostgreSQL умеет в массивы.
И структуры в качестве значений.
Но думайте "--- не упутили ли вы сущность!

\subsection{Ключи}

Можно писать \sql'UNIQUE', можно \sql'PRIMARY KEY'.
Второе цельно, быстрее ищется (создаёт индекс) и делает автоматически \sql'NOT NULL'; первое независимо.
\begin{sqlcode}
DROP TABLE IF EXISTS Stacecraft;
CREATE TABLE Spacecraft (
	name TEXT PRIMARY KEY,
	class INT CHECK(class BETWEEN 1 AND 3),
	capacity INT CHECK(capacity >= 0),
	-- важно, после столбцов:
	UNIQUE(class, capacity)
	-- йохохо, нельзя двух кораблей, имеющих равные классы И вместимости!
);
\end{sqlcode}
Все ограничения пожно писать и у столбца (действует только на него), и внизу (на что укажите).

\subsection{Проверки!}
\begin{sqlcode}
DROP TABLE IF EXISTS Stacecraft;
CREATE TABLE Spacecraft (
	name TEXT,
	class INT,
	capacity INT,
	planet TEXT, -- ?
	goverment TEXT, -- ?!
	capitan TEXT, -- ?
	flight_date DATE -- ?
);
\end{sqlcode}
Как-то плохо...
Можно попробовать составить предложение---предикат. Получим информацию о полётах, но никак о кораблях!

Тут помогут ключи!
Ключ пока тут "--- \sql'PRIMARY KEY(name, date)', или ещё \\\sql'PRIMARY KEY(capitan, date)'.
Какие атрибуты зависят от ключа? \sql'goverment' зависит, вообще говоря, не очень от ключа, а от \sql'planet'!
\sql'capacity' и \sql'class' зависит от ключа не целиком, только от имени корабля!
Это очень плохо (теоретическое обоснование на следующей лекции).

Правильнее три таблицы "--- корабли, полёты, планеты. А ещё и справочник!

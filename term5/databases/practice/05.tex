\chapter{Занятие 4}

Теперь будем писать запросы к спроектированной базе данных.

\section{Запросы}

\begin{sqlcode}
SELECT * FROM Planets;
SELECT *
	FROM Flight
		JOIN Planet ON (Flight.planet_id = Planet.id);
SELECT Planet.name, Flight.date 
	FROM Flight
		JOIN Planet ON (Flight.planet_id = Planet.id);
\end{sqlcode}
Полное форма \sql'JOIN' "--- \sql'INNER JOIN', внутреннее соединение.
Каждая строчка будет иметь пару.
\sql'NATURAL JOIN' "--- сама подбирает по одноимённым столбцам.
Без соединения можно через декартого произведение:
\begin{sqlcode}
SELECT *
	FROM Flight, Planet
	WHERE Flight.planet_id = Planet.id;

SELECT Planet.name
	FROM Planet
	WHERE id IN (1, 2, 10);
\end{sqlcode}
Можно не написать название таблицы, если и так понятно, откуда он.

\section{Вложенные запросы}
\begin{sqlcode}
SELECT id, name
	FROM Planet
	WHERE id IN (
		SELECT planet_id
			FROM Flight
			WHERE commander_id IN (
				SELECT id
					FROM Commander
					WHERE name='Ким'
			)
	);
\end{sqlcode}
Это всё как-то громоздко.
Ещё и номера рейсов к этим планетам не выяснить.
А вот соединение справится:
\begin{sqlcode}
SELECT Flight.id, Planet.name
	FROM Flight
		JOIN Commander ON Flight.commander_id = Commander.id
		JOIN Planet on Flight.planet_id = Planet.id
	WHERE
		Commander.name = 'Ким';
\end{sqlcode}
Порядок соединений не важен для внутренних соединений.

Хотим планеты без рейсов.
\begin{sqlcode}
SELECT *
	FROM Planet, Flight
	WHERE Flight.planet_id <> Planet.id; -- это нагенерит кучу бреда.
SELECT *
	FROM Planet
	WHERE Planet.id NOT IN (
		SELECT planet_id
			FROM Flight
	);
-- ВНЕЗАПНО, не работает
SELECT * FROM Flight WHERE planet_id IS NULL;
-- Ой, есть строчка! И её NULL валит проверку NOT IN
\end{sqlcode}
Ограничения "---  это полезно.
\begin{sqlcode}
SELECT *
	FROM Planet
	WHERE Planet.id NOT IN (
		SELECT planet_id
			FROM Flight
			WHERE planet_id IS NOT NULL
	);
-- отстой
SELECT *
	FROM Planet
	LEFT JOIN Flight ON Planet.id = Flight.planet_id;
-- Планеты без полётов попадут с NULLами в вывод по одному разу.
SELECT *
	FROM Planet
	LEFT JOIN Flight ON Planet.id = Flight.planet_id
	WHERE Flight.planet_id IS NULL;
\end{sqlcode}
Ещё есть \sql'RIGHT JOIN'.

Задолбало писать имена таблиц?
Можно дать синонимы!
\begin{sqlcode}
SELECT *
	FROM Planet p
	LEFT JOIN Flight f ON p.id = f.planet_id;
\end{sqlcode}

Чтобы было читабельнее, не стоит делать преждевременные оптимизации.
\begin{sqlcode}
SELECT *
	FROM Planet p
	JOIN Flight f ON p.id = f.planet_id
	JOIN (
		SELECT name, id
			FROM Commander
			WHERE name = 'Ким'
	) c ON f.commander_id = c.id;
SELECT *
	FROM Planet p
	JOIN Flight f ON p.id = f.planet_id
	JOIN Commander c ON (
		f.commander_id = c.id
		AND c.name = 'Ким'
	);
\end{sqlcode}
СУБД сама оптимизирует запросы, пишите читабельно и правильно, без грязных хаков.
А то вот:
\begin{sqlcode}
SELECT *
	FROM Planet
	LEFT JOIN Flight ON (Planet.id = Flight.planet_id AND Planet.government_id = 6)
	WHERE Flight.planet_id IS NULL;
\end{sqlcode}
Вернёт больше одной планеты, так работает \sql'LEFT JOIN'.
Хотите только планеты с дикатурой "--- используйте условие фильтрации в \sql'WHERE'.

\section{Агрегатные функции}
Позволяют что-то считать.
На сколько планет были полёты?
\begin{sqlcode}
SELECT Count(Planet.id) -- всего полётов
	FROM Planet p
	JOIN Flight f ON f.planet_id = p.id;
SELECT Count(DISTINCT Planet.id) -- различные планеты, уже лучше!
	FROM Planet p
	JOIN Flight f ON f.planet_id = p.id;
SELECT Count(DISTINCT Planet.id, Commander.id)
	FROM Planet p
	JOIN Flight f ON f.planet_id = p.id
	JOIN Commander c on c.id = f.commander_id;
\end{sqlcode}
Ещё есть \sql'MAX', \sql'MIN', \sql'AVG'.
И можно веселться.
\begin{sqlcode}
SELECT SUM(2); -- Почему бы и не?
\end{sqlcode}

Количество полётов на планету?
\begin{sqlcode}
SELECT Planet.name, (
		SELECT COUNT(*)
			FROM Flight
			WHERE planet_id = Planet.id
	)
	FROM Planet;
\end{sqlcode}
Что за жесть?
\begin{sqlcode}
SELECT Planet.name, COUNT(*)
	FROM Planet
	JOIN Flight ON Flight.planet_id = Planet.id
	GROUP BY Planet.name;
\end{sqlcode}
Что происходит? Мы группируем агрегирующую функцию по названию планеты.
Правда, планету без полётов не найдём... \sql'LEFT JOIN' найдёт её! Но одну, а не ноль.
\begin{sqlcode}
SELECT Planet.name, Planet.id, COUNT(*)
	FROM Planet
	JOIN Flight ON Flight.planet_id = Planet.id
	GROUP BY Planet.name;
\end{sqlcode}
Эм... Ругается! Не очень ясно, какой id брать "--- нет группировки, а какой номер тогда брать?
Некоторые старнные СУБД позволяют оставить случайное значение...

PostgreSQL знает, что если идёт группировка по ключу (\sql'UNIQUE' / \sql'PRIMARY KEY'), то всё в порядке.
Не стоит лишний раз это злоупотреблять.
\begin{sqlcode}
SELECT Planet.name, COUNT(*)
	FROM Planet
	JOIN Flight ON Flight.planet_id = Planet.id
	GROUP BY Planet.id; -- OK
\end{sqlcode}

Если хотим применить фильтр к результату агрегации, то \sql'WHERE' неприменим.
Нужно писать условия в \sql'HAVING'.
\begin{sqlcode}
SELECT Planet.name, COUNT(*)
	FROM Planet
	JOIN Flight ON Flight.planet_id = Planet.id
	GROUP BY Planet.id
	HAVING Count(*) > 20;
\end{sqlcode}
Среднее число полётов?
Нужен подзапрос.
\begin{sqlcode}
SELECT ANG(cnt)
FROM (
	SELECT Planet.name, COUNT(*)
		FROM Planet
		JOIN Flight ON Flight.planet_id = Planet.id
		GROUP BY Planet.id
	) AS cnt;
\end{sqlcode}


\chapter{}

Будем писать сложные запросы.

Базовый запрос: количество людей на каждом рейсе.
\begin{sqlcode}
SELECT
	F.id, F.planet_id, F.spacecraft_id, F.commander_id, COUNT(*) AS pax_cnt
	FROM Flight F JOIN Booking B ON F.id = B.flight_id
	GROUP BY F.id;
\end{sqlcode}

Колисество планет, куда летал капитан:
\begin{sqlcode}
SELECT
	C.id, C.name, COUNT(planet_id) AS planet_count
	FROM (
		SELECT
			F.id, F.planet_id, F.spacecraft_id, F.commander_id, COUNT(*) AS pax_cnt
			FROM Flight F JOIN Booking B ON F.id = B.flight_id
			GROUP BY F.id;
	) AS PaxFlight
	JOIN Commander.C ON C.id = PaxFlight.commander_id
	GROUP BY C.id;
\end{sqlcode}
Мы пока считаем количество рейсов, имеющих хотя бы одного пассажира, куда летал капитан, если капитан летал хоть куда-нибудь.
Чтобы починить дублирующиеся планеты, нужно починить.
\begin{sqlcode}
SELECT
	C.id, C.name, COUNT(DISTINCT planet_id) AS planet_count
	FROM (
		SELECT
			F.id, F.planet_id, F.spacecraft_id, F.commander_id, COUNT(*) AS pax_cnt
			FROM Flight F JOIN Booking B ON F.id = B.flight_id
			GROUP BY F.id;
	) AS PaxFlight
	JOIN Commander.C ON C.id = PaxFlight.commander_id
	GROUP BY C.id;
\end{sqlcode}
Теперь рейсы без полётов: добавим \sql'LEFT JOIN'. Это вернёт полёты без пассажиров, но у них будет 1 пассажир, а не 0.
Попробуем считать не все записи, а \sql'DISTINCT B.ref_num':
\begin{sqlcode}
SELECT
	C.id, C.name, COUNT(DISTINCT planet_id) AS planet_count
	FROM (
		SELECT
			F.id, F.planet_id, F.spacecraft_id, F.commander_id, COUNT(DISTINCT B.ref_num) AS pax_cnt
			FROM Flight F LEFT JOIN Booking B ON F.id = B.flight_id
			GROUP BY F.id;
	) AS PaxFlight
	JOIN Commander.C ON C.id = PaxFlight.commander_id
	GROUP BY C.id;
\end{sqlcode}
Это, внезапно, работает! Но имелся в виду другой, или ваша СУБД так не может.
Можно и по-другому, через сумму единичек:
\begin{sqlcode}
SELECT
	C.id, C.name, COUNT(DISTINCT planet_id) AS planet_count
	FROM (
		SELECT
			F.id, F.planet_id, F.spacecraft_id, F.commander_id,
			SUM(CASE B.flight_id IS NOT NULL WHEN true THEN 1 ELSE 0) AS pax_cnt
			FROM Flight F LEFT JOIN Booking B ON F.id = B.flight_id
			GROUP BY F.id;
	) AS PaxFlight
	JOIN Commander.C ON C.id = PaxFlight.commander_id
	GROUP BY C.id;
\end{sqlcode}
Этот трюк тоже работает.

Ещё и коммандиры без рейсов! Повторяя трюк, у нас уже \sql'DISTINCT', добавим \sql'RIGHT JOIN':
\begin{sqlcode}
SELECT
	C.id, C.name, COUNT(DISTINCT planet_id) AS planet_count
	FROM (
		SELECT
			F.id, F.planet_id, F.spacecraft_id, F.commander_id, COUNT(DISTINCT B.ref_num) AS pax_cnt
			FROM Flight F LEFT JOIN Booking B ON F.id = B.flight_id
			GROUP BY F.id;
	) AS PaxFlight
	RIGHT JOIN Commander.C ON C.id = PaxFlight.commander_id
	GROUP BY C.id;
\end{sqlcode}
Теперь считаем среднее число пассажиров в рейсе на планету.
Если бы мы скопипастили старую версию вложенного запроса, то было бы грустно.
\begin{sqlcode}
SELECT
	P.id, P.name, AVG(pax_cnt)
	FROM (
		SELECT
			F.id, F.planet_id, F.spacecraft_id, F.commander_id, COUNT(DISTINCT B.ref_num) AS pax_cnt
			FROM Flight F LEFT JOIN Booking B ON F.id = B.flight_id
			GROUP BY F.id;
	) AS PaxFlight
	RIGHT JOIN Commander.C ON C.id = PaxFlight.commander_id
	GROUP BY C.id;
\end{sqlcode}

Всё равно копипаст это зло.
Хочется, чтобы запрос жил подольше.

\section{Представления (\texorpdfstring{\tt VIEW}{VIEW})}

\begin{sqlcode}
CREATE VIEW PaxPerFlight AS
	SELECT
		F.id, F.planet_id, F.spacecraft_id, F.commander_id, COUNT(*) AS pax_cnt
		FROM Flight F JOIN Booking B ON F.id = B.flight_id
		GROUP BY F.id;
SELECT
	P.id, P.name, AVG(pax_cnt)
	FROM PaxPerFlight
	RIGHT JOIN Commander.C ON C.id = PaxFlight.commander_id
	GROUP BY C.id;
\end{sqlcode}

Таким образом, если в представлении лажа, можно его заменить.
\begin{sqlcode}
-- можно и DROP IF EXISTS...
CREATE OR REPLACE VIEW PaxPerFlight AS
	SELECT
		F.id, F.planet_id, F.spacecraft_id, F.commander_id, COUNT(DISTINCT B.ref_num) AS pax_cnt
		FROM Flight F LEFT JOIN Booking B ON F.id = B.flight_id
		GROUP BY F.id;
\end{sqlcode}

Почему это лучше отдельной таблицы?
Это всегда живые данные, не надо их руками обновлять.
Как оно устроено "--- скорее всего, знать не надо.
Можно считать, что просто подставляется текст запроса.

Представления входят в стандарт.
Чаще всего, они только на запись, а вот вставлять-обновлять в них нельзя\footnote{Но иногда можно!}.

Представления -- часть схемы, нужно иметь права на модифицирование схемы БД для их создания.
Как бороться с просто большими запросами "--- позже.

Продолжаем считать капитанов.
Трёх лучших по количеству планет?
\begin{sqlcode}
SELECT
	id, planet_count
	FROM (
	SELECT C.id, C.name, COUNT(DISTINCT planet_id) AS planet_count
		FROM PaxPerFlight
		RIGHT JOIN Commander C on C.id = PaxPerFlight.commander_id
		GROUP BY C.id
	) AS CmdPlanet
	ORDER BY CmdPlanet.planet_count DESC
	LIMIT 3; -- не стандарт, все умеют
)
\end{sqlcode}
Предположим, что это "--- представление (без ограничения на 3, естественно) \sql'CommanderPlanetCount'.

\begin{sqlcode}
-- огромный запрос про крутость капитанов, с рейтингом элита, по дистанции и пассажирам.
\end{sqlcode}

Шалость: почему поменялся порядок записей? Мы же в представлении посортировали?
Так вот: с тех пор в нашем запросе что-то могло обрабатываться в другом порядке.
\begin{Rem}
	Мнение Димы: сортировка в представлении "--- фишка PostgreSQL.
\end{Rem}

Хотим самых клёвых.
Можно ещё раз посотрировать и отсечь, но... По-другому можно!
\begin{sqlcode}
SELECT *
	FROM CommanderStats CS1
	JOIN CommanderStats CS2
	ON CS1.planet_count < CS2.planet_count
		OR CS1.planet_count = CS2.planet_count AND CS1.pax_cnt < CS2.pax_cnt;
\end{sqlcode}
Победителя тут нет "--- нет никого его круче.
\begin{sqlcode}
SELECT
	*
	FROM CommanderStats CS1
	LEFT JOIN CommanderStats CS2
	ON CS1.planet_count < CS2.planet_count
	Or CS1.planet_count = CS2.planet_count AND CS1.pax_cnt < CS2.pax_cnt
	WHERE CS2.id IS NULL;
\end{sqlcode}
Остался только самый клёвый.
Можно и просто считать, сколько выше данного человека.

А если бы мы писали всё сразу в одно тело, то...
\begin{sqlcode}
-- 40 строк чистого удовольсвтвия!
\end{sqlcode}
Никак уже не догадаться, что это по делу.
И что это работает.

\section{Обощённые табличные выражения}
\begin{sqlcode}
WITH
	FlightStats AS (
		SELECT ...
	),
	CmdStats AS (
		...
	)
	CmdOrdering AS (
		...
	)
SELECT
	...;
\end{sqlcode}
Так хоть как-то понятно.

Теперь можно считать какой-то мегастатистический запрос.
Старым методом "--- трёхзначное число строк?
Новым же "--- читабельно, плюс можно магии добавить.
\begin{sqlcode}
WITH
	FlightStats AS (
	),
	FlightStatsExt AS (
	)
SELECT
	DISTINCT FSE.commander_name, FSE.planet_id,
	SUM(pax_cnt) OVER (PARTITION BY commander_id) AS total_pax_per_commander,
	...;
\end{sqlcode}
Мы тут написали оконную функцию "--- мы выделили часть строк и применили к ним.
Ещё и работать может быстрее.

Какую ещё магию?
\begin{sqlcode}
	ROW_NOMBER() OVER (ORDER BY)
\end{sqlcode}

\chapter{}

Транзакция "--- возможность группировки запросов, чтобы их эффект был атомарен: в случае ошибки мы получим БД в исходном состоянии, а не частично применённую.
Мы можем её открыть, начать что-то делать и таки решить сохранить. Если по дороге данные, над которыми были изменения, изменились, или новые данные нарушают ограничения.
Сегодня мы рассмотрим, как же транзакции взаимодействуют, и как они видят деятельность друг друга.

Начнём транзакцию:
\begin{sqlcode}
BEGIN; -- BEGIN TRANSACTION ...
INSERT INTO T(id, value) VALUES (1, 10);
COMMIT; -- сохранить изменения.

SELECT * FROM T; -- увидим строчку

BEGIN; -- BEGIN TRANSACTION ...
INSERT INTO T(id, value) VALUES (2, 10);
ROLLBACK; -- отменить
\end{sqlcode}

Отдельные запросы вне транзакций считаются исполняемыми сразу.

Вложенные транзакции умеют только high-tier СуБД.

\begin{sqlcode}
BEGIN;
INSERT INTO T(id, value) VALUES (3, 10);
SELECT * FROM T; -- мы свои изменения видим!
COMMIT;

CREATE TABLE S(id INT PRIMARY , value INT CHECK (value > 0));
BEGIN;
INSERT INTO S(1, 100); --ok
COMMIT; BEGIN;
INSERT INTO S(2, -50); --fail, Postgres заблокируется в состоянии ошибки, пока не введут ROLLBACK
ROLLBACK;
\end{sqlcode}
Если транзакция была начата, то изменения не исчезнут сами по себе.

Как же система ведёт себя при нескольких транзакциях на одних данных.
\begin{sqlcode}
-- 1
BEGIN;
UPDATE T SET value=1000 WHERE id = 1; -- ok

--2
BEGIN;
UPDATE T SET value=10000 WHERE id = 1; -- blocks

--1
UPDATE T SET value=2000 WHERE id = 2;
COMMIT;

--2 - SUCCESS
SELECT * FROM T; -- видит 10000 и 2000.
COMMIT;
\end{sqlcode}
Конкретно Postgres блокировать отдельные строчки, что мы и видим.
Программист может вообще выбирать, что блокировать.

Неприятность:
\begin{sqlcode}
--1
BEGIN;
SELECT * FROM T WHERE id = 1; -- 1000

--2
BEGIN;
SELECT * FROM T WHERE id = 1; -- 1000

--1 +=100
UPDATE T SET value = 1100 WHERE id = 1; -- ok

--2 +=500
UPDATE T SET value = 1500 WHERE id = 1; -- block

--1
COMMIT; --ok
--2
COMMIT; --ok

SELECT * FROM T; -- 1500 =(
\end{sqlcode}

Надо было
\begin{sqlcode}
UPDATE T SET value = value + 100 WHERE id = 1;
UPDATE T SET value = 1500 WHERE id = 1 AND value = 1000;
\end{sqlcode}
Можно ещё добавить столбик и писать туда, какой она версии, и проверять.
Такой руками написанный Spin-lock.

Какие же мы видем данные? Только закомиченные.
Эта гарантия при создании транзакции называется \sql'READ COMMITED'.
Ещё есть \sql'READ UNCOMMITED', некоторые его не дают (дают сразу сильнее).
Ещё есть куда сильнее \sql'REPEATABLE READ': вы всегда видите данные, как в начале транзакции.
\begin{sqlcode}
-- 1, 2
BEGIN ISOLATION LEVEL REPEATABLE READ;
SELECT * FROM T;

-- 1
UPDATE T SET value = 1100 WHERE id = 2;
-- 2 не видит
-- 1
COMMIT;
-- 2 всё ещё не видит!
ROLLBACK;

-- 1,2
BEGIN ISOLATION LEVEL REPEATABLE READ;
-- 1
UPDATE T SET value = 1200 WHERE id = 2;
-- 2
UPDATE T SET value = 1201 WHERE id = 2; -- blocks
-- 1
COMMIT;
-- 2 -- failure: overwridden!
ROLLBACK;
\end{sqlcode}
Что же будет, если попробуем применить два инкремента?
Нас тоже пошлют, что может быть обидно.
\begin{sqlcode}
-- 1
BEGIN;
UPDATE T SET value = value + 100 WHERE value < 100;
-- 2
BEGIN;
DELETE FROM T WHERE value < 100; -- такая есть, но блокируется
-- 1
COMMIT;
-- 2 -- deleted 0.
COMMIT;

-- 1
BEGIN;
UPDATE T SET value = value + 100 WHERE value BETWEEN 200 AND 299;
-- 2
BEGIN;
DELETE FROM T WHERE value BETWEEN 200 AND 299; -- такая есть, но блокируется
-- 1
COMMIT;
-- 2 -- deleted 0.
COMMIT;

-- 1
BEGIN;
INSERT INTO T(id, value) VALUES (5, 9), (6, 10);
UPDATE T SET value = value + 1 WHERE value < 100;
-- 2
BEGIN;
DELETE FROM T WHERE value = 10; -- blocks
-- 1
COMMIT;
-- 2 -- deleted 0 ?!?!?!
COMMIT;
\end{sqlcode}
Что произошло? Вторая транзакция стала ждать какие-то строчки, дождалась доступа, перепроверила... А они не подходят.
\sql'REPEATABLE READ' упадёт, естественно.

Ещё есть уровень \sql'SERIALIZABLE', посильнее.
Он гарантирует, что какая-то последовательность транзакций будет корректно произведена, как будто они друг другу не мешают.

Мы будем писать на Java. Код будет

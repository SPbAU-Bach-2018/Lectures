\chapter{Занятие 10 ноября}

Загадка. Простой запрос. Что-то поменяли, но не в БД или в запросе. Стало работать куда медленне "--- одновременно запросили 11 вкладок.
Ответ "--- размер пула подключений.
Когда сделали меньше, всем приходится тормозить. Когда больше "--- шустренько, но от сильного увеличения уже не сильно, так как БД тоже надо работать больше.

Проанализируем простой запрос с помощью \sql'EXPLAIN ANALYSE':
\begin{sqlcode}
SELECT
	id, (SELECT COUNT(*) FROM Booking WHERE flight_id = id) AS cnt
	FROM Flight;
\end{sqlcode}
Увидим, что всё в лоб: подзапрос на каждую строку.
Если же перепишем через `JOIN`:
\begin{sqlcode}
EXPLAIN ANALYZE SELECT
	id, COUNT(*)
	FROM Flight
	JOIN Booking ON Booking.flight_id = Flight.id;
\end{sqlcode}
То увидим, что сканируются обе таблицы всего один раз.

\begin{sqlcode}
SELECT *
	FROM Flight
	WHERE (SELECT COUNT(*) FROM Booking WHERE flight_id = id) > 2;
\end{sqlcode}
Долга!
\begin{sqlcode}
SELECT *
	FROM Flight
	WHERE id IN (
		SELECT flight_id
		FROM Booking
		GROUP BY Flight.id
		HAVING COUNT(*) > 2);
SELECT Flight.*
	FROM Flight
	JOIN Booking ON Booking.flight_id = Flight.id
	GROUP BY Flight.id
	HAVING COUNT(*) > 2;
\end{sqlcode}
Всё из-за коррелирующих подзапросов "--- позапросов, обращающихся к полям внешнего запроса.

Структурное программирование предлагает такое:
\begin{sqlcode}
SELECT *
	FROM Flight
	WHERE GetBookedCount(id) > 2;
\end{sqlcode}
Клёво выглядит, тока не работает быстро.

\begin{sqlcode}
SELECT *
	FROM Booking
	WHERE flight_id = 2;
\end{sqlcode}
Просто скан и фильтр.
\begin{sqlcode}
SELECT *
	FROM Flight 
	WHERE id = 2;
\end{sqlcode}
Скан индекса, куда быстрее!
К сожалению, индекса для внешнего ключа не создано.
\begin{sqlcode}
CREATE INDEX ON Booking(flight_id);
\end{sqlcode}
Это крайне короткий синтаксис. \sql'\d Booking' покажет его.

А когда не поможет?
\begin{sqlcode}
SELECT
	id
	FROM Flight
	JOIN Commander ON ...
	WHERE Commander.rating = 'poor';
\end{sqlcode}
Скан капитанов, потом фильтр, потом хеш-таблица по полётам.
Индекс по рейтингу тут не помогает. Почему? Попробуем заставить.
\begin{sqlcode}
SET enable_seqscan=OFF;
\end{sqlcode}
Стал выполняться больше, толку мало.
Капитанов-то у нас 10, всё равно всех читать.

Теперь считаем количество полётов с троллями.
\begin{sqlcode}
SELECT
	COUNT(DISTINCT Flight.id)
	FROM Flight
	JOIN Booking
	JOIN Pax
	WHERE Pax.race = 'troll';
\end{sqlcode}
Почему сейчас не заиспользовался? 1000 пассажиров "--- довольно много.
\begin{sqlcode}
SELECT
	COUNT(DISTINCT Flight.id)
	FROM Flight
	JOIN Booking
	JOIN Pax
	WHERE Pax.id = 1;
\end{sqlcode}
Тут индекс заюзали, и сразу кстати хеш-джойны заменился на вложенный цикл...
Проблема в том, что рас мало. Количество записей довольно велико, и всё равно прочитаем почти всё.

Вернёмся к 
\begin{sqlcode}
SELECT
	id
	FROM Flight
	JOIN Commander ON ...
	WHERE Commander.rating = 'poor';
\end{sqlcode}
Попробуем индекс для \sql'Flight.commander_id'.
Что стало? Индекс заюзался!

Если делается какой-то сложный запрос, то можно его обернуть в материализованное представление "--- его результат хранится на диске и поддерживается, но это ваш последний помощник, тем более, если есть изменения.

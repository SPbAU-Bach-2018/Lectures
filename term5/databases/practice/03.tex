\chapter{Занятие 3}

Одним из способов проектирования БД является изображение ER-диаграм.

%рис1 (старинная схема, где все поля в овальчиках)
Потом аннотация изменилось, так как выяснилось, что на монитре это выглядит плохо.
Поэтому это трансформировалось во что-то такоу %рис 2 классическая схема с полями

Мы тут можем связывать объекты.
Cвязь 1 к 1: скорее всего, кто-то живёт дольше, пусть на него ссылаются.
Связи много к 1: указатель.
Связи 1 к много: какой-то список.
Связи много к много... Что-то непонятное)
Например, пассажиры и самолёты "--- на одном самолёте летают много пассажиров, пассажир летает много где.

\section{Внешние ключи}

Создаём как-то так:
\begin{sqlcode}
CREATE TABLE Student (
	id INT PRIMARY KEY,
	name TEXT
);
CRATE TABLE Payments (
	studentId INT FOREIGN KEY REFERENCES Student, -- по умолчанию первичный ключ
	payment FLOAT
);
\end{sqlcode}

Если вы ссылаетесь не на первичный ключ "--- можно писать после имени таблицы в скобках столбец.
Если хотите писать составной внешний ключ "--- пишутся внизу:
\begin{sqlcode}
	FOREIGN KEY (attr1, attr2) REFERENCES Table(attr1, attr2)
\end{sqlcode}
Ссылаться всегда должны на потенциальный ключ.

\subsection{Связь многие ко многим}

Просто создаём таблицу из двух внешних ключей на первичные ключи тех и тех.
Обычно её имя "--- вместе имена тех сущностей, что мы связываем.

\begin{sqlcode}
CREATE TABLE Person (...);
CREATE TABLE Airjet(...);

CREATE TABLE PersonAirjet (
	personId INT FOREIGN KEY REFERENCES Person,
	airjetId INT FOREIGN KEY REFERENCES Airjet,
	-- место и прочее, если хочется,
	PRIMARY KEY(personId, airjetIdId)
);
\end{sqlcode}

\subsection{Cвязь многие к 1}

Многие планеты могут иметь один строй. Просто создаём внешний ключ из планеты в строй.

Можно создать, аналогично предыдущему случаю, отдельную таблицу, но ключ сделать только из тех, из кого <<много>>
(так как для них должна быть только одна ссылка!).

\subsection{Связь 1 к 1}

Можно объявить просто внешний ключ ключом.

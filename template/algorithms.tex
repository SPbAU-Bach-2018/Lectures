\documentclass[12pt]{article}

% Автор: Сергей Копелиович

\usepackage[T2A]{fontenc}
\usepackage[utf8]{inputenc}
\usepackage[russian]{babel}
\usepackage{graphicx}
\usepackage{amsthm,amsmath,amssymb}
\usepackage[russian]{hyperref}
\usepackage{enumerate}
\usepackage{datetime}
\usepackage{minted}
\usepackage{fancyhdr}
\usepackage{lastpage}

\sloppy
\voffset=-20mm
\textheight=235mm
\hoffset=-25mm
\textwidth=180mm
\headsep=12pt
\footskip=20pt

% Основные математические символы
\def\EPS{\varepsilon}    %
\def\SO{\Rightarrow}     % =>
\def\EQ{\Leftrightarrow} % <=>
\def\t{\texttt}          %
\def\O{\mathcal{O}}      %
\def\NO{\t{\#}}          % #
\newcommand{\q}[1]{\langle #1 \rangle}               % <x>
\newcommand\URL[1]{{\footnotesize{\url{#1}}}}        %
\newcommand{\sfrac}[2]{{\scriptstyle\frac{#1}{#2}}}  % Очень маленькая дробь
\newcommand{\mfrac}[2]{{\textstyle\frac{#1}{#2}}}    % Небольшая дробь

% Отступы
\def\makeparindent{\hspace*{\parindent}}
\def\up{\vspace*{-\baselineskip}}
\def\down{\vspace*{\baselineskip}}
\def\LINE{\vspace*{-1em}\noindent \underline{\hbox to 1\textwidth{{ } \hfil{ } \hfil{ } }}}

% Мелкий заголовок
\newcommand{\THE}[1]{
  \vspace*{0.5em}
  {\bf \underline{#1}}%\hspace{0.5em}
  \vspace*{0.2em}
}
% Другой тип мелкого заголовка
\newcommand{\THEE}[1]{
  \vspace*{0.5em} $\bullet$
  {\bf #1}%\hspace{0.5em}
  \vspace*{0.2em}
}

% Код с правильными отступами
\newenvironment{code}{
  \VerbatimEnvironment

  \vspace{-0.8em}
  \begin{minted}{c}}{
  \end{minted}
  \vspace{-0.8em}

}

% Формула с правильными отступами
\newenvironment{formula}{
 
  \vspace{-1.0em}
}{
  \vspace{-1.7em}
  
}

% Определяем основные секции: \begin{LM}, \begin{THM}, \begin{DEF}
\renewcommand{\qedsymbol}{$\blacksquare$}

\theoremstyle{definition} % жирный заголовок, плоский текст
\newtheorem{Lm}{Lemma}[subsection]
\newtheorem{Thm}{Theorem}[subsection]

\theoremstyle{plain} % жирный заголовок, курсивный текст
\newtheorem{Def}{Def}[subsection]

\theoremstyle{remark} % курсивный заголовок, плоский текст
\newtheorem{Rem}[Thm]{Следствие} % Нумерация такая же, как и у теорем

\newcommand{\Lmautorefname}{Lm}
\newcommand{\Thmautorefname}{Thm}
\newcommand{\Defautorefname}{Def}

% Определяем ЗАГОЛОВКИ
\def\SectionName{unknown}
\def\AuthorName{unknown}

\newlength{\sectionvskip}
\setlength{\sectionvskip}{0.5em}
\newcommand{\Section}[3]{
  % Заголовок
  \pagebreak
  \refstepcounter{section}
  \vspace{0.5em}
  \addcontentsline{toc}{section}{\arabic{section}. #1}
  \begin{center}
    {\Large \bf Лекция по алгоритмам \NO{\arabic{section}}} \\ 
    \vspace{\sectionvskip}
    {\Large \bf #1} \\
    \vspace{\sectionvskip}
    {\large #2} \\
  \end{center}

  \LINE

  % Запомнили название и автора главы
  \gdef\SectionName{#1}
  \gdef\AuthorName{#3}

  % Заголовок страницы
  \lhead{Алгоритмы, весна 2014/15}
  \chead{}
  \rhead{\SectionName}
  \renewcommand{\headrulewidth}{0.4pt}

  \lfoot{Глава \NO{\arabic{section}}}
  \cfoot{\thepage/\pageref*{LastPage}}
  \rfoot{Автор конспекта: \AuthorName}
  \renewcommand{\footrulewidth}{0.4pt}
}
\newcommand{\Subsection}[1]{
  \subsection{#1}
  \vspace*{-1.5em}
  \makeparindent\unskip
}

\newcommand{\Header}{
  \pagestyle{empty}

  \begin{center}
    {\Large \bf Первый курс, весенний семестр} \\
    \vspace{\sectionvskip} 
    {\Large \bf Конспект по алгоритмам} \\
    \vspace{\sectionvskip}
    {Собрано {\today} в {\currenttime}}
  \end{center}

  \LINE

  \vspace{1em}
  \tableofcontents
  \pagebreak
}

\newcommand{\BeginConspect}{
  \pagestyle{fancy}
  \setcounter{page}{1}
}


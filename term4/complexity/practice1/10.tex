\chapter{Занятие 15.04.2016}

\section{Разбор задач}
\subproblem{49}{a}
	Разбирала Лиза Третьякова.

	Пусть в одной доле графа $r$ вершин, а в другой "--- $s$.
	Перенумеруем вершины так, чтобы вершины первой доли шли в начале.
	Рассмотрим матрицу смежности нашего двудольного графа, она имеет следующий вид:
	\[
		\begin{vmatrix}
			0_r & B \\
			B^{\top} & 0_s
		\end{vmatrix}
	\]
	Если $r\neq s$, то совершенный паросочетаний точно нет.
	А если $r=s$, то матрица $B$ квадратная.
	Проведём биекцию между выборками перманента в $B$ и совершенными паросочетаниями.
	\TODO

\subproblem{49}{б}
	Разбирала Лиза Третьякова.

	Заметим, что определитель "--- это такая же сумма, как и перманент, но некоторые слагаемые
	идут со знаком минус, а не плюс, на чётность ответа это не влияет.
	Тогда посчитаем определитель матрицы в поле $\mathrm{F}_2$, получим чётность числа полных паросочетаний.

\subproblem{50}{а}
	Разбирал Петя Смирнов.

	Будем считать, что протокол $\AMA$ на строке $x$ работает так:
	\begin{enumerate}
		\item
			Сначала Артур генерирует случайную строку $s_1$ фиксированной длины $p_1(|x|)$
		\item
			Потом Мерлин даёт подсказку $w(s_1)$
		\item
			Потом Артур генерирует случайную строку $s_2$ фиксированной длины $p_2(|x|)$
		\item
			Потом Артур выдаёт вердикт $A(x, s_1, w(s_1), s_2)$
	\end{enumerate}
	Мы знаем, что в таком протоколе Артур выдаёт правильный ответ с вероятностью хотя бы $\sfrac 23$
	(вероятность берётся по всем строкам $s_1$, $s_2$).

	Рисуем дерево протокола: корень, из него рёбра с $s_1$, из этих вершин "--- рёбра с $w$,
	из них "--- рёбра с $s_2$.
	В листах мы пишем 0/1 (в зависимости от вердикта $A$), потом в вершинах-ходах-Артура пишем
	среднее (потому что берём вероятность), в вершинах-ходах-Мерлина берём максимум (потому что
	Мерлин хочет максимизировать вероятность ответа <<да>>).

	Либо считаем не среднее, а сумму, а в конце делим на $2^{|s_1|+|s_2|}$.

	\begin{Rem}
		То, что мы берём дополнительные случайные биты из генератора случайных бит, генератор никак не портит.
	\end{Rem}

\subproblem{50}{б}
	Разбирал Петя Смирнов.

	Опять рисуем дерево игры.
	Мерлин должен понять, какую строчку ему выдавать по первому ходу Артура $t$.
	Зафиксировали $t=h(s_1)$ (тут $h$ "--- первый ход Артура), потом перебираем
	все допустимые $s_1$ (для которых $t=h(s_1)$), потом вообще все $s_2$ и считаем вероятность того,
	что Артур согласится (вероятность по допустимым $s_1$ и всем $s_2$).

	Либо считаем не вероятность, а сумму, а в конце делим сумму по всем $t$-шкам на $2^{|s_1|+|s_2|}$.

	\TODO см. фотографии доски для формализации.

	\begin{Rem}
		В общем случае удобно перебирать сначала то, что посылает Артур Мерлину, потом
		подсказки Мерлина, потом случайные биты.
	\end{Rem}

\problem{45}[набросок]
	Разбирала Оля Черникова.

	\begin{Rem}
		Решение методом $\IP=\PSPACE$ не засчитывается "--- этот факт стал извествен после того, как была дана задача.
	\end{Rem}

	На формальное определение $\IP$ мы забьём, будем описывать протокол словами.
	Вход задачи "--- $y$, мы хотим проверить существование $\sqrt y$ (это условное обозначение,
	соответствующее какому-нибудь корню из $y$).

	Предположим, что $m$ простое.
	Verifier действует так: выбирает случайное число $z$ и еще один случайный бит.
	Если бит "--- нолик, дадим prover'у число $z^2 \bmod m$, иначе дадим ему число $yz^2 \bmod m$.
	Попросим prover сказать, является ли число квадратичным вычетом.

	Формально не закончили.
	Формально могло бы звучать так: алгоритм у нас будет однораундовый.
	Если поймали за руку "--- говорим <<нет>>.
	Оценим вероятность того, что мы ошибёмся, отдельно в двух случаях: $x \in L$ и $x \notin L$.
	Мы посылаем пруверу сэмпл из случайного из двух распределений.
	В случае $x \in L$ они одинаковые, значит, прувер всегда выдаёт единицу с фиксированной вероятностью $p$.
	Ну и хорошо оценивается тогда вероятность ошибки в этом случае.
\problem{45}
	Сейчас будет вероятное решение от меня(Оли Черниковой), которое не было рассказано в классе,
	поэтому не проверенное и может содержать ошибки. Так как тетрадку я уже выкинула, поскольку решила, что 
	все решение и так достаточно хорошо есть в конспекте и мне моя тетрадка не нужна, поэтому попытаюсь востановить по памяти. 

	И так. У нас есть число m, который модуль. Задача разложения на простые числа принадлежит классу NP, поэтому разложить 
	на простые множетели мы справимся. Разложим. $m = p_1^{\alpha_1}p_2^{\alpha_2}\cdots p_n^{\alpha_n}$.

	Теперь, надо доказать, что можем решать по каждому $p_i^{\alpha_i}$ независимо. 
	\begin{lemma}
		Число квадратное по модулю m $\Lra$ число квадратное по каждому $p_i^{\alpha_i}$.
	\end{lemma}
	\begin{proof}
	        \begin{description}
	        \item $\La$
	        	Хотим показать, что если число квадратное по модулю 
	        	m, то оно квадратное по всем соответсвющим модулям. 

	        	Есть $y_1$, такой, что $y_1 \cdot y_1 = y(m)$. Но тогда, 
	        	возьмем тот же $y_1 \% (p_i^{\alpha_i})$ и в квадрате он 
	        	так же даст $y \% (p_i^{\alpha_i})$.
	        \item $\Ra$
	        	Здесь придется воспользоваться К.Т.О. 

	        	Условно, у нас есть $y_1, y_2, \cdots, y_n$, которые корни для 
	        	каждого конкретного модуля. Модули между собой взаимопросты. 

	        	По КТО получаем, что существует и единственно $w$, которое 
	        	дает соответствующие остатки по каждому из таких модулей, а само
	        	по модулю $m$. Теперь надо понять, что
	        	это число в квадрате дает как раз таки y.

	              	Рассмотрим число $w \cdot w$. По 
	              	каждому из простых модулей, это число такое же как $y$,
	              	значит это число как раз ровно $y$ само по себе, потому что
	              	КТО гарантирует единственность такого числа при данных остатках от 
	              	простых частей модуля. 
	        \end{description}
	\end{proof}

	Теперь нужно разобрать несколько случаев. 
	\begin{enumerate}
	\item $gcd(y, p_i^{\alpha_i}) = 1$
		Тогда мы хотим решать задачу по протокуолу, который уже был описан. 
		Там еще с количеством действий в целом, надо мучиться, но тут я в вас верю, что дожимается. 
	\item $gcd(y, p_i^{\alpha_i}) = p^{k}$. 

		Тогда утвержается, что число квадратное тогда и только тогда, когда y квадратное 
		само по себе. Что число квадрат не по модулю мы проверяем просто за полином. Тут проблем вообще нет.

		Окай. Пусть $gcd(y, p_i^{\alpha_i}) = d$. Значит $y = dc$, где c взаимопросто с $p_i$. Если
		$y \% (p_i^{\alpha_i}) = 0$, то это какой-то совсем клинический случай, естественно
		тогда число квадратное, по данному модулю.

		Теперь, что мы можем понять, про это счастье. Вот есть какое-то число $w$, такое что $w \cdot w = dc$.

		$w = p_i^{e}k$, то есть все это можно разбить на две незавсисимые части. То есть
		здесь надо все-такие проверять, что $y/d$ квадратное, по модулю $p_i^{\alpha_i}/d$. Но
		эти числа у нас взаимопростые и это мы проверять умеем.

		Остались части $w' = p_i^{e}$ и $d = w' \cdot w' = p_i^{e} \cdot p_i^{e} = p_i^{2e}$, это по
		модулю $p_i^{\alpha_i}$, то есть это либо 0, либо точный квадрат. Но ноль мы уже отдельно ранее разобрали.   
 
	\end{enumerate}
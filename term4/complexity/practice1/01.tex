\chapter{Занятие 10.02.2016}
\section{Введение}

\begin{Def}
	Множество бинарных слов $\{0, 1\}^*$ "--- все слова из нулей и единиц конечной длины.
\end{Def}

\begin{Def}
	Некоторое множество слов $L \subset \{0, 1\}^*$ называется \textit{языком}.
\end{Def}

\begin{Def}
	Язык $L$ разрешим алгоритмом $A$, если $A(x)=L(x)$ (тут $L$ "--- характеристическая функция для $L$),
	т.е. $A(x) = 1 \iff x \in L$.
\end{Def}

В этом семестре нас будут интересовать только эффективные системы доказательств:
дополнительное свойство для алгоритма "--- он работает за полиномиальное время
от суммарного размера входа (в частности, учитывается и размер утверждения, и размер доказательства).

\begin{Def}
	Система доказательств для языка $L$ "--- это полиномиальный 
	алгоритм $\Pi(x, w)$, возвращающий элементы $\{0, 1\}$, причём:
	\begin{itemize}
		\item $x    \in L \Ra \exists \omega \colon \Pi(x, \omega) = 1$
		\item $x \notin L \Ra \forall \omega \colon \Pi(x, \omega) = 0$
	\end{itemize}
\end{Def}
\begin{Rem}
	Отличие от прошлого семестра: тут мы требуем полиномиальность алгоритма.
	Полиномиальность от суммарного размера входа (т.е. и размера $x$, и размера $w$).
\end{Rem}

\begin{Def}
	Пусть $\Pi'(\omega)$ "--- полиномиальный алгоритм, причём отображение
	$\Pi'(\omega) \to x$ на мн-во $L$ является сюръективным.
\end{Def}

С пустыми языками мы работать не хотим "--- в последнем определении возникнут проблемы.

\begin{Rem}
	Эти определения, оказываются, эквивалентны в следующем смысле (неформально):
	если имеется один алгоритм, то его можно переделать в другой за полиномиальное время.
	Без доказательства.
\end{Rem}

\section{Корректировки задач}
\subsection{Задача 2}
	Передалали задачу 2, теперь в ней два пункта:
	\begin{enumerate}
		\item Матрица состоит только из нулей и единиц, кольцо "--- $\Z$.
		\item
			Матрица состоит из произвольных целых чисел, каждое число имеет размер $\mathcal O(n^{0.5})$,
			проверяющий алгоритм должен работать за $\mathcal O(n^2)$.
	\end{enumerate}
	Предполагается, что мы можем считывать произвольный элемент матрицы, все считывать необязательно.

\subsection{Задача 3}
	Аналогично, все элементы матрицы смежности считывать необязательно.
	Стоит заметить, что число из $[1\dots V]$ имеет размер $\mathcal O(\log V)$.

	Исправление: можно проверять доказательство за $\mathcal O(V \log V)$.
	Если в течение недели Дмитрий Олегович вспомнит решение за честное $\mathcal O(V)$, будет создана еще одна задача.

\subsection{5}
	В задаче 5 подразумевается язык выполнимых формул \textit{в КНФ}.

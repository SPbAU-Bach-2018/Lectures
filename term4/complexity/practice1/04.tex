\chapter{Занятие 02.03.2016}

К сожалению, Егора на занятии не было, поэтому записаны только разборы задач
со слов присутствующих.

\section{Разбор задач}
\subsection{Задача 16}\label{prob16}
	На занятии разбирал Игорь Лабутин.
	Записано со слов Пети Смирнова и записей Лизы Третьяковой.

	\begin{assertion}
		$f(n)=n^{\log n}=e^{\log^2 n}$ асимптотически больше любого полинома ограниченной степени
		и асимптотически меньше любой экспоненты.
	\end{assertion}
	Давайте построим язык $L \in EXP$.
	Строим машину Тьюринга, которая разрешает слово $n$ за экспоненциальное время:
	\begin{enumerate}
		\item
			Запускает универсальную машину Тьюринга на алгоритме с номером $n$ на входе $n$ в течение $f(|n|)$ шагов.
			Под алгоритмом подразумеваем машину Тьюринга, которая отвечает <<да/нет>> (либо принимает вход, либо не принимает).
		\item
			На предыдущем шаге мы узнали значение $x \coloneq \langle r \rangle(n)$.
			Если $x$ не определён (т.е. машина Тьюринга не завершилась), то считаем $n \notin L$.
		\item
			Если $x=0$, то считаем $n \in L$ (выдавая 1), иначе считаем $n \notin L$ (выдавая 0).
	\end{enumerate}
	Очевидно, что эта машина разрешает $L$ за не более чем экспоненциальное время.
	От противного покажем, что не существует машины $M$, которая бы разрешала $L$ за полиномиальное время.
	В самом деле, пусть машина $M$ разрешает $L$, причём тратит на это не более $P(|n|)$ шагов,
	где $P$ "--- полином фиксированной степени.
	Заметим, что у этой машины бесконечное число номеров: если мы допишем в машину лишнее недостижимое состояние,
	её свойства не поменяются.
	Пусть это номера $a_1, a_2, \dots$.
	Посмотрим на числа $L(a_1), L(a_2), \dots$.
	Заметим, что начиная с некоторого места при подсчёте соответствующего числа у нас внутренняя машина Тьюринга ($M$)
	обязательно завершила выполнение, так как $P(|n|)=o(f(|n|))$.
	Стало быть, начиная с некоторого места $L(a_i)=\lnot M(a_i)$ по построению языка $L$.
	С другой стороны, $L(a_i)=M(a_i)$, так как $M$ разрешает язык $L$.
	Противоречие.

\subsection{Задача 17}
	На занятии разбирал Никита Подгузов.
	\TODO

\subsection{Задача 18}
	На занятии разбирала Лиза Третьякова.
	\TODO

\subsection{Задача 19}
	На занятии разбирал Петя Смирнов.
	\TODO

\subsection{Задача 22}
	Разобрана на лекции.

\chapter{Занятие 23.03.2016}

\section{Разбор задач}
\problem{31}
	\subsubsection{От Егора}
		Разбирал Егор Суворов.

		Сначала избавимся от изолированных вершин: теперь в каждой компоненте связности есть хотя бы
		одно ребро (это потребуется потом).
		Давайте переформулируем задачу: мы знаем, что граф $G$ из $c$ компонент связности не содержит циклов
		$\iff$ граф $G$ является лесом
		$\iff$ любая компонента связности из $n_i$ вершин содержит ровно $n_i-1$ рёбер
		$\iff$ $|V|-|E|\ge c$.
		Последний переход стоит пояснить:
		\begin{description}
			\item[$\Ra$:]
				В любой компоненте связности хотя бы $n_i-1$ ребро, стало быть,
				всего рёбер хотя бы $\sum n_i-1 = |V|-c$.
				Значит, $|E|\ge|V|-c \iff |V|-|E|\le c$ в произвольном графе.
				Однако если в каждой компоненте ровно $n_i-1$ рёбер, то получаем равенство.
			\item[$\La$:]
				Так как $|V|-|E|\le c$ в произвольном графе, то на самом деле $|V|-|E|=c$.
				Значит, в каждой компоненте ровно $n_i-1$ ребро.
		\end{description}

		Теперь давайте посчитаем некоторую величину $c'$ такую, что $c' \ge c$ в произвольном графе,
		а в случае леса $c'=c$ (это такая оценка числа компонент связности сверху).
		Тогда $G$ является лесом~$\iff$~$|V|-|E| = c'$:
		\begin{description}
			\item[$\Ra$:]
				Если $G$ "--- лес, то $|V|-|E|=c'=c$
			\item[$\La$:]
				От противного: если $G$ "--- не лес, то $|V|-|E| < c \le c'$, следовательно,
				$|V|-|E| < c'$.
		\end{description}

		Теперь, собственно, определяем величину $c'$ и учимся считать её с логарифмической памятью.
		Давайте каждое ребро в $G$ заменим на два ориентированных.
		В каждой вершине зафиксируем порядок исходящих из неё рёбер.
		Для каждого ориентированного ребра $e=(a, b)$ определим следующее за ним ребро так:
		находим в списке смежности вершины $b$ ребро $(b, a)$ и берём следующее за ним ребро $(b, x)$
		(возможно, $x=a$, если вершина $b$ была листом).
		Аналогично определяем предыдущее ребро так, чтобы понятия <<следующее>> и <<предыдущее>>
		были взаимно обратны.
		\begin{Rem}
			Мы сейчас получили что-то очень похожее на алгоритм обхода граней в планарном графе.
		\end{Rem}
		Заметим, что теперь, если переходить с какого-то ребра к следующему, мы в какой-то
		момент обязательно прийдём в исходное.
		Таким образом, ориентированные рёбра разбились на сколько-то циклов, их количество и назовём $c'$.
		Сначала покажем, что это действительно то, что нам надо:
		\begin{itemize}
			\item
				Если одно ребро следует за другим, то они инцидентны $\Ra$
				все рёбра в одном цикле лежат в одной компоненте связности.
			\item
				Так как в каждой компоненте связности есть хотя бы одно ребро (изолированные
				вершины выкинули), то каждой компоненте связности соответствует хотя бы один цикл.
				Следовательно, $c' \ge c$.
			\item
				Если компонента связности является деревом, то из какого бы ребра мы не начали,
				у нас получится Эйлеров обход дерева, покрывающий все рёбра компоненты.
				Стало быть, если $G$ "--- лес, то $c' = c$, что и требовалось.
		\end{itemize}

		Осталось лишь научиться считать $c'$ за логарифмическую память.
		Это просто: мы посчитаем число ориентированных рёбер, являющихся минимальными (по номеру) в своём цикле.
		Перебираем ребро $i$, проходим по его циклу (храня только номер текущего ребра), пока не придём обратно в $i$,
		и храним бит "--- верно ли, что нам еще не встретилось ребра с номером, меньшим $i$.
		Если это так, увеличиваем $c'$ на единицу.

	\subsubsection{От семинариста}
		Идея решения: можно показать, что в $G$ есть цикл $\iff$
		можно начать такой же обход из какой-то вершины $a$ по ребру $(a,b)$ и вернуться в эту же вершину по другому ребру.
		Более детально не разбиралось.

\problem{24}
	Разбирал Игорь Лабутин.

	\subsubsection{а}
		Задача лежит в $\NL$, так как мы можем взять в качестве подсказки
		$n^2$ путей (в фиксированном порядке) от каждой вершины до каждой,
		а потом проверить, что каждый путь "--- действительно путь из нужной вершины
		в нужную.

	\subsubsection{б}
		Для начала научимся сводить задачу наличия пути в ографе к
		задаче о сильной связности за логарифмическую память.
		Пусть есть граф $G$ и мы хотим проверить, есть ли путь из $s$ в $t$.
		Для этого добавим для всех остальных вершин $v$ рёбра $v \to s$ и $t \to v$,
		получим граф $G'$.
		Докажем, что <<в $G$ есть путь>> $\iff$ <<$G'$ сильно связен>>:
		\begin{description}
			\item[$\Ra$:]
				Очевидно: взяли произвольные вершины $a$ и $b$ из $G'$,
				тогда из $a$ есть ребро в $s$, из $s$ в $t$ был путь даже в
				$G$ (и тем более есть в $G'$), а из $t$ в $b$ есть ребро.
			\item[$\La$:]
				Так как $G'$ сильно связен, то есть путь из $s$ в $t$,
				возьмём кратчайший такой путь.
				Заметим, что в нём не могут присутствовать рёбра $t \to v$,
				так как они бессмысленны "--- мы уже пришли в $t$.
				Аналогично не могут присутствовать рёбра $v \to s$ "---
				если такое есть, то в пути есть цикл и он не кратчайший.
				Стало быть, в $G'$ есть путь $s \to t$, не проходящий по свежедобавленным рёбрам,
				но тогда он есть и в $G$.
		\end{description}

		Теперь исходный пункт: возьмём произвольную задачу $A$ из $\NL$.
		Построим алгоритм, который по входу $x$ для задачи $A$ строит
		граф, сильная связность которого эквивалентна решению для $x$.
		Пусть алгоритму подали на вход $x$.
		Мы знаем, что задача $A$ лежит в $\NL$, то есть имеется недетерминированная
		МТ $M$ с логарифмической памятью, которая решает задачу.
		Построим граф конфигураций $M$ на входе $x$ (конфигурация "--- это внутреннее
		состояние МТ, позиция на ленте и логарифмическая память МТ), каждая
		конфигурация кодируется $\O(\log n)$ битами.
		Давайте выпишем граф конфигураций следующим образом: перебираем две пары конфигураций,
		а потом понимаем, можно ли из одной в другую перейти за один шаг МТ.
		Если да "--- выписываем в ответ соответствующее ребро.

		Теперь у нас такая задача: понять, есть ли в этом графе конфигураций хотя бы один
		путь из стартовой конфигурации $s$ в хотя бы одну из конечных $t_0, t_1, \dots$.
		Добавим фиктивную конфигурацию $t$ и проведём рёбра $t_i \to t$, теперь исходная
		задача равносильна существованию пути $s \to t$.
		А теперь сведём задачу существования пути к задаче сильной связности (см. выше),
		что и требовалось.

		\begin{Rem}
			Мы тут несколько раз воспользовались композицией МТ с логарифмической памятью
			и неограниченным выводом.
		\end{Rem}

\subsection{Задача 26 (не разобрана)}
	Разбирал Дима Розплохас.

	Сначала забудем про условие $L \in \EXP$ и покажем, что просто есть сложные для схем языки.
	Зафиксируем длину входа $n$.
	Схема задаёт какую-то функцию $\{0, 1\}^n \to \{0, 1\}$, всего таких функций $2^{2^n}$,
	получаем, что языков на словах длины $n$ хотя бы столько.
	Оценим количество схем из не более чем $s$ гейтов.
	Для этого оценим число схем из ровно $s$ гейтов, но введём дополнительный тип гейта "--- <<не используется>>
	(ноль входов, выводит какую-нибудь константу).
	\begin{Rem}
		Мы по определнию схем считаем, что у нас каждый гейт имеет не более двух входов.
		Год назад на другом курсе мы считали, что у нас базис из гейтов произвольный и каждый гейт
		из базиса имеет произвольное конечное число входов, тогда размер схем в разных базисах
		меняется лишь в константу раз.
		К сожалению, тут нам важны константы и выбор конкретного базиса.
		Например, для любой константы в условии можно подобрать очень крутой базис, в котором
		утверждение будет неправдой (без доказательства).
	\end{Rem}
	Итак, в каждом гейте (в том числе входном) не более двух входов, значит, всего типов гейтов
	(с учётом входов и выходов) не более $2^{2^2}$ на константу, всяко не более 1024.
	И еще мы должны выбрать для гейта, какие два других гейта для него входные
	(если меньше входов, то посчитаем больше, ничего страшного).
	Значит, всего у нас схем размера не больше $s$ не больше $(1024s^2)^s$.
	Теперь подставим $s=\frac{2^n}{10n}$ и покажем, что для некоторого $n$ схем сильно меньше, чем функций:
	\begin{gather*}
		(1024s^2)^s < 2^{2^n} \\
		s \log 1024s^2 < 2^n \\
		s (2\log s + 10) < 2^n \\
		\frac{2^n}{10n} (2(n-\log 10n) + 10) < 2^n \\
		2n-2\log 10n + 10 < 10n \\
		2n + 10 < 10n \\
		10 < 8n
	\end{gather*}
	Дальше, к сожалению, полностью корректного решения не получилось.
	Но идея была такая: 
	давайте покажем, что есть такой сложный язык, но даже лежащий в $\EXP$.
	Возьмём для каждой длины $n$ лексикографически наименьшее подмножество слов длины $n$,
	которое не распознаётся небольшими схемами, объединим все такие подмножества, получим язык $L$.
	У него схемная сложность большая.
	Дальше хочется показать, что он лежит в EXP.
	Давайте запишем этот язык формально через кванторы и формулы.
	\begin{Rem}
		Напоминание: мы говорили, что язык разрешается предикатными формулами, если
		для каждого размера входного слова $k$ есть формула на $k$ битовых переменных,
		которая истинна тогда и только тогда, когда слово принадлежит языку.
		\TODO Егор не очень понимает, как это соотносится с определениями с лекций.
	\end{Rem}
	\TODO что дальше происходит?

	Получаем примерно следующую конструкцию:
	\begin{itemize}
		\item Слово $y$ длины $n$ принадлежит $L$ тогда и только тогда, когда:
		\item Есть характеристическая функция $f$ языка (причём $f(y)=1$) такая, что:
		\item Любая схема размера размера не больше $\frac{2^n}{10n}$ где-то ошибается на этой функции, и,
		\item Нет лексикографической меньшей функции $f'$ с тем же свойством.
	\end{itemize}
	Под каждым квантором у нас стоит что-то экспоненциальной длины относительно $n$ 
	(например, $\exists f$ на самом деле говорит, что существует таблица истинности из $2^n$ битов).
	Теперь замечаем, что по полиномиальной иерархии это всё лежит в такой степенной башне:
	$\NEXP^{\NEXP^{\dots}}$.
	
	Дальше план доделывания от семинариста: надо в степенях стереть буквы $EX$ и получить,
	башню $\NEXP^{\NP^{\NP^{\dots}}}$.
	А потом надо схлопывать оракулов сверху вниз до $P$.

\problem{32}
	Разбирал Дима Лапшин.

	Это в точности решение с лекции.
	Научимся по описанию схемы вычислять её с логарифмической памятью.
	Давайте идти таким dfs'ом от конечного гейта схемы: в каждый момент
	мы храним путь от конечного гейта до текущего (по биту на гейт "---
	является ли он левым входом или правым входом предыдущего),
	а также результат вычислений соседнего гейта на каждом уровне, итого
	$\O(\log n)$ бит.
	Когда надо вычислить результат очередного гейта, мы идёт в левый вход,
	рекурсивно вычисляем результат там, возвращаемся, кладём результат на стек,
	идём в правый вход,	рекурсивно вычисляем его, потом возвращаемся и вычисляем
	уже непосредственно наш гейт.
	Для возврата в какой-то гейт (и получения его номера или номера родителя) нам
	надо заново пройтись по всем стеку от конечного гейта схемы, тогда в каждый
	момент надо хранить лишь константное число номеров гейтов.

	\begin{Rem}
		Мы тут снова неявно воспользовались задачей про компизицию машин Тьюринга
		с логарифмической памяти: ведь на вход нам даётся не готовая схема,
		а слово, которое надо распознать или отвергнуть, поэтому сначала нам надо
		построить схему (это можно сделать с логарифмической памятью по определению
		$\NC^1$), а уже потом её вычислить с логарифмической памятью.
	\end{Rem}

\problem{34}
	Оля считает, что эта задача была на лекции (видимо, от 17.03.2016),
	поэтому мы её не стали разбирать, а перешли к более интересной задаче 35.

	\TODO

\section{Задача 35 (не разобрана)}
	Разбирала Оля Черникова и завалилась.

	Сейчас хотим доказать стрелочку вправо: пусть есть $\P$-полный язык $L \in \NC$.
	Покажем, что $\NC=\P$.
	Возьмём произвольный язык $W \in \P$, покажем, что $W \in \NC$.
	Мы знаем, что $L$ "--- $\P$-полон, т.е. за логарифмическую память можно
	получить для слова $w$ такое слово $l$, что $w \in W \iff l \in L$.
	Дальше можно за логарифмическую память построить и вывести схему, которая проверит, что
	$l \in L$.
	Композиция двух вычислений с логарифмом памяти снова использует логарифм памяти.

	А проблема такая: мы построим схему, которая на вход принимает слово $l$, а надо
	построить схему, которая принимает слово $w$.
	То есть надо, чтобы схема каким-то образом преобразовывала $l$ в $w$, этим не может заниматься
	алгоритм, который сводит $W$ к \t{CircuitEval} "--- он может зависеть только от $|w|$, но не
	от самого $w$.

\problem{36}
	Сошлёмся на задачу 32: $\NC^1 \subset \mathsf{L}$, докажем, что $\mathsf{L} \neq PSPACE$.
	\subsubsection{От Нади}
		Разбирала Надя Бугакова.

		Давайте строить язык $A$, лежаший в $\PSPACE$, но не лежащий в $\mathsf{L}$ (мы так уже делали в \hyperref[prob16]{задаче 16}).
		Разрешать слово $x$ длины $n \approx \log x$ будем следующим образом: запускаем $\langle x \rangle(x)$
		(машину Тьюринга с номером $x$ на входе $x$), пока не произойдёт одно из следующих событий:
		\begin{enumerate}
			\item Машина завершается и принимает слово. Тогда говорим, что $x \notin A$.
			\item Машина завершается и отвергает слово. Тогда говорим, что $x \in A$.
			\item Машина использовала больше $\log^2 n$ памяти, тогда $x \notin A$.
			\item Машина сделала больше $2^n \approx x$ шагов, тогда $x \notin A$.
		\end{enumerate}
		Заметим, что любое слово мы разрешаем за полиномиальную память, так как для хранения
		всех счётчиков (количество используемой памяти и сделанных шагов) требуется порядка
		$n$ бит, что есть полином от длины входа.
		
		Теперь заметим, что не существует машины $M$, разрешающей $A$ и использующей $O(\log n)$ памяти
		(будем считать, что не более, чем $C\cdot \log n$).
		В самом деле: пусть такая машина есть.
		Тогда она всегда завершается и, так как различных состояний всего $\O(2^{C \cdot \log n}))=\O(n^C)$,
		она работает за $\O(n^C)$ "--- полиномиальное время.
		Так как она встречается бесконечное число раз, то с некоторого места $2^n > n^C$, т.е. она
		не будет отсекаться по времени.
		Аналогично она не будет отсекаться по памяти, так как с некоторого места $\log^2n > C \cdot \log n$.
		Значит, с некоторого места эта машина всегда завершается.
		Но тогда она ошибается на всех $x$, на которых завершается "--- мы так строили язык.

	\subsubsection{От семинариста}
		Можно сослаться на теорему об иерархии по памяти и тогда всё очевидно.

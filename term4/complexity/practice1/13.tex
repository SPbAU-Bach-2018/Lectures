\chapter{Занятие 04.05.2016}

\section{Напоминание теории}
	\begin{Def}
		Функция $f$ \textit{конструктивна по времени}, если мы её можем вычислить в точке $n$
		за $\O(f(n))$.
		Например, арифметические операции конструктивны с большим запасом, потому что
		их можно вычислять за $\O(\log f(n))$.
	\end{Def}

\section{Разбор задач}
\subproblem{44}{а}[заваленное]
	Разбирала Лиза Третьякова.
	Этот пункт доказать не справились.
	Ниже "--- попытка.

	Давайте вместо равенства доказывать включения:
	\[ \BPTime[f(x)] \subseteq \BPTime[g(x)]  \Ra \BPTime[f(h(x))] \subseteq \BPTime[g(h(x))] \]
	Из этого очевидно будет следовать задача.

	Возьмём язык $A \in \BPTime[f(h(x))]$.
	По определению есть вероятностная машина $M$ (которая может ошибаться в две стороны), которая
	всегда работает за $f(h(x))$.
	Введём язык $B$:
	\[ B \eqcolon \{ x\t{\#}[\t{0}\t{1}]^{h(|x|)-|x|-1} \mid x \in A \]
	\begin{Rem}
		Считаем, что $h(\alpha) \ge \alpha + 1$ (можно даже побольше запас взять).
		Если меньше, то доказывать типа нечего, мы это мельком упомянули, должно быть очевидно.
	\end{Rem}
	\begin{Rem}
		Считаем, что у МТ есть отдельная лента, на которой написана длина входа, причём безошибочно.
	\end{Rem}
	Покажем, что $B$ распознаётся за $f(n)$.
	Упс, мы даже вход считать не можем.
	Возможно, доделывается, если разрешить брать произвольный мусор (а не только из единиц), но не факт.

\subproblem{44}{б}
	Разбирала Лиза Третьякова.

	Сначала первое включение.
	Что такое $\DTime$?
	Это ДМТ, которые точно распознают язык.
	Заметим, что такая машина также является вероятностной (просто использующей ноль случайных бит), которая распознаёт язык с вероятностью $1$.

	Второе включение.
	Просто переберём все случайные биты (их $\O(f(n))$), на каждых случайных битах проэмулируем рандомизированную машину из $\BPTime$
	за $\O(f(n))$, посчитаем долю хороших случайных бит, сравним её с $\sfrac 23$.
	Время работы "--- какое надо.

\subproblem{44}{в}[попытка Лизы]
	Разбирала Лиза Третьякова.
	Этот пункт доказать не справились.
	Ниже "--- три попытки.

	Первое включение.
	Что такое $\BPP$?
	\[ \BPP = \cup_{c \in \N} \BPTime[n^c] \]
	Очевидно, что это вложено в $\BPTime[n^{\log n}]$, так как логарифм мажорирует любую константу (со временем).

	Второе включение.
	Нас просят показать, что $\BPTime[2^{\log^2n}] \subsetneq \BPTime[2^n]$.
	Давайте пойдём от противного:
	\[                                         ^n
		\BPTime[2^n] \subseteq \BPTime[2^{\log^2n}]
	\]
	Дальше с двух сторон вложим в $\DTime$ (по пункту <<б>>):
	\begin{gather*}
		\begin{array}{rcl}
			\DTime[2^n] \subseteq & \BPTime[2^n] & \subseteq \DTime[2^{\O(2^n)}] \\
			& \rotatebox[origin=c]{90}{\subseteq} & \\
			\DTime[2^{\log^2n}] \subseteq & \BPTime[2^{\log^2n}] & \subseteq \DTime[2^{\O(2^{\log^2 n})}]
		\end{array}
	\end{gather*}
	Не получилось.

\subproblem{44}{в}[попытка Димы]
	Разбирал Дима Розплохас, тоже не получилось.

	Хотели взять решение для $\DTime$ и заменить его на $\BPTime$, но не получилось:
	мы не умеем перебирать вероятностные машины, которые задают язык.
	Например, машина, которая просто выдаёт случайный бит, никакой язык не задаёт.

\subproblem{44}{в}[попытка семинариста]
	Разбирал семинарист, тоже не получилось.

	Заметим, что $\BPTime[\log n] = \BPTime[\log n]$.
	От противного докажем строгость второго включение (т.е. предположили, что равны):
	\[
		\BPTime[n^{\log n}9] = \BPTime[2^n]
	\]
	Подставили пункт <<а>> и $h(n)=2^n$:
	\[
		\BPTime[\left(2^n\right)^n] = \BPTime[\left(2^{2^n}\right)]
	\]
	Не получилось.

\problem{63}
	Разбирал семинарист.

	Отличие от PCP-теоремы в том, что тут о-малое, а там "--- о-большое.
	Т.е. это задача про то, что если мы ещё уменьшим число случайных бит, то $\P=\NP$.

	Давайте в предположении задачи придумаем алгоритм для $\SAT$.
	Так как $\SAT \in \NP$, то, по предположении задачи есть алгоритм, который
	может проверять какое-то доказательство для $\SAT$ по $o(\log n)$ случайным битам и $\O(1)$ точек.

	Построим полиномиальный алгоритм для $\SAT$.
	Пусть нам дали формулу $\phi$.
	Давайте зафиксируем набор случайных бит $r_1$.
	Мы знаем, куда ткнёт алгоритм из PCP, получаем функцию $f_{r_1}(w_{i_1}, \dots, w_{i_k})$ от константного
	числа аргументов, которая возвращает результат работы алгоритма (тут $w_{\alpha}$ "--- биты доказательства).
	Для случайных бит $r_2$ есть аналогичная функция $f_{r_2}$, и так далее.
	Мы знаем, что слово принадлежит языку тогда и только тогда, когда есть такие $w_{\alpha}$, являющиеся решением системы:
	\[
		\begin{cases}
			f_{r_1}(w_{i_1}, \dots, w_{i_k}) = 1 \\
			f_{r_2}(w_{i_{k+1}}, \dots, w_{i_{2k}}) = 1 \\
			\dots
		\end{cases}
	\]
	В этой системе имеется всего $2^{o(\log n)}=o(n)$ функций.
	Каждую функцию можно записать в виде формулы константной длины.
	Значит, мы можем записать формулу $\phi'(w_{\alpha})$ длины $o(n)$, эквивалентную системе.
	Таким образом, свели выполнимость формулы $\phi$ к выполнимости формулы $\phi'$, причём $|\phi'|=o(|\phi|)$.
	Значит, повторим не более чем $|phi'|$ раз такое сведение, мы получим формулу маленького размера.

\subproblem{54}{а}[уточнение]
	Разбирал семинарист.

	Кажется, мы не умеем решать проще, чем доказывать, что язык простых чисел лежит в $\P$.
	А это весьма неприятный алгоритм AKS.

\subproblem{54}{б}
	Разбирал семинарист.

	Если $\mathsf{USAT} \in \mathsf{UP}$, то $\mathsf{USAT} \in \NP$.
	А $\mathsf{USAT}$ "--- это $\co\NP$-полная задача, значит, у нас есть $\NP$-алгоритм для любой $\co\NP$-задачи.
	Получили вложение $\co\NP \subseteq \NP$.
	Но мы знаем, что одно является дополнением другого, значит, есть равенство.

\problem{46}
	Разбирал семинарист.

	Давайте запишем матрицу переходов машины между конфигурациями (конфигураций полиномиальное число,
	так как памяти логарифм).
	В двух строчках (в которой машина заканчивает работу "--- принимающая и отвергающая) стоят одни нули.
	В остальных строчках сумма чисел равна единице "--- либо одна единица, либо две половинки.
	Давайте домножим эту матрицу на вектор из нулей и одной единицы (единица в начальной конфигурации).
	Получим вектор, характеризующий, с какой вероятностью мы окажемся в какой конфигурации после одного шага.

	Давайте оценим, в какую степень надо возводить матрицу, чтобы норма матрицы стала очень маленькой
	(тогда будущие степени не повлияют на ответ).
	Оказывается, достаточно возвести матрицу в какую-то полиномиальную степень (скажем, $n^{18}$).
	Тогда рассмотрим $A^{n^{18}} \cdot v$ (добавив в $A$ петли в завершающих конфигурациях),
	у результата все вероятность сконцентрируются в строке завершения конфигурации, просто сравним, какая вероятность больше.

	Считаем не в вещественных числах, а в целых, просто домножив всю матрицу на знаменатель.

\section{Подсказки}
\problem{55}[подсказка]
	Надо посмотреть на теорему, которая говорит, что $\P$ и $\NP$ можно отделить оракулом.

\problem{62}[подсказка]
	Вместо того, чтобы считать перманент конкретной матрицы, надо научиться его выражать через перманент случайной матрицы.
	Это что-то похожее на тест на линейность в PCP-теореме, где мы вместо обращения к $f(x)$ обращались к $f(r) \oplus f(x \oplus r)$

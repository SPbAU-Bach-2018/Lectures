\setcounter{section}{24}
\section{Билет 25}
\subsection[PCP]{$\PCP$}
	\begin{Def}
		$\PCP[f(n), g(n)]$ "--- это класс таких языков, что существует полиномиальная машина Тьюринга $V$ (verifier),
		которая, используя $\O(f(n))$ битов рандома, неадаптивно обращаясь к $\O(g(n))$ битам подсказки, гарантированно принимает
		корректные подсказки и с вероятностью не более $\frac 12$ принимает некорректные подсказки.
	\end{Def}
	\begin{theorem}
		Без доказательства:
		\[ \NP = \PCP[\log n, 1] \]
	\end{theorem}
	\begin{theorem}
		Будем доказывать в следующих билетах:
		\[ \exists k \colon \NP = \PCP[poly(k), 1] \]
	\end{theorem}

\subsection{Связь с неаппроксимируемостью}
	Из $\PCP$-теоремы можно вывести, что некоторые задачи в предположении $\P\neq \NP$ не только плохо решаются,
	но даже плохо приближаются.

	\begin{Def}
		$\mathsf{MAX-3SAT}$ "--- посчитать по данной формуле $\phi$ в 3-КНФ, какое максимальное число клозов можно выполнить одновременно.
	\end{Def}
	\begin{Rem}
		Эта задача $\NP$-трудная, так как к ней сводится $\mathsf{SAT}$.
	\end{Rem}

	\begin{assertion}
		Есть алгоритм $A$ такой, что по любой формуле в 3-КНФ с $m$ клозами (из которых доля $\alpha$ могут быть выполнены)
		он выдаёт решение, которое выполняет не менее $\frac{m}{2}$ клозов.
	\end{assertion}
	\begin{proof}
		Жадный алгоритм: переберём все переменные по очереди, для каждой выбираем то значение, которое выполнит наибольшее число клозов из оставшихся.
		Без доказательства? \TODO
	\end{proof}

	\begin{assertion}
		Без доказательства: есть алгоритм $A$ такой, что по любой формуле в 3-КНФ с $m$ клозами (из которых доля $\alpha$ могут быть выполнены)
		он выдаёт решение, которое выполняет не менее $\sfrac78 m$ клозов.
	\end{assertion}

	\begin{lemma}
		Существует константа $0 \le c \le 1$ и полиномиальный алгоритм, который любую булеву формулу $\phi$ в 3-КНФ может переделать в 3-КНФ формулу $\phi'$, причём:
		\begin{itemize}
			\item Если $\phi$ была выполнима, то $\phi'$ тоже выполнима
			\item Если $\phi$ была невыполнима, то $\phi'$ сильно невыполнима, т.е. в ней нельзя выполнить долю клозов больше $c$
		\end{itemize}
	\end{lemma}
	\begin{proof}
		Мы знаем, что $\PCP[\log n, 1] = \NP$.
		Мы знаем, что $\mathsf{3SAT} \in \NP$, тогда давайте возьмём PCP-алгоритм для него.
		Давайте переберём все возможные случайные биты, которые потребует PCP-алгоритм для проверки выполнимости $\phi$, их всего полином.
		При конкретном наборе случайных бит можно записать формулу \textit{константной длины} от битов доказательства, которая будет проверять соответствующие биты.
		Давайте эту формулу от констаного числа бит доказательства запишем в 3-КНФ, она всё ещё останется константной длины.
		Обозначим эту константу $C$.

		Давайте теперь запишем конъюнкцию всех этих 3-КНФ блоков по всем случайным наборам, получим формулу $\phi'$.
		Если $\phi$ была выполнима, то и $\phi'$ выполнима, так как PCP-алгоритм всегда принимает, независимо от случайных бит.
		В противном же случае, хотя бы половина <<блоков>> должна быть невыполнима (т.е. в каждом блоке хотя бы один клоз должен быть невыполним).
		Значит, всего хотя бы $\frac{1}{2C}$ клозов должно быть невыполнимо в формуле $\phi'$ "--- эту долю и обозначим за $1-c$.
	\end{proof}
	\begin{conseq}
		Из этого сразу следует, что если есть полиномиальный алгоритм, решающий $\mathsf{MAX-3SAT}$ приближённо с долей больше $c$, то $\P=\NP$, так как мы умеем решать $\mathsf{3-SAT}$ точно.
		В самом деле: берём формулу $\phi$, строим $\phi'$, запускаем на ней наш чудо-алгоритм.
		Если $\phi$ была выполнима, то он выдаст долю клозов больше $c$, иначе "--- не выдаст (так как это невозможно по построению $\phi'$).
	\end{conseq}

\setcounter{section}{14}
\section{Билет 15}
	Раздел 1.12 в конспекте Оли.

	\begin{theorem}
		Если $s(n)$ конструктивна по памяти, то $\NSpace[s(n)] \subseteq \DTime[2^{\O(s(n))}]$
	\end{theorem}
	\begin{Rem}
		Этой теоремой не пользуемся, есть более крутая, см. ниже.
		Это просто пример.
	\end{Rem}
	\begin{proof}
		\begin{Rem}
			У НМТ из одного состояния может вести сколько угодно переходов, хоть во все остальные состояния.
			Поэтому просто перебирать все возможные способы работы НМТ не поможет.
		\end{Rem}
		Так как НМТ работает за $s(n)$ шагов, то она использует не более $\O(s(n))$ памяти.
		Значит, у неё всего $2^{\O(s(n))}$ конфигураций.
		Можно явно построить граф таких конфигураций (квадрат экспоненты "--- всё ещё экспонента) и пройтись по нему DFS'ом, пытаясь найти путь из исходной конфигурации в какую-нибудь из терминальных.
	\end{proof}

	\begin{theorem}
		Если имеется орграф на $n$ вершинах и наличие/отсутствие ребра в графе можно проверять за $\O(\log n)$ памяти,
		то задача проверка достижимости из вершины $a$ вершины $b$ лежит в $\DSpace[\log^2 n]$.
	\end{theorem}
	\begin{proof}
		Введём предикат $s(a, b, i)$ "--- можно ли попасть из вершины $a$ в вершину $b$, сделав не более $2^i$ переходов.
		Покажем индукцией по $i$, что его можно вычислять с использованием $C \cdot (i+1)\cdot \log n$ памяти, где $C$ "--- некоторая фиксированная константа.
		Считаем, что на одной рабочей ленте МТ будет записано $a$, на другой "--- $b$, на третьей "--- $i$.

		База при $i=0$: это просто проверка на наличие ребра в графе из условия задачи, нам только важно, что $C$ не меньше той константы, что спрятана в ошке в условии.
		Переход: пусть мы хотим научиться вычислять $s(a, b, i)$.
		Сохраним $a$, $b$, $i$ на стек, потратим не более $3 \cdot \log n$ памяти.
		Потом переберём промежуточную вершину $z$, используя на это ещё $\log n$ памяти.
		После фиксации вершины $z$ посчитаем предикат $s(a, z, i - 1)$, используя $C \cdot i \cdot \log n$ памяти.
		Итого будет использовано не более $4 \cdot \log n + C \cdot i \cdot \log n \le C \cdot (i + 1) \cdot \log n$ памяти (если $C \ge 4$).
		Посчитали, если результат отрицательный, то сразу возвращаем ноль, иначе считаем предикат $s(z, b, i - 1)$.
		
		А чтобы теперь посчитать глобальную достижимость, надо посчитать при $i=\log n$, получим память $\O(\log^2 n)$.
	\end{proof}

	\begin{theorem}
		Если $s(n)$ конструктивна по памяти, то $\NSpace[s(n)] \subseteq \DSpace[(s(n))^2]$
	\end{theorem}
	\begin{proof}
		Это следствие предыдущей теоремы: у НМТ есть не более $2^{C \cdot s(n)}$ конфигураций.
		Будем считать, что перед приёмом слова НМТ стирает все ленты, т.о. принимающая конфигурация ровно одна.
		Можно ли перейти из одной конфигурации в другую легко выяснить, прочитав обе конфигурации.

		Т.о. мы находимся в условии предыдущей леммы: есть граф на $2^{C \cdot s(n)}$ вершинах, хотим проверить достижимость.
		Это можно сделать за $\O(C^2 \cdot s^2(n)) = \O(s^2(n))$.
	\end{proof}

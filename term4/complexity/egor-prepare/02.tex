\setcounter{section}{1}
\section{Билет 2}

\subsection{Универсальная МТ}
	Тут мы говорим про ДМТ.
	Вход для универсальной ДМТ $U$ "--- пара из ДМТ и входа для неё "--- $(M, x)$.
	Интересно то, сколько шагов затратит $U$ на эмуляцию $M$, если $M$ работала за $t$ шагов.

	\begin{Rem}
		В общем случае алфавит $M$ может быть сильно больше алфавита $U$, но это лечится преобразованием машины $M$
		в машину $M'$, которая пользуется алфавитом $\{0, 1, \mathvisiblespace, \triangleright\}$ и кодирует каждый старый символ
		фиксированным числом бит.
		Замедление $M'$ по сравнению с $M$ будет лишь константное.
	\end{Rem}
	
	Если нам требуется моделировать только $M$ с ограниченным числом лент $\le k$, то
	это легко делается машиной $U$ с $k+1$ лентой с константным замедлением (т.е. $U_M$ будет работать для конкретной $M$ за $\O(t)$).
	На этой отдельной ленте мы храним описание $M$ и текущее состояние, а первые $k$ лент и позиции головок на них в точности соответствуют машине $M$.
	Прочитали за не зависящее от $t$ время символы под головками, нашли нужный переход, перешли.

	Если у нас машина $U$ имеет только одну ленту, то эмулировать можно за время $\O(t^2)$.
	Мы храним описание $M$ в начале ленты, потом храним $k$ рабочих лент вперемешку (сначала первые символы, потом вторые символы и т.д.).
	На каждой ленте символ, на который указывает головка машины, мы специальным образом помечаем, чтобы не потерять.
	А дальше совершение шага такое:
	\begin{enumerate}
		\item Доехали до упора влево, до описания $M$.
		\item Там мы где-то храним номер текущего состояния.
		\item \TODO
	\end{enumerate}
	
	Если у нас все машины одноленточные (и $U$, и $M$), то эмулировать довольно просто, мы это делали на логике в прошлом семестре.
	У нас есть описание эмулируемой машины фиксированного размера, мы его таскаем по ленте вслед за головкой, получаем замедление в константное число раз.

	А вот если лент у нас сколько угодно, то было бы разумно считать, что у $U$ число лент фиксированное, а вот эмулировать она может машину с произвольным числом лент.
	Можно либо делать за $\O(t^2)$ (см. 1.3.1 в Arora-Barak), либо за $\O(t \log t)$ (см. теорему 1.13 в Arora-Barak, это раздел 1.A).
	$\O(t^2)$ легко доделать до oblivious, а вот с доделкой $\O(t \log t)$ у меня возникли проблемы (это упражнение 9 в Arora-Barak к главе 1).
	\TODO

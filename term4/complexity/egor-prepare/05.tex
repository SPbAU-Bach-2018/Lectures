\setcounter{section}{4}
\section{Билет 5}
	\begin{theorem}
		Пусть есть массовая задача $R \in \widetilde{NP}$, причём $L(R)$ "--- $\NP$-полон.
		Пусть также есть алгоритм $A$ для разрешения языка $L(R)$.
		Тогда имеется алгоритм $B$ для решения задачи поиска $R$, работающий не более, чем в полином раз медленнее алгоритма $A$.
	\end{theorem}
	\begin{Rem}
		Смысл такой: есть мы для $\NP$-полной задачи умеем проверять существование ответа, то искать ответ мы умеем не сильно медленнее.
	\end{Rem}
	\begin{proof}
		Мы знаем, что так как $L(R) \in \mathsf{NPC}$, то к $L(R)$ можно свести задачу $\SAT$,
		т.е. есть такая п.о. полиномиально вычислимая $f$, что для любой формулы $\phi$:
		\[ \phi \in \SAT \iff f(\phi) \in L(R) \]
		Значит, у нас есть алгоритм для проверки выполнимости формулы не медленнее, чем в полином раз медленнее $A$.
		Давайте теперь научимся, используя этот алгоритм, находить решение.
		Это просто: мы берём формулу, проверяем выполнимость.
		Если выполнима, то подставляем $x_1=0$, если формула всё ещё выполнима "--- мы угадали одну переменную, идём дальше.
		Иначе точно надо подставить $x_1=1$ и начать угадывать следующие переменные.
		Всего у нас будет не более $k$ запусков проверки на выполнимость, если имеется $k$ переменных.

		Т.о. есть алгоритм $A'$, который ищет решение $\mathsf{SAT}$ довольно быстро.

		Давайте учиться по условию $x$ для задачи $R$ быстро искать ответ.
		У нас есть ДМТ $M$, которая по входу из условия и решения $R$ проверяет корректность.
		Аналогично доказательству теоремы Кука-Левина (4 билет) можно построить булеву формулу, эмулирующую $M$.
		В неё будет зашито условие $x$, а переменные будут в точности задавать решение для входа, формула
		будет верна тогда и только тогда, когда решение действительно является решением.
		А для этой формулы мы можем применить алгоритм $A'$, который быстро найдёт решение.
	\end{proof}
	\begin{Rem}
		Краткая схема доказательства:
		\begin{enumerate}
			\item Сводим $\mathsf{SAT}$ к нашему $\NP$-полному языку, теперь умеем быстро проверять выполнимость.
			\item Угадываем значения переменных по одной, теперь мы умеем быстро искать решения для булевых формул.
			\item Строим по машине, проверяющей решение $R$, эквивалентную булеву формулу, решением которой будут являться в точности решения исходной задачи.
				Мы уже умеем быстро искать решения для булевых формул, успех.
		\end{enumerate}
	\end{Rem}
	\begin{Rem}
		$\NP$-полнота существенна, например, про задачу поиска нетривиальных делителей $\mathsf{FACTOR}$ такого неизвестно.
		Мы знаем детерминированные полиномиальные (от длины числа) алгоритмы проверки на простоту (ABS-тест), но вот искать делители так же эффективно не умеем.
	\end{Rem}

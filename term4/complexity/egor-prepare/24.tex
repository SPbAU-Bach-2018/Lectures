\setcounter{section}{23}
\section{Билет 24}
	Раздел 1.16.1 в конспекте Оли и \href{http://neerc.ifmo.ru/wiki/index.php?title=\%D0\%A2\%D0\%B5\%D0\%BE\%D1\%80\%D0\%B5\%D0\%BC\%D0\%B0_\%D0\%92\%D0\%B0\%D0\%BB\%D0\%B8\%D0\%B0\%D0\%BD\%D1\%82\%D0\%B0-\%D0\%92\%D0\%B0\%D0\%B7\%D0\%B8\%D1\%80\%D0\%B0\%D0\%BD\%D0\%B8}{викиконспекты}.

	\begin{Def}
		Булева формула называется \textit{одновыполнимой}, если у неё есть ровно один выполняющий набор.
	\end{Def}
	\begin{Rem}
		Язык таких формул называется $\mathsf{USAT}$ (unique SAT).
	\end{Rem}

	\begin{theorem}
		По любой булевой формула $\phi$ можно за полиномиальное время построить такой набор $\phi_1, \dots, \phi_m$, что:
		\begin{itemize}
			\item Если $\phi$ невыполнима, то $\phi_i$ тоже невыполнимы
			\item Если $\phi$ выполнима, то с вероятностью хотя бы $\sfrac 12$ среди $\phi_i$ есть одновыполнимая формула.
		\end{itemize}
	\end{theorem}

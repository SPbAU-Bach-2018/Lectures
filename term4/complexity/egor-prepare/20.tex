\setcounter{section}{19}
\section{Билет 20}
	Ниже будем доказывать, что $\BPP \subseteq \Sigma_2$.

\subsection{Основная идея}
	Рассмотрим какой-нибудь язык $L \in \BPP$.
	Будем считать, что есть п.о. п.п. бинарное отношение $R$, которое проверяет слова из
	языка с ошибкой не более $2^{-n}$, где $n$ "--- длина проверяемого слова, мы так понижать ошибку умеем.
	Обозначим за $m$ число использованных при проверке случайных бит.
	Выберем как-то число $k$ (как "--- поймём позже, но пока считаем, что оно является полиномом от $n$, где $n$ "--- длина проверяемого слова).
	\begin{Rem}
		Что происходит при небольших $n$, нас не интересует "--- мы можем намертво <<зашить>> ответы для всех коротких строк в обход основного случая.
	\end{Rem}
	\begin{Rem}
		Спойлер: $k$ будет равно $m$ при $n > 1$.
	\end{Rem}

	Хотим показать, что $x \in L$ равносильно тому, что существует такой набор битовых строк
	$t_1, t_2, \dots, t_k$ ($|t_i|=m$), что из любой битовой строчки $r$ можно получить
	правильную подсказку для $R$:
	\begin{gather*}
		x \in L \\
		\Updownarrow \\
		\exists t_1, \dots, t_k \colon
		\forall r \colon
		\left[
			\begin{array}{l}
				R(x, r \oplus t_1) \\
				R(x, r \oplus t_2) \\
				\vdots \\
				R(x, r \oplus t_k) \\
			\end{array}
		\right.
	\end{gather*}
	\begin{Rem}
		Смысл за этим стоит следующий: мы берём все подсказки $w$ такие, что $(x, w) \in R$,
		сдвигаем их по $k$ различным векторам в разных направлениях и тогда $x \in L \iff$ мы так покрыли все подсказки.
	\end{Rem}

	Заметим, что если $x \notin L$, то для каждого конкретного $t_i$ вероятность события $R(x, r \oplus t_i)=1$ не превосходит $2^{-n}$.
	А вероятность того, что случится одно из $k$ событий (пусть и зависимых) вероятности не более $2^{-n}$ каждое, не превосходит $k \cdot 2^{-n}$.
	Если это число меньше единицы, то гарантированно найдётся такое $r$, что ни одно $t_i$ не сделает $R(x, r \oplus t_i)$.
	Значит, имеем ограничение на $k$:
	\[ k < 2^n \]
	\begin{Rem}
		Это экспоненциальное ограничение, а $k$ может быть максимум полиномиально, так как мы должны под квантором получить $k$ строчек.
		Так что это условие можно считать автоматически выполненным с некоторого $n$, чего нам достаточно "--- для мелких $n$ нам неважно.
	\end{Rem}

	А если $x \in L$, то хочется сказать, ситуация обратная "--- мы должны суметь покрыть тот маленький кусок
	подсказок, которые не удовлетворяют $R$, так как хороших подсказок у нас очень много.

\subsection{Аккуратная оценка вероятности}
	Давайте покажем, что если $x \in L$, то существует подходящий набор $t_i$, заодно узнаем, какого $k$ нам хватит.
	Выберем все $t_i$ случайно равновероятно и оценим сверху вероятность события <<нашлась хотя бы одна подсказка $r$, для которой всё сломалось>>.
	Если эта вероятность окажется меньше единицы, то нужный набор $t_i$ существует.

	Обозначим за $\chi(P)$ обозначим характеристическую функцию предиката $P$ (ноль, если $P$ ложно и единица иначе):
	\begin{gather*}
		\Pr_{t_1, \dots, t_k} \{ \exists r \colon R(x, r \oplus t_1) = \dots = R(x, r \oplus t_k) = 0 \} = \\
		= \Expect_{t_1, \dots, t_k} \chi\left(\exists r \colon R(x, r \oplus t_1) = \dots = R(x, r \oplus t_k) = 0\right) \le  \\
		\le \Expect_{t_1, \dots, t_k} \sum_r \chi\left(R(x, r \oplus t_1) = \dots = R(x, r \oplus t_k) = 0\right) = \\
		= \sum_r \Expect_{t_1, \dots, t_k} \chi\left(R(x, r \oplus t_1) = \dots = R(x, r \oplus t_k) = 0\right) = \\
		= \sum_r \Pr_{t_1, \dots, t_k} \{ R(x, r \oplus t_1) = 0 \land \dots \land R(x, r \oplus t_k) = 0 \} = \\
		= \sum_r \prod_{i=1}^k \Pr_{t_i} \{ R(x, r \oplus t_i) = 0 \}
		= \sum_r \left(\Pr_{t} \{ R(x, r \oplus t) = 0 \} \right)^k
		= \sum_r \left(\Pr_{t} \{ R(x, t) = 0 \} \right)^k \le \\
		\le \sum_r \left(\frac{1}{2^{n}} \right)^k
		= \sum_r \frac{1}{2^{nk}}
		= \frac{2^m}{2^{nk}}
	\end{gather*}
	Наше требование:
	\[ 2^m < 2^{nk} \iff nk > m \]
	В частности, можно положить $k=m$ (если $n>1$), это полином от $n$, что и требовалось.

\subsection{Следствие}
	\begin{conseq}
		$\BPP \subseteq \Pi_2$
	\end{conseq}
	\begin{proof}
		Пусть $L \in \BPP$.
		Тогда $\bar L \in \BPP$, так как $\BPP$ замкнут относительно отрицания.
		Но тогда по теореме $\bar L \in \Sigma_2$.
		Значит, $\bar{\bar L} = L \in \Pi_2$, что и требовалось.
	\end{proof}

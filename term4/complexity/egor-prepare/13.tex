\setcounter{section}{12}
\section{Билет 13}
\subsection[NC и logspace-вычислимость]{$\NC$ и logspace-вычислимость}
	\begin{Def}
		Семейство схем $\{C_n\}_{n\in\N}$ \textit{равномерно}, если имеется полиномиальный алгоритм $A$ такой,
		что $A(1^n)=C_n$.
	\end{Def}
	\begin{Rem}
		Равномерность нужна, чтобы мы не могли зашивать в схему какие-то невычислимые функции, зависящие от размера входа.
	\end{Rem}
	\begin{Def}
		Функция $f$ называется \textit{logspace-вычислимой}, если существуют вычисляющая её ДМТ со следующими свойствами:
		\begin{enumerate}
			\item Одна лента выделена под вход и не может изменяться на протяжении работы МТ (read-only)
			\item
				Одна лента выделена под результат, машина не может с неё читать (только писать) и не может перемещать головку влево.
				Т.е. если МТ что-то записала и поехала дальше, эта запись уже не меняется
			\item Суммарная используемая память на остальных (рабочих, read-write) лентах не превосходит $\O(\log n)$, где $n$ "--- длина входа функции.
		\end{enumerate}
	\end{Def}
	\begin{assertion}
		Размер выхода logspace-вычислимой функции может быть лишь полиномиален от входа.
	\end{assertion}
	\begin{proof}
		Просто посчитаем число конфигураций, оно не меньше времени работы, а оно не меньше размеры выхода:
		\[ \O(2^{\O(\log n)})=\O(2^{C\cdot \log n})=\O\left(\left(2^{\log n}\right)^C\right)=\O(n^C) \]
	\end{proof}
	\begin{exmp}
		$f(x)=x+1$ является \textit{logspace-вычислимой}: надо сначала запомнить позицию, начиная с которой идёт хвост вида $\t{011\dots11}$ (это $\log n$ памяти),
		потом вывести начало $x$ до этой позиции, а потом "--- новый хвост $\t{100\dots00}$.
	\end{exmp}

	\begin{Def}
		Семейство схем \textit{logspace-равномерно}, если оно равномерно и при этом алгоритм $A$ использует не более $\O(\log n)$ памяти (в том же смысле, что и logspace-вычислимые функции).
		То есть входная лента (с $n$ единицами) доступна только для чтения, выходная лента доступна только для записи и головка на ней двигается только вправо,
		в память считается только память рабочих лент.
	\end{Def}
	\begin{Def}
		Класс $\NC^i$ "--- это все задачи, которые решаются logspace-равномерными семействами схем глубины не более $\O(\log^i |x|)$.
		То есть задача $L$ лежит в этом классе тогда и только тогда, когда имеется полиномиальный (от длины входа) алгоритм $A$ такой,
		что он строит по длине входа некоторую схему $C_n$ с $n$ входами для которой потом верно:
		\[
			|x| = n \Ra C_n(x) = L(x)
		\]
	\end{Def}
	\begin{Rem}
		Logspace-равномерность тут нужна, чтобы алгоритм $A$ был не слишком умный и не мог бы что-то предподсчитать про входы размера $n$.
		Он должен просто выдавать схему (причём не зная входа, зная только его длину), а схема должна делать всю вычислительную работу.
	\end{Rem}
	\begin{Def}
		$\NC = \Cup_i \NC^i$
	\end{Def}

	\begin{lemma}
		$\NC \subseteq \P$
	\end{lemma}
	\begin{proof}
		Возьмём задачу $L \in \NC$.
		Для неё есть порождающий схему алгоритм $A$.
		Построим полиномиальный алгоритм проверки $x \in L$.

		Пусть дали $x$.
		Давайте запустим $A(1^{|x|})$, он отработает за полиномиальное время и выдаст схему полиномиального от $|x|$ размера.
		Дальше мы её просто вычислим.
	\end{proof}

\subsection[Сводимости в P]{Сводимости в $\P$}
	\begin{Rem}
		До этого моменты шли сведения, полиномиальные по времени.
		Теперь нам нужны более тонкие сведения.
	\end{Rem}
	\begin{Def}
		Когда мы говорим про сведение задач из класса $\P$ друг к другу, мы будем предполагать, что
		сведение использует $\O(\log n)$ памяти и, следовательно, полиномиальное время.
	\end{Def}

	\begin{Def}
		Язык $L$ "--- \textit{$\P$-полон}, если он принадлежит $\P$ и любой другой язык из $\P$
		к нему сводится (за логарифмическую память).
	\end{Def}

	\begin{lemma}
		Композиция двух logspace-вычислимых функций (т.е. функций, которые при вычислениях не модифицируют входную ленту, на выходную ленту могут только дописывать в конец, при этом
		использующих $\O(\log n)$ памяти) тоже может быть вычислена с использованием $\O(\log n)$ памяти.
	\end{lemma}
	\begin{proof}
		Пусть надо посчитать $f(g(x))$, обозначим $n=|x|$.
		Для начала заметим, что размер выхода $g$ полиномиален от $n$ "--- это означает, что вычисление $f(g(n))$
		(по известному $g(n)$) тоже требует $\O(\log n)$ памяти.

		Давайте возьмём машину $A$, считающую $f$, и машину $B$, считающую $g$.
		Добавим к машине $A$ рабочие ленты машины $B$, а также ленту с позицией головки машины $A$ на входе (т.е. на выходе $B$) "--- это всё еще $\O(\log n)$ памяти.
		Попросим машину $A$ начать вычислять.
		Когда ей потребуется считать $i$-й бит $g(n)$, перезапустим с самого начала машину $B$, на ещё одной ленте будем поддерживать
		текущую позицию машины $B$ в выходе.
		Биты в выводе $B$ до $i$-го будем игнорировать, а вот после вывода $i$-го бита подадим его на вход машине $A$ и остановим машину $B$.

		Таким образом у нас время работы может получиться квадратичным (мы на каждый бит входа, который использует $A$, перезапускаем $B$ с нуля),
		но память осталась логарифмической.
	\end{proof}
	\begin{Rem}
		В народном хозяйстве это означает, что сведение языков в $\P$ транзитивно: если $A \to B$ и $B \to C$ (всё с логарифмической памятью),
		то также с логарифмической памятью можно свести $A \to C$ напрямую.
	\end{Rem}

\subsection[Следствия из P-полноты]{Следствия из $\P$-полноты}
	\begin{theorem}
		Если $L$ "--- $\P$-полный, то:
		\[ L \in \NC \iff \P = \NC \]
	\end{theorem}
	\begin{proof}
		\begin{description}
			\item[$\Ra$:]
				Мы уже знаем, что $\NC \subseteq \P$, осталось показать второе включение.
				Возьмём произвольную задачу $L' \in \P$.
				Мы знаем, что она сводится к $L$ с использованием логарифмической памяти.

				\TODO задача 35 у первой группы, надо построить схему, которая будет преобразовывать слова из $L'$ в слова для $L$,
				а потом будет решать $L$ (так как $L \in \NC$).
			\item[$\La$:]
				Раз $L$ является $\P$-полной, то $L\in \P = \NC$, что и требовалось.
		\end{description}
	\end{proof}
	\begin{theorem}
		Если $L$ "--- $\P$-полный, то:
		\[ L \in \DSpace[\log] \iff \P = \DSpace[\log] \]
	\end{theorem}
	\begin{proof}
		\begin{description}
			\item[$\Ra$:]
				Очевидно, что $\DSpace[\log] \subseteq \P$ "--- если вычисление использует $C \cdot \log n$ памяти,
				то всего конфигураций не более полинома:
				\[ \O(2^{C\log n})=\O\left(\left(2^{\log n}\right)^C\right)=\O(n^C) \]
				Следовательно, время работы тоже ограничено полиномом.

				Покажем второе включение.
				Возьмём произвольную задачу $L' \in \P$.
				Так как $L$ "--- $\P$-полна, то есть logspace-сведение $f$:
				\[ x' \in L' \iff f(x') \in L \]
				Но так как $L \in \DSpace[\log]$, то есть logspace-функция $g$:
				\[ x \in L \iff g(x) = 1 \]
				А мы знаем, что композиция logspace-функций снова logspace, т.о.:
				\[ x' \in L' \iff g(f(x')) = 1\]
				Построили требуемую logspace-функцию, распознающую язык $L'$, значит, $L' \in \DSpace[\log]$, что и требовалось.
			\item[$\La$:]
				Так как $L$ "--- $\P$-полный, то $L \in \P = \DSpace[\log]$, что и требовалось.
		\end{description}
	\end{proof}

	\begin{Def}
		Язык $\mathsf{CIRCUIT\_EVAL}$ "--- множество пар $(C, x)$, где $C$ "--- схема, $x$ "--- её вход, причём $C(x)=1$.
	\end{Def}
	\begin{assertion}
		$\mathsf{CIRCUIT\_EVAL}$ является $\P$-полным языком.
	\end{assertion}
	\begin{proof}
		Очевидно, этот язык лежит в $\P$ "--- берём и вычисляем схему.

		Осталось показать, что любой язык $L \in \P$ сводится к $\mathsf{CIRCUIT\_EVAL}$.
		Давайте придумаем функцию сведения $f$.
		Пусть нам дали на вход слово $x$.
		Давайте слово-в-слово повторим доказательство теоремы Кука-Левина (где мы сводили задачи из $\NP$ к $\SAT$) и построим
		многослойную схему, которая эмулирует работу машины Тьюринга, разрешающей язык $L$.
		Слои у нас все одинаковые, связи между ними "--- тоже, логарифма памяти точно хватит (логарифм памяти "--- это некоторое константное число int'ов, если угодно).
		Получим на выходе схему, у которой нет входов (это точно не баг? \TODO) и которая вычисляется в единицу тогда и только тогда, когда машина принимает слово, что и требовалось.
	\end{proof}

\setcounter{section}{8}
\section{Билет 9}
\subsection{Определения}
	\begin{Def}
		Если $L$ "--- язык, то $\bar L$ "--- его \textit{дополнение}:
		\[ x \in L \iff x \notin \bar L \]
	\end{Def}
	\begin{Def}
		Если $C$ "--- класс языков, то $\co C$ "--- класс дополнений:
		\[ L \in C \iff \bar L \in \co C \]
	\end{Def}
	\begin{exmp}
		Язык $\SAT$ лежит в $\NP$.
		А его дополнение (это все невыполнимые булевы формулы плюс строчки, не кодирующие булевы формулы) лежит в $\co\NP$.
	\end{exmp}
	\begin{assertion}
		Если $D$ "--- класс языков, то:
		\[ C^D = C^{\co D} \]
	\end{assertion}
	\begin{proof}
		Мы можем просто после каждого вызова оракула инвертировать ответ.
	\end{proof}

	\begin{Def}
		Полиномиальная иерархия состоит из трёх семейств классов, определяемых рекурсивно (при $i=0$ по определению):
		\begin{align*}
			\Sigma_0 &= \P & \Pi_0 &= \P & \Delta_0 &= \P \\
			\Sigma_1 &= \NP^{\P} & \Pi_1 &= \co\NP^{\P} & \Delta_1 &= \P^{\P} \\
			\Sigma_2 &= \NP^{\co\NP^{\P}} & \Pi_2 &= \co\NP^{\NP^{\P}} & \Delta_2 &= \P^{\NP^\P} \\
			\Sigma_3 &= \NP^{\co\NP^{\NP^{\P}}} & \Pi_2 &= \co\NP^{\NP^{\co\NP^{\P}}} & \Delta_2 &= \P^{\NP^{\co\NP^\P}} \\
			&\vdots && \vdots && \vdots \\
			\Sigma_{i+1} &= \NP_{\Pi^i} & \Pi_{i+1} &= \co\NP_{\Sigma^i} & \Delta_{i+1} &= \P^{\Sigma^i\P}
		\end{align*}
	\end{Def}
	\begin{Rem}
		Класс $\Delta_i$ мы нигде не используем.
		Также классы могут обозначаться как $\Sigma^i \P$ вместо $\Sigma_i$, но так длиннее писать.
	\end{Rem}
	\begin{assertion}
		При $i \ge 0$:
		\begin{align*}
			\Sigma_i &= \co\Pi_i \\
			\Sigma_{i+1} &= \NP^{\Sigma_i} = \NP^{\Pi_i} \\
			\Pi_{i+1} &= \co\NP^{\Sigma_i} = \co\NP^{\Pi_i} \\
		\end{align*}
	\end{assertion}
	\begin{proof}
		Индукция по $i$.
		При $i=0$ всё очевидно, в переходе мы просто пользуемся определением $\Sigma_i$ и $\Pi_i$ и тем фактом, что в оракуле можно менять класс языков на его дополнение.
	\end{proof}
	\begin{Def}
		Класс $\PH$ (polynomial hierarchy):
		\[ \PH = \bigcup_{i \ge 0} \Sigma_i \]
	\end{Def}
	\begin{exmp}
		\begin{align*}
			\Sigma_2 = \Sigma^2 \P = \NP^{\NP} = \NP^{\co\NP} \\
			\Pi_2 = \Pi^2 \P = \co\NP^{\NP} = \co\NP^{\co\NP}
		\end{align*}
	\end{exmp}
	\begin{assertion}
		\[ \PH = \co\PH \]
	\end{assertion}
	\begin{proof}
		Включение $\co\PH \subseteq \PH$:
		\begin{gather*}
			\co\PH =
				\co\bigcup_{i \ge 0} \Sigma_i \subseteq
				\co\bigcup_{i \ge 0} \Pi_{i+1} =
				\bigcup_{i \ge 1} \co\Pi_i =
				\bigcup_{i \ge 1} \Sigma_i \subseteq
				\bigcup_{i \ge 0} \Sigma_i =
				\PH
		\end{gather*}
		Включение $\co\PH \supseteq \PH$:
		\begin{gather*}
			\co\PH =
				\co\bigcup_{i \ge 0} \Sigma_i \supseteq
				\co\bigcup_{i \ge 1} \Sigma_i \supseteq
				\co\bigcup_{i \ge 1} \Pi_{i-1} =
				\bigcup_{i \ge 0} \co\Pi_i =
				\bigcup_{i \ge 0} \Sigma_i =
				\PH
		\end{gather*}
	\end{proof}

	\begin{theorem}
		При $k \ge 1$:
		\begin{itemize}
			\item
				$L \in \Sigma_k$ тогда и только тогда, когда есть п.о. отношение $R \in \Pi_{k-1}$ такое, что:
				\[ x \in L \iff \exists y \colon R(x, y) \]
			\item
				$L \in \Pi_k$ тогда и только тогда, когда есть п.о. отношение $R \in \Sigma_{k-1}$ такое, что:
				\[ x \in L \iff \forall y \colon R(x, y) \]
		\end{itemize}
	\end{theorem}
	\begin{Rem}
		Это второе определение, через кванторы.
		Если его рекурсивно раскрыть, то получим, что $\Sigma_k$ "--- это чередующуяся цепочка из $k$ кванторов, начиная с $\exists$, а заканчивая п.о. п.п. отношением $R \in \P$.
		А $\Pi_k$ "--- то же самое, но начинается с $\forall$.
	\end{Rem}
	\begin{proof}
		Показываем эквивалентность индукцией по $k$, база при $k=0$ и $k=1$ очевидна.

		Переход от $k-1$ к $k$ (при $k \ge 2$).
		Сначала разбираемся с $\Sigma_{k}$.
		Надо показать в две стороны:
		\begin{description}
			\item[$\La$:]
				Пусть есть язык $L$ и отношение $R \in \Pi_{k-1}$ из условия.
				Построим машину из $\NP^R \subseteq \Sigma_k$, которая распознаёт язык $L$.
				Она будет недетерминированно угадывать $y$ и потом спрашивать у оракула корректность $R(x, y)$.
			\item[$\Ra$:]
				Пусть есть язык $L$ и НМТ $M$, использующая оракул для языка $L' \in \Sigma_{k-1}$
				(тут мы пользуемся тем, что $X^{\Pi_{k-1}} = X^{\Sigma_{k-1}}$).
				\begin{Rem}
					Если бы $M$ не использовала оракул для $\Sigma_{k-1}$, а использовал бы его только для $\Sigma_{k-2}$, то переход был бы очевиден:
					в качестве отношения $R(x, y)$ можно было бы взять следующее: машина $M$ принимает вход $x$ по подсказке $y$.
					Тогда это отношение бы лежало в $\P^{\Sigma_{k-2}} \subseteq \co\NP^{\Sigma_{k-2}} = \Pi_{k-1}$.
					Однако нам так не повезло, надо чуть аккуратнее.
				\end{Rem}
				По индукционному предположению о $\Sigma_{k-1}$ это значит, что есть такое п.о. отношение $R'\in \Pi_{k-2}$:
				\[ x' \in L' \iff \exists y' \colon R'(x', y') \]
				Формально говоря, $M$ принимает слово $x$ тогда и только тогда, когда есть последовательность недетерминированных выборов $y_0$,
				а также корректные ответы оракула для $L'$ на входах $x'_1, \dots, x'_l$ (где его запускала $M$): $a_1, \dots, a_l$.

				Давайте введём отношение $R(x, y)$, а в $y$ закодируем следующее:
				\begin{enumerate}
					\item Саму последовательность детерминированных выборов $y_0$
					\item Последовательность $a_i$
					\item Для каждого $a_i=1$ закодируем соответствующий $y'_i$ для отношения $R'$, позволяющий проверить ответ оракула для языка $L'$
				\end{enumerate}
				Осталось лишь сделать так, чтобы $R$ могло убедиться в том, что для $a_i=0$ ответ действительно ноль, т.е. что для всех подсказок $y'$ отношение $R'$ возвращает ноль.
				Скажем, что $R(x, y)$ имеет такой вид:
				\begin{enumerate}
					\item Для любых $y''_1, \dots, y''_l$:
					\item Для всех $i$ таких, что $a_i=1$ верно $R'(x'_i, y'_i)=1$
					\item Для всех $i$ таких, что $a_i=0$ верно $R'(x'_i, y''_i)=0$
					\item НМТ $M$ принимает вход $x$ по подсказке $y$, если ответы оракула на входах $x_i'$ равны $a_i$
				\end{enumerate}
				Заметим, что это отношение лежит в $\co\NP^{R'}$ по построению "--- мы требуем принятие по всем подсказкам, иногда вызываем оракул для $R'$, и делаем дополнительную полиномиальную работу на последнем шаге.
				Т.о. $R \in \co\NP^{\Pi_{k-2}} = \Pi_{k-1}$, что и требовалось показать.
		\end{description}
		Аналогичным образом должно получаться и с $\Pi_k$.
		\TODO
	\end{proof}

\subsection{Полные задачи}
	\begin{Def}
		Язык $\QBF_k$ состоит из замкнутых формул в предварённой форме с $k$ чередующимися кванторами, начиная с $\exists$,
		а под всеми кванторами "--- формулама в ДНФ или КНФ.
	\end{Def}
	\begin{assertion}
		Язык $\QBF_k$ является $\Sigma_k$-полным.
	\end{assertion}
	\begin{proof}
		Возьмём произвольный язык $L \in \Sigma_k$.
		Мы хотим построить полиномиально вычислимую функцию $f$ такую, что:
		\[ x \in L \iff f(x) \in \QBF_k \]
		По определению $\Sigma_k$ мы знаем, что есть такое п.о. п.п. $R \in P$, что:
		\[ x \in L \iff \exists x_1 \forall x_2 \dots Q_k x_k \colon R(x, x_1, \dots, x_k) \]
		Хочется каким-то образом преобразовать $R$ в формулу.
		Если бы нам разрешалось делать не формулы, а схемы, то мы бы могли, как в билете 4, построить схему квадратичного размера,
		которой на вход подаются $x_1, \dots, x_k$, в неё саму зашит $x$, и она вычисляет $R$.
		К сожалению, мы не умеем переделывать схемы в формулы напрямую "--- они экспоненциально разрастаются.
		Однако мы умеем (как в билете 4) делать это с добавлением промежуточных переменных, т.е. можно построить такую $\phi$ полиномиального размера, что:
		\[ R(x, x_1, \dots, x_k) \iff (\exists y \colon \phi(x, x_1, \dots, x_k, y)) \]

		Таким образом, если если $k$ нечётно, то последний квантор в формуле будет квантором существования и мы можем просто навесить ещё один,
		не сбивая чередование, получим честную формулу для $\QBF_k$.

		А вот если $k$ чётно и последний квантор "--- квантор всеобщности, то надо поступить хитрее.
		Надо записать $\lnot R$ в виде булевой формулы $\phi'$:
		\begin{gather*}
			\lnot R(x, x_1, \dots, x_k) \\
			\Updownarrow \\
			\exists y \colon \phi'(x, x_1, \dots, x_k, y) \\
			\Updownarrow \\
			\lnot \forall y \colon \lnot \phi'(x, x_1, \dots, x_k, y) \\
			R(x, x_1, \dots, x_k) \iff \forall y \colon \lnot \phi'(x, x_1, \dots, x_k, y)
		\end{gather*}
		Опять получим несбитое чередование кванторов, что и требуется.
	\end{proof}

\subsection{Коллапс}
	\begin{assertion}
		Если $\PH=\Sigma_k$, то $\PH=\Pi_k$
	\end{assertion}
	\begin{proof}
		Мы знаем, что $\co\PH = \PH$.
		\begin{align*}
			\PH &= \Sigma_k \\
			\co\PH &= \co\Sigma_k \\
			\PH &= \Pi_k
		\end{align*}
	\end{proof}

	\begin{theorem}
		Если $\Sigma_k=\Pi_k$ (при $k>0$), то $\PH=\Sigma_k=\Pi_k$ (иерархия коллапсирует до уровня $k$)
	\end{theorem}
	\begin{proof}
		Сначала покажем, что $\Sigma_{k+1} = \Pi_k$.
		Мы точно знаем нестрогое включение ($\Sigma_{k+1} \supseteq \Pi_k$, из определения с оракулами), покажем во вторую сторону.
		Пусть $L \in \Sigma_{k+1}$, т.е. есть такое $R \in \Pi_k=\Sigma_k$:
		\[ L = \{ x \colon \exists y \colon R(x, y) \} \]
		Так как $R \in \Sigma_k$, то есть такое $R' \in \Pi_{k-1}$, что:
		\[ R(x, y) \iff \exists z \colon R'(x, y, z) \]
		Значит:
		\[ x \in L \iff \exists (y, z) \colon R'(x, y, z) \]
		Отсюда немедленно $L \in \Sigma_k$.

		Теперь делаем следствия:
		\begin{gather*}
			\Sigma_{k+1} = \Pi_k = \Sigma_k \\
			\co\Sigma_{k+1} = \co\Pi_k \\
			\Pi_{k+1} = \Sigma_k = \Pi_k \\
			\Sigma_{k+1} = \NP^{\Sigma_k} = \Sigma_k \\
			\Pi_{k+1} = \co\NP^{\Sigma_k} = \Pi_k
		\end{gather*}
		Из последних двух строк автоматически получаем коллапс полиномиальной иерархии:
		\[ \Sigma_{k+i} = \Sigma_k = \Pi_k = \Pi_{k+i} \]
	\end{proof}

	\begin{theorem}
		Если $\Sigma_k=\Sigma_{k+1}$, то $\PH=\Sigma_k$ (иерархия коллапсирует до $\Sigma_{k}$)
	\end{theorem}
	\begin{proof}
		Как и в предыдущей теореме, достаточно показать, что $\NP^{\Sigma_k}=\Sigma_k$ и $\co\NP^{\Sigma_k}=\Pi_k$,
		тогда рекурсивное определение зациклится.

		Первое нам дано в условии:
		\[ \Sigma_k=\Sigma_{k+1}=\NP^{\Pi_k}=\NP^{\Sigma_k} \]
		Второе из него следует:
		\[
			\co\NP^{\Sigma_k} = \Pi_{k+1} = \co\Sigma_{k+1} = \co\Sigma_k = \Pi_k
		\]
	\end{proof}
	\begin{conseq}
		Как следствие, иерархия коллапсирует до уровня $k$, так как если $\PH=\Sigma_k$, то $\PH=\Pi_k$.
	\end{conseq}

	\begin{conseq}
		Если есть $\Sigma_{k+1}$-полная задача, лежащая в $\Sigma_k$, то иерархия коллапсирует до уровня $k$.
	\end{conseq}
	\begin{proof}
		Так как любая задача из $\Sigma_{k+1}$ сводится к задаче из $\Sigma_k$, то $\Sigma_{k+1} \subseteq \Sigma_k$
		(так как уровни иерархии замкнуты относительно сведения).
		С другой стороны, $\Sigma_{k=1} \supseteq \Sigma_k$, стало быть, $\Sigma_{k+1} = \Sigma_k$ и иерархия коллапсирует до $\Sigma_k$.
	\end{proof}

	\begin{conseq}
		Если существует $\PH$-полная задача, то иерархия коллапсирует.
	\end{conseq}
	\begin{proof}
		Эта $\PH$-полная задача лежит в конкретном уровне иерархии.
		Так как $\Sigma_{k+1}\supseteq \Pi_k$, то можно считать, что эта задача лежит в некотором $\Sigma_k$.
		Так как она $\PH$-полная, то все задачи из $\Sigma_{k+1}$ к ней сводятся и иерархия коллапсирует до $\Sigma_k$.
	\end{proof}

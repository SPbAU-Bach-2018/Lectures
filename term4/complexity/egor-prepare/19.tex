\setcounter{section}{18}
\section{Билет 19}
	\begin{theorem}
		$\BPP \subseteq \Ppoly$
	\end{theorem}
	\begin{proof}
		Возьмём задачу распознавания $L \in \BPP$.
		По определению $\BPP$ имеется п.о. п.п. бинарное отношение $R$ такое, что вероятность $\Pr_w \{ (x, w) \in R \}$ разная для случаев
		$x \in L$ и $x \notin L$.
		Давайте зафиксируем длину $x$ ($|x|=n$) и построим схему, которая будет корректно работать со словами длины $n$.
		Мы хотим найти одну подсказку $w$, на которой отношение $R$ будет вести себя безошибочно, а потом зашить эту подсказку в схему.

		Для этого давайте сделаем из $R$ отношение $R'$ с пониженной ошибкой, которое будет запускать $R$ полиномиальное число раз и выбирать ответ, который более вероятен.
		Мы знаем, что можно понизить вероятность ошиби $R'$ до $\frac{1}{2^{poly(n)}}$ для произвольного полинома, давайте тогда выберем в качестве полинома $2n$.
		Тогда для $R'$ вероятность ошибиться на конкретном входе (в любую сторону) будет не более $2^{-2n}$:
		\[
			\forall x \colon \Pr_w\{ R'(x, w) \neq L(x) \} < 2^{-2n} \\
		\]
		Обозначим за $W$ суммарное число подсказок (они ограничены по длине полиномом, так как \TODO).
		Тогда:
		\begin{gather*}
			\forall x \colon \frac{|\{ w \mid R'(x, w) \neq L(x) \}|}{W} < 2^{-2n} \\
			\forall x \colon |\{ w \mid R'(x, w) \neq L(x) \}| < 2^{-2n}W \\
			\sum_x |\{ w \mid R'(x, w) \neq L(x) \}| < 2^n \cdot 2^{-2n}W \\
			|\{ x, w \mid R'(x, w) \neq L(x) \}| < 2^{-n}W \\
			\sum_w |\{ x \mid R'(x, w) \neq L(x) \}| < 2^{-n}W \\
			\frac{\sum_w |\{ x \mid R'(x, w) \neq L(x) \}|}{W} < 2^{-n} \\
			\text{среднее значение не может быть меньше минимального значения} \\
			\exists w \colon |\{x \mid R'(x, w) \neq L(x)\}| < 2^{-n}
		\end{gather*}
		Так как мощность множества "--- целое число, то существует подсказка $w$ такая, что $R'(x, w)=L(x)$ для всех входов $x$ фиксированной длины.
		Давайте тогда преобразуем $R'$ в схему полиномиального размера, а случайные биты положим константами из $w$.
	\end{proof}

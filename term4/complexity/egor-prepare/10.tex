\setcounter{section}{9}
\section{Билет 10}
	\begin{theorem}
		Если $\NP \subseteq \Ppoly$, то $\PH = \Sigma_2$ (иерархия коллапсирует до $\Sigma_2$)
	\end{theorem}

\subsection{Идея}
	Покажем, что $\Sigma_3$-полный язык $\QBF_3$ лежит в $\Sigma_2$, этого хватит для коллапса.
	Нам надо построить некоторое отношение $R \in P$ такое, что:
	\[ x \in QBF_3 \iff \exists \alpha \colon \forall \beta \colon R(x, \alpha, \beta) \]

	Давайте рассуждать (точное решение будет ниже).
	Пусть нам дали формулу $F(x, y, z)$, хотим определить истинность следующего выражения (это задача $\QBF_3$):
	\[ \exists x \colon \forall y \colon \exists z \colon F(x, y, z) \]
	\begin{Rem}
		Напоминаем, что $F$ "--- это булева формула от многих переменных,
		а $x$, $y$, $z$ "--- какие-то конечные множества булевых переменных.
		Очевидно, что $|x|+|y|+|z|\le|F|$, иначе какая-то переменная в формуле не встречается и её можно убрать.
	\end{Rem}
	Заметим, что выражение $\exists z \colon F(x, y, z)$ "--- это в точности какая-то задача $A$ из $\NP$ с условием $(F, x, y)$ и поиском решения $z$.
	Значит, по предпосылке теоремы, она лежит в $\Ppoly$, т.е. есть какое-то семейство схем $C_0, C_1, C_2, \dots$ такое, что:
	схема $C_i$ решает $A$ для входов длины $i$:
	\[
		\exists z \colon F(x, y, z) \iff (F, x, y) \in A \iff C_{|(F, x, y)|}((F, x, y)) = 1
	\]
	К сожалению, мы не можем эти схемы напрямую использовать, несмотря на то, что $A$ "--- какая-то фиксированная задача, ни от чего не зависящая.
	Никто не сказал, что этим самые схемы можно конструктивно построить.

	Схитрим: запихнём посторение этих схем в квантор существования.
	Их бесконечно много, но мы можем сказать, что нас интересуют только схемы до размера $\O(|F|)$, просто потому что так ограничен размер тройки $(F, x, y)$.
	Так как задача $A \in \Ppoly$, мы заранее знаем, каким полиномом ограничен размер каждой $C_i$, значит, можем завести нужное число переменных в формуле.
	Важно ещё как-то проверить, что имеющиеся схемы действительно решают задачу $A$.
	Это несложно сделать, так как схемы с небольшим входом можно просто встроить в формулу,
	а схемы с б\'ольшими входами можно рекурсивно проверять через схемы с меньшими, например,
	должно быть верно следующее:
	\[ C(\forall x \colon \phi(x))) = C(\phi[x \coloneq 0]) \land C(\phi[x \coloneq 1]) \]

\subsection{Точное решение}
	Нам надо показать, что есть такое п.о. п.п. отношение $R$, что:
	\[ x \in \QBF_3 \iff \exists a \colon \forall b \colon R(x, a, b) \]
	Перефразируя, надо построить такое $R$, что для любого $F$:
	\begin{gather*}
		\exists a \colon \forall b \colon \exists c \colon F(a, b, c) \\
		\Updownarrow \\
		\exists \alpha \colon \forall \beta \colon R(F, \alpha, \beta)
	\end{gather*}
	\begin{Rem}
		Тут латинскими буквами будут обозначаться последовательности из фиксированного числа
		булевых переменных (как в $\QBF$ и вариациях), а греческими "--- битовые строки
		(подсказки для машин из полиномиальной иерархии).
	\end{Rem}

	Введём массовую задачу $A$ "--- это множество таких троек $(F, a, b)$
	(тут $F$ "--- некоторая булева формула, а $a$ и $b$ "--- значения переменных для подстановки),
	что $\exists c \colon F(a, b, c)$.
	Очевидно, она лежит в $\NP$, так как по подсказке ($c$) можно проверить решение.
	Значит, существует некоторое семейство формул полиномиального размера $C_0, C_1, C_2, \dots$ такое, что:
	\begin{gather*}
		\exists c \colon F(a, b, c) \\
		\Updownarrow \\
		C_{|(F, a, b)|}((F, a, b)) = 1
	\end{gather*}
	Раз семейство формул из $\Ppoly$, то есть такой многочлен $p$, что $|C_i| \le p(i)$.
	Значит, если мы рассмотрим схемы $C_0, \dots, C_{|F|+|a|+|b|}$, то их можно закодировать битовой
	строкой $\gamma$ полиномиальной (от $|F|$) длины.

	Введём специальное п.п. отношение $T(\gamma, \phi)$, где $\phi$ "--- произвольная формула длины не более $|F|$,
	использующая не более $|a|+|b|$ переменных (если использует больше, то полагаем $T=1$).
	Это отношение проверяет, что описание схем $\gamma$ корректно решает $A(\phi)$ в предположении,
	что для меньших формул ответ вычисляется корректно:
	\begin{itemize}
		\item Если в $\phi$ нет переменных, то это какая-то константа, надо подставить в схему и проверить, что константа вычислена правильно
		\item
			Если в $\phi$ есть переменные, то есть какой-то внешний квантор по переменной $x$.
			Надо вычислить $C(\phi)$, $C(\phi[x \coloneq 0])$ и $C(\phi[x \coloneq 1])$ и проверить, что
			при комбинации последних двух значений получается первое.
			Комбинация "--- либо конъюнкция, либо дизъюнкция, в зависимости от квантора.
			Если это правда, то $C(\phi)$ вычисляется правильно, если оба значения от меньших формул
			вычислены правильно.
	\end{itemize}
	\begin{Rem}
		$T$ "--- это такой индукционный переход для доказательства корректности семейства схем $\gamma$.
	\end{Rem}
	Получаем следующую эквивалентность:
	\begin{gather*}
		\exists a \colon \forall b \colon \exists c \colon F(a, b, c) \\
		\Updownarrow \\
		\exists \gamma, a \colon \forall \phi, b \colon C_{|(F, a, b)|}(F, a, b) \land T(\gamma, \phi)
 	\end{gather*}
 	Мы уже говорили, что $\gamma$ и $\phi$ имеют полиномиальный размер относительно $|F|$, а
 	выражение под кванторами тоже вычисляется за полином от $|F|$.
 	Так что мы как раз решили задачу $\QBF_3$ в $\Sigma_2$, что и требовалось.

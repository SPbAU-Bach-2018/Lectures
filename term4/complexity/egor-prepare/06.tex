\setcounter{section}{5}
\section{Билет 6}
	\begin{Def}
		Оракульная машина Тьюринга с оракулом для произвольной функции $f$ (не обязательно вычислимой) "--- это обычная машина Тьюринга (детерминированная или недетерминированная "--- неважно),
		у которой добавляется дополнительный переход <<вызови оракула для функции $f$ на такой ленте и запиши ответ на такую ленту>>.

		Обозначается как $M^f$ "--- машина $M$, у которой есть оракул $f$.
	\end{Def}
	\begin{exmp}
		Например, можно в качестве оракула взять характеристическую функцию языка $\mathsf{3-SAT}$, тогда наша МТ сможет за один шаг узнавать выполнимость
		произвольной формулы в 3-КНФ.
	\end{exmp}
	\begin{Rem}
		Можно брать несколько оракулов, не меняя определения: сказать, что у нас на вход оракулу помимо данных подаётся ещё и номер задачи, которую надо решать.
	\end{Rem}
	\begin{Rem}
		Часто пишут $M^{C}$, где $C$ "--- целый класс языков.
		Подразумевается, что для каждой машины берётся какой-то один язык из $C$.
	\end{Rem}

	\begin{Def}
		Если $\mathcal C$ и $\mathcal D$ "--- классы языков, то можно ввести класс $\mathcal C^{\mathcal D}$, который состоит из языков вида
		$C^D$, где $D \in \mathcal D$ (язык), а $C$ "--- решатель для некоторого языка из $\mathcal C$.
	\end{Def}
	\begin{exmp}
		$\P^A$ "--- класс языков, распознаваемых ДМТ с оракулом $A$ за полиномиальное время.
	\end{exmp}
	\begin{exmp}
		$\NP^A$ "--- класс языков, распознаваемых НМТ с оракулом $A$ за полиномиальное время.
	\end{exmp}
	\begin{assertion}
		Если $A \subseteq B$ "--- классы языков, то $C^{A} \subseteq C^{B}$.
	\end{assertion}
	\begin{proof}
		Должно быть очевидно "--- любой решатель с оракулом из $C^A$ должен лежать в $C^B$.
	\end{proof}
	\begin{Rem}
		\textbf{Опасный момент}: то, что стоит внизу в обозначении оракула "--- не конкретный класс,
		а лишь модель вычислений, в которую добавляется оракул.
		В случае $\P$ это ДМТ, в случае $\NP$ это НМТ.
		Там можно поставить и другие классы со своими моделями вычислений.
		Однако не надо думать про это как про <<основание>> степени: например, даже если $\P=\NP$, то вовсе
		необязательно $\P^A = \NP^A$, модели вычислений-то совсем разные, а добавление оракула их ещё больше меняет.
	\end{Rem}

	\begin{theorem}
		Есть такой язык $A$, что $\P^A = \NP^A$.
	\end{theorem}
	\begin{proof}
		\TODO
	\end{proof}

	\begin{theorem}
		Есть такой язык $B$, что $\P^B \neq \NP^B$.
	\end{theorem}
	\begin{proof}
		\TODO
	\end{proof}

\setcounter{section}{5}
\section{Билет 6}
	\begin{Def}
		Оракульная машина Тьюринга с оракулом для произвольной функции $f$ (не обязательно вычислимой) "--- это обычная машина Тьюринга (детерминированная или недетерминированная "--- неважно),
		у которой добавляется дополнительный переход <<вызови оракула для функции $f$ на такой ленте и запиши ответ на такую ленту>>.

		Обозначается как $M^f$ "--- машина $M$, у которой есть оракул $f$.
	\end{Def}
	\begin{exmp}
		Например, можно в качестве оракула взять характеристическую функцию языка $\mathsf{3-SAT}$, тогда наша МТ сможет за один шаг узнавать выполнимость
		произвольной формулы в 3-КНФ.
	\end{exmp}
	\begin{Rem}
		Можно брать несколько оракулов, не меняя определения: сказать, что у нас на вход оракулу помимо данных подаётся ещё и номер задачи, которую надо решать.
	\end{Rem}
	\begin{Rem}
		Часто пишут $M^{C}$, где $C$ "--- целый класс языков.
		Подразумевается, что для каждой машины берётся какой-то один язык из $C$.
	\end{Rem}

	\begin{Def}
		Если $\mathcal C$ и $\mathcal D$ "--- классы языков, то можно ввести класс $\mathcal C^{\mathcal D}$, который состоит из языков вида
		$C^D$, где $D \in \mathcal D$ (язык), а $C$ "--- решатель для некоторого языка из $\mathcal C$.
	\end{Def}
	\begin{exmp}
		$\P^A$ "--- класс языков, распознаваемых ДМТ с оракулом $A$ за полиномиальное время.
	\end{exmp}
	\begin{exmp}
		$\NP^A$ "--- класс языков, распознаваемых НМТ с оракулом $A$ за полиномиальное время.
	\end{exmp}
	\begin{assertion}
		Если $A \subseteq B$ "--- классы языков, то $C^{A} \subseteq C^{B}$.
	\end{assertion}
	\begin{proof}
		Должно быть очевидно "--- любой решатель с оракулом из $C^A$ должен лежать в $C^B$.
	\end{proof}
	\begin{Rem}
		\textbf{Опасный момент}: то, что стоит внизу в обозначении оракула "--- не конкретный класс,
		а лишь модель вычислений, в которую добавляется оракул.
		В случае $\P$ это ДМТ, в случае $\NP$ это НМТ.
		Там можно поставить и другие классы со своими моделями вычислений.
		Однако не надо думать про это как про <<основание>> степени: например, даже если $\P=\NP$, то вовсе
		необязательно $\P^A = \NP^A$, модели вычислений-то совсем разные, а добавление оракула их ещё больше меняет.
	\end{Rem}

	\begin{theorem}
		Есть такой язык $A$, что $\P^A = \NP^A$.
	\end{theorem}
	\begin{proof}
		Пусть $A$ "--- это язык из троек $(M, x, 1^n)$, причём ДМТ $M$ принимает вход $x$ быстрее, чем за $2^n$ шагов.
		\begin{Def}
			$\EXP$ "--- это все языки, распознаваемые ДМТ за время $2^{poly(|x|)}$:
			\[ \EXP = \bigcup_{poly(n)} \DTime[2^{poly(n)}] \]
		\end{Def}
		\begin{assertion}
			$A \in \EXP$.
		\end{assertion}
		\begin{proof}
			Просто запускаем ДМТ на входе, она делает не более $t\coloneq 2^n$ шагов, а моделировать МТ мы умеем
			за $\O(t^2)=\O(2^{2n})=\O(2^{poly(n)})$.
		\end{proof}
		Покажем, что $\EXP \subseteq \P^A \subseteq \NP^A \subseteq \EXP$, откуда сразу $\P^A=\NP^A=\EXP$:
		\begin{description}
			\item[$\EXP\subseteq \P^A$:]
				Есть язык $L \in \EXP$ и пусть нам дали слово $x$.
				По определению $\EXP$ есть машина, которая принимает слова из языка за $2^{poly(|x|)}$, можно вычислить этот $poly(|x|)$
				и спросить у оракула, примет ли машина слово за такое время "--- это и будет ответом.
			\item[$\P^A\subseteq \NP^A$:]
				Очевидно, так как любая ДМТ с оракулом также является НМТ с оракулом и распознаёт тот же самый язык.
			\item[$\NP^A\subseteq \EXP$:]
				Пусть есть язык $L \in \NP^A$.
				Пусть нам дали слово $x$.
				Мы знаем, что если есть подсказка для НМТ, то она не длиннее, чем $p(|x|)$ ($p$ зависит только от $L$).
				Переберём все такие подсказки за $2^{p(|x|)}$, каждую проверим на корректность.
				Для проверки корректности требуется промоделировать не более $q(|x|)$ шагов НМТ, на каждом шаге может потенциально
				быть запущен оракул для входа размером не более $q(|x|)$, который отработает за $2^{q_1(q(|x|))}$, где $q_1$ "--- фиксированный многочлен.
				Значит, итоговое время работы будет не более:
				\[
					2^{p(|x|)} \cdot q(|x|) \cdot 2^{q_1(q(|x|))} \le
					2^{p(|x|)+q(|x|)+q_1(q(|x|))}
				\]
				Т.е. двойка в степени полином, что и требовалось.
		\end{description}
	\end{proof}

	\begin{lemma}
		Возьмём произвольный язык $B$ (возможно даже неразрешимый) и построим унарный язык $U_B$:
		\[ U_B = \{ 1^{|x|} \mid x \in B \} \]
		Т.е. это все слова из $B$, в которых биты были заменены на единички.
		Тогда если $U_B \notin \P^B$, то $\P^B \neq \NP^B$.
	\end{lemma}
	\begin{proof}
		Достаточно заметить, что $U_B \in \NP^B$, тогда строгое включение будет доказано.
		Но это очевидно: можно считать, что НМТ требует в качестве подсказки для слова $1^n$ какое-нибудь слово $x$ длины $n$
		из $B$ и при помощи оракула проверяет, что оно действительно лежит в $B$.
		Также НМт надо проверить, что ему на вход подали одни единицы.
	\end{proof}

	\begin{theorem}
		Есть такой язык $B$, что $\P^B \neq \NP^B$.
	\end{theorem}
	\begin{proof}
		Воспользуемся предыдущей леммой и построим $B$ с нужным свойством ($U_B \notin \P^B$).
		Строить его будет по шагам, на каждом шаге у нас для конечного множества строк будет решено, лежат они в $B$ или нет.
		Изначально это не известно ни для одной из строк.
		Для произвольной строки мы либо добавим/исключим её в $B$ на каком-то шаге, либо она никогда не возникнет.
		Если она никогда не возникнет, то мы считаем, что она не лежит в $B$.

		Занумеруем машины с оракулом $B$ как $M_1, M_2, \dots$ (каждая машина имеет бесконечно много номеров).
		На $i$-м шаге мы будет достраивать $B$ так, чтобы машина $M_i$ не разрешала язык $B$ за время $\sfrac{2^n}{10}$ (при помощи диагонализации).
		Так как $M_i$ будет использовать оракул для $B$, то при каждом вызове оракула нам надо помещать или не помещать аргумент оракула в $B$.

		Итак, $i$-й шаг.
		У нас для конечного множества строк известно, лежат ли они $B$ или нет.
		Выберем натуральное $n$, большее, чем длины все известных строк.
		Запустим $M_i$ на входе $1^n$ в течение $\sfrac{2^n}{10}$ шагов.
		Если она запустит оракул на строке, про которую мы уже что-то знаем, мы обязаны ответить корректно.
		Если же оракул запускается на строке, про которую мы ещё ничего не знаем, мы начинаем считать, что эта строка в $B$ не лежит.
		После того, как машина $M_i$ закончила работать, ни про одну из строк длины $n$ мы ещё не могли решить, что она лежит в $B$
		(впрочем, про какие-то могли решить, что не лежат).
		Два варианта завершения работы $M_i$:
		\begin{enumerate}
			\item
				$M_i$ приняла строку $1^n$.
				Тогда говорим, что все строки длины $n$ не лежат в $B$.
				Никаких противоречий не возникнет, так как мы про строки длины $n$ не говорили, что они лежат в $B$.
				Получавем, что $M_i$ сказала неправду про язык $U_B$.
			\item
				$M_i$ отвергла строку $1^n$.
				Тогда заметим, что точно существует строка длины $n$, про которую ещё ничего не известно (т.е. про неё не спрашивал оракул машины $M_i$),
				так как всего он спрашивал максимум про $\sfrac{2^n}{10}$ строк, а всего их $2^n$.
				Тогда скажем, что эта строка лежит в $B$.
				Получавем, что $M_i$ сказала неправду про язык $U_B$.
		\end{enumerate}

		Мы так построили какой-то язык $B$.
		Докажем условие леммы от противного.
		Предположим, что существует полиномиальная машина $M$, разрешающая язык $U_B$ за полиномиальное время.
		Тогда при достаточно большом $k$ она начнёт отрабатывать за время, меньшее $\sfrac{2^k}{10}$.
		Тогда посмотрим на первый шаг после $k$-го, на котором встретилась эта машина $M$.
		На этом шаге машина точно завершилась, но по построению языка $B$ она выдала неправильный ответ.
		Противоречие.

		Условие леммы доказано, стало быть, $\P^B \neq \NP^B$.
	\end{proof}

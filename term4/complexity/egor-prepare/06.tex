\setcounter{section}{5}
\section{Билет 6}
	\begin{Def}
		Оракульная машина Тьюринга с оракулом для произвольной функции $f$ (не обязательно вычислимой) "--- это обычная машина Тьюринга (детерминированная или недетерминированная "--- неважно),
		у которой добавляется дополнительный переход <<вызови оракула для функции $f$ на такой ленте и запиши ответ на такую ленту>>.

		Обозначается как $M^f$ "--- машина $M$, у которой есть оракул $f$.
	\end{Def}
	\begin{exmp}
		Например, можно в качестве оракула взять характеристическую функцию языка $\mathsf{3-SAT}$, тогда наша МТ сможет за один шаг узнавать выполнимость
		произвольной формулы в 3-КНФ.
	\end{exmp}
	\begin{Rem}
		Можно брать несколько оракулов, не меняя определения: сказать, что у нас на вход оракулу помимо данных подаётся ещё и номер задачи, которую надо решать.
	\end{Rem}

	\TODO обозначения $\NP^{\NP}$ и почему нельзя менять нижний класс

	\begin{theorem}
		Есть такой язык $A$, что $\P^A = \NP^A$.
	\end{theorem}

	\begin{theorem}
		Есть такой язык $B$, что $\P^B \neq \NP^B$.
	\end{theorem}

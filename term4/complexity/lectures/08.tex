\section{Хеширование}        
Сегодняшняя лекция будет про хеширование, которое применяется в разных местах теории сложности. 

Нам потребуется определить семейство попарно независимых хешфункций. 
Мы рассматриваем функции $\{0, 1\}^n \to \{0, 1\}^k$, которы по n битов сделают k битов и 
хочется, что бы эти функции разбрасывали строчки достаточно равномерно, 
при этом не хочется, что бы этих функций было сверх много, но при этом много. Нужно,
что бы при взятии хешфункции случайнм образом склеивающиеся пары были равномерно распределены.

\begin{Def}
	Хочется иметь такое семейство функций $H$ такое, что если я буду брать случайным образом из 
	этого семейства хешфункцию($h \ra random(H)$), то 
	$\forall x, y P\{h(x) = y\} = \frac{1}{2^k}$.\\
	Вероятность, что мы отправим конкретную строчку в конкретную другую строчку она одинаковая.\\ 

	$\forall x, y, x', y', x\ne x' P\{h(x) = y, h(x') = y'\} = \frac{1}{2^{2k}}$\\
	А события что мы две разные строчки куда-то отправили должны быть независимые.\\ 

	Такое семейство называется семейством попарно независимых хешфункций. 
\end{Def}

Пусть $k = n$, тогда давай-те построим такое семейство, а из него
мы семейство для всех остальных k сделаем запросто.\\

Семейство у нас будет проиндексировано двумя индексами. Эти индексы будут 
строчками длин n из поля с $2^n$ элементами.\\ 
$a, b \in GF(2^n)$, 
$\{0, 1\}^{n} \lra GF(2^n)$\\

\begin{theorem}
	Хешфункция будет выглядить очень банальным образом, а именно:
	$h_{a, b}(x) = ax + b$\\

	Мы только что привели пример семейства попрно независимых хешфункций.
\end{theorem}
\begin{proof}
	Давай-те посмотрим, что у нас происходит со вторым условием. 
	Путсь $x \ne x'$ и функция x отправила в y, а x' в y'.
	
	Тогда имеет место система:
	$$\left\{
	\begin{aligned}
	ax + b = y \\
	ax' + b = y'
	\end{aligned}
	\right.
	$$

	Из этой системы мы можем однозначно востановить $a = \frac{y - y'}{x - x'}$ и $b$\\
	
	Из этого мы понимаем, что если случилась конъюнкция таких событий, то мы точно знаем a и b. 

	Тогда вероятность этого события какая надо.
	$$P = \left\{
	\begin{aligned}
	ax + b = y \\
	ax' + b = y'
	\end{aligned}
	\right. = \frac{1}{2^{2n}}
	$$

	Поскольку вероятность выбрать такое $a = \frac{1}{2^n}$, вероятность выбрать такое $b = \frac{1}{2^n}$ и выбираем мы a и b независимо. 
 
	Теперь вернемся к первому пункту. Эта функция не биекция. Нам вообще не это нужно. Зафиксировали y и x, ищем 
	подходящую хешфункцию.
	
	$ax + b = y$\\

	Какое бы мы не выбрали $a$, $b = y - ax$, то есть b выбирается однозначно, 
	значит, вероятность угадать b: $\frac{1}{2^n}$\\
\end{proof} 
\begin{theorem}
	В случе $k < n$ можно просто образать результат хешфункции до k бит и при этом 
	нужные нам условия будут выполняться.
\end{theorem}
\begin{proof}
	Если $k < n$, то обрежим последние биты у хешфункции [...k..n|...]. Каждой хешфункции соответсвует $2^{n - k}$
	прежних хешфункции и собрав по этим значениям целое, мы будем получать все тоже самое. 


	h(x) = yt\\
	Функция h переводит из x в у с хвостиком t. Функция h' этот хвстик обрубает.  

	\begin{enumerate}
	\item 
		$P\{h'(x) = y\} \le \sum_{|t| = n - k} P\{h(x) = yt\} = 2^{n - k}\frac{1}{2^n} = \frac{1}{2^k}$\\
		Вероятность явно не больше, чем сумма вероятностей по всем хвостикам, даже если события зависимы, 
		при этом если мы знаем, что вероятность у каждой функции не больше, чем $\frac{1}{2^k}$, а сумма 1, 
		то вероятность ровно $\frac{1}{2^k}$.
	\item	
		$P \{h'(x) = y \wedge h'(x') = y'\} = P\{\vee_{t, t'} h(x) = yt \wedge h(x') = y't' \}\\
		= \sum_{t, t'} P\{h(x) = yt \wedge h(x') = y't' \} = 2^{2(n - k)}\frac{1}{2^{2n}} = \frac{1}{2^{2k}}$\\

		Мы воспользовались тем, что если мы берем две разные строчки, то эти события дизъюнктны. И поэтому 
		дизъюнкция в точности равна сумме. 
	\end{enumerate}
\end{proof}

\subsection{Лемма Вэлианта-Вазирани}

Знаем мы про задачу выполнимости булевой формулы. Знаем, что она NP-полная и мы 
ее не умеем решать. 

То есть мы по произвольной формуле не умеем находить выполняющий набор. 
$x \colon F[x] = 1$\\

Но вдруг, если у формулы ровно один выполняющий набор, то его 
найти проще. Сейчас, мы докажем, что не проще. 

\begin{theorem}
	$SAT \to Unique-SAT$\\
	Решение задачи USAT не проще, чем решение задачи SAT.
\end{theorem}
\begin{proof}
	А точнее мы построим следующее вероятностное сведение:
	Возьмем формулу и чего-нибудь к ней допишем. Так, что если формула
	была невыполнимой то она и останется невыполнимой, ну просто потому что мы только больше ограничений к ней 
	дописали.

	А если формула была выполнимой, то останится ровно один выполняющий набор с большой вероятностью.  
	А допишим мы, что случайная хешфункция из нашего универсального семейства равна $0^k$. Как выбрать k еще не знаем.
	$F \wedge (h(x) = 0\cdots0)$\\

	То есть сведение выглядит следующим образом, взяли формулу $F$, которая дана нам на вход, берем случайную
	хешфункцию $h$ и записываем  $F \wedge (h(x) = 0\cdots0)$, где x это переменные в формуле. Если их записать подряд и 
	присвоить 0 и 1, то получится обычная строчка. 

	Понятно, что выполняющих наборов у этой формулы меньше, чем у $F$, хочется доказать, что их с большой вероятностью 1.	

	$S$ "--- множество выполняющих наборов для формулы $F$
	Давай-те выбирем k так, что бы выполнялись неравенства $2^{k - 2} < |S| \le 2^{k - 1}$.\\

	k возьмем случайным образом, тогде с вероятностью $\frac{1}{n + 1}$ мы нужное k угадаем.\\

	Утверждается, что если мы угалали $k$, то с вероятностью $\frac{1}{8}$ хотя бы, если формула была 
	выполнимая, то в новой один выполняющий набор.

	Нужно оценить вероятность того, что мы не убъем все выполняющие наборы, но тем неменее убъем почти все. 
	$P\{\text{остается ровно 1}\} \ge P\{\text{хотя бы 1}\} - P\{> 1\}$\\
	
	Давай-те оценим вероятнось каждой из этих штук. Как оценить, что остался хотя бы один выполняющий набор.
	Вероятность, что выжило больше одного не больше, чем чем сумма по всем парам, что оба выжили.
	$P\{> 1\} \le \sum_{x,x' \in S} P\{h(x) = 0 \wedge h(x') = 0\} = \binom{|S|}{2}\frac{1}{2^{2k}}$\\

	Эту вероятнусть нужно оценить сверху. Оценим п формуле включения-исключения. 
	$P\{\text{хотя бы 1}\} = \sum_{x \in S}P_{x\in S}\{h(x) = 0\} - sum_{x,x' \in S} P\{h(x) = 0 \wedge h(x') = 0\} \ge |S| \frac{1}{2^k} - \binom{|S|}{2}\frac{1}{2^{2k}} \\
	= \frac{|S|}{2^k} - |S|(|S| - 1)\frac{1}{2^{2k}} \ge \frac{|S|}{2^k}(1 - \frac{|S| - 1}{2^k}) = \frac{1}{8} $
        
        И того, наша процедура выдает формулу с одним выполняющим набором хотя бы с вероятностью $\frac{1}{8(n + 1)}$.\\

	Но можно сделать другое сведение, которое по произвольной формуле генерирует n + 1 другую формулу
	$F_1\cdots F_{n + 1}$, тогда с вероятностью хотя бы $\frac{1}{8}$ хотя бы одна из них имеет ровно один 
	выполняющий набор. Это будут F для разных значений k. То есть здесь мы просто перебрали все возможные
	значения $k$. 

	Если где-то заведется оракул, который умеет решать USAT, тогда мы вероятностно сможем решить SAT. 

	Много раз повторим и вероятность ошибки потихоньку станет константной. 
	$(1 - \frac{1}{8(n + 1)})^{8(n + 1} \to \frac{1}{e}$\\
\end{proof}

\subsection{Протокол Гольдвассер-Сипсера}
При помощи хеширования оценим размер множества. \\

\begin{theorem}
	Существует протокол из $AM$ для оценки размера множества. 
	Нас пытаются убедить в том, что мощность множества какая-та определенная. $2^{k - 2} \le |S| \le 2^{k - 1}$\\
\end{theorem}
\begin{proof}
	Артут захочит это множество захешировать в строчку длины k\\
	У нас есть множетсво $S \subset \{0, 1\}^{n}$ артур хочет захишировать в строчки длины k\\
	
	Далее Артур будет брать случаную строчку в образе и будет говорить Мерлину, в придъяви-ка мне прообраз 
	это точки из множества S.

       	Протокол такой:
       	\begin{enumerate}
       	\item Артур выберает случайным публичным образом хешфункцию и Артур выбирает случайным образом y.  
       	\item Мерлин должен найти такое x, что $x \in S$ и $h(x) = y$.
       	\item Артур потом детерменированно проверяет, что все верно. Артут умеет проверять, что $x \in S$. 
       	\end{enumerate}

	Это основано на том, что если S гораздо меньше чем нас убеждают, то у Мерлина будут трудности, потому что
	можем с большой вероятностью ткнуть в такую точку y, что у нее-то никакого праобраза в S то и нет.   

	\begin{lemma}
		$p = \frac{|S|}{2^k}$\\
		Пусть $|S| \le \frac{2^k}{2}$, тогда		 
		\begin{enumerate}
		\item                    	
			$Pr_{h,y}\{\exists x \in S \colon h(x) = y\} \le p$\\
		\item
			$Pr_{h,y}\{\exists x \in S \colon h(x) = y\} \ge \frac{3p}{4}$
	\end{enumerate}
	\end{lemma}
	\begin{proof}
		\begin{enumerate}
		\item Этот пункт должен быть более-менее очевидным.
		Взяли множество, его с помощью функции куда-то отправили. Если множество
		совсем маленькое, то оно не может занимать много место среди строчек размера $2^k$.

		Если множество размера $p\cdot 2^k$, то после того, как мы его захешировали сюда, то получилось
		не более, чем  $p\cdot 2^k$ строчек, а всего строчек $2^k$, то есть это банальность.  
		
		Первый пункт дает нам, что если множество сильно меньше, то у Мерлина очень мало шансов найти нужный x.
		\item 
	 	Хочется посмотреть, что у нас происходит с другой стороны.  	
	 	
	 	Временно зафиксируем y, h все еще случайно, давай-те смотреть на события $h(x) = y$.
	 	Будем так же это оценивать по формуле включения исключения. 
	 	$Pr_{h,y}\{\exists x \in S \colon h(x) = y\} \ge 
	 	\sum_{x \in S} P_h\{h(x) = y\} - \sum_{x, x', x < x' \in S} P_h\{h(x) = y, h(x') = y\} \\
	 	\ge |S|\frac{1}{2^k} - \binom{|S|}{2}\frac{1}{2^{2k}} = \frac{|S|}{2^k}(1 - \frac{|S| - 1}{2 \cdot 2^{k}}) \ge \frac{3p}{4}$ \\
		
		Это значит, у Мерлина есть покрайней мере такой шанс на успех.
		\end{enumerate}
	\end{proof}

	Теперь Мерлин может убедить нас, что множество достаточно большое. Вероятность того, что он нас обманет, она $p$.

	Честный Мерлин дает нам k, такое, что $2^{k - 2} \le |S| \le 2^{k - 1}$ \\
	Хороший Мерлин с вероятностью $\frac{3}{4}p \ge \frac{3}{16}$ нас убедит.

	Если Мерлин пытается нас сильно обмануть, то есть $|S| \le 2^{k - 3}$, то 
	вероятность того, что он нас убедит меньше, чем $\frac{1}{8}$.

	Если сделать этот эксперемент много раз, то мы сможем принять правильное решение с большой вероятностью.

        Если размер множество слишком большой, то мы уже решили, что в этом нас убеждать не будут. Мерлин доказывает нижнюю оценку для
        размера множества. 
\end{proof}

Теперь как это применить на пользу народного хозяйства? Если у нас нет приватных случайных бит, а только 
публичные, то мы можем протакол, который пользуется приватными переделать в публичные.

Пусть У нас V придумывает секретную случайную строчку r, P дает доказательство w, V после этого проверяет v(x, r, w) = 1.

Как переделать в публичный случайные биты. Вместо P пришел M и теперь убеждает Артура с публичными битами, что V с большой вероятностью
получил бы 1 в игре с P. То есть, что множество случайных строк приводящих к успеху достоточно большое.

Теперь $S = \{r \colon \exists w \colon v(x, r, w) = 1\}$. Теперь Мерлин принимает не только r, но и w. И артур проверяет, что все верно. 

 

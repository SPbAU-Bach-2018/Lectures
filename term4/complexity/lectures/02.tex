\section{Сводимости}
\subsection{Сводимость по Карпу}
Эта сводимость обычно имеется в виду в случае задач распознавания.
\begin{Def}
	Сведение по Карпу: 
	Язык $L_1$ сводится к языку $L_2$($L_1 \to L_2$) если, имеется полиномиально вычислимая f:
	$\forall x \colon x \in L_1 \Lra f(x) \in L_2$
\end{Def}

\begin{Rem}
	Мы не можем поменять местами ответ, то есть мы не можем сказать, что там где нет мы скажем да, а там где да сажем нет. 
	Это важно. Так как есть подсказка для второго языка, то эта же подсказка для первго языка, а подсказки у нас для "да".
\end{Rem}

\subsection{Сводимость по Левину}
Сводимость применяется в случае задач поиска. 
\begin{Def}
	Сведение задач по Левину:
	Задача $R_1$ сводится к задаче $R_2$($R_1 \to R_2$), если $\exists f, g, h \colon \forall x_1, y_1, y_2:$
	\begin{enumerate}
	\item $R_1(x_1, y_1) \Ra R_2(f(x_1), g((x_1, y_1)))$
   	\item $R_1(x_1, h((f(x_1), y_2))) \La R_2(f(x_1), y_2)$
    	\item $f$ и $h$ полиномиально вычислимы, а $g$ ограничена полиномом. 
	\end{enumerate}
\end{Def}
\begin{Rem}
	Функцию $g$ мы не собираемся запускать, но нам важно, что бы решение для второго случая 
	было полиномиально ограницено, если полиномиально ограничено в первом случае.  
\end{Rem}

Видимо, наше седение будет происходить следующим образом. Мы отображаем наше $x_1$ с помощью функции $f$, 
находим в $R_2$ нужные нам решения и отображаем их назад с помощью функции $h$.

\begin{Rem}
	Классы $P, NP, \tilde{P}, \tilde{NP}$ замкнуты относительно 
	этих сведений. 
\end{Rem}

\section{Полные задачи}	
\subsection{Определение полных задач}
\begin{Def}
	Задача A "--- трудная для класса C, $\forall B \in C \colon B \to A$.
	То есть любая задача из класса $C$ сводится к задаче $A$. 
\end{Def}

\begin{Def}
	Задача полная для класса C, если она трудная и принадлижит C.
\end{Def}

\begin{theorem}
	Если $A$ "--- NP-трудная и $A \in P$,
	то $P = NP$.
\end{theorem}
\begin{conseq}
	Если $A$ "--- NP-полная, то $A \in P \Lra P = NP$.
\end{conseq}

\subsection{Задача об ограниченной остановке}
\begin{Def}
	$1^t$ "--- обозначение для записи слова, состоящего из t единиц.
\end{Def}
\begin{Def}
	Задача об ограниченной остановке: 
	$\tilde{BH}(<M, x, 1^t>, w) =$ НМТ
	M с подсказкой w принимет вход $x$ за $\le t$ шагов.
\end{Def}
\begin{theorem}
	Задача об ограниченной остановке "--- $\tilde{NP}$-полная, 
	а соответствующей язык "--- NP-полный. 
\end{theorem}
\begin{Rem}
	Принадлежность $\tilde{NP}$ использует существование универсальной ДМТ, которая 
	может промоделировать вычисление ДМТ, описание которой дано ей на входе. Причем лишь с полиномиальным замедлением. 
\end{Rem}

\begin{proof}
	\begin{description}
	\item[NP-трудность]
	Есть фиксированная задача $L$ из класса $NP$, то есть есть соответствующая ей машины Тьюринга $M$, которая работает $p(n)$
	
	Воспользуемся сведением по Карпу. $f(x) = <M, x, 1^{p(|x|)}>)$.

	Давай-те убедимся, что это правильное сведение. Если у x существовала подсказка $w$, то и у новой машины
	эта же подсказка подходит и наоброт.

        Действительно, если в новом решение подошла подсказка $w$, значит для изначальной $M$ подходила подсказка $w$.
        Теперь, если у изначальной машины была подсказка $w'$ и слово принадлежало языку, то существовала подсказка,
        $w$, которую можно было получить за меньше, чем $p(n)$ шагов, значит уже это подсказка подойдет для $f(x)$.

        Функция $f$ является полиномиально вычислимой.
          
	\item[Принадлежность NP]
	Теперь, что бы удостоверится, что новая задача принадлежит $NP$, нужно значть, что 
	бывает универсальная детерминированная машина тьюринга, которая с неболее чем полиномиальным ухудшением 
	сможет промоделировать переданную машину тьюринга на t шагов. Пока в такую машину поверим, чуть поже будет рассказано,
	как именно ее построить.  
	\end{description}
\end{proof}

\subsection{$CIRCUIT\_SAT$}
\begin{Def}
	Булева схема это:
	\begin{enumerate}
	\item Ориентированный граф без циков.
	\item Бинарные (и унарные) операции над битами: $\wedge, \vee, \oplus$
	в вершинах графа
	\item Есть выделеные вершины, которые являются входом схемы. Вершины без входящих ребер.
	\item Есть выделенные вершины, которые являются выходом схемы.
	\end{enumerate} 
\end{Def}

\begin{Def}
	$CIRCUIT\_SAT = \{(C, x)|C \text{"--- схема}, С(x) = 1\}$\\
	То есть массовая задача для $CIRCUIT_SAT$ "--- это для данной схемы найти вход, на котором эта схема будет давать 1.

	В случае задачи распознавания, узнать, существует ли такой набор. 
\end{Def}

\begin{theorem}
	$CIRCUIT\_SAT$ "--- $NP$-полная задача. 
\end{theorem}
\begin{proof}
	\begin{description}
	\item[Принадлежность NP:]
		Если у нас есть схема и подсказка, то достаточно не сложно проверить результат за полином от размера схемы. 
		Можно, например, сделать топ сорт вершин и за квадрат последовательно вычислять значения в ячейках.
	\item[NP-трудность:]
		Что бы доказать $NP$-трудность, сведем уже известную нам $NP$-полную задачу $BH$ к нашей. 
		Тогда все задачи из $NP$ будут сводится к $BH$, $BH$ будет сводится к нам и как следствие все задачи из $NP$ 
		будут сводится к нам. 

		\begin{enumerate}
		\item 
			Каждый этаж схемы будет соответсовать конфигурация ДМТ.\\
			Конфигурацию шифруем как [состояние, положение головки, память до головки, символ под головкой, память после головки; так же для следующих лент]\\
		
		        Ну да, каждый символ алфавита шифруется каким-то количеством бит. 

		\item	Размер этажа можем ограничесть $const \cdot t$. Так как больше памяти, чем работает машина тьюринга мы не израсходуем.\\ 
		
		\item 
			Количество этажей схемы соответствует времени работы ДМТ. Каждый этаж конфигурация 
			на данном шаге выполнения ДМТ.
		\item 
			Переход между этажами реализует один шаг ДМТ\\
		\item 
			Вход схема "--- подсказка НМТ. \\
			Так, как схема строится для конкретной Машины тьюринга, входа и времени, то все остальное
			уже намертво вбито в нашу построенную схему.  

		\end{enumerate}

		Теперь подробнее разберем, как устроен переход между этажами. 

		Для этого научимся программировать с помощью схем несколько конструкций. 
		
		if x then y else z такую инструкцию можно записать, например в виде
		(x and y) | (not x and z)

		Понятно как проверить, что какой-то символ чему-то равен. Берем ячейки отвечающие за этот 
		символ и соответсвенно and. 

		Теперь хотим узнать, что будет в какой-то конкретной ячейке. 

		if (головка где-то недалеко)
			то что-то содержательное\\
		else
			оставляем символ таким же.\\ 
		
		Что-то содержательное - это много if. У нас будет отдельно в схеме хранится переходы по 
		состояниям. 

		if вида, если такое-то состояние и такие-то сиволы под головкой, то записать в данную ячейку то-то.

	        И в конце исправить положенеи головки. 

	        Получается не так мало if, но все равно константа, поэтому не так важно. 

	        Ячеек в схеме получается порядка $\Theta(|x|^2)$\\
	 
	        Можно уменьшить до $|x|log(|x|)$ с использованием oblivious универсальной машины тьюринга.
	        То есть если изначальная Машина Тьюринга была oblivious, то мы все сделали за линию, 
	        пересчитывая гейты только в окрестности головки и мы так могли бы сделать, так как все время знаем
	        положение головки в любой момент времени вне зависимости от входа. 
	\end{description}
\end{proof}

\subsection{3-SAT}
\begin{Def}
	$\tilde{3-SAT} = \{(F, A)|F \text{--- в 3-КНФ}, F(A) = 1\}$
	
	У нас есть формула в $3-$КНФ, хотим для нее найти выполняющий набор.

	В случае задач распознавания, узнать, существует ли выполняющий набор. 
\end{Def} 
\begin{theorem}
	$3-SAT$ "--- NP-полная задача.
\end{theorem}
\begin{proof}
	\begin{description}
	\item[Принадлежность NP:]
		Проверка выполняющего набора "--- не очень хитрое дело. Подставляем 0 и 1 и 
		дальше уже распарсить выражение, тем более в 3-КНФ - совсем просто. 
	\item[NP-трудность:]
		Сведем $CIRCUIT\_SAT$ к $3-SAT$. 
		
		Теперь как мы это будем делать:
		\begin{enumerate}
		\item Заведем для каждого гейта новую переменную.
		\item Теперь гейт выражает какую-то операцию $g(x, y)$. Для примера $\oplus$. 
		Тогда хотим выписать несколько клозов, которую этот гейт выразят. 

		Это мы сможем сделать не более чем за 8 клозов.
		\item 
		Склеим полученный результат. Напишем конъюнкцию всех этих дизъюнктов. 
		\item 
		Добавляем клоз, что на выходе получили 1. 
		\end{enumerate}
	\end{description}
\end{proof}
\begin{Rem}
	Это наиболее изестная $NP$-полная задача. 
\end{Rem}

\section{Универсальная машина тьюринга:}

ablivios машина тьюрига(пофигистическая)

Теперь, как мы это будем строить. 
Мы хотим добиться, что бы головка стояла на месте, а ленты как-то двигались. 

Научимся двигать одну ленту, а дальше все и так получится. 

Как и в прошлый раз разбили на блоки и хотим что бы суммарно в них заполнены были половина из 
возможных. 

Если у нас есть $2^i$ элементов в i-ом блоке,то мы можем все это раскидать в начало. 
То есть после преобразования у всех элементов будет заполнено ровно половину. Назовем это
балансировкой и занимает это по времени $2^i$, но такая балансировка происходит 
не чаще, чем $2^i$ шагов. 

Значит всего время работы $O(\sum_{i = 0}^{\log t}2^i \frac{t}{2^i}) = O(t\log t)$

Теперь улучшим машину, что бы она стала ablivios. Теперь будем проводить 
балансировку не глядя, что там происходит. 

\begin{lemma}
Существует U "--- ablivious TM, U(M, x, $1^t$) = M(x) за t шагов и 
делает при этом $O(t\log t)$ шагов. 
\end{lemma}
\begin{conseq}
...
\end{conseq}



\begin{Def}
$U(M, x) = M(x)$ "--- универсальная машина тьринга.
\end{Def}

Для простоты будем пытаться сделать и унивирасальную машину тьюринга одноленточной и исходная 
машина тоже была однолентчной. 

\begin{theorem}
Хотим построить такую машину, что бы время исполнения на ней было $O(t \log t)$ от исходного. 
\end{theorem}

\begin{proof}
\includegraphics[width=0.25\linewidth]{im2}

Разобъем ленту на блоки следующего вида.Будем подерщивать,что бы $|L_i| + |R_i| = 2^i$, 
то есть в парных блоках сумарно занята была ровно половина. 

Состояние машины тьюринга будим пытаться как-то тащить за собой. 
\end{proof}

\section{Оракульная машина тьюринга}
Оракульная машина тьюринга имеет доступ к оракулу, который за один шаг
дает ответ на вопрос. 

Формально: состояния $q_{in}$, $q_{out}$  и "фантастический переход" из 
$q_{in}$ в $q_{out}$,заменяющий содержимое третий ленты на ответ оракула. 

$M^B$ "--- оракульная машина M, котрый дали конкртетный оракул B. Класс NP может 
быть не замкнут относительно этих операций. 
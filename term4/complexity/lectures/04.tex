\section{Оракулы для равенства и не равенства P и NP}
\begin{theorem} (Baker, Grill, Solovay)
	Существуют задачи $A, B$ такие, что
	\begin{enumerate}
	\item $P^{A} = NP^{A}$\\
	\item $P^{B} \ne NP^{B}$\\
	\end{enumerate}
\end{theorem}

\begin{proof}
\begin{enumerate}
\item 
	Рассмотрим $A = \{(M, x, 1^{n})| M(x) = 1 $ за $\le 2^n$ шагов $\}$
	Язык машин тьюринга, которые на входе x останавливаются меньше чем за $2^n$ шагов.

	Хотим доказать, что $P^A = EXP$ и $NP^A = EXP$.

	\begin{Def}
		$EXP = \cup_{p(n)}DTime(2^{p(n)})$
	\end{Def}

	Для $NP^{A}$ можем перебрать все подсказки, за полином проверять и каждый раз моделировать,
	все равно получится не более экспаненты, значит $NP^{A} \subset EXP$.

	EXP содержится в $P^A$, можем просто задавать вопросы оракулу. 

	Ну и тривиальное включение $P^A \subset NP^{A}$.

	Из этого следует, что $P^A = NP^{A}$\\
\item 
	Построим языки B и $U_B$.
	
	$U_{B}$ это такой унарный язык, проекция языка $B$. Если у нас есть слово длины n, 
	то в проекции есть слово из n палочек, а если нет, то нет.

	Каким бы не было $B$, $U_B$ принадлежит $NP^{B}$. Просим подсказку
	это слово такой же длины и спрашиваем у оракула, принадлежит ли это $B$.
	
	Перебираем все детерминированные оракульные машины $M_1, \cdots, M_n, \cdots $
	
	Запускаем машину $M_n(1^n)$ и в какой-то момент она спрашивает оракула. 
	
	Если мы уже добавили в слово в словарь, то говорим да, делать нечего. 
	Если еще не добавили, то говрим нет. 
	
	Ждем время $2^{\frac{n}{10}}$. Теперь смотрим, что она сказала. 
	
	Если она сказала да, то мы принимаем решение, что строчек такой длины в оракуле нет. 
	
	Теперь, если он сказал нет,то находим любую строчку, про которуе не знаем, что нет и отвечаем про нее да. 
	
	Если предыдущие машины спрашивали про строчки этой длины, то резерв-то все равно у нас останется, что бы выбрать строчку.  
\end{enumerate}
\end{proof}

\begin{Def}
	Relative это навешивание оракула. То есть когда факт не зависит от довешивания оракулов на машине. 
\end{Def}

\section{Полиномиальные схемы}
\begin{Def}
	$L \in Size[f(n)]$, если существует семейство булевых схем $\{C_n\}_{n \in \N}$, такие, что 
	\begin{enumerate}
	\item $\forall n |C_n| \le f(n)$\\
	\item $\forall x(x \in L \Lra C_{|x|}(x) = 1)$
\end{enumerate}
\end{Def}

\begin{Def}
	$P/poly = \cup_{k \in \N}Size[n^k]$
\end{Def}

\begin{Rem}
	$P \subset P/poly$\\
	Можем построить схему переходов конфигураций машины тьюринга. Она окажется полиномиального размера. 
\end{Rem}
\begin{Rem}
	Обратное включение не верно. Возьмем язык из n палочек, если $M_n$ останавливается, 
	то выдаем нет, если не останавливается, то да. Схема тривиально, это для данного $n$ просто константа. 
\end{Rem}

\begin{Rem}
	Если $L \in P/poly$ тогда $\exists G \colon 1^n \to C_n$, то язык принадлежит $P$. То есть существует Машина, 
	которая принимает на вход n 1 и выдает схему и язык принадлежит $P/poly$, тогда язык принадлежит P. 
\end{Rem}

Мы не умеем доказывать, что $NP = P$, но может мы можем доказать $NP \subset P/poly$. Конечно же нет...

\begin{theorem}
	$NP \subset P/poly \Ra PH = \Sigma^2P$\\
\end{theorem}
\begin{proof}
	Что бы доказать эту теорему покажем, что $\Sigma^3P$-полный язык $QBF_3$ лежит в $\Sigma^2P$.

	Дали значит нам схемы, и сказали, что они решают SAT. Нужно проверить, правда это или нет. 

	Проверки корректности схем для SAT:
	$C_{|G|}(G) = C_{|G[x_1 = 0]}(G[x_1:= 0]) \vee C_{|G[x_1 = 1]}(G[x_1 = 1])$ и проверка корректности для тривиальных схем.

	То есть формула выполнима,если выподнима она, когда вместо первой переменной подставили 0 или если она выполнима, когда вместо переменной 
	поставили бы 1. 

	Хотим формулы с тремя кванторами решить с использованием двух.
	
	$(\exists x \forall y \exists z (F)) \in QBF_3 \Lra
	\exists$ схемы $C_1, \cdots, C_{|F|} \exists x, \forall y$\\
	$\forall G$ - булевы формулы длины $\le |F|$\\
	(семейство $\{C_i\}$ корректно для $G$)$\wedge C_{|F|}(F(x,y,z)) = 1$

        То есть у нас есть формула с тремя кванторами, хотим записать ее с двумя. 
        Говорим, что это то же самое, что у нас существуют схемы, существует x, 
        что для любого y, любая булева формула является корректной для этого семейства схем и 
        формула F является выполнимой. 
\end{proof}

Сколько функций $f \colon \{0, 1\}^n \to \{0, 1\}$ их $2^{2^n}$\\
А сколько схем такого размера $|C_n| = s(n)$ их примерно $2^{cs(n)}s(n)$

Значит есть функции, для которых нет схем. 

Теперь давай-те искать сложную функцию, прям экспаненциальную может быть сложно.
\begin{theorem}
	$\forall k \Sigma^4P \not \subset Size[n^k]$\\
\end{theorem}
\begin{proof}
	Из соображений мощности имеется $f\colon \{0, 1\}^n \to \{0, 1\}$, зависящая 
	от первых $ck\log n$ битов, для которой нет булевой схемы размера $n^k$\\

	Сконструируем такую функцию в $\Sigma^4P$\\
	$y \in L \Lra \exists f \forall c$(схема размера $n^k$)\\
	$f(x) \ne c(x)$ не принимается схемой.
	$\wedge ((f \le f') \vee f'(x') = c'(x'))$
	$\wedge f(y) = 1$ 
\end{proof}

\begin{conseq}
	$\forall k \Sigma^2 P \cap \Pi^2P \ne \subset Size[n^k]$
\end{conseq}
\begin{proof}
	$\Sigma^2P \cap \Pi^2P \subset Size[n^k] \Ra $\\
	$NP \subset P/poly \Ra$\\
	$PH = \Sigma^2P \cap \Pi^2P \subset Size[n^k]$\\
\end{proof}
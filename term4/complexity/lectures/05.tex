\section{Параллельные алгоритмы.}

Главаня вычислительная модель для параллельных алгоритмов "--- это булевысхемы. 

\begin{Def}
	Время "--- глубина схемы\\
	Работа "--- количество гейтов\\
\end{Def}

Если у нас есть булева схема глубины d, то за время $d$ мы можем вычислить результат, если 
каждый гейт поручить своему процессуру. 

Это не самое оптимальное количество процессоров, которое можно использовать. А именно,
если в схеме w gates(работа) и глубина схемы d, то процессоров достаточно что-то вроде $\frac{w}{d}$, что бы 
не сильно увеличить время. 

То есть разбиваем все на этажи, если этаж большой, то попилим еще на несколько кусков. За счет этого
время, конечно, увеличится. 

\begin{theorem}
	Brent's principle

	Если в схеме имеется w гейтов и схема имеет глубину d, тогда достаточно иметь $\frac{w}{d}$ процессов, 
	что бы время работы замедлялось не более чем в константу от оптимального. 
\end{theorem}
\begin{proof}
	Посчитаем, как изменится время работы.

	Был $g_i$ гейтов на i-ом уровне $\sum_{i = 1}^{d} g_i = w$\\
	$T = \sum_{i} \left \lceil \frac{g_i}{\frac{w}{d}} \right \rceil \le \sum_i (\frac{g_i}{\frac{w}{d}} + 1)
	\le d + \frac{w}{\frac{w}{d}} \le 2d$\\
\end{proof}

Если нам выдали меньше процессоров, чем мы способны заиспользовать. Если нам выдали $p' < p$, 
то время работы изменится $O(d\frac{p}{p'})$,  где $p$ столько процессоров, сколько мы могли заиспользовать
оптимально. 

Если же нам дали больше процессоров, то время могло не увеличится, так как ну что с ними теперь делать.

\subsection{Определение класса NC}

\begin{Def}
	Семейство схем $\{C_n\}_{n \in \N}$ равномерно, если имеется полиномиальный алгоритм $A$, такое что $A(1^n) = C_n$\\

	То есть схема порождается полиномиальным алгоритмом. Мы подаем количество входов, алгоритм нам выдает схему.
\end{Def}

Ясно, что равномерные полиномиальные схемы задают $P$.
 
\begin{Def}
	$Logspace$-равномерно: $A$ использует память $O(log n)$ \\
\end{Def}

\begin{Def}
	Глубина схемы эквивалентно времени параллельного вычисления. 
\end{Def}

\begin{Def}
	$NC^1 = \{L|$принимается семейством схем, которые logspace-равномерные и имеют глубину $O(log n) \}$
	\\
	$NC^i = \{L|$принимается семейством схем, которые logspace-равномерные и имеют глубину $O(log^i n) \}$
	\\
	$NC = \cup_{i} NC^i \subset P$\\
\end{Def}

Что бы этот язык за полином посчитать, мы сначала породим схему за полиномиальное время, а потом просто 
по этой схеме произведем вычисление. Так как ее размер полиномиален, так как никакю другую мы выдать не успеем. 

\subsection{P-полнота}

Все сведения, которые мы пока знаем "--- это полиномиальные сведения. Но в данном случае 
полиномиальные сведения недостаточно точные, поскольку с точки зрения полиномиального сведения все языки в P P-полные. 
Для того, что бы говорить для более слабых вычислениях нам нужны более точные сведения, а именно
в данном случае мы используем сведение логарифмическое по памяти и полиномильное по времени. 

\begin{Def}
Язык P-полные, если он принадлежит P и любой другой язык из P сводится к нему 
за полиномиальное время и логарифмическую память. 

Полиномиальное время, в целом, следует из-за логарифмической памяти. 
\end{Def}

\begin{theorem}
	Если L "--- P-полный, то 
	\begin{enumerate}
		\item  $L \in NC \Lra P = NC$(все параллезуется)
		\item  $L \in DSpace[log] \Lra P = DSpace[log]$
    \end{enumerate}
\end{theorem}
\begin{proof}
	\begin{enumerate}
	\item 
	То есть хотим доказать,что если $L$ сводится к $L' \in NC$, то $L \in NC$.

	Пусть R сведение $L$ к $L'$ логарифмическое по памяти и полиномиальное по времени. 
	Не сложно заметить, что существует логарифмическая по памяти 
	машина тьюринга R', которая принимает вход (x, i) (где i бинарное предствление числа не больше $|R(x)|$), если
	и только если i-ый бит числа $R(x)$ равен одному. Теперь, если мы решим задачу достижимости в графе 
	конфигураций $R'$ на входе (x, i) мы cможем посчитать i-ый бит в числе R(x). Мы можем решить 
	все эти задачи параллельно за $NC_2$, как обычно, напрмер, 
	с помощью возведения матрица в степень в графе конфигураций, 
	тогда мы смогли посчитать все биты $R(x)$. А раз у нас есть схема для $R(x)$ 
	мы можем использовать NC схема для языка L', что бы сказать, 
	принадлежит ли x языку L. И это все происходит в NC.
 
	\item Здесь все очевидно, если за логарифм памяти умеем сводить и второй умеем вычислять за логарифм, то более-менее понятно, 
	что $P = DSpace[log]$.
\end{enumerate}
\end{proof}

\begin{theorem}
	$P$-полный язык: $CIRCUIT_EVAL = \{$(схема С, вход x)| C(x) = 1$\}$. \\
\end{theorem}
\begin{proof}
	Очевидно, этот язык лежит в $\P$ "--- берём и вычисляем схему.

	Осталось показать, что любой язык $L \in \P$ сводится к $\mathsf{CIRCUIT\_EVAL}$.
	Давайте придумаем функцию сведения $f$. Пусть нам дали на вход слово $x$.
	Давайте слово-в-слово повторим доказательство теоремы Кука-Левина (где мы сводили задачи из $\NP$ к $\SAT$) и построим
	многослойную схему, которая эмулирует работу машины Тьюринга, разрешающей язык $L$.
	Слои у нас все одинаковые, связи между ними "--- тоже, логарифма памяти точно хватит (логарифм памяти "--- это некоторое константное число int'ов, если угодно).
	Получим на выходе схему, у которой нет входов 
	и которая вычисляется в единицу тогда и только тогда, когда машина принимает слово, что и требовалось.
	\end{proof}

\subsection{Примеры параллельных алгоритмов}
\subsubsection{Для дизъюнктов}
Самый простой параллельный алгоритм: $\vee_{i = 1}^{n}a_i$\\

Здесь мы можем нарисовать двоичное дерево логарефмической глубины, где в гейтох будут находится дизъюнкции, в листьях переменные.

\subsubsection{Перемножение матриц}
Перемножение матриц булевским способом. Вместо сложения у нас дизъюнкция, вместо умножения у нас конъюнкция:

$С = AB$, где $c_{ik} = \vee_{j = 1}^{n}a_{ij}\wedge b_{jk}$\\

Для простоты считаем, что матрицы квадратные. 

Каждый элемент матрицы будет считаться своей большой дизъюнкцией. 

Большую дизъюнкцию можем писать как дерево, конъюкции можем тоже отдельно привести, но это всего одна операция, поэтому дерево будет 
всего на единичке больше. 

Время $O(\log n)$  и работа $O(n^3)$\\

\subsubsection{Достижимость в графе} 
Теперь хотим проверим, существование путей между каждыми вершинами.

Для этого можем возвести матрицу смежности в степени $n$. Через и и или. $A^n$.\\
На диагонале нужно изначально добавить единицы. 

Понятно, что возводить мы можем за $\log n$ быстрым возведением в степень. 

То есть за $O(\log^2n)$ можем проверить достижимость. 

\subsubsection{Сложение чисел} 
Учимся складывать два числа за время $O(\log n)$\\

Давайте сначала схемой глубины $O(\log n)$ выясним, в каких разрядах будет при сложении возникать перенос.
А потом отдельно выясним значение каждого разряда: это просто сумма двух разрядов исходных чисел плюс перенос (который мы уже знаем).

Давайте научимся считать функцию $f$ от двух чисел одинаковой длины $k$, она будет выдавать два бита:
\begin{enumerate}
	\item Верно ли, что при сложении чисел возникнет перенос из разряда $k$?\\
	$g_i = a_i \wedge b_i$
	\item Верно ли, что при сложении этих чисел и еще одной единицы возникнет перенос из разряда $k$?\\
	$p_i = a_i \vee b_i$
\end{enumerate}

Определим операцию $\odot$ как:\\
$(a, b) \odot (a', b') = ((a' \vee (b' \wedge a), b \wedge b')$\\

Эта операция обладает несколькими свойствами. 
 
\begin{enumerate}
	\item $(0, 0) \odot (g_1, p_1) \odot (g_2, p_2) \odot \cdots$
	Если применить эту операцию i раз, то мы получаем $c_i$ в первой компоненте.
	\item Эта операция ассоциативна, из этого сразу будет следавать, что мы 
	сможем построить дерево логарифмической глубины. Где $c_i$ означает итоговое наличие или отсутствие переноса в i-ом разряде.
\end{enumerate}

Можно убедится по индукции, что если мы применим операцию $i$ раз, то получим $c_i$ в первой компоненте.\\
$(g_i \vee (p_i \wedge c_{i - 1}), *)$\\
Вторая компонента нужна для ассоциативности.\\

Давай-те теперь поймем, что $g_i \vee (p_i \wedge c_{i - 1}) = c_i$. Действительно, как мог возникнуть перенос. 
Если в этом разрязе уже был перенос, то никуда мы от него не денемся. Или перенос мог возникнуть в случае, если перенос 
был бы, если мы прибавим 1 и произошел перенос к нам из предыдущего разряда.

Можем уже сейчас все тупа записать для каждого переноса. Но так делать не хочется, 
так как получится большая работа. 

Хорошо бы еще понять, что операция будет ассоциативна. 
$((a, b) \odot (c, d)) \odot (e, f) = (c \vee (d \wedge a), b \wedge d) \odot (e, f) = (e \vee (f \wedge (c \vee (d \wedge a))), b \wedge d \wedge f)$\\
$(a, b) \odot ((c, d) \odot (e, f)) = (a, b) \odot (e \vee (f \wedge c), f \wedge d) = ((e \vee (f \wedge c)) \vee ((f \wedge d) \wedge a), b \wedge d \wedge f)$\\
$(e \vee (f \wedge (c \vee (d \wedge a))) = (e \vee (f \wedge c) \vee (f \wedge d \wedge a)))$

Тогда давай-те разбиваем по парам, запускаем рекурсивно, посчитали все для четных префиксов. 
Теперь добавим нечетные, получили еще один этаж, тогда работа это $W(n) = W(\frac{n}{2}) + \frac{n}{2}$, а время $t(n) = O(\log (n))$. 
Работа линия.

\subsubsection{Умножение} 
Что бы умножить два числа в столбик, мы должны просуммировать $n$ чисел. 
Которые являются либо 0 либо сдвигами числа. 

Нужно производить сложение n n битных чисел. 

Будем производить сложения $3-in-2$.
Мы делаем это совсем быстро, так как нам не нужно никуда ничего переносить.
Он строится просто: $a = x \oplus y \oplus z$, а $b$ "--- это маска возникших переносов при суммировании $x+y+z$ (т.е. просто $((x \& y) | (x \& z) | (y \& z)) << 1$).

Два числа можем сложить честно.

\subsection{$NC^1 < NSpace[log n] < NSpace[log n] < NC^2$}

\begin{Rem}
	В случае недетерминированости по памяти в ленте с подсказкой нельзя бегать туда-сюда, можно только в одну сторону.
\end{Rem}

\begin{Rem}
	Что такое вычисление ограниченное по памяти, которое не просто говорит да/нет, а что-то еще выдает. Это значит, что у него 
	есть лента, которая доступна только на запись. Можно считать, что каждый бит можем записать только один раз. Выход может быть длинее 
	ограничения. 
\end{Rem}

Как комбинировать два logspace в одно. (см. практику) 

\begin{theorem}
	$NC^1 \subset DSpace(log) \subset NSpace(log) \subset NC^2$\\
\end{theorem}
  
\begin{proof} 
	\begin{enumerate}
	\item Второе включение очевидно.

	\item Докажем последнее включение, т.е. что $NSPACE(\log) \subseteq NC^2$.
	
	Рассмотрим язык, лежащий в $NSPACE(\log)$, и недетерминированную машину Тьюринга $M$, которая его принимает. 

	\begin{Def}
		Псевдоконфигурацией в этой теореме назовем конфигурацию, в которую не включается входная лента.
	\end{Def}

	Рассмотрим граф всех возможных псевдоконфигураций машины $M$ (наличие каждого из ребер в этом графе, очевидно, зависит от входа).

	Наша задача --- проверить, есть ли (для данного входа) в этом графе путь из начальной псевдоконфигурации в принимающую.
	Заметим, что размер псевдоконфигурации ограничен $O(\log)$. Таким образом, нам надо решить задачу достижимости в графе, содержащем
	полиномиальное число вершин (обозначим это число через $k$).

	Покажем, что эта задача лежит в $NC^2$. 
	
	Пусть $A$ --- матрица смежности нашего ориентированного графа.
	Наша задача --- вычислить булеву степень\footnote{Булево произведение
	матриц --- это произведение, в котором в качестве операции сложения
	используется дизъюнкция, а качестве операции умножения --- конъюнкция.} 
	$A^k$ схемами глубины $\log^2k$.

	Заметим, что для этого достаточно $\log k$ последовательных умножений 
	пар матриц: $A^2$, $(A^2)^2$, \ldots (если получим в итоге чуть б\'ольшую
	степень, чем $k$ --- а именно, ближайшую к $k$ сверху степень двойки ---
	не страшно: $A^k=A^{k+1}=\ldots$). Для умножения же одной пары матриц 
	$B\cdot C$ достаточно глубины схем $O(\log k)$:
	параллельно вычислим все произведения вида $b_{il}\land c_{lj}$,
	затем вычислим их суммы $\bigvee_{l=1}^k b_{il}\land c_{lj}$ также параллельно,
	затрачивая на каждую сумму лишь логарифмическое время (глубину) ---
	сначала складываем слагаемые по два, затем полученные частичные сумму ---
	снова по два, и$\;$т$\;$д.

	Итак, мы доказали, что достижимость в графе можно выяснить в $NC^2$.
	На вход этой схеме мы должны подать элементы матрицы смежности,
	т.е. ребра графа ($1=\textrm{ребро есть}$, $0=\textrm{ребра нет}$).
	Понятно, что для любых двух псевдоконфигураций мы можем в $NC^2$ выяснить,
	получена ли одна из них из другой (входом для этой схемы являются
	биты входной ленты! псевдоконфигурации входами не являются: для каждой
	интересной нам пары псевдоконфигураций мы строим свою подсхему).

	\item Докажем теперь, что $NC^1 \subseteq DSPACE(\log)$. 
	Пусть есть теперь язык $L \in NC^1$. Докажем, что $L\in DSPACE(\log)$. 
	Пусть на вход нам подали $x$. Надо выяснить, верно ли, что
	$x\in L$. Мы сделаем это, построив композицию трех функций, вычисляемых
	с логарифмической памятью.

	Первая функция строит схему для входов длины той же, что у $x$.
	Существование соответствующей логарифмической по памяти машины Тьюринга
	следует из определения $NC^1$.

	Вторая функция преобразует эту схему в формулу
	(т.е. делает из ориентированного ациклического графа, 
	представляющего схему, дерево). Сделаем
	мы это следующим образом: каждый гейт формулы будет 
	обозначаться битовой строчкой, обозначающей путь к некоторому гейту
	исходной схемы от ее выходного гейта.
	(В этой строчке 1 будет означать, что мы пошли направо, а 0 --- налево.)
	Понятно, что перечислить эти вершины, их типы ($\lor, \land, \lnot$) 
	и ребра между ними мы можем, ограничившись логарифмической памятью,
	устроив в исходной схеме поиск в глубину (по ребрам, обратным заданным)
	(правда, чтобы вернуться к предыдущей вершине, придется пройтись еще
	раз от корня до нее, поскольку мы можем хранить только <<куда мы пошли>> ---
	налево или направо --- но не <<откуда пришли>>).

	Третья функция вычисляет значение формулы на входе $x$. 
	Для этого будем обходить дерево и вычислять значение левого поддерева,
	а, когда надо, и правого. <<Когда надо>> здесь означает
	следующее: если вершина помечена <<$\lor$>>, 
	то правое поддерево надо вычислять только в том случае, если значение в левом 
	поддереве $0$, а если помечена <<$\land$>>, то надо вычислять 
	только в том случае, если значение в левом поддереве $1$. 

	То, что для этого достаточно логарифмической памяти, очевидно.
	Вот, впрочем, формальное доказательство.

	{\small
	Заметим, что нам достаточно хранить только битовую строчку,
	обозначающую текущий гейт $c$ и переменную $b$, принимающую одно из пяти значений:
	\begin{enumerate}
	\item[$b=1$,] \leftskip 3mm
              если мы приступили к обработке данной вершины, 
	\item[$b=2$,] если уже обработали левое поддерево с результатом $0$,
	\item[$b=3$,] если уже обработали левое поддерево с результатом $1$,
	\item[$b=4$,] если уже обработали правое поддерево с результом $0$,
	\item[$b=5$,] если уже обработали правое поддерево с результом $1$.
	\end{enumerate}
	Начинаем мы с пустой строки $c$ (соответствующей выходному гейту)
	и $b=1$. Опишем работу алгоритма на гейте, вычисляющем <<$\land$>> 
	(для <<$\lor$>> --- аналогично). 

		\begin{description}
		\item[Если] $b=1$, то $c:=c\,0$, $b:=1$;	
		\item[если] $b=2$, то <<обрезаем>> последний бит $c$, и 
			\begin{description}
			\item[если] этот бит был $0$, то $b:=2$, а
			\item[если] этот бит был $1$, то $b:=4$;
			\end{description}
		\item[если] $b=3$, то $c:=c1, b:=1$;
		\item[если] $b=4$, то поступаем аналогично случаю $b=2$;
		\item[если] $b=5$, то <<обрезаем>> последний бит $c$ и
	\begin{description}
	\item[если] этот бит был $0$, то $b:=3$, а
	\item[если] этот бит был $1$, то $b:=5$.
	\end{description}
\end{description}
Понятно, что памяти мы затратили снова лишь $O(\log)$ на хранение $c$~и~$b$.
}
\end{enumerate}
\end{proof}

\begin{conseq}                                          	
	$PSpace = NPSpace$\\
\end{conseq}

\begin{Rem}
	Часть конспекта взята из: http://logic.pdmi.ras.ru/\~hirsch/students/complexity1/
\end{Rem}

Недетерменизм по памяти это не страшно, за недетерменизм по памяти мы платим только квадрат. Если нам 
дали ориентированный граф и спрашивают, если путь из одной вершины в другую. Как это сделать за логарифм. 

Если есть путь значит, есть какая-та вершина по дороге, до которой есть путь половинной длины и от нее есть путь 
половинной длины. У нас есть задание на вершине стека если мы взяли ABK. Нужно перебирая все промежуточные вершины обработать задания  
$AC(K/2)$ и $CB(K/2)$. Размер этого стека никогда не будет привышать логарифм.

Теперь мы посмотрим на граф состояний и посмотрим на наличие путей. То есть мы поняли, что
если речь идем про полиномиальную память, то ничего в недетерменизме не меняется.  

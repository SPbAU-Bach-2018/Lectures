\setauthor{Черникова Ольга}

\setcounter{section}{4}
\chapter{классы P и NP} 
\section{Индивидуальные задачи, массовые задачи, языки}
\begin{Def}
Работаем в алфавите $\{0, 1\}$\\
Множество слов длины $n$ в нем: $\{0, 1\}^n$\\
Множество всех конечных слов: $\{0, 1\}^*$\\ 
Длина слова x: $|x|$\\
\end{Def}
\begin{Def}
Индивидуальная задача "--- пара (уловие задачи или вход для задачи, какое-то решение) $\in \{0, 1\}^* \times \{0, 1\}^*$\\
\end{Def}
\begin{Def}
Массовая задача "--- некоторое множество индивидуальных задач, 
бинарное отношение на $\{0, 1\}^* \to \{0, 1\}^*$\\
\end{Def}

Наиболее интересные массовые задачи "--- бесконечные, 
с возможностью проверить корректность решения. 

\begin{exmp}
Пологаем $\N \subset \{0, 1\}^*$\\
$FRAC = \{(n, d) | n \vdots d, 1 < d < n\}$\\
\end{exmp}
\begin{Def}
Алгоритм решает задачу поиска для массовой задачи $R$, 
если для условия $x$ он находит решение $w$, 
удовлетворяющее $(x, w) \in R$
\end{Def}

\begin{Def}
Алгоритм решает задачу распознования(языки), задачу 
типа да нет. Для массовой задачи $R$:
$$L(R) = \{x | \exists w (x, w) \in R\} $$
\end{Def}

\begin{exmp}
$L(FRAC) =$ множество всех составных слов. 
\end{exmp}
\section{Детерминированная машина Тьюринга}
\begin{itemize}
\item конечный алфавит(включая начала ленты и пробелы)
\item несколько лент, то есть массивов, бесконечных в одну сторону. 
\item читающие/пишещие голвки, по одной для каждой ленты, 
каждая в один момент видет одну позицию. 
\item конечная множество состояний, в том числе начальное $q_s$, принимающие
$q_Y$ и отвергающие $q_N$.
\item управляющее устройство(программу), содержащее для каждых
$q, c_1, \cdots, c_k$ одну инструкцию вида $(q, c_1, \cdots) \to (q', c_1', \cdots, d_1 \cdots)$, 
где $q, q' \in Q$, $c_i, c_i' \in \Sigma$ "--- символы, обозреваемые 
головками. 
$d_i \in \{ \la, \ra, \cdot\}$ "--- направление движения

\begin{center}
\psfig{file=lecture1pic.eps, width=8cm}
\end{center}
\end{itemize}

Вычисление на ДМТ:
\begin{description}
\item[Начало работы:]
    \begin{itemize}
    \item состояние $q_s$
    \item на первой ленте вход(входное слово) и пробелы,\\
    остальные ленты заполнены пробелами.
    \item головки в крайней левой позиции.
    \end{itemize}
\item[Выполнение вычислений:]
шаг за шагом выполняем инструкции программы.
\item[Конец работы:] 
когда машина попадает в состояние $q_y$, $q_n$;
\end{description}      
\begin{Def}
ДМТ - принимает входное слово, 
если она заканчивает свою работу в $q_y$;
\end{Def}
\begin{Def}
ДМТ - отвергает входное слово, 
если она заканчивает свою работу в $q_n$;
\end{Def}
\begin{Def}
ДМТ $M$ распознает язык $A$, если принимает все 
$x \in A$, отвергает все $x \not \in A$, пишем, 
что $A = L(M)$.
\end{Def}
\begin{Rem}
ДМТ может также вычислять функцию(решать задачу поиска).
Значением этой функции на данном входе будем считать содержимое 
первой ленты после достижения $q_y$.
\end{Rem}
\begin{Def}
Время работы $M$ на входе $x$ "--- 
количество шагов(применений инструкций) до 
достижения $q_y$ или $q_n$.
\end{Def}

\begin{Def}
Используемая память "--- суммарное крайнее 
правое положение всех головок. При сублинейных ограничениях 
на память первая лента(где вход) $read-only$ и положение 
головки на ней не считается. 
\end{Def}
\section{Классы $\tilde P$ и $\tilde {NP}$}
\begin{Def}
Массовая задача $R$ полиномиально ограничена, если 
существует полином $p$, ограничивающий длину
кратчайшего решения:
$\forall x(\exists u(u, x) \in R \Ra 
\exists w(x, w)\in R \wedge |w| \le p(|x|))$
\end{Def}
\begin{Def}
Массовая задача $R$ полиномиально проверяема, если 
существует полином $q$, ограничивающий время проверки решения:
для любой пары $(x, w)$ можно проверить принадлежность $R$
за время $q(|(x, w)|)$.
\end{Def}
\begin{Def}
$\tilde {NP}$ "--- класс задач поиска, задаваемых полиномиально 
ограниченными, полиномиально проверяемыми массовыми задачами. 
\end{Def}
\begin{Def}
$\tilde P$"--- класс задач поиска из $\tilde {NP}$, 
разрешимых за полиномиальное время, то есть задаваемых
отношениями $R$, такими, что $\forall x \in \{0, 1\}^*$ за
полиномиальное время можно найти $w$, для которого $(x,w) \in R$.
\end{Def}

\begin{Rem}
Ключевой вопрос теории сложности: $\tilde P ?= \tilde {NP}$
\end{Rem}

\section{классы P и NP}
\begin{Def}
$NP$ "--- класс задач распознавания(языков), 
задаваемых полиномиально ограниченными полиномиально 
проверяемыми массовыми задачами, то есть $NP = \{L(R)| R \in \tilde{NP}\}$

Иначе говоря, $L \in NP$, если имеется п.о. п.п. $R$, такая, что 
$$\forall x \in \{0, 1\}^* x\in L \Lra \exists w (x, w) \in R$$
\end{Def}

\begin{Def}
P "--- класс задач распознавания(языков), разрешимых за
полиномиальное время; ясно, что $P \subset \{L(R)|R \in \tilde{P}\}$.\\
\end{Def}

Очевидно, $P \subset NP$.

\begin{Rem}
Ключивой вопрос теории сложности: $P ?= NP$
\end{Rem}

\section{НМТ}
Инструкции $k$-ленточной ДМТ можно записать как
функцию(таблицу)
$$\delta: Q \times \Sigma^k \la Q \times \Sigma^k 
\times \{\la, \ra, \dot\}^k$$
\begin{Def}
Недетерминированная машина Тьюринга(НМТ)
допускает больше одной инструкции для данных
$q \in Q$ и $c_1, \cdots, c_k \in \Sigma$ то есть
$\delta$ для нее "--- многозначная функция.
\end{Def}

Так появляется дерево вычислений:
\includegraphics[width=0.25\linewidth]{im1}

То есть дерево вычислений это все варианты переходов в НМТ и
на конце 0 если закончили в состояние $q_N$ и 1, если в $q_Y$. 

\begin{Def}
Время выполнения на недетерминированной машине тьюринга "--- это 
длина наибольшей ветке в дереве вычислений. 
\end{Def}

\begin{Def}
В машины(ДМТ, НМТ) с заведомо ограниченным
временем работы можно встроить \textbf{будильник} и считать 
время вычисления на входах одной 
длины всегда одним и тем же. 

НМТ принимает вход, если существует 
путь в дереве вычислений, заканчивающийся в $q_Y$.
\end{Def}

\begin{Def}
Недетерминированная машина Тьюринга(НМТ) "--- это просто ДМТ,
у которой есть дополнительный аргумент (подсказка w на второй ленте)

НМТ $M$ принимает вход $x$, если существует $w$, для которой 
вычисление заканчивается в $q_y$(пишем $M(x, w = 1)$).\\
\end{Def}

Вычислительный путь в старом определение это тоже
самое, что подсказка в новом. Можно считать, что 
длина подсказки определяется длиной входа. 

\begin{Def}
$NP$ "--- класс языков, принимаемых полиномиальными 
по времени НМТ.
\end{Def}
\section{Универсальная машина тьюринга:}
\begin{Def}
$U(M, x) = M(x)$ "--- универсальная машина тьринга.
\end{Def}

Для простоты будем пытаться сделать и унивирасальную машину тьюринга одноленточной и исходная 
машина тоже была однолентчной. 

\begin{theorem}
Хотим построить такую машину, что бы время исполнения на ней было $O(t \log t)$ от исходного. 
\end{theorem}

\begin{proof}
\includegraphics[width=0.25\linewidth]{im2}

Разобъем ленту на блоки следующего вида.Будем подерщивать,что бы $|L_i| + |R_i| = 2^i$, 
то есть в парных блоках сумарно занята была ровно половина. 

Состояние машины тьюринга будим пытаться как-то тащить за собой. 
\end{proof}

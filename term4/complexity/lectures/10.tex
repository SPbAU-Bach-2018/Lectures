\subsubsection{QUADEQ "--- NP-полная задача}
Еще из прошлой теоремы осталось два момента. 
Первый, почему задача, которую мы сводим NP-полна.

\begin{lemma}
	QUADEQ "--- NP-полная задача.
\end{lemma}
\begin{proof}
	Сведем Circuit SAT к QUADEQ. Давай-те нарисуем схему. Ее можно переписать
	в виде системы квадратичных уравнений.

	Как? Вот есть гид, которые из x и y делает z. Понятно, что если там конъюнкция, 
	то это произведение. Дизъюнкцию можно переписать через конъюнкцию и отричание, 
	то есть каждый гид можно записать, как какое-то квадратичное уравнение. Тут только
	есть одирн маленький момент. Потому что у нас были чисто квадратичные уравнения, там
	не было первях степеней. Поэтому, как только видете первую степень можете сразу вместо 
	нее записать квадрат. $z^2 = xy$
\end{proof}

\subsubsection{Еще про связь с неаппроксимируемостью}

Вторая вещь, оставленная безсознательная. Я вам сказал, что мотивация PCP теоремы, 
доказывать, что для каких-то задач нельзя построить приблеженных алгоритмов. Но я вам не сказал почему. 

\begin{lemma}
	PCP теорема, часто является доказательством не существования хорошо приблеженных решений.  
\end{lemma}
\begin{proof}
	Вот почему. Смотрите. PCP теорема позволяет сделать вот что, 
	она позволяет из формулы сделать другую формулу.
	Такую, что, если формула была выполнима, то она и останется выполнимой, 
	а если формула была невыполнимой, то она стент очень сильно невыполнимой. То есть 
	в ней выполнить не только 100\% клозов, но и даже некоторую константную долю даже нельзя. 

	$F_{\text{вып}} \to F'_{\text{вып}}$\\
	$F_{\text{невып}} \to F'_{\text{сильно невыполнимо}}$\\ 

	То есть некотрую фиксированную долю мы точно никогда не сможем выполнить. Некоторую конструкциями можно
	получить $\frac{1}{8}$. А мы умеем только совсем маленькую долю можем получить, но тем неменее фиксированная.

       	Теперь поймем, что этого достаточно, что бы оправдать не существование хорошего приблеженного алгоритма. В
       	самом деле если есть алгоритм, со степенью приблежения больше, чем 1-доля, то мы сможем точно решать выполнимость,
       	а именно нам дали на вход формулу, мы ее раздули до сильно невыполнимой, нам дали этот чудо алгоритм и этот
       	чудо приблеженный алгоритм выдаст больше, чем $1-\alpha$ клозов, а для невыполнимых не выдаст, потому что там столько нет.

	Теперь откуда берется это сведение. 
	Для этого и нужна PCP теорема. $PCP(O(log n), O(1)) = NP$. Нам нужно $log$ случайных чисел и константа просмотров. И мы говорили, 
	что это неадоптивно. 

	Теперь как из этого сделать другую формулу. Эта формулу будет моделировать все возможные запуски прувера в PCP. То есть 
	перебираем все случайные биты. Всех вариантов у нас полином. 

	В каждом куске в каждом блоке мы пишем те дизъюнкции, которые говорят нам, в каком случае мы примем доказательство. 
	Переменными в этой формуле будут те биты в доказательстве, которые мы будем запрашивать. 

	Вот мы запросили три бита доказательства, и после этого мы должны принять решения брать или не брать. Это какая-та 
	функция. Эту функцию мы можем записать в виде КНФ от этих переменных. Она будет по прежнему константного размера. 

	$(\underline{(a_1 \vee a_2 \vee )\wedge \cdots}{O(1)})\wedge()$\\

	И так, у нас теперь для каждых выпавших случайных битов $2^{c\log n}$ у нас будет константный кусок формулы.
	Мы получили новую формулу, и если существует выполняющий набор для нее, это значит что V у нас ее в PCP принимал,
	значит и ранее она была выполнимой .

        Теперь, PCP теорема нам говорит, что если формула невыполнима, то она будет отвергнута с некоторой константной
        вероятностью. Что это значит? Это значит, что среди наших блоков константная доля будет ложна. Блок ложен, 
        если хотя бы одна из дизъюнкций ложна, но дизъюнкций там константное число. Для примера их 100 штук, а вероятность отвергнуть 
        $\frac{1}{2}$\\
        Это значит, что доля ложных клозов у нас больше, чем $\frac{1}{2}\frac{1}{100}$.\\
\end{proof}                                                                     


$Pr\{\bar{f}(x) \ne f(x + r) + f(r)\} < \delta$\\
$Pr\{\bar{f}(y) \ne f(y + x + r) + f(x + r)\} < \delta$\\
$1 - P_{x + y} = Pr\{\bar{f}(x + y) \ne f(x + y + r) + f(r) < \delta$\\
\\
$NP \subset PCP(O(poly(n)), O(1))$\\
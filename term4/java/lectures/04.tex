2 задача: 
threadPool: t.join(); -- Thread t; -- поэтому join запущен у потока, другой только ждать
(а, а ещё это \^ ничего не возвращает, а это v возвращает результат выполнения задачи)

FJP: 	    t.join(); -- RecursiveTask t; -- join запуцщен у задачи, поэтому поток работает

Задачки априори делящиеся, делятся и распределяются между потоками в пуле

3 задача:
	1. любая задача такого вида должна начит=наться с проверки того, что она маленькая
	2. разбить на кусочки, форкнуть каждый кусочек и дождаться выполнения

	Не надо писать так:
	for(по всем подзадачам) {
		t.fork();
		t.join(); -- неэффективно загрузились потоки: у нас процессится только одна задачка -- насколько возможно, параллельно, но, тем не менее, большое количество веток простаивает, потому что ещё не форкнулось
	}

4 задача:
	дебильные ошибки:
		-спутать параметры: первым -- инициализация, то есть точечку создать надо было, например: Point::new \\ () -> {return new Point(0, 0, 0)}
	потом -- combiner: (p1, p2) -> new Point(...)

		-ошибка в формуле для центра масс

1 задача:
%решето Эратосфена на стримах %)
%just for fun
\begin{javacode}
List<Integer> primes = new ArrayList<>();

IntStream.iterate(2, i -> i + 1).filter(i -> {
		for(int prime: primes) {
			if(i % prime == 0) {
				return false;
			}	
		}
		return true;
	}).limit(100).forEach(primes::add);

//но не параллельно

//числа идут слева направо, проходят через ситечко уже нацеженных простых чисел
//и оседают в нём только в том случае, если ни на кого не делятся
\end{javacode}

List<Integer> $\ra$ IntStream //(всегда; чтобы работало быстрее; это мы чтобы один раз unboxing сделать, а не туда-сюда разворачивать)

Обычный filter с проверкой на простоту, просто parallelStream -- не меняется от этого \textbf{ничего}.


5 pадачка:
1 способ: try-catch
2 способ: 
	sc = new Scanner(str)
	sc.hasNextInt()

это как в "количестве уникальных слов" с пары
Только вместо Identity оставляем лямбду, оставляющую от строчки первый символ


\section{Возвращаемся к сетям}%aka к нашим барашкам
Напоминание:
человек, роутер -> цепочка маршрутизаторов
общение в интернете по http
запрос на vk.com:80

app		http
transp		tcp, udp		формируем dns-запрос нашему dns-серверу, узнали IP-адрес vk
network		IP			смотрится, павда ли 2 компьютера в одной локальной сети. Применяет маску, если нет, то напрямую нельзя -- надо в шлюз(gate)
link		Ethernet		mac-адрес шлюза, mac-адрес себя


Прилетел пакетик на шлюз. Мой пакетик? нет => выкидываем. да => передаём на уровень выше





UDP не гарантирует доставку данных, TCP -- гарантирует

2 генерала стоят на двух холмах рядом с непристун=пной крепостью
За каждым армия
Но нужно обязательно обе армии напасть
Они хотят договориться о времени штурма
Связь?
Лазутчики -- но через город
man-in-the-middle
Ответа нет, невозможно договориться, если нет знания, которым обладают эти и только эти двое.
%image1

К чему задача? Нет способа гарантировать доставку данных без внешнего знания.
Чем поможет это знание? С его помощью можно шифровать наши данные.
От недоставки мы не застрахованы, но пошлём трёх человек с одинаковым сообщением -- там разберутся.

То есть TCP ничего не гарантирует.

Зачем же придумали UDP? Например, передача видео по сети: если по дороге пропал один кадр -- это фигня, это лучше, чем если доставятся все кадры, но с задержкой. 
Вторй вариант -- торренты. Вместо того, чтобы скачивать файл из одного места, мы получаем табличку с пользователями, которые уже скачали, и уже у них по небольшому кусочку тырим. Там можно повторить запрос.

TCP -- засчёт чего гарантии? Дополнительные вопросы-ответы. Каждый пакет нумеруется. 8 9 11... -- тут мы просим переотправить нам 10. Если так и не получается, то шлём извиннения и рвём соединение. То есть он очень медленный, а особенно в ситуации плохой среды -- на несколько порядкой медленнее, чем udp. 
Также этот протокол поддерживает соединение. Официальненько Устанавливаем Соединение, периодически осведомляемся о здоровье-с противоположной партии, не изволили ли они, паче чаяния, соединение прервать. %куртуазненько общаются, короче




Network		ICMP
Internet Control Message Protocol
Что это? Начнём издалека.

Как устроен пакет IP кровня Network:
%imageo2
TTL -- jed time, когда пакет проходит через маршрутизатор, делается --, и если ноль, то пакет убивается
Зачем?
Если админ криворукий и у него пакеты ходят%каруселькаэ
по кругу между двумя маршрутизаторами

На точке доступа можно сравнить TTL пакетов, которые приходят с человека и тогда понять, выходит через него кто-то в интернет или нет: если есть различные TTl, то выходят.

ICMP позволяет получать сообщения об ошибках при работе с пакетами. Если кому-то по TTL пришлось выбросить пакет, то он обратно отошлёт сообщение о том, что он выброшен. Чтобы TCP понимал, как там у него дела у пакетиков отправленных.
Если IP-пакет потерялся, то он отправляет ICMP-пакет об ошибке. Если тот потерялся, то ничего страшного. Если пакеты не приходят -- соед=инение разорвалось.

traceroute -- позволяет узнать IP-адреса всех маршрутизаторов по дороге к какому-то адресу.
Как работает:
Можем послать пакет с TTL = 1 на куда-нибудь, куда надо, на первом же маршрутизаторе пакет умер, а нам вернулось сообщение с ошибкой и ... IP-адресом маршрутизатора :)
Но не совсем корректно: нет гарантии, что наши пакеты будут всегда идти одинаковым маршрутом.

Есть 2 типа ICMP-пакета: для ping-запроса и ping-ответа, а ещё один -- об ошибке.

ICMP-пакеты -- большая нагрузка для сети.

Типичная задача ICMP -- синхронизация времени.

Задачка: есть 2 компа со своим локальным временем. Как им синхронизоваться? 
Один ICMP-запрос, в который мы вкладываем наше время отправления 1, на втором компе записывает время полчения2, время отправления [обратно] 2, мы, получив обратно, приписываем время получения 1
t1, t2, t3, t4 условно
Как по ним понять дельту, на которую синхронизоваться?
Что такое $(t2 - t1)$: 
$(t4 - t1) - (t3 - t2)$ -- суммарное время в пути, поделим на два -- примерно в одну сторону
По идее, если мы идеально синхронизованы, то $t1 + \text{время\_в\_одну\_сторoну} = t2$. Поскольку не равно, то отсюда и дельта.



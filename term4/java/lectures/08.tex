\section{Что на джаве сейчас пишут?}

В основном, бекэнд и всякие серверные вещи. Потому что Джава не самый быстрый язык (нет, ладно, с Python успешно конкурирует).
А вот с плюсами уже не очень.
Поэтому осмысленно на ней писать большую сложную систему: 
\begin{enumerate}
\item
ООП + строгая типизация -- важны в больших проектах
\item
ООП -> удобная модульность
\item
программы запускаются на Java-машине -- контейнере, который умеет во всякие плюшки.
\end{enumerate}

А писать десктопные приложения на джаве можно, но вряд ли оно будет настолько большое и сложное, чтобы не написать его на чём-то другом.

JavaEE -- yабор соглашений к языку программирования. Она вся основана на \textit{аннотациях}

\section{Аннотации}
Например, \java"@Override" "--- это хорошо контролирует ошибки: если написали её к методу, отсутствующему в родительскуом классе, 
то бужет ошибка компиляции.

Основная задача аннотаций -- статическое расширение классов. 
Что значит расширение? Положить вместе с классом некоторую дополнительную статическую информацию, то есть менять её нельзя.
Где её использовать? Во-первых, во время компиляции "--- 
%голосом Дроздова
для отлова ошибок. 
И наоборот, когда ошибки на самом деле не нужны.

А ещё это даёт дополнительную функциональность "--- генерация кода.

А ещё можно использовать прямо во время работы и смотреть, например, на наличие/отсутствие каких-либо аннотаций в класcе.

\subsection{How-to: как написать аннотацию}
\begin{javacode}
@Target(ElementType.TYPE) //определяет, к чему можно будет приписывать эту аннотацию; в этом случае "--- к типам: классам, интерфейсам
@Retention(RetentionPolicy.RUNTIME) // информацию о наличии этой аннотации в классе можно будет получать в runtime'e
				    // на этапе компиляции её видно не будет
public @interface Mammal { // "@interface" === "Джава, смотри, я пишу свою аннотацию!"
	String sound();
	int color() default 0xffffff; // цвет по умолчанию -- белый
}
\end{javacode}
Научимся это приписывать к чему-нибудь.
\begin{javacode}
@Mammal(color = oxffa844, sound = "uuuu") // задаём полям значения
class Giraffe {
}
\end{javacode}

Retention:
\begin{enumerate}
\item
\java"SOURCE" "--- сбрасывается при создании .class файла
\item
\java"CLASS" "--- в .class хранится, используется в основном для класслоадеров (они работают с бинарными файлами, поэтому недоступность в runtime'e их мало волнует)
\item
\java"RUNTIME"
\end{enumerate}


\java"User.Permission" из примера -- это enum.

Ми-ми-маленькая хитрость: в аннотациях можно делать и массивы аннотаций, да.

\subsection{Как используем}
Это информация, которая навешена на класс сверху "--- не лезет во внутреннюю реализацию.
\subsection{Сериализация}
Можно, конечно, написать класс "СериализаторЯблока". Но когда на вход валится куча объектов "--- это писать огромный \java"if", разбирая все
возможные варианты входа? Это же ужас.

А что предлагается:
\begin{javacode}
@interface SerializedBy {
	Class<? extends Serializer> value();
}
// написать такую аннотацию, мы внутри кода сохраняем связь между объектом и классами, которые умеют его сериализовать
\end{javacode}

%Вопрос о студента Л.:
%Л: А чё, аннотацию можно отнаследовать?
%АМ: Можно. Не стоит!! - но можно.

\subsection{Ещё немного аннотаций}
\begin{itemize}
\item
\java"@NotNull" "--- позволяет избежать внезапных \java"NullPointerException".
\item
\java"@Nullable" "--- наоборот, здесь может всякое быть. Зачем эта аннотация нужна? 
Ну, скорее для определённости, что мы не забыли про это место, а подумали и уверены, что нам так надо.
\end{itemize}

\subsection{Ещё примеры}
Пусть мы пишем приложение, там есть сервер, в нём есть класс, отвечающий за связь с базой данных, "--- \java"DB".
Пусть он в качестве параметров конструктора должен что-нибудь получить "--- например, имя сервера и пароль.
Откуда нам их взять? 
Ну, наверно, из какого-то конфига.
А мы хотим это полем в нашем классе приложения.
Очень не хотим возиться со всей этой инициализацией руками "--- то ли дело написать \java"@Inject", 
дальше разберётся сама JVM, только настроить её надо.
\begin{javacode}
Application {
	@Inject
	DB db;
}
\end{javacode}

\subsection{Промежуточный итог}
Оченб мощная вещь. Прям боевая магия. Про всё это в слеующих сериях.

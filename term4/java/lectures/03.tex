\chapter{Дивертисмент: ликбез про сети}
Книжка: TCP/IP Illustrated Vol. 1 (1-2 chapters)

Чтобы отправить куда-то пакет "--- данные "--- нужно знать его адрес "--- IP-адрес. 
Состоит из четырёх байт (IPv4) "--- четырёх чисел: XXX.XXX.XXX.XXX
Соответственно, у каждого компьютера должен быть свой IP-адрес, если он хочет работать в сети.
Но давайте считать: 
порядка четырёх миллиардов адресов, 
при семи миллиардах человек, 
а на каждого человека ещё и не по одному компу, а по $2.5$.
Поэтому мы мечтаем перейти на IPv6, чтобы было побольше адресов 
(эти-то уже кончились давно, хотя планировались уникальными), 
но с ним как всегда: он почти везде поддерживается и почти везде не поддерживается.

Посмотрим:
\begin{javacode}
10.*.*.*
192.168.*.*
127.*.*.*
172.X.*.*
\end{javacode}
эти IP-адреса запрещены для испольщования в интернете.
Это сделано для локальных сетей: у почти всех компов в ней невалидные для интернета адреса и есть набор валидных, несколько штук.
Так что для внешнего мира это всего те несколько компьютеров, 
через которые все остальные машины в их локальной сети выходят в интернет.


\begin{exmp}
У студента Р. адрес есть, но невалидный, у \texttt{vk.com} есть нормальный адрес.
\textit{(это, как всегда, не совсем правда)}
Как студенту Р. отправить данные во внешний мир?
Можно отправить на точку доступа, а потом от ней "--- дальше.
А как не во внешний мир, а, например, другому студенту "--- Г.?
Кажется, что можно напрямую, если они в одной локальной сети.
Принципиально есть два варианта: напрямую к другому пользователю или через место, через которое выходим в интернет.

Это место "--- точка доступа "--- нужна в том числе для безопасности.
\end{exmp}
Чтобы отправлять данные напрямую, вводится локальная сеть. 
Чтобы понять, что два устройства лежат в одной локальной сети, вводится пониятие маски сети.
Два варианта записи:
\begin{itemize}
\item
\texttt{192.168.65.13/24} (первые $24$ единички)
\item
\texttt{255.255.254.0} ($23$ единички)
\end{itemize}

Компы находятся в одной локальной сети, если у них биты, соответствующие единичным битам маски, совпадают.
\texttt{M \& IP1 === M \& IP2} (можно проксорить)

И ещё одна вещь "--- gateway "--- шлюз "--- IP-адрес компьютера в нашей(это обязательно!) локальной сети, 
через который будет осуществлено взаимодействие со внешним миром.
Откуда шлюз знает, куда отправлять пакеты дальше?
В самом простом варианте у шлюза есть такая же штука: 
он есть в ещё какой-то локальной сети (например, wi-fi и провод).
И он понимает, что если мы шлём на комп из какой-то его локальной сети, то напрямую, 
а иначе у него у самого есть шлюз, на который он дальше пересылает.

Одна проблема: студент Р. написал \texttt{vk.com}, а не \texttt{72.47.10.9}
Чтобы это работало, есть специальная штука "--- \texttt{DNS}, \texttt{domain name service}.
Это такая служба, основной задачей которой является ответ на вопрос: какой IP-адрес у компа с указанным именем.
Но это надо у кого-то спросить, чтобы узнать "--- у     exttt{DNS-server}'a. Он необязательно лежит с нами в одной сети, но его IP-адрес нам надо щнать.
Например, у \texttt{DNS-server}'a гугла адрес \texttt{8.8.8.8}

Т.о. сначала студент Р. отправил запрос к \texttt{DNS-server}'у, получает адрес \texttt{vk.com}, отправляет данные напрямую/через шлюс(зависит от локальности)

Как работает DNS?
Ответ на вопрос: либо есть в кеше, либо спрашиваем у вышестоящего DNS-сервера.
Есть корневые \texttt{DNS-server}'a, их $13$ штук + резервные "--- что они знают? Довольно мало, он знает, какой комп знает всё про домены первого уровня
\texttt{.ru}, \texttt{.com}
Он, в свою очередь, знает, какой компьютер знает что-то про зону \texttt{vk.ru}(домен второго уровня)

Две модели общения при общении с \texttt{DNS-server}'ами:
\begin{enumerate}
\item
Плюсы "по уепочке обратно": 
все закешировали ответ.
Минусы: трафик.
\item
Плюсы "обращаться напрямую к следующим":
полегче с трафиком.
\end{enumerate}

Раз в год проподится \texttt{DNS hack-day}.


Что сделать, чтобы завести ''\texttt{vasya.com}'':
есть организация, отвечающая за домен ''\texttt{.com}'', 
ей надо заплатить, чтобы она узнала о том, что именно твой комп за этот домен отвечает. 
Ну, организация такая не одна: есть
VeriSign "--- американская, есть
InterNIC "--- кому она принадлежит? Американскому правительству, $50$\% акций + одна.


Если очень хочется пройти мимо провайдера "--- ''хочу напрямую к облачку'': 
есть точка подключения (штук 13 в Питере). 
За подклоючение к ним провайдеры платят денежку.
Есть длинный кабель, втыкающийся в эти точки "--- большие маршрутизаторы, 
они принадлежат частным компаниям (RunNet, Ростелеком, ещё кто-то).
У нас есть провод в Москву, в Хельсинки, спутниковая тарелка на ИТМО. 
Есть кабели по дну Атлантического океана, тарелки в Норвегии по тысяче терабит в секунду.
Антенны в США.
Кабели между материками.
Кабели оптоволоконные.

На обоих концах этого кабеля стоит маршрутизатор и принадлежит США.

Любая страна "--- это собственник маршрутизаторов, стоящих на её границе.
Великий китайский \texttt{firewall}, минуя их сервера выходить в интернет "--- в Китае это тстуденту Р.мный срок.



Возвращаемся к студенту Р. и \texttt{vk.com}
Как-то чудно студент Р. получил ответ, пока чудно.
Проблема: у студента Р. в браузере не одна вкладка.

Порт "--- это не идентификатор компьютера, это идентификатор [сетевого соединения нашего] приложения уже на нашем компьютере.
То есть студент Р. невяно пишет \texttt{vk:com:80}.
Из порта данные передаются в соответствующую вкладку.


А теперь о том, как студент Р. выходит с запрещённым IP-адресом в интернет.
Студент Р. пишет запрос к \texttt{vk.com} на порт 80. 
В запросе от студента Р. написан его IP-адрес, порт его браузера для ответа, написан IP-адрес вконтакте и порт на вконтакте, 
куда студенту надо.
Пакет с этими данными доходит до точки доступа. 
Если дальше начался интернет, то дальше этот пакет пропускать нельзя "--- там адрес плохой.
Поэтому точка доступа делает преобразование и заменяет пакет на другой: 
IP-адрес точки доступа, порт на ней, тот же адрес и порт вконтактика.
А кому передать ответ, точка доступа запомнит.
Преобразование, которое сделала точка доступа "--- это NAT: network address translation.
Резонный вопрос: сколько точка доступа держит?
Сколько портов: от $0$ до $(2^16 - 1) \ra$ точка доступа не может одновременно поддержать больше ~$65$K соединений.
(на самом деле сильно меньше: не все порты для этого + памяти-то у точки доступа не так много)
Запись обычно живёт не больше несколькихз десятков секунд: 
во-первых, вряд ли тебе уже через так долго ответят, 
во-вторых, записи вытесняются.
Это причина, почему во многих общественных местах запрещены торренты: 
торренты открывают огромное количество соединений, точка доступа не выдержит. %не выжержит же ни фига

%табличка
<картинка с табличкой тут>

\begin{itemize}
\item
В Application "--- это протоколы, на которых общаются друг с другом наши приложения.
http "--- общение страничек.
ssh "--- удалённый терминал.
...

\item
%В Link "--- yfib ecnhjqcndf abpbxtcrb j,of.ncz lheu c lheujv^ gj l`kntymrjve ghjdjle? gj db-ab
В Link "---  наши устройства физически общаются друг с другом: по беленькому проводу, по ви-фи.

\item
2 промежуточных уровня.

Зачем все эти 4 уровня сделаны?
Пишем хром. 
Нам хочется не думать о том, как хардварно работает наш пользователь. 
Я приложение, я хочу абстрагироваться от непосредственной передачи данных по физическому носителю.

\item
IP-уровень. Здесь появляется абстракция "IP-адрес", не зависящий от способа физической передачи данных по сети.

\item
Transport "--- вводится абстракция "порт". 
Чтобы приложения на одном IP-адресе разделядлись и получали свои данные.

\item
Уровень приложений "--- только наше приложение знает, как оно непосредственно передает данные.

\end{itemize}

И вся эта штука называется стек протоколов TCP/IP.

Почему стек?
Приолжение хочет отправить данные $\ra$ 
говорит компу "отправь туда-то" $\ra$ 
TCP добавляет в пакет шапочку(с нашим адресом и портами отправителя и получателя) и хвостик(контрольная сумма) $\ra$ 
данные передаются на уровень ниже и запаковываются в пакет уровня "IP" $\ra$ 
в этом пакете приписывается IP-адрес отправителя и получателя $\ra$ 
в зависимости от техноолоиии заворачивается в пакет link-level'a(пусть Ethernet) и отправляется в сеть.

Пусть эти два компа находились в одной локальной локальной сети.
На линк уровень нам пришёл пакет.
Но часто нам приходжит всё подряд. 
Поэтому вводится ещё и mac-адрес(вписывается на линк-уровне), адрес сетевой платы.
По получении в первую очередь смотрим на мак-адрес и, если он наш, разворачиваем и передаём выше.
Там проверяется, тот ли IP адрес. 
Если да, то разворачиваем и передаём на уровень выше. 
Посмотрели на порт, определили приложение, развернули, передали. 
Приложение может спросить, кто отправил mac, IP, ...

Вопрос: а mac-адрес "--- это не лишнее?
Нет. 
\begin{itemize}
\item Во-первых, он специфичен для Ethernet.

\item Во-вторых, работаем чеерз шлюз. 
Если сошёлся мак-адрес, но не IP, то шлюз \textit{заворачивает обратнро} и передает следующему шлюзу.
\end{itemize}

Если IP не интернетный ???
Разворачивание пакетов туда-сюда "--- довольно дорогая операция. %SwitchWare1,2,3

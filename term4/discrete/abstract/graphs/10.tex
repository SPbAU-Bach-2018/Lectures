% 8.3 (стр. 56)
\section{} % 10
	Граф совершенен, если для любого индуцированного подграфа $\omega=\chi$ (индуцированный, чтобы не смотреть на случай <<хитрый граф + независимая клика>>).
	Слабая гипотежа Бержа: $G$ совершенен $\iff$ $\bar G$ совершенен (иногда удобнее доказывать совершенность для дополнения).
	Сильная гипотеза Бержа (без док-ва): $G$ совершенен $\iff$ ни $G$, ни $\bar G$ не имеют индуцированного нечётного цикла длины хотя бы пять.

	% стр. 57
	Двойственность от кэпа (нужна в док-ве): $\alpha(G)=\omega(\bar G)$, $\chi(G)=\vartheta(\bar G)$ (минимальное число клик для покрытия, т.к. клика в $\bar G$ есть независимое мн-во в $G$).
	Если дополнение графа совершенно, то в нём $\chi(\bar G)=\omega(\bar G)$, отсюда по капитану $\alpha(G)=\vartheta(G)$.
	Если же сам граф совершенен, то в нём $\chi(G)=\omega(\bar G)$, откуда по капитану $\alpha(\bar G)=\vartheta(\bar G)$.
	Т.е. слабая гипотеза "--- равносильность этих двух равенств (т.к. для подграфов можно по индукции).
	А, ещё в любом графе $\vartheta(G)\ge \alpha(G)$ (так как независимые вершины не могут лежать в одной клике) и $\chi(G)\ge \omega(G)$.
	Т.о. для слабой гипотезы достаточно по индукции показывать, что в совершенном $G$ верно: $\chi(\bar G) \le \omega(\bar G)$.

	% стр. 58
	Операция расширения вершины на ребро: скопировали $x$ в $x'$, соседей оставили тех же, соединили $x \to x'$.
	Тогда если $G$ был совершенен, то расширенный $G'$ тоже совершенен.
	Док-во: индукция по числу вершин (база "--- $K_1$).
	Переход: любой собственный индуцированный подграф $G'$ "--- либо подграф $G$, либо расширенный подграф $G$, т.е. по предположению всё ок.
	Осталось показать, что $\chi(G') \le \omega(G')$.
	Если $x$ лежала в максимальной клике, то мы эту клику увеличили на 1, увеличили $\omega(G')$, а $x'$ можем покрасить в новый цвет.
	Пусть не лежал и был окрашен в цвет $i$.
	Знаем, что в каждой клике есть вершина цвета $i$ (т.к. граф был совершенен).
	Пусть в цвета $i$ окрашены вершины из мн-ва $S\cup \{x\}$.
	Тогда в графе $G-S$ все максимальные клики убились (он совершенен $\Ra$ можно покрасить в $\omega-1$ цвет).
	Т.к. $S \cup \{x\}$ "--- вершинно независимое, то $S \cup \{x'\}$ тоже вершинно независимо.
	Значит, можно его вернуть в $G-S$ и покрасить в новый цвет $\omega-1+1$, успех.

	% стр. 59
	Доказываем Бержа индукцией по числу вершин, надо показать, что в совершенном $G$ верно $\chi(\bar G) \le \omega(\bar G)$.
	Пусть $\mathcal{K}$ "--- мн-во всех клик, а $\mathcal{A}$ "--- мн-во всех максимальных независимых подмножеств.
	Если есть клика $K$, которая пересекается со всеми элементами $\mathcal{A}$, то можно удалить $K$,
	получим $\alpha(G)>\alpha(G-K)=\omega(\bar G - K)$, т.е. граф $\bar G - K$ можно покрасить в не более чем $\omega(\bar G)-1$ цвет.
	А т.к. $K$ было кликой в $G$, то в $\bar G$ оно независимое мн-во, можно докрасить в цвет $\omega(\bar G)$, отсюда $\chi(\bar G)\le\omega(\bar G)$.

	% стр. 60
	Теперь от противного покажем, что такая клика есть: пусть для любой клики $K_i$ есть максимальное независимое $A_i$ такое, что $K_i \cap A_i=\varnothing$.
	Посчитаем для каждой вершины $x$ количество $A_i$, её содержащих, обозначим $k(x)$ ($A_i$ могут повторяться, никаких проблем).
	Тогда вершины с $k(x)>0$ индуцируют подграф $H$ (непустой же), он тоже совершенен.
	Расширим в нём все вершины до клик размера $k(x)$ (останется $\alpha(H')=\alpha(H)$), он по лемме совершенен, т.е. $\omega(H')=\chi(H')$.
	В нём будет $\sum k(x) = \sum |A_i| = |\mathcal{A}|\alpha(G)$ вершин.
	Также знаем, что $\chi(H') \ge \sfrac{|V(H')|}{\alpha(H')}$ (т.к. вершины одного цвета "--- независимое мн-во), а $\alpha(H') = \alpha(H) \le \alpha(G)$, т.о.
	$\chi(H')\ge |\mathcal{A}|\alpha(G)/\alpha(G)=|\mathcal{A}|$.
	Посчитаем $\omega(H')$ по-другому: клика в нём "--- это раздутая клика в $H\subseteq G$.
	Пусть это была клика $K_r$, тогда $\omega(H')=\sum_{x \in K_r} k(x)=\sum_{i} |K_r \cap A_i|$ (последний переход "--- расписали $k(x)$).
	Но клика с независимым множеством пересекается максимум по вершине и $K_i \cap A_i=\varnothing$, т.е. $\omega(H') \le |\mathcal{A}|-1$.
	Упс: $\omega(H') < \chi(H')$, но $H'$ был совершенен.

% Раскраска графов (стр. 43)
% 7. k-раскрашиваемые графы. Теорема Брукса (стр. 43).
\section{} % 08
	Раскраска графа (по умолчанию "--- вершин) правильная, если смежные покрашены в разное.
	Граф бывает $k$-раскрашиваем, минимальное $k$ "--- его хроматическое число $\chi(G)$.
	Тут смотрим только на простые графы (если петля "--- сразу нельзя покрасить, если мультиребро "--- можно убрать),
	еще можно смотреть отдельно на компоненты.
	Случай $\chi(G)=1$ "--- пустой граф, $\chi(G)=2$ "--- двудольный (без нечётных циклов).
	Классическая задача: расписание экзаменов, два экзамена нельзя в один день, если есть пересечение по студентам.
	Ещё одна: какие-то химикаты нельзя хранить в одной комнате.
	Проверка на 3,4-раскрашиваемость (и выше) NP-полна, а поиск $\chi$ "--- и подавно.

	% стр. 45
	Жадный алгоритм: взяли порядок вершин, красим слева направо в минимальный цвет.
	Покрасит в не более $\Delta+1$ цвет.
	Значит, $\chi\le\Delta+1$ и в любом $k$-хроматическом есть вершина степени хотя бы $k-1$.
	В двудольных графах вообще не достигается, а вот для $K_n$ и $C_{2k+1}$ достигается.
	Теорема Брукса: если простой связный (\TODO уточнить формулировку) граф не полон и не нечётный цикл, то $\chi \le \Delta$.
	Случаи $\Delta \le 2$ разобрали отдельно ($K_1$, $K_2$, циклы).
	% стр. 47
	Если граф не регулярен, то найдём вершину степени $<\Delta$, запустим из неё dfs, в таком порядке запустим жадник
	(у каждой покрашены только дети, а для всех, кроме корня, есть ещё родитель, успех).
	Если регулярен, но есть точка сочленения: по ней пилим граф на блоки, каждый красим, потом перекрашиваем, чтобы цвет точки сошёлся.
	% стр. 49
	Если нет точек, то двусвязен.
	Ниже покажем, что тогда есть $x$, смежные с ней (но не между собой) $y$ и $z$, причём $G-y-z$ связен (чтобы по-прежнему был один корень дерева).
	Тогда расположим их в начале списка (покрасятся в один цвет), а остальные "--- dfs'ом из $x$.
	Все, кроме $x$, жадно покрасятся в $\le \Delta$, а $x$ "--- тоже, так как два соседа имеют один цвет.
	
	Лемма: в вершинно двусвязном регулярном ($\Delta \ge 3$) есть индуцированный путь $y \to x \to z$, причём $G-y-z$ связен.
	Выберем любую вершину $w$.
	Если точек сочленения в $G-w$ нет, то найдём в $G$ вершину на расстоянии 2 от $w$ (если нет, то $G$ был полон), нашли нужную тройку.
	А если есть, то посмотрим на крайние блоки $B_1$ и $B_2$ (листья в дереве блоков), в каждом ровно одна точка сочленения: $v_1$ и $v_2$
	(не упростить до одной общей точки сочленения $v_1=v_2$, т.к. бывает бамбук из блоков).
	Найдём в $B_1-v_1$ и в $B_2-v_2$ смежные с $w$ (есть, т.к. исходный двусвязен, удалить можно любую вершину).
	Они несмежны, т.к. иначе блоки схлопнутся.
	При этом $G-v_1-v_2$ связен, т.к. степень $w$ хотя бы 3, блоки были двусвязные, значит, можно из каждого безопасно удалить по вершине,
	а $w$ потом подцепить.
	\TODO зачем нам крайность блоков?

	% стр. 50
	Граф $k$-критический, если $\chi=k$ и любой собственный подграф можно раскрасить в $k-1$.
	В любом графе есть $\chi(G)$-критический подграф (удаляем рёбра/вершины пока не надоест).
	Критический связен, т.к. можно удалить менее сложную компоненту.
	В критическом нет вершинно разделяющего $S$, являющееся кликой.
	От противного: пусть есть, тогда $G-S$ развалился на компоненты, можно к каждой добавить $S$ (это т.н. $S$-компоненты), покрасить в $k-1$ цвет (по критичности).
	Т.к. $S$ клика, то все её вершины в разные цвета в разных компонентах, можно перенумеровать цвета и согласовать, упс.
	Следствие: точек сочленения нет, иначе есть $S=K_1$.
	Следствие: критический двусвязен.

	% стр. 51
	Теорема Дирака: если в $k$-критическом есть вершинно разделяющее мн-во $\{x, y\}$, то
	$G$ есть объединение двух $S$-компонент, у одной $x$ и $y$ всегда разных цветов, у другой "--- всегда одного.
	Следует из того, что у хотя бы одной $S$-компоненты не должно быть раскраски, при которой все вершины $S$ разного цвета
	(иначе сможем все компоненты перекрасить и согласовать).
	Аналогично, хотя бы у одной не должно быть раскраски, при которой все вершины $S$ одного цвета.

% 8. Нижние оценки на хроматическое число. Теорема Турана. Совершенные графы (стр. 52)
\section{} % 09
	Клику надо красить в разные цвета, поэтому $\omega(G) \le \chi(G)$.
	Но есть конструктивный Мицельский: для любого $k$ можно построить граф без треугольников с $\chi=k$.
	Начинаем с чего угодно (проще с $K_2$), потом делаем для увеличения $k$ так: для каждой $x_i$ создали $y_i$, соединили новую со всеми соседями $x_i$,
	а потом создали $z$ и подцепили ко всем $y_i$.
	% стр. 53
	Док-во по индукции: треугольников нет ($y_i$ несмежны, и соседи $x_i$/$y_i$ тоже несмежны),
	окрасить в $k+1$ можно ($y_i$ "--- так же, как и $x_i$, а $z$ в новый),
	а вот в $k$ нельзя (окрасим $z$ в цвет $k$, все $y_i$ окрасились в $k-1$, но тогда так же можно окрасить $x_i$ \TODO\textit{неправда, см. вопрос 11 в консультации}, упс).
	Теорема Эрдёша (нужна ли? без док-ва \TODO): для любого $k$ есть $k$-хроматический с обхватом $\ge k$ (обхват "--- минимальная длина цикла).
	Это более круто, чем Мицельский, конструктивного док-ва у Эрдёша не было, потом сделали.

	Граф $k$-долен, если можно разбить вершины на доли (покрасить в $k$ цветов).
	Бывает полный $k$-дольный граф, его дополнение "--- $k$ клик.
	Макс. число рёбер в $k$-дольном графе (при фикс. числе вершин) достигается в полном графе, где все доли почти равны (разница не больше единицы,
	такое разбиение $n$ единственно), это числе обозначим за $M(n, k)$, надо доказать (если есть разница больше единицы "--- передвинем вершину между долями).
	Считается так: число рёбер в полном графе минус число рёбер в дополнении (сколько-то клик одного размера, сколько-то клик другого размера),
	в формуле будет $n \bmod k$, проверить $M(3, 2)=2$.
	
	% стр. 55
	Туран: если в графе на $n=tk+r$ вершинах больше $M(n, k)$ рёбер, то есть клика размера $k+1$ ($\Ra \chi>k$).
	Возьмём граф $G$ на $n$ вершинах без больших клик с макс. числом рёбер.
	Покажем индукцией по $t$ (ага, по частному при фикс. остатке), что в графе не более $M(n,k)$ рёбер.
	База тривиальна: у нас $n=r$, т.е. граф $K_n$.
	Переход: сейчас в $G$ клики $K_{k+1}$ нет, но при добавлении любого ребра появляется (т.к. число рёбер максимально).
	Значит, сейчас есть клика $S$ размера $k$, в неё $\binom{k}{2}$ рёбер, а все остальные вершины смежны с не более чем $k-1$ вершиной из $S$ (иначе нашли клику $k+1$).
	При этом в $G-S$ тоже нет клики размера $k+1$, т.е. в нём не более $M(n-k,k)$ рёбер, складываем число рёбер в клике, в $G-S$ и в разрезе $[S;G-S]$, получили в точности $M(n,k)$.

% 8.3 (стр. 56)
\section{} % 10
	Граф совершенен, если для любого индуцированного подграфа $\omega=\chi$ (индуцированный, чтобы не смотреть на случай <<хитрый граф + независимая клика>>).
	Слабая гипотежа Бержа: $G$ совершенен $\iff$ $\bar G$ совершенен (иногда удобнее доказывать совершенность для дополнения).
	Сильная гипотеза Бержа (без док-ва): $G$ совершенен $\iff$ ни $G$, ни $\bar G$ не имеют индуцированного нечётного цикла длины хотя бы пять.

	% стр. 57
	Двойственность от кэпа (нужна в док-ве): $\alpha(G)=\omega(\bar G)$, $\chi(G)=\vartheta(\bar G)$ (минимальное число клик для покрытия, т.к. клика в $\bar G$ есть независимое мн-во в $G$).
	Если дополнение графа совершенно, то в нём $\chi(\bar G)=\omega(\bar G)$, отсюда по капитану $\alpha(G)=\vartheta(G)$.
	Если же сам граф совершенен, то в нём $\chi(G)=\omega(\bar G)$, откуда по капитану $\alpha(\bar G)=\vartheta(\bar G)$.
	Т.е. слабая гипотеза "--- равносильность этих двух равенств (т.к. для подграфов можно по индукции).
	А, ещё в любом графе $\vartheta(G)\ge \alpha(G)$ (так как независимые вершины не могут лежать в одной клике) и $\chi(G)\ge \omega(G)$.
	Т.о. для слабой гипотезы достаточно по индукции показывать, что в совершенном $G$ верно: $\chi(\bar G) \le \omega(\bar G)$.

	% стр. 58
	Операция расширения вершины на ребро: скопировали $x$ в $x'$, соседей оставили тех же, соединили $x \to x'$.
	Тогда если $G$ был совершенен, то расширенный $G'$ тоже совершенен.
	Док-во: индукция по числу вершин (база "--- $K_1$).
	Переход: любой собственный индуцированный подграф $G'$ "--- либо подграф $G$, либо расширенный подграф $G$, т.е. по предположению всё ок.
	Осталось показать, что $\chi(G') \le \omega(G')$.
	Если $x$ лежала в максимальной клике, то мы эту клику увеличили на 1, увеличили $\omega(G')$, а $x'$ можем покрасить в новый цвет.
	Пусть не лежал и был окрашен в цвет $i$.
	Знаем, что в каждой клике есть вершина цвета $i$ (т.к. граф был совершенен).
	Пусть в цвета $i$ окрашены вершины из мн-ва $S\cup \{x\}$.
	Тогда в графе $G-S$ все максимальные клики убились (он совершенен $\Ra$ можно покрасить в $\omega-1$ цвет).
	Т.к. $S \cup \{x\}$ "--- вершинно независимое, то $S \cup \{x'\}$ тоже вершинно независимо.
	Значит, можно его вернуть в $G-S$ и покрасить в новый цвет $\omega-1+1$, успех.

	% стр. 59
	Доказываем Бержа индукцией по числу вершин, надо показать, что в совершенном $G$ верно $\chi(\bar G) \le \omega(\bar G)$.
	Пусть $\mathcal{K}$ "--- мн-во всех клик, а $\mathcal{A}$ "--- мн-во всех максимальных независимых подмножеств.
	Если есть клика $K$, которая пересекается со всеми элементами $\mathcal{A}$, то можно удалить $K$,
	получим $\alpha(G)>\alpha(G-K)=\omega(\bar G - K)$, т.е. граф $\bar G - K$ можно покрасить в не более чем $\omega(\bar G)-1$ цвет.
	А т.к. $K$ было кликой в $G$, то в $\bar G$ оно независимое мн-во, можно докрасить в цвет $\omega(\bar G)$, отсюда $\chi(\bar G)\le\omega(\bar G)$.

	% стр. 60
	Теперь от противного покажем, что такая клика есть: пусть для любой клики $K_i$ есть максимальное независимое $A_i$ такое, что $K_i \cap A_i=\varnothing$.
	Посчитаем для каждой вершины $x$ количество $A_i$, её содержащих, обозначим $k(x)$ ($A_i$ могут повторяться, никаких проблем).
	Тогда вершины с $k(x)>0$ индуцируют подграф $H$ (непустой же), он тоже совершенен.
	Расширим в нём все вершины до клик размера $k(x)$ (останется $\alpha(H')=\alpha(H)$), он по лемме совершенен, т.е. $\omega(H')=\chi(H')$.
	В нём будет $\sum k(x) = \sum |A_i| = |\mathcal{A}|\alpha(G)$ вершин.
	Также знаем, что $\chi(H') \ge \sfrac{|V(H')|}{\alpha(H')}$ (т.к. вершины одного цвета "--- независимое мн-во), а $\alpha(H') = \alpha(H) \le \alpha(G)$, т.о.
	$\chi(H')\ge |\mathcal{A}|\alpha(G)/\alpha(G)=|\mathcal{A}|$.
	Посчитаем $\omega(H')$ по-другому: клика в нём "--- это раздутая клика в $H\subseteq G$.
	Пусть это была клика $K_r$, тогда $\omega(H')=\sum_{x \in K_r} k(x)=\sum_{i} |K_r \cap A_i|$ (последний переход "--- расписали $k(x)$).
	Но клика с независимым множеством пересекается максимум по вершине и $K_i \cap A_i=\varnothing$, т.е. $\omega(H') \le |\mathcal{A}|-1$.
	Упс: $\omega(H') < \chi(H')$, но $H'$ был совершенен.

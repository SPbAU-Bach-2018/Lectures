% Паросочетания в графах (стр. 15)
% 4. Понятие паросочетания. Теорема Бержа. (стр. 15)
\section{} % 03
	Берж "--- максимум, если нет дополняющих.

	Паросочетание "--- мн-во рёбер без общих концов, тогда интересны только простые графы.
	Вершина может быть (не) покрыта парсочем.
	Парсоч совершенен, если всех покрывает.
	Максимален, если больше всех по размеру, количество рёбер в нём "--- $\alpha'(G)$.
	Наибольшее по включению можно строить жадно, может не дать $\alpha'(G)$.
	% стр. 17
	Если есть парсоч $M$, то можно определить $M$-чередующийся путь (рёбра то в $M$, то не в $M$).
	Если оба конца $M$-чередующегося не покрыты, то он $M$-дополняющий.
	По такому пути можно инвертировать рёбра, увеличить размер (matching augmentation).

	% стр. 18
	Теорема Бержа: $M$ максимально $\iff$ нет $M$-дополняющих путей.
	$\Ra$ уже проверили.
	$\La$: взяли максимальное $M'$, рассмотрели $M \triangle M'$, в нём степени не более двух,
	т.е. там чётные циклы и пути.
	Но $|M'|>|M|$, значит, есть путь правильной нечётности, он $M$-дополняющий.

	% стр. 19
	Кун в двудольном (за куб от числа вершин): взяли непокрытую вершину, dfs, нашли $M$-дополняющий путь, пока не надоется.
	\TODO

	% стр. 20
	Эдмондс в произвольном: \TODO.

% 5. Независимые множества и покрытия графа. (стр. 21)
\section{} % 04
	$\alpha$ "--- независимое, $\beta$ "--- покрывающее, со штрихом  "--- из рёбер,
	$\omega$ "--- клики.
	Связь: $\alpha+\beta=n$.
	Если нет изолированных, то Галлаи: $\alpha'+\beta'=n$.
	Ещё $\alpha' \le \beta$, а в двудольных по Кёнигу-Эгервари $\alpha'=\beta$.

	Паросочетание "--- это рёберно независимое множество.
	Можно ввести вершинно независимое множество "--- $S$ из вершин, никакие две не смежны.
	Бывает максимальное, его размер "--- число независимости графа $\alpha(G)$ (без штриха).
	Можно жадно набрать наибольшее по включению, поиск максимального NP-труден,
	проверка существования размера $\ge k$ NP-сложна.
	В дополнении $G$ независимое перейдёт в клику (кликовое число $\omega(G)$ "--- размер макс. клики),
	$\omega(G)=\alpha(\bar G)$.

	Вершинное покрытие покрывает все рёбра, минимальное имеет размер $\beta(G)$.
	Теорема: $S$ независимо $\iff$ $V\setminus S$ вершинное покрытие.
	Следствие: $\alpha(G)+\beta(G)=n$ (два неравенства: воспользовались максимальностью $\alpha$
	и минимальностью $\beta$).

	Рёберное покрытие покрывает все вершины (есть только если $\delta>0$, нет изолированных).
	Размер минимального равен $\beta'(G)$.
	% стр. 25
	Лемма: минимальное рёберное покрытие $L$ есть объединение звёзд (дерево у которого все вершины, кроме одной "--- листы).
	Док-во: циклов в $L$ точно нет (можно удалить), а если есть ребро, у которого каждый конец степени $\ge 2$, то ребро можно удалить.
	Повторяем, в конце получаем звезду (от противного будет путь из трёх рёбер, упс).
	% стр. 26
	Дополнение независимого рёберного не есть покрытие, но теорема Галлаи: если в графе $\delta>0$, то $\alpha'+\beta'=n$ (sic, число вершин, не рёбер).
	Доказываем два неравенства.
	Берём макспарсоч $M$, он не покрыл $n-2\alpha'$ вершин (мн-во $U$).
	$U$ независимо, можно выбрать по ребру из вершины, покрыть вершины $n-\alpha'$ рёбрами, т.о. показали $n-\alpha'\ge\beta' \iff n\ge\alpha'+\beta'$.
	Обратное: берём минимальное покрытие $L$, каждая компонента $L$ "--- звезда из хотя бы двух вершин.
	Не-центров звёзд столько, сколько рёбер, т.е. $\beta'$, а центров (и звёзд) "--- $n-\beta'$.
	Найдём в $L$ макспарсоч $M$, он размера ровно $n-\beta'$ (в каждой компоненте можно выбрать ровно одно ребро).
	Т.о. $\alpha' \ge n - \beta' \iff \alpha'+\beta' \ge n$, что и требовалось.

	% стр. 27-28
	Очевидно, что $\alpha' \le \beta$: на каждое ребро парсоча нужна хотя бы одна вершина, так можно доказывать максимальность $\alpha'$ (но не получится в $C_5$).
	В двудольном графе по Кёнигу-Эгервари $\alpha'=\beta$.
	Добавляем исток/сток, тогда $S$ их разделяет $\iff$ покрывает все рёбра исходного (т.к. сток/исток нельзя брать в $S$).
	А мн-во непересекающихся путей между $x$ и $y$ образует парсоч в исходном графе.
	Но по Менгеру это равно минимальному размеру $S$, что и требовалось.
	Переформулировка теоремы в матричном виде "--- представили $A$ как матрицу $n \times m$ смежности двудольного графа на $n+m$ вершинах,

% 3. k-связные графы. Теорема Менгера. (стр. 10)
\section{} % 02
	Уитни-1: двусвязен $\iff$ есть два пути.
	Гёринг: минимальное отделяющее $X$ от $Y$ равно кол-ву путей.
	Менгер: то же самое для вершин.
	Уитни-2: $k$ связен $\iff$ есть $k$ путей.

	% стр. 10
	Теорема Уитни: граф на $\ge 3$ вершинах двусвязен $\iff$ для между любой парой вершин
	есть два непересекающихся по внутренности пути (это же цикл).
	Определение: если $X, Y \subseteq V$, то путь между $X$ и $Y$ "--- любой простой,
	начинается в $X$, кончается в $Y$, по $X \cup Y$ в середине не идёт.
	Допускается путь из единственной вершины из $X \cap Y$.
	% стр. 11
	$R$ отделяет $X$ от $Y$ (вершинно отделяющее мн-во), если любой путь из $X$ в $Y$
	цепляет какой-то вершиной $R$ (в т.ч. конечной).
	Например, $R=X$ подходит.
	% стр. 12
	Теорема Гёринга: кол-во вершин $k$ в минимальном отделяющем $X$ от $Y$ равно макс. числу $l$ попарно
	непересекающихся (в т.ч. по концам) путей из $X$ в $Y$.
	Очевидно $l \le k$ (так как путь проходит через $R$), покажем $k \le l$.
	При $k=0$ очевидно, при $k=1$ очевидно т.к. есть путь.
	При $k\ge 2$ индукция по размеру графа ($n \ge 2$), база $\overline{K_2}$ очевидна (т.к. тогда $|X|=|Y|=2$).
	Если есть $z\in X \cap Y$, то удалили, индукция, добавили тривиальный путь $\{z\}$.

	Если $X \cap Y =\varnothing$, то взяли ребро $e$ на пути между ними ($k \ge 1$), его начало $\notin Y$, конец $\notin X$.
	Удалили $e$ (вершины оставили), если $k$ не изменилось, то успех.
	% стр. 13
	Если изменилось, то хотим найти какое-нибудь отделяющее $R$, причём не хотим $R \supseteq X, Y$.
	Для этого возьмём отделяющее в графе $G-e$, назовём $R'$.
	$R' \cup \{e\}$ отделяющее в $G$, значит, $R_x=R'\cup\{x\}$ и $R_y$ тоже отделяющие.
	Одно из них не содержит целиком либо $X$, либо $Y$, назовём его $R$.
	Тогда если $R$ не содержит целиком $X$, удалим $\bar X=R\setminus X$, воспользуемся индукцией для $R$ и $X$,
	найдём $k$ непересекающихся путей из $X$ в $R$.
	Аналогично для $Y$.
	Потом состыкуем (осторожно с пересечениями).

	% стр. 14
	Теорема Менгера: количество $\kappa(x, y)$ вершин в минимальном отделяющем несмежные вершины $x$ и $y$ (нельзя брать $x$ и $y$ в мн-во)
	равно макс. числу попарно непересекающихся простых путей из $x$ в $y$.
	Док-во: $X$ "--- соседи $x$, $Y$ "--- соседи $y$, тогда разделяет $x$/$y$ $\iff$ разделяет $X$ и $Y$,
	т.к. путь $X \to Y$ есть обрезанный путь $x \to y$, тогда нашли $\kappa(x, y)$ путей по Гёрингу.

	Теорема Уитни: граф $k$-связен $\iff$ между любыми двумя вершинами есть $k$ путей без общих внутренних вершин.
	$\La$: вершин $\ge k+1$, удалить мало вершин для потери связности нельзя, упс.
	$\Ra$: сначала применили Менгера, теперь $x$ и $y$ соединены ребром $e$.
	Тогда если в $G-e$ путей между $x$ и $y$ хотя бы $k-1$, то успех.
	Иначе их не более $k-2$, тогда можно найти разделяющее мн-во $R$ размером $k-2$ (оно не разделяющее в $G$, иначе упс).
	Тогда оно отделяет некую $z$ либо от $x$, либо от $y$, но тогда можно добавить в $R$ либо $y$, либо $x$, и получим разделяющее в $G$ размера $k-1$, упс.

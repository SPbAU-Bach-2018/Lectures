% 9. Реберная раскраска графов (стр. 61)
\section{} % 11
	Кёниг: в двудольном $\chi'=\Delta$.
	Визинг: в любом графе $\Delta \le \chi' \le \Delta+1$.

	Тут уже графы без петель, но с мультирёбрами.
	Рёберное хроматическое число $\chi'(G)$ "--- минимальное кол-во цветов, в которые можно покрасить рёбра,
	никакие две смежных не покрашены в один цвет.
	Задача: провести турнир <<каждый с каждым>> в минимальное число дней.
	Рёбра одного цвета "--- парсоч, поэтому $\alpha'(G)\chi'(G) \ge |E(G)|$.
	Также рёбра из одной вершины разных цветов, поэтому $\chi' \ge \Delta$.
	% стр. 62
	Кёниг: в двудольном графе $\chi' = \Delta$.
	Достроим до $\Delta$-регулярного: сначала добавим вершин в долю, потом берём вершину степени $<\Delta$,
	обязательно найдём ей пару в другой доле, соединим, повторить.
	А в $\Delta$-регулярном есть парсоч (по Холлу), его можно покрасить и перейти к $\Delta-1$.

	% стр. 63
	Теорема Визинга: в простом графе $\Delta \le \chi' \le \Delta + 1$.
	Док-во от противного: нашли в $G$ подграф $H$ с максимальным числом рёбер, который можно окрасить в $\Delta+1$ цвет.
	В каждой вершине есть хотя бы один отсутствуюий цвет, т.к. цветов строго больше степени.
	Взяли ребро $x\to y_1$ не из $H$.
	Пусть в $x$ отсутствует $c$.
	Если он отсутствует в $y_1$, то покрасили ребро, упс.
	Пусть в $y_1$ отсутствует $c_1$.
	Если его нет и в $x$, то покрасили ребро, упс.
	Значит, $c_1$ есть в $x$, и это ребро ведёт в $y_2$.
	Если в $y_2$ отсутствует $c$, то покрасили сдвигом, упс.
	Пусть в $y_2$ отсутствует $c_2$, если его нет и в $x$, то упс (перекрасили сдвигом), и так далее.

	% стр. 64
	Неуспех, если в какой-то момент зациклились после вершины $y_k$: $c_j=c_k$, при этом все $c_i$ встречаются в $x$, и во всех $y_i$ встречается $c$.
	Тогда сдвинем цвета: покрасим $x \to y_1, \dots, y_{j}$ сдвигом, уберём цвет у ребра $x \to y_{j+1}$.
	Теперь построим граф $S$, оставив только рёбра цветов $c$ и $c_j$.
	В $S$ компоненты связности "--- чётные циклы и пути.
	Также в $S$ степень $x$ не больше единицы (т.к. в ней нет $c$), степень $y_k$ не больше единицы (т.к. в ней нет $c_k=c_j$),
	степень $y_{j+1}$ не больше единицы (т.к. раньше в неё вело ребро $c_j$).
	Значит, они втроем не могут быть в одной компоненте.

	Если $x$ и $y_{j+1}$ в разных компонентах, то поменяем цвета в компоненте с $y_{j+1}$, покрасим ребро $x\to y_{j+1}$ в цвет $c$.
	А если они лежат в одной, то поменяем цвета в компоненте $y_k$, сдвинем цвета рёбер, покрасим $x \to y_k$ в цвет $c$.

	Можно обобщить до мультирёбер (без док-ва): в произвольном графе $\chi' \le \Delta + \mu$ (где $\mu$ "--- максимальное по $\mu(x, y)$ "--- кол-во параллельных рёбер между вершинами).
	Следствие "--- $\chi' \le \sfrac 32 \cdot \Delta$ (без док-ва).

% 10. Хроматический многочлен графа (стр. 65)
\section{} % 12
	Хроматический многочлен $P_G(z)$ "--- функция, причём для целого $k \ge 0$ значение $P_G(k)$ есть
	кол-во способов покрасить $G$ в $k$ цветов.
	Для $K_n$ "--- убывающая степень, для пустого "--- $z^n$, для дерева "--- $z(z-1)^{n-1}$.
	% стр. 66
	Лемма: $P_{G-e}(z) = P_G(z) + P_{G \setminus e}(z)$, т.к. вершины-концы $e$ либо разного цвета, либо одного.
	Теорема: для любого графа $G$ на $n$ вершинах $P_G(z)$ "--- это полином степени $n$ из $\Z[z]$,
	причём коэфф. при $z^n$ всегда единица, а остальные "--- ненули с чередующимся знаком.
	Индукция по числу вершин/рёбер (лексикографически), для $\bar K_n$ всё верно.
	Переход: лемма, написали общий вид через $a_i \in \Z_{+}$, сложили, проверили знаки.

	% стр. 67
	Следствие доказательства: коэффициент при $z^{n-1}$ равен $-E(G)$.
	По правилу произведения можем перемножать хроматические полиномы для компонент связности.
	Теорема: ноль является корнем хроматического полинома кратности ровно $k$, где $k$ "--- число компонент связности.
	Достаточно показать, что для любого связного графа это корень кратности ровно один.
	Без док-ва.

	% стр. 68
	Вершина симплициарна, если её соседи образуют клику.
	Нумерация в порядке исключения $x_n, \dots, x_1$, если $x_i$ симплициарна для подграфа из всех оставшихся.
	Если граф допускает такую нумерацию, то можно красить вершины в порядке $x_1, \dots, x_n$, у каждой покрашенные
	соседи образуют клику, поэтому остаётся $z - \tilde{d_i}$ цветов, получаем полином как произведение линейных.
	% стр. 69
	Теорема Дирака: граф допускает нумерацию $\iff$ он хордален (отсутствуют индуцированные циклы длины больше трёх).
	$\Ra$ очевидно, так как мы в какой-то момент удалим вершину цикла, её соседи в клике, получили хорду, упс.
	$\La$: подграф хордального хордален, поэтому достаточно показать, что в хордальном есть симплициарная.
	Верно даже более сильное: для любой $x$ среди всех $y_i$, находящихся на макс. расстоянии от $x$, найдётся симплициарная.
	Индукция по числу вершин, база $n=1$ очевидна.

	Переход: если $x$ смежна со всеми остальными, то они от неё на расстоянии 1, можно найти симплициарную в $G-x$,
	в $G$ она тоже симплициарна ($x$ ничего не испортит), успех.
	% стр. 70
	Если же это не так, то рассмотрим мн-во $T$ вершин, максимально удалённых от $x$, возьмём одну из компонент
	связности подграфа $T$, назовём $H$.
	А смежные с ними вершины не из $T$ назовём $S$ (предыдущий слой).
	Покажем, что $S$ "--- клика, взяв любые две вершины $u$ и $v$ оттуда, найдя кратчайший путь между
	ними в графе $G-T-S$ (такой есть, так как можно просто спуститься до $x$), он длины хотя бы два.
	Потом добавляем к нему кратчайший путь между $u$ и $v$ через $T$, получили цикл длины $\ge 4$.
	Хорд в каждой половине цикла нет, между половинами тоже быть не может (далёкие слои bfs'а),
	значит, только между $u$ и $v$.
	Теперь смотрим только на подграф $S \cup H \eqcolon G'$, берём $u \in S$, по индукции ищём дальнейшую от неё симплициарную $y$.
	Если $G'$ не клика, то $y \in H$ и тогда $y$ симплициарна в $G$.
	А если $G'$ клика, то берём любую $y \in H$, она тоже симплициарна в $G$.

	Также известно, что хордальные графы совершенны (без док-ва).

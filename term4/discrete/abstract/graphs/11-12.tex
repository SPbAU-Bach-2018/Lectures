% 9. Реберная раскраска графов (стр. 61)
\section{} % 11
	Тут уже графы без петель, но с мультирёбрами.
	Рёберное хроматическое число $\chi'(G)$ "--- минимальное кол-во цветов, в которые можно покрасить рёбра,
	никакие две смежных не покрашены в один цвет.
	Задача: провести турнир <<каждый с каждым>> в минимальное число дней.
	Рёбра одного цвета "--- парсоч, поэтому $\alpha'(G)\chi'(G) \ge |E(G)|$.
	Также рёбра из одной вершины разных цветов, поэтому $\chi' \ge \Delta$.
	% стр. 62
	Кёниг: в двудольном графе $\chi' = \Delta$.
	Достроим до $\Delta$-регулярного: сначала добавим вершин в долю, потом берём вершину степени $<\Delta$,
	обязательно найдём ей пару в другой доле, соединим, повторить.
	А в $\Delta$-регулярном есть парсоч (по Холлу), его можно покрасить и перейти к $\Delta-1$.

	% стр. 63
	Теорема Визинга: в простом графе $\Delta \le \chi' \le \Delta + 1$.
	Док-во от противного: нашли в $G$ подграф $H$ с максимальным числом рёбер, который можно окрасить в $\Delta+1$ цвет.
	В каждой вершине есть хотя бы один отсутствуюий цвет, т.к. цветов строго больше степени.
	Взяли ребро $x\to y_1$ не из $H$.
	Пусть в $x$ отсутствует $c$.
	Если он отсутствует в $y_1$, то покрасили ребро, упс.
	Пусть в $y_1$ отсутствует $c_1$.
	Если его нет и в $x$, то покрасили ребро, упс.
	Значит, $c_1$ есть в $x$, и это ребро ведёт в $y_2$.
	Если в $y_2$ отсутствует $c$, то покрасили сдвигом, упс.
	Пусть в $y_2$ отсутствует $c_2$, если его нет и в $x$, то упс (перекрасили сдвигом), и так далее.

	% стр. 64
	Неуспех, если в какой-то момент зациклились после вершины $y_k$: $c_j=c_k$, при этом все $c_i$ встречаются в $x$, и во всех $y_i$ встречается $c$.
	Тогда сдвинем цвета: покрасим $x \to y_1, \dots, y_{j}$ сдвигом, уберём цвет у ребра $x \to y_{j+1}$.
	Теперь построим граф $S$, оставив только рёбра цветов $c$ и $c_j$.
	В $S$ компоненты связности "--- чётные циклы и пути.
	Также в $S$ степень $x$ не больше единицы (т.к. в ней нет $c$), степень $y_k$ не больше единицы (т.к. в ней нет $c_k=c_j$),
	степень $y_{j+1}$ не больше единицы (т.к. раньше в неё вело ребро $c_j$).
	Значит, они втроем не могут быть в одной компоненте.

	Если $x$ и $y_{j+1}$ в разных компонентах, то поменяем цвета в компоненте с $y_{j+1}$, покрасим ребро $x\to y_{j+1}$ в цвет $c$.
	А если они лежат в одной, то поменяем цвета в компоненте $y_k$, сдвинем цвета рёбер, покрасим $x \to y_k$ в цвет $c$.

	Можно обобщить до мультирёбер (без док-ва): в произвольном графе $\chi' \le \Delta + \mu$ (где $\mu$ "--- максимальное по $\mu(x, y)$ "--- кол-во параллельных рёбер между вершинами).
	Следствие "--- $\chi' \le \sfrac 32 \cdot \Delta$ (\TODO док-во?).

% 10. Хроматический многочлен графа (стр. 65)
\section{} % 12
	

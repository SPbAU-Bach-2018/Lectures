% 8. Нижние оценки на хроматическое число. Теорема Турана. Совершенные графы (стр. 52)
\section{} % 09
	Клику надо красить в разные цвета, поэтому $\omega(G) \le \chi(G)$.
	Но есть конструктивный Мицельский: для любого $k$ можно построить граф без треугольников с $\chi=k$.
	Начинаем с чего угодно (проще с $K_2$), потом делаем для увеличения $k$ так: для каждой $x_i$ создали $y_i$, соединили новую со всеми соседями $x_i$,
	а потом создали $z$ и подцепили ко всем $y_i$.
	% стр. 53
	Док-во по индукции: треугольников нет ($y_i$ несмежны, и соседи $x_i$/$y_i$ тоже несмежны),
	окрасить в $k+1$ можно ($y_i$ "--- так же, как и $x_i$, а $z$ в новый),
	а вот в $k$ нельзя (окрасим $z$ в цвет $k$, все $y_i$ окрасились в $k-1$, но тогда так же можно окрасить $x_i$ \TODO\textit{неправда, см. вопрос 11 в консультации}, упс).
	Теорема Эрдёша (нужна ли? без док-ва \TODO): для любого $k$ есть $k$-хроматический с обхватом $\ge k$ (обхват "--- минимальная длина цикла).
	Это более круто, чем Мицельский, конструктивного док-ва у Эрдёша не было, потом сделали.

	Граф $k$-долен, если можно разбить вершины на доли (покрасить в $k$ цветов).
	Бывает полный $k$-дольный граф, его дополнение "--- $k$ клик.
	Макс. число рёбер в $k$-дольном графе (при фикс. числе вершин) достигается в полном графе, где все доли почти равны (разница не больше единицы,
	такое разбиение $n$ единственно), это числе обозначим за $M(n, k)$, надо доказать (если есть разница больше единицы "--- передвинем вершину между долями).
	Считается так: число рёбер в полном графе минус число рёбер в дополнении (сколько-то клик одного размера, сколько-то клик другого размера),
	в формуле будет $n \bmod k$, проверить $M(3, 2)=2$.

	% стр. 55
	Туран: если в графе на $n=tk+r$ вершинах больше $M(n, k)$ рёбер, то есть клика размера $k+1$ ($\Ra \chi>k$).
	Возьмём граф $G$ на $n$ вершинах без больших клик с макс. числом рёбер.
	Покажем индукцией по $t$ (ага, по частному при фикс. остатке), что в графе не более $M(n,k)$ рёбер.
	База тривиальна: у нас $n=r$, т.е. граф $K_n$.
	Переход: сейчас в $G$ клики $K_{k+1}$ нет, но при добавлении любого ребра появляется (т.к. число рёбер максимально).
	Значит, сейчас есть клика $S$ размера $k$, в неё $\binom{k}{2}$ рёбер, а все остальные вершины смежны с не более чем $k-1$ вершиной из $S$ (иначе нашли клику $k+1$).
	При этом в $G-S$ тоже нет клики размера $k+1$, т.е. в нём не более $M(n-k,k)$ рёбер, складываем число рёбер в клике, в $G-S$ и в разрезе $[S;G-S]$, получили в точности $M(n,k)$.

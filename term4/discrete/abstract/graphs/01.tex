% Связность в графах (стр. 1)
% 1. Вершинная и реберная связность графа (стр. 1)
% 2. Двусвязные графы (стр. 4)
\section{} % 01
	% стр. 1
	Для обеих связностей:
	несвязный/пустой граф 0-связен;
	если граф $k$-связен, то и $k-1, \dots, 0$-связен.
	По умолчанию будет вершинная связность.

	Граф рёберно $k$-связен, если можно удалить строго меньше $k$ рёбер и он останется связным.
	Рёберная связность $\lambda(G)$ "--- такое максимальное $k$, оно же
	размер минимального рёберного разреза (есть разрез, если при удалении граф распался на компоненты связности).
	При $|V|=1$ $\lambda(G)$ неопределено "--- такой граф сколь угодно связен.

	% стр. 2
	Подмножество вершин $S$ называется вершинным разделяющим множеством/вершинным разрезом,
	если $G-S$ несвязен.
	Вершинная связность $\kappa(G)$ "--- минимальный размер $|S|$ такой, что либо это разрез,
	либо $|V\setminus S|=1$ (чтобы положить $\kappa(K_n)=n-1$, $K_5=4$).
	Осторожно: связный $K_1$ имеет $\kappa=0$, т.е. он не 1-связен.
	Альтернативное определение: граф $k$-связен (вершинно, по умолчанию), если $|V|\ge k+1$
	и при удалении менее чем $k$ вершин остаётся связен;
	тогда $\kappa$ "--- максимальное $k$, для которого граф $k$-связен.

	% стр. 3
	$\delta(G) = \min \deg v$, $\Delta(G) = \max\deg v$.
	Тогда $\lambda(G)\le\delta(G)$ (удалили смежные с вершиной).
	И ещё $\kappa(G)\le \lambda(G)$; два случая: если в какой-то компоненте есть вершина,
	не инцидентная с разрезом, убили разрез.
	Если нет (пример "--- конвертик), то выберем любое ребро разреза $x \to y$,
	удалим остальные за $\lambda-1$ вершину, если ещё в какой-то компоненте вершины остались "--- удаляем $x$ или $y$, добивая разрез,
	если нет "--- то мы удалил все вершины, кроме двух, удаляем ещё одну, $|V|=1$, успех.

	% стр. 4-5
	Рассмотрим графы с $\kappa=1$, два ребра похожи, если равны или лежат на одном простом цикле,
	это эквивалентность (доказать транзитивность), рёбра разбились на блоки (каждый блок "--- либо ребро, либо несколько циклов).
	Теорема: $G$ двусвязен $\iff$ любые два ребра похожи $\iff$ через любые две вершины есть простой цикл.
	Если двусвязен, то любые два смежных ребра похожи (можно удалить общую вершину и найти путь), пользуемся транзитивностью и связностью.
	Если есть две вершины, то можно в каждой выбрать по ребру.
	Если через любые две есть цикл, то нет точек сочленения: пусть есть, удалили,
	получили компоненты, но раз нет пути $x \to y$, то не было цикла через $x$, $y$, упс.
	% стр. 6
	Теперь рёбра разбиты на блоки, можно построить двудольный $B(G)$ для блоков и точек сочленения (соседствует с разными блоками $\iff$ точно точка, доказать,
	отсюда сделать вывод <<точки "--- границы блоков>>), $B(G)$ ацикличен (нашли цикл через разные блоки).
	Листы "--- только блоки, <<крайние блоки>>, не точки.
	% стр. 7
	Алгоритм Хопрофта-Тарьяна для поиска: обошли DFS-ом, посчитали для каждого поддерева, насколько рёбра прыгают вверх, не забыли про корень.

	% стр. 8
	Если $H$ "--- подграф $G$, то простой путь (по рёбрам) $P$ "--- его ручка $\iff$ концы $P$ лежат в $H$, а внутренние "--- нет.
	Разложение на ручки: $P_0$ (цикл), $P_1, P_2, \dots$ (ручки для объединения предыдущих).
	% стр. 9
	Разложим на ручки $\Ra$ двусвязен, т.к. цикл двусвязен, добавление/подразбиение ребра её не нарушает.
	Если двусвязен, то можно на ручки: взять цикл, потом итерация: найти ребро не в разбиении, найти цикл через него и разбиение, получить ручку.

	Рёберная двусвязность: не должно быть мостов (но недостаточно).
	Если вершнно двусвязен, то и рёберно двусвязен (не в обратную, <<бабочка>>).
	Замкнутая ручка "--- простой цикл, имеющий со старым графом лишь одну общую точку.
	Рёберно двусвязен $\iff$ можно разложить на цикл, ручки, замкнутые ручки (без доказательства \TODO).
	Можно ещё вводить для орграфов, там будем смотреть на сильную связность.

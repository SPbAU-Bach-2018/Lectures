% 6. Совершенные паросочетания в произвольном графе. Теорема Татта, формула Татта-Бержа. (стр. 29)
\section{} % 05
	Теорема Татта: в произвольном $G$ есть совершенный парсоч $\iff$ для любого $S \subseteq V$
	количество нечётных компонент в $G-S$ ($\codd(G-S)$) не больше $|S|$.
	Необходимость очевидна (посмотрели на рёбра парсоча, выходящие из компонент $G-S$).
	Достаточность: у нас граф точно на чётном числе вершин и не равен $K_n$ (иначе неинтересно).
	Если совершенного нет, то берём несмежные вершины, соединяем, если оно всё ещё не появится, так насытим граф.
	Условие при этом не нарушится.
	Найдём в насыщенном $G$ мн-во $U$ вершин степени $n-1$, в лемме покажем, что $G-U$ есть объединение несвязных полных графов.
	Тогда в чётных компонентах можно найти парсоч, в нечётных "--- почти парсоч, лишние вершины подцепить к $U$, а остаток из $U$ между собой ($U$ же со всем подряд связно).
	Нашли совершенный парсоч, противоречие с насыщенностью, упс.

	% стр. 32
	Лемма, от противного: пусть есть левая компонента, тогда в ней $\ge 3$ вершин (иначе она $K_n$),
	найдём индуцированный путь $a \to b \to c$, найдём несмежную с $b$ вообще во всём графе ($b \notin U$), она "--- $d \notin U$.
	Значит, добавление и $e_1=\{a, c\}$, и $e_2=\{b, d\}$ создаёт совершенные парсочи (по насыщенности), рассмотрели их симм. разность, это циклы.
	Если $e_1$ и $e_2$ на разных циклах, то циклы можно инвертировать, получить парсоч в исходном, упс.
	Если на одном, то в нём есть хорды $b \to a, c$, можно отрезать одной из них кусок цикла и инвертировать кусок с хордой, упс.

% 6.2 (стр. 34)
\section{} % 06
	Лемма: для любого $S$ верно: $\codd(G-S)-|S|\equiv n \mod 2$.
	Док-во: убрали чётные компоненты, $\codd(G-S)+|S|\equiv n$, заменили знак ($\bmod 2$), успех.
	Теорема Петерсена: связный кубический (3-регулярный) граф $G$ с не более чем двумя мостами имееет совершенный парсоч.
	% стр. 35
	Док-во от противного по Татту: если не имеет, то есть $S$, причём из Татта $\codd(G-S) - |S|\ge 1$.
	Т.к. $n$ чётно, то справа на самом деле двойка.
	Рассмотрим нечётную компоненту графа $G-S$, тогда из неё торчит нечётно рёбер (сумма степеней).
	Если $m_i=1$, то имеем мост $\Ra$ не более двух таких $m_i$ (по условию теоремы).
	Суммируем $m_i$ по нечётным компонентам, получаем $\ge 3(\codd(G-S)-2)+2$, используем оценку на $\codd$,
	получаем $\ge 3|S|+2$, что-то много рёбер входит в $S$, упс.
	Контрпример с тремя мостами: три почти-$K_5$ (домики), соединены крышами с вершиной.

	Без док-ва теорема Плесника (без док-ва): если $G$ $k$-регулярен и $(k-1)$-рёберносвязен с чётным числом вершин,
	то можно удалить любое $k-1$ ребро и всё равно найти совершенный парсоч.
	Матрица Татта: квадратная $n \times n$, на диагонали нули, над ней "--- $x_{ij}$ (ноль, если ребра нет, не-ноль иначе),
	под ней "--- с другими знаками (антисимметрично).
	Можно рассмотреть определитель, для нечётных $n$ он ноль (так как должен поменять знак для минус матрицы).
	А для чётных он равен $P^2$, где $P$ "--- полином, пфаффиан матрицы (тут и далее "--- без доказательств).
	Определение: рассмотрели все разбиения $n$ на пары, отсортировали их как можно разумнее,
	сконкатенировали, получили перестановку из $S_n$ (у которой первый элемент всегда единица).
	Пфаффиан "--- сумма произведений соответствующих $t_{x_i, x_{i+1}}$ со знаком перестановки.
	Тогда это какая-то сумма по паросочетаниям.
	Можно так проверять наличие совершенного: подставили случайные $t_{ij}$, посчитали пфаффиан (или определитель),
	порадовались с хорошей вероятностью.
	А ещё некоторые графы пфаффовы "--- допускают такую ориентацию, что если поменять знаки в матрице Татта,
	то в пфаффиане все слагаемые с одним знаком, так можно считать число паросочетаний.
	Планарные, внезапно, пфаффовы.

% 6.4 (стр. 39)
\section{} % 07
	Дефицит графа "--- $\defic(G) = n-2\alpha'(G)$ (сколько вершин не покрыли максимальным).
	Теорема Бержа: $n-2\alpha'(G)=\codd(G-S) - |S|$ (превращается в Татта для совершенного парсоча),
	так можно показывать максимальность паросочетания.
	% стр. 40
	Сначала покажем, что для любого $S$: $n-2\alpha' \ge m \coloneq \codd(G-S)-|S| \iff 2\alpha' \le n-m$.
	Это просто: рассмотрим макспарсоч, он не покрыл $x$ нечётных компонент, тогда из оставшихся рёбра пошли в $S$.
	Второе нер-во: добавим к графу $m$ вершин, соединим их со всеми вообще, проверим условие Татта для нового графа, оно есть:
	% стр. 41
	отдельно пустое $S$ (была лемма про чётность $\codd(G-S)-|S|$, значит, $m \equiv n$),
	отдельно не удаляющее все новые (останется одна компонента),
	отдельно удаляющее все новые (удалили-то хотя бы $m$ $\Ra$ разность $\le 0$).
	% стр. 42
	Найдём в нём макспарсоч (размера $\sfrac{(n+m)}{2}$), выкинем добавленные вершины, осталось хотя бы $\sfrac{(n-m)}{2}$ рёбер (это парсоч в исходном),
	отсюда $2\alpha' \ge n-m$.

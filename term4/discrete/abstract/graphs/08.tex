% Раскраска графов (стр. 43)
% 7. k-раскрашиваемые графы. Теорема Брукса (стр. 43).
\section{} % 08
	Жадник: красит в $\Delta+1$ цвет по порядку, Брукс "--- можно даже в $\Delta$ (если не полный и не нечётный цикл).

	Раскраска графа (по умолчанию "--- вершин) правильная, если смежные покрашены в разное.
	Граф бывает $k$-раскрашиваем, минимальное $k$ "--- его хроматическое число $\chi(G)$.
	Тут смотрим только на простые графы (если петля "--- сразу нельзя покрасить, если мультиребро "--- можно убрать),
	еще можно смотреть отдельно на компоненты.
	Случай $\chi(G)=1$ "--- пустой граф, $\chi(G)=2$ "--- двудольный (без нечётных циклов).
	Классическая задача: расписание экзаменов, два экзамена нельзя в один день, если есть пересечение по студентам.
	Ещё одна: какие-то химикаты нельзя хранить в одной комнате.
	Проверка на 3,4-раскрашиваемость (и выше) NP-полна, а поиск $\chi$ "--- и подавно.

	% стр. 45
	Жадный алгоритм: взяли порядок вершин, красим слева направо в минимальный цвет.
	Покрасит в не более $\Delta+1$ цвет.
	Значит, $\chi\le\Delta+1$ и в любом $k$-хроматическом есть вершина степени хотя бы $k-1$.
	В двудольных графах вообще не достигается, а вот для $K_n$ и $C_{2k+1}$ достигается.
	Теорема Брукса: если простой связный граф не полон и не нечётный цикл, то $\chi \le \Delta$.
	Случаи $\Delta \le 2$ разобрали отдельно ($K_1$, $K_2$, циклы).
	% стр. 47
	Если граф не регулярен, то найдём вершину степени $<\Delta$, запустим из неё dfs, в таком порядке запустим жадник
	(у каждой покрашены только дети, а для всех, кроме корня, есть ещё родитель, успех).
	Если регулярен, но есть точка сочленения: по ней пилим граф на блоки, каждый красим, потом перекрашиваем, чтобы цвет точки сошёлся.
	% стр. 49
	Если нет точек, то двусвязен.
	Ниже покажем, что тогда есть $x$, смежные с ней (но не между собой) $y$ и $z$, причём $G-y-z$ связен (чтобы по-прежнему был один корень дерева).
	Тогда расположим их в начале списка (покрасятся в один цвет), а остальные "--- dfs'ом из $x$.
	Все, кроме $x$, жадно покрасятся в $\le \Delta$, а $x$ "--- тоже, так как два соседа имеют один цвет.

	Лемма: в вершинно двусвязном регулярном ($\Delta \ge 3$) есть индуцированный путь $y \to x \to z$, причём $G-y-z$ связен.
	Выберем любую вершину $w$.
	Если точек сочленения в $G-w$ нет, то найдём в $G$ вершину на расстоянии 2 от $w$ (если нет, то $G$ был полон), нашли нужную тройку.
	А если есть, то посмотрим на крайние блоки $B_1$ и $B_2$ (листья в дереве блоков), в каждом ровно одна точка сочленения: $v_1$ и $v_2$
	(не упростить до одной общей точки сочленения $v_1=v_2$, т.к. бывает бамбук из блоков).
	Найдём в $B_1-v_1$ и в $B_2-v_2$ смежные с $w$ (есть, т.к. исходный двусвязен, удалить можно любую вершину; а блоки крайние "--- фиг из них куда-то выйдем).
	Они несмежны, т.к. иначе блоки схлопнутся.
	При этом $G-v_1-v_2$ связен, т.к. степень $w$ хотя бы 3, блоки были двусвязные, значит, можно из каждого безопасно удалить по вершине,
	а $w$ потом подцепить.

	% стр. 50
	Граф $k$-критический, если $\chi=k$ и любой собственный подграф можно раскрасить в $k-1$.
	В любом графе есть $\chi(G)$-критический подграф (удаляем рёбра/вершины пока не надоест).
	Критический связен, т.к. можно удалить менее сложную компоненту.
	В критическом нет вершинно разделяющего $S$, являющееся кликой.
	От противного: пусть есть, тогда $G-S$ развалился на компоненты, можно к каждой добавить $S$ (это т.н. $S$-компоненты), покрасить в $k-1$ цвет (по критичности).
	Т.к. $S$ клика, то все её вершины в разные цвета в разных компонентах, можно перенумеровать цвета и согласовать, упс.
	Следствие: точек сочленения нет, иначе есть $S=K_1$.
	Следствие: критический двусвязен.

	% стр. 51
	Теорема Дирака: если в $k$-критическом есть вершинно разделяющее мн-во $\{x, y\}$, то
	$G$ есть объединение двух $S$-компонент, у одной $x$ и $y$ всегда разных цветов, у другой "--- всегда одного.
	Следует из того, что у хотя бы одной $S$-компоненты не должно быть раскраски, при которой все вершины $S$ разного цвета
	(иначе сможем все компоненты перекрасить и согласовать).
	Аналогично, хотя бы у одной не должно быть раскраски, при которой все вершины $S$ одного цвета.

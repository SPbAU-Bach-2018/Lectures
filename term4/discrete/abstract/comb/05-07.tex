\section{} % 05
	Композиция экспоненциальных $e^{F(z)}$ (ввели по определению как сумму ряда) "---
	это взяли $n$ различных предметов, разложили по скольки-то неразличимым ящикам, над каждым
	ящиком $a_{n_i}$ действий ($a_0 = 0$), это \textit{экспоненциальная формула}.
	Например, $e^{e^z-1}=B(z)$ "--- числа Белла (разбили $n$ элементов на неупорядоченные непустые блоки).
	Вывод рекурренты из композиции: продифференцировали $e^{e^z-1}$, получили формулы
	для $c_n$ по $a_n$ и предыдущим $c_n$.
	Иногда надо наоборот, вычислять $a_n$ (ещё одна формула).
	Так можем посчитать число связных графов (замкнутая формула для ПФ для всех графов не нужна, кстати).

	Композиционная формула $G(F(z))$ "--- сначала разложили различимые по неразличимым ящикам,
	действия над ящиком, а потом ещё $b_k$ действий со всеми ящиками.
	А если хотим различимые ящики, то надо $G=\frac{1}{1-z}$ (ЭПФ),
	т.к. мы сначала ящики неупорядочили, а потом переставили $n!$ способами.
	Если переставляем лишь циклически, вылезет логарифм: $1-\ln(1-F(z))$.

\section{} % 06
	Хотим рекурренты (адские) для вычисления коэффициентов $G(F(z))$.
	Играемся с примерами вроде <<разбить людей на пары, в каждой паре $a_2$ действий, а с парами $b_k$ действий>>,
	получаем Фаа ди Бруно: $c_n$ есть сумма во всем разбиениям $n=1k_1+2k_2+\dots+nk_n$,
	какой-то почти мультиномиальный коэффициент на $b_{\sum k_i}$ на произведение $\left(\frac{a_i}{i!}\right)^{k_i}$.
	
	Давайте теперь ещё и отделим $k$, обозначим функцию $B_{n,k}(a_1, \dots, a_n)$ "--- это полиномы
	Белла, целые коэффициенты от $a_i$.
	Теперь формула для $c_n$ выражается через $B_{n,k}$ и простую сумму по всем $k$.
	Если есть рекурренты для Белла, то они никак не зависят от $a_i$, $b_i$, давайте выберем удобные и отсюда найдём $B_{n,k}$.
	Положим $b_k=t^k$ ($t$ "--- метка) и любую $F(z)$, композицию можно вычислить по экспоненциальной формуле.
	Отсюда получили рекурренту для $B_{n,k}$ (сумма по $i=k\dots n$ каких-то $B_{i,k}$ с коэффициентами-переменными).

	Если считать, что $a_i=1$ (кроме $a_0$), то в каждом ящике ничего не делаем,
	тогда $B_{n,k}(1, \dots, 1)$ "--- кол-во способов разложить $n$ различимых по $k$ непустым неразличимым ящикам,
	т.е. число Стирлинга второго рода $S(n, k)$.
	Тогда если подставить в рекурренту, получим рекурренту для Стирлинга
	(там будет сумма по всем разбиениям $n$ на $k$ слагаемых, ничего страшного).

\section{} % 07
	А если надо ящик циклически упорядочивать, то $a_n=(n-1)!$, внутре у ней ($F(z)$) логарифм.
	Если подставить в экспоненциальную, то все сократится, получим $\frac{1}{1-z}$ "---
	это и число перестановок $n$ элементов, и кол-во разбиений на блоки, причём каждый блок циклически упорядочен.
	Это потому что любая перестановка есть произведение циклов.
	Получаем, что числа $n!$ есть коэффициенты, аналогичные $B_n$ (которые вылезают при $a_n=1$).
	Снова подставим $b_k=t^k$, получим кол-во перестановок с $k$ циклами,
	а коэффициенты $c(n,k)$ при $t^k\frac{z^n}{n!}$ "--- числа Стирлинга первого рода.
	Можно для них записать формулу (сумма по всем разбиениям $n$),
	можно записать производящую функцию ($e^{t(-\ln(1-z))}=(1-z)^{-t}$).

	Потом можно взять частную производную по $t$, получить $(1-z)C'_z=tC(z, t)$,
	отсюда рекуррента $c_{n+1}(t) = nc_n(t)+tc_n(t)$.
	А отсюда "--- $c(n+1,k)=nc(n,k)+c(n,k-1)$.
	Можно ещё из производящей можно написать бином Ньютона, тогда коэффициент
	при $\frac{z^n}{n!}$ это сумма $\sum c(n,k)t^k$, оно же "--- разложение восходящей
	факториальной степени $(t)^n$ по степеням $t^k$, это ещё один смысл Стирлинга.

	Можно ввести числа со знаком $s(n, k)$, если в факториальной степени
	подставить $-t$ вместо $t$.
	Тогда $s(n, k)$ получатся коэффициентами в убывающей факториальной степени.
	Ещё можно заметить, что $S(n,k)$ появлялись с условием $t^n=\sum S(n, k)(t)_k$,
	поэтому $s$ и $S$ взаимно обратны (если перемножить как матрицы, то получим либо ноль,
	либо единицу, если $n=k$).
	Трактуем так: есть базис из $t^k$, есть базис из $(t)_k$, числа Стирлинга "--- переходы из одного базиса в другой.

	Если в перестановке нет единичных циклов (но есть остальные), надо просто
	написать $e^{-\ln(1-z)-z} = \frac{e^{-z}}{1-z}$, это как раз формула для числа беспорядков, что хорошо.

	Давайте теперь положим $a_n=(n-1)!x_n$, где $x_i$ "--- переменные.
	Обозначим коэффициент при $\frac{z^n}{n!}$ за $\tilde Z(x_1, \dots, x_n)$ (дополненный цикловой индекс).
	Алгебраически можно подставить в Фаа ди Бруно и действительно обнаружить, что
	это сумма по всем перестановкам с мономами, характеризующими кол-во циклов каждой длины.
	А цикловой индекс $Z(\dots)$ "--- это то же самое, но поделить на $n!$.
	Рекуррентная формула для дополненных индексов "--- посмотрели, в каком цикле лежит элемент $n+1$.

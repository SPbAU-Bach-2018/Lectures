\section{} % 05
	Композиция экспоненциальных $e^{F(z)}$ (ввели по определению как сумму ряда) "---
	это взяли $n$ различных предметов, разложили по скольки-то неразличимым ящикам, над каждым
	ящиком $a_{n_i}$ действий ($a_0 = 0$), это \textit{экспоненциальная формула}.
	Например, $e^{e^z-1}=B(z)$ "--- числа Белла (разбили $n$ элементов на неупорядоченные непустые блоки).
	Вывод рекурренты из композиции: продифференцировали $e^{e^z-1}$, получили формулы
	для $c_n$ по $a_n$ и предыдущим $c_n$.
	Иногда надо наоборот, вычислять $a_n$ (ещё одна формула).
	Так можем посчитать число связных графов (замкнутая формула для ПФ для всех графов не нужна, кстати).

	Композиционная формула $G(F(z))$ "--- сначала разложили различимые по неразличимым ящикам,
	действия над ящиком, а потом ещё $b_k$ действий со всеми ящиками.
	А если хотим различимые ящики, то надо $G=\frac{1}{1-z}$ (ЭПФ),
	т.к. мы сначала ящики неупорядочили, а потом переставили $n!$ способами.
	Если переставляем лишь циклически, вылезет логарифм: $1-\ln(1-F(z))$.

\section{} % 06
	\TODO

\section{} % 07
	\TODO

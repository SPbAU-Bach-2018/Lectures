\section{} % 08
	Построим ЭПФ для числа корневых деревьев $T(z)$, корневых лесов $F(z)=e^{T(z)}$.
	А вот леса на $(n-1)$ вершине похожи на деревья на $n$ вершинах, пишем $T(z)=zF(z) \iff T(z)e^{-T(z)}=z$,
	обозначим $G(z)=ze^{-z}$, надо решить уравнение $G(T(z))=z$.
	Вернёмся к формальным степенным рядам: композиция образует моноид,
	можно ввести обратный в смысле композиции (у ряда с $a_0=0$ есть тогда и только тогда,
	когда $a_1 \neq 0$, единственен).
	Отсюда есть рекуррента для коэф. $T(z)$ через коэф. $G(z)$.

	Можно рассмотреть поле рядов Лорана, вычет "--- как в матане.
	Теорема Лагранжа: если $f(g(z))=g(f(z))=z$, то коэффициент $b_n$ равен
	вычету ряда $1/(nf^n(z))$.
	Док-во: распишем $g(f(z))=z$, дифференцируем по $z$, делим на $f^n(z)$,
	слева остался только $z^{-1}$, т.к. $f^k(z)\cdot f'(z)=\O((f^{k+1})')$ при $k\neq -1$,
	а у производной вычет ноль.

	Для решений уравнений $g(z)=zr(g(z))$, в котором у $g$ есть обратный к композиции,
	а у $r$ свободный коэф. не ноль, можно рассчитать коэф. $g(z)$ по формуле через $[z^{n-1}]r^n(z)$
	\TODO

\section{} % 09
	\TODO

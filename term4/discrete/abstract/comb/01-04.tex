\section{} % 01
	Сложение производящих функций (обычных/эксп.) "--- было два непересекающихся мн-ва
	$X$ и $Y$, на каждом были метки, получили $X \cup Y$.
	
	Произведение экспоненциальных "--- разбили $n$ различимых элементов на два подмножества,
	на одном $a_n$ способов, на другом "--- $b_n$ способов (обобщается до произведения $k$ функций).
	Примеры: разбить студентов на две группы и выбрать в одной группе старосту,
	число беспорядков (перестановка = неподвижные точки + беспорядок),
	раскладка $n$ различимых предметов по $k$ различимым ящикам с ограничениями
	вроде <<в первый только чётное, во второй только нечётное>> (показать формулки,
	в т.ч. без ограничений и для $\ge1$; получим $\hat S(n,k)=\sum (-1)^i\binom{k}{i}(k-i)^n$).

	Произведение обычных "--- разбили $n$ неразличимых предметов на две группы,
	на одном $a_n$ способов, на другом "--- $b_n$ (обобщается до $k$ множителей).
	Примеры: раскладка $n$ неразличимых по $k$ ящикам (в каждом ящике есть ограничение на
	число предметов: 0 или 1, сколько угодно, не-ноль, показать, что формулы на члены правильные).

	Ещё произведение обычных, если мы бьём линейно упорядоченное мн-во на префикс и суффикс.
	Пример: преподаватель пилит семестр на лекции и практики, на лекциях уезжает один раз, на практиках два раза
	($\binom{n+1}{4}$).
	Пример: числа Каталана из путей Дика.
	Отдельно посчитали тривиальный при $n=1$, отдельно нетривиальные.
	В нетривиальном выделили парную скобку, разбили на два блока (внутри скобок и снаружи).

\section{} % 02
	Марки "--- задача о рюкзаке (наклеить марок на 18 рублей такими-то), но порядок важен.
	Тогда рекуррента простая линейная, можно решить.
	Можно умнее: сначала зафиксировали число марок $k$, а потом
	выбрали номиналы у $i$-й "--- это $(f(z))^k$.
	Получили композицию $f(z)$ и функции $\frac{1}{1-z}$ (запрещено $a_0 \neq 0$),
	больше композиции обычных не умеем.
	Смысл: распределяем $n$ неразличимых предметов по $k$ различимым блокам ($k$ заранее неизвестно),
	в каждом блоке $a_{n_i}$ способов (считаем $a_0=0$, т.е. пустые блоки нельзя, иначе странно получится).
	Альтернативно: пилим линейно упорядоченное на куски.

\section{} % 03
	Если порядок неважен, тоже ДП, но уже удаляем типы марок: $\Phi(n; 4, 6, 10)=\Phi(n-10; 4, 6, 10)+\Phi(n; 4, 6)$.
	Тут уже так просто не написать производящие.
	Можно умнее: ищем решения $4a+6b+10c=n$, это раскладка неразличимых предметов по трём различимым ящикам
	с ограничениями, выразили через произведение обычных ПФ, успех.
	Разбиений на слагаемые без учета порядка "--- $p(n)$ (полагают $p(0)=1$, $p(-n)=0$).
	Переформулировка: решить уравнение $\sum x_i=n$, $0 \le x_i\le x_{i+1}$.
	Переформулировка: $1x_1+2x_2+3x_3+\dots=n$, опять разложили по ящикам.
	Можно записать решение как бесконечное произведение (хоть мы его формально и не вводили).

	Ещё задача: кол-во разбиений, когда каждое число входит от 0 до 1 раза ($(1+z)(1+z^2)(1+z^3)\dots$).
	Можно переписать как $1+z^2=\frac{1-z^4}{1-z^2}$, посокращать, получим разбиения на нечётные слагаемые.

\section{} % 04
	Определить разбиение $n$ на ровно $k$ слагаемых (натуральных) без учёта порядка.
	Это $p_k(n)$, причём $\sum p_k(n) = p(n)$.
	А раскладка неразличимых шариков по неразличимым ящикам "--- это разбиение на не более чем $k$ слагаемых,
	зовётся $P_k(n)$, для любого $k \ge n$ имеем $P_k(n)=p(n)$.
	Показать связь между $p_k$ и $P_k$ (разность соседних, сумма) и
	$P_k(n)=p_k(n+k)$ (откусываем от слагаемого единичку), отсюда неважно, кого изучать "--- $P_k$ или $p_k$.

	Утверждение: $p_k(n)$ есть сумма $p_i(n-k)$ по $i$ от 1 до $\min(k, n-k)$ (при больших $i$ либо
	занулится $p_o(n-k)$, либо будет много слагаемых).
	Док-во: $i$ "--- это просто посчитано число слагаемых $\ge 2$.
	Утверждение: $p_k(n)=p_{k-1}(n-1)+p_k(n-k)$, т.к. либо убили единицу,
	либо откусили единицу от всех.

	Диаграмма Ферре (как Юнг, только точки вместо квадратов): каждая строка "--- слагаемое.
	Можно передоказать утверждения выше.
	Двойственная диаграмма "--- повернули на 90 градусов.
	Утверждение: $P_k(n)$ равно числу разбиений на сколько-то частей, каждая $\le k$ (перешли к двойственной).
	Но тогда можно записать производящую функцию с $k$ множителями (привет от Эйлера).
	Утверждение: $p_k(n)$ равно числу разбиений на сколько-то частей, каждая $\le k$, а одна $=k$.
	Это опять производящая функция (почти такая же, домножилась на $z^k$).

	Формула Эйлера: раскрыли скобки в бесконечном произведении в знаменателе ($Q(z)$),
	после очередной меняются только старшие коэффициенты.
	Чередуются 0, $\pm1$ по правилу: $Q(z)=1+\sum_{d=1} (-1)^d(z^{\frac{3d^2-d)}{2}}+z^{\frac{3d^2+d)}{2}})$.
	Числа $\omega(d)=\frac{3d^2-d}{2}$ и $\omega(-d)$ (симметричное) "--- пентагональные числа,
	суммы префиксов $1+4+7+10+\dots$.
	Док-во: посмотри, когда получим какой знак: когда разбиений $n$ на нечётные слагаемые больше/меньше,
	чем на чётные.
	Почти для всех совпадает, не совпадает только для пентагональных.
	Проводим биекцию для разбиений: взяли Ферре, разбили на трапеции (трапеция "--- кусок снизу,
	возрастает наверх ровно на единицу), $d$ "--- длина нижней строчки, а $l$ "--- высота самой верхней трапеции.
	Первый блок "--- все из $\ge 2$ трапеций, у которых $d \ge l$; из одной трапеции с $d > l$.
	Второй блок "--- $\ge 2$ трапеции, у которых $d < l$; из одной трапеции с $d < l-1$.
	Третий "--- одная трапеция с $d=l$, $d=l-1$.
	Первый блок переводится во второй и обратно: отрезали нижнюю строку, наклонили на 45 градусов, приклеили к диагонали.
	Чётность меняется.

	Отсюда, зная явную формулу для $Q(z)$, можно быстро получить рекурренту для $p(n)=p(n-1)+p(n-2)-p(n-5)-p(n-7)-\dots$.

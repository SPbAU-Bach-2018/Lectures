\chapter{Занятие 25.03.2016 (комбинаторика)}
\section{Разбор домашней работы}

\subsection{Задача 1}
	\TODO проблема с бесконечными произведениями (их нет)

	Напишем сначала формулу для числа разбиений на слагаемые не больше $n$,
	где каждое повторяется не более трёх раз:
	\begin{align*}
		f(z) &\coloneq (1+z+z^2+z^3)(1+z^2+z^4+z^6)\dots \\
		f(z) &= \prod_{i=1}^n (1+z^i+z^{2i}+z^{3i}) = \prod_{i=1}^n \frac{1-z^{4i}}{1-z^i}
	\end{align*}
	Теперь напишем формуля для числа разбиений на слагаемые не больше $n$,
	где повторяются только нечётные слагаемые:
	\begin{align*}
		g(z) &\coloneq (1+z+z^2+\dots)(1+z^2)(1+z^3+z^9+\dots)(1+z^4)\dots \\
		g(z) &=
			\prod_{i=0}^{\sfrac n2} \frac{1}{1-z^{2i+1}} \cdot
			\prod_{i=1}^{\sfrac n2} (1 + z^{2i}) \\
		g(z) &=
			\prod_{i=0}^{\sfrac n2} \frac{1}{1-z^{2i+1}} \cdot
			\prod_{i=1}^{\sfrac n2} \frac{1-z^{4i}}{1-z^{2i}} \\
		g(z) &=
			\prod_{i=1}^{n} \frac{1}{1-z^i} \cdot
			\prod_{i=2,4,6,8,\dots}^{n} (1-z^{2i}) \\
	\end{align*}
	И всё бы хорошо, если бы ряды были до бесконечности "--- они бы совпали.
	Но они у нас конечны, и это создаёт проблему "--- при фиксированном $n$ они не равны.
	Например, $g(z)$ может иметь сколь угодно больше ненулевые степени $z^n$, а степень $f(z)$ ограничена.

	Давайте продемонстрируем, что $f(z)-g(z) \vdots x^{n+1}$, это будет обозначать, что при любом
	$n$ первые $n$ коэффициентов у них совпадает, т.е. получим, что требовалось в задаче
	(так как мы не можем получить слагаемые больше $n$ при разложении $n$ на слагаемые).
	Для начала поделим эту разность на $\prod_{i=1}^n \frac{1}{1-z^i}$.
	\begin{gather*}
		\frac{f(z) - g(z)}{\prod \dots}
			= \prod_{i=1}^n (1-z^{4i}) - \prod_{i=2,4,6,8,\dots}^{n} (1-z^{2i}) 
			= \TODO
	\end{gather*}

\subsection{Задача 2}
	Тут не очень хорошо получилось с формулой, на самом деле предлагается пользоваться такой формулой
	(верной при $k \le \sfrac n2$):
	\[
		p_k(n) = \sum_{r=1}^{k} p_r(n-k)
	\]
	Выводим:
	\begin{gather*}
		p_k(n)
			= \sum_{r=1}^{k} p_r(n-k)
			= \sum_{r=1}^{k-1} p_r(n-k) + p_k(n_k)
			= \sum_{r=1}^{k-1} p_r((n-1)-(k-1)) + p_k(n_k)
			= p_{k-1}(n-1) + p_k(n_k)
	\end{gather*}

	Для случая $k>\sfrac n2$ второе слагаемое в искомой формуле просто зануляется (как и должно быть),
	получаем естественное равенство.

\subsection{Задача 3}
	Нарисуем диаграмму Ферре, заметим, что в конце идёт куча единичек.
	Их можно убирать или добавлять.

\subsection{Задача 4}
	Нарисуем диаграмму Ферре, посмотрим, перевернём, успех.

\subsection{Задача 5}
	\TODO как у меня: 

\subsection{Задача 6}
	Возьмём два старших слагаемых, прибавим к одному из них единичку.

\subsection{Задача 7}
	То же самое, что и 6, но геморроя больше.

\section{Разбор классной работы}
\subsection{Задача 1}
	Для одного слагаемого производящая функция должна быть такая: $e^z-1-z$, потому что
	мы разрешаем ящики с 2 и более элементами.
	Тогда ответ на задачу (производящая функция) "--- это $f(z)=e^{e^z-1-z}$, а для
	чисел Белла производящая функция "--- это $B(z)=e^{e^z-1}$.
	Можно написать:
	\begin{gather*}
		f(z) = B(z) \cdot e^{-z} = \sum_{n=0}^{\infty} \frac{z^n}{n!} \sum_{k=0}^n (-1)^{n-i} \binom{n}{i} B_i
	\end{gather*}

\subsection{Задача 3}
	Сколько способов упорядочить человек в группе из $n$ человек?
	Сначала выбираем центрального $n$ способами, а остальных $n-1$ упорядочиваем циклически
	еще $(n-2)!$ способами:
	\[ n(n-2)! = \frac{n!}{n-1} \]
	Теперь суммируем по $n$ и пишем производящую функцию $f(z)$:
	\begin{align*}
		f(z) &=
			\sum_{n=2}^\infty \frac{n!}{n-1} \cdot \frac{z^n}{n!} =
			\sum_{n=1}^\infty \frac{z^{n+1}}{n} \\
		\frac{f(z)}{z} &=
			\sum_{n=1}^\infty \frac{z^n}{n} \\
		\frac{f(z)}{z} &= -\ln (1 - z) \\
		f(z) &= -z\ln (1 - z) \\
	\end{align*}
	И ответ получается такой:
	\[
		e^{f(z)} = e^{-z\ln (1-z)} = \left(e^{\ln(1-z)}\right)^{-z} = (1-z)^{-z}
	\]

\section{Организационное}
	Оставшиеся задачи "--- домашнее задание.
	Разбирать графы будем на занятии по графам, уже выложенная
	домашняя работа по графам сдаётся к тому же моменту.

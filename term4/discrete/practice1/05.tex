\chapter{Занятие 11.03.2016 (графы)}
\section{Разбор домашней работы}

\subsection{Задача 1}
	\begin{Rem}
		Под $k$-связностью подразумевается <<хотя бы $k$-связен>> (то есть $(k+1)$-связность тоже подходит).
	\end{Rem}
	Давайте покажем, что $G'$ вершинно $k$-связен.
	Если $k=0$, то всё очевидно (ничего показывать не надо).
	Если $k=1$, то тоже всё очевидно (граф был связен, подцепили к нему неизолированную вершину, связность сохранили).
	Теперь $k > 1$.
	Удалим произвольную $(k-1)$ вершину, покажем, что граф остался связен.
	Два случая:
	\begin{itemize}
		\item
			Пусть мы удалили вершину $y$ и еще $k-2$ вершины из $G$.
			Тогда, очевидно, остаток связен, так как $G$ являлся $k$-связным
			(и уж тем более $(k-2)$-связным), а вершину $y$ мы удалили и проверять
			её достижимость не надо.
		\item
			Пусть мы сохранили вершину $y$, но удалили $k-1$ вершину из $G$.
			Тогда оставшиеся в $G$ вершины связны между собой даже без учёта вершины $y$,
			так как $G$ был $k$-связным.
			А вершина $y$ осталась смежна хотя бы с одной из них, так как её степень была
			равна $k$.
			Получаем, что весь остаток связен
	\end{itemize}

\subsection{Задача 2}
	Добавим в граф вершину $y$, соединим её со всеми вершинами множества $Y$.
	Заметим, что граф при этом всё еще остаётся $k$-связным по предыдущей задаче.
	Применяем теорему Менгера для вершин $x$ и $y$, получаем, что между ними
	существует $k$ не пересекающихся по внутренности путей.
	Удалив из каждого пути его конец (вершину $y$) получаем в точности нужный $k$-веер.
	\begin{Rem}
		Существование $k$-веера "--- очень мощная техника доказательства разных фактов про связность.
		Главное "--- помнить про условия: $|Y|\ge k$ и $x\notin Y$.
	\end{Rem}

\subsection{Задача 3}
	Возьмём максимальный простой цикл $C$.
	Если в нём содержится хотя бы $2k$ вершин "--- мы уже победили.
	Если в нём содержатся все $n$ вершин "--- мы тоже победили, так как $n\ge2k$.
	Если же содержатся не все, то возьмём какую-нибудь вершину $x$, которая в $C$ не лежит.
	Дальше разберём два случая:
	\begin{enumerate}
		\item
			Если $|C|< k$, то найдём $|C|$-веер из вершины $x$ в цикл $C$,
			такой существует по предыдущей задаче (так как граф $k$-связен и уж
			всяко $|C|$-связен).
			Заметим, что на самом деле мы нашли непересекающиеся пути из $x$
			в каждую вершину цикла.
			Возьмём две смежные вершины цикла и добавим вершину $x$ между ними,
			используя два пути из веера.
		\item
			Если $|C|\ge k$, то найдём $k$-веер из вершины $x$ в цикл $C$,
			такой существует по предыдущей задаче.
			Заметим, что так как $|C|< 2k$, то в цикле обязательно найдутся
			две смежные вершины, в которых приходят пути из веера.
			Тогда между ними можно аналогичным образом добавить вершину $x$ в цикл.
	\end{enumerate}
	В любом случае мы удлинили цикл $C$ и получили тем самым противоречие.

\subsection{Задача 4}
	Для $k = 1$ утверждение либо неверно (так как у нас может быть дерево
	без простых циклов, если мы не считаем вершину этаким вырожденным циклом),
	либо очевидно (если считаем).
	Индукция по $k$ при $k\ge 2$.
	База: для $k=2$ есть вершинная теорема Менгера или просто одна из теорем с лекций
	про существование общего цикла для любых двух вершин.

	Докажем переход от $k-1$ к $k$ ($k \ge 3$).
	Пусть есть цикл $C$, на котором лежат вершины $x_1$, \dots, $x_{k-1}$ (не умаляя общности
	считаем, что именно в таком порядке).
	Еще есть вершина $x_k$.
	Если она лежит на цикле, то мы победили.
	Если нет "--- два случая:
	\begin{enumerate}
		\item
			Если $|C|<k$, то $|C|=k-1$ и мы можем найти $(k-1)$-веер из $x_k$ в цикл.
			В частности, мы найдём непересекающиеся пути в $x_1$, $x_2$,
			тогда можно просто встроить $x_k$ между ними.
		\item
			Если $|C|\ge k$, то мы можем найти $k$-веер из $x_k$ в цикл.
			Рассмотрим отрезки цикла $[x_1; x_2]$, \dots, $[x_{k-2}, x_{k-1}]$, $[x_{k-1}; x_1]$.
			Их $k-1$, значит, по принципу Дирихле хотя бы в один у нас есть два непересекающихся
			пути из $x$.
			Тогда внутрь этого отрезка можно встроить вершину $x$, сохранив концы
			(возможно вырезав какие-то внутренние вершины отрезка), что и требовалось.
	\end{enumerate}

\subsection{Задача 5}
	\subsubsection{Разбор от семинариста}
	В следующем двусвязном графе не существует цикла $1 \to 3 \to 2 \to 4 \to 1$:
	\begin{center}
		\neatograph[scale=0.6]{g05p5}{
		    A[label="1",pos="0,0!"];
		    B[label="2",pos="0,3!"];
		    C[label="3",pos="3,3!"];
		    D[label="4",pos="3,0!"];
		    A1[label="",pos="1.5,0.75!",style=dot];
		    B1[label="",pos="0.75,1.5!",style=dot];
		    C1[label="",pos="1.5,2.25!",style=dot];
		    D1[label="",pos="2.25,1.5!",style=dot];
		    A -- B -- C -- D -- A;
		    A1 -- B1 -- C1 -- D1 -- A1;
		    A -- A1; A -- B1;
		    B -- B1; B -- C1;
		    C -- C1; C -- D1;
		    D -- D1; D -- A1;
		}
	\end{center}
	
	\subsubsection{Пример от Севы}
	Трёхдольный граф $G$, в каждой доле $k$ вершин,
	рёбра есть только между долями (всего $3k^2$ рёбер).
	Заметим, что две вершины из разных долей связаны всегда (пока живы),
	а вот из одной доли живы, пока есть хотя бы одна вершина из какой-то другой доли.
	Таким образом, $G$ ровно $2k$-связен: для разрушения связности надо разрушить две доли целиком.

	Тогда берём вершины в таком порядке: $k$ вершин из одной доли, потом $k$ вершин из другой доли.
	Простого цикла, который обходит в таком порядке, не существует (при $k > 2$),
	так как между любыми двумя вершинами должна встретиться вершина третьей доли,
	итого должно быть хотя бы $2k-2$ вершин третьей доли, а $2k-2>k$.

\subsection{Задача 6}
	Так как граф не двудольный, то есть цикл нечётной длины хотя бы три.
	Возьмём кратчайший такой цикл.
	Если нет вершины $x$, лежащей вне этого цикла, то, значит, все оставшиеся
	в графе рёбра являются хордами.
	Если есть хотя бы одна хорда, то можно найти нечётный цикл меньшей длины, противоречие.
	Если хорд нет, то наш граф является циклом и не трёхсвязен.

    Теперь можно считать, что есть вершина $x$, лежащая вне этого цикла.
	Давайте проведём из неё 3-веер в цикл.
	Получили четыре нечётных цикла: один исходный, и еще три можно получить, выбрав
	два пути из веера и обойдя исходный цикл в правильную сторону (чтобы сошлась нечётность).
	\begin{center}
		\neatograph[scale=0.6]{g05p6}{
		  A -- B -- C -- D -- E -- F -- G -- A;
		  x [color=red];
		  x -- A [color=red];
		  x -- D [color=red];
		  x -- F [color=red];
		}
	\end{center}

\subsection{Задача 7}
	Возьмём две самые дальние друг от друга вершины: $x$ и $y$.
	Любой путь между ними имеет длину (в рёбрах) хотя бы $d$.
	Также между ними имеется хотя бы $k$ непересекающихся путей.
	Значит, в них в сумме хотя бы $k(d-1)$ вершин, а в графе "--- $k(d-1)+2$.

	Пример: давайте возьмём $(d-1)$ штук графов $K_k$, соединим их <<по цепочке>>.
	\TODO картинка, доказательство

\section{Разбор классной работы}
\subsection{Задача 3.1}
	Ткнули в первую вершину.
	Смотрим, сколько у неё вариантов соседей.
	Их $(n-1)$.
	Выкинули эти две вершины.
	Ткнули в первую из оставшихся, для неё $(n-3)$ варианта.
	Итого ответ для чётных $n$ "--- $(n-1)!!$, для нечётных "--- 0.

\subsection{Задача 3.2}
	Либо ноль, либо один.
	Если дерево с нечётным числом вершин "--- ноль.
	Иначе рассмотрим лист, инцидентное ему ребро обязательно надо удалить.
	Удаляем его вместе с двумя вершинами, получили лес.
	Каждая компонента "--- дерево, ответы "--- ноль и один, надо перемножить.
	Получаем либо 0, либо 1.

\subsection{Задача 3.3}
	Для двудольного паросочетание просто задаёт нам перестановку.
	То есть ответ "--- число перестановок, $n!$.
	А если мы выкинули рёбра паросочетания, то не умаляя общности можно считать,
	что мы выкинули рёбра из паросочетания, соответствующего тождественной перестановке.
	То есть надо посчитать число перестановок без неподвижных точек "--- число беспорядков "--- $!n$.

\subsection{Задача 2.1}
	Утверждение неверно, пример:
	\begin{center}
		\digraph[scale=0.6]{g02p1}{
			edge[dir=none];
			rankdir=LR;
			{rank=same; 1 5 }
			{rank=same; 2 6 }
			{rank=same; 3 7 }
			{rank=same; 4 8 }
			1 -> 2[color=red]; 2 -> 3; 3 -> 4[color=red];
			5 -> 6; 6 -> 7[color=red]; 7 -> 8;
			1 -> 5[color=red]; 2 -> 6[color=red]; 3 -> 7[color=red]; 4 -> 8[color=red];
		}
	\end{center}
	Здесь между вершинами 1 и 8 есть Гамильтонов путь и они не связаны ребром.

\subsection{Задача 2.2}
	Есть пять рёберно непересекающихся путей из 1 в 12: $1 \to 2, 3, 4, 6 \to 12$ и $1 \to 12$.
	Значит, надо удалить хотя бы пять рёбер.
	Просто удаляем все смежные с 12.

\subsection{Задача 2.4}
	Что такое граф-гиперкуб?
	Вершины "--- битовые строчки длины $k$, ребром соединены те, что отличаются ровно в одном бите.
	Надо найти много ($k)$ непересекающихся путей из одной строки в другую.
	Взяли две различные вершины, не умаляя общности считаем, что у них общий только префикс, причём длины $l$
	(просто переставили биты).
	Рассматриваем два варианта путей:
	\begin{enumerate}
		\item
			Переключаем все биты в различающемся хвосте, начиная с некоторого, по циклу.
			Таких путей $k-l$. Например:
			\[
				\t{000\textit{01\textbf{0}1}} \to
				\t{000\textit{011\textbf{1}}} \to
				\t{000\textit{\textbf{0}110}} \to
				\t{000\textit{1\textbf{1}10}} \to
				\t{000\textit{1010}}
			\]
		\item
			Переключаем один бит из общего префикса, потом переключаем все в хвосте, потом переключаем бит обратно.
			Таких путей $l$.
	\end{enumerate}
	Итого $k$ путей, все различны.

\subsection{Задача 1.16}
	Возьмём двусвязный граф, построим нужное разбиение.
	Взяли произвольный цикл, назвали его $G_0$.
	Пусть есть какая-то вершина $x$, не лежащая на нём.
	Тогда возьмём какую-нибудь вершину $y$ на цикле, найдём цикл, содержащий $x$ и $y$.
	Оборвали этот цикл рядом с $x$ в момент первого пересечения с $G_0$,
	получили ручку.
	Продолжаем процесс, пока не закончатся вершины (в каждый момент находим цикл между очередной
	вершиной и из вершиной из старого графа, обрываем его, когда он заходит в старый граф).

	Теперь в обратную сторону.
	Если мы удаляем одно ребро на цикле, то связность, очевидно, не теряется.
	Если на ручке "--- тоже не теряется, так как можно пройти по другой стороне ручки.
	А два ребра удалить можно: если были ручки "--- удаляем два соседних ребра на последней ручке,
	если не было "--- два соседних ребра на цикле.

\subsection{Задача 1.15}
	Просто берём граф Петерсена и применяем алгоритм.
	\TODO картинка

\subsection{Задача 1.13}
	Надо воспользоваться тем, что граф блоков и точек сочленения "--- это дерево.
	Все блоки $k$-связны.
	Покажем, что граф $k$-связен.
	Пусть мы удалили $k-1$ ребро (из графа, т.е. внутри блоков), тогда ни один из блоков не развалился, стало быть,
	внутри блока связность не потерялась, а внутри дерева "--- тем более.

	В обратную сторону: пусть есть $k$-связный граф, и в нём есть $l$-связный блок (где $l<k$),
	то есть из него можно удалить $l$ рёбер так, чтобы он стал несвязным.
	Давайте их удалим, заметим, что граф стал несвязным: в самом деле, взяли две вершины в этом блоке,
	которые стали несвязными.
	Тогда проблема: мы не можем пройти между этими вершинами, потому что выход за пределы блока нам не поможет.

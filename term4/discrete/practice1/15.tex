\chapter{Занятие 20.05.2016 (графы)}

\section{Разбор классной работы}

\subsection{Задача 2}
	Правильная формула: $m \le 3n - 6$.
	Берём $K_6$, в нём 15 рёбер.
	Если удаляем одно ребро, то условие не будет выполняться.
	Если удалим два, то тоже не выполняется (\TODO что-то про оценку числа пересечений через число удалённых рёбер).
	А для трёх надо нарисовать пример: квадратик, а внутри "--- отрезок.

\subsection{Задача 4}
	На каждой оси координат расположим по доле графа "--- половину вершин в одной
	половине оси, другую половину "--- в другой половине оси.
	Получаем, что число пересечений не превышает:
	\[ 4 \cdot \binom{t}{2} \binom{s}{2} \]
	Как догадаться: увидеть биномиальный коэффициент в ответе.

\section{Разбор домашней работы}
\subsection{Задача 1}
	Интересно доказывать для критического графа, иначе можно удалить произвольное ребро, доказывать станет сложнее.
	От противного: пусть $k \ge \frac32 \Delta(G) + 1$.

	Пусть у нас есть $k$-критический граф ($\xi'(G)=k$).
	Удалим какое-то ребро $e=(x, y)$ такое, что между $x$ и $y$ есть максимальное число рёбер ($\mu(G)$).

	Теперь граф можно покрасить в $k-1$ цвет.
	Посмотрим, сколько цветов остались доступными в вершинах $x$ и $y$:
	\begin{align*}
		|\Omega(x)| &\coloneq k - \Delta(x) - 1 \ge k - \Delta(G) - 1 \\
		|\Omega(y)| &\coloneq k - \Delta(y) - 1 \ge k - \Delta(G) - 1 \\
		|\Omega(x)| + |\Omega(y)| &\ge 2k - 2\Delta(G) - 2 \\
%		&\text{но между $x$ и $y$ есть общие рёбра} \\
%		|\Omega(x) \cap \Omega(y)| &\ge \mu(G) - 1 \\
%		|\Omega(x) \cup \Omega(y)| &\le 2k - 2\Delta(G) - 2 - (\mu(G) - 1) \\
	\end{align*}
	
	Сколько цветов осталось свободными между $x$ и $y$ из исходных $k-1$?
	Не больше, чем $k-1-(\mu(G)-1)$, так как между ними осталось $\mu(G)-1$ ребро.
	Воспользуемся формулой с лекции:
	\begin{gather*}
		\xi'(G) \le \Delta(G) + \mu(G) \\
		-\mu(G) \le \Delta(G) - \xi'(G)
	\end{gather*}
	Пусть $\Omega(x)$ "--- свободные цвета у вершины $x$.
	Пишем оценку (\TOODO: тут в первой строчке потеряна $-1$, из-за этого получился в конце 0, а не 1,
	и мы ничего содержательного сказать не можем):
	\begin{gather*}
		|\Omega(x) \cup \Omega(y)| \le k - 1 - \mu(G) + 1 \le k + \Delta(G) - k = \Delta(G) \\
		|\Omega(x) \cap \Omega(y)|
		\ge |\Omega(x)| + |\Omega(y)| - |\Omega(x) \cup \Omega(y)|
		\ge |\Omega(x)| + |\Omega(y)| - \Delta(G)
		\ge 2k - 3\Delta(G) - 2
		\ge 2(\frac32 \Delta(g) + 1) - 3\Delta(G) - 2
		\ge 3\Delta(g) - 3\Delta(G)
		\ge 0
	\end{gather*}

\section{Задача 2}
	Заметим, что если мы возьмём индуцированный подграф хордального, то он окажется хордальным.
	Значит, надо показать, что для любого хордального графа на $n$ вершинах верно, что $\xi(G)=\omega(G)$.

	Индукция по $n$, база $n=1$ очевидна, переход $n \to n+1$.
	Возьмём хордальный граф из $n+1$ вершины, посмотрим на симплициарный порядок вершин
	$x_1, \dots, x_{n+1}$, мы знаем, что соседи вершины $x_i$ с меньшими номерами образуют клику.
	В частности, вообще все соседи $x_{n+1}$ образуют клику.

	Обзоначим $k \coloneq \xi(G)$.
	Если $\deg x_{n+1} \ge k$, то в $G$ имеется клика размера $k+1$, противоречие.
	Значит,$\deg x_{n+1} < k$.
	Удалим вершину $x_{n+1}$, получим граф, для него точно $\xi(G-x_{n+1})=\omega(G-x_{n+1})$ (по предположению индукции).
	Когда добавляем вершину обратно, $\omega$ могло увеличиться не более, чем до $\deg x_{n+1}+1$ в лучшем случае, т.е. $\omega(G)\le k$.
	С другой стороны, мы знаем, что в любом графе $\omega(G)\ge k$.
	Отсюда $\omega(G)=k$.

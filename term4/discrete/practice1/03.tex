\chapter{Занятие 26.02.2016 (графы)}
\section{Работа домашней работы}
\subsection{Задача 1}
	\TODO

\subsection{Задача 2}
	Давайте удалим произвольную $k-1$ вершину.
	Посчитаем суммарную степень в графе.
	Всего вершин у нас осталось $n-k+1$, степень каждой хотя бы $\delta(G)-(k-1)=\frac{n-k}{2}$,
	суммарная степень:
	\[ \frac{n-k}{2} \cdot (n-k+1) = \frac{(n-k+1)(n-k)}{2} \]
	Так как граф простой, то мы получаем, что он равен $K_{n-k+1}$, т.е. точно связен.
	Противоречие.

\subsection{Задача 3}
	В этой задаче надо понять, сколько раз мы считаем каждое ребро.
	Рассмотрим вершину из $S$.
	Ребро, смотрящее наружу, считается ровно один раз.
	Ребро, смотрящее внутрь $S$, считается ровно два раза, их мы как раз и вычитаем.

	Далее надо доказать, что граф Петерсена трёхсвязен.
	Предположим, что это не так.
	\subsubsection{Односвязность}
		Пусть граф Петерсена односвязен, т.е. можно удалить одну вершину ($|S|=1$) и граф распадётся.
		Посчитаем по формуле, сколько рёбер удалилось из графа:
		\[ |\partial(S)| = \sum \deg x - 2|E(G[S])| = 3 - 2 \cdot 0 = 3 \]
		Пусть остаток графа распался на хотя бы две компоненты суммарного размера 9.
		Тогда одна из них точно нечётного размера.
		Значит, в неё входило нечётное число рёбер из удалённой вершины
	\TODO

\subsection{Задача 6}
	Докажем первый нужный факт от противного.
	Пусть есть вершина $x$ из $S$, у которой все соседи лежат в $S$.
	Посмотрим на вершины $\bar S$.
	Заметим, что у любой вершины из $\bar S$ есть сосед из $S$, потому что в противном случае
	расстояние от соответствующей вершины до $x$ будет хотя бы три "--- надо сначала
	перейти внутри $\bar S$, потом перескочить в $S$, потом перескочить внутри $S$.
	Значит, размер разреза $|\partial(S)|$ не меньше $|S'|$, потому что у каждой вершины $S'$
	хотя бы одно ребро участвует в разрезе.

	А теперь рассмотрим разрез $[x; V-x]$.
	Его размер равен $\deg x$.
	Так как $x$ соединена только с вершинами $S$, то её степень строго меньше $|S|$.
	Итого:
	\[ [x; V - x] = \deg x < |S| \le |S'| \le |\partial(S)| \]
	Получаем, что существует разрез меньше, чем исходный, противоречие.

	Теперь покажем, что $\lambda(G)=\delta(G)$.
	Мы для всех графов знаем, что $\lambda(G) \le \delta(G)$.
	Покажем, что $\lambda(G) \ge \delta(G)$.
	Возьмём произвольную вершину $v$ из $S$.
	Пусть у неё идёт $a$ рёбер внутрь $S$ и $b$ рёбер в $\bar S$.
	Давайте каждому ребру $v \to u$, ведущему внутрь, сопоставим ребро из $u \in S$ ведущее
	в некоторую вершину другой доли $u' \in \bar S$.
	Получили набор из $a+b$ различных рёбер, ведущих между $S$ и $\bar S$.
	Значит, их не больше, чем $|[S; \bar S]|=\lambda(G)$.
	Значит, степень вершины $v$ не больше $\lambda(G)$:
	\[ \delta(G) \le \deg v \le \lambda(G) \]
	Что и требовалось показать.

\subsection{Задача 5}
	Воспользуемся формулой из задачи 2 и оцениваем снизу число рёбер в разрезе $\partial(S)$:
	мы знаем, что степень каждой вершины не более $\delta(G)$, а суммарное число рёбер в
	$G[S]$ не может быть больше, чем в $K_{|S|}$.
	Итого:
	\[ |\partial(S)| \ge |S|\delta(G) - 2\cdot\frac{|S|(|S|-1)}{2} \]
	Также мы знаем, что $\delta(G) > |\partial(S)|$:
	\begin{gather*}
		\delta(G) > |S|\delta(G) - |S|(|S|-1) \\
		|S|(|S|-1) > (|S|-1)\delta(G) \\
		|S| >  \delta(G)
	\end{gather*}

\subsection{Задача 7}
	Идея решения: давайте запустим DFS, все рёбра из дерева обхода ориентируем вниз,
	а обратныё ребра "--- наверх.
	Дальше будем обосновывать сильную связность.
	Важно именно DFS, BFS не прокатит:
	\digraph[scale=0.5]{g03h7}{
		1 [rank=0];
		2 [rank=1];
		3 [rank=1];
		1 -> 2 -> 3;
	}

	Давайте посмотрим на дерево DFS'а и обратные рёбра.
	Заметим, что обратные рёбра не могут идти между ветками (т.е. для любого ребра не из дерева
	одно из его концов является предком другого).
	Это легко понять: берём ребро не из дерева, мы из какой-то вершины вышли раньше.
	Из той, которой вышли, мы не пошли по этому ребру только потому что второй конец уже был в обработке,
	т.е. он и являлся предком.

	Заметим, что мы можем добраться из корня дерева до любой вершины в такой ориентации.
	Давайте теперь покажем, что из любой вершины $v$ (кроме корня), можно добраться на хотя бы один уровень выше
	(скажем, в её родителя $u$).
	Для этого возьмём исходный граф, удалим из него ребро $v \to u$, так как он был двусвязен,
	то был путь из $v$ в $u$.
	Возьмём в этом пути первое ребро, которое выводит нас из поддерева $v$.
	Такое точно есть, так как вершина $u$ в поддереве не лежит.
	Заметим, что оно подняло нас на несколько уровней выше $v$, что и требовалось.

\subsection{Задача 8}
	Переформулируем: скажем, что для любой тройки есть два простых непересекающихся пути:
	$x \to y$ и $x \to z$.
	\begin{Rem}
		Это обобщается: если в графе для любого набора из $k$ вершин и еще одной
		выполняется это свойство (существует так называемый <<$k$-веер>>), то граф
		$k$-связен.
	\end{Rem}
	Понятно, что если условие верно, то граф двусвязен.
	В самом деле, пусть он лишь односвязен: т.е. при удалении вершины $v$
	вершины $x$ и $y$ перестают быть связными.
	Однако у нас был $k$-веер для начальной вершины $x$ и конечных вершин $v$, $y$.
	Значит, есть путь $x \to y$, не проходящий через $v$, противоречие.

	Теперь пусть у нас есть двусвязный граф, покажем, что существуют 2-веера.
	Пусть есть три вершины "--- $x$, $y$, $z$.
	Возьмём какой-нибудь цикл $C$, на котором лежат вершины $x$ и $y$ (такой есть из-за двусвязности).
	Возьмём кратчайший простой путь от $z$ до какой-нибудь вершины цикла ($\alpha$).
	Если $\alpha\neq y$, то, пойдя в две стороны от $x$ по циклу, мы с одной стороны найдём
	$\alpha$ (и можем продолжить путь до $z$), а с другой "--- до $y$, нашли нужный веер.
	Если же $\alpha=y$, то поймём, что существует и какой-то другой путь до цикла, не проходящий через $y$.
	В самом деле: пусть любой путь до цикла проходит через $y$.
	Тогда, в частности любой путь до $x$ проходит через $y$.
	Тогда удалим $y$, получим, что вершины $x$ и $z$ стали несвязными, противоречие.

\section{Работа классной работы №1}
\subsection{Задача 1.8}
	\begin{enumerate}
	\item Контрпример: два ребра.
	\item Существует: $K_2$
	\item У нас граф выглядит как путь из двух рёбер и никак иначе, т.е. <<да>>.
	\item Неправда, т.к. возьмём два ребра, смежные с этими вершинами.
	\item Контрпример "--- гантеля.
		\digraph[scale=0.5]{g01p8}{
			edge[dir=none];
			rankdir=LR;
			1 [rank=0];
			2 [rank=0];
			3 [rank=1];
			4 [rank=2];
			5 [rank=3];
			6 [rank=3];
			1 -> 2 -> 3 -> 4 -> 5 -> 6;
			1 -> 3; 4 -> 6;
		}
	\item Не может быть двусвязным, потому что иначе любые две вершины должны лежать на одном цикле.
	\item Правда, так как это отрицание пункта 4
	\end{enumerate}
	\TODO проверить

\subsection{Задача 1.10}
	Рассмотрим дерево из блоков и точек сочленения.
	Каждый блок <<расконденсируем>>, т.е. превратим одну вершину блока в реальный подграф, соответствующий блоку.
	Каждая вершина исходного графа, кроме точек сочленения, в этой расконденсации будет ровно один раз.
	А точки сочленения "--- несколько раз.
	Заметим, что у такого графа остовные деревья биективно отображаются на остовные деревья исходного графа:
	\begin{itemize}
		\item Если был старый остов, добавляем в него все рёбра, идущие из точек сочленения в блоки
		\item Если есть новый остов, то просто склеиваем все образы каждой точки сочленения.
	\end{itemize}
	Так как все остовы такого графа для каждой точки сочленения содержат все смежные с ней рёбра (ведущие в блоки),
	то уже можно говорить, что остовные деревья такого графа есть декартово произведение остовов графов, что и требовалось.

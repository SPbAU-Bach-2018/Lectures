\chapter{Занятие 04.03.2016 (комба)}
Занятие записано с конспектов присутствующих, может чего-то не хватать, решения могут быть записаны
не слишком аккуратно.

\section{Разбор домашней работы}
\subsection{Задача 1}
	Смешаем все колоды вместе, теперь мы получили колоду, в которой есть 52 типов карт, по четыре карты каждого типа.
	Переформулируем задачу в терминах ящиков и шариков.

	У нас имеется 52 различимых ящика (по одному на тип карты), нам требуется в них разложить пять неразличимых шариков (шарик помечает выбранную карту).
	При этом есть ограничение "--- в каждом ящике лежит не более четырёх шариков (так как карт каждого типа четыре).
	\begin{Rem}
		Аналогичное рассуждение произведено в конспекте, примеры 1.5 и 1.6, начиная со страницы 5.
	\end{Rem}
	Пишем производящую функцию, которая характеризует это ограничение:
	\[
		f(z) = 1 + z + z^2 + z^3 + z^4
	\]
	Всего ящиков 52, поэтому ответом будет являться коэффициент при $z^6$ функции $(f(z))^{52}$.
	Чтобы получить более явную формулу, можно заметить следующее:
	\begin{gather*}
		\begin{aligned}
			(1-z)f(z) &= 1-z^5 \\
			f(z) &= \frac{1-z^5}{1-z}
		\end{aligned} \\
		(f(z))^{52} = \frac{(1-z^5)^{52}}{(1-z)^{52}}
	\end{gather*}

\subsection{Задача 2}
	Переформулируем задачу в терминах раскладки неразличимых шариков по ящикам.
	
	Всего шариков будет 78 (шарик соответствует грамму), а ящиков будет восемь (по числу гирек).
	В $i$-й ящик можно будет класть либо ноль шариков, либо $a_i$ шариков (где $a_i$ "--- вес $i$-й гирьки).
	Тогда образуется биекция между раскладками шариков по ящикам и способами составить нужный вес.
	Производящая функция для $i$-го ящика будет выглядеть так:
	\[ f_i = 1 + z^{a_i} \]
	А ответом на задачу будет являться коэффициент при $z^{78}$ следующей производящей функции:
	\begin{gather*}
		(1+z)(1+z)(1+z^2)(1+z^5)(1+z^{10})(1+z^{10})(1+z^{20})(1+z^{50}) \\
		(1+z)^2(1+z^2)(1+z^5)(1+z^{10})^2(1+z^{20})(1+z^{50}) \\
	\end{gather*}

\subsection{Задача 3}
	\begin{Rem}
		В условии считаем, что за экзамен можно получить одну из четырёх оценок: 2, 3, 4, 5.
		Для прохода требуется, чтобы сумма оценок по четырём экзаменам была хотя бы 17.
	\end{Rem}
	Переформулировка: у нас есть четыре различимых ящика (экзамены) и неизвестное количество неразличимых шариков (баллы за экзамены).
	В каждый ящик можно положить от 2 до 5 шариков, т.е. производящая функция для одного ящика такая:
	\[ f(z) = z^2 + z^3 + z^4 + z^5 = \frac{z^6-z^2}{z-1} = z^2\frac{z^4-1}{z-1} \]
	А производящая функция для количества способов набрать определённое число баллов за экзамены выглядит так:
	\[
		(f(z))^4 = z^{8}\frac{(z^4-1)^4}{(z-1)^4} = z^8\underbrace{\frac{1-4z^4+6z^8-4z^{12}+z^{16}}{(1-z)^4}}_{g(z)}
	\]
	Так как за экзамены можно набрать не более 20 баллов, то ответом на задачу будет сумма коэффициентов этой функции при $z^{17}, z^{18}, z^{19}, z^{20}$.
	\TODO

\subsection{Задача 4}
	\TODO

\subsection{Задача 5}
	\TODO

\subsection{Задача 6}
	\TODO

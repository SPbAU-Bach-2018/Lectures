\chapter{Занятие 29.04.2016 (графы)}

\section{Разбор классной работы}
\subsection{Задача 1}
	Возьмём $P(z)$, подставим $z=1$.
	С одной стороны это сумма коэффициентов.
	С другой стороны это количество способов раскрасить граф в один цвет.
	Так можно раскрасить только пустой граф (и ровно одним способом).
	Получили комбинаторное доказательство.

	Также можно доказывать рекурсивно по индукции.

\subsection{Задача 2}
	Уточнение условия:
	количество правильных раскрасок не больше $k(k-1)^{n-1}$ вообще для всех связных графов.
	А для связных графов, не являющихся деревьями, оно строго меньше.

	Для начала разберёмся с деревом.
	Зафиксировали произвольный корень, его можно покрасить в $k$ цветов.
	Дальше красим по уровням, у каждой вершины покрашен только родитель, значит,
	её можно покрасить в $k-1$ цвет.
	Итого $k(k-1)^{n-1}$ раскрасок.

	Теперь покажем, что если к дереву добавить хотя бы одно ребро, то хотя бы одна раскраска станет некорректной.
	Покрасим дерево в два цвета (по чётности расстояний от корня).
	Если мы добавляем ребро между вершинами на одном уровне, то эта раскраска ломается.
	Если же мы добавляем ребро между вершинами на разных уровнях, то они точно не соединены в дереве
	и оба конца ребра можно перекрасить в третий цвет (так как $k \ge 3$).
	Раскраска дерева останется корректной, но вот ребро сломается.
	\begin{Rem}
		Если $k=2$, то исходная оценка всё ещё верна, но вот строгое неравенство уже достигается не на всех не-деревьях, а только на двудольных.
	\end{Rem}

\subsection{Задача 3}
	Возьмём центр колеса и, не умаляя общности, покрасим его в цвет номер $1$.
	Оставшиеся вершины колеса должны быть покрашены в оставшиеся $k-1$ цвет,
	то есть $P(W_n) = z\cdot P(C_n)$ (тут $C_n$ "--- цикл из $n$ вершин, он же "--- колесо без центра).

	Посчитаем сначала полином для пути: $P(P_n)$.
	Считаем комбинаторно: мы красим первую вершину пути в $z$ цветов, каждую из оставшихся "--- в $z-1$ цвет.
	Получаем следующее условие при $n \ge 1$:
	\[
		P(P_n) = z(z-1)^{n-1}
	\]

	Теперь посчитаем для $C_n$ рекурсивно по формуле из конспекта.
	Сначала напишем рекурренту при $n \ge 3$:
	\[
		P(C_n) = P(\underbrace{P_n}_{C_n - e}) - P(\underbrace{C_{n-1}}_{C_n\setminus e})
	\]
	Граничное условие при $n = 2$ (посчитали комбинаторно):
	\[ P(C_2)=z(z-1)=P(P_2) \]
	Раскручиваем рекурренту (необязательно):
	\begin{align*}
		P(C_n) &= P(P_n) - P(P_{n-1}) + P(P_{n-2}) - \dots + (-1)^{n-3}P(P_3) + (-1)^{n-2}\underbrace{P(C_2)}_{P(P_2)} \\
		P(C_n) &= \sum_{k=2}^n (-1)^{n+k}P(P_k) \\
		P(C_n) &=
			\sum_{k=2}^n (-1)^{n+k}z(z-1)^{k-1} =
			(-1)^{n+1}\sum_{k=2}^n (-1)^{k-1}z(z-1)^{k-1} = \\
		&=	(-1)^{n+1}\sum_{k=2}^n (-z(z-1))^{k-1} =
			(-1)^{n+1}\sum_{k=1}^{n-1} (-z(z-1))^{k} \\
		P(C_n) &=
			(-1)^{n+1}\left(\sum_{k=0}^{n-1} (-z(z-1))^{k} - (-z)\right) =
			(-1)^{n-1}\left(\frac{(-z(z-1))^n-1}{-z(z-1)-1} - (-z)\right) = \TODO \\
	\end{align*}

\subsection{Задача 4}
	Пусть лесенка состоит из $n$ слоёв, обозначим её за $G_n$.
	Обозначим верхнее ребро лесенки за $e$.
	\begin{center}
		\begin{tabular}{ccc}
			\begin{minipage}{2cm}\begin{center}
				\begin{dot2tex}[options=-tmath]
					graph G {
						node [shape=point]
					    rankdir=LR
						{ rank=same a1 -- a2 -- a3 -- a4 }
						{ rank=same b1 -- b2 -- b3 -- b4}
						a1 -- b1 [label=e]; a2 -- b2; a3 -- b3; a4 -- b4
					}
				\end{dot2tex}
			\end{center}\end{minipage}
			&
			\begin{minipage}{2cm}\begin{center}
				\begin{dot2tex}[options=-tmath]
					graph G {
						node [shape=point]
						{ rank=same a1 b1 }
						{ rank=same a2 -- b2 }
						{ rank=same a3 -- b3 }
						{ rank=same a4 -- b4 }
						a1 -- a2 -- a3 -- a4
						b1 -- b2 -- b3 -- b4
					}
				\end{dot2tex}
			\end{center}\end{minipage}
			&
			\begin{minipage}{2cm}\begin{center}
				\begin{dot2tex}[options=-tmath]
					graph G {
						node [shape=point]
						x
						{ rank=same a2 -- b2 }
						{ rank=same a3 -- b3 }
						{ rank=same a4 -- b4 }
						x -- a2 -- a3 -- a4
						x -- b2 -- b3 -- b4
					}
				\end{dot2tex}
			\end{center}\end{minipage}
			\\
			$G_4$ & $G_4 - e$ & $G_4 \setminus e$
		\end{tabular}
	\end{center}

	Если у нас лесенка без верхней ступени, то ответ равен $G_{n-1}(z-1)^2$, так как две висячие вершины красим независимо.
	Если же верхнюю ступень стянуть, получается $G_{n-1}(z-2)$, так как верхнюю вершину надо покрасить в один из $z-2$ цветов (смежные с ней обязательно имеют разный цвет).
	Пишем рекурренту:
	\[
		P(G_n) = G_{n-1}(z-1)^2 - G_{n-1}(z-2) = G_{n-1}(z^2-2z+1-z+2) = G_{n-1}(z^2-3z+3)
	\]
	Остановка произойдёт при $n=1$: $P(G_1)=z(z-1)$.
	Итого:
	\[
		P(G_n) = (z^2-3z+3)^{n-1}z(z-1)
	\]

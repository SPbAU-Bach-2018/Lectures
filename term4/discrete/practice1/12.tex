\chapter{Занятие 29.04.2016 (графы)}

\section{Разбор классной работы}
\subsection{Задача 1}
	Возьмём $P(z)$, подставим $z=1$.
	С одной стороны это сумма коэффициентов.
	С другой стороны это количество способов раскрасить граф в один цвет.
	Так можно раскрасить только пустой граф (и одним способом).
	Получили комбинаторное доказательство.

	Также можно доказывать рекурсивно по индукции.

\subsection{Задача 2}
	Уточнение условия:для всех графов, кроме деревьев, строго меньше (для деревьев равенство).

	Для начала разберёмся с деревом.
	Зафиксировали произвольный корень, его можно покрасить в $k$ цветов.
	Дальше красим по уровням, у каждой вершины покрашен только родитель, значит,
	её можно покрасить в $k-1$ цвет.
	Итого $k(k-1)^{n-1}$ раскрасок.

	Теперь покажем, что если к дереву добавить хотя бы одно ребро, то хотя бы одна раскраска станет некорректной.
	Если это ребро находится между вершинами на одном уровне, то сломается покраска дерева в два цвета (по уровням)
	Воспользуемся тем, что $k \ge 3$.
	Оставим раскраску дерева как раньше, но концы ребра покрасим в третий цвет. 
	Если же было $k=2$, то для части графов число покрасок останется.

\subsection{Задача 3}
	Возьмём центр колеса, не умаляя общности покрасим его в цвет $k$.
	Осталось посчитать число раскрасок цикла из $n$ вершин в $k-1$ цвет.

	Посчитаем сначала для пути $P(P_n)$.
	Комбинаторно: мы красим первую вершину пути в $z$ цветов, каждую из оставшихся "--- в $z-1$ цвет.
	Получаем при $n \ge 1$:
	\[
		P(P_n) = z(z-1)^{n-1}
	\]

	Теперь посчитаем для $C_n$ рекурсивно по формуле.
	Сначала напишем рекурренту при $n \ge 3$:
	\[
		P(C_n) = \underbrace{P(P_n)}_{C_n - e} - \underbrace{P(C_{n-1})}_{C_n\\e}
	\]
	При $n = 2$ имеем $P(C_2)=z(z-1)$.
	Удачно получилось, что $P(C_2)=P(P_2)$.
	Раскручиваем рекурренту (\TODO счёт):
	\[
		P(C_n) =
		\sum_{k=n}^{2} (-1)^{n-k}z(z-1)^{k-1} =
		(-1)^{n-1}\sum_{k=n}^{2} (-1)^{k-1}z(z-1)^{k-1} =
		(-1)^{n-1}\sum_{k=n-1}^{1} (-z(z-1))^{k} =
		(-1)^{n-1}\left(\frac{(-z(z-1))^n}{-z(z-1)-1} - 1\right) =
		(-1)^{n-1}\left((-1)^n\frac{z^n(z-1)^n}{z-z^2-1} - 1\right) =
		\t{???}
	\]

\subsection{Задача 4}
	Пусть лесенка состоит из $n$ слоёв, обозначим её за $G_n$.
	Если у нас лесенка без верхней ступени, то ответ равен $G_{n-1}(z-1)^2$, так как две висячие вершины красим независимо.
	Если же верхнюю ступень стянуть, получается $G_{n-1}(z-2)$, так как верхнюю вершину надо покрасить в один из $z-2$ цветов (смежные с ней обязательно имеют разный цвет).
	Пишем рекурренту:
	\[
		P(G_n) = G_{n-1}(z-1)^2 - G_{n-1}(z-2) = G_{n-1}(z^2-2z+1-z+2) = G_{n-1}(z^2-3z+3)
	\]
	Остановка произойдёт при $n=1$: $P(G_1)=z(z-1)$.
	Итого:
	\[
		P(G_n) = (z^2-3z+3)^{n-1}z(z-1)
	\]

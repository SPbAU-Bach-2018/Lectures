\chapter{Занятие 08.03.2016 (графы)}

\section{Разбор классной работы}
\subsection{Задача 1.2}
	Чтобы покрасить в два цвета, красим сначала целиком одну долю, а потом вторую:
	\[ x_1, x_2, \dots, x_n, y_1, y_2, \dots, y_n \]
	Первые $n$ вершин друг с другом не смежны, они будут покрашены в цвет один.
	Вершины второго блока смежны только с вершинами первого блока, они будут покрашены в цвет два.

	Чтобы покрасить в $n$ цветов, красим в таком порядке:
	\[ x_1, y_1, x_2, y_2, \dots, x_n, y_n \]
	Докажем по индукции, что вершины $x_i$ и $y_i$ будут покрашены в цвет $i$.
	База при $i=1$ очевидно.
	Переход: ясно, что при покраске $x_i$ или $y_i$ все покрашенные соседи
	будут иметь номера $y_1, \dots, y_{i-1}$ или $x_1, \dots, x_{i-1}$, соответственно,
	и покрашены в цвета $1, 2, \dots, i-1$.
	Тогда жадный алгорим обязан покрасить соответствующую вершину в цвет $i$.

\subsection{Задача 1.3}
	У нас в графе есть $K_4$, т.е. требуется хотя бы четыре цвета для покраски.
	Оказывается, что четырёх не хватает: давайте зафиксируем цвет вершины $x$.
	Тогда в каждой из двух клик наверху мы должны покрасить нижние три вершины в
	цвета 2, 3, 4, так как эти вершины смежны с $x$.
	Теперь посмотрим на каждую клику, в каждой осталась ровно одна непокрашенная вершина,
	но для них остался лишь один цвет.
	Таким образом, получаем, что две верхние вершины в каждой клике покрашены одинаково,
	но между ними есть ребро, противоречие.

	В пять цветов покрасить несложно: делаем то же самое, но одну из верхних вершин красим в пятый цвет.

\subsection{Задача 1.5}
	Рассмотрим цепочку из шести вершин (очевидно, его можно покрасить в два цвета):
	\[ v_1 \to v_2 \to v_3 \to v_4 \to v_5 \to v_6 \]
	Первымы должны быть покрашены вершины $v_2, \dots, v_5$ "--- они имеют степень пять.
	Мы имеем право выбрать им любой порядок.
	Давайте выберем такой:
	\[ v_2, v_5, v_3, v_4 \]
	Тогда вершины $v_2$ и $v_5$ будут покрашены в первый цвет (так как соседей не покрашено),
	вершина $v_3$ "--- во второй цвет (есть сосед $v_2$), а вот вершина $v_$ "--- в третий
	цвет (так как есть сосед $v_5$ первого цвета и сосед $v_3$ второго цвета).

\subsection{Задача 1.9}
	\TODO тут и далее прочитать и поправить

	Перепишем отношение:
	\begin{gather*}
		\frac{\xi(G)\cdot(\xi(G) - 1)}{2} \le m \\
		\binom{\xi(G)}{2} \le m \\
	\end{gather*}
	Идея такая: разобьём изначальный граф $G$ на блоки по цветам, всего $\xi(G)$ блоков.
	Стянем вершины в одних блоках, получим граф из $\xi(G)$ вершин (возможно, с мультирёбрами)
	и $m$ рёбер.
	При этом любые две вершины соединены хотя бы одним ребром, иначе эти два цвета можно объединить в один.
	Получаем, что рёбер должно быть хотя бы столько же, сколько пар блоков.

\subsection{Задача 1.10}
	Индукция по $m$.
	Если $m=1$, то всё просто "--- красим все вершины в разные цвета.
	Переход: пусть есть граф $G$ с $m$ блоками.
	Давайте удалим блок $B_m$, остался граф $G-B_m$, в котором $m-1$ блок.
	Его можно раскрасить в $n-|B_m|-(m-1)+1=n-|B_m|-m+2$ цветов, сделаем это.
	Вернём блок обратно, покрасим каждую вершину в новый цвет, всего получится $n-m+2$ цвета,
	надо один цвет куда-нибудь убрать.
	
	Пусть в нашем блоке есть вершины $x_1, \dots, x_k$.
	Давайте возьмём вершины $y_i$ "--- вершина из блока $B_i$, которая не смежна с какой-то вершиной из $x_j$.
	Получим $m-1$ различную вершину $y_1, \dots, y_{m-1}$, обозначим это множество за $Y$.
	Заметим, что если в $Y$ есть вершина с цветом $c$, который нигде в графе $G$ больше не встречается,
	можно покрасить соответствующую несмежную ей вершину из $B_m$ покрасить в цвет $c$ и получить раскраску в $n-m+1$ цвет.

	Теперь предположим, что такой вершины у нас нет.
	Рассмотрим граф $G-Y$, в нём $n-m+1$ вершина.
	Однако же в нём всё еще присутствуют все цвета, которые были в графе $G$, а их было $n-m+2$.
	Т.е. цветов больше, чем вершин, противоречие.

\section{Задача 1.11}
	Разобьём граф $G$ на $\xi(G)$ блоков по цветам.
	Любые два блока соединены ребром (иначе цвета можно стянуть).
	Соответственно, в $\bar G$ никакик два блока ребром связаны не будут.
	Значит, можно применить предыдущую задачу и получить, что $\bar G$ можно покрасить
	в $n - \xi(G) + 1$ цвет:
	\begin{gather*}
		\xi(\bar G) \le n - \xi(G) + 1 \\
		\xi(G) + \xi(\bar G) \le n + 1
	\end{gather*}

	Теперь вторая часть.
	Мы знаем, что $\xi(\bar G)$ не меньше, чем $\omega(\bar G)$ "--- размер максимальной клики в $G$.
	А $\omega(\bar G)=\alpha(G)$ (максимальное независимое множество в $G$).
	\[
		\xi(G) \cdot \xi(\bar G) \ge \xi(G) \cdot \omega(\bar G) = \xi(G) \cdot \alpha(G) \ge n
	\]
	\TODO оценить каждый блок цвета как $\alpha$, они же независимы

\section{Задача 1.12}
	Решение от Димы Розплохаса: возьмём обход DFS, разные уровни "--- в разные цвета.
	Глубина дерева не больше самого длинного пути, ура.

\chapter{Занятие 27.05.2016 (графы)}

\section{Разбор классной работы}
\subsection{Задача 1}
	Есть почти колесо на 8 вершинах.
	Стянули вершины 2 и 3; 6 и 7, получили $K_{3,3}$

\subsection{Задача 2}
	Показать, что граф $K_4$ не внешнепланарный.
	От противного: нарисовали во внешней грани вершину, соединили её со всеми вершинами,
	получили планарный $K_5$, упс.

	Показать, что граф $K_{2,3}$ не внешнепланарный.
	От противного: нарисовали во внешней грани вершину, соединили её с вершинами второй доли,
	получили планарный $K_{3,3}$, упс.

	Покажем, если если в графе есть минор $K_4$ или $K_{2,3}$, то он не внешнепланарен.
	Для этого надо показать, что от стягивания/удаления рёбер или вершин внешнепланарность не портится.
	\TODO

	Покажем в обратную сторону от противного: пусть в граф нет ни $K_4$, ни $K_{2,3}$.
	Добавим в него вершину и соединим её со всеми.
	Получим граф без $K_5$ и без $K_{3,3}$, он планарен.
	\TODO


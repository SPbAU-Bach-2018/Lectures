\chapter{Занятие 22.04.2016 (графы)}

\section{Разбор домашней работы}
\subsection{Задача 6}
	Цепь в графе сравнимости "--- это просто клика.
	С использованием теоремы Мирского: размер максимальной цепи (что в точности есть $\omega(G)$)
	равен минимальному количеству антицепей (независимых множеств), на которые можно разбить (что есть $\chi(G)$).
	Любой индуцированный подграф тоже является графом сравнимости, так что для него тоже $\omega=\chi$.
	Альтернативно можно через Дилуорса, но тогда надо к дополнению перейти.

	Без теорем \textbf{не доказали}.

\section{Разбор куска классной работы II подгруппы}
\subsection{Задача ???}
	Рассмотрим декартово произведение графов $G$ и $H$, $G \square H$.
	В нём вершины кодируются парами $(x, y)$, где $x \in G$, $y \in H$.
	Ребро между $(x_1, y_1)$ и $(x_2, y_2)$ есть тогда и только тогда, когда верно одно из следующего:
	\begin{enumerate}
		\item $x_1=x_2$ и в $H$ есть ребро между $y_1$ и $y_2$.
		\item $y_1=y_2$ и в $G$ есть ребро между $x_1$ и $x_2$.
	\end{enumerate}
	Давайте поймём, как выглядит граф $P_n \square K_2$.
	Это такая лесенка.

	Давайте поймём, во сколько цветов можно раскрасить рёбра графа $C_n \square K_2$.
    Это трёхмерное колесо, в нём есть вершины степени три.
    Значит, хотя бы в три.
	Но если $n$ чётно, то в три точно можно.
	Если нечётно, тоже можно: красим верхний и нижний циклы одинаково, потом аккуратно разбираемся с гранью,
	где получились рёбра цвета три.

\subsection{Задача ???}
	Есть $G \square H$, причём мы знаем, что $\xi'(H) = \Delta(H)$.
	Надо посчитать $\xi'(G \square H)$.

	Про граф $G$ мы знаем, что его можно покрасить не более чем в $\Delta(G) +1$ цвет.
	Давайте точно так же покрасим все $|H|$ копий графа $G$.
	Теперь красим копии графа $G$ "--- каждая копия соединяет копии вершин.
	Каждую копию можно покрасить в $\Delta(H)$, итого будет $\Delta(G)+\Delta(H)+1$ цветов.
	Но один можно сберечь: мы в каждой вершине $G$ сберегли цвет $c$, он же будет сохранён одинаковый
	во всех копиях $G$, поэтому его можно переиспользовать при покраске соответствующего $H$.
	Получили, что на самом деле можно в $\Delta(G)+\Delta(H)$ цветов, а это нижняя оценка.

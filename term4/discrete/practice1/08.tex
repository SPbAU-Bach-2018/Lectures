\chapter{Занятие 01.04.2016 (комбинаторика)}
\section{Разбор домашней работы}
Она же "--- задачи с прошлой классной работы, кроме задач 1 и 3.

\subsection{Задача 1}
	Мы знаем производящую функцию для чисел Белла:
	\begin{gather*}
		e^{e^z-1} =
		\frac1e e^{e^z} =
		\frac1e \sum_{k=0}^{\infty} \frac{e^{kz}}{k!} =
		\frac1e \sum_{k=0}^{\infty} \sum_{n=0}^{\infty} \frac{(kz)^n}{n!} \cdot \frac{1}{k!} =
		\frac1e \sum_{n=0}^{\infty} \sum_{k=0}^{\infty} \frac{(kz)^n}{n!} \cdot \frac{1}{k!} =
		\frac1e \sum_{n=0}^{\infty} \frac{z^n}{n!} \sum_{k=0}^{\infty} \frac{k^n}{k!}
	\end{gather*}
	Что и требовалось показать.

\subsection{Задача 2}
	Разбираемся с действиями над отдельной группой.
	Разрешаем только чётные степени (большие нуля), коэффициент "--- $(2k)!$ (все перестановки $2k$ элементов):
	\[
		F(z) \coloneq \sum_{k=1}^\infty (2k)! \cdot \frac{z^{2k}}{(2k)!} = \sum_{k=1}^\infty z^{2k} = \frac{1}{1-z^2} - 1
	\]
	Сами стопки тоже надо упорядочить:
	\[
		G(z) \coloneq \sum_{k=0}^\infty k! \cdot \frac{z^k}{k!} = \frac{1}{1 - z}
	\]
	Теперь берём композицию:
	\[
		G(F(z)) = \frac{1}{1-\frac{1}{1-z^2}+1} = \frac{1}{\left(\frac{2-2z^2-1}{1-z^2}\right)} = \frac{1-z^2}{1-2z^2}
	\]

\subsection{Задача 3}
	Для элементов внутри блока:
	\[
		F(z) \coloneq \sum_{k=1}^\infty \frac{k!}{k} \cdot \frac{z^k}{k!} = -\ln(1-z) - 1
	\]
	Для блоков:
	\[
		G(z) \coloneq \sum_{k=0}^\infty k \cdot \frac{z^k}{k!}
	\]

\subsection{Задача 4}
	Была на лекции.

\subsection{Задача 5}
	Так же, как и все остальные, плюс счёт.

\subsection{Задача 6}
	Так же, как и полиномы Белла.

\section{Разбор классной работы}
\subsection{Задача 1}
	Решается как угодно.
	Можно подставить в лоб и посчитать, а можно по формуле вроде (a_1/k_1!)^{k_1} * ...

\subsection{Задача 2}
	Решается так же, как и полиномы Белла, и как последняя задача из домашнего задания.
	Давайте напишем ЭПФ, которая характеризует действия внутри каждого блока (перестановка его элементов,
	причём элементов больше нуля):
	\[ G(z) \eqcolon \sum_{n=1}^\infty n! \frac{z^n}{n!} = \frac{1}{1-z} - 1 = \frac{z}{1-z} \]
	А с группой мы комбинаторных действий не делаем, зато навешиваем <<метку>> $t^k$:
	\[ F_t(z) \eqcolon \sum_{n=0}^\infty t^k\frac{z^k}{n!} = e^{tz} \]
	Теперь делаем композицию двух ЭПФ "--- это мы разобъём $n$ элементов на блоки,
	каждый упорядочим, а потом сгруппируем разбиения по количеству блоков при помощи метки $t^k$.
	Из комбинаторных соображений получаем:
	\begin{gather*}
		F_t(G(z)) = \sum_{n=0}^{\infty} \frac{z^n}{n!} \sum_{k=0}^\infty t^k L_{n, k}
	\end{gather*}
	А теперь раскладываем композицию по формулам:
	\begin{align*}
		F_t(G(z)) &=
			F_t\left(\frac{z}{1-z}\right) =
			e^{\frac{tz}{1-z}} =
			\sum_{k=0}^\infty \frac{t^kz^k}{(1-z)^kk!} = \\
			&= \sum_{k=0}^\infty \frac{t^k}{k!} \cdot \frac{z^k}{(1-z)^k} =
			\sum_{k=0}^\infty \frac{t^k}{k!} \cdot \sum_{n=k}^\infty z^n \binom{n-k+k-1}{k-1} = \\
			&= \TODO
	\end{align*}
	Последний переход "--- это мы просто вспомнили, что есть деление на $(1-z)$ "--- это
	суммирование на префиксах, отсюда вылезает биномиальный коэффициент.

\subsection{Задача 4}
	Что значит, что перестановка в степени $r$ даёт идентичную?
	Это значит, что $r$ делится на длину каждого цикла перестановки.
	Получим ответ, будет сумма по делителям, дальше не сворачивается.

\subsection{Задача 5}
	Тут делается похоже, но получаем почти логарифм.
	Получим sqrt((1-z)/(1+z)).

\chapter{Занятие 12.02.2016 (графы)}
\section{Классная работа}
\subsection{Задача 2}
	Заметим, что у каждой вершины $k$-связного графа степень хотя бы $k$.
	Значит, $\sum \deg v \ge nk$.
	А так как $\sum\deg v = 2|E|$, то $|E| \ge \frac{nk}{2}$, что и требовалось
	(округление вверх появляется, так как рёбер целое число).

\subsection{Задача 1}
	\subsubsection{Решение через первую}
		От противного: пусть некоторый граф на $n$ вершинах вершинно $(n-1)$-связен.
		Значит, по первой задаче в нём хотя бы $\frac{n(n-1)}{2}$ рёбер, то есть
		граф является полным.

	\subsubsection{Альтернативное решение}
		Так как граф не является полным, то есть хотя бы две несмежные вершины: $x$ и $y$.
		Возьмём множество размера $n-2$: все вершины, кроме $x$ и $y$.
		Оно является разделяющим, значит, $\kappa(G) \le n - 2$.

\subsection{Задача 4}
	Возьмём два графа $K_{\delta+1}$: $A$ и $B$, ответом будет $A+B$ с добавленными рёбрами.
	Выберем в $A$ произвольные $\kappa$ вершин: $x_1, \dots, x_\kappa$,
	а в $B$ "--- произвольные $\lambda$ вершин $y_1, \dots, y_\kappa, \dots, y_\lambda$.
	Проведём рёбра $x_i \to y_i$ для $1 \le i \le \kappa$, а также
	рёбра $x_1 \to y_i$ для $\kappa < i \le \lambda$.

	Очевидно, минимальная степень в этом графе будет действительно $\delta$.

	Поймём, что граф вершинно $\kappa$-связен и не более.
	В самом деле: есть вершинный разрез мощности $\kappa$: удаляем все $x_i$
	причём так как $\kappa < \delta$, в половинке $A$ останутся вершины.
	Разреза меньшей мощности быть не может, покажем это:
	\begin{enumerate}
		\item Если мы удаляем из половинки $A$ не более $\delta-1$ вершины (а это, очевидно, так), оставшиеся вершины индуцируют непустой связный подграф, так как граф $K_{\delta+1}$ является $\delta$-связным.
		\item Аналогично с половинкой $B$.
		\item Если рассмотреть рёбра паросочетания $x_i \to y_i$, то из них хотя бы одно останется в графе (так как рёбер $\lambda>\kappa$).
		\item Таким образом получаем, что оставшийся после удаления менее чем $\kappa$ вершин граф связен.
	\end{enumerate}

	Поймём, что граф рёберно $\lambda$-связен и не более.
	В самом деле: рёберный разрез мощности $\lambda$ есть: удаляем все рёбра между $A$ и $B$.
	Разреза меньшей мощности быть не может по рассуждениям, аналогичным предыдущим:
	\begin{enumerate}
		\item Половинки $A$ и $B$ являются рёберно $\delta$-связными, значит, при удалении из каждой не менее чем $\lambda<\delta$ рёбер они остаются связными
		\item Также остаётся хотя бы одно ребро между половинками $A$ и $B$
		\item Стало бы, после удаления менее чем $\lambda$ рёбер граф остаётся связным.
	\end{enumerate}
	
\subsection{Задача 3}
	Из четвёртой задачи:

	\begin{center}
		\neatograph[scale=0.6]{g01p3}{
			A1[label="."] [pos="0,2.5!"]; A2[label="."] [pos="0.5,1.5!"]; A3[label="."] [pos="0,0.5!"]; A4[label="x1"][pos="2,2!"]  ; A5[label="x2"][pos="2,1!"]  ;
			B1[label="y1"][pos="4,2!"]  ; B2[label="y2"][pos="4.5,1!"]  ; B3[label="y3"][pos="4,0!"]  ; B4[label="."] [pos="6,1.5!"]; B5[label="."] [pos="6,0.5!"];
			A1 -- A2 -- A3 -- A4 -- A5 -- A1;
			A1 -- A3 -- A5 -- A2 -- A4 -- A1;
			B1 -- B2 -- B3 -- B4 -- B5 -- B1;
			B1 -- B3 -- B5 -- B2 -- B4 -- B1;
			A4 -- B1;
			A5 -- B2;
			A4 -- B3;
		}
	\end{center}

\subsection{Задача 5}
	Так как $\delta \le \Delta \le 3$, то мы точно знаем, что $\kappa \le \lambda \le 3$.
	Покажем, что $\lambda \le \kappa$, откуда будет следовать $\lambda = \kappa$.
	Возьмём какое-нибудь минимальное по размеру разделяющее множество вершин $A$.
	Оно разделило граф на две непустые компоненты: $S_1$ и $S_2$.

	Заметим, что если в $A$ есть некоторая вершина $v$, не соединённая ребром либо с $S_1$, либо с $S_2$,
	то её можно из $A$ исключить "--- если в граф $G - A$ не существовало пути между $S_1$ и $S_2$,
	то в графе $G - (A \setminus \{v\})$ такого пути тоже быть не может "--- вершина $v$ относится только к одному из $S_1$, $S_2$.
	Значит, по минимальности $A$ каждая вершина имеет хотя бы ребро, идущее в $S_1$ и идущее в $S_2$.
	Тогда каждая вершина из $A$ может иметь не более одного ребра, идущего в $A$, т.е. подграф $G$, индуцированный множеством $A$
	есть паросочетание (необязательно совершенное или непустое).

	Разберём несколько случаев:
	\begin{itemize}
		\item
			$|A|=1$.
			В этом случае есть единственная вершина, у неё не более трёх рёбер, каждое из которых идёт либо в $S_1$, либо в $S_2$.
			Очевидно, что в хотя бы одну компоненту идёт не более одного ребра, его и возьмём в искомое рёберно разделяющее множество.
		\item
			$|A|=2$.
			Заметим, что если вершины между собой не соединены, то можно удалить по одному ребру у каждой аналогично предыдущему рассуждению, граф распадётся.
			Если же они соединены, то из каждой из них идёт ровно одно ребро в $S_1$ и ровно одно ребро в $S_2$.
			Удалим все рёбра из $A$ в $S_1$ "--- их ровно два, после этого $S_1$ и $S_2$ перестанут быть связны.
			\begin{Rem}
				Важно, что мы удалём рёбра с одной сторны, в противном случае могло бы и не получиться.
				Например, в примере ниже надо удалить либо все рёбра, смежные с $S_1$, либо все, смежные с $S_2$:
				\begin{center}
					\neatograph[scale=0.6]{g01p5}{
						S1[pos="0,1!"];
						x[pos="1,2!"]; y[pos="1,0!"];
						S2[pos="2,1!"];
						S1 -- x -- S2;
						S1 -- y -- S2;
						x -- y;
					}
				\end{center}
			\end{Rem}
		\item
			$|A|=3$.
			Так как подграф, индуцированный $A$, является паросочетанием, то в нём есть хотя бы одна изолированная вершина.
			Из неё можно удалить одно ребро и она перестанет связывать $S_1$ и $S_2$.
			С оставшейся парой можно разобраться так же, как в предыдущем пункте, тогда $S_1$ и $S_2$ тоже перестанут быть связными.
	\end{itemize}
	
\subsection{Задача 6}
	Следует из пятой задачи

\subsection{Задача 7}
	Так как $\kappa(G_3)=3$, то $|G_3|>3$, т.е. хотя бы четыре вершины.
	Граф $K_4$ подходит "--- он три-регулярный и является 3-связным (как полный граф).

	Граф $G_2$ должен иметь чётное число вершин (так как 3-регулярный), 4 вершины
	иметь не может (так как получим $K_4$).
	Шесть вершин иметь тоже не может: от противного.
	Пусть у нас 6 вершин, и 3 ребра из каждой, всего девять рёбер.
	Когда мы удалим две вершины, у нас пропадёт не более шести рёбер,
	останется граф на четырёх вершинах с минимум трёмя рёбрами,
	при этом степень каждой из оставшихся вершин будет хотя бы $3-1=1$.
	А единственный несвязный граф с $|V|=4$ и $|E|=3$ является треугольником с изолированной вершиной, не подходит.

	Теперь пример для восьми вершин:

	\begin{center}
		\neatograph[scale=0.6]{g01p7}{
			1[pos="0.5,1.5!"]; 2[pos="0,0.75!"]; 3[pos="1,0.75!"]; 4[pos="0.5,0!"];
			5[pos="2.5,0!"]; 6[pos="3,0.75!"]; 7[pos="2,0.75!"]; 8[pos="2.5,1.5!"];
			1 -- 2 -- 3 -- 4 -- 5 -- 6 -- 7 -- 8 -- 1;
			1 -- 3;
			2 -- 4;
			5 -- 7;
			6 -- 8;
		}
	\end{center}

	Этот граф двусвязен, так как в нём есть Гамильтонов цикл.

\subsection{Задача 9}
	У нас есть какие-то точки сочленения и блоки (блоков $k$ штук, точек сочленения $m$),
	в блоке $i$ лежит $B_i$ вершин.
	Каждый блок включает все входящие в него точки сочленения.
	Каждые два блока пересекаются не более чем по одной точке сочленения.
	Просуммируем размер всех блоков: $\sum B_i$.
	Заметим, что мы получили почти размер графа, но каждую точку сочленения $v$
	посчитали столько раз, в скольки блоках она есть.

	Можно нарисовать такое двудольное дерево: в одной доле будут блоки,
	а в другой "--- точки сочленения; ребро между блоком и точкой проводится,
	если точка лежит в блоке.
	Подвесим это дерево за блок.
	Теперь в нём по уровням чередуются блоки и точки сочленения.
	Нас интересует сумма по всем точкам сочленения: степень соответствующей вершины в дереве минус один.
	Возьмём из каждого блока (кроме корневого) ребро наверх, в точку сочленения, закрасим его.
	В суммые мы закрасили $k-1$ ребро (для всех блоков, кроме одного).
	Закрасили ровно требуемое количество рёбер: для каждой точки сочленения закрашены все рёбра
	вниз и не закрашено ребро наверх (к корню).

	Итого ответ: $\sum_{i=1}^k B_i - (k - 1)$.

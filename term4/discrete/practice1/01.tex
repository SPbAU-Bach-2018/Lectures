\chapter{Занятие 12.02.2016 (графы)}
\section{Контрольная работа}
\subsection{Задача 2}
	Заметим, что у каждой вершины $k$-связного графа степень хотя бы $k$.
	Значит, $\sum \deg v \ge nk$.
	А так как $\sum\deg v = 2|E|$, то $|E| \ge \frac{nk}{2}$, что и требовалось
	(округление вверх появляется, так как рёбер целое число).

\subsection{Задача 1}
	\subsubsection{Решение через первую}
		От противного: пусть некоторый граф на $n$ вершинах вершинно $(n-1)$-связен.
		Значит, по первой задаче в нём хотя бы $\frac{n(n-1)}{2}$ рёбер, то есть
		граф является полным.

	\subsubsection{Альтернативное решение}
		Так как граф не является полным, то есть хотя бы две несмежные вершины: $x$ и $y$.
		Возьмём множество размера $n-2$: все вершины, кроме $x$ и $y$.
		Оно является разделяющим, значит, $\kappa(G) \le n - 2$.

\subsection{Задача 4}
	Возьмём два графа $K_{\delta+1}$: $A$ и $B$, ответом будет $A+B$ с добавленными рёбрами.
	Выберем в $A$ произвольные $\kappa$ вершин: $x_1, \dots, x_\kappa$,
	а в $B$ "--- произвольные $\lambda$ вершин $y_1, \dots, y_\kappa, \dots, y_\lambda$.
	Проведём рёбра $x_i \to y_i$ для $1 \le i \le \kappa$, а также
	рёбра $x_1 \to y_i$ для $\kappa < i \le \lambda$.
	\TODO
	
\subsection{Задача 3}
	Из четвёртой задачи:

	\begin{center}
		\neatograph[scale=0.6]{g01p3}{
			A1[pos="0,2!"]; A2[pos="0,1!"]; A3[pos="0,0!"]; A4[pos="1,1.5!"]; A5[pos="1,0.5!"];
			B1[pos="2,2!"]; B2[pos="2,1!"]; B3[pos="2,0!"]; B4[pos="3,1.5!"]; B5[pos="3,0.5!"];
			A1 -- A2 -- A3 -- A4 -- A5 -- A1;
			A1 -- A3 -- A5 -- A2 -- A4 -- A1;
			B1 -- B2 -- B3 -- B4 -- B5 -- B1;
			B1 -- B3 -- B5 -- B2 -- B4 -- B1;
			A4 -- B1;
			A5 -- B2;
			A4 -- B3;
		}
	\end{center}

\subsection{Задача 5}
	\TODO
	
\subsection{Задача 6}
	Следует из пятой задачи

\subsection{Задача 7}
	Так как $\kappa(G_3)=3$, то $|G_3|>3$, т.е. хотя бы четыре вершины.
	Граф $K_4$ подходит "--- он три-регулярный и является 3-связным (как полный граф).

	Граф $G_2$ должен иметь чётное число вершин (так как 3-регулярный), 4 вершины
	иметь не может (так как получим $K_4$).
	Шесть вершин иметь тоже не может: от противного.
	Пусть у нас 6 вершин, и 3 ребра из каждой, всего девять рёбер.
	Когда мы удалим две вершины, у нас пропадёт не более шести рёбер,
	останется граф на четырёх вершинах с минимум трёмя рёбрами,
	при этом степень каждой из оставшихся вершин будет хотя бы $3-1=1$.
	А единственный несвязный граф с $|V|=4$ и $|E|=3$ является треугольником с изолированной вершиной, не подходит.

	Теперь пример для восьми вершин:

	\begin{center}
		\neatograph[scale=0.6]{g01p7}{
		    1[pos="0.5,1.5!"]; 2[pos="0,0.75!"]; 3[pos="1,0.75!"]; 4[pos="0.5,0!"];
	    	5[pos="2.5,0!"]; 6[pos="3,0.75!"]; 7[pos="2,0.75!"]; 8[pos="2.5,1.5!"];
			1 -- 2 -- 3 -- 4 -- 5 -- 6 -- 7 -- 8 -- 1;
			1 -- 3;
			2 -- 4;
			5 -- 7;
			6 -- 8;
		}
	\end{center}

	Этот граф двусвязен, так как в нём есть Гамильтонов цикл.

\subsection{Задача 9}
	У нас есть какие-то точки сочленения и блоки (блоков $k$ штук, точек сочленения $m$),
	в блоке $i$ лежит $B_i$ вершин.
	Каждый блок включает все входящие в него точки сочленения.
	Каждые два блока пересекаются не более чем по одной точке сочленения.
	Просуммируем размер всех блоков: $\sum B_i$.
	Заметим, что мы получили почти размер графа, но каждую точку сочленения $v$
	посчитали столько раз, в скольки блоках она есть.

	Можно нарисовать такое двудольное дерево: в одной доле будут блоки,
	а в другой "--- точки сочленения; ребро между блоком и точкой проводится,
	если точка лежит в блоке.
	Подвесим это дерево за блок.
	Теперь в нём по уровням чередуются блоки и точки сочленения.
	Нас интересует сумма по всем точкам сочленения: степень соответствующей вершины в дереве минус один.
	Возьмём из каждого блока (кроме корневого) ребро наверх, в точку сочленения, закрасим его.
	В суммые мы закрасили $k-1$ ребро (для всех блоков, кроме одного).
	Закрасили ровно требуемое количество рёбер: для каждой точки сочленения закрашены все рёбра
	вниз и не закрашено ребро наверх (к корню).

	Итого ответ: $\sum_{i=1}^k B_i - (k - 1)$.

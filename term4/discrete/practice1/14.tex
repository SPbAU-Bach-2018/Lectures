\chapter{Занятие 11.05.2016 (комба)}

\section{Разбор классной работы}
\section{Задача 2}
\subsection{Первая попытка}
	У нас точно есть одно тождественное преобразование.
	Также вокруг каждой из трёх осей есть три нетривиальных поворота, всего девять.
	Ещё бывают симметрии вдоль каждой плоскости, всего три.
	Ещё бывают симметрии вдоль плоскостей, проходящих через диагонали граней,
	их три направления, в каждом направлени "--- две, всего шесть.

	Получилось всего 19 движений, но это мало, мы ещё не учли всякие композиции.

\subsection{Вторая попытка}
	Давайте пронумеруем грани.
	Посмотрим, куда перейдёт первая грань "--- в одну из шести остальных.
	А потом, когда уже одна грань зафиксирована, остаётся ещё четыре смежных с ней,
	которые можно двигать по кругу и отражать (как в движениях квадрата), всего восемь.

	Итого $6 \cdot 8 = 48$.

\section{Задача 3}
	Можно уменьшать перебор с 48 элементов до 24 следующим образом: если первая грань переезжает
	в смежную с ней, то все эти четыре варианта равнозначны.
	Восемь вариантов получалось в задаче 1.

\section{Задача 5}
	
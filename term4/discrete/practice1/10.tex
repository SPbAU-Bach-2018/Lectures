\chapter{Занятие 15.04.2016 (графы)}

\section{Разбор домашней работы}
Первые шесть задач записаны не с разбора

\subsection{Задача 1.1}
	Можно перерисовать и покрасить граф так (отсюда сразу $\chi(G) \le 4$):
	\begin{center}
		\begin{dot2tex}[scale=0.8,options=-tmath]
			graph G {
				node [color=yellow]; 1
				node [color=red]; 6 5
				node [color=green]; 2 7
				node [color=blue]; 3 4
				1 -- {2 -- 3; 4 -- 5}
				{2; 3} -- 6 -- 7 -- {4; 5}
				{ rank=same; 2 3 4 5 }
				{ rank=same; 6 7 }
			}
		\end{dot2tex}
	\end{center}
	Так как в нём есть треугольники, то $\chi(G)\ge \omega(G) \ge 3$.
	\TODO

\subsection{Задача 1.7}
	Тут критический граф.
	Левое неравенство через задачу 1.6

	Правое неравенство "--- пишем сумму всех степеней:
	$2m \ge n\delta$
	$2m \ge (k-1)n$
	$2m/n >= xi(G) - 1$

\subsection{Задача 1.8}
	Берём $k$-критический подграф, в нём степени вершин хотя бы $k-1$ и в нём хотя бы $k$ вершин.

\subsection{Задача 1.9}
	Взяли $k$-критический подграф из $t$ вершин, в нём у каждой вершины хотя бы $k-1$ ребро.
	Кроме подграфа есть еще $n-k$ вершин, а весь граф был связен, значит, кроме критического подграфа есть еще хотя бы $n-t$ рёбер.
	Итого хотя бы $t(k-1)+n-t$ рёбер.
	Мы также знаем, что $t \ge k$.
	Значит, в сумме хотя бы $\binom{k}{2}+n-k$ рёбер.

\section{Разбор классной работы}
\subsection{Задача 1}
	Пусть у нас есть две смежные вершины: $x$ и $y$, а их степени "--- $d_x$ и $d_y$.
	Тогда они по принципу Дирихле образуют хотя бы $d_x + d_y - n$ треугольников.
	Просуммируем по всем рёбрам, получаем оценку на число треугольников снизу:
	\[
		\sum_{(x,y)\in E} d_x + d_y - n = \sum_{x \in V} d_x^2 - nm \ge
	\]
	Оцениваем первое слагаемое по неравенству о средних (\TODO):
	\[
		\ge \frac{1}{n} \left(\sum d_x \right)^2 - nm =
		\frac{4m}{n} - nm
	\]
	Дальше /3 вылезает, потому что мы посчитали каждый треугольник три раза (упорядоченно)

\subsection{Задача 4}
	Слева мы считаем тройки вершин $(a, b, c)$ (порядок $b$ и $c$ не важен), причём есть рёбра $a \to b$, $a \to c$.
	\TODO
	Пример, когда неправда, когда неравенство нестрогое: $K_{2,l-1}$.

\subsection{Задача 2}
	От противного: пусть обхват графа хотя бы пять.
	Возьмём вершину $x$ степени $d_x$, возьмём всех её соседей.
	Обхват хотя бы три, значит, они все между собой попарно не связны.
	Более того, так как обхват не четыре, ни у каких двух из них нет общих соседей.
	Нарисуем такое двухуровневое дерево, посчитаем число листьев, оно будет равно
	\[ \sum_{(x, y_i) \in E} d_{y_i}-1 \le n - d_x - 1 \] % справа - из комбы
	Просуммируем по всем $x$:
	\begin{gather*}
		\sum_{x \in V} d_x^2 - 2m \le n^2 - 2m - n \\
		\sum_{x \in V} d_x^2 \le n^2 - n = n(n-1) \\
		\sum_{x \in V} d_x^2 \ge \frac{4m^2}{n} \\
		\frac{4m^2}{n} \le n(n-1) \\
		4m^2 \le n^2(n-1) \\
		2m \le n\sqrt{n-1} \\
		m \le \frac12 n\sqrt{n-1} \\
	\end{gather*}

\subsection{Задача 3}
	Подсказка: можно так же оценивать количества вершин слоями.

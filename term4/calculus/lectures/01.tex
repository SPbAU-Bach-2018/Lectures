\setauthor{Дмитрий Лапшин}

\setcounter{chapter}{6}
\chapter{Поверхностные интегралы (продолжение)}
\setcounter{section}{4}
\section{Замкнутые и точные формы}

Пусть $G$ "--- область в $\R^2$, и есть форма $\omega = P \d x + Q \d y$.
Тогда
\[ \d \omega = \left(\partd Qx - \partd Py \right) \d x \wedge \d y \]

\begin{Def}
	Форма $\omega$ называется замкнутой, если $\d \omega = 0$.
	В нашем частном случае, это означает, что во всех точках $G$
	\[ \partd Qx = \partd Py \]
\end{Def}

\begin{Def}
	Форма $\omega$ называется точной, если существует её первообразная $F\colon G \ra \R$
	\[ \d F = \omega \]
\end{Def}

\begin{Rem}
	Напомним: первообразная существует тогда и только тогда, когда интеграл по замкнутой кривой всегда равен нулю.
\end{Rem}

\begin{Def}
	Форма $\omega$ называется локально точной, если у каждой точки есть окрестность, в оорой существует первообразная.
\end{Def}

\begin{theorem}
	$\omega = P \d x + Q \d y$, $P$ и $Q$ непрерывно дифференцируемы.
	Тогда замкнутость равносильна локальной точности.
\end{theorem}
\begin{proof}\begin{description}
\item[$\La$:]
	Из локальной точности следует, что в окрестности каждой точки существует первообразная
	\begin{gather*}
		\d F = \omega \\
		\partd Fx \d x + \partd Fy \d y = \omega = P \d x + Q \d y
	\end{gather*}
	Мы хотим:
	\begin{gather*}
		\partd Fx = P \quad \partd Qy = Q \\
		\partd Py = \partdd Fyx \quad \partd Qx = \partdd Fxy
	\end{gather*}
	Последние два совпадают, так как считают производную непрерывно дифференцируемой.

\item[$\Ra$:]
	Хотим в $B_r$ показать первообразную.
	Возьмём кусрчногладкую замнкутую несамопересекающуюся кривую $\gamma$ внутри шарика.
	Она ограничит внутри какую-то область $D$, тогда по формуле Грина:
	\[ \int\limits_\gamma \omega = \int\limits_\gamma P \d x + Q \d y = \int\limits_D \left( \partd Qx - \partd Py \right) \]
	А это ноль из замкнутости, успех.
\end{description}\end{proof}

\begin{conseq}
	Замкнутая форма в любом круге имеет первообразную.
\end{conseq}

\begin{exmp}
	Замкнутости не хватает для точности:
	\begin{gather*}
		\omega\colon \R^2 \setminus \{(0,0)\} \ra \R \\
		\omega = \frac{x \d y - y \d x}{x^2 + y^2}
	\end{gather*}
	Она замкнутая, можно проверить.
	Но не точная: возьмём кривую
	\begin{gather*}
		\ointctrclockwise\limits_{x^2 + y^2 = 1} \frac{x \d y - y \d x}{x^2 + y^2}
		= \int\limits_0^{2\pi} \left(\cos t \sin' t - \sin t \cos' t\right)
		= 2\pi
	\end{gather*}
\end{exmp}

\begin{Def}[Первообразная вдоль пути]
	$\omega$ "--- замкнутая форма в $G$, путь $\gamma\colon [a,b] \ra G$.
	$f$ называется первооразной $\omega$ вдоль $\gamma$, если $f\colon [a, b] \ra \R$,
	и для каждой $\tau \in [a, b]$ существует её окрестность (в $[a,b]$), и в соотвествущей в $G$ окрестности первообразная формы $F$ выражается как
	\[ f(t) = F(\gamma(t)) \]
\end{Def}

\begin{theorem}
	$\omega$ "--- замкнутая форма в $G$, $\gamma\colon [a, b] \ra G$.
	Тогда с точностью до константы существует единственная первообразная $\omega$ вдоль $\gamma$.
\end{theorem}
\begin{proof}\begin{description}
\item[Единственность:]
	Пусть $f_1$ и $f_2$ "--- две первообразные дволь пути.
	Возьмём $\tau \in [a,b]$.
	\[ f_1 = F_1 \circ \gamma \quad f_2 = F_2 \circ \gamma \]
	Мы про простые первообразные знаем, что $F_2 = F_1 + C$, откуда $f_2 = f_1 + C$.

	Показали, что в каждой окрестности отличаются на какую-то константу.
	Теперь возьмём отрезок $[a, b]$ "--- компакт, покрытый окрестностями точек.
	Значит, есть конечное подпокрытие, а в каждой области есть константа.
	Идём из окрестности в окрестность по пересечению, получим, что везде константы одинаковые.

\item[Существование:]
	Рассмотрим кривую $\gamma$, покроем её окрестностями точек, лежащих на ней.
	Сама кривая $\gamma([a, b])$ "--- компакт (непрерывный образ компакта).
	Тогда можно взять конечное подпокрытие кривой $U_i$.
	Теперь хотим просто взять на отрезке покрытие интервалами, которые все вместе перейдут в подпокрытие кривой.
	Для этого возьмём все прообразы $U_i$, получим много интервалов (много, так как прообраз одного $U_i$ не обязательно один интервал).
	Но мы получили покрытие интервалами отрезка.
	Возьмём ещё раз конечное подпокрытие из $k$ интервалов.

	Получили конечное множество интервалов на $[a,b]$.
	Если идти слева направо, то интервалы пересекаются.
	Давайте построим последовательность точек: $t_0 = a$, $t_k = b$, $t_i$ ставим в пересечении $i$ и $i+1$ интервала слева направо.

	Теперь начнём строить первообразную.
	На каждом отрезке $[t_i,t_{i+1}]$ введём $f(t) = F_i(\gamma(t))$.
	Так можно, поскольку каждый такой отрезок покрывается одним интервалом, который пришёл из какого-то $U_i$, а там была первообразная.

	Теперь осталось согласовать $f$ в точках $t_1,\dots,t_{k-1}$.
	Вспомним, что первообразные можно поправлять на константу, поправим сначала в первой точке, потом во второй и так далее.
	Получим уже честную функцию $f$.
\end{description}\end{proof}

\begin{conseq}
	$\omega$ "--- замкнутая форма в $G$, $\gamma\colon [a, b] \ra G$, $f$ "--- первообразная вдоль пути.
	Тогда
	\[ \int\limits_\gamma \omega = f(b) - f(a) \]
\end{conseq}
\begin{proof}
	Взяли первообразную, разбили на отрезки из доказательства.
	\[
		\int\limits_\gamma
		= \sum_{i=0}^{k-1} \int\limits_{\gamma|_{[t_i,t_{i+1}]}} \omega
		= \sum_{i=0}^{k-1} \left( F_{i+1}(\gamma(t_{i+1})) - F_i(\gamma(t_i)) \right)
		= \dots
	\]
	Вспомним, что мы сделали, чтобы $f$ была корректна в концах отрезков, откуда
	\[
		\dots
		= F_k(\gamma(t_k)) - F_0(\gamma(t_0))
		= f(b) - f(a)
	\]
\end{proof}

\begin{Rem}
	Так можно ввести интеграл по негладким кривым!
\end{Rem}

\begin{Def}[Гомотопные пути]
	Пусть есть два пути $\gamma_0, \gamma_1\colon [a,b] \ra G$, начинающиеся и заканчивающиеся в одинаковых точках.
	Пути $\gamma_0$ и $\gamma_1$ гомотопны как пути с неподвиными концами, если существует такое непрерывное отображение
	$\gamma\colon [a,b] \times [0,1] \ra G$, что
	\begin{gather*}
		\forall t \in [a, b], \gamma_0(t) = \gamma(t, 0) \land \gamma_1(t) = \gamma(t, 1) \\
		\forall u \in [0, 1], \gamma_0(a) = \gamma(a, u) \land \gamma_0(b) = \gamma(b, u)
	\end{gather*}
\end{Def}

Физический смысл: висит резинка на двух гвоздях в форме одной кривой, мы её как-то сжимая, растяшивая и двигая переводим в другую кривую.

\begin{Def}
	Пусть есть два замкнутых пути $\gamma_0, \gamma_1\colon [a,b] \ra G$.
	Пути $\gamma_0$ и $\gamma_1$ гомотопны как замкнутые пути, если существует такое непрерывное отображение
	$\gamma\colon [a,b] \times [0,1] \ra G$, что
	\begin{gather*}
		\forall t \in [a, b], \gamma_0(t) = \gamma(t, 0) \land \gamma_1(t) = \gamma(t, 1) \\
		\forall u \in [0, 1], \gamma(a, u) = \gamma(b, u)
	\end{gather*}
\end{Def}

Не все пути гомотопны "--- на плоскости с дыркой путь вокруг дырки не гомотопен пути, не захватывающем дырки.

\begin{Def}
	Нулевой (вырожденный путь) "--- константа.
\end{Def}

\begin{Def}
	Путь стягиваем, если он гомотопен нулевому.
\end{Def}

\begin{Rem}\hfill\begin{verse}\it
	Не суй голову в петлю, \\
	Гомотопную нулю!
\end{verse}\end{Rem}

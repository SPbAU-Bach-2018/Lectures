\begin{exmp}[Метод Дарбу]
	Хотим понять скорость роста коэффицентов.
	$\sum_{n = 0}^{\infty} a_nz^n$ сходится в круге $|z| < R$, тогда он сходится абсолютно в круге $|z| \le r < R$ (отступили от границы), значит $a_nr^n \ra 0 \Ra a_n = o(r^{-n})$.

	Пусть есть полюс: \TODO картинка.
	% картинка, взять фото у Юры Кравченко
	\[ g(z_0) = \frac{c_{-1}}{z - z_0} + \ldots + \frac{c_{-m}}{(z - z_0)^m} \]
	$f(z) - g(z)$ будет сходиться в большом круге.
	Явная формула для случая полюса?
	\[ \frac{1}{(z - a)^n} = \sum_{n = 0}^{\infty} С_{n + m - 1}^{m - 1} \left( \frac{z}{a} \right)^n \Ra C_{n + m - 1}^{n} \sim \frac{n^{m - 1}}{(m - 1)!} \]
	Но проблемы могут быть не только из-за особой точки или полюса:
	\begin{gather*}
		f(z) = \frac{\sqrt{1 - \alpha z}}{(1 - z)^2} \quad 0 < \alpha < 1 \\
		g(z) = \frac{\sqrt{1 - \alpha}}{(1 - z)^2}
		h(z) = f(z) - g(z) = 
		\frac{1}{(1 - z)^2}(\sqrt{1 - \alpha z} - \sqrt{1 - \alpha}) = 
		\frac{1}{(1 - z)^2} \frac{(1 - \alpha z)  - (1 - \alpha)}{\sqrt{1 - \alpha z} + \sqrt{1 - \alpha}} =
		\frac{\alpha}{1 - z} \frac{1}{\sqrt{1 - \alpha z} + \sqrt{1 - \alpha}}
	\end{gather*}
\end{exmp}

\begin{gather*}
	g(z) = \sum_{n=0}^{+\infty} b_n z^n \quad b_n = n \sqrt{1-\alpha} \\
	\frac1{\sqrt{1-\alpha z} + \sqrt{1-\alpha}} \text{ голоморфна в круге радиуса $\sfrac 1{\alpha}$}
\end{gather*}
	значит там её коэф-ы $o(r^{-n})$
\begin{gather*}
	\frac1{1-z} = 1 + z + z^2 + z^3 + \dots \\
\end{gather*}
Значит коэф-ы произведения $O(1)$, то есть $a_n = \sqrt{1-\alpha} n + O(1)$.
Ваще это из книжки <<Грин, Кнут. Математические методы анализа алгоритмов.>>

\begin{exmp}[Суммирование рядов]
	\begin{gather*}
		\sum_{n=3}^{+\infty} f(z)
	\end{gather*}
	А мы хотим это посчитать по теореме о вычетах.
	Для этого нужна функция, имеющая по особенности во всех натуральных точках.
	\begin{gather*}
		\res_{z=n} \frac{f(z)}{\sin \pi z} = \frac{f(z)}{(\sin \pi z)'} \biggr|_{z=n} = \frac{(-1)^n f(n)}{\pi}
	\end{gather*}
	Ну почти получилось.
	\begin{gather*}
		\res_{z=n} f(z) \ctg \pi z = \frac{f(z) \cos \pi z}{(\sin \pi z)'} \biggr|_{z=n} = \frac{f(n)}\pi
	\end{gather*}
	Клёво! Осталось посчитать интеграл.
	\begin{gather*}
		f(z) = \frac1{n^2} \quad g(z) = \frac{\ctg \pi z}{z^2} \quad \res_{z=n} g(z) = \frac1{\pi n^2}
	\end{gather*}
	Есть ещё особенность в нуле.
	\begin{gather*}
		\frac{\ctg \pi z}{z^2} = \frac{\cos \pi z}{z^2 \sin \pi z}
		= \frac{1 - \frac{\pi^2 z^2}2 + \frac{\pi^4 z^4}{24} - \dots}
		{z^2 \pi z \left(1 - \frac{\pi^2 z^2}6 + \frac{\pi^4 z^4}{5!} - \dots\right)} = \\
		= \frac1{\pi z^3} \left(1 - \frac{\pi^2 z^2}2 + \frac{\pi^4 z^4}{24} - \dots\right)
		\left(1 - \frac{\pi^2 z^2}6 + \dots\right)^{-1}
		= \frac1{\pi z^3} \left(1 - \frac{\pi^2 z^2}3 + \dots\right) \\
		\res_{z=0} g(z) = -\frac{\pi}3
	\end{gather*}
	Теперь считаем. Мы знаем, что котангенс ограничен на окружности полуцелого радиуса.
	\begin{gather*}
		\int\limits_{|z|=n+0,5} \frac{\ctg \pi z}{z^2} \d z
		= 2\pi i \left(-\frac{\pi}3 + 2\sum_{k=1}^n \frac1{\pi k^2}\right) \\
		\left| \int\limits_{|z|=n+0.5} \frac{\ctg \pi z}{z^2} \d z \right|
		\le M \frac{2\pi (n + 0,5)}{(n + 0,5)^2} \ra 0 \\
		2\pi i \left(-\frac{\pi}3 + 2\sum_{k=1}^n \frac1{\pi k^2} \right) \ra 0 \\
		\sum_{k=1}^{+\infty} \frac1{k^2} = \frac{\pi^2}6
	\end{gather*}
\end{exmp}

\section{Конформные отображения}

\begin{Def}
	Линейное отображение называется конформным, если оно сохраняет угол между векторами и ориентацию.
\end{Def}

\begin{theorem}
	$L\colon \C \ra \C$ "--- линейное над $\R$.
	Тогда $L$ конформно тогда и только тогда, когда $L(z) = \lambda z$ для какого-то $\lambda \in \C \setminus \{0\}$.
\end{theorem}
\begin{proof}\begin{description}
\item[$\La$:] Просто посмотрим $|\lambda|$ и $\arg \lambda$ "--- всё растянется и повернётся, конофрмно.
\item[$\Ra$:] Возьмём базис. Значит его образ тоже базис. Осталось показать, что их длина не поменяется. Для этого посмотрим на $e_1 + e_2$.
	Сначала мы были под углом $\frac{\pi}4$, должны были остаться, значит длины образов равны.
\end{description}\end{proof}

\begin{Def}
	$f$ конформно в точке $z$, если $\d_z f$ линейно конформно и не равно нулю.
\end{Def}

\begin{Def}
	Отображение $f\colon \Omega_1 \ra \Omega_2$ конформно, если оно биективно и конформно в каждой точке.
\end{Def}
\begin{Rem}
	Если $\Omega_1, \Omega_2 \subset \C$, то определение равносильно биективной голоморфности.
	Для $\bar \C$ надо ещё описать, что такое комфорность на бесконечности.
	$f$ конформна на бесконечности, если $f(1/z)$ комформна в нуле.
	$f$ конформна в точке, образ который бесконечен, если $\sfrac1f$ комформна в той же точке.
\end{Rem}

\begin{Def}
	$f \in H(\Omega)$ "--- определена на области, $f \ne const$.
	Тогда $f(\Omega)$ "--- область.
\end{Def}
\begin{Rem}
	Непрерывности или вещественной дифференцируемости не хватит: $|z|^2\colon \C \ra \R$.
\end{Rem}
\begin{proof}
	Покажем, что $f(\Omega)$ "--- открыто.
	Для этого хотим показать, что $\exists \epsilon > 0\colon \forall |z-a|<\epsilon, |f(z) - b| \ne 0$.
	От противного: пусть не так, и для каждого $\epsilon$ есть точка $z_{\epsilon}$, что $f(z_{\epsilon}) = b$.
	Получили, что $z_n \ra a \land f(z_n) = b$, по теореме о единственности $f(z) \equiv b$.

	Далее, на компакте достигается минимум
	\[ r = \min_{|z-a| = \epsilon} |f(z) - b| > 0 \]
	Докажем, что $B_{\sfrac r2}(b) \subset f(\Omega)$.
	Возьмём $w \in B_{\sfrac r2}(b)$.
	\begin{gather*}
		f(z) - w = f(z) - b + b - w \\
		|f(z) - b| \ge r > |b - w|
	\end{gather*}
	По теореме Руше $N_{f-w} = N_{f-b} \ge 1$, то есть существует $z$, что $|f(z) - w| = 0$.

	Теперь связность.
	Была ломаная, связывающая две точки.
	После преобразования оно перейдёт в гладкую ломаную.
	Можем подогнать ломаной "--- производные ограничены, значит какими-то маленькими шажками можно пройти.
\end{proof}

\begin{Def}
	$f\colon \C \ra \C$ однолистна, если она инъективна ($f(z_1) = f(z_2) \Ra z_1 = z_2$).
\end{Def}

\begin{theorem}
	Если $f \in H(\Omega)$ и однолистна, то про $f' \ne 0$ нигде.
\end{theorem}
\begin{proof}
	$a \in \Omega$. Пусть $f'(a) = 0$ и $b \coloneq f(a)$.
	Сделаем всё, как в прошлом доказательстве.
	Найдём радиус, возьмём круг $B_{\epsilon}(a)$, теорему Руше применим.
	\[ N_{f-w} = N_{f-b} \ge 2 \]
	так как $a$ "--- ноль кратности 2.
	Значит, $f(z) = w$ имеет два решения, но вдруг это решение двойной кратности?

	Не могло так быть. Вспомним рассуждения вокруг теоремы о единственности.
	Давайте предположим, что везде, где бы мы не взяли, корни кратные.
	Получили последовательность $z_n$, сходящуюся к $a$, что $f'(z_n) = 0$.
	Значит вся производная ноль, вся функция константа, а она не однолистна.
\end{proof}

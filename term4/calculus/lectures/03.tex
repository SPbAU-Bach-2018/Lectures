\section{Голоморфные функции}

\begin{Def}
	$f\colon \Omega \ra \C$, $z_0 \in \Omega$, $\Omega$ открытая и связная.
	$f$ называется голоморфной в $z_0$, если существует предел
	\[ \lim_{z \ra z_0} \frac{f(z) - f(z_0)}{z - z_0} \in C \]
\end{Def}

\begin{Rem}
	Равносильное условие: сущесвтует такой $k \in \C$, что
	\[ f(z) = f(z_0) + k(z - z_0) + o(|z - z_0|) \]
\end{Rem}

\begin{Rem}
	Это очень похоже на частную производную, если подходить к $z_0$ только по горизонтали или вертикали.
	\begin{gather*}
		f = \begin{pmatrix} g \\ h \end{pmatrix}\colon \Omega \subset \R^2 \ra \R^2 \\
		\begin{pmatrix} g(x, y) \\ h(x, y) \end{pmatrix}
		= \begin{pmatrix} g(x_0, y_0) \\ h(x_0, y_0) \end{pmatrix}
		+ A \begin{pmatrix} x - x_0 \\ y - y_0 \end{pmatrix}
		+ o(\sqrt{(x - x_0)^2 + (y - y_0)^2})
	\end{gather*}
	В чём же разница?
	\begin{gather*}
		f(z) = g(z) + ih(z) \quad z = x + iy \quad k = l + im \\
		(l + im)((x - x_0) + i(y - y_0))
		= (l(x - x_0) - m(y - y_0)) + i(m(x - x_0) + l(y - y_0)) \\
		\begin{pmatrix}
			l(x - x_0) - m(y - y_0) \\
			m(x - x_0) + l(y - y_0)
		\end{pmatrix}
	\end{gather*}
	Как видно, в качестве матрицы $A$ должны быть только матрицы вида
	\[\begin{pmatrix}
		l & -m \\
		m & l
	\end{pmatrix}\]
	Запишем эти условия:
	\begin{gather*}
		A = \begin{pmatrix}
			\partd gx & \partd gy \\
			\partd hx & \partd hy
		\end{pmatrix} \\
		\partd gx = \partd hy \quad \partd gy = -\partd hx
	\end{gather*}
	Вывод: $f$ голоморфна в $z_0$ тогда и только тогда, когда она вещественно дифференцируема и выполняется равенство выше.
\end{Rem}

\begin{assertion}[Условие Коши---Римана (Эйлера---Даламбера)]
	Если функция голоморфна в $z_0$, то
	\[ \partd fx = -i \partd fy \]
\end{assertion}

Поравзлекаемся с дифференицалами:
\begin{gather*}
	\d f = \partd fx \d x + \partd fy \d y \\
	z = x + iy \quad \bar z = x - iy\\
	\d z = \d x + i \d y \quad \d \bar z = \d x - i \d y \\
	\d f = A \d z + B \d \bar z \\
	A + B = \partd fx \quad iA - iB = \partd fy \\
	\d f
	= \underbrace{\frac12 \left( \partd fx - i \partd fy \right)}_{=\partd fz} \d z
	+ \underbrace{\frac12 \left( \partd fx + i \partd fy \right)}_{=\partd f{\bar z}} \d \bar z
\end{gather*}
Условие Коши-Римана можно тогда переписать:
\[ \partd f{\bar z} = 0 \]
и тогда если $f$ голоморфна, то
\[ \d f = \partd fz \]

\begin{conseq*}
	$f$ голоморфная на $\Omega$ и $\Re f = const$.
	Тогда $f = const$.
\end{conseq*}
\begin{proof}
	$f = g + ih$, $g = const$.
	\begin{gather*}
		0 = \partd gx = \partd hy \quad 0 = \partd gy = -\partd hx \\
		\Ra \partd hx = \partd hy = 0 \Ra h = const \Ra f = const
	\end{gather*}
\end{proof}

\begin{conseq*}
	$f$ голоморфна и дважды дифференуируема.
	Тогда
	\begin{gather*}
		\partD gx + \partD gy = 0 \\
		\partD hx + \partD hy = 0 \\
	\end{gather*}
	Тажке, такая функиця называется гармонической.
\end{conseq*}
\begin{proof}
	\begin{gather*}
		\partD gx = \partd{}x\left( \partd gx \right) = \partd{}x\left( \partd hy \right) = \partdd hxy \\
		\partD gy = \partd{}x\left( \partd gy \right) = \partd{}x\left( -\partd hx \right) = -\partdd hyx
	\end{gather*}
\end{proof}

\begin{Rem}
	$g\colon \Omega \subset \C \ra \R$ гармноническая функция.
	Тогда существует гармоническая $h\colon \Omega \ra \R$, что $f = g + ih$ голоморфная и единственная с точностью до прибавления константы.
	Такие $g$ и $h$ называют гармонически сопряжёнными.
\end{Rem}

\begin{theorem}[Коши]
	$f\colon \Omega \ra \C$ голоморфная ($f \in H(\Omega)$).
	Тогда форма
	\[ f(z) \d z \]
	замкнута.
\end{theorem}
\begin{proof}
	\textbf{Первое доказательство}.
	Возьмём любую точку, из её окрестности возьмём прямоугольник $G$ c правильно ориентированным контуром $\partial G$.
	Если $\partd fx$ и $\partd fy$ непрерывны, то доказательство очень простое:
	\begin{gather*}
		\int\limits_{\partial G} f(z) \d z
		= \int\limits_{\partial G} (g(z) + ih(z))(\d x + i \d y)
		= \int\limits_{\partial G} (g(z) \d x - h(z) \d y)
		+i\int\limits_{\partial G} (h(z) \d x + g(z) \d y) = \cdots
	\end{gather*}
	Формула Грина!
	\begin{gather*}
		\cdots = \int\limits_G \left( \left( -\partd hx - \partd gy \right) + i\left( \partd gx - \partd hy \right)\right)
	\end{gather*}
	Оба коэффициенты по условию Коши---Римана нули.
\end{proof}
\begin{Rem}
	$f\colon \Omega \ra \C$ "--- произвольная вещественно дифференцируемая.
	Тогда
	\[ \ointctrclockwise\limits_{\partial G} f(z) \d z = 2i \int\limits_G \partd f{\bar z} \d x \d y \]
\end{Rem}
\begin{proof}
\[
	\partd f{\bar z}
	= \frac12 \left( \partd fx + i\partd fy \right)
	= \frac12 \left( \partd gx + i\partd hx + i\partd gy - \partd hy\right) \]
\end{proof}

\begin{proof}
	\textbf{Второе доказательство}.
	Теперь честное доказательство теоремы, без непрерывности.
	Обозначим
	\[ \alpha(P) \lrh \int\limits_P f(z) \d z \]
	Берём наш прямоугольник $P$ и предположим, что на нём $\alpha(P) \ne 0$.
	Разрежем наш прямоугольник на 4 подпрямоугольника
	%OH GOD WHYYY
	\newcommand{\tildes}[2]{%
		\ifnumgreater{#1}{0}%
			{\tilde{\tildes{\number\numexpr#1-1\relax}{#2}}}%
			{#2}%
	}
	$\tildes1P$, $\tildes2P$, $\tildes3P$ и $\tildes4P$.
	\[ \alpha(P) = \alpha(\tildes1P) + \alpha(\tildes2P) + \alpha(\tildes3P) + \alpha(\tildes4P) \]
	Не умаляя общности,
	\[ |\alpha(\tildes1P)| \ge \frac14 |\alpha(P)| \]
	Тогда $P_1 \lrh \tildes1P$. Повторим процесс рекурсивно.
	\[ P \supset P_1 \supset P_2 \supset P_3 \supset \dots \]
	Выберем общую точку $z_0$.
	\[ f(z) = f(z_0) + k(z - z_0) + \beta(z-z_0) \quad \beta(z-z_0) = o(z - z_0)\]
	\TODO Я здесь кусок потерял, будет \TODO
	Оценим
	\begin{gather*}
		\left| \int\limits_{P_n} \beta(z) \d z\right|
		\le perimeter(P_n) \cdot \max |\beta(z)|
		= \frac{perimeter(P)}{2^n} \max |\beta(z)|i = \\
		= \frac{perimeter(P)}{2^n} |z - z_0| \cdot o(1)
		\le \left(\frac{perimter(P)}{2^n}\right)^2 o(1) \\
		\frac1{4^n} |\alpha(P)|
		\le |\alpha(P)|
		\le \frac{perimeter(P)^2}{4^n} o(1)
	\end{gather*}
\end{proof}

\begin{conseq}
	$f \in H(\Omega)$, $\gamma$ "--- стягиваемая петля.
	\[ \int\limits_\gamma f(z) \d z = 0 \]
\end{conseq}

\begin{conseq}
	$f \in H(\Omega)$.
	Тогда у $f$ есть локальная первообразная, которая голомофрная.
\end{conseq}
\begin{proof}
	$f(z) \d z$ замкнута, значит в окрестности точки есть первообразная $d F = f(z) \d z$, откуда
	$F' = \partd Fz = f$.
\end{proof}

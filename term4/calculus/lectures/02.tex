\begin{theorem}
	$\omega$ "--- замкнутая форма в $G$, $\gamma_0$ и $\gamma1$ "--- гомотопные пути.
	Тогда
	\[ \int\limits_{\gamma_0} \omega = \int\limits_{\gamma_1} \omega \]
\end{theorem}

\begin{theorem}
	$\omega$ "--- замкнутая форма в $G$, $\gamma$ Э--- стягиваемый путь.
	Тогда
	\[ \int\limits_\gamma \omega = 0 \]
\end{theorem}

Доказательства не будет "--- они будут тривиальными следствиями следующей теоремы.

\begin{Def}
	$\omega$ "--- замкнутая форма в $G$, $\gamma\colon [a, b] \times [c, d] \ra G$ непрерывна.
	$f$ называют первообразной формы $\omega$ относительно $\gamma$, если для каждой $\gamma(\tau, \nu)$
	существует окрестность $U$, что первообразная $F$ в ней
	\[ F(\gamma(t, u)) = f(t, u) \]
\end{Def}

\begin{lemma}[Лебега]
	$K$ "--- компакт в метрическом пространстве, $U_\alpha$ "--- его покрытие открытыми множествами.
	Тогда существует такой $r > 0$, что для всех $x \in K$ шар $B_r(x)$ содержится в каком-то элементе покрытия.
\end{lemma}
\begin{proof}
	Возьмём $a \in K$, значит точно есть $U$ из покрытия, накрывающая $U$.
	Значит, существует $r_a > 0$, что $B_{r_a}(a) \subset U$.

	Шары $B_{r_a / 2}(a)$ "--- покрытие компакта, значит существует конечное подпокрытие:
	\[ B_{r_{a_1} / 2}(a_1) \cup \dots \cup B_{r_{a_n} / 2}(a_n) \supset K \]
	Возьмём $r = \frac12 \min \left\{ r_{a_1}, \dots, r_{a_n} \right\}$ "--- он подходит.
	Проверим это: возьмём какой-то $x \in K$, он лежит в каком-то $B_{r_{a_i} / 2}(a_i)$,
	а там уже
	\[ B_r(x) \subset B_{r_{a_i} / 2}(\mathbf x) \subset B_{r_{a_i}}(\mathbf a_i) \subset U\]
\end{proof}

\begin{theorem}
	$\omega$ "--- замкнутая форма в $G$, $\gamma\colon [a, b] \times [c, d] \ra G$ непрерывна.
	Тогда существует первообразная $\omega$ относительно $\gamma$.
\end{theorem}
\begin{proof}
	Возьмём непрерывный образ компакта $K = \gamma([a, b] \times [c, d])$.
	У каждой точки есть окрестность, в которой есть первообразная $F$.
	Возьмём $r$ из леммы Лебега $r$ и нарежем $[a, b] \times [c, d]$ на прямоугольники, диаметр каждого из которых меньше $r$.
	Это возможно, так как $\gamma$ равномерно непрерывна (потому что непрерывна на компакте), и по определению сущесвтует такое $\delta$,
	что образы точек, отстающих менее, чем на $\delta$, отстают менее, чем на $r$.

	Тогда достаточно нарезать $[a, b] \times [c, d]$ на прямоугольники диаметра менее $\delta$.
	Пусть $U_{ij}$ "--- окрестность, накрывающая $[t_{i-1}, t_i] \times [u_{j-1}, u_j]$.
	На каждой есть первообразная $F_{ij}$. Согласуем их.
	Зафиксируем $j$, рассмотрим $U_{ij}$ и $U_{i+1,j}$.
	У них есть общая точка, и пересечение открыто, значит на ней первообразные отличаются на константу.
	Подправим одну из них, чтобы совпали.

	Получили $f_j = F_{ij}(\gamma(t, u))$, непррывную на $[a, b] \times [u_{j-1}, u_j]$.
	Теперь посмотрим на строчки вида $\bigcup\limits_i U_{ij}$ и $\bigcup\limits_i U_{i,(j+1)}$.
	У них тоже есть пересечение, можно сразу подправить целую строчку на константу.

	\begin{Rem}
		На самом деле порядок склейки не важен.
		Можно было поиском в глубину, по очереди доклеивая каждую ко всем предыдущим, лишь бы пересекались.
		Ну как получилось, так получилось.
	\end{Rem}
\end{proof}

\begin{proof}
	Докажем теоремы 1 и 2:
	\begin{enumerate}
	\item
		$\gamma$ "--- гомотопия $\gamma_0$ и $\gamma_1$, $f$ "--- первообразная $\omega$ относительно $\gamma$.
		\begin{align*}
			\int\limits_{\gamma_0} \omega &= f(b, 0) - f(a, 0) \\
			\int\limits_{\gamma_1} \omega &= f(b, 1) - f(a, 1)
		\end{align*}
		Заметим, что $\gamma(a, u) = const$,
		\[ f(a, 0) = F(\gamma(a, 0)) = F(\gamma(a, 1)) = f(a, 1) \]
		Аналогично $f(b, 0) = f(b, 1)$.

	\item
		$\gamma$ "--- гомотопия $\gamma_0$ и $\gamma_1$, $f$ "--- первообразная $\omega$ относительно $\gamma$.
		$\gamma_1$ "--- вырожден.
		\begin{align*}
			\int\limits_{\gamma_0} \omega &= f(b, 0) - f(a, 0) \\
			0 = \int\limits_{\gamma_1} \omega &= f(b, 1) - f(a, 1)
		\end{align*}
		Тут нет неподвижности концов. Надо действовать так: зафиксируем $a$ и будем ползти по $u$.
		Внутри каждой исходной области \TODO
	\end{enumerate}
\end{proof}

\begin{Def}
	$G$ "--- односвязная область, если любой замкнутый путь в $G$ стягивается.
\end{Def}

\begin{theorem}
	$G$ "--- односвязная область, $\omega$ "--- замкнутая форма в $G$.
	Тогда $\omega$ "--- точная.
\end{theorem}
\begin{proof}
	Раз любой замкнутый путь стягивается, то по любому замкнутому пути интеграл 0.
\end{proof}

\begin{exmp}\hfill\begin{enumerate}
\item
	Звёздная обрасть "--- область, есть точка, из которой по отрезку внутри обрасти можно дойти до любой точки области.
	\[ \exists a \in G\colon \forall x \in G, [a, x] \subset G \]
	Покажем, что она звёздная.
	\begin{proof}
		Берём любую кривую, начинаем её гомотетично\footnote{по подобию, уменьшая коэффициент} стягивать в точку $a$.
		Оно стянется.
	\end{proof}

\item
	Выпуклые фигуры односвязные.

\item
	$\C \setminus 0$ не односвязная.
\end{enumerate}\end{exmp}

\begin{Rem}
	Дальше будут встречаться обозначения:
	\begin{itemize}
		\item $ \mathbb T = \{z \mid |z| = 1\}$ "--- единичная окружность.
		\item $r\mathbb T = \{z \mid |z| = r\}$
		\item $ \mathbb D = \{z \mid |z| < 1\}$
		\item $r\mathbb D = \{z \mid |z| < r\}$
		\item $ \bar \mathbb D = \{z \mid |z| \le 1\}$
		\item $r\bar \mathbb D = \{z \mid |z| \le r\}$
	\end{itemize}
\end{Rem}

\textbf{Упражнение}
Пусть есть односвязная область $G \in \C$, $f\colon \mathbb T \ra G$ непрерывна.
Показать, что можно непрерывно продолжить до $f\colon \bar \mathbb D \ra G$.


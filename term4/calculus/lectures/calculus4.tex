\begin{theorem}
	$f \in H(\Omega)$ и однолистна $\Ra$ $\forall z \in \Omega\colon f'(z) \neq 0$
\end{theorem}

\begin{conseq}
	$f(z) = c_0 + \frac{c_1}{z} + \frac{c_2}{z^2} + \frac{c_3}{z^3} + \ldots$, 
	$f$ -- голоморфна в окрестности $\inftyt$ и однолистна 
	$\Ra$ 
	$c_{-1} \neq 0$
\end{conseq}

\begin{proof}
	$g(z) = f(1 / z)$ голоморфна в окрестности 0 и однолистна
	$\Ra$ 
	$g'(0) \neq 0$, $g(z) = c_0 + c_1 z + c_2 z^2 + \ldots$ и
	$g'(0) = c_{-1}$
\end{proof}

\begin{conseq}
	$f$ имеет полюс в (.) $a$ и однолстна в окрестности (.) $a$
	$\Ra$ 
	это полюс 1-го порядка
\end{conseq}

\begin{proof}
	$g(z) = \frac{1}{g(z)}$ у $g(z)$ в (.) $a$ устранимая особая (.) 
	$\Ra$
	голоморфная и однолистная 
	$\Ra$ 
	$g'(a) \neq 0$

	$f(z) = \frac{c_{-m}}{(z-a)^m} + \ldots + \frac{c_{-1}}{z - a} +  c_0 + \ldots$ф
	$\Ra$
	ноль порядка $m$
\end{proof}


\begin{proof}
Доказывать существование не будем.
Докажем единственность.

1) $\Omega = \tilde{\Omega} = \mathbb{D}$ $z_0 = \tilde{z} = \alpha_0 = 0$
	тода $f = z$
	$f: D \ra D$, $f(0) = 0$ 
	$\Ra$ (по лемме Шварца)
	$|f(z)| \leq |z|$

	$f^{-1}: \mathbb{D} \ra \mathbb{D}$
	$f(0) = 0$
	$\Ra$
	$|f^{-1}(z)| < |z|$

	Тогда 
	$|f(z)| \leq |z|$
	$|f^{-1}(z)| \leq |z|$
	$\Ra$ ()
	$|f(z)| \leq |z|$
	$|z| \leq |f(z)|$
	$\Ra$
	$|f(z)| = |z|$
	$\Ra$ (лемма Шварца)
	$f(z) = e^{iz}z$

	$f'(z) = e^{i\phi}$ но $\arg f'(0) = 0 \Ra e^{i\phi} = 1$

2) $\Omega$ и $\tilde{\Omega}$ произвольные

	$\phi: \D \ra \Omega$; $\phi(0) = z_0$ и $\phi'(0) > 0$

	$\psi: \tilde{\Omega} \ra \D$; $\psi(\tilde{z_0}) = 0$ и $\arg \psi'(\tilde{z_0}) = -\alpha_0$

	пусть $f_1$ и $f_2: \Omega \ra \tilde{\Omega}$,  конформные и $f_i(z_0) = \tilde{z_0}$ и $\arg f'_i(z_0) = \alpha_0$
	$g_i(z) = \psi(f_i(\phi(z)))$, $g_i: \D \ra \D$ конформные

	$\D \ra^\phi \Omega \ra^{f_i} \tilde{\Omega} \ra^{\psi} \D$

	$g_i(0) = \psi(f_i(\phi(0))) = \psi(f_i(z_0)) = \psi(\tilde(z_0)) = 0$

	$g_i'(0) = \psi'(f_i(\phi(0))) \cdot f'(\phi(0)) \cdot \phi'(0) = \psi(\tilde{z_0}) \cdot f'_i(z_0) \phi'(0)$ тут аргумент 0 
	$\Ra$ 
	$g_i(z) = z$
	$\Ra f_1 \eqiv f_2$
\end{proof}



\begin{assertion}
	$f\colon \Omega \ra \Omega$ "--- комплексный потенциал в $\Omega$, $g\colon \tilde\Omega \ra \Omega$ "--- конформная.
	Тогда $f \circ g$ "--- комплексный потенциал в $\tilde \Omega$.
\end{assertion}
\[ f \in H(\Omega), g \in H(\Omega) \Ra f \circ g \in H(\Omega) \]
Если мы взяли линию тока, то она в её маленькой окрестности только растянулась.
\[ \Im f = const \Ra \Im f \circ g = const \]

Попробуем решить такую задачу.
Есть заданная скорость на бесконечности, есть кривая через всю плоскоть, на бесконечности её касательная горизонтальная.
Хотим найти поле скоростей.
Надо найти похожую задачу, где всё понятно, и применить конформное отображение.
Всё понятно нам про настоящую полуплоскость c cкоростью $v$.
Там потенциал $vz$.

Расмотрим отображение в настоящую полуплоскость $g$ ($g(\infty) = \infty$, $g'(\infty) > 0$ \TODO это из теоремы Римана "--- выбрали нулевой аргумент).
Если $f$ "--- комплексный потенциал, то $\bar f' = f'$ "--- поле скоростей.
Тогда $vg'(\infty)$ "--- скорость бесконечности.
Осталось подобрать коэфициент $v = \frac{v_{\infty}}{g'(\infty)}$.

Поле точечного источника в начале координат "--- поле скоростей имеет вектора на лучах из точки (из соображений симметрии).
\[ V = \vec r \phi(r) = \frac{z}{|z|} \phi(|z|) \]
Пишем формулу для потока через окружность
\[ N = \int\limits_{|z| = r} \left< \vec V, \vec n \right> \d \sigma = \int\limits_{|z| = r} \phi(|z|) \d \sigma = 2\pi r \phi(|z|) \]
Но заметим, что поток постоянен для любого контура вокруг начала координат, откуда
\begin{gather*}
	\phi(r) = \frac N{2\pi}\frac1r \\
	V = \frac{Nz}{z\bar z \cdot 2\pi} = \frac N{2\pi \bar z} \\
	\bar f' = V \Ra f = \frac N{2\pi} \Ln z
\end{gather*}

Поле скоростей и комлпексный потенциал для точечного вихря.
Если повернём, то всё перейдёт в себя.
\begin{gather*}
	V = \frac{iz}{|z|} \psi(|z|) \\
	\Gamma
	= \int\limits_{|z|=r} \left<\vec V, \vec \sigma \right> \d \sigma
	= \int\limits_{|z|=r} \psi(r) \d \sigma
	= 2 \pi r \psi(r) \\
	\psi(r) = \frac{\Gamma}{2\pi}\frac1r \\
	V = \frac{i\Gamma}{2\pi \bar z} \\
	f = \frac{i\Gamma}{2n} \Ln z
\end{gather*}

А если исток с вихрем, то 
\[ f = \frac{N + i\Gamma}{2n} \Ln z \]

Теперь решаем задачу обтекания какой-то фигни, зная $v_{\infty}$.
Аналогично, стягиваем голоморфно в отрезок, чтобы $g(\infty) = \infty$, $g'(\infty) > 0$.
У отрезка весь потенциал $v_\infty z$.

Если обтекаем окружность, то нужная функция "--- Жуковского, домноженная на нужную константу.
\[ f(z) = v_{\infty} \left(z + \frac1z\right) \]
Но мы ещё забыли, что кроме обтекания есть ещё вихрь вокруг круга.
\[ f(z) = v_{\infty} \left(z + \frac1z\right) + \frac{i\Gamma}{2\pi} \Ln z \]
Это общее решение, правда $\Gamma$ нужно из физических соображений.

А для всяких клякс
\[ f(z) = \frac{v_{\infty}}{g'(\infty)} \left(g(z) + \frac1{g(z)}\right) + \frac{i\Gamma}{2\pi} \Ln g(z) \]

Так раньше всё расчитывали "--- всё сводится к поиску комфорных отображений.
Так обсчитывались первые самолёты, винты кораблей.
Сейчас так делаются только грубые прикидки, а дальше используются более сложные модели.

\begin{Def}
	Точка с нулевой скоростью называется критической.
\end{Def}

В случае обтекания круга
\begin{gather*}
	f'(z) = 0 \\
	f'(z) = v_{\infty} - \frac{v_{\infty}}{z^2} + \frac{i\Gamma}{2\pi}\frac1z \\
	z = \frac1{4\pi v_{\infty}} \left( \Gamma i \pm \sqrt{16\pi^2 v_{\infty}^2 - \Gamma^2} \right)
\end{gather*}
\begin{description}
\item[$|\Gamma| \le 4\pi v_{\infty}$:]
	$|z| = 1$.
	Это значит, что на окружности есть две точки: где поток расщипляется вокруг окружности и схлопывается за ней.

\item[иначе:]
	Точка всего одна, и тогда это выглядит так:
	какая-то линия потока проходит сбоку от откружности, огибает её и пересекая саму себя уходит дальше.
\end{description}

И тут на прощание была куча весёлых картинок красивых отображений.

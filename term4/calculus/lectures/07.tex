\begin{conseq}
	$f, g \in H(\Omega)$, $\forall z \in B_r(z_0), f(z) = g(z)$.
	Тогда $f = g$.
\end{conseq}

\begin{Def}[Аналитическое продолжение]
	Пусть есть две области $\Omega_1 \cap \Omega_2 \ne 0$, $f_1 \in H(\Omega_1)$.
	Если $f_2 \in H(\Omega_2)$ и $f_1 = f_2$ на $\Omega_1 \cap \Omega_2$, то $f_2$ называыется аналитическим продолжением $f_1$.
\end{Def}

\begin{conseq}
	Из следсвтия видно, что если продоожение есть, то оно однозначно.
\end{conseq}

\begin{Def}
	$f \in H(\Sigma)$, $z_0 \in \Sigma$.
	Если $f \nequiv 0$, а $f(z_0) = 0$, то $z_0$ "--- нуль (корень) $f$.
\end{Def}

\begin{Rem}
	$z_0$ "--- корень.
	\begin{gather*}
			f(z) = \sum_{n=0}^{+\infty} a_n (z - z_0)^n \\
			f(z) = \sum_{n=k}^{+\infty} a_n (z - z_0)^n \quad a_k \ne 0
	\end{gather*}
	Это $k$ "--- порядок (кратность) нуля (корня).
\end{Rem}

\begin{theorem}
		$f \in H(\Sigma)$, $f \ne 0$.
		Множество нулей $f$ дискретно (состоит из изолированных точек).
\end{theorem}
\begin{proof}
	Пусть $z_0$ "--- ноль $f_0$.
	\begin{gather*}
		f(z) = \sum_{n=k}^{+\infty} a_n (z - z_0)^n = (z - z_0)^k \underbrace{\sum_{n=0}^{+\infty} a_{n+k} (z - z_0)^n}_{\lrh g(z)}
	\end{gather*}
	$g(z)$ голоморфна в окрестности $z_0$ и $g(z_0) = a_k \ne 0$.
	$|g(z)|$ непрерывна в окрестности $z_0$ и $|g(z)| \ge \frac{|a_k|}2 > 0$.
	Тогда в окрестности $z_0$ оба множителя $f$ не обращаются в 0 в окрестности $z_0$.
\end{proof}

\begin{conseq}
	$f \in H(\Sigma)$, $f \nequiv 0$.
	В любом компакте существует лишь константое число корней.
\end{conseq}
\begin{proof}
	$K \subset \Sigma$, $\{\omega_k\} \subset K$ "--- нули, значит есть сходящаяся подпоследовательность
	$\omega_{n_k} \ra \omega_0 \in K$.
	Но $f$ непрерывна, значит $f(\omega_0) = 0$.
	Нашли неизолированный корень.
\end{proof}

\begin{theorem}[о единственности]
	$f, g \in H(\Omega)$, $\{\omega_k\} \subset \Omega$, $\forall k, f(\omega_k) = g(\omega_k)$.
	Если у последовательности $\{\omega_k\}$ имеет предельную точку в $\Omega$, то $f \equiv g$.
\end{theorem}
\begin{proof}
	Рассмотрим $h = f - g$.
	Если есть предельная точка, то у $h$ есть неизолированный ноль, и $h \equiv 0$.
\end{proof}

\begin{Rem}
	Были у нас функции $e^x$, $\sin x$, $\cos x$, определяемые без рядов.
	Потом мы из рядов создали $e^z$, $\sin z$, $\cos z$.
	Это единственные продолжения соотвесвтующих $e^x$, $\sin x$, $\cos x$, сохраняющие аналитичность.
\end{Rem}

\begin{theorem}[связь нулей и полюсов]
	$f \in H(\Omega)$, $a$ "--- полюс.
	Тогда существует $g$, голоморфная в окрествности $a$, что в ней
	\[ \forall z \ne a, f(z) = \frac1{g(z)} \]
	и $g(a) = 0$.
\end{theorem}
\begin{proof}
	Сначала проверим, что $f(z)$ в окрестности полюса не ноль.
	$a$ "--- полюс, значит $|f(z)| \xlongrightarrow{z \ra a} +\infty$, значит в некоторой оркестности $|f(z)| > 0$.
	Далее, $g$ автоматически определена в окрестности, а в самой точке $a$ по непрерывности доопределили $g(z) = 0$.

	Теперь, $g$ голоморфна в проколотой окрестности точки $a$ и непрерывна во всей окрестности, значит она логоморфна.
\end{proof}

\begin{Def}
	Кратность полюса $f$ "--- кратность соотвествующего нуля $1/f$.
\end{Def}

\begin{theorem}
	$f \in H(\Sigma \setminus \{a\})$.
	Тогда $a$ "--- полюс тогда и только тогда, когда в разложении в ряд Лорана в кольце с центром в точке $a$
	в главной части есть лишь конечное ненулевое число коэффициентов.
\end{theorem}
\begin{proof}\begin{description}
\item[$\Ra$:]
	$a$ "--- полюс, значит $g=1/f$ имеет ноль в $a$, значит
	\begin{gather*}
		g(z)
		= \sum_{n=k}^{+\infty} a_n (z - a)^n
		= (z - a)^k \underbrace{\sum_{n=0}^{+\infty} a_{n+k} (z - a)^n}_{\lrh h(z) \ne 0} \\
		f(z)
		= (z - a)^{-k} \underbrace{\frac1{h(z)}}_{\text{голоморфна}}
		= (z - a)^{-k} \sum_{n=0}^{+\infty} b_n (z - a)^n
		= \sum_{n=-k}^{+\infty} b_{n+k} (z-a)^n
	\end{gather*}

\item[$\La$:]
	\[
		f(z)
		= \sum_{n=-k}^{+\infty} a_n (z-a)^n
		= \underbrace{(z-a)^{-k}}_{\ra \infty} \underbrace{\sum_{n=0}^{+\infty} a_{n-k} (z-a)^n}_{\ra a_{n-k} \ne 0}
	\]
\end{description}\end{proof}

\begin{Rem}
	Для полной красоты осталось понять, что порядок полюса и есть $-\min\{k \mid a_k \ne 0\}$ из разложения в ряд Лорана.
	Но если посмотреть на доказательство, то та самая кратность корня и переходит, куда надо.
	Осталось понять, что $b_0 \ne 0$, но $b_0 = \frac1h(a) \ne 0$.
\end{Rem}

\begin{theorem}
	$f \in H(\Omega \setminus \{a\})$, $a$ "--- особая точка, $f(z) = \sum_{n=-\infty}^{+\infty} a_n (z-a)^n$.
	Тогда
	\begin{enumerate}
		\item $a$ устранимая особая точка тогда и только тогда, когда главная часть ряда Лорана обращена в ноль.
		\item $a$ полюс, если в главной части ряда Лорана лишь конечное ненулевое число ненулевых членов.
		\item $a$ существенная особая точка, если в главной части ряда Лорана бесконечно много ненулевых членов.
	\end{enumerate}
\end{theorem}

\begin{proof}
	А уже все доказали!
\end{proof}

\begin{theorem}[Сохоцкого]
	$a$ "--- существенная особая точка $f$.
	Тогда
	\[ \forall A \in \C, \exists z_n \ra a\colon f(z_n) \ra A \]
	или то же самое
	\[ \forall \epsilon > 0, \cl f(\{0 < |z - a| < \epsilon\}) = \C \]
\end{theorem}
То есть вокруг существенной особой точки функция ведёт себя абсолютно хаотично.
\begin{proof}
	$a$ "--- сущесвтенная особая точка, значит $f$ неограничена в её окрестности (иначе устранимая).
	\[ \exists z_n \ra a\colon |f(z_n)| \ra +\infty \]
	Пусть в $0 < |z - a| < \epsilon$ $f$ не принимает значение $A$.
	Тогда заведём $g(z) = \frac1{f(z) - A}$ "--- голоморфна в $0 < |z - a| < \epsilon$.
	Тогда $f(z) = A + \frac1{g(z)}$.
	Поймём, что $a$ "--- существенная особая точка $g$.
	Она точно не устранимая "--- иначе была бы или устранимой и для $f$ (в случае $g(a) \ne 0$) или пределом ($g(z) = 0$).
	Если она полюс, то $f \ra A$, противоречие.

	Тогда есть $z_n \ra a$, что $|g(z_n)| \ra +\infty$.
	Но тогда опять $f(z_n) \ra A$.
\end{proof}

Есть более мощный факт:
\begin{theorem}[Пикара]
	$a$ "--- сущесвтенно особая точка $f$.
	Тогда для любого $\epsilon > 0$ множество
	\[ f(\{0 < |z - a| < \epsilon\}) \]
	или равно $\C$, или $\C$ без \textit{одной} точки.
\end{theorem}
Доказывать не будем.

\begin{exmp}
	\[ e^{1/z} = \sum_{n=0}^{+\infty} \frac1{n!} \frac1{z^n} \]
	Можно доказать, что она достигает в окрестности нуля всех значений, кроме собсвтенно нуля.
\end{exmp}

\begin{Def}
	$\Omega$ "--- область. $f$ называется мероморфная, если сущесствует последовательность $\{a_n\} \subset \Omega$,
	что $f \in H(\Omega \setminus \{a_n\})$ и в $a_n$ находятся полюса.
	Можно думать, что в этих точках есть значение бесконечность.
\end{Def}

\subsection{Отступление про логарифм}

\begin{theorem}
	$\Omega$ "--- односвязная область, $f \in H(\Omega)$, $\forall z \in \Omega, f(z) \ne 0$.
	Тогда
	\[ \exists g \in H(\Omega)\colon e^g = f \]
\end{theorem}
\begin{proof}
	\begin{gather*}
		f \in H(\Omega)
		\Ra f' \in H(\Omega)
		\Ra \frac{f'}f \in H(\Omega)
		\Ra \text{$\frac{f'(z)}{f(z)}\d z$ "--- замкнута} \Ra \\
		\xLongrightarrow{\text{$\Omega$ односвзяная}} \text{$\frac{f'(z)}{f(z)}\d z$ "--- точная}
		\Ra \exists F \in H(\Omega)\colon F(z) = \frac{f'(z)}{f(z)}
	\end{gather*}
	Возьмём $g = F + c$. Хотим $e^g = f$.
	\begin{gather*}
		e^g = f
		\La e^{-g}f = 1 \\
		e^{-g}f = e^{-F}e^{-c}f
	\end{gather*}
	Достаточно показать, что $e^{-F}f = const \ne 0$.
	\begin{gather*}
		\left(e^{-F}f\right)'
		= -F'e^{-F}f + e^{-F}f'
		= e^{-F}\underbrace{(f' - Ff)}_{= 0}
		= 0
		g(z) = \ln |f(z)| + i \arg f(z) \in H(\Omega)
	\end{gather*}
	Подберём $c$ так, чтобы получить единицу.
\end{proof}
\begin{Rem}
	Хотим понять, как же выглядит эта обратная функция.
	\begin{gather*}
		z = e^w = e^{u+iv} = e^u (\cos v + i\sin v) \\
		|z| = e^u \Ra u = \ln |z| \\
		v = \arg z \quad \text{(с точностью до $2\pi$)} \\
		g(z) = \ln |f(z)| + i\arg f(z) \in H(\Omega)
	\end{gather*}
	Таким образом, $\arg f(z) \in H(\Omega)$, и в теореме можно выбрать такие углы, чтобы аргумент был непрерывен.
\end{Rem}

\begin{conseq}
	Рассмотрим плоскость $\C$ без кривой из нуля, идущей на бесконечность.
	На этой области существует логарифм $\log z$: хотим применить теорему выше, $f = z$ голоморфна и непрерывна.
	\begin{Rem}
		Есть формула $\log z = \ln |z| + i\arg z$, требующая непрерывность аргумента.
		У нас она есть, так как выбор значения в одной точке однозначно продолжается по непрерывности на всю плоскость,
		а бегать по кругу нам не даст кривая.
	\end{Rem}
	\begin{Rem}
		По этой же причине все варианты непрерывной функции аргумента отличаются на $2\pi k$.
	\end{Rem}
\end{conseq}

Обозначения:
\begin{itemize}
	\item $\ln x$ "--- логарифм положительных вещественных чисел.
	\item $\log z$ "--- хоть какое-то значение логарифма комплексного числа.
	\item $\arg z$ "--- хоть какой-то аргумент комплексного числа.
	\item $\Ln z$ "--- голоморфный логарифм, определённый на односвязной области, не содержащей ноль.
	\item $\Arg z$ "--- непрерывный аргумент, определённый на односвязной области, не содержащей ноль.
		Определяется областью и значением в одной точке.
\end{itemize}

\begin{conseq}
	$\Gamma$ "--- несамопересекающаяся кривая из 0 в бекснонечность.
	В $\C \setminus \Gamma$ существует функция $z^p \in H(\C \setminus \Gamma)$:
	\[ z^p = e^{p\Ln z} \]
	\begin{enumerate}
	\item
		$p \in \R$: $|z^p| = |z|^p$
		\begin{proof}
			\[ |z^p| = |e^{p\Ln z}| = e^{\Re p\Ln z} = e^{p\ln z} = |z|^p \]
		\end{proof}

	\item
		$p \in \Z$: всё как раньше.
		\begin{proof}
			\[ z^n = e^{n\Ln z} = e^{n(\ln |z| + i\arg z)} = e^{n\log z} \underbrace{e^{2i\pi k}}_{=1} = (e^{\log z})^n \]
		\end{proof}
	\end{enumerate}

	\begin{Rem}
		Не всегда $\Ln z + \Ln w = \Ln zw$, $z^pw^p = (zw)^p$.
		Если при этих соотношениях нигде нет прыжка через границу односвязной области, то нормально.
	\end{Rem}
\end{conseq}

\begin{exmp}
	\begin{gather*}
		\int\limits_{0}^{+\infty} \frac{t^{p+1}}{1+t}\d t = \cdots \quad 0 < p < 1 \quad t = x^2 \\
		\cdots = 2\underbrace{\int\limits_{0}^{+\infty} \frac{x^{2p-1}}{1+x^2} \d x}_{\lrh I} \\
		f(z) = \frac{e^{(2p-1)\Ln z}}{1+z^2} \quad \{z \mid \Im z \ge 0 \land z \ne 0\} \quad \Ln 1 = 0
	\end{gather*}
	Функция на изначальной области совпадает с нужной.

	Начнём считать интеграл по области: полуокружность большого радиуса $c_R$ против часовой, от неё к нулю,
	обходя его по окружности радиуса $c_{\epsilon}$.
	Единственная особая точка "--- $i$.
	\begin{gather*}
		\int\limits_{\Gamma_{R,\epsilon}} f(z) \d z = 2\pi i \res_{z=i} f(z) \\
		\res_{z=i} f(z)
		= \frac{e^{(2p-1)\Ln z}}{(1+z^2)'}\biggr|_{z=i}
		= \frac{e^{(2p-1)\Ln i}}{2i} \\
		\Ln i = \ln |i| + i \Arg i = \frac{i\pi}2 \\
		\int\limits_{\Gamma_{R,\epsilon}} f(z)\d z = \pi e^{(2p-1)\sfrac{i\pi}2} \\
		\int\limits_{\Gamma_{R,\epsilon}}
		= \int\limits_{c_R} + \int\limits_{c_\epsilon} + \underbrace{\int\limits_{\epsilon}^R}_{\ra I} + \int\limits_{-R}^{-\epsilon} \\
		|1 - z^2| \ge |z^2| - 1 \ge 1 - |z^2|
		\left| \int\limits_{c_R} \right|
		\le \pi R \max_{|z| = R \land \Im z \ge 0} |f(z)|
		\le \pi R \max \frac{|z|^{2p-1}}{|z|^2 - 1}
		= \pi R \frac{R^{2p-1}}{R^2 - 1} \ra 0 \\
		\left| \int\limits_{c_\epsilon} \right|
		\le \pi \epsilon \max_{|z| = \epsilon \land \Im z \ge 0} |f(z)|
		\le \pi \epsilon \max \frac{|z|^{2p-1}}{1 - |z|^2}
		= \pi \epsilon \frac{\epsilon^{2p-1}}{1 - \epsilon^2} \ra 0
	\end{gather*}
	Остался последний фрагмент
	\begin{gather*}
		\int\limits_{-R}^{-\epsilon} \\
		z = -x \quad \Ln z = \ln |z| + i\Arg z = \ln x + i \Arg (-x) = \ln x + \pi i \\
		f(-x)
		= \frac{e^{(2p-1)\Ln (-x)}}{1+(-x)^2}
		= \frac{e^{(2p-1)(\ln x + \pi i)}}{1+x^2}
		= \frac{x^{2p-1}e^{(2p-1)\pi i}}{1+x^2}
		\int\limits_{-R}^{\epsilon} f(z) \d z = e^{(2p-1)\pi i}\int\limits_{\epsilon}^R \ra e^{(2p-1)\pi i} I \\
		I \left(1 + e^{(2p-1)\pi i}\right) = e^{(2p-1)\sfrac{i\pi}2}\pi \\
		I = \frac{\pi e^{(2p-1)\sfrac{i\pi}2}}{e^{(2p-1)\pi i} + 1}
		= \dots = \frac{\pi}{2\cos{p\pi - \frac{\pi}2}}
		= \frac{\pi}{2\sin(p\pi)}
	\end{gather*}
\end{exmp}

\begin{exmp}
	\begin{gather*}
		I = \int\limits_0^1 \frac{\ln(1-t)}t \d t
		= 2 \int\limits_0^1 \frac{\ln(1-x^2)}x \d x
		= 2 \int\limits_0^1 \frac{\ln(1-x)}x \d x + 2\int\limits_0^1 \frac{\ln(1+x)}x \d x
		= 2I - 2\int\limits_{-1}^0 \frac{\ln(1-x)}x \d x \\
		\int\limits_{-1}^0 \frac{\ln(1-x)}x = \frac I2 \quad \int\limits_{-1}^1 \frac{\ln(1-x)}x = \frac{3I}2 \\
		f(z) = \frac{\Ln(1 - z)}z \quad \Ln 1 = 0 \quad \{z \mid \Im z \ge 0 \land z \ne 1\}
	\end{gather*}
	Бегаем по контуру: радиус $R$ против часовой до $-1$, потом почти до 1, огибаем её на растоянии $\epsilon$.
	Особых точек нет.
	\begin{gather*}
		\int\limits_{\Gamma_\epsilon} f(z) \d z = 0
		= \underbrace{\int\limits_{-1}^{1-\epsilon}}_{\ra \frac{3I}2} + \int\limits_{c_\epsilon} + \int\limits_c \\
		\left| \int\limits_{c_\epsilon} \right| \le \frac{\pi}2 \epsilon \max |f(z)| \le \frac{\pi\epsilon}2 \frac{|\ln \epsilon| + const}{\sfrac12} \ra 0 \\
		z = e^{i\phi}
		\int\limits_c = \int\limits_{\approx\epsilon}^\pi \frac{\ln |1-e^{i\phi}| i\Arg(1-e^{i\phi})}{e^{i\phi}} ie^{i\phi} \d \phi
		= \int\limits_{\approx\epsilon}^\pi \left(-\Arg(1-e^{i\phi}) + i\ln|1-e^{i\phi}|\right)
	\end{gather*}
	Перейдём везде к вещественной части. \TODO очень нужна картинка про углы.
	\begin{gather*}
		\frac{3I}2 + o(1) - \int\limits_{\approx\epsilon}^\pi \Arg (1 - e^{i\phi}) \d \phi = 0 \\
		\frac{3I}2 = \int\limits_{0}^\pi \Arg(1 - e^{i\phi}) \d \phi = \int\limits_0^\pi -\frac{\pi - \phi}2 \d \phi
		= -\frac{\pi^2}4
	\end{gather*}
\end{exmp}

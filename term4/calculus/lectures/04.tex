\begin{theorem}
	$f\colon \Omega \ra \C$, $f \in C(\Omega)$, но $f \in H(\Omega \setminus \Delta)$, где $\Delta$ "--- фрагмент горизонтальной прямой.
	Тогда $f(z) \d z$ всё равно замкнута.
\end{theorem}
\begin{proof}
	Режем на прямоугольнкики. Если прямая не задела "--- смотри на предыдущий случай.
	Если задела "--- разрежем по прямой. Для каждого такого прямоугольника отступим на прямой на $\epsilon$.
	Для таких прямоугольников всё доказано, а дальше по непрерывности хотим показать, что интегралы сойдутся.
	\[
		\int\limits_P - \int\limits_{P_\epsilon} = \int\limits_{\text{боковые фрагменты}} + \int\limits_{\text{сторона по прямой}} (f(z) - f(z - i\epsilon))
	\]
	Первое можно оценить как $2\epsilon \max |f|$, а второе как $\delta \cdot \text{длина стороны}$
\end{proof}

\begin{conseq}
	Если $f$ голоморфна в $\Omega$ за исклчением некоторого количества точек, не имеющих предела,
	то тоже форма замнута.
\end{conseq}
\begin{proof}
	Раз нет предельной точки, то либо можно выбрать окрестность без них, либо она совпадает с центром окрестности.
	Во втором случае вырежем целый диаметр, и смотри предыдущую теорему.
\end{proof}

\begin{Def}
	$\gamma\colon [b, c] \ra \C$ замкнута кривая, $a$ не лежит на $\gamma$.
	Возьмём полярное отображение кривой в полярные координаты из точки $a$ в виде $(r(t), \phi(t))$.
	Теперь назовём индексом кривой относительно $a$
	\[ \Ind (\gamma, a) = \frac{\phi(c) - \phi(b)}{2\pi} \in \Z \]
	В силу замкнутости, $\phi(c) - \phi(b)$ делится нацело на $2\pi$.
	Неформально "--- сколько оборотов вокруг точки делает кривая.

	Альтернативно, можно считать индекс так (не доказываем, что это так):
	проводим луч из точки, каждый раз, когда кривая проходит через луч справа налево, прибавляем 1, когда слева направо, вычитаем 1.
\end{Def}

\begin{theorem}
	$\gamma$ "--- замкнутая кривая, не проходящая через 0. Тогда
	\[ \int\limits_\gamma \frac{d z}z = 2\pi i \Ind(\gamma, 0) \]
\end{theorem}
\begin{proof}
	\begin{gather*}
		z = re^{i\phi} \quad \d z = (r'(t) e^{i\phi(t)} + i\phi'(t)r(t)e^{i\phi(t)}) \d t \\
		\frac{d z}z = \left(\frac{r'(t)}{r(t)} + i\phi'(t)\right) \d t \\
		\int\limits_\gamma \frac{\d z}z = \int\limits_\gamma \left(\frac{r'(t)}{r(t)} + i\phi'(t) \right)\d t
		= \left(\ln r(t) + i\phi(t)\right)\biggr|_{t=b}^c
		= 2\pi i\Ind(\gamma, a)
	\end{gather*}
\end{proof}

\begin{conseq}
	\[ \Ind(\gamma, a) = \frac1{2\pi i} \int\limits_\gamma \frac{\d z}{z-a} \]
\end{conseq}

\begin{theorem}[Интегральная теорема Коши]
	$f \in H(\Omega)$, $a \in \Omega$, $\gamma$ стягивается и не проходит через $a$.
	Тогда
	\[ \int\limits_\gamma \frac{f(z)}{z-a} \d z = 2\pi i f(a) \Ind(\gamma, a) \]
\end{theorem}
\begin{proof}
	\[ g(z) = \begin{cases} \frac{f(z) - f(a)}{z-a} & z \ne a \\ f'(a) & z = a \end{cases} \]
	$g \in C(\Omega)$, $g \in H(\Omega \setminus \{a\})$.
	Тогда $f(z) \d z$ замнутая и
	\begin{gather*}
		0 = \int\limits_\gamma g(z) \d z
		= \int\limits_\gamma \frac{f(z) - f(a)}{z - a} \d z
		= \int\limits_\gamma \frac{f(z)}{z-a} \d z - f(a) \int\limits_\gamma \frac{\d z}{z - a}
	\end{gather*}
\end{proof}

\begin{conseq}
	$f$ "--- голоморфная в окрестности единичного круга.
	\[
		\int\limits_{\mathbb T} \frac{f(z)}{z - a} \d z = \begin{cases}
			2\pi i f(a) & a \in \mathbb D \\
			0 & a \notin \bar {\mathbb D}
		\end{cases}
	\]
\end{conseq}

\begin{theorem}
	$f \in H(r\mathbb D)$.
	Тогда $f$ аналитична в $r\mathbb D$.
\end{theorem}

\begin{proof}
	$0 < r_1 < r_2 < r$, $|\zeta| = r_2$, $|z| < r_1$.
	\begin{gather*}
		f(z) = \frac1{2\pi i} \int\limits_{r_2\mathbb T} \frac{f(\zeta)}{\zeta - z} \d \zeta = \dots \\
		\frac1{\zeta - z} = \frac1\zeta \cdot \frac1{1- \frac z\zeta} = \sum_{i=0}^\infty \frac{z^n}{\zeta^{n+1}} \\
		\dots = \frac1{2\pi i} \int\limits_{r_2 \mathbb T} f(\zeta) \sum_{i=0}^\infty \frac{z^n}{\zeta^{n+1}} \d \zeta = \dots
	\end{gather*}
	Равномерно сходится: мажорируется рядом $\sum_{n=0}^\infty \max |f| \left(\frac{r_1}{r_2}\right)^n$
	\begin{gather*}
		\dots = \frac1{2\pi i} \sum_{n=0}^\infty \int\limits_{r_2 \mathbb T} \frac{f(\zeta)}{\zeta^{n+1}} \d \zeta \cdot z^n \\
		a_n = \frac1{2\pi i} \int\limits_{r_2 \mathbb T} \frac{f(\zeta)}{\zeta^{n+1}} \d \zeta
	\end{gather*}
\end{proof}

\begin{conseq}
	$f$ толоморфна тогда и только тогда, когда она аналитична.
\end{conseq}
\begin{proof}
	Берём окрестность точки, берём в ней круг, там теорема.
\end{proof}

\begin{conseq}
	Если $f$ голоморфна, то она бесконечно дифференцируема.
\end{conseq}

\begin{conseq}
	$f \in H(\Omega) \Ra f' \in H(\Omega)$.
\end{conseq}

\begin{conseq}
	Если $f \in H(\Omega)$, то $\Re f$ и $\Im f$ гармонические.
\end{conseq}

\begin{theorem}[Морера]
	$f\colon \Omega \ra \C$, $f(z) \d z$ замкнута.
	Тогда $f$ голомофрна.
\end{theorem}
\begin{proof}
	$z_0 \in \Omega$, $B$ "--- круг с центром в $z_0$, в котороем есть первообразная.
	$F$ "--- первообразная, значит $F$ голоморфна в $B$, значит по 3 следствию производная тоже голоморфна, а это $F' = f$.
\end{proof}

\begin{conseq}
	$f \in C(\Omega) \land f \in H(\Omega \setminus \Delta) \Ra f \in (\Omega)$.
\end{conseq}

\begin{conseq}
	$f \in C(\Omega)$, $f$ голоморфна, кроме некоторого набора точек без предельной.
	Тогда $f$ голоморфна на всей $\Omega$.
\end{conseq}

\begin{theorem}[Интегральная теорема Коши, вторая версия]
	$f \in H(\Omega)$, $K \subset \Omega$ "--- компакт с кусочно гладной границей.
	Тогда
	\[
		\int\limits_{\partial K} \frac{f(z)}{z - a} \d z = \begin{cases}
			2\pi i f(a) & a \in \Int K \\
			0 & a \notin K
		\end{cases}
	\]
\end{theorem}
\begin{proof}
	Пусть $a \notin K$.
	\begin{gather*}
		\frac{f(z)}{z-a} \in H(B_\epsilon(K))
		\Ra \partd{}{\bar z} \left( \frac{f(z)}{z-a} \right) = 0 \\
		\int\limits_{\partial K} \frac{f(z)}{z-a} \d z = 2i \int\limits_K \partd{}{\bar z} \left(\frac{f(z)}{z-a}\right)\d x\d y = 0
	\end{gather*}

	Пусть $a \in \Int K$.
	Рассмотрим $\tilde K = K \setminus \bar B_r(a)$.
	Мы уже знаем, что
	\[ \int\limits_{\partial \tilde k} \frac{f(z)}{z-a} \d z = 0 \]
	но
	\begin{gather*}
		\int\limits_{\partial \tilde k} \frac{f(z)}{z-a} \d z
		= \int\limits_{\partial K} \frac{f(z)}{z-a} \d z
		- \underbrace{\ointctrclockwise\limits \frac{f(z)}{z-a}}_{=2\pi i f(a)}
	\end{gather*}
\end{proof}

\textbf{Упражнение.}
\TODO

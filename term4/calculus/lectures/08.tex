\begin{theorem}
	Если $f$ меломорфна, то $f'$ тоже меломорфна.
	Полюса у $f$ и $f'$ в одних и тех же точках.
	Если у $f$ полюс $r_0$ порядка $n$, то у $f'$ он порядка $n+1$.
\end{theorem}
\begin{proof}
	Если $f$ голоморфна на открытом множестве, то $f'$ там тоже голоморфна,
	значит множество особых точек $f'$ содержится в множестве таковых у $f$.

	Пусть $z_0$ "--- особая точка $f$.
	Если это полюс порядка $n$, то в окрестности есть рахложение
	\begin{gather*}
		f(z)
		= \sum_{k=-n}^{+\infty} c_k(z-z_0)^k
		= \frac{c_{-n}}{(z-z_0)^n} + \frac{c_{-n+1}}{(z-z_0)^{n-1}} + \dots +  \frac{c_{-1}}{z - z_0} + \sum_{k=0}^{+\infty} c_k (z-z_0)^k \\
		f'(z)
		= \frac{-nc_{-n}}{(z-z_0)^{n+1}} + \dots + \frac{-c_{-1}}{(z-z_0)^2} + \sum_{k=0}^{+\infty} kc_k (z-z_0)^{k-1}
	\end{gather*}
	Значит $z_0$ "--- полюс $f'$ порядка $n+1$.
\end{proof}

\begin{exmp}
	\[ \frac{\sin z}z \quad \frac{\cos z - 1}{z^2} \]
	0 "--- устранимая особая точка.
	\[ \frac{e^z}z \quad \frac{\sin z}z \]
	0 "--- полюс 10-го и 4-го порядка.
	\[ e^{\sfrac1z} \quad \sin\frac1z \quad \cos\frac1z \]
	0 "--- существенная особая точка.
\end{exmp}

\section{Вычеты}

\begin{Def}[Бесконечный предел]
	\[\lim_{n \ra +\infty} z_n = \infty \]
	если $\lim_{n \ra +\infty} |z_n| = +\infty$.
\end{Def}

\begin{enumerate}
	\item $\lim_{n \ra +\infty} z_n = \infty \Ra \lim_{n \ra +\infty} \frac1{z_n} = 0$
	\item $\lim_{n \ra +\infty} z_n = \infty \land |w_k| \ge \delta > 0 \Ra \lim_{n \ra +\infty} z_nw_n = \infty$
	\item $\lim_{n \ra +\infty} z_n \land \text{$w_k$ ограничена} \Ra \lim_{n \ra +\infty} (z_n \pm w_n) = \infty$
	\item $\lim_{n \ra +\infty} z_n \land \text{$w_k$ ограничена} \Ra \lim_{n \ra +\infty} \frac{w_n}{z_n} = 0$
\end{enumerate}

\begin{Def}
	Предел $\lim_{n \ra +\infty} z_n = A$ можно определить по Коши или по Гейне.
\end{Def}

\begin{Rem}
	$\bar \C = \C \cup \{\infty\}$
\end{Rem}

Сфера Римана: давайте поставим на 0 комплексной плоскости сферу диаметра 1:
\[ u^2 + v^2 + \left(w - \frac12\right)^2 = \frac14 \]
Построим отображение $\C$ на эту сферу: проводим в точку $x$ на плоскости луч из северного полюса, смотрим точку пересечения сферы.
Сам северный полюс отображается на $\infty$.

\begin{theorem}
	При отображении выше точка $z = x + iy$ отображается в
	\[ u = \frac x{1+|z|^2} \quad v = \frac y{1+|z|^2} \quad w = \frac{|z|^2}{1 + |z|^2} \]
\end{theorem}
Доказать: упражнение, вспомнить 11 класс.

\begin{conseq}
	$z, \tilde z \in \C$.
	Расстояние между ними на сфере
	\[ \frac{|z - \tilde z|}{\sqrt{1+|z|^2}\sqrt{1+|\tilde z|^2}} \]
\end{conseq}
\begin{proof}
	Тупо раскрываем скобочки.
\end{proof}

\begin{conseq}
	Расстояние между $z \in \C$ и бесконечностью на сфере:
	\[ \frac1{\sqrt{1+|z|^2}} \]
\end{conseq}
\begin{proof}
	\[
		u^2 + v^2 + (w-1)^2
		= \frac{x^2}{(1+|z|^2)^2} + \frac{y^2}{(1+|z|^2)^2} + \frac1{(1+|z|^2)^2}
		= \frac1{1+|z|^2}
	\]
\end{proof}

\begin{conseq}
	Cходимость в $\bar \C$ равносильна сходимости на сфере Римана.
\end{conseq}
\begin{proof}
	Если $z_n \ra z_0$ в $\bar C$, то видно, что формулы выше стремятся к нулю.

	Обратно: пусть $|z_n - z| \nrightarrow 0$, но $A \lrh \frac{|z - \tilde z|}{\sqrt{1+|z|^2}\sqrt{1+|\tilde z|^2}}$.
	Тогда $|z_{n_k} - z_0| \ge \epsilon > 0$, и
	\[ \sqrt{1+|z_{n_k}|^2} \ra +\infty \Ra z_{n_k} \ra \infty \ra A \ra 1 \ne 0\]
	Если возьмём $\frac1{\sqrt{1+|z|^2}} \ra 0$, то очевидно $z_n \ra \infty$.
\end{proof}

\begin{conseq}
	$\bar \C$ компакт.
\end{conseq}

\begin{Def}
	Пусть $f$ голомофрна в окрестности $\infty$.
	Тогда $\infty$:
	\begin{description}
		\item[устранимая особая точка] если есть конечный предел,
		\item[полюс] если есть бесконечный предел,
		\item[существенная особая точка] если предела нет.
	\end{description}
\end{Def}

\begin{Rem}
	Если $f$ голоморфна вне круга радиуса $R$, то она расскладывается там в ряд Лорана,
	и $g = \sfrac1f$ раскладывается в ряд Лорана в круге радиуса $\sfrac1R$ без центра.
\end{Rem}

\begin{theorem}\begin{enumerate}
\item
	Следующее равносильно:
	$\infty$ "--- устранимая особая точка $f$,
	$f$ ограничена в окрестности $\infty$,
	$f$ в ряде Лорана не имеет членов с положительными степенями.

\item
	$\infty$ полюс тогда и только тогда, когда в ряде Лорана коенчное число ненулевых членов с полодительными степенями,
	и наибольшая такая степень равна порядку полюса.

\item
	$\infty$ существенная особая тока тогда и только тогда, когда в ряде Лорана бесконечно много ненулывых членов с положительной степенью.
\end{enumerate}\end{theorem}

\begin{Def}
	$f$ голоморна в $\infty$, если она голоморфна в окрестности $\infty$ и $\infty$ "--- устранимая особая точка.
\end{Def}

Теперь можно много предыдущих теорем переписать для $\bar \C$.
Например:

\begin{theorem}[Лиувилля]
	$f \in H(\bar \C) \Ra f \equiv const$
\end{theorem}
\begin{Def}
	\[ f \in H(\bar C) \Ra f \in H(\C) \land f \in C(\C) \]
	Непрерына на компакте, значит ограничена, и ссылаемся на старую теорему.
\end{Def}

\begin{Def}
	$f$, $z_0 \ne \infty$ "--- изолированная особая точка $f$.
	Вычетом функции $f$ в точке $z_0$ называется
	\[ \res_{z = z_0} f \eqDef c_{-1} \]
	где
	\[ f(z) = \sum_{n=-\infty}^{+\infty} c_n(z-z_0)^n \]
\end{Def}
Иногда так писать неудобно, потому что функция всегда одна и та же, а точки разные, и пишут наоборот "--- точку справа, функцию внизу.

\begin{Def}
	\[ \res_{z = \infty} f \eqDef -c_{-1}\]
\end{Def}

\begin{theorem}[Коши о вычетах]
	Пусть $f$ голомофрна в $\Omega$ без какого-то конечного числа особых точек $a_j$,
	$K \subset \Omega$ "--- компакт, и особые точки не лежат на его границе.
	Тогда, внезапно,
	\[ \int\limits_{\partial K} f(z) \d z = 2\pi i\sum_{a_j \in \Int K} \res_{z=a_j} f \]
	То есть посчитать интеграл можно просто сложив несколько вычетов.
\end{theorem}

\begin{lemma}
	$f(z) = \sum_{n=-\infty}^{+\infty} c_n z^n$ сходится в кольце $R_1 < |z| < R_2$.
	Пусть $R_1 < r < R_2$.
	Тогда
	\[ \int\limits_{|z| = r} f(z) \d z = 2\pi i c_{-1} = 2\pi i\res_{z=0} f \]
\end{lemma}
\begin{proof}
	На окружности сходится равномерно,
	\begin{gather*}
		\int\limits_{|z| = r} f(z)\d z
		= \sum_{n=-\infty}^{+\infty} \int\limits_{|z|=r} z^n \d z \\
		\int\limits_{|z| = r} z^n \d z
		= \int\limits_0^{2\pi} rie^{i\phi}r^nr^{in\phi} \d \phi
		= ir^{n+1} \int\limits_0^{2\pi} e^{i(n+1)\phi} \d \phi
		= \begin{cases} 2\pi i & n = -1 \\ 0 & n \ne -1 \end{cases}
	\end{gather*}
\end{proof}

\begin{proof}
	Возьмём компакт $\bar K$, равный $K$ без окрестностей особых точек $B_j$.
	Ориентируем его границу так, чтобы область всегда была слева по обходу.
	Внутри него вообще нет особых точек, и интеграл по его границе 0.
	\begin{gather*}
		\int\limits_K f(z) \d z
		= \int\limits_{\bar K} f(z) \d z + \sum_{a_j \in \Int K} \int\limits_{|z-a_j|=r_j} f(z) \d z
		= 0 + \sum_{a_j \in \Int K} 2\pi ic_j
	\end{gather*}
\end{proof}

\begin{conseq}
	$f$ голомофрна в $\C$, кроеме конечного числа особых точек.
	Тогда сумма вычетов $f$ в этих точках и на бесконечности равна нулю.
\end{conseq}
\begin{proof}
	Возьмём достаточно большой круг.
	Теорема нам говорит, что интергал по границе круга равен сумме вычетов всех особых точек.
	С другой стороны, по лемме этот же интеграл равен коэффициенту при $z^{-1}$ в разложении
	\[ f(z) = \sum_{n=-\infty}^{+\infty} c_n z^n \]
	и он по определению равен $-\res_{z=\infty} f$.
\end{proof}

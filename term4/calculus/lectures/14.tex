\TODO Глеб

\begin{Rem}
	$f \in H(\Omega)$, $f'(a) \ne 0$, $a \in \Omega$.
	Тогда существует окрестность точки $a$ в которой $f$ однолистна.
\end{Rem}

\begin{proof}
	\begin{gather*}
		f'(a) \ne 0 \Ra \forall z \in B_{\delta}(a), |f'(z)| > \epsilon \\
		f(z) = f(w) + f'(w) (z-a) + O((z-w)^2)
	\end{gather*}
	Стоит не просто ошка, а оценка $C(z-w)^2$ с универсальной для всех точек константой.
	Если найдём такое, то в радиусе $C|z-w| < \epsilon$ всё будет хорошо.
	Такое есть из соображений: посмторим на Тейлора, там будет несколько слагаемых для производных второго порядка, из них выводится.
\end{proof}

\begin{Rem}
	Если $f \in H(\Omega)$ и $\forall z \in \Omega, f'(z) \ne 0$, то $f$ может не быть однолистной.
\end{Rem}

\begin{proof}
	$f(z) = e^z$
\end{proof}

\begin{Def}
	$\Omega_1$, $\Omega_2$ "--- области.
	Их называют комформно эквивалентными, если существует комформная биекция $f\colon \Omega_1 \ra \Omega_2$.
\end{Def}

\begin{theorem}
	$\C$ и $\mathbb D = \{|z| < 1 \}$ комформно не эквивалентны.
\end{theorem}

\begin{proof}
	Пусть есть такое отображение $f\colon \C \ra \mathbb D$, значит оно голоморфно на всей плоскости и ограничено, значит по теореме Лиувилля константна.
\end{proof}

\begin{theorem}[Римана о конформных отображениях]
	$\Omega$, $\tilde \Omega$, \TODO
\end{theorem}

\begin{proof}
	Cуществование доказывать не будем 
\end{proof}

\TODO

\begin{conseq}
	Пусть $f$ голоморфна на всей плоскости, а образ не содержит какую-то кривую.
	Тогда она константа.
\end{conseq}
\begin{proof}
	$\gamma$ "--- та самая кривая. По теореме Римана $\exists g\colon \bar \C \setminus \gamma \ra \mathbb T$ конформная.
	Рассмотрим $h(z) = g(f(z))$, $h\colon \C \ra \mathbb D$. Отсюда $h = const$ и $f=const$.
\end{proof}

\begin{Rem}
	По конформной эквивалентности односвязные области делятся на три класса:
	\begin{enumerate}
		\item $\bar \C$
		\item $\bar \C \setminus \{z_0\}$: $f\colon \bar C \setminus \{z_0\} \ra \C$, $f(z) = \frac1{z-z_0}$.
		\item Всё остальное: теорема Римана.
	\end{enumerate}
	2 и 3 не эквивалентны, так как $\C$ и $\mathbb D$.
	1 и 3 не эквивалентны, так как если $f\colon \bar C \ra \mathbb D$ конформна, то $f = const$.
	1 и 2 не эквивалентны, так как $f \in H(\bar \C) \Ra f = const$.
\end{Rem}

\begin{Def}
	Дробно-линейное отображение или преобразование (ДЛП):
	\[ f(z) = \frac{az+b}{cz+d} \quad ad-bc \ne 0 \]
\end{Def}

Упражнение: однолистно.

\begin{theorem}
	Если $f \in H(\bar \C \setminus \{z_0\})$ и однолистно, то $f$ дробно-линейно.
\end{theorem}
\begin{gather*}
	Посмотрим на эту особую точку $z_0$.
	Она не может быть устранимой особой точкой, иначе можно продолжить, и по Луивиллю константа.
	Она не может быть существенной особой точкой.
	От противного: ну пусть. Рассмотрим окрестность вокруг $z_0$ $W$ и кольцо $W_1$ вокруг окрестности.
	По теореме Соковского $\cl f(W) = \C$, но вне этого замыкания образа должен быть открытый образ $f(W_1)$, противоречие.

	Значит полюс, но не может быть полюсом 2 порядка или выше.
	Иначе $\sfrac 1{f(z)}$ однолистна в окрестности $z_0$, но $\left(\sfrac 1{f(z)}\right)'|_{z=z_0} = 0$.
	Ура.

	\[ g(z) = f(z) - \frac A{z - z_0} \]
	Она имеет устраниую особую точку в $z_0$, значит опять по Луивиллю константа.
	Значит $f(z) = B + \frac A{z-z_0}$
\end{gather*}

\begin{theorem}
	Любое конформное отображение из $\mathbb D$ в $\mathbb D$ имеет вид
	\[ f(z) = e^{i\theta} \frac{z-a}{1-z\bar a} \]
	где $\theta \in \R$, $a \in \mathbb D$.
\end{theorem}
\begin{proof}
	Покажем, что они подходят.
	Возьмём точку на единичной окружности и посмотрим, куда она переходит.
	\[ 
		\left| f(e^{i\phi}) \right|
		= \left| e^{i\theta} \frac{e^{i\phi} - a}{1-e^{i\phi}\bar a} \right|
		= \left| e^{i(\theta - \phi)} \frac{e^{i\phi} - a}{e^{-i\phi} - \bar a}\right|
		= 1
	\]
	Значит граница в границу. Значит всё внутри должно перейти внутрь, а точка $a$ перешла в точку $0$.
	Значит действительно подходит.

	Почему других не бывает. Зафиксируем, что $a \ra 0$ и $\arg f'(a) = \theta$.
	Если $g \colon \mathbb D \ra \mathbb D$ "--- конформна, то по единственности в теореме Римана $f = g$.
\end{proof}

Упражнения.
\begin{enumerate}
\item
	Композиция ДЛП "--- ДЛП, причём тут замешано произведение матриц: коэффициенты у $f_2(f_1(z))$ считаются так:
	\[
		\begin{pmatrix} \tilde a & \tilde b \\ \tilde c & \tilde d \end{pmatrix}
		= \begin{pmatrix} a_1 & b_1 \\ c_1 & d_1 \end{pmatrix}
		\begin{pmatrix} a_2 & b_2 \\ c_2 & d_2 \end{pmatrix}
	\]

\item
	Любое конформное отображение верхней полуплоскости в $\mathbb D$ имеет вид
	\[ e^{i\theta} \frac{z-a}{z-\bar a} \]
	где $\theta \in \R$, $\Im a > 0$.
\item
	Любое конформное отображение верхней полуплоскости в $\mathbb D$ имеет вид
	\[ e^{i\theta} \frac{z-a}{z-\bar a} \]
	где $\theta \in \R$, $\Im a > 0$.
\end{enumerate}

Почти всё из книжек
<<Картан. [Очень много слов про комплексные функции от одной и многих переменных]>>,
<<Лаврентоьев, Шабат. Методы теории функций комлексной переменной>>
и <<Евграфов. Аналитические функции>>.

\chapter{17.03.2016}

\section{Организационное}
	Что-то нас становится мало, не надо так, а то Илья начнёт посещаемость отмечать.

\section{Задача 1}
	Пусть $w = \sfrac{2}{z}$.
	Найдите образы кривых ($z=x+iy$):
	\begin{enumerate}[label=\asbuk{enumi}.]
		\item $x^2+y^2-2x=0$
		\item $(x-1)^2+y^2=4$
		\item $x=1$
		\item $y=0$
		\item Область: $0 < \Re z < 1$
	\end{enumerate}

	\subsection{а}
		\begin{Rem}
			Напоминаем, что уравнение окружности на комплексной плоскости выглядит так:
			\[
				A\underbrace{\bar z z}_{x^2+y^2} +
				B\underbrace{\frac{z+\bar z}{2}}_{x} +
				С\underbrace{\frac{z-\bar z}{2}}_{y} +
				D = 0
			\]
			Тут четыре переменных, но уравнение однородное, так что три степени свободы.
			Его можно получить из обычного уравнения окружности:
			\begin{gather*}
				(x-x_0)^2+(y-y_0)^2=R^2 \\
				x^2-2xx_0+x_0^2+y^2-2yy_0+y_0^2=R^2 \\
				\underbrace{1}_{A}\cdot(x^2+y^2)+x\cdot\underbrace{(-2x_0)}_{B}+y\cdot\underbrace{(-2y_0)}_{C}+\underbrace{(x_0^2+y_0^2-R^2)}_{D} = 0
			\end{gather*}
			Разумеется, в обратную сторону тоже можно: по коэффициентам получить описание окружности.
		\end{Rem}
		Переписываем исходное уравнение:
		\begin{gather*}
			x^2 + y^2 - 2x = 0 \\
			(x-1)^2 + y^2 = 1 \\
			\bar z z - 2\frac{z+\bar z}{2} + 0\frac{z-\bar z}{2} + (1^2+0^2-1^2) = 0 \\
			\bar z z - z - \bar z = 0 \\
			w = \frac{2}{z} \quad z = \frac{2}{w} \\
			\frac{2}{\bar w} \cdot \frac{2}{w} - \frac{2}{w} - \frac{2}{\bar w} = 0 \\
			\frac{4}{\bar w w} - \frac{2\bar w + 2w}{\bar w w} = 0 \\
			4 - 2\bar w - 2w = 0 \\
			w + \bar w = 2 \\
			\Re w = 1
		\end{gather*}
		Получили вертикальную прямую $x=1$.

	\subsection{б}
		\begin{gather*}
			(x-1)^2+y^2=4=2^2 \\
			\bar z z - 2\frac{z+\bar z}{2} + 0 + (1^2+0^2-2^2) = 0 \\
			\bar z z - z - \bar z = 3 \\
			z = \frac{2}{w} \\
			\frac{4}{\bar w w} - 2\frac{w + \bar w}{\bar w w} = 3 \\
			4 - 2(w + \bar w) = 3 \bar w w \\
			3 \bar w w + 4\frac{w + \bar w}{2} - 4 = 0 \\
			\bar w w - 2\cdot\frac{-2}{3}\cdot\frac{w + \bar w}{2} - \frac{4}{3} = 0 \\
			\text{получили окружность с центром в $(-\sfrac23, 0)$, считаем радиус}
			-\frac{4}{3} = \left(\frac{-2}{3}\right)^2 - R^2 \\
			R^2 = \frac49+\frac43 = \frac{16}{9} \\
			R^2 = \frac{4}{3}
		\end{gather*}
		Получаем окружность с центром в $(-\sfrac{2}{3}, 0)$ и радиусом $\sfrac 4 3$.

	\subsection{в}
		Замечаем, что у нас в условии ровно образ из пункта <<а>>, а
		$w(z)$ "--- инволюция ($w(w(z))=z$).
		То есть образ в пункте <<в>> есть условие пункта <<а>> "--- окружность с центром в $(1, 0)$.

	\subsection{г}
		\begin{gather*}
			y = 0 \\
			\frac{z + \bar z}{2} = 0 \\
			z + \bar z = 0 \\
			\frac{2}{w} + \frac{2}{\bar w} = 0 \\
			\frac{2\bar w + 2w}{\bar w w} = 0 \\
			2\bar w + 2w = 0 \\
			w + \bar w = 0 \\
			x = 0
		\end{gather*}
		Прямая переходит сама в себя.

	\subsection{д}
		Чтобы понять, куда переходит область, надо сначала понять, куда переходит граница,
		а потом проверить, в какой из двух кусков перешла область.
		Что граница переходит в границу "--- теорема (кажется, из конца второго семестра?).

		Была граница из двух прямых: $x=0$, $x=1$, первая перешла сама в себя, а вторая "--- в
		кружочек вокруг $(1, 0)$.
		Дальше можно взять произвольную точку из полосы и посмотреть, куда она перейдёт,
		а потом просто заштриховать кусок плоскости, ограниченный границами (у нас гарантированно
		компонента связности переходит в компоненту связности).
		Единственный кусок плоскости, который граничит с ними обоими "--- полуплоскость
		без кружочка:
		\cautoimg{06_p1}

\section{Задача 2}
	Пусть $w_1$, $w_2$, $w_3$ "--- различные точки в $\C$,
	а $z_1$, $z_2$, $z_3$ "--- еще три различные точки в $\C$.
	Докажите, что существует и единственно дробно-линейное преобразование (ДЛП) $w$,
	которое переводит $w(z_k)=w_k$.
	\begin{Rem}
		Напоминаем, что ДЛП "--- это преобразование вида
		\[ z \to \frac{az+b}{cz+d} \]
		Также можно задавать его трёмя коэфициентами вместа четырёх, как-нибудь отнормировав все коэффициенты.
	\end{Rem}

	\subsection{Существование}
		Давайте сначала построим ДЛП (а точнее "--- уравнение для него), которое переводит $z_1$ в $w_1$:
		\[ w - w_1 = z - z_1 \]
		Теперь добавим точку $w_2$.
		Например, скажем, что обе стороны уравнения должны на ней обращаться в бесконечность:
		\[
			\frac{w-w_1}{w-w_2} = \frac{z-z_1}{z-z_2}
		\]
		А теперь давайте это на что-нибудь домножим, чтобы в точке $w_3$ равенство тоже сохранялось:
		\begin{equation}\label{mobius_three_points}
			\frac{w-w_1}{w-w_2} \cdot \frac{w_3-w_2}{w_3-w_1} = \frac{z-z_1}{z-z_2} \cdot \frac{z_3-z_2}{z_3-z_1}
		\end{equation}
		Стоит проверить, что это ДЛП, раскрыв скобки и приведя подобные слагаемые:
		\TODO
		\begin{Rem}
			\TODO а точка <<бесконечность>> и деление на ноль "--- нормально вообще или просто интуитивный способ вывести, который надо проверить?
			Непонятно, спросил у Ильи.
		\end{Rem}

	\subsection{Единственность}
		Пусть есть преобразование $w_1$: $w_1(z_k)=w_k$.
		Предположим, что есть другое преобразование $w_2 \neq w_1$.
		Так как ДЛП образуют группу, возьмём обратное к нему, $\xi_2$: $\xi_2(w_k)=z_k$.
		Возьмём их композицию, это тоже ДЛП:
		\[ \xi_2(w_1(z)) = \frac{az+b}{cz+d} \]
		Мы знаем, что оно оставляет точки $z_k$ на месте:
		\begin{gather*}
			z_k = \frac{az_k+b}{cz_k+d} \\
			cz_k^2+dz_k = az_k+b \\
			cz_k^2+(d-a)z_k - b = 0
		\end{gather*}
		Это квадратное уравнение относительно $z_k$, у него три различных корня.
		Значит, оно "--- тождественный ноль.
		Следовательно, $c=0$, $d-a=0 \iff a = d$, $b=0$, т.е. 
		\[ \xi_2(w_1(z)) = \frac{az_k}{a} = z_k \]
		Получили, что $\xi_2$ и $w_1$ являются обратными друг для друга.
		Но в группе обратный элемент единственен, стало быть, $w_1=w_2$.

\section{Задача 3}
	\begin{enumerate}[label=\asbuk{enumi}.]
		\item
			Найдите ДЛП, которое переводит:
			\begin{align*}
				w(0) &= 4\\
				w(1+i) &= 2+2i\\
				w(2i) &= 0
			\end{align*}
			А потом найдите образ кружочка $|z-i|<1$.
	\end{enumerate}

	\subsection{а}
		Давайте воспользуемся формулой \ref{mobius_three_points} и посчитаем в лоб.
		\begin{gather*}
			\frac{w-w_1}{w-w_2} \cdot \frac{w_3-w_2}{w_3-w_1} = \frac{z-z_1}{z-z_2} \cdot \frac{z_3-z_2}{z_3-z_1} \\
			\begin{aligned}
			\end{aligned}
		\end{gather*}
		\TODO

		Получится такое уравнение между $z$ и $w$:
		\[
			z = \frac{i(4-w)}{2}
		\]

	\subsection{б}
		Подставим уравнение в условие на кружочек:
		\begin{gather*}
			|z - i| < 1 \\
			\left|\frac{i(4-w)}{2} - i\right| < 1\\
			|i|\cdot\left|\frac{4-w}{2} - 1\right| < 1\\
			\left|\frac{2-w}{2}\right| < 1\\
			|2-w| < 2\\
		\end{gather*}
		Получаем, что кружочек радиуса 1 с центром в $i$ перешёл в кружочек радиуса 2 с центром в $2$.

\section{Задача 4}
	Найдите ДПЛ $w$, которое переводит открытый круг $\mathcal{D}$ в левую половину плоскости:
	\[
		\{ |z| < 1 \} \to \{ \Re z < 0 \}
	\]

	Давайте сначала найдём ДЛП, которое перегонит границу кружочка (окружность) в границу полуплоскости (прямую).
	\TODO проверить решения ниже
	\subsection{Решение от семинариста}
		Мы хотим, чтобы окружность перешла в прямую.
		Значит, знаменатель ДЛП должен где-то на этой окружности занулиться.
		Давайте возьмём какое-нибудь такое ДПЛ, а потом его домножим на константу и, возможно, прибавим константу,
		чтобы из произвольной прямой получить нужную нам.

		Например, берём точку $z_0=-1$ и пишем преобразование:
		\[ w = \frac{1}{z-z_0} = \frac{1}{1+z} \]
		Давайте теперь поймём, куда перейдут оставшиеся точки на окружности (они гарантированно
		перейдут в прямую), например подставим точки 1 и $i$:
		\begin{gather*}
			w(1) = \frac{1}{1+1} = \frac 1 2\\
			w(i) = \frac{1}{1+i} = \frac{1-i}{1-i^2}=\frac{1-i}{2}=\frac12-\frac{i}{2}\\
		\end{gather*}

		Ответ получим такой:
		\[ -\left(\frac{1}{1+z}-\frac12\right) \]

	\subsection{Решение от Егора (инверсии)}
		\TODO

\section{Домашнее задание}
	Берём задачник Евграфова и делаем хотя бы семь номеров (всего десять) из параграфа 35 (до следующего занятия):
	\begin{enumerate}
		\item 35.04: 2 и 3
		\item 35.05: 1 и 2
		\item 35.06: 2 и 3
		\item 35.11: 3 и 5
		\item 35.18: 2 и 3
	\end{enumerate}

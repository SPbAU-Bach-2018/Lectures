\chapter{31.03.2016}

\section{Полезные конформные отображения}
	Просто справочный материал, этим можно пользоваться.
	Но также можно рассматривать как упражнения и доказать самостоятельно.

	\subsubsection{Полуплоскость на круг}
		Любое конформное преобразование полуплоскости $\Im z > 0$ на круг $|w|<1$ выглядит так:
		\[
			w(z) = \frac{z-z_0}{z-\bar z_0} e^{i\alpha} \quad \Im z_0 > 0,\, \alpha \in \R
		\]

	\subsubsection{Круг в самого себя}\label{day160331_circle_auto}
		Любое конформное преобразование круга $|z|<1$ на самого себя выглядит так:
		\[
			w(z) = \frac{z-z_0}{1-z\bar z_0} e^{i\alpha} \quad |z_0| < 1\, \alpha \in \R
		\]
		Например, при $z_0=0$ получаем тождественное отображение.
		Обычно чтобы определить отображение однозначно, задают еще два параметра:
		значение отображения в какой-нибудь точке и значение производной аргумента в этой же точке.

\section{Разбор 35.18 из дз №2}
	Есть функция Жуковского:
	\[ w(z) = \frac12\left(z+\frac1z\right) = \frac12\left(z+\frac{\bar z}{|z|^2}\right) \]
	Есть какие-то линии, надо понять, куда они перейдут.
	\subsection{Пункт 2}
		Линия такая (дуга окружности):
		\[
			\begin{cases}
				|z|=1 \\
				\arg z \in \left(-\frac{3\pi}{4}, -\frac{\pi}{4}\right)
			\end{cases}
		\]
		Так как $|z|=1$, то функция Жуковского получается равна $\sfrac 12(z+\bar z)$,
		что есть просто $\Re z$.
		Значит, образ линии есть просто проекция дуги на вещественную ось.
		Надо лишь посчитать косинусы углов из условия, получим:
		\[ -\frac{\sqrt2}{2} < w(z) < \frac{\sqrt 2}{2} \]

	\subsection{Пункт 3}
		Теперь у нас есть просто окружность $|z|=2$.
		Параметризуем её:
		\[ z = 2(\cos \phi + i\sin \phi) \quad \phi \in [0, 2\pi) \]
		Подставляем в функцию Жуковского:
		\begin{gather*}
			w(z) =
				\frac12\left(2(\cos\phi + i\sin\phi) + \frac{2(\cos(-\phi) + i\sin(-\phi))}{4}\right) =
				\cos\phi + i\sin\phi + \frac{\cos\phi - i\sin\phi}{4} = \\
				= \frac54\cos\phi + i\frac34\sin\phi
		\end{gather*}
		Получаем эллипс, для него можно записать уравнение:
		\begin{gather*}
			\left(\underbrace{\frac45 \Re w}_{\cos\phi}\right)^2 + \left(\underbrace{\frac 43 \Im w}_{\sin\phi}\right) = 1 \\
			\frac{16}{25}\Re^2 w + \frac{16}{9}\Im^2 w = 1
		\end{gather*}

\section{Задача 1}\label{day160331_task1}
	Найти отображение круга $|z|<1$ на круг $|w|<1$, причём:
	\begin{gather*}
		w\left(\frac 12\right) = \frac12 \\
		\arg w'\left(\frac 12\right) = \frac \pi 2
	\end{gather*}

	Мы попытались разобрать, не получилось "--- укопались в числах.
	Мораль: подифференцировать, составить систему уравнений и решать её можно, но сложно.
	\hyperref[day160407_task1]{В следующий раз} разобрали способом похитрее и попроще.

\section{Задача 2}
	Отобразить полуплоскость $\Re z - \Im z > 0$ на внешность круга $|w-3i|>2$, причём:
	\begin{gather*}
		w(1) = \infty \\
		\arg w'(2) = \frac \pi 4
	\end{gather*}

	Эта задача похожа на первую, её не разбирали, стали решать третью.

\section{Задача 3}
	Перевести луночку в верхнюю полуплоскость:
	\[
		\begin{cases}
			|z| < 1 \\
			|z - 3| < 3
		\end{cases}
		\mapsto
		\{ \Im w > 0 \}
	\]
	\begin{center}
		\begin{tabular}{cc}
		До & После \\
		\includegraphics{08_p3_a.pdf} & \includegraphics{08_p3_b.pdf} \\
		\end{tabular}
	\end{center}

	Давайте действовать аналогично \hyperref[day150331_task5]{предыдущей похожей задаче}.
	\begin{Rem}
		Вычислений на паре не было "--- они какие-то не очень приятные.
		Я их добавил через Вольфрам.
	\end{Rem}
	Найдём точки пересечения окружностей, одну переведём в 0, а другю "--- в бесконечность:
	\begin{gather*}
		z_{1,2} = \frac 16 \pm \frac{\sqrt{35}}{6} \\
		w(z) = \frac{z-z_2}{z-z_1} = \frac{z - \sfrac 16 - i\sfrac{\sqrt{35}}6}{z-\sfrac 16 + i\sfrac{\sqrt{35}}6}
	\end{gather*}
	Каждая окружность перейдёт в некоторую прямую, проходящую через начало координат.
	Чтобы понять, в какую, надо взять точку на каждой окружности и подставить её в преобразование,
	например:
	\begin{align*}
		z_3 &= -1 & z_4 &= 6 \\
		w_3 &= \dots & w_4 &= \dots \\
	\end{align*}
	У нас было четыре компоненты связности, в образе тоже получим четыре компоненты связности.
	Чтобы понять, куда перешла лунка, надо взять в ней точку и подставить в преобразование.
	\begin{align*}
		z_5 &= \frac 1 2 \\
		w_5 &= \dots
	\end{align*}
	После этого всего мы узнаем, куда нашим преобразованием переведётся лунка "--- в какой-то угол:

	\begin{center}
		\includegraphics{08_p3_c.pdf} \\
		\includegraphics{08_p3_d.pdf}
	\end{center}

	Его надо домножить на $e^{i\alpha}$, чтобы совместить начало с лучом $\arg z = 0$,
	а потом возвести преобразование в степень, чтобы угол стал развёрнутым.

\section{Задача 4}
	Куда переходит полоса при преобразовании $w=e^z$?
	Полоса такая: $\{ \Re z < 0 \cap \{ 0 < \Im z < \pi\}$.
	Распишем преобразование: 
	\begin{gather*}
		z = a + ib \\
		w=e^z=e^{a+ib}=e^a\cdot e^{ib} = e^a (\cos b + i\sin b)\\
	\end{gather*}
	\begin{center}
	\end{center}
	Получаем, что $e^z$ в каком-то смысле переписывает число из обычной формы в тригонометрическую.

	Так как $0 < \Im z < \pi$, то аргумент образа будет от $0$ до $\pi$, т.е. в верхней полуплоскости.
	А так как $\Re z < 0$, то модуль образа (экспонента от вещественная части) будет произвольным, но меньше единицы.
	Получаем, что полоса переходит в верхний полукруг без точки 0:
	\begin{center}
		\includegraphics{08_p4_a.pdf} \quad \includegraphics{08_p4_b.pdf}
	\end{center}

\TODO что-то еще было после задач сказано про возведение квадрат, логарифмы и так далее

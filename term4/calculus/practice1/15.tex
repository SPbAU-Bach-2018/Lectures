\chapter{26.05.2016}

\section{Разбор К/Р №2}
\subsection{Задача 4}
	\[
		\int_0^\infty e^{-ax^2} \cos bx \d x
	\]

	Сначала замечаем чётность функции и преобразовываем:
	\[
		\int_0^\infty e^{-ax^2} \cos bx \d x =
		\frac \int_{-\infty}^{\infty} e^{-ax^2} \cos bx \d x
	\]
	Потом угадываем контур.
	Правильный прямоугольник "--- против часовой стрелки, граница по вещественной оси от $-R$ до $R$,
	по мнимой оси "--- от 0 до некоторого $h$.

	Давайте теперь рассмотрим функцию $f(z)=e^{-az^2}$ (косинус магически вылезет потом).
	Особых точек внутри прямоугольника нет, значит, её интеграл по контуру равен ноль.

	Давайте посмотрим на предел её интегралов по нижней границе прямоугольника (I), это несобственный интеграл:
	\[ \int_{-\infty}^{\infty} e^{-ax^2} \d x \]
	В прошлом семестре мы считали такой интеграл:
	\[ \int_{-\infty}^\infty e^{-x^2} \d x = \sqrt \pi \]	
	\begin{Rem}
		Его можно посчитать, домножив на его самого по переменной $y$, перейдя в полярные координаты,
		а потом извекая корень.
		Мы этим когда-то занимались.
	\end{Rem}
	Делаем замену переменной:
	\[
		\int_{-\infty}^{\infty} e^{-ax^2} \d x =
		\frac{1}{\sqrt a} \int_{-\infty}^{\infty} e^{-(x\sqrt a)^2} \d (x\sqrt a) =
		\frac{1}{\sqrt a} \sqrt \pi
	\]

	Докажем, что пределы интегралов по границам II и IV равны нулю.
	\TODO

	Теперь посмотрим на предел её интегралов по верхней границе (III):
	\[
		\int_{III} =
		\int_{+\infty}^{-\infty} e^{-a(x+ih)^2} \d x =
		\int_{+\infty}^{-\infty} e^{-ax^2}e^{ah^2}e^{-2aixh} \d x =
		e^{ah^2} \int_{+\infty}^{-\infty} e^{-ax^2}\left(\cos(-2axh)+i\sin(-2axh)\right)\d x
	\]
	Давайте выберем $h=\sfrac{b}{2a}$.
	Мы знаем, что $I+II+III+IV=0$, значит, можно приравнять вещественные и мнимые части и приравнять:
	\[
		\Re \int_{III} = e^{ah^2} \int_{+\infty}^{-\infty} e^{-ax^2}\cos bx \d x = \sqrt{\frac{\pi}{a}} \\
		\Im \int_{III} = -e^{ah^2} \int_{+\infty}^{-\infty} e^{-ax^2}\sin bx \d x = 0 \\
	\]
	Вторая строчка неинтересна, а вот первая как раз позволяет нам посчитать ответ.

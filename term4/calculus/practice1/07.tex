\chapter{24.03.2016}

\begin{Rem}
	Когда мы используем ангармоническое преобразование и одна из точек равна бесконечности,
	то можно считать, что $\frac{\infty}{\infty}=1$ и $\infty+x=\infty$, сократить дроби и
	решить получившееся уравнение относительно $w$ и $z$.
	Никакого формализма, это всё надо по-хорошему доказывать и проверять, но в случае с ангармоническим работает.
\end{Rem}

\begin{Def}
	$z_1$  и $z_2$ называются \textit{симметричными} относительно окружности $\Gamma$ радиусом $R$,
	если они лежат на одном луче с началом в центре окружности $z_0$ и
	\[ |z_0-z_1||z_0-z_2| = R^2 \]
	\cautoimg{07_def.pdf}
\end{Def}

\section{Задача 1}
	Показать, что $z_1 \neq z_2$ симметричны относительно $\Gamma$ $\iff$
	любая окружность/прямая $C$, проходящая через $z_1$ и $z_2$ пересекает $\Gamma$ под прямым углом.

	\begin{Rem}
		Егору совершенно не понравился разбор этой задачи.
	\end{Rem}
	Сначала докажем влево.
	Это какая-то куча ссылок на школьную геометрию.
	Теперь докажем вправо.
	Взяли окружность $C_1$ через $z_1$ и $z_2$, взяли точку её пересечения с $\Gamma$ $z_3$.
	Если через центр $\Gamma$ и $z_3$ проходит касательная к окружности $C_1$, то мы победили.
	Иначе возьмём касательную к окружности, она пройдёт через точку $z_3'$, в этой точке
	из школьной геометрии знаем, что произведение расстояний равно квадрату радиуса,
	стало быть, $z_3$ и $z_3'$ одновременно лежат на некоторой окружности с центром в $z_0$
	и на $C_1$, при этом они с одной стороны, значит, равны.

\section{Задача 2}
	Пусть $z_1$ и $z_2$ симметричны относительно окружности $|z-z_0|=R$,
	а $w$ "--- некоторое ДПЛ (на самом деле вроде достаточно конформного отображения).
	Доказать, что тогда образы $z_1$ и $z_2$ симметричны относительно образа окружности.
	Если образ "--- прямая, то свойство тоже сохраняется
	(что такое симметричность относительно прямой должно быть понятно).

	Воспользуемся задачей 1.
	До преобразования свойство для точек $z_1$ и $z_2$ было верно.
	Так как конформное преобразование сохраняет все углы и пересечения между окружностями/прямыми, то
	после преобразования свойство всё еще сохранится: какую бы мы окружность/прямую через образы
	$z_1$ и $z_2$ не взяли, можно взять её прообраз (тоже окружность/прямую), а её пересечение с $C$
	было под прямыми углами.
	Стало быть, после преобразования тоже все углы прямые, что и требовалось.

\section{Задача 3}
	Найти какое-нибудь конформное преобразование, переводящее области:
	\[ \{ |z|<1 \} \mapsto \{ \Im z > 0 \} \]
	\begin{center}
		\begin{tabular}{cc}
		До & После \\
		\includegraphics{07_p3_a.pdf} & \includegraphics{07_p3_b.pdf} \\
		\end{tabular}
	\end{center}
	Можно использовать ДПЛ, возведение в степень, экспоненты...

	Сначала хотим перевести границу в границу, то есть окружность в прямую.
	Давайте выберем какие-нибудь три точки на окружности и выберем, куда они перейдут
	(в надежде, что нам повезёт):
	\begin{align*}
		z_1 &= i & w_1 = 1 \\
		z_2 &= 1 & w_2 = 0 \\
		z_3 &= -i & w_3 = -1
	\end{align*}
	Теперь пишем ангармоническое соотношение и решаем:
	\begin{gather*}
		\frac{w-1}{w-0} \cdot \frac{-1-0}{-1-1} = \frac{z-i}{z-1} \cdot \frac{-i-1}{-i-i} \\
		\frac{(w-1)}{w} \cdot \frac 12 = \frac{z-i}{z-1} \cdot \frac{i+1}{2i} \\
		\frac{w-1}{2w} = \frac{(z-i)(1-i)}{2(z-1)} \\
		(w-1)(z-1) = w(z-i)(1-i) \\
		wz-w-z+1 = w(z-i-iz-1) \\
		w(z-1-z+i+iz+1) = z-1 \\
		w(i+iz) = z-1 \\
		w=\frac{z-1}{i(z+1)}
	\end{gather*}
	Осталось понять, куда перейдёт внутренность круга, взяв точку $z_4$ внутри круга и применив отображение:
	\[ w(0) = \frac{-1}{i(0+1)} = \frac{-1}{i} = i \]
	Значит, получилось правильное отображение.

\section{Задача 4}
	Найти какое-нибудь конформное преобразование, переводящее области:
	\[ \{ |z|<1 \} \cap \{\Im z > 0\} \mapsto \{ \Im z > 0 \} \]
	\begin{center}
		\begin{tabular}{cc}
		До & После \\
		\includegraphics{07_p4_a.pdf} & \includegraphics{07_p4_b.pdf} \\
		\end{tabular}
	\end{center}

	Подсказка: сначала полезно перегнать полукруг в первый квадрант, а потом его перегнать в полуплоскость
	(при помощи $w=z^2$).
	Перегнать полукруг в первый квадрант можно так: нижняя сторона переходит в нижнюю сторону,
	а полуокружность "--- в левую сторону квадранта.
	\begin{center}
		\begin{tabular}{cc}
		До & После \\
		\includegraphics{07_p4_c.pdf} & \includegraphics{07_p4_d.pdf} \\
		\end{tabular}
	\end{center}

	Давайте переведём полуокружность в квадрант, выбрав какие-то точки (как минимум "--- углы полуокружности)
	и куда они переходят:
	\begin{align*}
		z_1 &= -1 & w_1 &= 0 \\
		z_2 &= 0 & w_2 &= 1 \\
		z_3 &= 1 & w_3 &= \infty \\
	\end{align*}
	Получим следующее отображение (мы знаем, что раз $w_3=\infty$,
	то в знаменателе должно быть $z-z_3$, числитель подгоняем):
	\[ w(z)=-\frac{z+1}{z-1} \]
	Подставим точку $z_4=\sfrac i2$, проверим, куда она перейдёт:
	\[ w(\frac i2) = -\frac{i+2}{i-2} = \frac{3+4i}{5} \]
	Ух ты, повезло.
	Осталось возвести это преобразование в квадрат, чтобы получить полуплоскость:
	\[ w(z) = \left(\frac{z+1}{z-1}\right)^2 \]
	\begin{Rem}
		Остался серьёзный вопрос: а что бы мы делали, если не повезло?
		Вот и Илья тоже не знает.
	\end{Rem}

\section{Задача 5}
	Найти какое-нибудь конформное преобразование, переводящее области:
	\[ \{ |z-i|<1 \} \cap \{ |z-1| < 1 \} \mapsto \Im z > 0 \]
	\begin{center}
		\begin{tabular}{cc}
		До & После \\
		\includegraphics{07_p5_a.pdf} & \includegraphics{07_p5_b.pdf} \\
		\end{tabular}
	\end{center}
	Такая область называется <<луночкой>>.

	Давайте сначала заметим, что если мы переведём границы лунки в две прямые, то лунка перейдёт в некоторый угол,
	который степенными преобразованиями уже можно перевести в нужный нам развёрнутый угол (он же полуплоскость).
	Давайте оставим точку 0 на месте, а точку $1+i$ переведём в бесконечность, например, таким преобразованием:
	\[ w(z) = \frac{z}{z-(1+i)} \]
	\begin{Rem}
		Мы знаем, какая точка переходит в бесконечность $\Ra$ знаем знаменатель.
		Числитель должен обращаться в ноль в нуле, при этом сам нулю быть равен не может,
		самое простое, что подходит "--- $z$.
	\end{Rem}
	Теперь посмотрим, куда переходят окружности.
	Мы знаем, что они перейдут в прямые, значит, достаточно лишь понять, куда перейдут две точки на каждой окружности,
	а потом провести прямые через эти точки.
	\begin{align*}
		z_1 &= 0 & w_1 &= 0 \\
		z_2 &= 2 & w_2 &= \frac{2}{2-(1+i)} = \frac{2}{1-i} = 1 + i \\
		z_3 &= 2i & w_3 &= \frac{2i}{2i-(1+i)} = \frac{2i}{i-1} = 1-i \\
	\end{align*}
	Окружность $|z-1|=1$ переходит в прямую $x=y$, а окружность $|z-i|=1$ перейдёт в прямую $x=-y$.
	Теперь посмотрим, куда перешла лунка, взяв точку $0.5+0.5i$:
	\[ w(0.5+0.5i) = \frac{0.5+0.5i}{-(0.5-0.5i)} = -1 \]
	\begin{center}
		\begin{tabular}{cc}
		До & После \\
		\includegraphics{07_p5_c.pdf} & \includegraphics{07_p5_d.pdf} \\
		\end{tabular}
	\end{center}
	Осталось лишь повернуть это так, чтобы получился первый квадрант и возвести в квадрат, чтобы получить верхнюю полуплоскость:
	\begin{gather*}
		\left(\frac{z}{z-(1+i)} \cdot e^{-\sfrac 3 4 i \pi}\right)^2 =
		\left(\frac{z}{z-(1+i)} \cdot \left(-\frac{\sqrt 2}{2} -i\frac{\sqrt 2}{2}\right)\right)^2 = \\
		\left(\frac{z(1+i)}{z-(1+i)} \cdot \left(-\frac{\sqrt 2}{2}\right)\right)^2 =
		\frac12 \cdot \frac{z^2(1+i)^2}{(z-(1+i))^2} = \\
		\frac12 \cdot \frac{z^2\cdot2i}{(z-(1+i))^2} =
		\frac{iz^2}{(z-(1+i))^2}
	\end{gather*}

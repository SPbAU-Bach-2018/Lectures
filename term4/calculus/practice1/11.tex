\chapter{28.04.2016}

\section{Разложение в ряд Лорана}
	Хотим разложить функцию в ряд Лорана:
	\[
		f(z) = \frac{z+2}{z^2-2z-3} = \frac{\sfrac{-1}4}{z+1} + \frac{\sfrac 54}{z-3}
	\]
	Эта функция аналитична везде, кроме особых точек $z=-1$ и $z=3$.
	Давайте раскладывать отдельно в трёх областях:
	\begin{itemize}
		\item $|z| < 1$
		\item $1 < |z| < 3$
		\item $|z| > 3$
	\end{itemize}

	Общий алгоритм: если у нас есть дробь $\frac{1}{a-(z-z_0)}$ и мы хотим разложить её в точке $z_0$
	при $|z-z_0|<|a|$, то удобно поделить на $a$ и получить геометрическую прогрессию:
	\[
		\frac{1}{a-(z-z_0)} = \frac{\sfrac 1a}{1-\frac{z-z_0}{a}} = \sum_{n=0}^{\infty} \frac{(z-z_0)^n}{a^{n+1}}
	\]
	Если же $|z-z_0|>|a|$, то удобно поделить на $z-z_0$:
	\[
		\frac{1}{a-(z-z_0)} = - \frac{\sfrac 1{(z-z_0)}}{1-\frac{a}{z-z_0}} =
		- \sum_{n=0}^\infty \frac{a^n}{(z-z_0)^{n+1}}
	\]
	\begin{Rem}
		Два случая (в отличие от дискретки) потому что нам надо не просто какой угодно ряд, который
		формально равен, а сходящийся.
		Иначе всё бессмысленно.
	\end{Rem}

	Теперь поехали по областям из примера.
	Пусть мы в области $1<|z|<3$ и хотим разобраться с дробью $\frac{\sfrac 54}{z-3}$.
	Для этого разложим сначала дробь $\frac{1}{3-z}$.
	Мы знаем, что $|z|<3$, значит, мы в первом случае, будет удобно сделать так:
	\[
		\frac{1}{\underbrace{3}_{a}-z} = \frac{\sfrac 13}{1-\sfrac z3} = \frac13 \sum_{n=0}^\infty \frac{z^n}{3^n} =
		\sum_{n=0}^\infty \frac{z^n}{3^{n+1}}
	\]

	Для второй дроби:
	\[
		\frac{1}{1+z} = \frac{-1}{\underbrace{-1}_{a}-z}
	\]
	Тут у нас $|z|>|-1|=1$, так что удобно делить на $z$:
	\[
		\frac{-1}{-1-z} =
		\frac{\sfrac{-1}{z}}{\sfrac{-1}{z}-1} = 
		\frac{\sfrac{1}{z}}{1-(\sfrac{-1}{z})} =
		\frac1z \sum_{n=0}^\infty \frac{(-1)^n}{z^n} =
		\sum_{n=0}^\infty \frac{(-1)^n}{z^{n+1}} =
	\]

	Теперь нужны коэффициенты и сложить:
	\[
		\frac{-1}{4} \cdot \sum_{n=0}^\infty \frac{(-1)^n}{z^{n+1}}
		+
		\frac{5}{4} \cdot \left(-\sum_{n=0}^\infty \frac{z^n}{3^{n+1}}\right) =
		- \frac{1}{4} \cdot \sum_{n=0}^\infty \frac{(-1)^n}{z^{n+1}}
		- \frac{5}{4} \cdot \sum_{n=0}^\infty \frac{z^n}{3^{n+1}}
	\]

\section{Типы особых точек}
	\begin{itemize}
		\item
			$\lim_{z\to z_0} f(z) = a \in \C$ "--- предел существует и конечен, тогда это устранимая особенность
		\item
			$\lim_{z\to z_0} f(z) = \infty$ "--- полюс
		\item
			$\not\exists \lim_{z\to z_0} f(z)$ "--- существенная особенность
	\end{itemize}

	\begin{Rem}
		На комплексной плоскости надо подходить к предельной точке по всем возможным кривым,
		а не только слева и справа.
	\end{Rem}

	Задачи:
	\subsection{а}
		\[
			\lim_{z\to 0} e^{1/z}
		\]
		Предела нет: если подходить с направления $+\infty \to 0$, то предел $\infty$.
		А если с минуса по вещественной прямой, то предел "--- $0$.

	\subsection{б}
		\[
			\lim_{z\to 0} e^{\sfrac{1}{z^2}}
		\]
		Предела нет: направления $1$ и $i$.

	\subsection{в}
		\[
			\lim_{z\to 0} \frac{\sin z}{z} = \frac{z - \frac{z^3}{3!} + \frac{z^5}{5!} - \dots}{z} = 1 - \frac{z^2}{3!} + \frac{z^4}{5!} - \dots
		\]
		Сходится по тем же причинам, что и сам синус сходится.

	\subsection{г}
		При $n>1$:
		\[
			\frac{\sin z}{z^n} = \frac{1}{z^{n-1}} = \frac{1}{z^{n-1}} + \dots + \frac{c_{-1}}{z} + c_0 + c_1z + \dots
		\]
		А вот тут имеем полюс порядка $n-1$.
		Коэффициент $c_{-1}$ называется вычетом:
		\[
			\res_{z_0} f = c_{-1}
		\]

	Вопрос: пусть есть функция с полюсом конечного порядка (т.е. главная часть имеет конечное число слагаемых)
	и мы хотим посчитать вычет в точке $z_0$.
	Как это делать?
	Вот была функция:
	\begin{align*}
		f(z) &= \sum_{-N}^\infty c_n (z - z_0)^n \\
		f(z) &= \frac{c_{-N}}{(z-z_0)^N} + \dots + \frac{c_{-1}}{z-z_0} + с_0 + с_1(z-z_0) + \dots \\
		f(z)(z-z_0)^N &= c_{-N} + \dots + c_{-1}(z-z_0)^{N-1} + с_0(z-z_0)^N + с_1(z-z_0)^{N+1} + \dots \\
		\lim_{z \to z_0} \frac{\d^{N-1}}{\d z^{N-1}} [ f(z)(z-z_0)^N ] = c_{-1}(N-1)! \\
	\end{align*}

	\TODO выше и ниже должны быть пределы из последнего перехода

	Частный случай $N=1$ (подставляем):
	\[
		c_{-1} = (f(z)(z-z_0))^{(0)} = f(z)(z-z_0)
	\]
	Например, пусть $f(z)=\frac{\phi(z)}{\psi(z)}$, причём $\phi(z_0)\neq 0$, $\psi(z_0)=0$ и $\psi'(z_0) \neq 0$.
	Тогда у нас полюс точно кратности 1, можно подставить:
	\[
		c_{-1} = f(z)(z-z_0) = \frac{\phi(z)}{\psi(z)} (z-z_0) =
		\frac{\phi(z)}{\left(\frac{\psi(z)-\psi(z_0)}{z-z_0}\right)} =
		\frac{\phi(z)}{\psi'(z_0)}
	\]

\section{Основная теорема вычетов}
	У нас есть область (ограниченое открытое связное множество) $D$, у неё есть некоторая хорошая граница.
	\begin{Rem}
		Кто сможет придумать нехорошую границу "--- аплодисменты.
	\end{Rem}
	У нас есть некоторое конечное число точек $z_k \in D$, в которых у функции $f$ находятся полюса.
	Тогда интеграл по контуру $C=\partial D$ представляется так:
	\[
		\oint_C f(z) \d z = 2 \pi i \sum_{k=1} \res_{z_k} f
	\]

\section{Наш первый интеграл}
	\[
		\int_{-\infty}^{\infty} \frac{\d x}{x^4 + 1}
	\]
	Давайте рассмотрим интеграл по контуру полукруга $C_r$ в верхней полуплоскости с центром в нуле и радиусом $R$.
	Его можно посчитать через основную теорему вычетов, узнав полюса функции в полукруге.

	Посчитаем, получим:
	\[
		\lim_{R \to \infty} \oint_{C_r} f(z) \d z =
		\lim_{R \to \infty} \int_{-R}^{R} f(z) \d z +
		\lim_{R \to \infty} \oint_{\gamma_R} f(z) \d z =
		\int_{-\infty}^{\infty} f(z) \d z +
		\lim_{R \to \infty} \oint_{\gamma_R} f(z) \d z
	\]
	Давайте оценим второе слагаемое "--- интеграл по $\gamma_R$.
	Будем оценивать максимум модуля:
	\begin{gather*}
		\oint_{\gamma_R} f(z) \d z \le
		\pi R \max_{z \in \gamma_R} |f(z)| =
		\pi R \max_{z \in \gamma_R} |\frac{1}{z^4+1}| \circled{\le} \\
		\begin{aligned}
			|z^4 + 1| &\ge |z|^4 - 1 \quad \text{так как} \\
			||x| - |y|| &\le |x - y| \quad \text{обратное неравенство треугольника} \\
			||z^4| - |-1|| &\le |z^4-(-1)| \\
			||z|^4 - 1| &\le |z^4+1| \\
			|z|^4 - 1 &\le |z^4+1| \\
		\end{aligned}
		\circled{\le} \pi R \max_{z \in \gamma_R} \frac{1}{|z|^4+1} \le
		\pi R \frac{1}{R^4-1}
	\end{gather*}
	Это стремится к нулю при $R\to \infty$, значит, второе слагамое занулится.

	Теперь считаем первый интеграл.
	Для этого надо посмотреть полюсы.
	Особенность у нашей функции бывает в точках $|z^4|=1$, это корни из единицы, причём лежащие в верхней полуплоскости
	(так как у нас верхний полукруг):
	\begin{gather*}
		z_1 = \frac{-1+i}{\sqrt2} \\
		z_1^2 = -i \\
		z_2 = \frac{1+i}{\sqrt2} \\
		z_2^2 = i \\
	\end{gather*}
	Давайте найдём в них вычеты:
	\begin{gather*}
		\res_{z_1} \frac{\overbrace{1}_{\phi}}{\underbrace{z^4+1}_{\psi}} =
		\frac{\phi(z_1)}{\psi'(z_1)} =
		\frac{1}{4z_1^3} \\
		\res_{z_2} \frac{1}{z^4+1} = \frac{1}{4z_2^3}
	\end{gather*}
	Получаем, что при больших $R$ наш интеграл по контуру в точности равен:
	\[
		2\pi i \cdot \frac14 \left(\frac{1}{z_1^3}+\frac{1}{z_2^3}\right) =
		\frac{\pi i}{2} \cdot \left(\frac{1}{-iz_1}+\frac{1}{iz_2}\right) =
		\frac{\pi}{2} \cdot \left(-\frac{1}{z_1}+\frac{1}{z_2}\right) =
		\frac{\pi}{2} \cdot \left(-\frac{\sqrt2}{-1+i}+\frac{\sqrt2}{1+i}\right) =
		\frac{\pi}{\sqrt 2} \cdot \left(\frac{1}{1-i}+\frac{1}{1+i}\right) =
		\frac{\pi}{\sqrt 2} \cdot \frac{1+i+1-i}{2} =
		\frac{\pi}{\sqrt 2}
	\]

	Ответ: $\frac{\pi}{\sqrt 2}$

\section{Наш второй интеграл}
	\[
		\int_{-\infty}^{\infty} \frac{x^2+3}{(x^2+4x+6)^2} \d x
	\]

	Ищем полюса:
	\begin{gather*}
		x^2+4x+6 = 0 \\
		x = \frac{-4\pm\sqrt{16-4\cdot1\cdot6}}{2} =
		\frac{-4\pm2\sqrt{-2}}{2} =
		-2\pm\sqrt{-2} =
		-2\pm\sqrt{2}i
	\end{gather*}
	Контур берём такой же, как и в предыдущей задаче.
	Давайте сначала оценим интеграл по дуге окружности.
	Оценка, как обычно, тупая и грубая:
	
	\begin{gather*}
		\int_{\gamma_R} f(z) \d z \le
		\pi R \max_{\gamma R} \left|\frac{z^2+3}{(z^3+4x+6)^2}\right| \circled{\le} \\
		|z^2+3| \le |z|^2+3 = R^2 + 3 \\
		|z^3+4x+6| \ge R^2-4R-6 > 0 \\
		\circled{\le} \pi R \cdot \frac{R^2+3}{R^2-4R-6}
	\end{gather*}
	Дальше время кончилось.
	Мы успели выяснить, что интеграл по прямой "--- то же самое, что предел интегралов по контуру,
	но осталось посчитать интеграл по контуру (там уже будут полюсы порядка 2, надо аккуратнее).
	Надо пример дома доделать.

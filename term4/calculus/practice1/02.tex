\chapter{18.02.2016}

\section{Поверхностные интегралы II рода}

\subsection{Определение}
Есть поверхность:
\[ S=\vec r(D)=(x(u, v), y(u, v), z(u, v)) \]
\begin{Def}
	Нормаль $\vec N$ к поверхности:
	\[ \vec N = \frac{\vec r'_u \times \vec r'_v}{\|\vec r'_u \times \vec r'_v\|} \]
\end{Def}

\begin{Def}
	Определение интеграла II рода через интеграл I рода (в конспекте за III семестр
	называлось теоремой 7.3.4):
	\begin{equation}\label{int_surf_II_def}
		\iint\limits_S P \d y \d z + Q\d z \d x + R\d x \d y = \iint\limits_S \left( (P, Q, R) \cdot \vec N \right) \d S = \iint\limits_S \left(P N_x + Q N_y + R N_z\right) \d S
	\end{equation}
\end{Def}
Также есть полезное следствие из этого определения (оно возникало в процессе доказательства теоремы 7.3.4 на лекциях).
Тут надо быть осторожно с ориентацией (пока что мы на паре не научились определять, какая ориентация, поэтому пока
что считаем по этой формуле с точностью до ориентации):
\begin{equation}\label{int_surf_II_coords}
	\iint\limits_S P \d y \d z + Q\d z \d x + R\d x \d y =
	\iint\limits_D
	\begin{vmatrix}
		P & Q & R \\
		x'_u & y'_u & z'_u \\
		x'_v & y'_v & z'_v \\
	\end{vmatrix}
	\d u \d v
\end{equation}

\subsection{Формула Стокса}
Пусть у нас есть поверхность $S$, а её граница "--- некоторая кривая $L$.
Тогда верно следующее:
\begin{equation}
	\int\limits_L P \d x + Q \d y + R \d z =
	\iint\limits_S
	\left(\partd Ry - \partd Qz\right) \d y \d z +
	\left(\partd Pz - \partd Rx\right) \d z \d x +
	\left(\partd Qx - \partd Py\right) \d x \d y
\end{equation}

\subsection{Домашнее задание}
Имеется четыре темы, на каждую тему по две задачи:
\begin{enumerate}
	\setcounter{enumi}{-1}
	\item Поверхностные интегралы I рода
	\item Поверхностные интегралы II рода
	\item Формула Гаусса-Остроградского
	\item Формула Стокса
\end{enumerate}
Если решаешь задачу на соответствующую тему в классе, дома её делать не надо.

ДЗ будет прислано в субботу, в нём будут темы, которые мы успели на паре: 0, 1, 3.
Гаусса-Остроградского мы совсем не успели.

\section{Задача 1}\label{day160218_task1}
	Пусть $S$ "--- часть эллипсоида (на ориентацию нормали забили):
	\begin{align*}
		x &= a \cos u \cos v & u \in \left[\frac \pi 4; \frac \pi 3 \right]\\
		y &= b \sin u \cos v & v \in \left[\frac \pi 6; \frac \pi 4 \right]\\
		z &= c \sin v
	\end{align*}
	\TODO картинка

	Считаем интеграл:
	\[ \iint\limits_S \frac{\d y \d z}{x} + {\d z \d x}{y} + \frac{\d x \d y}{z} \]

	Пользуемся формулой \ref{int_surf_II_coords}.
	Для этого сначала посчитаем производные:
	\begin{align*}
		x'_u &= -a \sin u \cos v & x'_v &= -a \cos u \sin v \\
		y'_u &=  b \cos u \cos v & y'_v &= -b \sin u \sin v \\
		z'_u &= 0 & z'_v &= c \cos v \\
	\end{align*}
	Поверхность есть образ $D=[\frac \pi 4; \frac \pi 3] \times [\frac \pi 6; \frac \pi 4]$.
	Пишем формулу:
	\begin{gather*}
		\iint\limits_S \frac{\d y \d z}{x} + {\d z \d x}{y} + \frac{\d x \d y}{z} =
		\iint\limits_D \begin{vmatrix}
			P & Q & R \\
			x'_u & y'_u & z'_u \\
			x'_v & y'_v & z'_v \\
		\end{vmatrix} \d u \d v = \\
		= \iint\limits_D \begin{vmatrix}
			\frac{1}{a \cos u \cos v} & \frac{1}{b\sin u\cos v} & \frac{1}{c\sin v} \\
			-a \sin u \cos v & b \cos u \cos v & 0 \\
			-a \cos u \sin v & -b \sin u \sin v & c \cos v
		\end{vmatrix} \d u \d v = \\
		= \iint\limits_D \left(
			\frac{bc \cos u \cos^2 v}{a \cos u \cos v}
			- \frac{-ac \sin u \cos^2 v}{b\sin u\cos v} 
			+ \frac{ab\sin^2 u \cos v \sin v - (-ab)\cos^2 u \sin v \cos v}{c\sin v}
		\right) \d u \d v = \\
		= \iint\limits_D \left(
			\frac{bc}{a} \cos v
			+ \frac{ac}{b} \cos v
			+ \frac{ab(\sin^2 u + \cos^2 u)}{c} \cos v
		\right) \d u \d v = \\
		= \iint\limits_D \left(
			\frac{bc}{a} \cos v
			+ \frac{ac}{b} \cos v
			+ \frac{ab}{c} \cos v
		\right) \d u \d v = \\
		= \underbrace{\left(\frac{bc}{a} + \frac{ac}{b} + \frac{ab}{c}\right)}_{C}
		\int\limits_{\sfrac{\pi}{4}}^{\sfrac{\pi}{3}}
		\int\limits_{\sfrac{\pi}{6}}^{\sfrac{\pi}{4}}
			\cos v \d v \d u =
		C \cdot \left(\frac{\pi}{3} - \frac{\pi}{4}\right) \cdot \left.\left(\sin v\right)\right|_{\sfrac{\pi}{6}}^{\sfrac{\pi}{4}} =
		\frac{\pi C}{12} \cdot \left(\frac{1}{\sqrt 2} - \frac 12\right) =
		\frac{\pi C}{12} \cdot \frac{\sqrt 2 - 1}{2}
	\end{gather*}

\section{Задача 2}\label{day160218_task2}
	$S$ "--- часть параболоида, на ориентацию нормали забили:
	\[
		\begin{cases}
			z = x^2 + y^2; \\
			0 \le z \le 2
		\end{cases}
	\]
	Хотим посчитать
	\[ \iint_S y \d z \d x \]

	У нас есть поверхность $(x, y) \to (x, y, x^2+y^2)$.
	Тут $(x, y)$ берётся из круга радиуса $\sqrt 2$ с центром в начале координат.
	Пользуемся формулой \ref{int_surf_II_coords}:
	\begin{gather*}
		\iint\limits_S y \d z \d x =
		\iint\limits_D \begin{vmatrix}
			0 & y & 0 \\
			1 & 0 & 2x \\
			0 & 1 & 2y \\
		\end{vmatrix} \d x \d y = \\
		= \iint\limits_D -y(1\cdot 2y - 2x \cdot 0) \d x \d y =
		\iint\limits_D -2y^2 \d x \d y =
		-2 \int\limits_{-\sqrt 2}^{\sqrt 2} y^2 \left(\int\limits_{-\sqrt{2-y^2}}^{\sqrt{2-y^2}} \d x \right) \d y = \\
		= -2 \int\limits_{-\sqrt 2}^{\sqrt 2} y^2 \cdot 2\sqrt{2-y^2} \d y =
		-2 \int\limits_{-\sqrt 2}^{\sqrt 2} 2y\sqrt{2y^2-y^4} \d y \circleeq \\
		\text{делаем замену $t=y^2$ и считаем интеграл при $y \ge 0$} \\
		\d t = 2 y \d y \\
		\d y = \frac{\d t}{2y} \\
		\circleeq -4 \int\limits_0^2 \sqrt{2t-t^2} \d t =
		-4 \int\limits_0^2 \sqrt{1-(t-1)^2} \d t \circleeq \\
		\text{интеграл "--- площадь полукруга радиуса 1 с центром в $(1, 0)$} \\
		% \TODO картинка
		\circleeq -4 \cdot \frac{\pi\cdot 1^2}{2} = -2\pi
	\end{gather*}

\section{Задача 3 (не разобрана)}\label{day160218_task3}
	Хотим посчитать криволинейный интеграл:
	\[
		\int\limits_{OA} y z \d x + 3 x z \d y + 2 x y \d z \\
	\]
	Кривая вот такая:
	\[
		\begin{cases}
			x = t \cos t \\
			y = t \sin t \\
			z = t^2 = x^2=y^2 \\
			0 \le t \le 2 \pi \\
			O(0, 0, 0) \\
			A(2\pi, 0, 4\pi^2)
		\end{cases}
	\]
	Это вот такая незамкнутая кривая, навёрнутая на параболоид, делает один поворот вокруг центра координат.
	Если её замкнуть, добавив отрезок $AO$, то мы вырежем на параболоиде один конечный кусок (второй кусок будет бесконечный).
	\TODO картинка
	Давайте воспользуемся формулой Стокса.
	У нас на паре не получилось :(

	Но, говорят, считается через криволинейные интегралы II рода.
	Эту задачу разбираем \hyperref[day160225_task1]{в следующий раз}.

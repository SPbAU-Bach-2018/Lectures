\chapter{12.05.2016}

\section{Лемма Жордана}
	\begin{lemma}
		Если $a\ge 0$ и $\lambda > 0$, а $C_r$ "--- полудуга (<<пузырёк>>):
		\[
			C_r = \{ |z| = R \land \Im z \ge -a \}
		\]
		При этом:
		\[ \lim_{R\to\infty} \max_{C_r}|f(z)| = 0 \]
		То тогда:
		\[ \lim_{R\to\infty} \oint_{C_r} f(z) e^{i\lambda z} \d z = 0 \]
	\end{lemma}
	\begin{Rem}
		Используется для оценки интегралов по контуру, когда мы считаем какие-то интегралы.
		Например, если мы считаем вещественный интеграл, то нам будет нужно только $a=0$.
	\end{Rem}

\section{Задача 1}
	\[ \int_{-\infty}^{\infty} \frac{x \sin x}{1 + x^2} \d x \]

	Перепишем интеграл:
	\begin{gather*}
		A \coloneq \frac{e^{ix}-e^{-ix}}{2i} \\
		\int_{-\infty}^{\infty} \frac{x \sin x}{1 + x^2} \d x =
		\frac1{2i} \int_{-\infty}^{\infty} \frac{x}{1+x^2}\cdot e^{ix} - \frac{x}{1+x^2} \cdot e^{-ix} \d x \circleeq \\
		\text{Разобьём интеграл на два в надежде, что оба сойдутся} \\
		\circleeq \frac1{2i} \left(\int_{-\infty}^{\infty} \frac{x}{1+x^2}\cdot e^{ix} \d x - \int_{-\infty}^{\infty} \frac{x}{1+x^2} \cdot e^{-ix} \d x \right) =
		\frac1{2i} \left(\int_{-\infty}^{\infty} \frac{x}{1+x^2}\cdot e^{ix} \d x - \int_{\infty}^{-\infty} \frac{-x}{1+(-x)^2} \cdot e^{-i(-x)} \d (-x) \right) =
		\frac1{2i} \left(\int_{-\infty}^{\infty} \frac{x}{1+x^2}\cdot e^{ix} \d x - \int_{\infty}^{-\infty} \frac{x}{1+x^2} \cdot e^{ix} \d x \right) =
		2 \cdot \underbrace{\frac1{2i} \int_{-\infty}^{\infty} \frac{x}{1+x^2}\cdot e^{ix} \d x}_{I}
	\end{gather*}
	Возьмём контур в виде полуокружности радиуса $R$.
	Применим лемму Жордана для $a=0$, $\lambda=1$ и $f(z)=\frac{x}{1+x^2}$.
	Условие для $f(z)$ выполнено, так как $|f(z)|$ стремится к нулю при $z \to \infty$.
	Значит, интеграл по дуге окружности стремится к нулю при $R \to \infty$.
	\begin{Rem}
		Лемма Жордана "--- довольно тонкая штука, иначе бы мы не смогли оценить интеграл.
		\TODO
	\end{Rem}

	\TODO
	Получаем единственный вычет:
	\[ ie{-1}/(-4) \]
	Ответ тогда такой:
	\[ 2I = \frac{\pi}{e} \]

\section{Задача 2}
	\[ \int_0^{\infty} \frac{\cos x}{x^2+a^2} \d x \]

	Посмотрим на полюса 
	Вычет в $x=ia$.

	...
	$\frac{2e^{-a}}{2\cdot2ia}$
	...

	Ответ такой:
	\[ \frac{\pi}{e^a\cdot2a} \]

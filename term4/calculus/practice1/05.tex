\chapter{10.03.2016}

\section{Организационное}
	Просьба ко всем сдавшим домашней задания в электронном виде по электронной почте:
	распечатать и сдать также бумажно, чтобы можно было проставить комментарии.

	Несколько хороших книжек по ТФКП:
	\begin{itemize}
		\item Теория: Лаврентьев и Шабат <<Методы ТФКП>>.
		\item Задачник: Евграфов <<Аналитические функции>>.
		\item Теория: Привалов <<ТФКП>>.
	\end{itemize}

\section{Загадка века}
	Ниже "--- парадокс, как объяснить?
	\[
		i \pi =
		\ln e^{i\pi} =
		\ln (-1) =
		\ln ((-i)^2) =
		2\ln (-i) =
		2\left(\ln \left(1 + i\left(-\frac{\pi}{2}\right)\right)\right) =
		-2\pi
	\]
	\begin{Rem}
		На доске была написана какая-то чушь, мы её так и не обсудили,
		но кто-то крикнул: <<логарифм "--- многозначная функция>>.
	\end{Rem}

\section{Задача 1}
	\begin{Def}
		Функция $f$ называется однолистной на $A \subseteq \C$, если
		она является инъекцией из $A$ в $\C$ (т.е. никакие два значения в $A$ не совпадают).
	\end{Def}

	\begin{Def}
		$w(z) = \frac12 \left(z + \frac1z\right)$ "--- функция Жуковского.
	\end{Def}
	Вопрос: во что переходят следующие области под действием этой функции?
	\begin{Rem}
		Так удобно изучать функции комплексного переменного: график не нарисовать (он четырёхмерный),
		а вот смотреть, куда переходят конкретные точки или даже кривые "--- вполне.
	\end{Rem}
	\begin{enumerate}[label=\asbuk{enumi}.]
		\item $|z|=r_0$, а $r_0$ вот такое:
			\begin{enumerate}[label=\arabic*.]
			\item $r_0 \neq 1, 0$
			\item $r_0 \to 1$
			\item $r_0 \to 0$
			\end{enumerate}
		\item $|z|<1$
		\item $\arg z = \phi$
	\end{enumerate}
	
	А в каких вообще областях функция Жуковского является однолистной?
	Описать эти области.

	\subsection{Однолистность}
		Пусть у нас есть точки $z_1\neq z_2\neq0$, давайте приравняем значения в них, чтобы
		узнать, когда они могут совпадать:
		\begin{gather*}
			\frac 12 \left(z_1 + \frac1{z_1}\right) = \frac 12 \left(z_2 + \frac1{z_2}\right) \\
			z_1 - z_2 = \frac1{z_2} - \frac1{z_1} \\
			z_1 - z_2 - \frac{z_1-z_2}{z_1z_2} = 0 \\
			(z_1 - z_2)\left(1 - \frac{1}{z_1z_2}\right) = 0 \\
			1 = \frac{1}{z_1z_2} \\
			z_1z_2 = 1
		\end{gather*}
		Таким образом, две разные точки могут перейти в одно и то же значение только если
		они друг другу обратны (т.е. длина обратна, а аргументы противоположны по знаку).
		Таким образом, чтобы функция была однолистной, надо, чтобы в ней не было никаких
		двух обратных друг другу точек.
		Например, если мы берём любую из следующих областей, получаем однолистную функцию:
		\begin{itemize}
			\item $\Im z > 0$
			\item $\Im z < 0$
			\item $|z| < 1$
			\item $|z| > 1$
		\end{itemize}
		А вообще областей, конечно, много.

	\subsection{а}
		Пусть $z=r_0e^{i\phi}$ при $0 \le \phi < 2\pi$.
		Тогда $z$ переходит в:
		\begin{gather*}\label{day160310_task1a_uv}
			w(z) = \frac12 \left(r_0e^{i\phi}+\frac1{r_0} e^{-i\phi}\right) = u+iv \\
			u = \frac12 \left(r_0+\frac1{r_0}\right) \cos \phi \\
			v = \frac12 \left(r_0-\frac1{r_0}\right) \sin \phi \\
		\end{gather*}
		Теперь хотим избавиться от зависимости $u$ и $v$ от переменной $\phi$,
		чтобы получить какое-то уравнение кривой.
		Для этого естесственно попробовать возвести их в квадрат в надежде, чтобы убились $\cos^2\phi + sin^2\phi$.
		Так просто возводить в квадрат нехорошо, потому что коэффициенты при синусах/косинусах разные.
		Давайте поделим $u$ и $v$ на них, а уже потом сложим квадраты:
		\begin{gather*}
			\begin{aligned}
				\frac{2u}{r_0+\frac1{r_0}} &= \cos \phi &
				\frac{2v}{r_0-\frac1{r_0}} &= \sin \phi \\
				\frac{4u^2}{\left(r_0+\frac1{r_0}\right)^2} &= \cos^2 \phi &
				\frac{4v^2}{\left(r_0-\frac1{r_0}\right)^2} &= \sin^2 \phi \\
			\end{aligned} \\
			\frac{4u^2}{\left(r_0+\frac1{r_0}\right)^2} +
			\frac{4v^2}{\left(r_0-\frac1{r_0}\right)^2} = 1
		\end{gather*}
		Получили в общем случае эллипс.
		А формулы для $u$ и $v$ "--- его параметризация.
		
		\subsubsection{а1}
			Если $r_0 \neq 0, 1$, то у нас знаменатели внизу ненулевые и мы получаем в точности некоторый эллипс.

		\subsubsection{а2}
			Пусть $r \to 1$.
			Тогда знаменатель при $v^2$ стремится к нулю, то есть $v$ с приближением $r \to 1$
			должно уменьшаться к нулю сколь угодно близко.
			То есть <<высота>> эллипса уменьшается к нулю.

			Посмотрим на ширину: её легко выяснить из формул для $u$ и $v$.
			При $r \to 1$ получаем коэффициент при $\cos \phi$ равный единице,
			то есть наш эллипс будет стремиться к открытому отрезку от $-1$ до $1$.

		\subsubsection{а3}
			Пусть $r \to 0$.
			Тогда ширина эллипса (коэффициент при $\cos \phi$) стремится к $+\infty$,
			а его высота "--- к $-\infty$.
			Получаем, что эллипс так раздувается в две стороны, до бесконечности.

	\subsection{б}
		У нас изначально было приближение отрезка эллипсом <<снаружи>> (т.е. без отрезка),
		а потом они начинают непрерывно растягиваться на бесконечность при $r \to 0$.
		Таким образом замостим всю плоскость, кроме замкнутого отрезка от $-1$ до $1$.
		Это можно совсем формально проверить, решив уравнение для каждой точки относительно
		$r$ и $\phi$.

		\cautoimg{05_p1}
		\begin{Rem}
			Случай полностью аналогичен $|z|>1$, так как исходная функция симметрична относительно взятия обратного.
		\end{Rem}

	\subsubsection{в}
		У нас есть фигура $\arg z = \phi$ "--- это такой луч.
		То есть в уравнениях \ref{day160310_task1a_uv} теперь хотим избавляться от зависимости от $r$,
		чтобы получить уравнение кривой.
		Давайте теперь перенесём влево всё, кроме $r_0$ и возьмём полусумму:
		\begin{gather*}
			\begin{aligned}
			\frac{2u}{\cos\phi} &= r_0+\frac1{r_0} \\
			\frac{2v}{\sin\phi} &= r_0-\frac1{r_0} \\
			\frac{u}{\cos\phi}+\frac{v}{\sin\phi} &= r_0 \\
			\end{aligned} \\
			\text{Теперь можно подставить это дело в исходное выражение для $u$} \\
			\TODO \text{Тут надо восстанавливать, с ходу не получилось} \\
			\frac{u^2}{\cos^2\phi} - \frac{v^2}{\sin^2\phi} = 1
		\end{gather*}
		Получили гиперболу, если ничего в знаменателях не занулилось.
		А если занулилось, то гипербола вырождается в прямую (?).
		\TODO еще стоит разобраться, куда выгибаются гиперболы при изменении угла,
		или куда переходит область $\alpha \le \phi \le \beta$.

\section{Задача 2}
	Есть функция $e^z$.
	Надо найти области однолистности.
	Потом надо узнать, во что переходят следущие области:
	\begin{enumerate}[label=\asbuk{enumi}.]
		\item $z=x+ic$, $c=\const$, $x \in \R$
		\item $z=c+iy$, $c=\const$, $y \in \R$
		\item $a+2\pi k < \Im z < b + 2\pi k$, $k \in \Z$, $a, b \in R$
	\end{enumerate}

	\subsection{а}
		Было $z=x+ic$.
		Перешло в:
		\[ e^z=e^x\cdot\underbrace{e^{ic}}_{\const} \]
		То есть горизонтальная прямая перешла в луч из нуля под углом $c$.

	\subsection{б}
		Было $z=c+iy$.
		Перешло в:
		\[ e^z=\underbrace{e^c}_{\const}\cdot e^{iy} \]
		То есть вертикальная прямая перешла в окружность радиуса $e^c$ вокруг нуля,
		причём не инъективно, а <<наматываясь вокруг>>.

	\subsection{Однолистность}
		Пусть $e^{x_1+iy_1}=e^{x_2+iy_2}$.
		Это значит, что $x_1=x_2$, а $y_1$ и $y_2$
		отличаются лишь на $2\pi k$ ($k \in \Z$), т.е. задают один и тот же угол.
		Таким образом, в области однолистности не должно быть точек с одинаковой вещественной частью,
		отличающихся в мнимой части на кратное $2\pi$.

	\subsection{в}
		Тут мы как раз пользуемся наблюдением из однолистности: у нас образ области не зависит от $k$.
		Область "--- это семейство горизонтальных прямых $c=\const$, а их образы "--- лучи с углом $c$.
		Объединение таких лучей "--- это угол от $a$ до $b$.

\section{Задача 3}
	Найти области однолистности следующей функции:
	\[
		\sin z = \frac{e^{iz}-e^{-iz}}{2i}
	\]
	А также понять, куда переходит следующая область:
	\[
		\begin{cases}
			-\pi < \Re z < \pi \\
			\Im z > 0
		\end{cases}
	\]

	\begin{Rem}
		Может быть полезно заметить, что синус есть композиция функции Жуковского и экспоненты:
		\[ \sin z = \frac{1}{2} \left(\frac{e^{iz}}{i}+\frac{i}{e^{iz}}\right) \]
	\end{Rem}

	Давайте воспользуемся этим наблюдением, и представим синус как композицию четырёх функций:
	\[
		z
		\xrightarrow{f_1}
		iz
		\xrightarrow{f_2}
		e^{iz}
		\xrightarrow{f_3}
		\frac{e^{iz}}{i}
		\xrightarrow{f_4}
		\frac12\left(\frac{e^{iz}}{i}+\frac{i}{e^{iz}}\right)
	\]
	Чтобы синус был однолистным, нам надо, чтобы все четыре функции были однолистными.
	Функции $f_1$ и $f_3$ неинтересны "--- они биекции и всегда однолистны.
	А остальные нам знакомы: $f_2$ "--- экспонента, $f_4$ "--- функция Жуковского.
	То есть $\sin z' \neq \sin z''$ в следующем случае:
	\begin{gather*}
		\begin{cases}
			\Im iz' - \Im iz'' \neq 2\pi k, k \in \Z \setminus \{ 0 \} \\
			\frac{e^{iz'}}{i} \cdot \frac{e^{iz''}}{i} \neq 1
		\end{cases} \\
		\begin{cases}
			\Re z' - \Re z'' \neq 2\pi k,\, k \in \Z \setminus \{ 0 \} \\
			\frac{e^{i(z'+z'')}}{i^2} \neq 1 \iff
			z'+z'' \neq \pi 2\pi k,\, k \in \Z
		\end{cases}
	\end{gather*}
	Сформулируем то же самое словами: две точки склеиваются синусом, если верно хотя бы одно из двух:
	\begin{enumerate}
		\item
			Их сумма равна $\pi + 2\pi k$, где $k \in \Z$.
			В частности, сумма вещественных частей равна $\pi$, а мнимые противоположны по знаку.
		\item
			Их вещестенные части отличаются на $2\pi k$, где $k \in \Z$.
	\end{enumerate}
	Тогда нам, например, подходит отрезок $[-\sfrac \pi 2, \sfrac \pi 2]$:
	никакие две точки не могут отличаться на $2\pi k$ по вещественной части и никакие
	два различных числа не могут дать в сумме $\pi$.

	Можно еще раздуть этот отрезок до <<стакана>>, разрешив неотрицательную мнимую часть.

	Если же раздувать его, разрешив произвольную мнимую часть, то надо запретить граничные
	точки (иначе $\sfrac{\pi}2+i$ и $\sfrac{\pi}2-i$ склеятся), получаем открытую полосу:
	\[ -\frac \pi 2 < \Re z < \frac \pi 2 \]

	Понять на занятии, куда область переходит, мы, по-видимому, не успели. \TODO

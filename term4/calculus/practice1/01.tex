\setauthor{Егор Суворов}

\chapter{11.02.2016}

\section{Организационное}
Преподаватель: Злотников Илья Константинович.
Лучше звать без отчества.

Email: \href{mailto:zlotnikk@rambler.ru}{zlotnikk@rambler.ru}.

Система получения зачёта: раз в две недели задаётся обязательная домашняя работа.
По ней можно набирать баллы, надо набрать необходимый минимум по каждому дз.
25\% от баллов за домашнее ставятся за то, что оно оформлено в \TeX (а не от руки).
Если не набрать минимум "--- в конце семестра будет дополнительное дз (отдельное на каждую тему),
которое поможет добрать не хватающих баллов.
Сдавать написанное от руки лучше лично, по электронной почте можно сдавать только при болезни.
Если написано в \TeX "--- то стоит сдавать по почте.

Также надо писать общие контрольные: если пишем с первого раза, то достаточно решить все задачи кроме одной,
если со второго раза "--- надо решать все задачи.

По ТФКП было мало лекций, поэтому пока что займёмся подсчётом поверхностных интегралов.

\section{Напоминание интегралов I рода}
Пусть $S$ "--- поверхность в $\R^3$, задана функциями $x(u, v)$, $y(u, v)$, $z(u, v)$
(это просто отображение из $D$ в $\R^3$).
Мы вводили интеграл функции $f$ по поверхности следующим образом:
\begin{equation}\label{int_surf_I}
	\int_{S} f(x, y, z) \d S =
	\int\int f(x(u, v), y(u, v), z(u, v)) \sqrt{EG-F^2} \d u \d v
\end{equation}
При этом $E$, $F$, $G$ "--- некоторая характеристика поверхности:
\begin{gather*}
	E=\left(\partd{x}{u}\right)^2 + \left(\partd{y}{u}\right)^2 + \left(\partd{z}{u}\right)^2 \\
	G=\left(\partd{x}{v}\right)^2 + \left(\partd{y}{v}\right)^2 + \left(\partd{z}{v}\right)^2 \\
	F=\partd{x}{u} \cdot \partd{x}{v} + \partd{y}{u} \cdot \partd{y}{v} + \partd{z}{u} \cdot \partd{z}{v}
\end{gather*}

Например, полезный частный случай: у нас поверхность является графиком функции $z(x, y)$.
Тогда можно взять отображение $(u, v) \to (u, v, z(u, v))$.
Из формулы \ref{int_surf_I} можно вывести следующую:
\begin{equation}\label{int_surf_I_graph}
	\int_{S} f(x, y, z) \d S = \int_D f(x, y, z(x, y)) \sqrt{\left(\partd z x\right)^2+\left(\partd z y\right)^2+1}\d x \d y
\end{equation}

Считаем массу гладкой поверхности (если знаем плотность $\rho(x, y, z)$ в каждой точке):
\[ M = \int_S \rho(x, y, z) \d S \]
Если у нас поверхность не гладкая, то начинаются проблемы с частными производными.
Если поверхность кусочно-гладкая, то её стоит разбить на несколько кусочков, посчитать интеграл отдельно
по кускам и сложить "--- получим интеграл по всех кусочно-гладкой поверхности.

\section{Задача 1}
\begin{gather*}
	\int_S \frac{\d S}{\sqrt{x^2+y^2+z^2}} \\
	S \eqcolon
		\begin{cases}
			x = r \cos u \\
			y = r \sin u \\
			z = v
		\end{cases} \\
	0 \le u \le 2 \pi \\
	0 \le v \le H
\end{gather*}
Воспользуемся другой формулой с лекций: пусть $\vec r(u, v) = (x, y, z)$. Тогда:
\begin{equation}\label{int_surf_I_dot}
	E = \|\vec r'_u\|^2;
	\quad
	G = \|\vec r'_v\|^2;
	\quad
	F = \vec r'_u \cdot r'_v
\end{equation}
Теперь считаем $\vec r$ и его производные:
\begin{align*}
	\vec r(u, v) &= (x(u, v), y(u, v), z(u, v)) = (r \cos u, r \sin u, v) \\
	\vec r'_u &= (-r \sin u, -r \sin v, 0) \\
	\vec r'_v &= (0, 0, 1) \\
	E = r^2 \\
	G = 1 \\
	F = 0 \\
	\sqrt{EG-F^2}=r
\end{align*}
Считаем интеграл по поверхности:
\begin{gather*}
	\int_S \frac{\d S}{\sqrt{x^2+y^2+z^2}} =
	\int_0^{2\pi} \int_0^H \frac{r \d v \d u}{\sqrt{r^2+v^2}} =
	\int_0^{2\pi} \left(\int_0^H \frac{\d v}{\sqrt{r^2+v^2}}\right)r \d u = \\
	= 2 \pi r \int_0^H \frac{\d v}{\sqrt{r^2+v^2}} =
	2 \pi r \left( \ln \left| v + \sqrt{v^2 + r^2} \right| \right)_0^H =
	2 \pi r \left( \ln \left| H + \sqrt{H^2 + r^2} \right| - \ln r \right) = \\
	= 2 \pi r \ln \frac{H + \sqrt{H^2 + r^2}}{r} =
\end{gather*}

\section{Задача 2}
Пусть есть сфера $x^2+y^2+z^2=R^2$, плотность поверхности есть $\rho = \rho_0 \sqrt{x^2+y^2}$.
Хотим найти массу.
Для этого надо взять интеграл по поверхности:
\[ \int_S \rho(x, y, z) \d S \]

\subsection{Способ 1}
\TODO перенести с двух фотографий, в одном месте взяли по частям, в конце сказали <<это просто площадь четвертинки>>

\subsection{Способ 2 (сферическая замена)}
$\alpha$ "--- широта ($-\sfrac \pi 2 \le \alpha \le \sfrac \pi 2$), $\beta$ "--- долгота ($0 \le \beta \le 2\pi$):
\begin{gather*}
	x = R \cos \alpha \cos \beta \\
	y = R \cos \alpha \sin \beta \\
	z = R \sin \alpha \\
	\vec r'_\alpha = (-R \sin \alpha \cos \beta, -R \sin\alpha \sin\beta, R \cos \alpha) \\
	\vec r'_\beta  = (-R \cos \alpha \sin \beta, R \cos\alpha \cos\beta, 0) \\
	E = \|\vec r'_\alpha\|^2 = R^2(\sin^2\alpha(\cos^2\beta + \sin^2\beta)+\cos^2\alpha) = R^2\\
	G = \|\vec r'_\beta\|^2 = R^2(\cos^2\alpha(\sin^2\beta+\cos^2\beta)) = R^2\cos^2\alpha \\
	F = \vec r'_\alpha \cdot \vec r'_\beta = R^2 \cdot 0 + 0 = 0 \\
	\sqrt{EG-F^2} = \sqrt{R^4\cos^2\alpha} = R^2\sqrt{\cos^2\alpha}
\end{gather*}
Считаем интеграл:
\begin{gather*}
	\int_S \rho_0 \underbrace{\sqrt{x^2+y^2}}_{R^2\cos^2\alpha} \d S =
	\rho_0 \int_0^{2\pi} \int_{-\sfrac \pi 2}^{\sfrac \pi 2} \sqrt{R^2\cos^2\alpha} R^2 \sqrt{\cos^2\alpha} \d \alpha \d \beta = \\
	= 2\pi \rho_0 R^3 \int_{-\sfrac \pi 2}^{\sfrac \pi 2} \cos^2\alpha \d \alpha =
	2\pi \rho_0 R^3 \int_{-\sfrac \pi 2}^{\sfrac \pi 2} \frac{1 + \cos 2\alpha}{2 \cdot 2} \d (2 \alpha) =
	\frac{\pi}{2} \rho_0 R^3 \int_{-\pi}^{\pi} (1 + \cos t) \d t = \\
	= \frac{\pi}{2} \rho_0 R^3 \left( \underbrace{\int_{-\pi}^{\pi} \d t}_{2\pi} + \underbrace{\int_{-\pi}^{\pi} \cos t \d t}_{0} \right) =
	\frac{\pi}{2} \rho_0 R^3 \cdot 2\pi =
	\rho_0 \pi^2 R^3
\end{gather*}

\section{Задача 3}
Поверхность $S$ задаётся как граница следующей области:
\[ 0 \le \sqrt{x^2+y^2} \le z \le 2 \]
Хотим посчитать интеграл:
\[ \int_S z^2 \d S \]

\subsection{Введение}
Это поверхность конуса, растущего из начала координат и ограниченного плоскостью $z=2$.
Это кусочно-гладкая поверхность.
Надо отдельно посчитать для верхнего кружочка и отдельно посчитать для внешней поверхности конуса.

\subsection{Верхний кружочек}
\begin{gather*}
	\int_{\substack{x^2+y^2 \le 4\\ z = 2}} z^2 \d S = 4\cdot \pi \cdot 4 = 16 \pi
\end{gather*}

\subsection{Внешняя поверхность}
Считаем по формуле \ref{int_surf_I_graph}.
Для этого надо сказать, что поверхность "--- график функции $z(x, y)$ на круге $\sqrt{x^2+y^2}\le 2$.
\begin{gather*}
	\int_S z^2 \d S =
	\int_{x^2+y^2\le 4} z^2\sqrt{1+\frac{x^2}{x^2+y^2}+\frac{y^2}{x^2+y^2}} \d x \d y =
	\sqrt 2 \int_{x^2+y^2\le 4} (x^2+y^2) \d x \d y = \\
	\text{переходим в полярные координаты} \\
	= \sqrt 2 \int_0^{2\pi} \int_0^2 r^2 \cdot \underbrace{r}_{J} \d r \d \phi =
	2 \sqrt 2 \pi \int_0^2 r^3 \d r =
	2 \sqrt 2 \pi \left(\frac{r^4}{4}\right)|_0^2 =
	8 \sqrt 2 \pi
\end{gather*}

\subsection{Итоговый ответ}
\[ 8\sqrt 2 \pi + 16 \pi \]

\section{Окончание занятия}
К следующему занятию стоит вспомнить, как считать поверхностные интеграла второго рода (от формы).
Также стоит вспомнить формулу Стокса.

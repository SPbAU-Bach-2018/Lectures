\setauthor{Егор Суворов}

\chapter{11.02.2016}

\section{Организационное}
Преподаватель: Злотников Илья Константинович.
Лучше звать без отчества.

Email: \href{mailto:zlotnikk@rambler.ru}{zlotnikk@rambler.ru}.

Система получения зачёта: раз в две недели задаётся обязательная домашняя работа.
По ней можно набирать баллы, надо набрать необходимый минимум по каждому заданию.
25\% от баллов за домашнее ставятся за то, что оно оформлено в \TeX (а не от руки).
Если не набрать минимум "--- в конце семестра будет дополнительное задание (отдельное на каждую тему),
которое поможет добрать нехватающих баллов.
Сдавать написанное от руки лучше лично, по электронной почте можно сдавать только при болезни.
Если написано в \TeX "--- то стоит сдавать по почте.

Также надо писать общие контрольные: если пишем с первого раза, то достаточно решить все задачи кроме одной,
если со второго раза "--- надо решать все задачи.

По ТФКП было мало лекций, поэтому пока что займёмся подсчётом поверхностных интегралов.

\section{Напоминание интегралов I рода}
Пусть $S$ "--- поверхность в $\R^3$, задана функциями $x(u, v)$, $y(u, v)$, $z(u, v)$
(это просто отображение из $D$ в $\R^3$).
Мы вводили интеграл функции $f$ по поверхности следующим образом:
\begin{equation}\label{int_surf_I}
	\int\limits_{S} f(x, y, z) \d S =
	\int\int f(x(u, v), y(u, v), z(u, v)) \sqrt{EG-F^2} \d u \d v
\end{equation}
При этом $E$, $F$, $G$ "--- некоторая характеристика поверхности:
\begin{gather*}
	E=\left(\partd{x}{u}\right)^2 + \left(\partd{y}{u}\right)^2 + \left(\partd{z}{u}\right)^2 \\
	G=\left(\partd{x}{v}\right)^2 + \left(\partd{y}{v}\right)^2 + \left(\partd{z}{v}\right)^2 \\
	F=\partd{x}{u} \cdot \partd{x}{v} + \partd{y}{u} \cdot \partd{y}{v} + \partd{z}{u} \cdot \partd{z}{v}
\end{gather*}

Например, полезный частный случай: у нас поверхность является графиком функции $z(x, y)$.
Тогда можно взять отображение $(u, v) \to (u, v, z(u, v))$.
Из формулы \ref{int_surf_I} можно вывести следующую:
\begin{equation}\label{int_surf_I_graph}
	\int\limits_{S} f(x, y, z) \d S = \int\limits_D \left( f(x, y, z(x, y)) \sqrt{\left(\partd z x\right)^2+\left(\partd z y\right)^2+1} \right) \d x \d y
\end{equation}

Так мы умеем считаем массу гладкой поверхности (если знаем плотность $\rho(x, y, z)$ в каждой точке):
\[ M = \int\limits_S \rho(x, y, z) \d S \]
Если у нас поверхность не гладкая, то начинаются проблемы с частными производными.
Если поверхность кусочно-гладкая, то её стоит разбить на несколько кусочков, посчитать интеграл отдельно
по кускам и сложить "--- получим интеграл по всей кусочно-гладкой поверхности.

\section{Задача 1}
\begin{gather*}
	\int\limits_S \frac{\d S}{\sqrt{x^2+y^2+z^2}} \\
	S \coloneq
		\begin{cases}
			x = r \cos u \\
			y = r \sin u \\
			z = v
		\end{cases} \\
	0 \le u \le 2 \pi \\
	0 \le v \le H
\end{gather*}
Этой такой цилиндр высоты $H$ и радиуса $r$, гладкая поверхность.
Воспользуемся другой формулой с лекций: пусть $\vec r(u, v) = (x, y, z)$. Тогда:
\begin{equation}\label{int_surf_I_dot}
	E = \|\vec r'_u\|^2;
	\quad
	G = \|\vec r'_v\|^2;
	\quad
	F = \vec r'_u \cdot r'_v
\end{equation}
Теперь считаем $\vec r$ и его производные:
\begin{gather*}
	\begin{aligned}
		\vec r(u, v) &= (x(u, v), y(u, v), z(u, v)) = (r \cos u, r \sin u, v) \\
		\vec r'_u &= (-r \sin u, r \cos v, 0) \\
		\vec r'_v &= (0, 0, 1)
	\end{aligned} \\
	\begin{aligned}
		E &= \|\vec r'_u\|^2 =  r^2 \\
		G &= \|\vec r'_v\|^2 = 1 \\
		F &= \vec r'_u \cdot \vec r'_v = -r\sin u \cdot 0 + r\cos v \cdot 0 + 0 \cdot 1 = 0 \\
		\sqrt{EG-F^2} &= r
	\end{aligned}
\end{gather*}
Считаем интеграл по поверхности:
\begin{gather*}
	\int\limits_S \frac{\d S}{\sqrt{\underbrace{x^2+y^2}_{r^2}+\underbrace{z^2}_{v^2}}} =
	\int\limits_0^{2\pi} \int\limits_0^H \frac{\overbrace{\sqrt{EG-F^2}}^{r} \cdot \d v \d u}{\sqrt{r^2+v^2}} =
	\int\limits_0^{2\pi} \left(\int\limits_0^H \frac{\d v}{\sqrt{r^2+v^2}}\right)r \d u = \\
	= 2 \pi r \int\limits_0^H \frac{\d v}{\sqrt{r^2+v^2}} =
	2 \pi r \left.\left( \ln \left| v + \sqrt{v^2 + r^2} \right| \right)\right|_0^H =
	2 \pi r \left( \ln \left| H + \sqrt{H^2 + r^2} \right| - \ln r \right) = \\
	= 2 \pi r \ln \frac{H + \sqrt{H^2 + r^2}}{r}
\end{gather*}

\section{Задача 2}
Пусть есть сфера $x^2+y^2+z^2=R^2$, плотность поверхности есть $\rho = \rho_0 \sqrt{x^2+y^2}$.
Хотим найти массу.
Для этого надо взять интеграл по поверхности:
\[ M = \int\limits_S \rho(x, y, z) \d S \]

\subsection{Способ 1}
Сначала распилим сферу на две полусферы (чтобы можно было выразить $z$ через $x$ и $y$): $z \ge 0$ и $z \le 0$.
Интегралы по ним равны, так как $\rho$ не зависит от $z$.
Посчитаем интеграл только по верхней полусфере и умножим его на два, пересечение полусфер имеет меру ноль, на интеграл не повлияет.
Для верхней полусферы можно выразть:
\begin{align*}
	z(x, y) &= \sqrt{R^2-x^2-y^2} \\
	\partd{z}{x} &= \frac{-2x}{2\sqrt{R^2-x^2-y^2}} = -\frac{x}{z(x, y)} \\
	\partd{z}{y} &= \frac{-2y}{2\sqrt{R^2-x^2-y^2}} = -\frac{y}{z(x, y)}
\end{align*}
Таким образом, верхняя полусфера "--- график функции $z(x, y)$ на круге радиуса $R$.
Пользуемся формулой \ref{int_surf_I_graph}:
\begin{gather*}
	M =
	2\int\limits_{S\cap \{z \ge 0\}} \rho(x, y, z(x, y)) \d S =
	2\iint\limits_{x^2+y^2\le R^2} \left(\rho_0(x^2+y^2) \sqrt{1+\left(\partd z x\right)^2+\left(\partd z y\right)^2} \right)\d x\d y = \\
	= 2\rho_0\iint\limits_{x^2+y^2\le R^2} \left((x^2+y^2) \sqrt{\frac{\overbrace{z^2+x^2+y^2}^{R^2}}{z^2(x,y)}} \right)\d x\d y =
	2\rho_0R\iint\limits_{x^2+y^2\le R^2} \frac{x^2+y^2}{z(x,y)} \d x\d y
\end{gather*}
Так как подыинтегральная функция зависит только от расстояния до $(x, y)$ от начала координат, разумно заменить координаты на полярные.
\begin{gather*}
	M =
	2\rho_0R \int\limits_{0}^{2\pi} \int\limits_0^R \frac{r^2}{\sqrt{R^2-r^2}} \cdot r \cdot \d r \d \phi =
	4\pi\rho_0R \int\limits_0^R \left(\frac{r^2}{\sqrt{R^2-r^2}} \d r\right)\cdot r \circleeq \\
	\text{берём по частям:} \quad
	\int \frac{r^2}{\sqrt{R^2-r^2}} \d r = -\sqrt{R^2-r^2} + C \\
	\circleeq 4\pi\rho_0R \left( \underbrace{\left.\left(\sqrt{R^2-r^2}\cdot r\right)\right|_0^R}_{0} - \int\limits_0^R -\sqrt{R^2-r^2} \d r \right) =
	4\pi\rho_0R \int\limits_0^R \sqrt{R^2-r^2} \d r \circleeq \\
	\text{последний интеграл "--- площадь четверти круга радиуса $R$} \\
	\circleeq 4\pi\rho_0R \cdot \frac{\pi R^2}{4} =
	\pi^2\rho_0R^3
\end{gather*}

\subsection{Способ 2 (сферическая замена)}
$\alpha$ "--- широта ($-\sfrac \pi 2 \le \alpha \le \sfrac \pi 2$), $\beta$ "--- долгота ($0 \le \beta \le 2\pi$):
\begin{gather*}
	\begin{aligned}
		x &= R \cos \alpha \cos \beta \\
		y &= R \cos \alpha \sin \beta \\
		z &= R \sin \alpha
	\end{aligned} \\
	\begin{aligned}
		\vec r'_\alpha &= (-R \sin \alpha \cos \beta, -R \sin\alpha \sin\beta, R \cos \alpha) \\
		\vec r'_\beta  &= (-R \cos \alpha \sin \beta, R \cos\alpha \cos\beta, 0)
	\end{aligned} \\
	\begin{aligned}
		E &= \|\vec r'_\alpha\|^2 = R^2(\sin^2\alpha(\cos^2\beta + \sin^2\beta)+\cos^2\alpha) = R^2\\
		G &= \|\vec r'_\beta\|^2 = R^2(\cos^2\alpha(\sin^2\beta+\cos^2\beta)) = R^2\cos^2\alpha \\
		F &= \vec r'_\alpha \cdot \vec r'_\beta = R^2 \cdot 0 + 0 = 0 \\
		\sqrt{EG-F^2} &= \sqrt{R^4\cos^2\alpha} = R^2\sqrt{\cos^2\alpha}
	\end{aligned}
\end{gather*}
Считаем интеграл:
\begin{gather*}
	\int\limits_S \rho_0 \underbrace{\sqrt{x^2+y^2}}_{R^2\cos^2\alpha} \d S =
	\rho_0 \int\limits_0^{2\pi} \int\limits_{-\sfrac \pi 2}^{\sfrac \pi 2} \sqrt{R^2\cos^2\alpha} \cdot R^2 \cdot \sqrt{\cos^2\alpha} \cdot \d \alpha \cdot \d \beta = \\
	= 2\pi \rho_0 R^3 \int\limits_{-\sfrac \pi 2}^{\sfrac \pi 2} \cos^2\alpha \cdot \d \alpha =
	2\pi \rho_0 R^3 \int\limits_{-\sfrac \pi 2}^{\sfrac \pi 2} \frac{1 + \cos 2\alpha}{2} \cdot \frac{\d (2 \alpha)}{2} =
	\frac{\pi}{2} \rho_0 R^3 \int\limits_{-\pi}^{\pi} (1 + \cos t) \d t = \\
	= \frac{\pi}{2} \rho_0 R^3 \left( \underbrace{\int\limits_{-\pi}^{\pi} \d t}_{2\pi} + \underbrace{\int\limits_{-\pi}^{\pi} \cos t \d t}_{0} \right) =
	\frac{\pi}{2} \rho_0 R^3 \cdot 2\pi =
	\pi^2 \rho_0 R^3
\end{gather*}

\section{Задача 3}
Поверхность $S$ задаётся как граница следующей области:
\[ 0 \le \sqrt{x^2+y^2} \le z \le 2 \]
Хотим посчитать интеграл:
\[ \int\limits_S z^2 \d S \]

\subsection{Введение}
Это поверхность конуса, растущего из начала координат и ограниченного плоскостью $z=2$.
Это кусочно-гладкая поверхность.
Надо отдельно посчитать для верхнего кружочка и отдельно посчитать для внешней поверхности конуса.

\subsection{Верхний кружочек}
Интеграл константной функции на круге площади $4\pi$:
\begin{gather*}
	\int\limits_{\substack{x^2+y^2 \le 4\\ z = 2}} z^2 \d S = 4\cdot \pi \cdot 4 = 16 \pi
\end{gather*}

\subsection{Внешняя поверхность}
Считаем по формуле \ref{int_surf_I_graph}.
Для этого надо сказать, что поверхность "--- график функции $z(x, y)$ на круге $\sqrt{x^2+y^2}\le 2$.
\begin{gather*}
	\begin{aligned}
		z(x, y) = \sqrt{x^2+y^2} \\
		\partd{z}{x} = \frac{x}{\sqrt{x^2+y^2}} \\
		\partd{z}{y} = \frac{y}{\sqrt{x^2+y^2}} \\
	\end{aligned}
	\int\limits_S z^2 \d S =
	\int\limits_{x^2+y^2\le 4} z^2\sqrt{1+\underbrace{\frac{x^2}{x^2+y^2}+\frac{y^2}{x^2+y^2}}_{2}} \d x \d y =
	\sqrt 2 \int\limits_{x^2+y^2\le 4} (x^2+y^2) \d x \d y \circleeq \\
	\text{переходим в полярные координаты} \\
	\circleeq \sqrt 2 \int\limits_0^{2\pi} \int\limits_0^2 r^2 \cdot \underbrace{r}_{J} \d r \d \phi =
	2 \sqrt 2 \pi \int\limits_0^2 r^3 \d r =
	2 \sqrt 2 \pi \left.\left(\frac{r^4}{4}\right)\right|_0^2 =
	8 \sqrt 2 \pi
\end{gather*}

\subsection{Итоговый ответ}
\[ 8\sqrt 2 \pi + 16 \pi \]

\section{Окончание занятия}
К следующему занятию стоит вспомнить, как считать поверхностные интеграла второго рода (от формы).
Также стоит вспомнить формулу Стокса.

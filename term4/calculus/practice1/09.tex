\chapter{04.04.2016}

\section{Задача 1}\label{day160404_task1}
	Осталась \hyperref[day160331_task1]{с прошлого раза}, её разбирает Илья.

	Найти отображение круга $|z|<1$ на круг $|w|<1$, причём:
	\begin{gather*}
		w\left(\frac 12\right) = \frac12 \\
		\arg w'\left(\frac 12\right) = \frac \pi 2
	\end{gather*}

	Мы знаем \hyperref[day160331_circle_auto]{общий вид преобразований круга в себя}.
	Давайте не будем искать сразу искомое отображение $w(z)$, а вместо этого представим его как композицию двух других: $w = g^{-1} \circ f$.
	На диаграмме ниже показаны все отображения и три области "--- $z$ (исходная), $w$ (искомая), $v$ (промежуточная, мы можем выбрать её какой угодно).
	\begin{center}
		\begin{dot2tex}[options=-tmath]
			digraph G {
				rank=TB;
				z; w; v;
				{rank=same; z w};
				z -> w [label="w(z)"];
				z -> v [label="f(z)"];
				w -> v [label="g(w)"];
			}
		\end{dot2tex}
	\end{center}

	Положим $v$ равным единичному кругу (так как мы знаем общий вид преобразований круг в круг).
	На преобразования $f$ и $g$ наложим лишь одно ограничение "--- чтобы они
	переводили точки $z_0=\frac 12$ и $w_0=w(z_0)=\frac 12$ в ноль, тогда в этой точке будет проще дифференцировать.
	\begin{Rem}
		Нам тут повезло, что мы знаем $w(z_0)$ из условия.
	\end{Rem}
	Итого, рассматриваем следующие два преобразования (какие-то параметры мы еще, правда, не знаем):
	\begin{alignat*}{3}
		z &\xlongrightarrow{f} e^{i\phi} \frac{z-z_0}{1-z\bar z_0} &=& e^{i\phi} \frac{z-\frac12}{1-\frac z2} &\quad \phi \in \R& \\
		w &\xlongrightarrow{g} e^{i\theta} \frac{w(z)-w_0}{1-w(z)\bar w_0} &=& e^{i\theta} \frac{w(z)-\frac12}{1-\frac{w(z)}2} &\quad \theta \in \R& \\
	\end{alignat*}
	Мы знаем, что $f=g\circ w$, можем приравнять правые части:
	\begin{gather*}
		e^{i\phi} \frac{z-\frac12}{1-\frac z2} = e^{i\theta} \frac{w(z)-\frac12}{1-\frac{w(z)}2} \\
		\frac{z-\frac12}{1-\frac z2} = e^{i(\theta-\phi)} \frac{w(z)-\frac12}{1-\frac{w(z)}2} \\
		\text{берём производную по $z$} \\
		\frac{z'\left(1-\frac z2\right)-\left(z-\frac12\right)\left(-\frac{z'}{2}\right)}{\left(1-\frac z2\right)^2} =
		e^{i(\theta-\phi)} \frac{w'(z)z'\left(1-\frac{w(z)}2\right) - \left(w(z)-\frac12\right)\left(-\frac{w'(z)z'}2\right)}{\left(1-\frac{w(z)}2\right)^2} \\
		\text{подставляем $z=z_0=\frac 12$, $w=\frac12$} \\
		\frac{\frac34 z' - 0}{\left(1-\frac z2\right)^2} =
		e^{i(\theta-\phi)} \frac{\frac34 w'(z)z' - 0}{\left(1-\frac{w(z)}2\right)^2} \\
		\frac{1}{\left(1-\frac z2\right)^2} = e^{i(\theta-\phi)} \frac{w'(z)}{\left(1-\frac{w(z)}2\right)^2} \\
		\left(\frac{1-\frac{w(z)}2}{1-\frac z2}\right)^2e^{i(\phi-\theta)} = w'(z) \\
		\arg \left(\left(\frac{1-\frac{w(z)}2}{1-\frac z2}\right)^2e^{i(\phi-\theta)} \right) = \arg w'(z) \\
		\arg \underbrace{\left(\frac{1-\frac{w(z)}2}{1-\frac z2}\right)^2}_{\in \R_{+}} + (\phi-\theta) = \frac \pi2 \\
		\phi - \theta = \frac\pi 2 \\
	\end{gather*}
	Отлично, теперь можно подставить во вторую строчку ставшее известным $\phi-\theta$ и получить уравнение от $z$ и $w$:
	\begin{gather*}
		\frac{z-\frac12}{1-\frac z2} = e^{i(\theta-\phi)} \frac{w(z)-\frac12}{1-\frac{w(z)}2} \\
		\frac{z-\frac12}{1-\frac z2} = e^{-i\frac{\pi}{2}} \frac{w(z)-\frac12}{1-\frac{w(z)}2} \\
		\frac{z-\frac12}{1-\frac z2} = -i \cdot \frac{w(z)-\frac12}{1-\frac{w(z)}2} \\
	\end{gather*}

\section{Задача 2}
	Найти отображение верхней полуплоскости на саму себя с такими условиями ($z_0$, $w_0$, $\alpha$ "--- параметры задачи):
	\begin{gather*}
		\Im z > 0 \mapsto \Im w > 0 \\
		w(z_0) = w_0 \\
		\arg w'(z_0) = \alpha
	\end{gather*}
	То есть фиксирована одна точка, её образ и аргумент производной отображения в этой точке.

	Должна получиться по аналогии с предыдущей.
	Мы умеем отображать полуплоскость лишь на круг.
	Давайте скажем, что $u$ в нашем случае опять является кругом, а $f$ и $g$ "--- какие-то отображения на него.
	Опять же будем переводить в ноль точки $z_0$ и $w_0$, чтобы было потом проще дифференцировать.
	\begin{Rem}
		Внезапно, в общем виде отображений из полуплоскости на круг точка, переходящая в ноль, тоже обозначалась $z_0$.
		Совпадение.
	\end{Rem}
	При этом мы искренне надеемся, что оставшейся свободы в угле поворота нам хватит, чтобы нужные $f$ и $g$ вообще существовали.
	\begin{alignat*}{2}
		f(z) &= \frac{z-z_0}{z-\bar z_0} e^{i\theta} &\quad \alpha \in \R& \\
		g(w) &= \frac{w-w_0}{w-\bar w_0} e^{i\phi} &\quad \phi \in \R&
	\end{alignat*}
	Опять приравниваем правые части:
	\begin{gather*}
		\frac{z-z_0}{z-\bar z_0} e^{i\theta} = \frac{w-w_0}{w-\bar w_0} e^{i\phi} \\
		\frac{z-z_0}{z-\bar z_0} = \frac{w-w_0}{w-\bar w_0} e^{i(\phi-\theta)} \\
		\text{обозначим $\phi-\theta\eqcolon\tau$} \\
		\frac{z-z_0}{z-\bar z_0} = \frac{w(z)-w_0}{w(z)-\bar w_0} e^{i\tau} \\
		\text{берём производную по $z$ в точке $z_0$} \\
		\frac{z'(z-\bar z_0) - (\overbrace{z-z_0}^0)(\dots)}{(z-\bar z_0)^2} = \frac{w'(z)z'(w(z)-\bar w_0) - (\overbrace{w(z)-w_0}^0)(\dots)}{(w(z)-\bar w_0)^2} e^{i\tau} \\
		\frac{z'(z_0-\bar z_0)}{(z_0-\bar z_0)^2} = \frac{w'(z_0)z'(w_0-\bar w_0)}{(w_0-\bar w_0)^2} e^{i\tau} \\
		\frac{z_0-\bar z_0}{(z_0-\bar z_0)^2} = \frac{w_0-\bar w_0}{(w_0-\bar w_0)^2} \cdot w'(z_0) \cdot e^{i\tau} \\
		\frac{w_0-\bar w_0}{z_0-\bar z_0} = w'(z_0) \cdot e^{i\tau} \\
		\arg \frac{w_0-\bar w_0}{z_0-\bar z_0} = \arg w'(z_0) \cdot e^{i\tau} \\
		\arg (w_0-\bar w_0) - \arg (z_0-\bar z_0) = \tau + \arg w'(z_0)
	\end{gather*}
	Почти готово: так как $w_0$ и $z_0$ лежали в верхней полуплоскости, то аргумент $w_0 - \bar w_0$ равен $\sfrac{\pi}2$, аналогично для $z_0-\bar z_0$.
	То есть слева стоит ноль:
	\begin{gather*}
		0 = \tau + \alpha \\
		\tau = -\alpha \\
	\end{gather*}
	Теперь подставляем ставшее явным $\tau$ в третью строчку, получаем уравнение:
	\[
		\frac{z-z_0}{z-\bar z_0} = \frac{w(z)-w_0}{w(z)-\bar w_0} e^{-i\alpha}
	\]

\section{Задача 3}
	Перегнать область из пересечения двух окружностей в полоску $-1 < \Re w < 1$.
	\begin{center}
		\begin{tabular}{cc}
		До & После \\
		\includegraphics{09_p3_a.pdf} & \includegraphics{09_p3_b.pdf} \\
		\end{tabular}
	\end{center}

	Давайте при помощи ДЛП перегоним эти окружности в две прямые, по шагам.
	Сначала перегоняем пересечение окружностей в бесконечность как-нибудь, для этого достаточно, чтобы в знаменателе стояло $z-i$.
	Например, можно дополнительно сделать ноль неподвижной точкой и взять следующее преобразование:
	\[ z \xlongrightarrow{w_1} \frac{z}{z-i} \]
	Посмотрим, куда перешли окружности "--- в две параллельные прямые.
	Рассмотрев точки $1$, $-1$ на первой окружности, легко понять, куда она перешла:
	\begin{align*}
		w_1(1) &= \frac{1}{1-i} = \frac{1+i}{2} \\
		w_1(-1) &= \frac{-1}{-1-i} = \frac{1-i}{2} \\
	\end{align*}
	Вторая окружность тоже перешла в вертикальную прямую (потому что должна быть параллельна образу первой окружности), причём проходящую через ноль.
	\begin{center}
		\includegraphics{09_p3_c.pdf}
	\end{center}

	Теперь надо эти прямые раздвинуть между собой (расстояние было равно 0.5, а надо 2):
	\[ z \xlongrightarrow{w_2} 4z \]
	И потом сдвинуть влево на единицу:
	\[ z \xlongrightarrow{w_3} z - 1 \]
	Получим так (и нам опять повезло):
	\[
		w(z) = w_3(w_2(w_1(z))) = \frac{4z}{z-i} - 1 = \frac{3z+i}{z-i}
	\]

	Альтернативное решение от семинариста: можно удачно выбрать три точки на исходной картинке и перевести их
	ангармоническим соотношением в какие-нибудь три точки на целевой картинке.

\section{Задача 4}
	Перегнать:
	\[
		\left\{ -\frac \pi 2 < \Re z < \frac \pi 2 \right\}
		\cap
		\{ \Im z > 0 \}
		\mapsto
		\{ \Im z > 0 \}
	\]
	При этом хотим, чтобы точки перешли так:
	\begin{align*}
		-\frac \pi 2 &\mapsto -1 &
		0 &\mapsto 0 &
		\frac \pi 2 &\mapsto 1
	\end{align*}
	\begin{center}
		\begin{tabular}{cc}
		До & После \\
		\includegraphics{09_p4_a.pdf} & \includegraphics{09_p4_b.pdf} \\
		\end{tabular}
	\end{center}

	Решаем сначала первую часть "--- перегон фигуры, не обращая внимания на точки.
	У нас есть столбик, давайте его <<уроним>> на левый бок: сдвинем вправо и домножим на $i$ для поворота.
	\begin{center}
		\begin{tabular}{cc}
		Сдвинули & Повернули \\
		\includegraphics{09_p4_c.pdf} & \includegraphics{09_p4_d.pdf} \\
		\end{tabular}
	\end{center}
	\[
		w_1(z) = i\left(z + \frac \pi 2\right)
	\]
	Теперь возьмём экспоненту: $w_2(z)=e^z$.
	Она переведёт точки $a+bi$ в точки $e^a \cdot e^{ib}$.
	Так как $a < 0$, то модуль образа будет строго меньше единицы.
	Так как $0 < b < \pi$, то аргумент образа будет от 0 до $\pi$.
	Получили полукруг:
	\begin{center}
		\includegraphics{09_p4_e.pdf}
	\end{center}

	Давайте теперь сделаем ДЛП, которое переведёт круг в какой-нибудь угол.
	Переведём точку $a_2=1$ в бесконечность, а точку $a_1=-1$ "--- в ноль.
	\[
		w_3(z) = \frac{z+1}{z-1}
	\]
	Получили прямой угол, если возведём в квадрат ($w_4(z)=z^2$), получим как раз верхнюю полуплоскость.
	\begin{center}
		\begin{tabular}{cc}
		После $w_3$ & После $w_4$ \\
		\includegraphics{09_p4_f.pdf} & \includegraphics{09_p4_g.pdf} \\
		\end{tabular}
	\end{center}
	Осталось скомбинировать все $w_i$, чтобы получить явную формулу.
	Мы этим, кажется, не занимались.

	Теперь смотрим, куда по факту перешли точки из условия.
	Нам несказанно везёт: они изначально лежали границе.
	Стало быть, после преобразований они тоже лежат на границе.
	А три точки на прямой в три точки на той же самой прямой сдвигами и растяжениями вдоль этой прямой мы уж как-нибудь переведём.

	\TODO две понятно как переводить, а с трёмя почему полчится?

\section{Области однолистности}
	\TODO
	Для корня нельзя, чтобы был нестягиваемый путь вокруг нуля, иначе будет беда с непрерывностью.
	Ноль "--- точка ветвления корня.

	Для логарифма точка ветвления тоже ноль, обычно его рассматривают на всей плоскости, кроме неотрицательных
	вещественных чисел.

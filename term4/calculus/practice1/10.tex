\chapter{21.04.2016}
	В прошлый раз (14.04.2016) была контрольная работа.
	В этот раз "--- её разбор.

	Каждую задачу решил хотя бы один человек.

\section{Задача 1}
	Вычислите интеграл:
	\[ \int\limits_L (y^2+z^2)\d x + (x^2+z^2)\d y + (x^2+y^2)\d z \]
	Тут $L$ "--- кривая, полученная в результате пересечения полусферы:
	\[ S=\{(x, y, z)\in \R^3 \mid x^2+y^2+z^2=2Rx \} \cap \{ z > 0 \} \]
	И поверхности:
	\[ S_1 = \{(x, y, z) \in \R^3 \mid x^2+y^2=2rx, r \in \R \}\]
	При этом верно, что $0 < r < R$.
	Кривая пробегает так, что ограниченная ей на внешней стороне полусферы $S$ наименьшая область остаётся слева.

	Сначала разберёмся, что у нас за сфера:
	\begin{align*}
		x^2+y^2+z^2 &= 2Rx \\
		x^2-2Rx+y^2+z^2 &= 0 \\
		x^2-2Rx+R^2+y^2+z^2 &= R^2 \\
		(x-R)^2+y^2+z^2 &= R^2
	\end{align*}
	То есть это сфера радиуса $R$ с центром в $(R, 0, 0)$.

	Теперь вторая поверхность.
	Это бесконечная поверхность, которая не зависит от $z$, посмотрим на её проекцию на плоскость $OXY$:
	\begin{align*}
		x^2+y^2 &= 2rx \\
		x^2-2rx+y^2 &= 0 \\
		x^2-2rx+r^2+y^2 &= r^2 \\
		(x-r)^2+y^2 &= r^2
	\end{align*}
	Это такой цилиндр, вытянутый из окружности радиуса $r$ в точке $(r, 0)$ на плоскости.

	Вид сверху:
	\begin{center}
		\includegraphics{10_p1.pdf}
	\end{center}
	Это у нас полусфера, а с ней пересекается цилиндр.

	Давайте скажем, что $S_2$ "--- это поверхность $S$, вырезанная $S_1$ (шапка сферы), тогда $\partial S_2 = L$.
	Пишем формулу Стокса:
	\begin{gather*}
		\int\limits_L (y^2+z^2)\d x + (x^2+z^2)\d y + (x^2+y^2)\d z =
		\iint\limits_{S_2} (2y-2z)\d y \d z + (2z - 2x) \d z \d x + (2x - 2y) \d x \d y = \\
		2\iint\limits_{S_2} (y-z)\d y \d z + (z - x) \d z \d x + (x - y) \d x \d y \circleeq
	\end{gather*}
	Давайте заметим, что $S_2$ "--- это график вот функции на области $D$ ($D$ "--- проекция $S_2$, окружность):
	\begin{gather*}
		(u, v) \mapsto (u, v, z(u, v)) \\
		z = \sqrt{R^2-(u-R)^2-v^2} = \sqrt{2uR-u^2-v^2} \\
		\partd{z}{u} = \frac{2R-2u}{2z} = \frac{R_u}{z} \\
		\partd{z}{v} = \frac{-2v}{2z} = \frac{-v}{z}
	\end{gather*}

	Теперь продолжаем писать интеграл, используем формулу для интегралов по поверхности II рода через определитель
	(у нас как раз верхняя сторона графика, поэтому ориентация правильная):
	\begin{gather*}
		\circleeq 2\iint\limits_{D} \begin{vmatrix}
			v - z & z - u & u - v \\
			1 & 0 & \frac{R-u}{z} \\
			0 & 1 & \frac{-v}{z}
		\end{vmatrix}
		\d u \d v =
		2 \iint\limits_{D} \left((v-z)\frac{u-R}{z}+(z-u)\frac{v}{z}+u-v\right)\d u \d v = \\ =
		2 \iint\limits_{D} \left((v-z)\frac{u}{z}+(z-u)\frac{v}{z}-(v-z)\frac{R}{z}+u-v\right)\d u \d v = \\ =
		2 \iint\limits_{D} \left(\underbrace{\frac{vu}{z}-u+v-\frac{uv}{z}+u-v}_{0}-(v-z)\frac{R}{z}\right)\d u \d v =
	\end{gather*}
	% for page break
	\begin{gather*}
		=
		2R \iint\limits_{D} \left(1-\frac{v}{z}\right) \d u \d v =
		2R \underbrace{\left(\iint\limits_{D} 1 \d u \d v\right)}_{\mu D} - 2R\left(\iint\limits_{D} \frac{v}{z} \d u \d v\right) = \\ =
		2R\pi r^2 - 2R\iint\limits_{D} \frac{v}{\sqrt{2ur-u^2-v^2}} \d u \d v = \\ =
		2R\pi r^2 - 2R\int_{-r}^r \left(\int\limits_{-\sqrt{r^2-u^2}}{+\sqrt{r^2-u^2}} \frac{v}{\sqrt{2ur-u^2-v^2}} \d v\right) \d u \circleeq \\
		\text{внутри стоит интеграл от нечётной функции по симметричному отрезку, он всегда ноль} \\
		\circleeq 2R\pi r^2
	\end{gather*}
	\begin{Rem}
		Также у внутреннего интеграла можно честно угадать первообразную и увидеть, что она является чётной функцией:
		\[
			\int \frac{v}{\sqrt{2ur-u^2-v^2}} \d v = -\sqrt{2ur-u^2-v^2} + C
		\]
		Как угадать: у нас сверху что-то линейное по $v$, а снизу "--- корень от его квадрата, значит, ответом будет чуть подправленный знаменатель.
	\end{Rem}

\section{Задача 2}
	Вычислите следующий интеграл:
	\[
		\iint\limits_S (1+2x)\d y \d z + (2x+3y) \d z \d x + (3y + 4z) \d x \d y
	\]
	Тут $S$ "--- внутренняя сторона поверхности $|x-y+z|+|y-z+x|+|z-x+y|=a$.

	Давайте сначала поймём, что же это за поверхность.
	Егор заметил, что уравнение очень похоже на что-то из манхэттенской метрики "--- слева бы стояло в точности расстояние
	до нуля в манхэттенской метрике, если бы только под модулями были $x$, $y$, $z$, а не что-то более сложное.
	Однако при помощи линейного преобразования координат нашу поверхность можно перевести в более простую:
	\begin{align*}
		(x, y, z) &\mapsto (\alpha, \beta, \gamma) \\
		(x, y, z) &\mapsto (\underbrace{x+y-z}_{\alpha}, \underbrace{x-y+z}_{\beta}, \underbrace{-x+y-z}_{\gamma}) \\
		|x-y+z|+|y-z+x|+|z-x+y| &\mapsto |\beta| + |\gamma| + |\alpha| = a
	\end{align*}
	То есть получили, что от линейного преобразования наша поверхность переходит в поверхность октаэдра
	\begin{Rem}
		Поверхность октаэдра "--- это точки, равноудалённые от нуля в манхэттенской метрике.
		В двумерном случае равноудалённые от нуля точки образуют ромбик (это должно быть очевидно).
		Вывести трёхмерный случай из двумерного можно, зафиксировав одну координату (скажем, $\gamma$) и посмотрев на срез
		фигуры при фиксированном $\gamma$ "--- мы получаем точки, равноудалённые от нуля на $a-|\gamma|$, то есть ромбик.
		Объединение ромбиков как раз образует октаэдр.
	\end{Rem}
	(октаэдр "--- это окружность в манхэттенской метрике, в двумерном случае имеем ромбик ).
	Назовём этот октаэдр $K'$, а само преобразование "--- $\phi$.
	Давайте заранее посчитаем якобиан $\phi$, он нам потом потребуется:
	\[
		|J| =
		\left|\det \begin{pmatrix}
			1 & 1 & -1 \\
			1 & -1 & 1 \\
			-1 & 1 & 1
		\end{pmatrix}\right| =
		|1\cdot(-1-1) - 1 \cdot(1-(-1)) + (-1)\cdot(1-1)| =
		|-2-2| = |-4| = 4
	\]

	Значит, до линейного преобразования она ограничивала его прообраз "--- это какой-то многогранник.
	Что самое важное "--- прообраз является компактом (назовём этот компакт $K$, тогда $S=\partial K$).
	Значит, можно применить формулу Гаусса-Остроградского.
	Так как она считает интеграл по внешней поверхности, а нам надо по внутренней, потребуется поставить минус:
	\begin{gather*}
		\iint\limits_{\partial K} (1+2x)\d y \d z + (2x+3y) \d z \d x + (3y + 4z) \d x \d y = \\
		-\iiint\limits_{K} \left(\partd{1+2x}{x} + \partd{2x+3y}{y} + \partd{3y+4z}{z}\right)\d V =
		-9\iiint\limits_K \d V \circleeq
	\end{gather*}
	Нам, по сути, требуется посчитать объём прообраза многогранника.
	Можно либо вспомнить, что объём от линейного преобразования меняется в $|J|$ раз, либо честно изменить координаты в интеграле и получить то же самое
	(не запутайтесь с тем, что $\phi$ переводит из старых в новые, а не наоборот, как обычно):
	\begin{gather*}
		\circleeq -9\iiint_{K'} \frac{\d V}{|J|} = -\frac{9}{4} \mu K' \circleeq
	\end{gather*}
	Теперь вопрос в том, что такое объём октаэдра.
	Октаэдр составлен из восьми одинаковых прямоугольных тетраэдров (по одному на каждую комбинацию знаков трёх координат),
	причём у каждого тетраэдра <<катеты>> по длине равны ровно $a$.
	Это легко увидеть, нарисовав пересечение октаэдра с каждой из координатных осей "--- оно будет в точке $a$ (это следует из уравнения поверхности).
	Объём прямоугольного тетраэдра со <<катетами>> длины $a$ равен $\sfrac{a^3}{6}$ "--- это можно либо запомнить, либо честно взять трёхмерный интеграл
	(там получится интеграл площади прямоугольного равнобедренного треугольника от 0 до $a$).
	Значит, объём октаэдра знаем, можем посчитать ответ:
	\begin{gather*}
		\mu K' = 8 \cdot \frac{a^3}{6} = \frac{4a^3}{3} \\
		\circleeq -\frac{9}{4} \cdot \frac{4a^3}{3} = \frac{-9a^3}{3} = -3a^3
	\end{gather*}

\section{Задача 3}
	Найдите функцию, конформно отображающую $|z-1-i|<2$ на $|w|<1$ так, чтобы $w(i)=0$ и $\arg w'(i) = \frac{\pi}{2}$.

	Мы умеем хорошо отображать единичные круги в самих себя.
	Давайте тогда подвинем исходный круг с центром в ноль и уменьшим радиус в два раза, чтобы потом отобразить результат во что надо:
	\[ z \to \frac{z - i - 1}{2} \]
	Мы знаем общий вид отображений единичного круга на самого себя:
	\[
		w(z) = \frac{z-z_0}{1-\bar z_0 z} e^{i\alpha}
	\]
	Помним, что мы уже $z$ как-то поменяли (чтобы перевести круг в центр), чтобы получить единичный круг, подставляем,
	получаем, что мы ищем ответ на задачу в виде
	\[
		w(z) =
		\frac{\frac{z-i-1}{2}-z_0}{1-\frac{(z-i-1)\bar z_0}{2}} e^{i\alpha} =
		\frac{z-i-1-2z_0}{2-(z-i-1)\bar z_0} e^{i \alpha}
	\]
	Теперь подставляем условие $w(i)=0$:
	\begin{gather*}
		0 = w(i) =
		\frac{i-i-1-2z_0}{2-(i-i-1)\bar z_0} e^{i \alpha} =
		\frac{-1-2z_0}{2+\bar z_0} e^{i \alpha} \\
		z_0 = -\frac{1}{2} \\
		\text{выяснили $z_0$, можно подставить} \\
		w(z) =
		\frac{z-i-1-2\cdot(-\sfrac 12)}{2-(z-i-1)\cdot(-\sfrac 12)} e^{i\alpha} =
		\frac{z-i}{2+\frac{z-i-1}{2}} e^{i\alpha} =
		\frac{2(z-i)}{4+z-i-1} e^{i\alpha}
		\frac{2(z-i)}{3+z-i} e^{i\alpha}
	\end{gather*}
	Теперь берём производную в точке $i$ и приравниваем:
	\begin{gather*}
		w'(i) =
			\frac{2\cdot(3+z-i)-2(z-i)\cdot 1}{(3+z-i)^2}e^{i\alpha} =
			\underbrace{\frac{6+2z-2i-2z+2i}{9}}_{\in\R}e^{i\alpha} \\
		\frac{\pi}{2} = \arg w'(i) = \alpha
	\end{gather*}
	Теперь мы знаем оба параметра ответа, можно окончательно записать:
	\[
		w(z) = \frac{2(z-i)}{z-i+3} \underbrace{e^{i\sfrac{\pi}{2}}}_{i}
	\]

\section{Задача 4}
	Отобразить $|z-i|>1 \cap |z+i|>1$ на верхнюю полуплоскость.
	\begin{center}
		\begin{tabular}{cc}
		До & После \\
		\includegraphics{10_p4_a.pdf} & \includegraphics{10_p4_b.pdf} \\
		\end{tabular}
	\end{center}

	Для начала хотим перевести окружности в прямые.
	Это можно сделать при помощи ДЛП, если его особенность будет лежать сразу на двух окружностях.
	То есть особенность должна быть в нуле.
	Отсюда сразу понятен знаменатель "--- $z$.

	Давайте возьмём следующее преобразование: (в числителе выбрана двойка, чтобы были попроще вычисления)
	\[ w_1(z) = \frac{2}{z} \]
	Две окружности перейдут в две параллельные прямые.
	Также надо взять ещё одну точку на окружности и узнать наклон прямых.
	Окажется, что они перешли в прямые $\Im z = \pm 1$.
	Еще надо подставить внутреннюю точку и убедиться, что мы действительно получили полоску между прямыми (а не внешнюю область):
	\begin{center}
		\includegraphics{10_p4_c.pdf}
	\end{center}

	Переводить разного вида полоски в полуплоскость мы умеем только преобразованием $e^z$.
	В частности, мы знаем, что полуплоскость прекрасно получается из бесконечной полоски от $\Im z = 0$ до $\Im z = \pi$:
	\begin{align*}
		z &= a + bi & a \in \R, 0 < b < \pi \\
		e^z &= e^{a+bi} = e^a\cdot e^{ib} = e^a(\cos b + i \sin b) = & a \in \R, 0 < b < \pi \\
		&= x + iy & x \in R, 0 \le y
	\end{align*}
	Значит, надо текующую полоску подвинуть и отмасштабировать, чтобы подправить ширину (сейчас "--- 2, а надо $\pi$):
	\[
		w_2(w_1) = (w_1 + i)\cdot \frac{\pi}{2}
	\]
	Получаем такое преобразование:
	\[
		e^{\left(\sfrac{2}{z}+i\right)\cdot\sfrac{\pi}{2}}
	\]

\section{Задача 5}
	Отобразить внешность эллипса $\frac{x^2}{a^2}+\frac{y^2}{b^2}=1$ на внешность единичного круга.

	\begin{Rem}
		Пытаться отображать при помощи ДЛП эллипс в окружность бесполезно.
		Оно работает только с обобщёнными \textit{окружностями}.
	\end{Rem}

	Как обычно, основная возня в том, чтобы перевести границу эллипса в границу круга (окружность).
	После этого мы проверим, что внешность перешла во внешность.
	Если это не так "--- сделаем инверсию единичного круга, внешность и внутренность поменяются местами.

	Как перегонять границу в границу?
	Раз эллипс "--- значит функция Жуковского.
	Правда, ей обычно удобнее работать в обратную сторону: перегонять единичную окружность в эллипс.

	Функция Жуковского выглядит так:
	\[
		w = \frac12\left(z+\frac1z\right)
	\]
	Давайте подставим в неё какую-нибудь точку в полярных координатах:
	\begin{gather*}
		z=re^{i\phi} \\
		w=\frac12\left(re^{i\phi}+\frac1r e^{-i\phi}\right)=
		\frac{r+\frac1r}{2}\cos\phi + i\frac{r-\frac1r}{2}\sin\phi \\
		\text{Выразим косинус и синус} \\
		\cos\phi = \frac{2}{r+\frac1r}\Re z \\
		\sin\phi = \frac{2}{r-\frac1r}\Re z \\
		\text{Запишем основное тригонометрическое тождество} \\
		\cos^2\phi + \sin^2\phi = 1 \\
		\frac{u^2}{{\underbrace{\left(2\left(r+\frac1r\right)\right)}_{\alpha}}^2} +
		\frac{v^2}{{\underbrace{\left(2\left(r-\frac1r\right)\right)}_{\beta}}^2} = 1
	\end{gather*}
	Получили, что образ точки $z$ удовлетворяет уравнению некоторого эллипса, а сам эллипс определяется лишь модулем этой точки.
	Более того, так можно получить произвольную точку на этом эллипсе "--- надо лишь выбрать правильный угол $\phi$.
	Значит, окружность радиуса $r$ переходит в эллипс с полуосями $\alpha$ и $\beta$.

	Сразу получать эллипс с конкретными полуосями будет сложно.
	Давайте сначала получим эллипс с правильным соотношением полуосей ($\sfrac{a}{b}$), а потом его просто отмасштабируем.
	Мы знаем, как зависят полуоси эллипса от $r$, пишем уравнение:
	\begin{gather*}
		\frac{a}{b} = \frac{\alpha}{\beta} = \frac{r+\frac1r}{r-\frac1r} = \frac{r^2+1}{r^2-1} \\
		ar^2-a = br^2+b \\
		r^2(a-b) = a+b \\
		r = \sqrt{\frac{a+b}{a-b}}
	\end{gather*}
	Тут нам важно, что $a>b$.
	Если это не так, то надо дополнительно разбираться с поворотом эллипса на 90 градусов (мы не стали) и с крайним случаем <<$a=b$>>.
	Дальше просто домножим преобразование на константу ($\sfrac{b}{\beta}$), чтобы отмасштабировать эллипс нужным образом.

	Таким образом, мы умеем из единичной окружности получать эллипс за три шаа:
	\begin{enumerate}
		\item $w_1(z) = z \cdot r$ "--- получили окружность радиуса $r=\sqrt{\sfrac{(a+b)}{(a-b)}}$
		\item $w_2(w_1) = \frac12\left(w_1+\frac1{w_1}\right)$ "--- функция Жуковского, переводит в эллипс с правильным соотношением полуосей
		\item $w_3(w_2) = w_2 \cdot \frac{b}{\beta}$ "--- выправляем длину полуосей
	\end{enumerate}
	Их композиция такая:
	\[
		w=\frac12\left(zr+\frac{1}{zr}\right)\cdot\frac{\beta}{b}i
	\]
	Ответом будет являться обратное к нему.
	В принципе, такой ответ уже принимался.

	Обратное преобразование к первому и третьему пункту искать просто.
	Осталось найти обратное к функции Жуковского.
	На паре мы не справились, Илья расскажет в следующий раз.

\chapter{19.05.2016}

\section{Задача 1}
	\[
		\int_0^\infty \frac{\sin x}{x} \d x
	\]

	Решала Лиза Третьякова.
	Контур "--- две окружности, верхний полубублик.

\section{Задача 2}
	\[
		\int_0^\infty \frac{\ln x}{(x^2+1)^2)} \d x
	\]
    
    Решал Дима Лапшин.
    Контур такой же.

    Ответ "--- $-\sfrac{\pi}{8}$.

\section{Задача 3}
	\[
		\int_0^\infty \frac{\ln x}{(x+a)^2+b^2)} \d x
	\]
	
	Разбирал Илья.
	Контур "--- хитрый.

	Если мы считаем в лоб, то будет проблема, так как нужный интеграл сократится:
	у нас сверху от OX идём слева направо, а снизу "--- справа налево.
	Надо полечить так: возвести числитель в квадрат, логарифм в квадрате сократится,
	а обычный останется.

\section{Магия}
	\begin{Rem}
		Если хочется посчитать $\int_0^\infty f(x) \d x$ вещественной функции, то мы
		не можем нарисовать контур от $0$ до $\infty$.
		Будет полезно посчитать интеграл $f(x)\ln x$, тогда может получиться лучше "---
		там приравняются вещественные и мнимые части.
	\end{Rem}

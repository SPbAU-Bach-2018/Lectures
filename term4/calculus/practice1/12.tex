\chapter{05.05.2016}

\section{Постановки задач}
	\TODO
	Пусть у нас есть логарифм.
	Это многозначная функция.
	Например, можно взять такую точку $z$:
	\begin{align*}
		z &= 1 + i = \sqrt 2 e^{i\sfrac{\pi}4} = \sqrt 2 e^{i(\sfrac{\pi}4+2\pi)} \\
		\ln z &= \ln 2 + i\frac{\pi}4 \\
		\ln z &= \ln 2 + i\frac{\pi}4 + 2\pi \\
	\end{align*}
	Поэтому если мы смотрим на одну точку, то понять её значение непонятно.
	Более того, даже просто говорить, сколько раз добавить $2i\pi$ к значению логарифма, это не очень хорошо.
	Не все функции такие понятные, как логарифм, может быть ещё более адская многозначность.

	Поэтому частенько говорят про <<ветки>> многозначных функций при помощи кривых: берут кривую
	и фиксируют где-то на ней значение $\Ra$ автоматически посчитаются значения на остальных точках кривой.
	Обычный логарифм обозначается $\ln z$, а вот логарифм с фиксированной веткой-кривой $\Gamma$
	обозначается $(\ln z)_{\Gamma}$.
	Значение в конкретной точке в запись не входят "--- оно не влияет на вычисления.

	То есть в задачах нас просят переписать функции в таком виде, что все функции внутри однозначны.
	Подозревать функцию в многозначности можно, если она является обратной к неоднолистной функции.
	Например, $\ln$ обратен к экспоненте, а она склеивает точки.

\section{Однозначные функции}
	Базовая однозначная функция "--- $f(z)=z$.
	Если у нас есть две однозначные функции $F(z)$ и $G(z)$, то мы можем получить следующие однозначные функции:
	\begin{enumerate}
		\item $F(z) \pm G(z)$
		\item $F(z) \cdot G(z)$
		\item $\sfrac{F(z)}{G(z)}$
		\item $F(G(z))$
		\item $F'(z)$
		\item $\int_{\Gamma} F(t) \d t$ "--- для фиксированной кривой $\Gamma$
	\end{enumerate}

\section{Задача 1}
	
\section{} % 01
	Напоминание.
	Пусть $G$ "--- область (линейно связная, открытая) в $\R^2$ и есть 1-форма $\omega = P\d x + Q\d y$.
	Тогда по формуле Грина $\d \omega = (Q'_x - P'_y)\d x \wedge \d y$, дальше будем с такими работать.
	(мнемоника: $\d (P \d x) = (\d P) \wedge (\d x)$, $\d x \wedge \d y = -\d y \wedge \d x$, $\d x \wedge \d x = 0$).
	Дифф. $k$-форма "--- это такая хрень с дифференциалами (при каждом слагаемом "--- $k$ разных), можно их таскать/заменять, но интуитивного смысла "--- как у тензоров (почти нет).
	Можно смотреть, как на обобщённую функцию "--- если очень захотеть, может что-то и посчитать (в точке даст внешнюю форму, которая уже линейный оператор),
	но обычно мы её просто таскаем туда-сюда под интегралами.
	Например, $0$-форма "--- это просто функция.

	% стр. 1
	Определение: форма \textit{замкнута}, если $\d \omega=0$ ($\iff Q'_x=P'_y$).
	1-форма $omega$ \textit{точная}, если есть такая функция (0-форма) $F$, что $\d F = \omega$
	(называется \textit{первообразная}).
	Помним, что есть первообразная $\iff$ интеграл по любой замкнутой кривой ноль.
	Форма \textit{локально точная}, если у каждой точки в некоторой окрестности есть первообразная.

	% стр. 2
	Теорема: если $P$ и $Q$ непрерывно дифф., то замкнутость $\iff$ локальная точность.
	$\La$: посмотрели в окрестность, нашли $\d F = \omega$, расписали левую часть в частных,
	выразили $P$ и $Q$ через $F$, очевидно нужное равенство, т.к. порядок производных неважен в случае непрерывной дифф. $P$ и $Q$ (второй семестр).
	% стр. 3
	$\Ra$: взяли любой шарик, взяли замкнутую простую кривую внутри, она ограничила область, тогда можно Грина и по замкнутости ноль.
	Важно то, что кривая ограничила именно область, а не дырявое нечто.
	Следствие: замкнутая форма в любом круге имеет первообразную.

	Пример проблемы: форма из плоскости без нуля $\frac{x \d y - y \d x}{x^2+y^2}$ замкнута (проверка в лоб),
	но не точная "--- взяли единичную окружность, по ней $\oint=2\pi$ (криволинейный интеграл).

\section{} % 02
	Если есть замкнутая $\omega$, то $f$ "--- первообразная вдоль пути $\gamma$, если
	у каждой точке есть окр. на пути и на ней есть первообразная $\d F = \omega$, причём $F(\gamma(t))=f(t)$
	(сама первообразная ищется в окрестности окрестности).
	Теорема: у замкнутой с точностью до константы есть единственная первообразная вдоль $\gamma$.
	Единственность: если есть две разные, то в каждом отрезке отличаются на константу (т.к. обычные
	первообразные на константу), а отрезок $[a,b]$ покрывается конечным числом отрезков, ура, константы одинаковые.
	Существование: взяли вокруг точек путей области, компакт, взяли прообразы (неоднозначно), снова компакт, построим на кусках $[a,b]$ первообразные, согласуем по константам.

	% стр. 4
	Интеграл вдоль пути равен разности на концах первообразной вдоль пути, т.к. по аддитивности
	разбили путь на куски из конструкуции, знаем хорошее про обычные первообразные, они согласованы, всё посокращалось.

\section{} % 03
	% стр. 5
	Лемма Лебега: $K$ "--- компакт в метрическом, есть его покрытие открытыми, тогда есть $r>0$ такое,
	что вокруг любой точки шар с радиусом $r$ лежит в неком элементе покрытия (необязательно в $K$).
	Док-во: взяли вокруг каждой точки шар $B_{r_a}$, лежащий в каком-то открытом элементе,
	поделили радиусы пополам (чтобы вокруг любой точки из шара можно было такой же шар построить),
	выбрали конечное подпокрытие, выбрали минимум радиуса.

\section{} % 04
	% стр. 4
	$\gamma_0$ и $\gamma_1$ с общими концами гомотопны в $G$, если есть непрерывное $\gamma\colon [a,b] \times [0,1] \to G$
	(вторая координата "--- время), что в момент ноль имеем $\gamma_0$, в момент 1 имеем $\gamma_1$,
	а начало/конец гвоздями прибиты.
	А если пути замкнуты, то разрешаем двигать начало вдоль пути, но конец к нему приклеен.
	Нулевой путь "--- одна точка, константа.
	Путь стягиваем, если гомотепен нулевому.
	Определение: если <<2D-путь>> $\gamma \colon [a,b]\times[c,d]\to G$ непрерывен и $\omega$ замкнута,
	то бывает первообразная $f$ относительно $\gamma$: для любой точки $\gamma$ можно выбрать окрестность
	$\gamma$, которую накрыть окрестностью $G$, в которой есть правильная первообразная $F$.

	% стр. 5
	Теорема: первообразная замкнутой относительно непрерывного $\gamma$ всегда есть.
	Взяли непрерывный образ компакта ($\gamma$), у каждой точки "--- окрестность, где есть $F$,
	по Лебегу нашли $r$, нарезали по непрерывности $[a,b]\times[c,d]$ на соотв. прямоугольники.
	На каждом взяли первообразную, посогласовывали соседние прямоугольники (т.к. окрестности
	их образов пересекаются по открытому, а в пересечении получается первообразная).

	% стр. 6
	Следствие: интеграл замкнутой по гомотопным одинаков (т.к. $\int = f(b,0)-f(a,0)$ и $f(a, 0)=F(\gamma(a, 0))=F(\gamma(a, 1))=f(a,1)$).
	Следствие: интеграл замкнутой по стягиваемому ноль (т.к. $f(a,0)=F(\gamma(a, 0))=F(\gamma(b, 0))=f(b,0)$, но в конспекте по-другому).

	$G$ односвязна, если все пути стягиваются.
	В односвязных любая замкнутая точна, т.к. интеграл по любой петле ноль.
	Примеры: звёздная (есть точка, из которой можно дойти \textbf{по отрезку} до любой другой),	выпуклые.
	А вот плоскость с дыркой не односвязна (но линейно связна).

	Обозначения: $\mathbb{T}$/$r\mathbb{T}$ "--- единичная окружность,
	$\mathbb{D}$/$r\mathbb{D}$ "--- открытый диск, $r\bar\mathbb{D}$ "--- замкнутый диск.

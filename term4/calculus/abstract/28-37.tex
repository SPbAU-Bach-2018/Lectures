\section{} % 28
	$\Omega$ односвязна, $f$ голоморфна и не ноль, тогда есть $g$ такая, что $e^g=f$ (типа логарифм).
	Док-во: $f$ голоморфна, $f'$ голоморфна, $\frac{f'}{f}$ голоморфна, замкнута (именно замкнута),
	а по односвязности локально точна, отсюда есть первообразная.
	Взяли $g=F+c$, показываем $e^g=f$, для этого достаточно $e^{-F}f=const$,
	для этого берём производную, она ноль.

	В явном виде "--- $\ln|f(z)| + i\arg f(z)$.
	Следствие: если если есть $\C$ без кривой из нуля на бесконечность, то на ней есть логарифм $\log$.
	Обозначаем: $\ln$ обычный вещественный, $\log$ "--- какой-то комплексный, $\arg$ "--- какой-то аргумент,
	$\Ln z$/$\Arg z$ "--- непрерывные на конкретной области и с фикс. значением в одной точке.
	
	Следствие: если зафиксировать $p$, то в такой же односвязной области можно ввести $z^p$ как $e^{p \Ln z}$.
	Если $p$ вещественно, то модуль сохраняется.
	Если $p$ целое, то неважно, какой выбирать $\Ln z$.
	Замечание: не всегда $\Ln a + \Ln b = \Ln ab$, не всегда $a^pb^p=(ab)^p$ именно из-за этого.

\section{} % 29
	\TODO

\section{} % 30
	\TODO

\section{} % 31
	\TODO

\section{} % 32
	\TODO

\section{} % 33
	\TODO

\section{} % 34
	\TODO

\section{} % 35
	\TODO

\section{} % 36
	\TODO

\section{} % 37
	\TODO

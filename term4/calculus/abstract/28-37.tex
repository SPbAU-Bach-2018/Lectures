\section{} % 28
	$\Omega$ односвязна, $f$ голоморфна и не ноль, тогда есть $g$ такая, что $e^g=f$ (типа логарифм).
	Док-во: $f$ голоморфна, $f'$ голоморфна, $\frac{f'}{f}$ голоморфна, замкнута (именно замкнута),
	а по односвязности локально точна, отсюда есть первообразная.
	Взяли $g=F+c$, показываем $e^g=f$, для этого достаточно $e^{-F}f=const$,
	для этого берём производную, она ноль.

	В явном виде "--- $\ln|f(z)| + i\arg f(z)$.
	Следствие: если если есть $\C$ без кривой из нуля на бесконечность, то на ней есть логарифм $\log$.
	Обозначаем: $\ln$ обычный вещественный, $\log$ "--- какой-то комплексный, $\arg$ "--- какой-то аргумент,
	$\Ln z$/$\Arg z$ "--- непрерывные на конкретной области и с фикс. значением в одной точке.
	
	Следствие: если зафиксировать $p$, то в такой же односвязной области можно ввести $z^p$ как $e^{p \Ln z}$.
	Если $p$ вещественно, то модуль сохраняется.
	Если $p$ целое, то неважно, какой выбирать $\Ln z$.
	Замечание: не всегда $\Ln a + \Ln b = \Ln ab$, не всегда $a^pb^p=(ab)^p$ именно из-за этого.

\section{} % 29
	\TODO

\section{} % 30
	\TODO

	$\sum_{n=1}^\infty \frac{1}{n^2} = \frac{\pi^2}{6}$, т.к.
	заменим числитель на $x^n$, получим $\phi(x)$, хотим узнать $\phi'(1)$
	(в нуле ноль).
	Можно взять производную, получить $\sfrac{-\ln(1-x)}{x}$, надо посчитать
	интеграл от 0 до 1, см. предыдущий билет.

\section{} % 31
	Мероморфная $f$ с полюсами $a_i$ (конечное число) и устранимой/полюсом в $\infty$
	представляется в виде $C+G(z)+\sum G_k(z)$, где $G_k(z)$ "--- главные (отрицательные) части Лорана в точках $a_i$,
	а $G(z)$ "--- главная часть в $+\infty$ (т.е. все строго положительные $z^k$, т.к. на бесконечности именно они дают проблему).
	Док-во: смотрим на $h(z)=f(z)-G(z)-\sum G_k(z)$, у неё вообще нет полюсов не в бесконечности:
	они могли быть только в $a_k$, но в них либо всё ок, либо устранимая особенность.
	А в бесконечности ряд Лорана для $f(z)-G(z)$ (и, видимо, $h(z)$) имеет неположительные степени,
	тогда на бесконечности устранимая особенность.
	Значит, $h$ непрерывна на $\C$ и имеет предел, по Лиувиллю константа, ура.

	Следствие: если есть просто мероморфная (бесконечное число полюсов в $\C$) и есть последовательность $R_n \to \infty$,
	причём максимумы $f$ по радиусам $R_n$ к нулю, то $f(z)=\lim_n \sum G_k(z)$.
	\TODO

\section{} % 32
	\TODO

\section{} % 33
	Если $f$ голоморфна в $\Omega$ за исключением $P$ полюсов (с учётом кратности) внутри контура,
	то интеграл по контуру от $f'/f$ равен $\2 \pi i (N-P)$,
	где $N$ "--- число нулей (с учётом кратности) внутри контура.

	Док-во: сначала теорема о вычетах, теперь в нулях распишем по кратности нуля $f=(z-a)^mg$
	напишем $f'/f=m/(z-a)+g'/g$, посчитали вычет.
	В полюсах почти аналогично честно расписали, получили $-m/(z-a)$, успех.

	Следствие: если $f$ просто голоморфна, то $P=0$, формула упрощается.
	Следствие: а если так, то принцип аргумента: $2\pi N$ равно приращению аргумента $f(z)$ на контуре
	($\Delta_C(\arg(f(z)))$, т.к. первообразная $f'/f$ есть $\Ln f$, при интеграле убивается
	вещественная часть, $i$ сокращается, остаётся только изменение аргумента.

\section{} % 34
	Теорема Руше: Если $f, g$ голоморфны и на стягиваемом контуре верно $|g|\le |f|$, то $N_f = N_{f+g}$.
	Док-во: считаем его через приращение аргумента, разница равна $\Delta_C(1+g/f)$, но эта кривая
	ходит в мелком круге с центром в 1, вокруг нуля не обойдёт.

	Основная теорема алгебры: берём многочлен, $f$ "--- старший коэффициент, $g$ "--- хвост,
	берём окружность достаточно большого радиуса, внутри у $z^n$ и у многочлена одинаковое число корней,
	ура.

\section{} % 35
	Рассмотрим $f(z, w)=\sum a_{n,k}z^nw^k$, хотим преобразовать так, чтобы
	коэффициент при $z^n\zeta^{-1}$ был $a_{n,n}$, потом проинтегрируем по контуру по $\zeta$,
	всё остальное занулится, а это останется с коэф. $2\pi i$.
	Для этого берём ф-цию $f(\zeta, z/\zeta)/\zeta$, меняем сумму с интегралом,
	выносим интеграл по $\zeta$, он ноль при $n \neq k$ и $2\pi i$ иначе.
	Например, если $f(z,w)=\sum (z+w)^n=1/(1-z-w)$, то получаем на диагонали в точности $\binom{2n}{n}$.
	Производящая функция "--- $1/\sqrt{1-4z}$.

	А чтобы считать произведение Адамара ($\sum A_kB_kz^k$), надо перемножить производящие
	и потом выполнить диагонализацию.

\section{} % 36
	\TODO

\section{} % 37
	\TODO

\section{} % 08
	% стр. 10
	Если есть замкнутая кривая не содержащая $a$, то $Ind(\gamma, a)=\frac{\phi(c)-\phi(b)}{2\pi}$,
	где кривая непрерывно задана в полярных координатах (тогда $\phi$ оборачивается несколько раз).
	Альтернативно (без док-во): провели луч из точки, посчитали разность числа пересечений в каждую сторону.

	Если кривая не проходит через ноль, то $\int{\d z}{z}=2\pi i Ind$, т.к. честно расписали
	$z$ в полярных координатах, получили первообразную $\ln r(t) + i\phi(t)$.
	Следствие: формула для $Ind(\gamma, a)=\frac{1}{2\pi i}\int\frac{\d z}{z - a}$.
	Интегральная формула Коши: если $f$ голоморфна и кривая стягиваема,
	то $\int_{\gamma} \frac{f(z)}{z - a}\d z=2\pi i f(a) Ind(\gamma, a)$,
	т.к. вводим $g(z)=\frac{f(z)-f(a)}{z-a}$ (в $a$ по непрерывности $f'(a)$).
	Она голоморфна (т.к. непрерывно и голоморфна без одной точки), тогда форма замкнута,
	интеграл $g$ ноль, вынесли слагамое, успех.

	Следствие: если $f$ голоморфна возле $\mathbb{D}$, то такой интеграл по окружности
	либо $2\pi i f(a)$ (если $a$ внутри), либо ноль (если снаружи).

\section{} % 09
	% стр. 10
	Если $f$ голоморфна в $r\mathbb{D}$, то в нём же и аналитична.
	Взяли $r_2$, в котором голоморфна, докажем, что в $r_1<r_2$ аналитична.
	Выразим $f(z)$ через интеграл по $r_2\mathbb{T}$ (переменная $\zeta$), потом $\frac{1}{\zeta-z}=\sum \frac{z^n}{\dots}$,
	подынтеграл равномерно сх-ся (мажорируется рядом), меняем местами, получили Тейлора и формулу для $a_n$.

	% стр. 11
	Следствие: голоморфна $\iff$ аналитична, т.к. если аналитична, то нужная производная есть.
	Если голоморфна, то бесконечно дифф. (т.к. аналитична).
	Если голоморфна, то производная тоже.
	Если голоморфна, то $\Re f$ и $\Im g$ гармонические, т.к. они дважды дифф.

\section{} % 10
	Морера: если $f(z) \d z$ локально точна (перепутано с замкнутостью), то $f$ голоморфна.
	Док-во: возьмём точку, в ней есть первообразная (значит, голоморфна), а её производная ($f$) тоже голоморфна.

	Следствие: если $f \in C(\Omega)$ и голоморфна без отрезка, то голоморфна.
	Следствие: если $f$ непрерывна и голоморфна без набора точек (без предельных), то голоморфна.
	
	Интегральная теорема Коши для компакта: если $K$ "--- компакт с кусочно-гладкой границей,
	а $f$ голоморфна на области, то интеграл по границе от $\frac{f(z)}{z-a}$ либо ноль (если $a \notin K$),
	либо $2\pi i f(a)$ (если лежит).
	Док-во, если не лежит: раз под интегралом голоморфное, то его частная производная по $\bar z$ ноль
	(условие Коши), значит, интеграл из какого-то там замечания ноль.
	А если лежит, то вычли из компакта шарик, получили компакт, по нему ноль, а по шарику уже знаем что надо
	из обычного Коши.

\section{} % 11
	Если $f\in H(R\mathbb{D})$, $r<R$ и $f$ разложена в Тейлора, то $|a_n| \le \frac{M(r)}{r^n}$,
	где $M(r)=\max_{|z|=r} |f(z)|$.
	Док-во: берём формулу для $a_n$ через интеграл, оцениваем по модулю втупую.
	Функция целая, если голоморфна на всей $\C$.

	Лиувилль: если целая и ограничена, то константа.
	Написали разложение в ряд, он везде сх-ся к функции, но на больших кругах
	коэффициенты стремятся в ноль, значит, ноль.

\section{} % 12
	Неконстантный многочлен из $\C[x]$ имеет корень.
	Если не так, то смотрим на голоморфную $\frac{1}{P(z)}$, она вне какого-то круга
	больше единицы по модулю, а внутри (круг "--- компакт) достигнет минимума.
	Значит, $1/P(z)$ отделена от нуля, значит, $P(z)$ ограничена.

\section{} % 13
	Теорема о среднем (среднее на окружности есть значение в центре): $f$ голоморфна, внутри $\Omega$ есть замкнутый круг $a+r\bar\mathbb{D}$.
	Тогда $f(a)=\frac{1}{2\pi}\int_0^{2\pi} f(a+re^{i\phi})\d\phi$, т.к. посчитали интеграл
	от $\frac{f(z)}{z-a}$ по той же окружности.

	Следствие: при тех же условиях $f(a)=\frac{1}{\pi r^2} \int_{|z-a|\le r} f(x) \d x \d y$.
	Док-во: считаем $a=0$, расписали интеграл через двойной с заменой координат на полярные,
	получили $f(0)$ на площадь круга.

\section{} % 14
	Принцип максимума: если в точке есть нестрогий локальный максимум модуля, то в ней же $f=const$.
	Это верно для любой голоморфной (и подходящей в теорему о среднем).
	Случай $f(a)=0$ тривиален, иначе повернём всё так, чтобы $f(a)\in \R_+$.
	Посчитаем через теорему о среднем для круга (не окружности), $f(z)\le|f(z)|\le f(a)$,
	получим, что везде равенства, т.е. $|f(z)|=const$.
	Теперь аналогично, дописав везде $\Re$ и $\Im$ (можно занести под интеграл), $\Re f(z) = const$.

	Следствие: если $f$ голоморфна на ограниченном линейно связном $\Omega$ и непрерывна на его замыкании,
	то модуль в $\Omega$ не больше максимума модуля на границе.
	Док-во: если максимум модуля на границе, то всё ок.
	Если внутри "--- то возьмём круг, где функция постоянна, возьмём другую точку,
	соединим ломаной, покроем её такими кругами (конечным числом), успех, всё равно.

	Лемма Шварца: если $f\colon \mathbb{D} \to \mathbb{D}$ и $f(0)=0$, то $|f(z)|\le |z|$
	и равенство есть только в случае $f(z)=e^{i\phi} z$.
	Док-во: объявим $g(z)=f(z)/z$ (в нуле "--- $f'(0)$ по непрерывности).
	Она голоморфна в круге без точки, непрерывна $\Ra$ голоморфна.
	Модуль на круге радиуса $r$ не больше $1/r$, устремили к единице, итого $|| \le 1$, успех.
	А если максимум достигся, то $g=const$ и $|g|=1$, успех.

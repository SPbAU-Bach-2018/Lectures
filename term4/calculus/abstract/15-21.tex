\section{} % 15
	Ряд Лорана формально "--- как Тейлор, но в две стороны бесконечность.
	Члены от $-1$ до $-\infty$ "--- главная часть, он обычный ряд относительно $\sfrac{1}{z}$.
	Считаем, что Лоран сх-ся, когда сх-ся главная и неглавная часть, а весь Лоран сх-ся тогда в кольце $R_1<|z|<R_2$.
	Из обычных рядов, если $R_1<r_1<r_2<R_2$, то в $r_1 \le r_2$ есть равномерная сходимость, т.е.
	там можно почленно дифф. и интегрировать.

	Единственность: если $f$ голоморфна в кольце и равна ряду Лорана, то $a_n$ однозначны
	по формуле, как в Тейлоре с интегралом: $a_n=\frac{1}{2\pi i}\int_{r\mathbb{T}}\frac{f(\zeta)}{\zeta^{n+1}}\d \zeta$.
	Док-во: берём и считаем интеграл, меняем интеграл с суммой местами (равномерная сходимость), успех.

	Нер-во Коши верно и для Лорана: $|a_n|\le\frac{M(r)}{r^n}$.

\section{} % 16
	Если $f$ голоморфна в кольце $R_1 < R_2$, то в этом же кольце Лоран.
	\TODO

\section{} % 17
	$a$ изолированная особая, если $f$ голоморфна в лишь в проколотой окрестности.
	Если предел есть, то устранима; если бесконечен "--- полюс; если нет "--- существенная.

	Эквивалентности: $f$ можно доопределить до голоморфной $\Ra$ в ряде Лорана нет
	главной части (т.к. если доопределить, то получим Тейлора) $\Ra$ $a$ устранима
	(подставляем $a$ в Тейлора) $\Ra$ $f$ ограничена в окрестности (раз непрерывна и ограничена)
	$\Ra$ можно доопределить до голоморфной.
	Док-во последнего: разложим в Лорана (можем же), по нер-ву Кош с $r\to 0$ имеем $a_{-n}\to 0$,
	доопределим $f(a)=a_0$.

\section{} % 18
	Если $f$ голоморфна, то $\forall n \colon f^{(n)}(a)=0$ равносильно тому, что $f=0$ в некотором шарике
	(раз голоморфна, то раскладывается в Тейлора).
	Также они равносильны тому, что $f=0$ везде:
	положим $\bar\Omega$ равным мн-ву точек, в окрестности которых ноль ($a \in \bar\Omega$), оно открытое.
	Но каждая точка лежит с предельной, так как все производные непрерывны.

	Следствие: если голоморфные $f$ и $g$ равны в шарике, то они равны.
	Аналитическое продолжение $f_1$ "--- такая голоморфная $f_2$ на большем множестве,
	что она совпадает с $f_1$ в общих точках.
	Если пересечение имеет какой-нибудь шарик, то продолжение единственно.

\section{} % 19
	Корень бывает только у $f \nequiv 0$.
	Кратность "--- младший коэфф. Тейлора (не Лорана, мы же определены), не равный нулю.
	Теорема: мн-во нулей дискретно.
	Взяли ноль кратности $k$, тогда $f=(z-a)^kg(z)$, $g$ отделена от нуля в окрестности, поэтому $f$ тоже.
	Следствие: если голоморфная $f \nequiv 0$, то в любом компакте лишь конечное число корней, иначе есть сходящаяся
	подпоследовательность, упс, ещё корень по непрерывности.

	Теорема о единственности: если есть последовательность с пределом в $\Omega$, причём $f(a_k)=g(a_k)$, то $f=g$ везде,
	т.к. рассмотрели $f-g$.
	Полезно тем, что мы знаем о единственности доопределения $e^x$, $\sin$.

\section{} % 20
	Если $a$ "--- полюс $f$, то есть голоморфная $g$ в окрестности $a$, обратная к $f$ в проколотой, и равная нулю в $a$.
	В проколотой легко определить, в $a$ по непрерывности, получаем голоморфную (единственную).
	Кратность/порядок полюса "--- это кратность такого нуля.

	Если $f$ голоморфна в проколотой окрестности, то $a$ "--- полюс $\iff$ в ряде Лорана лишь конечное число ненулевых в главной части.
	Если полюс, то нашли $g(z)=(z-z_0)^kh(z)$, выразили $f$, ура.
	В обратную так же.
	А ещё порядок равен вот индексу ровно этого младшего коэффициента.

	Следствие: если $a$ "--- особая точка, то устранима $\iff$ главной части нет,
	полюс $\iff$ конечное число слагаемых, существенна $\iff$ бесконечное число слагаемых.

\section{} % 21
	Если $a$ существенна, то можно найти последовательность $\to a$, причём $f(z)$ будет стремиться к любому числу.
	Или, что равносильно, в любой окрестности $a$ замыкание образа $f$ есть $\C$.
	Док-во: в окрестности функция неограничена, пусть не принимает значение $A$.
	Тогда заведём $g(z)=1/(f(z)-A)$ "--- голоморфная, тогда отсюда выражается $f(z)$.
	$a$ "--- особенность $g$, точно не устранимая и не полюс (иначе $f \to A$).
	А если существенная, то есть последовательность с пределом в $\infty$ и в ней опять $f \to A$.

	Теорема Пикара (без док-ва): в существенной особенности образ (без замыкания) либо $\C$, либо $\C$ без одной точки.

\section{} % 15
	Ряд Лорана формально "--- как Тейлор, но в две стороны бесконечность.
	Члены от $-1$ до $-\infty$ "--- главная часть, он обычный ряд относительно $\sfrac{1}{z}$.
	Считаем, что Лоран сх-ся, когда сх-ся главная и неглавная часть, а весь Лоран сх-ся тогда в кольце $R_1<|z|<R_2$.
	Из обычных рядов, если $R_1<r_1<r_2<R_2$, то в $r_1 \le r_2$ есть равномерная сходимость, т.е.
	там можно почленно дифф. и интегрировать.

	Единственность: если $f$ голоморфна в кольце и равна ряду Лорана, то $a_n$ однозначны
	по формуле, как в Тейлоре с интегралом: $a_n=\frac{1}{2\pi i}\int_{r\mathbb{T}}\frac{f(\zeta)}{\zeta^{n+1}}\d \zeta$.
	Док-во: берём и считаем интеграл, меняем интеграл с суммой местами (равномерная сходимость), успех.

	Нер-во Коши верно и для Лорана: $|a_n|\le\frac{M(r)}{r^n}$.

\section{} % 16
	Если $f$ голоморфна в кольце $R_1 < R_2$, то в этом же кольце Лоран.
	\TODO

\section{} % 17
	\TODO

\section{} % 18
	\TODO

\section{} % 19
	\TODO

\section{} % 20
	\TODO

\section{} % 21
	\TODO


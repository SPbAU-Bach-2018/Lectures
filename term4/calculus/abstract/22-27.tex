\section{} % 22
	$f$ мероморфна, если голоморфна в области без точек $a_1, \dots, a_i, \dots$,
	а в них имеет полюса.
	Теорема: $f'$ тогда тоже мероморфна, полюса в одних точках, порядок увеличивается на единицу.
	Док-во: в старых не-полюсах точно нет особенностей, а в полюсе пишем Лорана и почленно дифференцируем.

\section{} % 23
	Определим $\bar \C = \C \cup \{ \infty \}$.
	Предел равен $\infty$, если модуль туда улетел.
	С таким пределом можно:
	брать обратный с ограниченным в числителе (будет ноль),
	умножать на отделённый от нуля (будет $\infty$),
	добавлять ограниченный (будет $\infty$).
	
	Сфера Римана: поставили на плоскость сферу диаметра 1,
	проецируем из северного полюса ($\infty$ "--- это он).
	Можно явно написать формулы преобразования.
	Можно мерять расстояние <<на сфере>> (евклидово), до бесконечности получим $1/\sqrt{1+|z|^2}$.
	Сходимость на $\bar \C$ равносильна сходимости на сфере Римана.
	Следствие: $\C$ компакт.

	Если ф-ция голоморфна возле $\infty$, то $\infty$ либо устранима (если есть конечный предел),
	либо полюс (есть бесконечный), либо существенна.
	Равносильно: $\infty$ устранима, в Лоране только неположительные.
	$\infty$ полюс $\iff$ конечное число ненулей при $z^n$.
	$\infty$ существенен $\iff$ бесконечное число ненулей в не-главной части.
	Определение: $f$ голоморфна в $\infty$, если голоморфна рядом и $\infty$ устранима.

	Теперь можно переписывать теоремы.
	Лиувилль: если голоморфна на $\bar \C$, то константа,
	так как ограничена на компакте и тогда ссылаемся на старого Лиувилля.

\section{} % 24
	Вычет в изолированной особой точке $\res_{z=z_0} f$ "--- коэффициент при $z^{-1}$.
	А в бесконечности полагаем равным минус коэффициенту при $z^{-1}$ в точке 0.
	Лемма: если $f$ сходится в кольце, то интеграл по окружности внутри равен $2 \pi i \res_{z=0} f$,
	т.к. на окружности $f$ сх-ся равномерно, меняем местами сумму и интеграл,
	всё занулилось, кроме $n=-1$.

	Теорема Коши о вычетах: если голоморфна в $\Omega$ без конечного числа особых, то
	интеграл по границе компакта (если особые не на ней) равен $2\pi i$ на сумму вычетов.
	Док-во: вырезали из $K$ окрестности особых точек, внутри особых точек нет $\Ra$ интеграл по границе ноль.
	А потом добавляем вырезанные окружности, успех.
	Следствие: если $f$ голоморфна в $\bar \C$, кроме конечного числа точек,
	то сумма вычетов везде и на бесконечности равна нулю.
	Док-во: взяли интеграл по большому кругу, с одной стороны это сумма вычетов внутри,
	с другой "--- это минус вычет в бесконечности (т.к. обход границы другой), успех.

\section{} % 25
	\TODO

\section{} % 26
	\TODO

\section{} % 27
	\TODO


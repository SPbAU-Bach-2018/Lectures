\section{} % 22
	$f$ мероморфна, если голоморфна в области без точек $a_1, \dots, a_i, \dots$,
	а в них имеет полюса.
	Теорема: $f'$ тогда тоже мероморфна, полюса в одних точках, порядок увеличивается на единицу.
	Док-во: в старых не-полюсах точно нет особенностей, а в полюсе пишем Лорана и почленно дифференцируем.

\section{} % 23
	Определим $\bar \C = \C \cup \{ \infty \}$.
	Предел равен $\infty$, если модуль туда улетел.
	С таким пределом можно:
	брать обратный с ограниченным в числителе (будет ноль),
	умножать на отделённый от нуля (будет $\infty$),
	добавлять ограниченный (будет $\infty$).
	
	Сфера Римана: поставили на плоскость сферу диаметра 1,
	проецируем из северного полюса ($\infty$ "--- это он).
	Можно явно написать формулы преобразования.
	Можно мерять расстояние <<на сфере>> (евклидово), до бесконечности получим $1/\sqrt{1+|z|^2}$.
	Сходимость на $\bar \C$ равносильна сходимости на сфере Римана.
	Следствие: $\C$ компакт.

	Если ф-ция голоморфна возле $\infty$, то $\infty$ либо устранима (если есть конечный предел),
	либо полюс (есть бесконечный), либо существенна.
	Равносильно: $\infty$ устранима, в Лоране только неположительные.
	$\infty$ полюс $\iff$ конечное число ненулей при $z^n$.
	$\infty$ существенен $\iff$ бесконечное число ненулей в не-главной части.
	Определение: $f$ голоморфна в $\infty$, если голоморфна рядом и $\infty$ устранима.

	Теперь можно переписывать теоремы.
	Лиувилль: если голоморфна на $\bar \C$, то константа,
	так как ограничена на компакте и тогда ссылаемся на старого Лиувилля.

\section{} % 24
	Вычет в изолированной особой точке $\res_{z=z_0} f$ "--- коэффициент при $z^{-1}$.
	А в бесконечности полагаем равным минус коэффициенту при $z^{-1}$ в точке 0.
	Лемма: если $f$ сходится в кольце, то интеграл по окружности внутри равен $2 \pi i \res_{z=0} f$,
	т.к. на окружности $f$ сх-ся равномерно, меняем местами сумму и интеграл,
	всё занулилось, кроме $n=-1$.

	Теорема Коши о вычетах: если голоморфна в $\Omega$ без конечного числа особых, то
	интеграл по границе компакта (если особые не на ней) равен $2\pi i$ на сумму вычетов.
	Док-во: вырезали из $K$ окрестности особых точек, внутри особых точек нет $\Ra$ интеграл по границе ноль.
	А потом добавляем вырезанные окружности, успех.
	Следствие: если $f$ голоморфна в $\bar \C$, кроме конечного числа точек,
	то сумма вычетов везде и на бесконечности равна нулю.
	Док-во: взяли интеграл по большому кругу, с одной стороны это сумма вычетов внутри,
	с другой "--- это минус вычет в бесконечности (т.к. обход границы другой), успех.

\section{} % 25
	Если $a$ "--- полюс первого порядка, то $\res=\lim_{z\to a} (z-a)f(a)$ (раскрыть Лорана).
	Если полюс первого порядка и $g$, $h$ "--- голоморфны в окрестности, $g(a)\neq0$, $h(a)=0$, $h'(a)\neq 0$, то
	$\res \frac{g}{h} = \lim \frac{g}{h'}$ (подставили предыдущее).
	Если полюс $k$-го порядка, то надо домножить на $(z-a)^k$, взять $k-1$ производную, потом предел, потом поделить на $(k-1)!$
	(просто из Лорана) "--- в ней уже нестрашно, если возьмём $k$ больше, чем на самом деле (получим просто ноль).

	Если особая устранимая в $\infty$ (т.е. Лоран только неположительный),
	то $\res_{\infty} = \lim (f(\infty) - f(z))\cdot z$ (опять же из Лорана).
	А если же при этом $f(z)=\phi(\sfrac 1z)$ и $\phi$ голоморфна около нуля, то $\res=-\phi'(0)$
	(подставили).
	Также общая формула в $\infty$: $\res = -\res_0 \frac{f(1/z)}{z^2}$ (опять по Лорану).

	Для чётной $f$ вычеты в $0$ и в $\infty$ нулевые (т.к. в ряде только чётные степени).

\section{} % 26
	Считаем на верхнем полукруге радиуса $R$ плюс кусок прямой.
	Полукруг оценили как длина дуги на максимум (знаменатель по треугольнику отделён от нуля), всё ноль.
	Вычеты "--- какие-то $\sfrac{-1}{2n}$ на какие-то кратные углы.
	Итого ответ $\pi/(n\sin \sfrac{\pi}{2n})$, там сократятся какие-то полуразности ешек и прочее.

	Лемма Жордана: если есть последовательность $R_n \to \infty$,
	на дужках соответствующих радиусов с ушами вниз до $y=-a$ супремум $|g(z)|=M_n \to 0$,
	то предел интегралов по дужкам функции $e^{i\lambda z} g(z)$ равен нулю (при $\lambda >0$).
	Док-во: нижние дужки оценим как длину на супремум (который $\to 0$, т.к. $e^{i\lambda z}$ ограничено),
	их длина примерно $a$ (ограничена).
	Четверть окружности из первого квадранта: (стр. 26) \TODO

\section{} % 27
	Лемма о полувычете: если есть полюс порядка 1, то интеграл по дуге радиуса $\epsilon\to 0$ от $\alpha$ до $\beta$ равен $(\beta-\alpha)i \res$.
	Написали $f$ как $\frac{c-1}{z-a}+g(z)$, взяли интеграл, второй занулился (оценили как длину на максимум), первый посчитался.

	Главное значение интеграла "--- предел суммы по симметрично обрезанным отрезкам (например, вокруг нуля на $\pm\epsilon$).
	Если сходился, то главное значение ($v.p.$) совпадает, линейно, аддитивно при разбитии не особыми точками.
	Интеграл $\frac{\sin x}{x}$ от 0 до $\infty$ считать так: это половина от интеграла до двух бесконечностей,
	надо представить в виде ешки (вылезет главное значение), каждый потом по стандартному контуру из двух кругов.
	По большому будет Жордан и ноль, по маленькому "--- полувычет ($-\pi i$), ответ "--- $\frac{\pi}{2}$.

\section{} % 38
	% стр. 24 конспекта Тани
	Линейное отображение конформно, если оно сохраняет углы между прямыми и ориентацию.
	Если $L \colon \C \to \C$ линейно над $\R$, то конформно $\iff L(z)=\lambda z$ ($\lambda \in \C \setminus \{0\}$).
	Док-во: $\La$ очевидно (поворот + масштаб), $\Ra$: базис перейдёт под линейным куда-то,
	но длины могут увеличиться лишь пропорционально (т.к. в треугольнике сохраняются углы),
	итого поворот + масштаб.

	По опр. любое $f$ конформно в точке $z$, если конформно линейное $\d_z f$.
	А конформно в $\infty$, если конформна $f(1/z)$ в точке 0.
	По опр. любое $f$ "--- конформно отображение, если биекция и $f$ голоморфна.
	\TODO

\section{} % 39
	Теорема: если $f$ голоморфна, не константа, то образ "--- область.
	(условия важны: $f(z)=|z|^2$ даже вещ. дифф).
	Док-во: \TODO

\section{} % 40
	$f$ однолистна, если инъективна.
	Теорема: если голоморфная однолистна, то $f'(z)\neq 0$.
	Док-во: пусть $f'(a)=0$ и $b=f(a)$.
	\TODO


\section{} % 41
	\TODO

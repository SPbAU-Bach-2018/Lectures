\section{} % 38
	% стр. 24 конспекта Тани
	Линейное отображение конформно, если оно сохраняет углы между прямыми и ориентацию.
	Если $L \colon \C \to \C$ линейно над $\R$, то конформно $\iff L(z)=\lambda z$ ($\lambda \in \C \setminus \{0\}$).
	Док-во: $\La$ очевидно (поворот + масштаб), $\Ra$: базис перейдёт под линейным куда-то,
	но длины могут увеличиться лишь пропорционально (т.к. в треугольнике сохраняются углы),
	итого поворот + масштаб.

	По опр. любое $f$ конформно в точке $z$, если конформно линейное $\d_z f$.
	А конформно в $\infty$, если конформна $f(1/z)$ в точке 0.
	По опр. любое $f$ "--- конформно отображение, если биекция и $f$ голоморфна.

\section{} % 39
	Теорема: если $f$ голоморфна, не константа, то образ "--- область.
	(условия важны: $f(z)=|z|^2$ даже вещ. дифф).
	Док-во: \TODO

\section{} % 40
	$f$ однолистна, если инъективна.
	Теорема: если голоморфная однолистна, то $f'(z)\neq 0$.
	Док-во: пусть $f'(a)=0$ и $b=f(a)$.
	Тогда у ф-ции $f(z)-b$ (однолистна) в каком-то круге есть корень кратности хотя бы два.
	\TODO

	Следствие: если разложить $f$ в Лорана в $\infty$, то если она возле неё голоморфна и однолистна,
	то $c_{-1}\neq 0$.
	Надо рассмотреть $g=1/f$, её производная в нуле "--- как раз $c_{-1}$.

	Следствие: если есть полюс в $a$ и однолистна в её окрестности, то полюс первого порядка.
	Надо рассмотреть $g=1/f$, у неё есть устранимая особенность, она голоморфна и однолистна,
	тогда полюс большого порядка быть не может (иначе производная ноль).

	А если $f'(z)\neq 0$, то есть окрестность, где однолистна, так как
	в нек. окр. отделена от нуля, а ошка сильно меньше линейного члена производной (что за рукомахательство?).
	Но вот если везде $f'(z) \neq 0$, то однолистность везде необязательна, т.к. $e^z$.

\section{} % 41
	Теорема: $\C$ и $\mathbb{D}$ не конформно эквивалентны, так как иначе есть голоморфная $f$
	ограниченная, а она константа.

	Теорема Римана: если есть две односвязные в $\bar \C$, у которых не одноточечная граница,
	то есть единственное конформное, если фиксируем образ одной точки и аргумент $f'$ в ней же.
	Существование не доказываем.
	Единственность: если обе области есть $\mathbb{D}$ и $z=z_0=\alpha_0=0$, то
	по Шварцу $|f(z)| \le |z|$, для обратного отображения это тоже так,
	значит, $f$ "--- это поворот, но тогда $f(z)=z$.
	Если же это странные области, то сначала переведём первую в $\mathbb{D}$ так, чтобы $\phi'(a) > 0$,
	а вторую "--- из $\mathbb{D}$ так, чтобы $\arg \psi'(z_0')=-\alpha_0$.
	Получим два отображения $g_i$ из круга в круг, \TODO, каждое есть $z$ плюс однолистное, т.е. $f_1=f_2$.

	Следствие (мощный Лиувилль): если $f$ не принимает значений на некоторой кривой, то константа.
    Т.к. если не принимает, то есть конформное из обл. значений в круг, а тогда их композиция
    голоморфна на всём $\bar C$, тогда константа.
    Итого бывает три типа областей: $\bar \C$, без точки, без хотя бы двух точек.

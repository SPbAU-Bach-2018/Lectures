\section{} % 05
	% стр. 7
	$f$ из открытой связной $\Omega$ в $\C$ голоморфна в $z_0$, если есть предел-как-в-производной, причём $\in C$.
	Равносильное условие через $+o(|z-z_0|)$.
	Если записать $f$ как ф-цию $\Omega\to\R^2$, то её дифференциал "--- это почти условие на голоморфность,
	только матрица Якоби должна быть вида $\begin{pmatrix}a&-b\\b&a\end{pmatrix}$ (умножение комплексных чисел "--- умножение матриц).
	Итого $f=g+ih$ голоморфна $\iff (g'_x=h'_y \land g'_y+h'_x=0)$.
	Можно переписать через комплексные: $f'_x = -if'_y$.
	Можно выразить полные дифференциалы $\d f$, $\d z$, $\d \bar z$, отсюда
	$f'_x + if'_y=2f'_{\bar z}$, что д.б. нулём для голоморфных.

	% стр. 8
	Следствие: если у голоморфной $\Re f=const$, то $f=const$, т.к. производные $h$ равны нулю.
	Следствие: если $g$/$h$ дважды дифференцируемы, то $\partdd{g}{x}+\partdd{g}{y}=0$ (аналогично с $h$,
	каждая из них является \textit{гармонической}), т.к. расписали и заменили внутреннюю производную.
	Замечание: если есть одна гармоническая $g$, то есть единственная ($+C$) гармоническая $h$, образующая голоморфную.
	Такие называют гармонически сопряжёнными.

\section{} % 06
	Коши о дифф. форме: если $f$ голоморфна, то $f \d z$ замкнутая (что следует из определения) и локально точная.
	Для второго хватит показать, что в нек. окр. точки интеграл по любому контуру-прямоугольнику ноль.
	Если частные производные $f$ непрерывны, то можно Грином и успех (соотв. разности нули по голоморфности).

	Теперь без непрерывности: обозначим интеграл по прямоугольнику $P$ за $\alpha(P)$,
	если есть какой-то ненулевой, то разрежем его на четыре куска, выберем тот, в котором хотя бы четверть значения,
	повторим, получим цепочку, выберем общую точку.
	Распишем в ней $f$ через ошку, распишем так же интеграл, первые два слагаемых нули, остался интеграл ошки.
	Ошку заменим на $o(1)$, оценим $|z-z_0|$ как периметр мелкого прямоугольника,
	итого интеграл по мелкому не больше $\frac{perimeter^2}{4^n}\cdot o(1)$, суммируем, получаем ноль.

	Замечание: интеграл по $\partial G$ от $f(z)$ равен $2\pi\int_G\partd{f}{\bar z}\d x\d y$,
	так как можно расписать частную производную.

\section{} % 07
	% стр. 9
	Следствие: если $f$ голоморфна, то интеграл по стягиваемой ноль.
	% стр. 10
	Следствие: у голоморфной есть локальная первообразная, тоже голоморфная.

	Теорема: если функция непрерывна в $\Omega$, а в $\Omega\setminus\Delta$ (без куска прямой)
	голоморфна, то форма всё равно локально точна.
	Док-во: тоже режем на прямоугольники, если не задел "--- ок, если задел "--- порезали
	его этой прямой, отступили на $\varepsilon$, посчитали разницу, боковые фрагменты просто короткие,
	а низ и верх малы по непрерывности.

	Следствие: если голоморфна в $\Omega\setminus S$ ($S$ не имеет предельной точки), то форма тоже локально точна.
	Док-во: вокруг каждой точки можно либо найти невырезанную окрестность, либо выколоть сразу диаметр.


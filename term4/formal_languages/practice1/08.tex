\chapter{Занятие 08.04.2016}
\section{Разбор задач}

\problem{40}
Разбирал Юра Ребрик.

Используем лемму о накачке. Зафиксировали $n$, взяли слово $0^{4n}10^n10^{4n}$.

Посмотрим, куда попадет середина $vxy$ (в обозначениях леммы) для произвольного разбиения.

Разберем два случая
\begin{itemize}
\item
В $vy$ нет единички.
Тогда $vxy$ лежит либо слева/справа от единичек, либо по центру.
В первом случае, если мы много раз продублируем ее, то эти две единички сдвинутся либо слишком влево, либо слишком вправо.
Пусть, мы продублировали правую часть и единички сдвинулись влево. $|x|$ ограничен сверху числом $5n$, значит, и $z$ тоже. 
В какой-то момент у нас ноликов справа от последней единички будет хотя бы $100n$, поскольку $|z| \leq 4n$, то $|y| \geq 96n$, а должно выполняться неравенство $|y| \leq |x|$.

Аналогично в случае, если мы продублировали левую часть.

Если мы дублируем нолики между единичками, то $y$ разрастается, в то время как ограничения на длину $|x|, |z|$ не меняются.

\item
В $vy$ хотя бы одна единичка, удалим, получим не более одной единицы в слове, оно не из языка.
\end{itemize}

\subproblem{41}{a}
Разбирал Всеволод Степанов.

Мы умеем с помощью обычных КС-грамматик принимать язык $\{0^n1^n2^k\}$ и язык $\{0^k1^n2^n\}$.

Например, для первого языка можно записать такие продукции: 
\begin{align*}
S &\to AB \\
A &\to 0A1 \mid \epsilon \\ 
B &\to B2 \mid \epsilon
\end{align*}.

А конъюктивные грамматики, по сути, умеют брать пересечение языков. Если мы пересечем $\{0^n1^n2^k\}$ и $\{0^k1^n2^n\}$, то получим в точности $\{0^n1^n2^n\}$.

\subproblem{41}{b}
Разбирала Лиза Третьякова.

\begin{align*}
S &\to (SAb \& Cb) \mid b \\
C &\to bCaa \mid aC \mid baa \\
A &\to aA \mid \epsilon
\end{align*}

Разбираем значение нетерминалов.

$A$ "--- просто несколько (возможно, ноль) букв $a$.

$C$ "--- слова вида $(ba^*)^{n - 1} b a^{2n} b$: когда мы добавляем очередную группу (то есть $b$), то мы добавляем и два символа $a$ вправо, к тому, что будет последней группой.

Теперь смотрим на $S$. Это либо $b$ (крайний случай), либо надо смотреть на конъюнкцию.
Она означает то, что $S$ обязано раскрыться в несколько групп, при этом последняя группа имеет нужную нам длину (за это отвечает $Cb$), 
а еще мы можем удалить последнуюю группу и применить это же правило для остатка ($SAb$). 
То есть, с помощью $Cb$ мы зафиксировали количество групп и размер последней, теперь удаляем последнюю группу, говорим, что новая последняя тоже имеет нужную длину (раскроем в $Cb$), и так далее.

\subproblem{41}{c}
Разбирал Дима Лапшин.

\begin{align*}
S &\to (a^*b)^* \& C_1 \& C_2 \\
C_1 &\to ab \mid aC_1 a^* b \\
C_2 &\to a^*b \mid (C_3b C_4 \& (a^*b)C_2)\\
C_3 &\to b | a^n b a^n \\
C_4 &\to \{a, b\}^*
\end{align*}

$C_1$ говорит, что количество $a$ в первой группе совпадает с количеством групп

$C_2$ говорит, что или есть одна группа $a^*b$, либо есть несколько групп, при этом первые две равны, если откусить первую, то там то же самое. То есть, что все группы равны между собой.

А $S$ говорит, что группы обладают одновременно свойствами и $C_1$ и $C_2$, то есть, ровно те слова, что нам нужны.
\subproblem{41}{d}
Разбирала Надя Бугакова.

\begin{align*}
S &\to a^*b \mid Sa^*b \& a^*bS \& T \\ 
T &\to T_1 \mid T_2 \\
T_1 &\to a^+T'b \\
T' &\to aT'a \mid bS \\
T_2 &\to T'a^+b
\end{align*}

Видим, что $S$ задает некоторое количество групп $a^*b$, при этом для них должно выполняться какое-то свойство $T$.

$T$ же, в свою очередь, говорит, что количество букв $a$ в первой группе и в последней разное. 
Например, если он раскрылся в $T_1$, то мы сначала сказали, что в первой группе больше нуля букв $a$, в последней их 0, после чего раскрыли $T'$.
А $T'$ добавляет одинаковое количество букв к первой и последней группе, после чего раскрывается в $S$, то есть, в несколько групп с каким-то свойством.
Получили, что первая группа больше последней.

Аналогично, $T_2$ скажет, что последняя группа больше первой.

Теперь еще раз посмотрим на $S$, воспользуясь тем, что знаем про $T$.
$S$ говорит, что у нас первая и последняя группа различны ($T$), а еще если откусить первую/последнюю группу, то получим то же самое. 
Для любой пары индексов групп можно откусить что-то с начала и что-то с конца, получить, что эти группы будут первой и последней в слове, тогда они различны.

Таким образом, $S$ переходит только в слова, у которых все группы различны.
Несложно понять, что любое такое слово можно получить.

\subproblem{41}{e}
Разбирала Лиза Третьякова.

\begin{align*}
S &\to C \& D \\
C &\to XCX \mid c \\
D &\to (aA \& aD) \mid (bB \& bD) \mid cE \\
E &\to XE \mid \epsilon \\
A &\to XAX \mid cEa \\
B &\to XBX \mid cEb \\ 
X &\to a \mid b
\end{align*}

$E$ "--- просто слово $\{a, b\}^*$

$A$ в итоге обязательно раскрывается в $xcyaz, x,y,z \in \{a, b\}^*$.
Аналогично для $B$.

$C$ это слова вида $xcy$, где $|x| = |y|, x, y \in \{a, b\}^*$.

Теперь понимаем, как выглядит $D$. 
Пусть раскрыли его в $aA$, тогда в итоге получим $a\{a, b\}^n cEa \{a, b\}^n$.

С другой стороны, мы тогда также можем раскрыть и в $aD$. 
Пусть мы потом сказали, что раскрываем $D$ в $bB$.
Тогда в итоге получим $b\{a, b\}^m cEb \{a, b\}^m$. 
Теперь заметим, что у нас будет всего одна $c$, поэтому $m = n - 1$.
Тогда мы исходный $D$ раскрыли в $ab\{a, b\}^{n-1} cEab \{a, b\}^{n - 1}$.

Несложно заметить, что в итоге $D$ раскрывается в $ucvu$.
А так как $S$ должно раскрыться в $ucvu$, при этом, $|u| = |vu|$ (свойство $C$), то $v = \epsilon$, $S = ucu$.

\subproblem{41}{f}
Разбирал Игорь Лабутин.

$A \to wcwc \{a, b\}^* \& \{a, b\}^*cwcw$. 
Первое слово совпадает со вторым, а второе совпадает с третьим.

\subproblem{41}{g}
Разбирала Оля Черникова.

Есть три условия:
\begin{enumerate}
    \item
        Длина первой части в два раза больше длины второй.
        \begin{align*}
        S &\to XXSX \mid c \\
        X &\to a \mid b
        \end{align*}

    \item
        Слово имеет вид $wxcw$. Делаем как в пункте е, только теперь $S \to D$, $D \to \dots \mid Ec$ "--- убрали ограничение на равные длины слева и справа от $c$, мусор теперь будет слева от $c$.
    \item
        Слово имеет вид $xwcw$. Слово это что-то + слово из пункта Е
    
\end{enumerate}

\subproblem{41}{h}
\TODO

\problem{36}
Разбирал Дима Розплохас.

$L = a^m b a^n b a^n b a^* b a^m b a^* b$. Это КС-язык. Докажем, что его корень таковым не является.

Его корень это в точности слова $a^n b a^n b a^n b$ (у корня должно быть ровно три буквы $b$, на конце при этом не должно быть буквы $a$, еще и получится, что $n=m$).
А для такого языка очевидно, что он не является КС. Можно, например, воспользоваться для этого леммой о накачке, вроде такое уже делали.

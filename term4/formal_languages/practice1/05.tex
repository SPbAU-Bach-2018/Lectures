\chapter{Занятие 18.03.2016}
\section{Разбор задач}

\TODO пункты а и б задачи 22 считаются очевидными

\subsection{Задача 22c}
	Разбирала Надя Бугакова.
	\TODO решение как у меня (находим первую точку с балансом ноль).

\subsection{Задача 22d}
	Разбирал Егор Суворов.
	\TODO

\subsection{Задача 22e}
	Разбирал Егор Суворов.
	\TODO

\subsection{Задача 22f}
	\subsubsection{Неверное решение}
		Разбирала Надя Бугакова и завалилась.

		Возьмём строку $S$ из языка длины хотя бы 5.
		\begin{lemma}
			В $S$ обязательно где-то встречается подстрока \t{bb}.
		\end{lemma}
		\begin{proof}
			От противного: пусть нет подстроки \t{bb}.
		\end{proof}

		По лемме запишем строку как $X\t{bb}Y$.
		Покажем, что либо в $X\t{bb}$, либо в $\t{bb}Y$ есть подстрока из нашего языка
		(в которой \t{a} в два раза меньше, чем \t{b}).
		В этом месте захотелось сказать, что после того, как мы эту подстоку удалим,
		у нас слева и справа останутся тоже строки языка.
		Но это неправда: \t{a\textbf{abb}bb}.

	\subsubsection{Верное решение}
		Разбирал Никита Подгузов.

		Обозначим язык из задачи за $S \coloneq \{ |w|_a - 2|w|_b = 0 \}$.
		Теперь введём два новых языка:
		\begin{itemize}
			\item $T \coloneq \{ |w|_a - 2|w|_b = 1 \}$
			\item $R \coloneq \{ |w|_a - 2|w|_b = -1 \}$
		\end{itemize}
		Будем всё друг через друга выражать.

		\begin{Def}
			Назовём величину $|w|_a - 2|w|_b$ \textit{балансом}.
		\end{Def}
		
		\TODO записать грамматику см фотку 2
		Разбираемся с $S$:
		\begin{enumerate}
			\item
				Либо пустое слово.
			\item
				Если слово из $S$ начинается с \t{a}, то дальше должно идти что-то с балансом $-1$,
				то есть слово из $R$.
			\item
				Аналогично, если слово заканчивается на \t{a}
			\item
				Если слово и начинается, и заканчивается на \t{b}, то после первой \t{b} у нас
				баланс строго отрицателен (и равен $-2$), а перед последней "--- строго положителен
				(и равен $+2$).
				Заметим, что мы можем увеличивать баланс только на $+1$, таким образом у нас обязательно
				в середине строки найдётся место, где баланс равен нулю, там её можно распилить
				на две строки с балансом ноль.
		\end{enumerate}
		Разбираемся с $T$:
		\begin{enumerate}
			\item
				Если начинается или заканчивается с \t{a}, то остаток имеет баланс ноль, т.е. слово из $S$.
			\item
				Если начинается и заканчивается на \t{b}, то после первой \t{b} у нас
				баланс равен $-2$, а перед последней "--- $3$.
				Значит, обязательно найдём точку, где баланс ноль.
				Тогда распили на строку из $S$ и $T$.
		\end{enumerate}
		Разбираемся с $R$:
		\begin{enumerate}
			\item
				Если начинается или заканчивается на \t{b}, то баланс на оставшемся куске должен быть $+1$,
				т.е. слово из $T$.
				Если начинается или заканчивается с \t{a}, то остаток имеет баланс ноль, т.е. слово из $S$.
			\item
				Если начинается и заканчивается на \t{a}, то после первой \t{a} у нас
				баланс равен $+1$, а перед последней "--- $-2$.
				Так как прыгать вниз мы умеем максимум на $-2$, то в какой-то момент баланс будет
				либо $0$, либо $-1$.
				Тогда распилим либо на $SR$, либо на $RS$.
		\end{enumerate}

		Доказать, что все слова из языка корректны, можно от противного: найдём слово
		минимальной длины, которое в языке лежит, а не должно.
		Посмотрим, по какому правилу оно было порождено из нескольких корректных слов меньшей длины,
		получим противоречие.

\subsection{Задача 22g}
	Разбирала Надя Бугакова.
	Идея такая: сначала отдельно разбёрем случай, когда $|x|\neq |y|$.
	А теперь скажем, что $x$ и $y$ отличаются хотя бы в одной букве,
	а все остальные буквы произвольны.
	\TODO см фотку 1

\subsection{Задача 15a}
	Разбирала Оля Черникова.
	\TODO в другую сторону доказали Лиза, записать

	\TODO записать аккуратно
	Зафиксируем какой-то конкретный 2DFA.
	Давайте придумаем какую-то характеристику для префиксов слов следующего вида:
	если одно слово начинается с $S$, а другое с $T$, то мы дальше будем зависеть только от
	оставшихся суффиксов \TODO Оля еще что-то про неразличимость говорила.

	Первое, что мы добавим в наше описание "--- в каком состоянии $d$ мы оказались, если
	обработали префикс $S$ (т.е. первый момент, когда мы съели последний символ $S$).
	Потом мы пойдём как-то обрабатывать суффикс.
	Возможно, захотим вернуться в $S$.
	Давайте научимся не храня целиком $S$ понимать, что произойдёт, когда мы вернёмся.

	Характеристика строки $S$ "--- это мы для каждого состояния $p$ храним, в каком состоянии
	$p'$ мы окажемся в букве справа от $S$, если начали из последней буквы $S$ в состоянии $p$.

	Таким образом понимаем, что характеристик $S$ конечно.
	А теперь покажем, что правых контекстов конечно (следовательно, язык будет приниматься DFA).
	Правый контекст "--- это в точности характеристика.
	Доказательство: взяли две строки с одинаковой характеристикой, дописали одинаковые суффиксы,
	одинаково приняли/отвергли результат, что успех.

\subsection{Задача 15e}
	Разбирала Оля Черникова.
	\TODO дописать разбор

	\begin{Rem}
		Мы теперь знаем, что машина Тьюринга с $\O(1)$ памяти "--- это то же самое,
		что и DFA.
		Это круто.
	\end{Rem}

\subsection{Задача 25}
	Разбирал Дима Розплохас.

	Давайте построим контекстно-свободную грамматику для $SUFFIX(L)$.
	В ней будут все старые терминалы, нетерминалы и правила, но мы также
	добавим для каждого нетерминала $X$ нетерминал $X'$, соответствующий $SUFFIX(X)$
	(т.е. суффиксам всех слов, которые можно получить из нетерминала $X$).
	После этого объявим нетерминал $S'$ стартовым.

	Строим правила для $X'$.
	Пусть было правило для $X$, например: $X \to cbDeF$.
	Будем стирать по одному символу, начиная слева:
	\TODO оформить красиво
	\begin{tabular}{rcr}
		\t{X} &\to& \t{cbDeF} \\
		\t{X} &\to&  \t{bDeF} \\
		\t{X} &\to&   \t{DeF} \\
		\t{X} &\to&  \t{D'eF} \\
		\t{X} &\to&    \t{eF} \\
		\t{X} &\to&     \t{F} \\
		\t{X} &\to&    \t{F'} \\
	\end{tabular}
	Формально это доказывается через дерево разбора: посмотрим на дерево разбора для слова,
	соответствующего $SUFFIX(L)$, а его можно легко достроить до дерева разбора для слова из $L$.

\subsection{Задача 15e}
	\TODO оформить красиво
	Пусть наша машина Тьюринга будет не сразу пытаться проверять, что у нас идут подряд идущие числа,
	а по очереди увеличивать $k$ от единицы, пока получается.
	Если зафейлилось, то зафейлилось, и памяти было мало.
	Если не зафейлилось, то надо еще в конце проверить, что мусора не осталось.

\section{Замечание про задачу 15}
	Мы доказали, что $DFA=DSpace[1]$.
	Если докажем 15c, то узнаем, что всё, что строго меньше $DSpace[\log \log n]$,
	схлопывается в $DFA$.
	Пример нерегулярного из $DSpace[\log \log n]$ мы привели.
	К сожалению, что происходит с языками с памятью между $\log \log n$ и $\log n$, никто не понимает.

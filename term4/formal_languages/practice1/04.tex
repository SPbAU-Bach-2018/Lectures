\chapter{Занятие 11.03.2016}
\section{Разбор задач}

\subproblem{13}{d}
	Разбирал Дима Розплохас.

	Возьмём такой алфавит: $\Sigma=\{0,1,2\}$,
	а язык "--- такой: $L = \{(012)^{3k+1} \mid k \in \N \}$.
	Этот язык регулярный, потому что есть регулярное выражение:
	\[ \t{(012012012)^*012} \]
	При этом хочется показать, что язык $L'=L_{\frac13-\frac13}$ будет нерегулярным.
	\begin{Rem}
		$L'$ "--- это все слова из $L$, длина которых делится на три, из которых убрана средняя треть.
	\end{Rem}
	То есть слово для некоторого $k$ из $L$ (длины $9k+3$) переходит в такое слово:
	\[ \t{(012)^k02(012)^k} \]
	Его слова имеют очень простую структуру: подстрока \t{02} встречается ровно один раз,
	а слева и справа "--- одинаковые подстроки.
	Докажем, что такой язык нерегулярный леммой о накачке.
	В самом деле, пусть $L'$ распознаётся автоматом с $n$ состояниями.
	Тогда рассмотрим слово $\t{(012)^n02(012)^n} \in L'$, тогда по лемме о накачке мы можем
	получить из него новое слово, тоже лежащее в $L'$.
	В самом деле: по этой лемме среди первых $n$ символов можно найти подстроку,
	которую можно повторить сколько угодно раз.
	Повторим её один раз, тогда паттерн нарушится и наше слово в $L'$ лежать не будет,
	противоречие, стало быть, язык нерегулярен.

	\begin{Rem}
		Можно чуть упростить доказательство, взяв гомоморфизм:
		пускай $L'$ регулярный.
		\TODO
	\end{Rem}

	\begin{Rem}
		Возьмём язык из слов, в которых нет двух подряд идущих единиц.
		Он тоже проходит как пример.
		\TODO а почему?
	\end{Rem}

\problem{14}
	Разбирал Никита Подгузов.
	Положим $A = \{ 2^k \mid k \in \N \}$.
	В двоичной системе счисления, очевидно, получаем регулярный язык.
	Покажем, что в троичной это не так (леммой о накачке): положим $B_3(A) \eqcolon L$.
	Предположим, что $L$ регулярен.
	Тогда по лемме о накачке существует число $n$ и слово $w=\overline{abc} \in L$ такое, что $|a|+|b| \le n$
	и подстроку $b$ можно повторить сколько угодно раз, причём слово всё еще будет лежать в $L$.

	Давайте введём числа (тут мы за $a$ обозначаем и подстроку, и соответствующее ей
	в троичной системе число):
	\begin{align*}
		a' &\coloneq a \cdot 3^{|b|+|c|} \\
		b' &\coloneq b \cdot 3^{|c|} \\
		c' &\coloneq c \\
	\end{align*}
	По сути мы просто взяли числа, соответствующие подстрокам $a$, $b$, $c$ и дописали нули так, чтобы
	$a'+b'+c'=w=2^k$.
	Давайте теперь посмотрим, что произойдёт, если мы рассмотрим слово $w'=abbc\in L$,
	то есть сдвинем префикс $ab$ влево и допишем $b$.
	\begin{align*}
		w' &= a\cdot3^{|b|+|b|+c|} + b\cdot3^{|b|+c} + b\cdot3^{|c|} + c' =
		      a'\cdot3^{|b|} + b'\cdot3^{|b|} + b' + c' =
		      (a'+b')\cdot3^{|b|} + b' + c' = \\
		   &= (2^k-c')\cdot3^{|b|} + b' + c' =
		      2^k\cdot3^{|b|}-c'\cdot3^{|b|}+b'+c' =
		      2^k\cdot3^{|b|}+b'+c'(1-3^{|b|})
	\end{align*}
	Так как $w' \in L$, то $w' = 2^{k+x}$ (так как в $L$ лежат только степени двойки).
	Давайте разделим $w'$ на $w$, должны получить $2^x$ "--- целое число:
	\begin{align*}
		\frac{w'}{w}
			&= \frac{2^k\cdot3^{|b|}+b'+c'(1-3^{|b|})}{2^k} =
			3^{|b|}+\frac{b'+c'(1-3^{|b|})}{2^k}
	\end{align*}
	\TODO дальше проблема возникнет в каких-то остатках.

	\begin{Rem}
		От семинариста:
		можно посмотреть на строчки $\overline{ac}$, $\overline{abc}$, $\overline{abbc}$ и на соответствующие числа.
		А теперь давайте рассмотрим и посчитаем следующее выражение:
		\begin{gather*}
			\frac{\overline{abbc}-\overline{abc}}{\overline{abc}-\overline{ac}} =
			\frac{(\overline{ab}-a)\cdot3^{|b|+|c|}}{(\overline{ab}-a)\cdot3^{|c|}} =
			\frac{(\overline{ab}-a)\cdot3^{|b|}}{\overline{ab}-a} =
			3^{|b|}
		\end{gather*}
		С одной стороны это получится степень тройки, а с другой "--- что-то не являющееся степенью тройки
		\TODO
	\end{Rem}

\subproblem{15}{b}
	Разбирал Дима Лапшин.

	В одну сторону скучно: если распознаётся 2DFA, то точно распознаётся машиной Тьюринга с нулевой памятью
	(вся память "--- в состоянии), тем более распознаётся с $O(1)$ памяти.

    В другую сторону: пусть есть машина Тьюринга с $O(1)$ памяти, то есть с константой памяти (пусть $M$ ячеек).
    Всего возможных состояний памяти $2^M$.
    Давайте построим новую машину с нулевой памятью: $Q\times 2^M \times [M]$, она просто будет в состоянии
    хранить память и положение головки на рабочей ленте.
    А МТ с нулевой памятью "--- это просто 2DFA.

\problem{16}
	Разбирал Егор Суворов.

	Пусть имеется строка $S$, синхронизирующая автомат из $k$ состояний.
	Давайте построим последовательность строк $\underbrace{a_1}_{b_1}$, $\underbrace{a_1a_2}_{b_2}$, $\dots$
	со следующим свойством: строка $b_i$, если её скормить автомату, переводит состояния 1 и $i$ в одно и то же.
	\begin{Rem}
		Если это так, то строка $b_i$ переводит все состояния $1, 2, \dots, i$ в одно и то же,
		так как она содержит в качестве префиксов $b_1, b_2, \dots, b_{i-1}$.
		В частности, $b_n$ будет синхронизирующей строкой для автомата.
	\end{Rem}
	\begin{Rem}
		Если построим такую последовательность строк, причём $|a_i| \le k^2$, то решим
		задачу, так как тогда $b_n$ будет являться конкатенацией $k$ строк длины не более $k^2$ каждая,
		т.е. $|b_n| \le k^3$.
	\end{Rem}

	Давайте строить по шагам.
	Положим $a_1=\epsilon$, так как нам вообще ничего склеивать не надо.
	Теперь предположим, что мы построили $b_{i-1}$ и хотим построить $b_i$ (для $i \ge 2$).
	Возьмём состояния 1 и $i$ и посмотрим, во что они перейдут после того, как скормить автомату
	строку $b_{i-1}$.
	Пусть они перешли в состояния $x$ и $y$.
	Теперь рассмотрим строку $S$, данную нам в условии: заметим, что если положить $a_i=S$, то
	состояния $x$ и $y$ синхронизируются.
	Однако такая $a_i$ может быть слишком длинной, давайте её оптимизировать для данных двух конкретных $(x, y)$ следующим образом:
	\begin{enumerate}
		\item
			Смотрим на последовательность пар состояний, куда переходили $x$ и $y$
			в процессе обработки строки $a_i$: $(x=x_0, y=y_0), (x_1, y_1), \dots (x_{|a_i|}, y_{|a_i|})$.
			Если имеется два индекса $0 \le \alpha < \beta \le |a_i|$ такие, что
			соответствующие пары равны ($x_\alpha=x_\beta, y_\alpha=y_\beta$), то можно
			смело выкинуть из $a_i$ подстроку с символа $\alpha+1$ до символа $\beta$ (включительно),
			потому что мы просто прошли по циклу.
		\item
			Если же таких индексов не нашлось, то все пары в последовательности пар состояний различны.
			Всего возможных пар $k^2$, значит, длина строки $a_i$ не превосходит $k^2-1$, что и требовалось.
	\end{enumerate}

	\begin{Rem}
		От семинариста: в Екатеринбурге занимаются синхронизирующими автоматами.
		К сожалению, даже там никто не умеет строить строить серии автоматов,
		где для синхронизации надо строку длиннее $k^2$.
		Правда, для каких-то специфичных классов автоматов умеют доказывать, что
		можно их синхронизировать за $k^2$.
	\end{Rem}

\subproblem{17}{a}
	Разбирал Дима Лапшин.

	Пусть было два регулярных языка $L_1$ и $L_2$, распознающиеся автоматами $M_1$ и $M_2$
	с $k_1$ и $k_2$ состояниями, соответственно.
	Мы знаем, что язык $U=L_1\cup L_2$ непуст.
	Значит, один из $L_1$, $L_2$ непустой.
	Значит, в соответствующем автомате есть путь из начальной вершины в какую-нибудь терминальную.
	Найдём в нём какой-нибудь простой пусть, в нём будет не более $k_i$ состояний.
	Получили, что надо.

\subproblem{17}{b}
	Разбирала Лиза Третьякова.

	Давайте построим автомат, принимающий в точности объединение языков.
	В нём будет $k_1k_2$ состояний (состояние "--- пара старых состояний),
	переходы понятные, а терминал $\iff$ терминал хотя бы в одном из состояний.
	Так как у нас есть строка $s$, не принимаемая автоматом, то есть достижимая нетерминальная вершина.
	До неё можно найти простой путь, он будет длины не более $k_1k_2$.

\problem{18}
	Разбирала Оля Черникова.

	Пусть есть бесконечный язык $A$.
	Применим лемму о накачке: есть слово $w=abc \in A$, причём все слова $ab^ic$ тоже лежат в $A$.
	Зафиксируем это $w$ и его разбиение.
	Заметим, что язык $B$ из слов $ab^{2i}c$ бесконечный и регулярный, так как можно построить регулярное выражение
	(или понятный автомат).
	А язык $A\setminus B$ тоже регулярный (это мы знаем), и бесконечный, так как содержит слова вида $ab^{2i+1}c$.

\problem{19}
	Разбирал Дима Розплохас.

	Возьмём язык $D \coloneq B \setminus A$, он бесконечный и регулярный.
	По предыдущей задаче регулярный и бесконечный язык можно разбить на два регулярных и бесконечных языка: $D_1$ и $D_2$.
	Тогда язык $A \cup D_1$ тоже регулярный и бесконечный.
	Значит, $A \ll A \cup D_1 \ll B$.

\problem{20}
	Разбирала Лиза Третьякова.

	Сначала доказываем влево: пусть $BB\subseteq B$.
	Покажем два включения.
	Включение $B\subseteq B^{+}$ очевидно.
	Индукцией по $k$ покажем $B\supseteq B^{+}$.
	Индукционное предположение: конкатенация не более чем $k$ слов из $B$ лежит в $B$.
	База при $k=1$ очевидна.
	Переход: пусть для $k-1$ это верно.
	Тогда взяли конкатенацию из $k$ слов из $B$.
	Конкатенация первых $k-1$ лежит в $B$, то есть на самом деле мы имеем конкатенацию всего
	двух слов из $B$, а это лежит в $B$, так как $BB\subseteq B$.

	Теперь доказываем вправо: пусть $B=B^{+}$.
	Тогда, в частности, $B^{+} \subseteq B$, а $BB \subseteq B^{+}$, комбинируем, получили.

\subproblem{15}{d}
	Разбирал Никита Подгузов.

	Надо показать, что язык не распознаётся DFA, т.е. не является регулярным.
	Докажем, что множество правых контекстов бесконечное.
	Покажем, что оно больше произвольного числа $n$.
	Для этого берём число $k>n$.
	\subsubsection{Неверное решение}
		Это оставшиеся у меня с пары записи:
		рассмотрим правые контексты у строк \t{0}, \t{00}, \dots, $\t{0^k}$.
		Правый контекст у последней строки "--- это множество из одной некоторая строка $s$.
		Правые контексты у остальных строк "--- это множества размера один, из строк вида $\t{0^ks}$.
		Все эти контексты все различны.
		Итого бесконечно много правых контекстов.

	\subsubsection{Верное решение}
		Небольшая правка:
		рассмотрим правые контексты у строк \t{0\$}, \t{00\$}, \dots, $\t{0^k\$}$.
		У каждой из них правый контекст состоит из единственной строки, причём все
		эти строки различной длины.
		Итого бесконечно много правых контекстов.

\subproblem{15}{e}[подсказка]
	Запустите свою машину Тьюринга на строке $\t{\underbrace{00\dots00}_k\$}$ и убедитесь, что она её не принимает,
	причём не тратит много памяти.
	Например, посчитать длину первого блока нулей мы не можем "--- нам надо $\O(\log k)=\O(\log n)$ памяти,
	хотя в обычном случае у нас бы была строка длины $2^k$ и у нас бы получилось $\O(\log \log n)$ памяти.
	В частности, решения Оли и Никиты это завалило.

\section{Организационное}
	\TODO уточнить у Коли
	Поправка к правилам: люди вызываются в каком-то порядке из числа решивших задачу.
	Скорее всего, от меньшего числа попыток к большему.
	Если человек вызван, его нет, а потом задачу разобрали, человеку ставится минус (в ущерб рейтингу).
	Если же задачу не разобрали (например, проспали все решившие), задача снимается (без ущерба для рейтинга).

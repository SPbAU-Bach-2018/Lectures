\setauthor{Всеволод Степанов}

\chapter{Формальные грамматики}
\section{Контекстно-свободные грамматики (context-free)}
Всё, что вы сейчас увидите "--- это то, как люди пытались описать синтаксис
построения предложения в естественный языках.
Относительно получается: какие-то стандартные невычурные фразы
(особенно в английском языке, где фиксирован порядок слов) даже получается
описывать, однако кажется, что естественный язык намного богаче.
Зато оказалось, что грамматиками можно задавать различные программистские
вещи, в том числе языки программирования, регулярные выражения и так далее.

\begin{Rem}
    Формальные грамматики "--- formal grammar.
\end{Rem}

\begin{Def}
    КС-грамматика (контекстно-свободная грамматика, context-free grammar, cf-grammar)
    $G$ "--- это четвёрка $(\Sigma, N, P, S)$ такая, что:
    \begin{itemize}
        \item
            $\Sigma$ "--- конечный алфавит, его элементы мы назваем \textit{терминалами}.
        \item
            $N$ "--- некоторое конечное множество, причём $\Sigma \cap N = \varnothing$.
            Его элементы мы называем \textit{нетерминалами}/\textit{переменными}/\textit{синтаксическими категориями}
            (это то, как его называли еще лингвисты: существительное, глагол, дополнение, предложение).
        \item
            $S \in N$ "--- \textit{стартовый символ}/\textit{стартовая переменная}.
        \item
            $P$ "--- конечное множество продукций вида $A \to \alpha$, $A \in N$, $\alpha \in (N \cup \Sigma)^*$.
            
            $P$ --- конечное подмножество $2^{N \times (N \cup \Sigma)^*}$.
    \end{itemize}
\end{Def}
\begin{Rem}
    Иногда будем употреблять слово \textit{символ} для элементов $\Sigma \cup N$.
\end{Rem}
\begin{Rem}
Контекстно-свободная в том смысле, что слева в продукциях стоит только один символ. 
Например, правила $aAa \Ra aABa$ недопустимы, а они, в каком-то смысле, зависят от контекста: $A$ переходит в $AB$ только если его окружают $a$.
\end{Rem}
\begin{Rem}
    Довольно бессмысленно рассматривать бесконечные системы: наверняка бесконечной системой
    описать бесконечный язык несложно, а вот если можно описать конечной системой "--- это хорошо.
\end{Rem}
\begin{Rem}
    Часто есть договоренность, заглавнеы буквы "--- нетерминалы, при этом, $S$ "--- стартовый.
\end{Rem}
\begin{Rem}
    $P$ можно рассматривать, например, вот так: есть синтаксическая категория, например, <арифметическое выражение>. 
    Мы его можем раскрыть как <число>, например, $239$ "--- корректное арифметическое выражение, а можем, например, как <арифметическое выражение> <плюс> <арифметическое выражение>. 
    Например, $239 + 30 + 179$ будет принадлежать такой грамматике.
\end{Rem}
\begin{Rem}
   Кстати, с контестно-свободными грамматиками мы должны быть знакомы, например, по форме Бэкуса-Наура (она же БНФ). %%вставить какой-то конкретный пример (например, if)
\end{Rem}

Сейчас у нас есть операция преобразования одного нетерминала в какую-то последовательность терминалов и нетерминалов. 
Но с последовательностями терминалов и нетерминалов мы пока ничего не умеем делать, надо все это дело формально определить.

\begin{Def}        
    $\alpha B \gamma$, $\alpha, \gamma \in (\Sigma \cup N)^*, B \in N$.

    Говорят, что $\alpha B \gamma \Ra \alpha \beta \gamma$, если в $P$ есть продукция $B \to \beta$, $\beta \in (\Sigma \cup N)^*$. 
    Читается ``$\alpha B \gamma$ выводит $\alpha \beta \gamma$''
\end{Def}

\begin{Def}
    Если при этом $\alpha \in \Sigma^*$, то пишут
    $\alpha B \gamma \xLongrightarrow[lm]{} \alpha \beta \gamma$ (левосторонне выводит)

    Аналогично, если $\gamma \in \Sigma^*$, то
    $\alpha B \gamma \xLongrightarrow[rm]{} \alpha \beta \gamma$ (правосторонне выводит)
\end{Def}

\begin{Def}
$s \xLongrightarrow{*} t$, если $\exists s_0\dots s_n \colon s_0 = s, s_n = t, \forall i \in [0, n] s_i \Ra s_{i+1}$.

Аналогично определяются $s \xLongrightarrow[lm]{*}$ и $s \xLongrightarrow[rm]{*}$
\end{Def}

\begin{Def}
$L(G) = \{w \in \Sigma^*: s \xLongrightarrow{*} w\}$
\end{Def}

\begin{exmp}
$\Sigma = \{0, 1\}, N = \{S\} \\
P = \{S \to 0S0, S \to 1S1, S \to 0, S \to 1, S \to \epsilon\}$

Еще одна договоренность: переходы, описанные выше, можно записать как $S \to 0S0|1S1|0|1|\epsilon$.

$S \Ra 1S1 \Ra 10S01 \Ra 10101 \in \Sigma^*$

$L(G)$ "--- язык палиндромов над $\Sigma = \{0, 1\}$.

формальными грамматиками, как видно, неплохо описываются всякие штуки, которые можно определить рекурсивно.
\end{exmp}
\begin{Rem}
Палиндромы не являются регулярным языком по лемме о накачке.
\end{Rem}

\begin{Def}
Контекстно-свободным языком называется язык $L$ такой, что $\exists$ КС-грамматика $G\colon L(G) = L$
\end{Def}

\begin{theorem}
Регулярные языки $\subset$ контекстно-свободные языки.
\end{theorem}
\begin{proof}
Рассмотрим регулярный язык $L$ и ДКА $A \colon L(A) = L$

$A = (\Sigma, Q, q_0, T, \delta)$.

$G = (\Sigma, Q, q_0, P)$

Пусть были в состоянии $A$ и перешли по символу $с$ в состояние $B$. 
Тогда добавим продукцию $A \Ra сB$.

Теперь еще надо нетерминал иногда убивать.
Для этого добавим еще продукции $A \Ra \epsilon \forall A \in T$.

Докажем, что это работает. 
Мы получаем терминал тогда и только тогда, когда у нас было слово из нескольких терминалов слева и символа, соответствующего терминальному состоянию справа.
Это значит, что в ДКА мы прошли путь из стартового состояния в терминальное по полученному слову, значит, это слово принималось этим ДКА.
\end{proof}
\begin{Rem}
Из доказательства, в частности, следует, чтобы получить грамматики, равные регулярным языкам, надо сказать, что продукции могут в правой части иметь только выражения вида $aA|\epsilon$
\end{Rem}

\begin{exmp}
Вспомним пример с палиндромами.
$S \Ra 1S1 \Ra 10S01 \Ra 10101$.

Можно нарисовать дерево разбора этой штуки: 
\begin{dot2tex}[tikz,scale=.55,options=-t math]
digraph G {
    ranksep="0.0";
    mindist="0";
    S1[label="S"];
    11[label="1"];
    S2[label="S"];
    12[label="1"];
    01[label="0"];
    S3[label="S"];
    02[label="0"];
    13[label="1"];                                
    {rank=same; 11; S2; 12}
    {rank=same; 01; S3; 02}
    S1 -> 11;
    S1 -> S2;
    S1 -> 12;
    S2 -> 01;
    S2 -> S3;
    S2 -> 02;
    S3 -> 13;
}
\end{dot2tex}

Для того, чтобы в результате разбора были только нетерминалы, необходимо и достаточно чтобы ими были все листья (крона дерева)
\end{exmp}
\begin{Def}
Крона дерева с корнем в $v$ = 
\begin{itemize}
\item $w$, если $v$ это лист с пометкой $w$.
\item конкатенация крон $v_1 \dots v_n$, если $v$ "--- не лист, есть дети $v_1 \dots v_n$.
\end{itemize}
\end{Def}

\begin{exmp}
Научимся парсить арифметические выражения, единиц и знаков <<плюс>>

$S \Ra 1|S+S$. Например, $S \Ra S + S \Ra S + S + S \Ra 1 + S + S \Ra 1 + S + 1 \Ra 1 + 1 + 1$. %TODO: дерево

Заметим, что то же самое можно было получить другим порядком применения продукций. Например, $S \Ra S + S \Ra S + S + S \Ra 1 + S + S \Ra 1 + 1 + S \Ra 1 + 1 + 1$ %TODO: дерево
\end{exmp}

\begin{Def}
КС-грамматика $G$ неоднозначная, если $\exists w \in L(G) \colon \exists$ 2 различных левосторонних вывода (дерева разбора).
\end{Def}
\begin{exmp}
<<x * y>> в языке C может означать и как <<бинарный оператор умножения, примененный к х и у>> и как <<определение указателя с названием у на объект типа х>>.

Предыдущую грамматику можно поправить: $S \Ra 1|1+S$. Заодно сразу получим правую ассоциативность плюсика (до этого ее не было).
\end{exmp}                                        

\setauthor{Всеволод Степанов}

\section{Иерархия формальных грамматик Хомского}

\begin{enumerate}
\setcounter{enumi}{-1}

\item
Неограниченныые грамматики

\item
Контекстно-зависимые

\item
Контекстно-свободные

\item
Автоматные грамматики (регулярные)

\end{enumerate}

Вложенность снизу вверх (нулевой класс самый большой).

Этим грамматикам соответствуют следующие вычислители:
\begin{enumerate}
\setcounter{enumi}{-1}
\item
Машина Тьюринга

\item
Линейно ограниченная машина Тьюринга (памяти столько же, сколько длина входа)

\item
МП-автоматы

\item
ДКА

\end{enumerate}

\begin{Def}
Регулярные грамматики "--- леволинейные грамматики $\cup$ праволинейные грамматики.

Леволинейные "--- только продукции вида $A \to a, A \to Ba$.
Праволинейные "--- $A \to a, A \to aB$.

(А еще если хотим, чтобы принималось пустое слово, сделаем, как делали в нормальной форме Хомского, продукцию $S \to \epsilon | S'$.

\end{Def}
\begin{Rem}
Если разрешить и леволинейные и праволинейные одновременно, то получим что-то между КС-грамматиками и регулярными.
\end{Rem}

\begin{theorem}
Языки регулярных грамматик равны регулярным языкам
\end{theorem}
\begin{proof}
\begin{description}
\item[ДКА $A$ $\to$ регулярная праволинейная грамматика $G$]

$N = Q$
$S = q_0$
Для перехода из $p$ в $q$ по символу $a$ добавим продукцию $p \to aq$

Если $q_0$ "--- терминальное, то создаем стартовый символ $S$, из него делаем продукцию $S \to \epsilon | q_0$

Если пришли из состояния $p$ в состояние $t \in T$ по символу $a$, то делаем продукцию $p \to a$

$\hat \delta(q_0, w) = p \Lra $ в $G\colon S \xLongrightarrow{*} wP, w \notin L(a), s \xLongrightarrow{*} w, w \in L(A)$
%TODO: леволинейные грамматики, сказать про них
\item[Регулярная грамматика $G$ $\to$ НКА $A$]

НУО, будем считать, что грамматика леволинейная. Для праволинейной будет все то же самое.

$Q = N \cup \{Z\}$, $Z$ "--- специальное состояние, означающее, что слово принято, что больше нет нетерминалов.
$q_0 = S$
$T = \{Z\}$
Переходы: был переход $A \to aB$, в автомате рисуем переход из состояния $A$ в состояние $B$ по букве $a$. 
А для перехода $A \to a$ делаем переход из $A$ по букве $a$ в $Z$.

$\hat\delta(q_0, w) = p \Ra $ в $G\colon S \xLongrightarrow{*} wP, p \neq Z, S \xLongrightarrow{*} w, p = Z$. 
доказываем простой индукцией.
\end{description}
\end{proof}

\begin{Def}
Контекстно-зависимые грамматики (Context-sensitive)

Определение почти такое же, как у КС-грамматик, но только другие ограничения на продукции:
$\alpha B \gamma \to \alpha \beta \gamma, \alpha, \gamma \in (\Sigma \cup N)^*, B \in N, \beta \in (\Sigma \cup N)^+$. 

Тут мы опять не можем получить пустое слово (так как $\beta$ содержит хотя бы один символ), можно добавить как обычно новое стартовое состояние и продукцию из него в $\epsilon$.
\end{Def}
\begin{Rem}
Контекстно-зависимые в смысле, что нетерминал можно преобразовывать только в случае, если он есть в определенном контексте "--- слева и справа от него стоит то, что надо.
\end{Rem}
\begin{Def}
Неукорачивающие грамматики "--- если была продукция $\alpha \to \beta$, то $|\alpha| \leq |\beta|$
\end{Def}
\begin{theorem}
Языки КЗ-грамматик равны языкам неукорачивающих грамматик
\end{theorem}
\begin{proof}
\begin{description}
\item[КЗ $\to$ неукорачивающие]. 

Тривиально, так как все продукции в КЗ неукорачиващие.

\item[неукорачивающие $\to$ КЗ].

Для каждого терминала $a$ создадим новый соответствующий нетерминал $A$, во всех продукциях заменяем все вхождения $a$ на $A$, добавляем продукцию $A \to a$.

Для продукции $x_1 \dots x_n \to y_1 \dots y_m, m \geq n, x_i, y_i \in (N \cup \Sigma)$ создадим уникальные для этой продукции нетерминалы $Z_1 \dots Z_n$, добавим следующие продукции:
$\\
X_1 X_2 \dots X_n \to Z_1 X_2 \dots X_n \\
Z_1 X_2 \dots X_n \to Z_1 Z_2 x_3 \dots X_n \\
Z_1 \dots Z_{n - 1} X_n \to Z_1 \dots Z_n \\
Z_1 \dots Z_n \to Y_1 Z_2 \dots Z_n \\
\dots \\
Y_1 \dots Y_{n - 1} Z_n \to Y_1 \dots Y_{n - 1} Y_n \dots Y_m$

После применения первой продукции единственный способ избавиться от $Z_i$, в силу их уникальности, только применить все остальные продукции, 
тем самым переведя $X_1 \dots X_n$ в $Y_1 \dots Y_m$
\end{description}
\end{proof}
\begin{Rem}
Без первого шага сведение некорректное: если все $x_i$ "--- терминалы, то продукция $x_1 x_2 \dots x_n \to Z_1 x_2 \dots x_n$ не является корректной продукцией для КЗ-грамматики.
\end{Rem}

\begin{Def}
Нормальная форма Куроды (для КЗ-грамматик)

Грамматика $G$ в нормальной формы Куроды, если все продукции имеют следующий вид: $AB \to CD, A \to CD, A \to a$. 
Иногда разрешают еще $A \to B$, но от этого можно избавиться (удаление цепных продукций как в нормальной форме Хомского)
\end{Def}
\begin{Rem}
Важный факт на будущее: можно разрешить продукции $A \to \epsilon$, получим нормальную форму Куроды для произвольных грамматик.
\end{Rem}
\begin{Rem}
Есть небольшая путаница в обозначениях, будьте внимательны. Лучше уточнять, какая именно форма имеется в виду.
\end{Rem}
\begin{theorem}
Любая КЗ-грамматика $G$ представима в виде НФ Куроды
\end{theorem}
\begin{proof}
Будем разрешать продукции $A \to B$.
\begin{enumerate}
\item
Из правых частей исключим терминалы.
Для любого символа $a \in \Sigma$ создадим нетерминал $A$ и продукцию $A \to a$
Во всех продукциях $A \to \alpha$, кроме созданных, в $\alpha$ заменяем любой терминал на соответствующий нетерминал

\item
Остались только продукции $A \to a$ и $\alpha \to \beta \in (Nwedge+), |\alpha| \leq |\beta|$. 

Продукции $A \to a$ не трогаем, они и так подходят под нормальную форму.
Рассмотрим остальные продукции $\alpha \to \beta$. Разберем случаи
\begin{itemize}
\item
$|\beta| \leq 2$. Уже в нормальной форме Куроды (с цепными продукциями)
\item
$|\beta| \geq 3, |\alpha| = 1$. Тогда это продукция $A \to B_1 \dots B_n$.
Создадим для этой продукции уникальные нетерминалы $A_1 \dots A_{n - 2}$, заменим исходную продукцию на следующие: $\\
A \to B_1 A_1 \\
A_1 \to B_2 A_2 \\ 
\dots \\
A_{n - 2} \to B_{n - 1} B_n
$
\item
$|\beta| \geq 3, |\alpha| \geq 2$. Эта продукция имеет вид $A_1A_2 \dots A_n \to B_1B_2B_3 \dots B_m$.
Запишем $\\
A_1 A_2 \to B_1 C \\
C A_3 \dots A_n \to B_2 \dots B_n$.
Второе правило можно раскрывать далее либо по второму, либо по третьему пункту (длина и левой и правой части уменьшилась на 1).
\end{itemize}
\end{enumerate}
\end{proof}

\begin{Def}
Произвольная грамматика "--- грамматика, у которой единственное ограничение на продукции "--- слева стоит хотя бы один нетерминал.
\end{Def}
\begin{theorem}
Любая формальная грамматика представима в виде НФ Куроды для произвольной грамматики
\end{theorem}
\begin{proof}
Пропало правило про то, что продукции могли быть неукорачивающими.

Дана произвольная грамматика $G$. Добавим нетерминал $Z$ и единственную продукицю $Z \to \epsilon$.
Для любой продукции $\alpha \to \beta, |\alpha| = n, |\beta| = m$. В случае, когда $n > m$ заменим ее на $\alpha \to \beta Z \dots Z$ (повторим $Z$ $n - m$ раз)
\end{proof}

\begin{theorem}
Языки неограниченных грамматик совпадают со всеми перечислимыми языками.
\end{theorem}
\begin{proof}
Не будет строгого доказательства, все <<на пальцах>>
\begin{description}
\item[Произвольная грамматика $G \to$ НМТ $T$]
Процесс такой: на ленте записана текущая строка из терминалов и нетерминалов. Мы можем выбрать какую-то подстроку, применить к ней какую-то продукцию.
Для этого надо проверить, что подстрока соответствует левой частми продукции, после чего сделать замену, подвинув правую часть ленты вправо/влево, если правая часть продукции короче/длиннее.

Поскольку машина будет недетерминированной, можно будет применить нужную продукцию, чтобы получить в итоге нужное слово.

\item[МТ $\to$ произвольная грамматика]
Пускай есть МТ, принимающая на вход слово, и говорящая YES/NO. 

Есть конфигурация МТ: слово на ленте, положение головки, текущее состояние. 
Храним мы ее так: у нас в каждый момент времени записана какая-то строка из нетерминалов, каждый нетерминал это либо пара (символ на ленте, -1), либо (символ на ленте, состояние).
Пара второго типа будет встречаться в единственном экземпляре и будет означать, что головка сейчас находится над данным символом, при этом МТ имеет ровно такое вот состояние.

Соответственно, переходы у МТ описываются легко: мы просто берем, и делаем продукции из пары (символ на ленте, состояние) плюс еще по символу справа/слева в новую конфигурацию этих трех символов "---
возможно, подвинулась головка, то есть, пара второго типа оказалась правее/левее, еще могло измениться состояние и измениться символ под головкой.

Научились МТ эмулировать, хотим, чтобы, если мы пришли в YES, то в итоге было выписано изначальное слово, которое было записано на ленте.
                                                
Построим грамматику для следующего языка: $\{\wedge w\#w\}$, вот она:
$\\
S \to \wedge S_1\# \\
S_1\# \to AS_1\#a \mid BS_1\#b \mid \epsilon \\
0A \to A0 \\
1A \to A1 \\
0B \to B0 \\
1B \to B1 \\
\wedge A \to \wedge a \\
\wedge B \to \wedge b$.

Почему работает: берем, генерим какой-то нетерминал, во вторую копию $w$ в начало пихаем соответствующую букву, а сам нетерминал мы умеем двигать влево. 
При этом, мы не умеем менять местами два нетерминала, а превратить его в терминал сможем только тогда, когда достигнем начала слова.

Вернемся к МТ. Возьмем, изначальную конфигурацию продублируем, получим $\wedge w\#w$.
Над исходной строкой (первая копия), описывающей состояние, проведем эмуляцию МТ, в случае, если пришли в YES, затрем все что слева от \#, получим исходное состояние (справа от \# ничего не трогали),
переведем его в строчку из терминалов.
Если перешли в NO, то ничего не делаем, вывод не заканчиваем и все хорошо.
Затирать можно, например, так
$\\            
\gamma A\# \to A\#, \gamma \in (N \cup \Sigma) \\
\wedge \# \to \epsilon$

Хотим показать, что для МТ есть грамматика, которая порождает ровно тот же язык, который принимает МТ. 
Для этого пишем сначала грамматику, генерирующую удвоенные слова (только терминалы заменим на нетерминалы, чтобы с ними точно можно было дальше работать).
При этом, еще скажем, что последний символ, который эта грамматика сгенерирует (самый левый), будет еще говорить, что на него указывает головка (мы говорили выше про то как выглядят состояния МТ).

После этого добавим продукции, описывающие работу МТ, и продукции, которые по приходу в YES/NO делают то, что надо.
\end{description}
\end{proof}
\begin{Rem}
Кстати, это почти то же самое, что и задача ассоциативного исчисления, которая была в прошлом семестре на логике. 
Она неразрешима, и тут мы тоже не можем сказать, принадлежит ли слово произвольной грамматике: никак нельзя понять, МТ еще работает, но не готова нам выдать ответ, или она зависла. 
\end{Rem}

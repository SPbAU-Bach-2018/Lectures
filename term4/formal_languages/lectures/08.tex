\setauthor{Всеволод Степанов}

\begin{assertion}
Можно говорить, что у МП-автомата всего лишь одно состояние, принимаемые языки будут те же самыми
\end{assertion}
\begin{proof}
кладем на стек пару из буквы и состояния, принимаем, конечно, по пустому стеку
\TODO
\end{proof}

\begin{theorem}
    Языки МП-автоматов = КС-языки
\end{theorem}
\begin{proof}
    МП-автомат $A$ $\to$ КС-грамматика $G$.

    Будем считать, что автомат принимал по пустому стеку (мы уже показывали, что не важно, как он принимает, но так нам будет удобнее).

    Для каждой буквы посмотрим на момент, когда мы ее положили на стек, и когда ее сняли с него. 
    Рассмотрим слово, которое нам <<помогло>> ее снять, то есть, на подстроку $w$ между двумя состояниями.
    Получаем какой-то язык слов, которые могут перевести автомат из одного состояния в другое, сняв ровно одну букву со стека.
    
    Зададим $N = Q \times \Gamma \times Q$ "--- множество нетерминалов. 
    Будем обозначать нетерминал как $[qgp]$, это соответствует тройке (из какого состояния перешли, какой символ сняли со стека, в какое состояние перешли).
    Скажем, что этот нетерминал выводит ровно те слова, которые могут перевести автомат $A$ из $q$ в $p$, снимая со стека $g$, построим необходимые продукции для этого.

    Для каждого состояния $q$ рассмотрим все переходы из него вида $a, g/u$, где $u=u_1\dots u_n$ в состояние $p$.
    Мы сняли $g$, но положили $u$, теперь его надо снять.
    Для каждой последовательности $r_1 \dots r_n$ создадим продукцию $[q g r_n] \to a[p u_1 r_1][r_1 u_2 r_2]\dots [r_{n-1} u_n r_n]$.

    Осталось добавить стартовый символ $S$. 
    Будут продукции вида $S \to [q_0z_0q_i] \forall q_i \in Q$ "--- $S$ соответствует ровно всем тем словам, которые берут и снимают со стека $z_0$, оставляя его пустым.

    Формально покажем, что принимаем ровно те слова, что надо. Просто индукция не проходит "--- так как у нас продукции рекурсивные, могут циклиться.
    Но можно делать индукцию по количеству переходов при приеме слова, тогда все получится. %%расписать как

    Стоит заметить, что, вообще говоря, получили экспоненциальную грамматику (от размера $u$). 
    Можно этого избежать, надо просто сказать, что у нас на стек будут добавляться короткие слова. 
    Будем класть не все слово целиком, а только две буквы "--- одну букву из слова, а другую "--- фиктивную, которую мы однозначно со стека снимем при следующем переходе.

    КС-грамматика $G$ $\to$ МП-автомат $A$.

    У нас будет автомат с одним состоянием, как это было описано выше.
    $w \in L(G)$, тогда существует левостороний вывод $s \xLongrightarrow{*} w$. 
    Переходу $s \xLongrightarrow{*} uA\alpha, u \in \Sigma^* \alpha \in (\Sigma \cup N)^*$ будет соответствовать стек, 
    на вершине которого будет $A$, а потом $\alpha$, а текущяя буква слова будет $u$.  % это было просто неформальное введение

    Для перехода $B \beta \in (\Sigma \cup N)^*$ будет соответствовать переход $\epsilon, B/\beta$.
    
    Если на стеке лежит сейчас нетерминал, то все ок, у нас будут переходы, соответствующие продукциям для этого нетерминала.
    А если же терминал, то надо сказать, что именно его мы и ожидаем сейчас в слове, сделаем просто переход $a, a/\epsilon$ -- снимаем нетерминал со стека, и читаем его в слове.
    \TODO примерчик происходящего: $S \to 0A1 \to 00A1 \to 0011$. Стек будет выглядеть так: $S, 0A1, A1 ,0A1, A1, 11, 1, \epsilon$

    Доказательство корректности опять по индукции по длине вывода.
\end{proof}          

\subsection{Детерминированные конечные автоматы с магазинной памятью (ДМП)}
К сожалению, в отличие от обычных автоматов, нельзя так легко избавиться от недетерминированности. 
Там мы пользовались тем, что у нас всего не очень много допустимых конфигураций, но тут это не так "--- у нас есть стек, на котором, вообще говоря, может быть что угодно, поэтому состояний бесконечно много.
Умеем делать только что-нибудь уровня перебора, что имеет экспоненциальную сложность.
Поэтому имеет смысл рассматривать детерминированные автоматы с магазинной памятью, так как для них понятно, как принимать слово.
К сожалению, они не будут равномощны недетерминированным автоматам, мы это покажем.


\begin{Def}
МП-автомат называется детерминированным, если 
\begin{itemize}
    \item
        В $\delta$ для $(q \in Q, a \in \Sigma \cup \{\epsilon\}, g \in Gamma$ есть переход ровно в одну пару $(p \in Q, \gamma \in Gamma^*)$.
        $\delta\colon Q \times (\Sigma \cup \{\epsilon\}) \times \Gamma \to Q \times \Gamma^*$.
    \item
        Если есть переход $(q, a \in \Sigma, g) \to ()$, то нет перехода $(q, \epsilon, g) \to ()$ "--- либо есть переходы по обычным символам, или только по $\epsilon$.
\end{itemize}
\end{Def}

\begin{Def}
    $L$ "--- префиксный язык, если $\nexists w_1, w_2 \in L$: $w_1$ "--- префикс $w_2$.
\end{Def}

В отличие от МП-автоматов, у ДМП-автоматов прием по пустому стеку и по терминальному состоянию это не одно и то же, сейчас это покажем.

\begin{theorem}
Язык $L$ принимается ДМП автоматом по пустому стеку $\leftrightarrow$ $L$ - префиксный и принимается ДМП-автоматом по терминальному состоянию
\end{theorem}
\begin{proof}
$\Rightarrow$

Пусть есть слово $w \in L$, при этом еще существует непустое слово $u$ такое, что $wu \in L$.
Получаем противоречие, так как, когда мы примем префикс $w$ слова $wu$, у нас будет пустой стек и мы закончим работу.
Из всех состояний нашего автомата надо добавить переход $\epsilon, z_1/z_1$ в новое, терминальное, состояние. 
Тут мы, как и на прошлой лекции, еще добавили новое стартовое состояние и новый символ $z_1$, из нового начального состояния в старое есть переход $\epsilon, z_0/z_0z_1$.

$\Leftarrow$
$A$ "--- ДМП-автомат, принимающий по терминальному состоянию префиксный язык $L(A)$.

Это значит, что при приеме любого слова мы только один раз бываем в терминальном состоянии "--- иначе можно принять еще и его префикс.
На самом деле, это не правда, можем прийти в терминальное состояние, потом сделать $\epsilon$-переход, попасть снова в терминальное состояние. 
Но для дальнейшего перехода это совсем не важно.

Все переходы, которые были из терминальных состояний, удаляем (потому что они при приеме слова из языка все равно не использовались), добавляем петли $\epsilon, \forall/\epsilon$

Есть еще следующая проблема: могли в исходном автомате слово не принять, но попасть в какое-то состояние, имея пустой стек. 
Чтобы это слово не принялось в новом автомате, нужно, как обычно, чтобы было <<двойное дно>> "--- опять новое начальное состояние, символ $z_1$ в самом низу стека, который снимется в самом конце.
\end{proof}

\begin{theorem}
Языки, принимаемые ДМП-автоматами по терминальным состояниям, строго вложены в КС-языки, а регулярные строго вложены в ДМП-автоматы, принимающие по терминальным состояниям.
\end{theorem}
\begin{proof}
Нестрогие вложенности очевидны: к ДКА для регулярного языка легко приделать стек "--- просто убираем $z_0$ и кладем его обратно.
Вторая нестрогая вложенность тоже очевидна "--- ДМП это частный случай МП, которые в точности равны КС-языкам.

Надо теперь доказать, что классы не равны.

Регулярные $\neq$ ДМП-автоматам.

$L = \{0^n10^n\}$
Этот язык не регулярный. Для доказательства используем лемму о накачке. Зафиксировали $n \in \N$, возьмем слово $w = 0^n10^n$, для любого его подходящего разбиения на три части лемма о накачке ломается.

А вот ДМП-автомат для этого языка есть "--- просто кладем нолики на стек, пока не придет единичка, а потом достаем их обратно. 
Кончился стек "--- переходим в терминальное состояние.

Для второго неравенства возьмем язык $\{ww^R, w \in \{0, 1\}^*\}$.
Неформаьльно легко объяснить, почему он подходит "--- не понятно, когда ДМП-автомату надо прекращать класть на стек и начать снимать с него. 
А то, что это КС-язык мы доказывали, например, на практике.

Докажем это формально, но для другого языка $L = \{0^n1^n\} \cup \{0^n 1^{2n}\}$.

А теперь формально покажем, что этот язык не принадлежит ДМП-автоматам, принимающим по терминальному состояню.
От противного, пусть нашелся такой автомат.

Возьмем его терминальное состояние, добавим переходы, чтобы принимать еще и $0^n 1^{2n}$.
А теперь сделаем копию (убирая начальность у начального состояния), добавим $\epsilon$-переходы из терминальных состояний исходного в соответствующие терминальные состояния копии.
Заменим в копии единички на двойки. 
\TODO картинка!

Получим, что этот автомат будет принимать язык $\{0^n1^n\} \cup \{0^n1^{2n}\} \cup \{0^n1^n2^n\}$. 
Покажем, что принимается действительно этот язык.
Пусть $w$ принимается автоматом. Если не было $\epsilon$-перехода в копию, то оно принимается исходным автоматом, то есть, лежит в $\{0^n1^n\} \cup \{0^n1^{2n}\}\}$.

Если $\epsilon$-переход был, то уберем его, перенесем все переходы из копии в исходный автомат, получили какое-то слово $u$ из ноликов и единичек.
Был какой-то префикс $p$ слова, который мы приняли перед тем, как был $\epsilon$-переход в копию.
$p$ это либо $0^n1^n$, или $0^n1^{2n}$. 
В первом случае $u$ это либо $0^n1^n$, тогда $w = 0^n1^n$, либо $u=0^n1^{2n}$, тогда $w = 0^n1^n2^n$.
Во втором случае $u = 0^n1^{2n}, w = 0^n1^2n$. 

Это мы формально показали, что принимается не более чем этот язык. Несложно понять, что каждое слово из этого языка принимается (из построения).

Заметим, что $\{0^n1^n2^n\}$ не КС-язык (а, значит, и $\{0^n1^n\} \cup \{0^n1^{2n}\} \cup \{0^n1^n2^n\}$ тоже)
Пользуемся леммой о накачке. Зафиксировали $n$, взяли слово $0^n1^n2^n$. 
Центральная часть разбиения не могла одновременно захватить все три цифры, поэтому, если мы ее повторим несколько раз, то получим слово не из языка, противоречие.

Итого, мы взяли ДМП-автомат, получили, что он принимает то, что не принимает КС-язык, при этом мы знаем, что есть вложенность.
Противоречие
\end{proof}
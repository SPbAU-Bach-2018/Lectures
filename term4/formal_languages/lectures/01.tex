\setauthor{Всеволод Степанов}

\section{Введение}
Курс будет читаться придерживаясь первых семи глав книги Хопкрофта, Мотвани и Ульмана "Введение в теорию автоматов языков и вычислений".

На практиках будем решать задачки, разбирать домашки.
Помимо этого, говорят, что мы иногда будем что-то кодить, а еще есть вероятность, что будет что-то типа контестов (с автоматической системой проверки), если успеют допилить. \\

Хочется как-то взять и научиться для языков строить какие-то вменяемые системы формальных правил, описывающих эти языки.
Для некоторых это сделать просто (например, языки программирования, у них, как правило, есть весьма четкая и строгая спецификация, 
а у какого-нибудь brainfuck она вообще занимает всего лишь страницу, при этом он Тьюринг-полный).

С естественными языками все не так гладко "--- принимались разные попыки их как-то формализовать, структуризировать, или же изобрести новый язык, с которым все это сделать, но все как-то безуспешно.
Тем не менее, попытки не прошли совсем зря, какие-то результаты были и этим пользуются. \\

Работать будем с \textbf{конечным} непустым алфавитом $\Sigma$. 
Почти всегда нам его размер будет не особо важен, будем работать с алфавитом $\Sigma = \{0, 1\}$, будем проговаривать, если это не так. 

\begin{Def}
Словом будем называть \textbf{конечную} последовательность элементов алфавита. 
Слово может быть пустым, оно такое одно, обозначается как $\epsilon$.
\end{Def}
\begin{Def}
Язык "--- множество слов. Язык может быть пустым.
\end{Def}
\begin{Rem}
Конечность нам почти везде будет важна, без нее почти все сломается.
\end{Rem}

Заметим, что языки у нас бывают либо конечными, либо счетными: всегда можно просто выписать все слова, посортировав их по длине, 
а равные по длине -- лексикографически в силу конечности алфавита.

А множество всех языков будет равномощно $2^\N$. В частности, это значит, что есть неразрешимые языки (привет, матлогика). 
Это весьма печально, так как для того, чтобы описать язык, надо, по сути, научиться для слова понимать, принадлежит ли оно этому языку, то есть, решить задачу распознавания.

Можно попробовать ограничиться лишь разрешимыми языками, но это тоже плохая идея. 
Большинство алгоритмов просто будут работать очень и очень долго, про них лишь будет известно, что когда-то они завершаются, но непонятно, когда. Это все очень плохо и неприменимо на практике.
Поэтому, глобальная цель курса "--- изучать какие-то простые системы для распознавания каких-то подмножеств разрешимых языков. 
Стоит отметить, что далеко не для всех языков нам удастся это сделать.


\chapter{Детерминированные конечные автоматы}
В этом параграфе везде будем работать с алфавитом $\Sigma = \{0, 1\}$, но для других алфавитов все будет аналогично.

\begin{Def}
Детерминированным конечным автоматом (ДКА, DFA) называют пятерку $(\Sigma, Q, q_0, T, \delta)$, где:

$Q$ "--- конечное множество всех состояний

$q_0 \in Q$ -- начальное состояние

$T \subseteq Q$ -- множество всех терминальных состояний (может быть и пустым)

$\delta: Q \times \Sigma \to Q$ -- функция перехода

Автоматам умеем ``скармливать'' строчки: итерируемся слева направо по символам строки. 
Пусть сейчас мы находились в состоянии $q$, в строке был символ $c$, тогда наше новое состояние это $\delta(q, c)$.
\end{Def}

\begin{Def}
Обобщенная функция перехода $\hat \delta(q, w) =     
    \begin{cases}
        q, w = \epsilon \\
        \hat \delta(\delta(q, c), s), w = cs, c \in \Sigma, s \in \Sigma^*
    \end{cases}
$
\end{Def}

\begin{Def}
    Автомат принимает слово $w$, если $\hat \delta(q_0, w) \in T$.
\end{Def}
\begin{Rem}
    На самом деле, если мы прогнали слово через автомат, то мы заодно для каждого его префикса поняли, принимается ли он автоматом.
\end{Rem}

\begin{Def}          
$A$ "--- ДКА, $L(A) = \{w \colon А \text{ принимает } w\}$ "--- язык, распознаваемый $A$.
\end{Def}

\begin{exmp}
Рассмотрим, в качестве примера, автомат, принимающий только числа (записанные в двоичном виде, разумеется), делящиеся на 6. 

Будем поддерживать следующий инвариант: пусть мы скормили какой-то префикс строки, стоим в вершине $i$, тогда остаток этого префикса при делении на 6 будет равен $i$.
Тогда дописывание к слову справа нуля "--- умножение на 2, дописывание 1 "--- еще добавить 1.
Терминальное состояние будет только одно -- вершина с номером 0.
\begin{dot2tex}[tikz,scale=.75,options=-t math]
digraph G {
    rankdir=LR;
    ranksep="0.0";
    mindist="0";


    node [shape = none] ""
    node [shape = doublecircle]; 0;
    node [shape = circle] 1 2 3 4 5;
    "" -> 0;
    0 -> 0 [ label = "0" ];
    0 -> 1 [ label = "1" ]; 
    1 -> 2 [ label = "0" ];
    1 -> 3 [ label = "1" ];
    2 -> 4 [ label = "0" ];
    2 -> 5 [ label = "1" ];
    3 -> 0 [ label = "0" ];
    3 -> 1 [ label = "1" ];
    4 -> 2 [ label = "0" ];
    4 -> 3 [ label = "1" ];
    5 -> 4 [ label = "0" ];
    5 -> 1 [ label = "1" ];
}
\end{dot2tex}
\end{exmp}

\begin{Def}
Размер ДКА "--- количество состояний в нем
\end{Def}

\begin{exmp}
Еще один полезный пример "--- автомат, принимающий только пустое слово:
\begin{dot2tex}[tikz,scale=.75,options=-t math]
digraph G {
    rankdir=LR;
    ranksep="0.0";
    mindist="0";

    node [shape = none] ""
    node [shape = doublecircle]; 0;
    node [shape = circle] 1;
    "" -> 0;
    0 -> 1 [ label = "0" ];
    0 -> 1 [ label = "1" ]; 
    1 -> 1 [ label = "0" ];
    1 -> 1 [ label = "1" ];
}
\end{dot2tex}
\end{exmp}

\begin{Def}
Тупиковое состояние "--- состояние, из которого невозможно дойти до хоть какого-нибудь терминального.
\end{Def}
\begin{Rem}
Иногда такие состояния называют дьявольскими. Это вполне себе принятое обозначение, но, все-таки, обычно используют термин ``тупиковое состояние''.
\end{Rem}
В автомате для языка $\{\epsilon\}$, нарисованном выше, вершина 1 как раз является тупиковой.
Есть договоренность, что такие вершины при изображении автомата можно опускать, при этом, если из вершины нет перехода по какому-то символу, то он, на самом деле, ведет в тупиковое состояние.
\begin{Rem}
При этом, однако, надо помнить, что это состояние может присутствовать, например, когда считаем размер автомата.
\end{Rem}

\begin{exmp}
Вот так вот будет выглядеть автомат, если убрать тупиковое состояние:
\begin{dot2tex}[tikz,scale=.75,options=-t math]
digraph G {
    rankdir=LR;
    ranksep="0.0";
    mindist="0";

    node [shape = none] ""
    node [shape = doublecircle]; 0;
    "" -> 0;
}
\end{dot2tex}

Не забываем, что в нем, на самом деле, две вершины.
\end{exmp}

Мы сопоставили каждому ДКА какой-то язык. В то же время, каждому языку может соответствовать несколько ДКА: например, можно просто добавить кучу вершин, недостижимых из начальной.

\begin{Def}
Два автомата называются эквивалентными, если они распознают один и тот же язык. 
\end{Def}

Понятно, что мы сейчас разбили автоматы на классы эквивалентности, хочется взять и найти в каждом классе какого-то канонического представителя.
Кажется вполне логичным, что чем меньше автомат, тем лучше. Например, с точки зрения скорости вычислений.

Для начала возьмем и построим алгоритм, который берет и строит по данному ДКА эквивалентный ему, но меньшего размера, а потом докажем, что этот алгоритм нам даст канонического представителя 
(в частности, надо будет показать, что мы получаем минимальный автомат, и при этом он ровно один).

\section{Алгоритм минимизации ДКА}
\begin{Def}
Два состояния $p, q$ называются различимыми, если $\exists w \in \Sigma^*: \hat \delta(p,  w) \in T \oplus \hat \delta(q, w) \in T$ "---
найдется такое слово, что если по нему из этих двух состояний перейти, то ровно одно из двух итоговых состояний будет терминальным.

Соответственно, состояния неразличимы, если множества слов, которые они принимают, совпадают.
\end{Def}
\begin{Rem}
Отношение неразличимости, как несложно убедиться, будет являться отношением эквивалентности.
\end{Rem}

Собственно, сам алгоритм:
\begin{enumerate}
\setcounter{enumi}{-1}
\item
Возьмем, обойдем автомат \t{dfs}-ом как орграф, некоторые вершины посетим, некоторые нет. 
Удалим непосещенные вершины, так как мы в них никогда не придем, вне зависимости от входа.

\item
Хотим для каждой пары состояний узнать, различима она, или нет. Конечная цель "--- разбить все вершины на классы эквивалентности.

Изначально мы знаем, что пара состояний $(t \in T, q \notin T)$ неразличима: в качестве $w$ в определение можно подставить $\epsilon$.

\item
\begin{lemma}
Если $(r, s)$ "--- пара различимых состояний, $\delta(p, x) = r, \delta(q, x) = s$, то $(p, q)$ -- тоже пара различимых состояний.
\end{lemma}
\begin{proof}
Достаточно просто взять и к различающей $(r, s)$ строке добавить в начало $x$.
\end{proof}

Рассмотрим граф, вершины в котором "--- все возможные пары состояний. Между двумя вершинами $(p_1, q_1), (p_2, q_2)$ проведем ребро с символом $c$, если $\delta(p_1, c) = p_2, \delta(q_1, c) = q_2$.
Обойдем этот граф по обратным ребрам, запуская \t{dfs} из всех пар вида (терминальная, нетерминальная), и наоборот.

\item
Посещенные \t{dfs}-ом вершины это в точности пары различимых состояний. 

Очевидно, что если мы вершину посетили, то соответствующая ей пара состояний различима "--- просто посмотрим на путь из стартовой вершины обхода до данной, применим по индукции к каждой вершине этого пути лемму.

То, что непосещенной вершине соответствует пара неразличимых состояний тоже понятно. 
Пусть они различимы, тогда, по определению, существовал бы путь до пары $(p, q): p \in T, q \notin T$, или наоборот. То есть, вершина была бы достижима из исходных. 
Противоречие.
\item
Итак, мы смогли взять и разбить вершины на классы эквивалентности. 
Возьмем и каждый класс стянем в одну вершину, если был переход из $p$ в $q$ по символу $c$, то теперь будет переход из $[p]$ в $[q]$ по тому же символу, где $[p], [q]$ -- классы эквивалентности для $p, q$. 
От кратных ребер, понятное дело, избавились.
Если $q \in T$, то скажем, что в полученном автомате $[q] \in T$. Соответственно, начальной вершиной будет $[q_0]$.

Надо понять, что мы сейчас получили корректный автомат.

Для начала заметим, что каждое терминальное состояние в новом автомате состоит только из терминальных состояний старого, так как любая нетерминальная вершина различима с терминальной.

Теперь надо посмотреть на переходы и понять, что из одного состояния по каждому символу есть ровно один переход. 

Легко заметить, что хотя бы один точно будет, так как он был в исходном автомате. 
Пусть их хотя бы два, при этом они ведут в разные вершины, то есть, в разные классы эквивалентности $[p], [q]$. %%КАРТИНКА

Пусть в $p$ был переход из $s$, а в $q$ из $r$, $[s] = [r]$. Понятно, что вершины $p, q$ различимы, так как они находятся в разных компонентах. 
Ну а тогда по лемме вершины $s, r$ тоже различимы, то есть, лежат в разных классах эквивалентности, противоречие.

\item
Итак, у нас есть корректный автомат, в нем вершин не больше, чем в исходном.

А теперь заметим, что скармливанию строки исходному автомату можно сопоставить скармливание строки сжатому "--- просто взяли последовательность состояний $q_0, q_1, \dots, q_{|w|}$, отобразили ее в 
$[q_0], [q_1], \dots, [q_{|w|}]$, при этом результат (строка принята/не принята) не изменился: $q_{|w|} \in T \iff [q_{|w|]} \in T$.

\item
Осталось еще оценить его асимптотику. Всего у нас $\O(n^2)$ состояний и $\O((n\Sigma)^2)$ ребер. 
Но, на самом деле, далеко не все ребра мы будем рассматривать: из каждой пары вершин есть всего $\Sigma$ допустимых переходов, так как мы переходим одновременно по одинаковому символу.
И итоговая асимтотика будет, на самом деле, $\O(n^2 \Sigma)$

Как правило, $\Sigma$ в асимптотике опускают, так как это какая-то константа, которая нас не очень волнует. Тогда асимптотика просто $\O(n^2)$.
\end{enumerate}
\begin{Rem}
Существует алгоритм Хопкрофта, работающий за $\O(n \log n)$, который даже вполне применим на практике. 
Рассказ занимает где-то одну лекцию, он реально сложный для понимания, поэтому рассматривать его не будем, желающие могут с ним ознакомиться.
\end{Rem}

\begin{exmp}
Если применить этот алгоритм к автомату, принимающему числа, делящиеся на 6, получим следующее (при желании можно честно выписать пошагово что алгоритм делает и убедиться, что все будет действительно именно так):
\begin{dot2tex}[tikz,scale=.75,options=-t math]
digraph G {
    rankdir=LR;
    ranksep="0.0";
    mindist="0";

    node [shape = none] ""
    node [shape = doublecircle]; 0;
    node [shape = circle] "1, 4" "2, 5";
    "" -> 0;
    0 -> 0 [ label = "0" ];
    0 -> "1, 4" [ label = "1" ]; 
    "1, 4" -> "2, 5" [ label = "0" ];
    "1, 4" -> 3 [ label = "1" ];
    "2, 5" -> "1, 4" [ label = "0" ];
    "2, 5" -> "2, 5" [ label = "1" ];
    3 -> 0 [ label = "0" ];
    3 -> "1, 4" [ label = "1" ];
}

Мы взяли и уменьшили автомат на 2 вершины.
\end{dot2tex}

\end{exmp}

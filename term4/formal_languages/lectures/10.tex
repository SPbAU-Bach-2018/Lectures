\setauthor{Всеволод Степанов}

\section{Лемма Огдена}
\begin{theorem}[Лемма Огдена (обобщенная лемма о накачке для КС-языков)]

Пусть $\Gamma$ "--- КС-грамматика. Тогда $\exists n \in \N$ такое, что $\forall w \in L(\Gamma) \colon |w| \geq n$, 
при этом, если выделить хотя бы $n$ позиций в $w$, будет верно, что $\exists u, v, x, y, z$ такие, что
$x$ содержит выделенную позицию и
одно из двух верно: $u, v$ содержат выделенную позицию или $y, z$ содержат ее
$vxy$ содержит не более чем $n$ выделенных позиций, и при этом
$uv^ixy^iz \in L(\Gamma)$ 

\end{theorem}
\begin{proof}
Рассмотрим все продукции, посмотрим на их правые части, длину самой большой из них обозначим за $l$.
Крайний случай: если $l = 1$, скажем на всякий случай, что $l = 2$.

Возьмем $m = |N|, n = l^{2m+3}$.

Берем произвольное слово $w$, можем построить дерево разбора, соответствующее тому, как мы из $S$ получили $w$.

Рассмотрим следующий путь из корня в этом дереве: начинаем с $u_1$ "--- корень, а дальше по индукции для любого $i \geq 2$ выбираем $u_i$ так: 
смотрим на детей $u_{i-1}$, выбираем того, у которого в поддереве наибольшее количество выделенных позиций.
В итоге жадно получили какой-то путь $u_1 \dots u_p$ ($u_p$ "--- лист).

Детей у каждой позиции не более чем $l$, смотрим на количество выделенных позиций у $u_{i-1}$, смотрим на количество выделенных позиций, у $u_i$,
понимаем, что это количество уменьшилось не более чем в $l$ раз
(мы шли в сына с наибольшим числом выделенных вершин, если их количество уменьшилось более чем в $l$ раз, то просто не сойдется сумма и будет противоречие).
У корня было хотя бы $l^{2m+3}$ выделенных позиций, тогда у $i$-й вершины их хотя бы $l^{2m+4-i}$.
Откуда получаем, что $p > 2m+3$: $u_p$ "--- лист, у него в поддереве не более одной выделенной вершины (он сам), если $p$ маленькое, то их должно быть больше.
\begin{Rem}
Тут таки важно, что $l \geq 2$, как мы потребовали в начале. 
\end{Rem}

Рассмотрим все вершины на нашем пути, посмотрели на $u_i$ и его ребенка $u_{i+1}$. 
Посмотрим на всех детей $u_i$, которые левее, чем $u_{i+1}$. 
Если для какого-то такого ребенка верно, что у него в поддереве хотя бы одна выделенная позиция, то назовем $u_i$ левоветвящейся.
Если то же самое верно для правых детей "--- назовем правоветвящейся.
Ветвящаяся вершина "--- либо лево-, либо правоветвящаяся (либо одновременно и то и то)

Рассмотрим последние $2m + 3$ ветвящиеся вершины, тогда верно (принцип Дирихле), что среди них либо количество левоветвящихся хотя бы $m + 2$, либо правоветвящихся хотя бы $m + 2$.
Ветвящихся вершин хотя бы столько, так как количество выделенных вершин в поддереве среди $u_i$ уменьшалось хотя бы $m+3$ раза, как было показано выше.
То есть, какие-то вершины оказывались левее/правее $u_{i+1}$.

НУО, левоветвящихся хотя бы $m + 2$. Обозначим $v_1 \dots v_{m+2}$ "--- последние левоветвящиеся вершины из пути $u_1 \dots u_p$.
По принципу Дирихле есть $i < j$ такие, что нетерминалы, записанные в вершинах $v_i, v_j$ совпадают.
\TODO Картинка (похожа на просто лемму о накачке).

Среди всех таких пар $(i, j)$ выбрали пару с максимальным $i$.
Обозначили за $x$ поддерево $v_j$, $v, y$ "--- левая и правая части поддерева $v_i$ (выкинули поддерево $v_j$). 
$u, z$ "--- остальное. 
Заметим, что у $x$ хотя бы одна выделенная позиция есть (по построению, $u_p$ лежит в $x$).

Так как у нас $v_i, v_j$ "--- левоветвящиеся, то у $v_i$ есть ребенок слева от пути, у которого есть выделенные позиции. 
Тогда в $v$ есть выделенные позиции.

Так как в пути хотя бы $m + 2$ вершины, то $i > 1$ (выкидываем $v_1$ и принцип Дирихле все равно работает).
Тогда над $v_i$ тоже есть хотя бы одна левоветвящаяся вершина, значит, в $u$ тоже есть выделенная позиция.

Аналогично, если бы у нас было бы много правоветвящихся вершин, то было бы верно, что в $y, z$ есть выделенные позиции.

Теперь хотим понять, почему $vxy$ содержит не более чем $n$ выделенных позиций.
Заметим, что после $v_i$ у нас было не более чем $2m + 3$ ветвящихся вершины (потому что мы брали только последние $2m + 3$ ветвящихся)

Поскольку каждый раз у нас количестов выделенных позиций уменьшалось не более чем в $l$ раз, при этом, 
оно могло уменьшаться только в ветвящихся вершинах (если вершина не ветвящаяся, то все выделенные позиции у одного ребенка), то верно, что
в поддереве $v_i$ не более $n = l^{2m + 3}$ выделенных позиции.

То, что можно сколько угодно раз повторять $v, y$ доказывается абслютно так же, как и для обычной леммы о накачке.
Просто говорим, что мы нетерминал, стоящий в $v_i$, превратили в его же ($v_j$), а еще слева и справа выписали $v, y$. 
А теперь можно $v_j$ опять вот так вот разложить, и так сколько угодно раз. 
А можно еще вообще сразу сказать, что мы ничего дублировать не будем и  вместо $v_i$ нарисуем сразу $v_j$ со своим поддеревом, просто $v, y$ вырежем.
\end{proof}
\begin{Rem}
Можно было привести в нормальную форму и получить $l = 2$, доказательство бы от этого не изменилось.
\end{Rem}
\begin{conseq}
Обычная лемма о накачке, когда выделяем все позиции в лемме Огдена.
\end{conseq}

\section{Нормальная форма Грейбах}
\begin{Rem}
Грейбах "--- женщина, поэтому именно так, а не Грейбаха. 
Как алгоритм Ахо-Корасик.
\end{Rem}

Это когда есть только продукции вида $\\
A \to a \\
A \to aB \\
A \to aBC$

Ну и обычный формализм "--- если хотим пустое слово, то делаем новый стартовый символ.

На практике это соответствует очень удобным автоматам с магазинной памятью.

Еще есть ослабленная НФ Грейбах "--- это только продукции $A \to a \beta, a \in \Sigma, \beta \in N^*$.

\begin{theorem}
Любая КС-грамматика представима в виде ослабленной НФ Грейбах.
\end{theorem} 
\begin{Rem}
Можно и обычную НФ получить, но на лекции не получилось.
\end{Rem}
\begin{proof}
$G$ "--- КС-грамматика

\begin{enumerate}
\item
Удаление $\epsilon$-продукций.

\item
Удаление непосредственных левых рекурсий. Это когда $A \to A \beta$. 
Есть еще косвенная левая рекурсия, это когда мы в итоге получим слева $A$, но не сразу, а через несколько продукций (например, $A \to B \to Aa$).
Аналогично бывают и правые рекурсии.

Непосредственная левая рекурсия на самом деле просто говорит, что мы порождаем что-то, где нет $A$ в качестве первого символа, а потом кучу раз дописываем в конец $\beta$.

Пусть была продукция $A \to A \beta_1 \mid A\beta_2 \mid \dots \mid A \beta_k \mid \gamma_1 \mid \dots \mid \gamma_m$, $\gamma_i$ не начинаются с $A$.
Тогда заменим ее на $\\
A' \to \beta_1 A' \mid \dots \mid \beta_k A' \mid \beta_1 \mid \beta_k \\
A \to \gamma_1 A' \mid \dots \mid \gamma_m A' \mid \gamma_1 \mid \dots \mid \gamma_m$ 

Понятно, что получили то же самое, но теперь нету непосредственной левой рекурсии.

\item
А теперь удалим еще и косвенную.
\begin{verbatim}
for i = 1 .. n
    for j = 1 .. i - 1
        for продукция $A_i \to A_j \gamma$
            for продукция $A_j \to \beta$
                создадим $A_i \to \beta \gamma$
            Убираем продукцию $A_i \to A_j \gamma$
    Устраним непосредственные левые рекурсии для $A_i$
\end{verbatim}
Здесь $\beta, \gamma \in (\Sigma \cup N)^*$.

Докажем, что после $i$-го шага из $A_i$ есть только продукции $A_i \to A_{>i} \dots \mid a$.
Как обычно, делаем это по индукции, база $i=1$ верна (это просто отсутствие непосредственных левых рекурсий).

Переход: есть продукции $A_i \to A_i \dots$, это непосредственные левые рекурсии, мы их убиваем. 
Есть еще $A_i \to B=A_{<i}$. Мы знаем по предположению индукции, что $B$ раскрывается в $A_{\geq i} \dots$, равенство убьем, получим корректный переход.

\item
Удаление нетерминалов в началах правых частей
\begin{verbatim}
for i = n .. 1
    for j = i + 1 .. n
        for продукция $A_i \to A_j \gamma$
            удаляем ее
            for продукция $A_j \to \beta$
                создаем $A_i \to \beta \gamma$
\end{verbatim}
Будет поддерживаться (на каждой итерации внешнего цикла) инвариант что все продукции для $A_{\geq i}$ имеют вид $A_{\geq i} \to a \gamma$, доказывается как и прошлый пункт.

\item
Теперь в начале всегда терминалы, но их может быть несколько.
Формализм "--- введем соответствующие нетерминалы для каждого терминала, заменим лишние терминалы на нетерминалы.     

\end{enumerate}
\end{proof}
\begin{Rem}
Вообще говоря, приведенное доказательство неправильное.
Проблема в том, что, когда мы убираем непосредственные левые рекурсии каждую итерацию, у нас появляются какие-то новые терминалы.
Во-первых, у нас из-за этого появляются новые нетерминалы (почему внешний цикл не зациклится?), а еще есть проблема, что у нас может сломаться инвариант $A_i \to A_{>i} \dots$. 

Эти проблемы решить очень просто: скажем, что у нового, штрихованного терминала номер будет минимальным среди всех. 
Тогда все продукции, где он в левой части, будут удовлетворять инвариант (и итерацию внешнего цикла по нему делать не надо), а сам по себе он нигде не является самым левым нетерминалом в правой части, поэтому второй проблемы тоже нет.
\end{Rem}
На этом курс закончился. Лекций больше не планируется, конец.

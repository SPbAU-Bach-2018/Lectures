\section{} % 10
	Алгоритм проверки слова $x$ на принадлежность грамматике в форме Хомского.
	У нас трёхмерная динамика по подотрезкам: можно ли такой подотрезок слова получить из нетерминала $X$.
	Переход: если отрезок длины один, то надо сразу получить букву (цепных-то нет), а если большей длины "--- то первая
	продукция точно $A \to BC$, надо перебрать границу распила и по какому правилу получим.
	Итого квадрат состояний, из каждого "--- линия на $|G|$ переходов.

\section{} % 11
	Пусть $L$ "--- КС-язык.
	Тогда есть такое $n$, что любое слово длины хотя бы $n$ можно распилить на пять кусков,
	причём средние три в сумме длины не более $n$, второй и четвёртый в сумме непусты и их можно
	повторять/убирать сколько угодно раз.
	Док-во: взяли грамматику в форме Хомского, убрали пустую строку (оно на лемму не влияет).
	Положим $n=2^{|N|}$, возьмём любое слово такой длины, смотрим на дерево разбора (оно бинарное в Хомском),
	рассмотрим путь до самой глубокой вершины, он имеет хотя бы $|N|+1$ нетерминал по Дирихле (иначе у дерева мало листьев; аккуратно с $\pm1$,
	высотой дерева в рёбрах/вершинах, и листьями-терминалами).
	В этом пути нашли самый глубокий повтор нетерминала, он распилил слово на пять частей, надо проверить все условия.

	Примеры не-КС: $\t{0}^n\t{1}^n\t{2}^n$ (следовательно, нельзя пересекать КС-языки), язык тандемных повторов ($\{ww\}$).
	А вот объединять КС-языки всё ещё можно.

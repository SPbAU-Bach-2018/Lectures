\section{} % 04
	Академические регулярки (АРВ): они проще обычных.
	По приоритету: $\varnothing$ (пустой язык), $\epsilon$ (пустая строка), символ, скобочки, звезда Клини (повторить ноль и более раз),
	конкатенация, объединение $\mid$: $\t{(0|11*)*}=\t{(0|(1(1)*)*}$.
	Язык $L$ регулярен, если есть АРВ $R$ и $L(R)=L$.

	Теорема Клини: автоматные $\iff$ АРВ.
	Можно по DFA явно построить АРВ кубической динамикой, похожей на Флойда.
	$R_{ijk}$ "--- АРВ для слов, переводящих автомат из состояния $i$ в $j$ с промежуточными состояниями
	номерами не более $k$ (например, если $k<i$, то по петле в $i$ пройти можно не более одного раза).
	Инициализируем с $k=0$ (пустым языком, тут он как раз нужен, или одним символом на петле).
	Переход "--- мы можем пройти по вершине $k$ несколько раз, надо дописать Клини (получим объединение
	ответя для прошлого слоя и ещё три выражения с прошлого слоя).
	В конце надо объединить все пути из начальной в каждую терминальную.
	На слое $0$ у нас длина выражений константа, на каждом следующем увеличиваем размер в четыре раза (1+3),
	итого $4^{|Q|}$.

\section{} % 05
	Теорема Клини из АРВ в автомат.
	Инвариант: в начальное нет переходов, одно терминальное (из него нет переходов),
	начало и терминал отличаются.
	Надо построить для пустого языка, для $\epsilon$, для буквы, для Клини (надо провести $t \to s$,
	$s' \to s$, $s, t \to t'$, чтобы принимался ноль; инварианты нужны, чтобы нельзя было принять кусок, вернуться
	в начало, а только потом пройти в конец), для конкатенации (нарисовали рядом, дописали $\epsilon$-переход)
	и для объединения (добавили $s_0$ и $t_0$, сделали четыре $\epsilon$-перехода).

\section{} % 06
	Лемма о накачке/разрастании (pumping lemma), они бывают для разных классов языков.
	Для регулярных: если $L$ регулярен, то для него есть константа $n$ такая,
	что если взять слово $w\in L$ длины хотя бы $n$,
	то его можно распилить на три части (первые две суммарной длины не более $n$) и тогда
	все слова вида $xy^iz$ лежат в языке.

	Доказательство: рассмотрели автомат для языка, положили $n \coloneq |Q|$.
	Взяли произвольное слово, выписали первые $n+1$ состояние, где-то там нашли цикл,
	повторили сколько надо раз, успех.

	Так удобно доказывать нерегулярность: показать, что для любого $n$ есть слово, которое как бы мы не распилили,
	при некотором $i$ в языке лежать не будет.

\section{} % 07
	КС-грамматика "--- это чётвёрка: алфавит/терминалы $\Sigma$, нетерминалы/переменные/синтаксические категории $N$ ($N \cap \Sigma=\varnothing$),
	стартовый символ $S$, продукции/правила $P$ (слева всегда ровно один нетерминал, справа "--- $(N\cup\Sigma)^*$).
	Все мн-ва конечны.
	Символ "--- это элемент $\Sigma \cup N$.
	Нетерминалы "--- заглавные буквы, терминалы "--- строчные.
	Можно выводить по правилам: $\alpha B \gamma \Ra \alpha\beta\gamma$ (левое выводит правое).
	Левосторонний вывод "--- если меняем самый левый нетерминал в строке ($\alpha \in \Sigma^*$), есть правосторонний.
	Можно выводить несколько раз: $s \xLongrightarrow{*} t$.
	Грамматика задаёт язык выводимых из $S$ нетерминалов.

	Дерево разбора: нарисовали дерево, $S$ сверху, крона вершины $v$ "--- это строка, написанная в листьях её поддерева (дети упорядочены).
	Грамматика неоднозначна, если строка с двумя деревьями разбора (при этом порядок построения дерева неважен).
	Бывают языки, для которых нет однозначных ($\{\t{a}^n\t{b}^n\t{c}^m\}\cup\{\t{a}^m\t{b}^n\t{c}^n\}$, без док-ва).
	Регулярные языки задаются КС грамматиками (можно построить по автомату, заметить спецвид таких грамматик "--- леволинейные/праволинейные).
	Палиндромы нерегулярны, поэтому есть строгое включение.

\section{} % 08
	Удаление $\epsilon$-продукций из языка, в котором не лежит $\epsilon$: находим все $\epsilon$-порождающие нетерминалы,
	убиваем для них $\epsilon$-порождаемость, перебираем все правила, выкидываем из них все подмножества $\epsilon$-порождающих,
	грамматика экспоненциально разрослась (экспонента от максимальной длины правой части), размеры правых частей сохранились.
	Время работы "--- линия от размера грамматики плюс ответ.

	Удаление цепных продукций ($A \to B \to C \to xy$): построили орграф нетерминалов,
	для каждого нашли достижимые (dfs за $\O(E)$) и написали сразу $A \to xy$ (для всех правил), а цепные продукции удалили.
	Грамматика разраслась квадратично, $\epsilon$-продукций не появилось, размеры правых частей сохранились.
	Время работы "--- размер грамматики на число нетерминалов.

	Удаление бесполезных символов: посмотрели, из каких нетерминалов можно получить конечную строку из терминалов (dfs),
	а какие достижимы из $S$, оставили пересечение.
	Также удалили из алфавита невстречающиеся в правилах (а все встречающиеся точно можно встретить в языке).
	Получили грамматику размера не больше исходной, все свойства сохранились.
	Время работы "--- линейно от размера грамматики.

	Нормальная форма Хомского для грамматики: есть продукции только $A\to BC$, $A\to \alpha$ и одна продукция $S\to\epsilon$
	(чтобы получить пустую строку, иначе нельзя), все символы полезные.
	Приводим в форму: убили всех терминалов из длинных продукций,
	распилили $A \to XYZ$ на $A\to XA_1, A_1\to YZ$ (вводя нетерминалы), пока что разрастание лишь в константу раз.
	Потом удалили $\epsilon$-продукции (теперь без экспоненты, т.к. правые части мелкие; возможно, добавили правило $S\to\epsilon$), цепные продукции, бесполезные символы.
	И только сейчас можно удалять терминалы из продукций (потому что мы что-то копировали в удалении $\epsilon$-продукций),
	добавив по нетерминалу для каждого терминала.

\section{} % 09
	\TODO

\section{} % 15
	Иерархия Хомского (строгая): регулярные грамматики.
	Это все леволинейные грамматики ($A \to Bc$ и $\epsilon$) плюс все праволинейные грамматики ($A \to bC$ и $\epsilon$).
	Они задают регулярные языки: легко перестраиваются в автоматы.
	Если разрешить и $A \to Bc$, и $A \to bC$, то получим линейные грамматики, они чутка помощнее (но ещё не КС),
	мы проходили на практике, в иерархию Хомского не входят.

	Контекстно-свободные грамматики, вычисляются МП-автоматами.
	Пример нерегулярного: $\t{0}^n\t{1}^n$.

	Контекстно-зависимые грамматики.
	Справа лежит то же, что и слева, но можно заменить ровно один нетерминал на что угодно непустое.
	Ещё добавляем $S\to\epsilon$, если очень хочется.
	Тогда они эквивалентны неукорачивающим (есть только правила $\alpha\to\beta$ при $|\alpha|\le|\beta|$).
	Вложены в них очевидно, так как в КЗ правила неукорачивают.
	Обратное вложение: сначала создали нетерминалы для терминалов, потом создаём вместо 
	одной продукции $X_1X_2\to Y_1Y_2Y_3$ цепочку $X_1X_2 \to Z_1X_2 \to Z_1Z_2 \to Y_1Z_2 \to Y_1Y_2Y_3$.
	Теперь вычислитель "--- это машина Тьюринга с линейной памятью (без док-ва.
	Прпимер не-КС: $\t{0}^n\t{1}^n\t{2}^n$ (не-КС по накачке, а КЗ потому что см. википедию и можно
	сначала построить $\t{0}^n(\t{1}\t{2})^n$, а потом потребовать отсортировать $\t{1}$ и $\t{2}$;
	swap соседних можно выразить через дополнительный нетерминал, чтобы оставаться в КЗ).

	Неограниченные грамматики.
	Слева стоит хотя бы один нетерминал, справа "--- что угодно.
	Языки таких совпадают со всеми перечислимыми языками (см. вопрос 15).
	Пример не-КЗ: язык всех МТ, которые останавливаются.

\section{} % 15
	Для КЗ-грамматик: только $AB\to CD$, $A\to CD$, $A \to a$ (без $\to \epsilon$).
	Иногда разрешают $A \to B$, но от цепных можно избавиться.
	Приведение КЗ в Куроду: из правых частей исключили терминалы, потом разбираем длины
	правил: если не в Куроде, но слева один символ, то просто ввели дополнительные нетерминалы и раскрыли по цепочке.
	А если слева много символов (и справа хотя бы три), то тоже по цепочке, но надо аккуратно смотреть, что требование контекста сохранится
	(или спецсимвол не умрёт).

	Для произвольной грамматики: ещё разрешаем $A \to \epsilon$ для произвольного нетерминала.
	Тогда чтобы привести произвольную в Куроду, добавим ко всем неукорачивающим правилам кучу $Z$
	(где $Z \to \epsilon$), получим неукорачивающую (только скажем, что $Z \to z'$, чтобы формально было аккуратно),
	переделаем её в КЗ, а для КЗ построим Куроду, успех.

	Произвольная грамматика полуразрешается НМТ: можно на ленте иметь текущую строку,
	недетерминированно выбирать подстроку, проверять, заменять, завершаться, если не осталось нетерминалов.
	Обратно: любую НМТ можно эмулировать произвольными грамматиками, это описано в конспекте, примерно как для ассоциативного исчисления доказывали на логике когда-то.
	Т.о. языки произвольных грамматик есть перечислимые.

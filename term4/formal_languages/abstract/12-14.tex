\section{} % 12
	МП-автомат по умолчанию недетерминирован и с $\epsilon$-переходами.
	Стек рисуем вершиной влево, дном вправо, если переход по ребру пихает в стек \t{abc}, то \t{a} оказывается вершиной.
	МП-автомат "--- семёрка из алфавита $\Sigma$, стекового алфавита $\Gamma$, состояний, стартового состояния, терминальных состояний,
	нулевого символа $z_0 \in \Gamma$ и функции перехода $\delta(q, a, \gamma) \to (q', \gamma_1\gamma_2\dots)$ (тут $a$ "--- символ входа или $\epsilon$,
	съедаем со стека $\gamma$, пихаем обратно $\gamma_i$).
	Рисуют переход из $q$ в $q'$ по ребру, на нём пишут $\t{a, \gamma/\gamma_1\gamma_2\dots}$.
	Моментальное состояние "--- тройка из состояния, хвоста непрочитанного входа и стека.
	Можно переходить из одного моментального в другое за один шаг (символ $\vdash$).
	Если стек опустел "--- переходов делать больше нельзя, так как каждый шаг снимает символ со стека.

	МП-автомат принимает $w$ по терминальному состоянию, из из начального состояния $(q_0, w, z_0)$
	можно съесть слово и перейти в терминальное состояние (стек неважен).
	Принимает по пустому магазину, если можно из начального как-то получить пустой стек, съев всё слово (тогда $T$ не нужны).
	Можно передалать терминалы в пустой магазин: заменим одним переходом перед началом работы на стеке символ $z_0$ на $z_0z_0'$
	(чтобы не принять случайно, что не надо), а потом в терминальном состоянии кушаем весь стек.
	Тогда $z_0'$ съедим только из терминального.
	Можно переделать пустой магазин в терминалы: перед началом работы так же заменим $z_0$ на $z_0z_0'$, а потом из каждого
	состояния разрешим, съедая $z_0'$, переходить в новое фиктивное терминальное состояние.

	Палиндромы можно принимать МП-автоматами, недетерминированность нужна, чтобы найти середину.


	Построение КС-грамматики из МП-автомата: \TODO страница 23


\section{} % 13
	Построение МП-автомата из КС-грамматики: \TODO страница 24

\section{} % 14
	Детерминированный МП-автомат: из каждого состояния и символа на стеке: либо есть ровно один $\epsilon$-переход,
	либо есть переходы по каким-то различным буквам (два $\epsilon$ или $\epsilon$ с чем-то нельзя).
	Т.е. у нас действия автомата всегда однозначны.
	Префиксный язык "--- если никакое слово не является префиксом другого.
	Тогда $L$ принимается ДМП по пустому стеку $\iff$ $L$ "--- префиксный и принимается по терминальному состоянию
	(т.е. принимать по пустому стеку не надо, это уменьшает класс языков, поэтому дальше ДМП будут принимать по терминалу).
	$\Ra$: если язык не префиксный, то мы из префикса не сможем продолжить работу и, следовательно, принять более длинное слово,
	значит $L$ префиксный.
	А дальше можно просто переделать как и недетерминированный МП.
	$\La$: если не префиксный, то при приёмке слова мы после встречи одного терминала можем ходить только по $\epsilon$-переходам (иначе примем префикс).
	Добавим фиктивное дно для стека (чтобы он не опустел), а из терминальных удалим все переходы и научим их потреблять стек.

	Регулярные строго вложены в ДМП-автоматы по терминалам.
	Вложение очевидно (взяли старый автомат), строгость: $\t{0}^n\t{1}^n$, он нерегулярен, но ДМП строится (кладём нули, пока не встретим единицу).
	ДМП автоматы строго вложены в КС-языки (так как вложены в МП-автоматы).
	Строгость: \TODO страница 27

\begin{enumerate}
	\item
		Разделим треугольник на четыре равных равносторонних треугольника, соединив середины сторон.
		В вершинах этих треугольников наши точки находиться не могут, так как вершины лежат на периметре большого треугольника.
		Точек пять, что на одну больше, чем маленьких треугольников \rightarrow по принципу Дирихле в одном из маленьких треугольников
		будут лежать две точки. Расстояние между ними будет меньше, чем длина стороны этого треугольника, то есть $0.5$ см,
		что и требовалось.
	
	\item
		Рассмотрим произвольную горизонтальную прямую и три вертикальные прямые такие, что они пересекают
		горизонтальную прямую по узлам одного цвета(это можно сделать, так как точек пересечения
		на прямой бесконечно много, а цветов всего два -- по расширенному принципу Дирихле)
		(этот цвет условно назовём белым для удобства дальнейшего повествования; второй цвет пусть будет красный).
		Теперь будем смотреть на пересечения этих трёх прямых с другими горизонтальными
		(на каждой рассмотренной прямой нас будут интересовать только три точки -- пересечения с нашими вертикальными прямыми).
		
		Если хотя бы на одной прямой из этих трёх точек две -- белые, то мы победили. 
		Иначе рассмотрим четыре горизонтальные прямые: на каждой из них хотя бы две из трёх точек пересечения -- красные.
		Если хотя бы на одной прямой все три точки красные, то мы, опять же, победили, потому что они синхронизируются
		с точками на любой из соседних прямых. Иначе у нас есть только три варианта, как расположить белую точку, и четыре прямые.
		Значит, какой-то вариант повторится -- эту пару прямых мы и возьмём в ответ.

	\item
		От противного: пусть все углы больше $\frac{\pi}{n}$.

		Введём систему координат, связанную с одной из прямых(пусть эта прямая будет выполнять роль оси абсцисс). 
		Отсортируем остальные по увеличению угла между ними и положительным направлением оси абсцисс
		(этот угол не превосходит $\pi$),
		параллельно перенесём их так, чтобы все они пересекались в одной точке -- начале координат.
		Посмотрим на угол между положительным направлением оси абсцисс и последней прямой, 
		пересекающейся с ним под наибольшим углом.
		Этот угол(назовём его $\alpha$) можно получить последовательным сложением углов между соседними прямыми:
		\[\alpha = \sum_{i = 2 \cdots n, i < j} \alpha_{i(i + 1) > \frac{\pi}{n} \cdot (n - 1) = \frac{\pi(n - 1)}{n}}\]

		Заметим, что дополнительный к $\alpha$ угол также больше $\frac{\pi}{n}$(по нашему предположению).
		Это значит, что в сумме
		\[\alpha + \frac{\pi(n - 1)}{n} > \pi\]
		Получили противоречие с тем, что сумма двух дополнительных углов равна $\pi$.

		Вывод: существует угол, не больший $\frac{pi}{n}$.

	\item
		Однозначных чисел: одно.

		Двузначных чисел: $8 + 10 = 18$.

		Трёхзначных: общее количество чисел минус количество чисел без семёрки.
		$9 \cdot 10 \cdot 10 - 8 \cdot 9 \cdot 9 = 252$

		Всего получается $271$.

	\item
		По формуле включений исключений:
		$$
		|A \bigcup B \bigcup C| = 
		= |A \bigcup B| + |C| - |(A \bigcup B) \bigcap C| = 
		= |A| + |B| - |A \bigcap B| + |C| - |(A \bigcap C) \bigcup (B \bigcap C)| = 
		= |A| + |B| + |C| - |A \bigcap B| - |A \bigcap C| - |B \bigcap C| + |A \bigcap B \bigcap C|
		$$

		Осталось показать, что 
		$$
		|A| + |B| + |C| - |A \bigcap B| - |A \bigcap C| - |B \bigcap C| = 
		= - |A| - |B| - |C| + |A \bigcup B| + |A \bigcup C| + |B \bigcup C|
		$$
		Расписав правую часть по формуле включений-исключений, получаем в точности искомое.

	\item
		Заметим, что $\binom{m + n + 1}{n}$ -- это количество $n$-элементных подмножеств $(m + n + 1)$-элементного множества.
		Их можно посчитать иначе: количество подмножеств, в которых обязательно нет последнего элемента($\binom{m + n}{n}$),
		количество подмножеств, в которых обязательно взят последний элемент, но не взят предпоследний($\binom{m + n - 1}{n - 1}$),
		количество подмножеств, в которых обязательно взяты последние два элемента и не взят пред-предпоследний и т.д.
		
	\item
		\begin{enumerable}
			\item
				$$
				\sum_{i = 0}^n i = \sum_{i = 0}^n \binom{i}{1} = \binom{n + 1}{1 + 1} = \frac{n(n + 1)}{2}
				$$
			\item
				Заметим следующее(испльзуем метод двойного подсчёта):
				$$
				\binom{n + 1}{2 + 1} = \sum_{i = 0}^n \binom{i}{2} 
				= \frac{1}{2} \cdot (\sum_{i = 0}^n i^2 - \sum_{i = 0}^n i)
				$$
				Отсюда искомая сумма легко выражается(с учётом первого подпункта):
				$$
				\sum_{i = 0}^n i^2 = \frac{(n - 1)n(n + 1)}{3 \cdot 2} \cdot 2 + \frac{n(n + 1)}{2} 
				= \frac{n(n + 1)(2n + 1)}{6}
				$$
			\item
				Наученные опытом предыдущих подпунктов, отправляемся смотреть сразу на $\sum_{i = 0}^n \binom{i}{3}$:
				$$
				\binom{n + 1}{3 + 1} = \sum_{i = 0}^n \binom{i}{3} = \frac{1}{6}\sum_{i = 0}^n(i^3 - i)

				\sum_{i = 0}^n = \frac{(n - 2)(n - 1)n(n + 1)}{4 \cdot 3 \cdot 2} \cdot 6 + \frac{n(n + 1)}{2}
				= \frac{n(n + 1)(n^2 -3n + 4)}{4}
				$$
		
		\end{enumerable}
\end{enumerate}

\begin{enumerate}
	\item
		\begin{itemize}
			\item
				$\lambda\ x\ z.\ (\lambda\ y. y(y\ y))x\ z$

				Не является редексом на верхнем уровне, но в начале стоит редекс.
			\item
				$\lambda\ x.\ x((\lambda\ y.y)\ x)$
				
				Не нормальная форма, потому что есть где применить редукцию.
		\end{itemize}
	\item
		По данным $m$ и $n$ выдаёт $(m - n)$, применив к $m$ операцию \texttt{pred} $n$ раз:
		$minus = \lambda\ m\ n.\ m\ \texttt{pred}\ n$
		
		Поскольку наша функция \texttt{pred} реализована таким образом, что выдаёт $0$, если ей 
		передали $0$, то можно сразу написать функции \texttt{\\ge} и \texttt{\\le}($m \ge n$ и $m \le n$)
		$$\\ge = \lambda\ m\ n.\ iszero\ (\texttt{minus}\ n\ m)$$
		$$\\le = \lambda\ m\ n.\ iszero\ (\texttt{minus}\ m\ n)$$
		
		Тогда $equals$ получается из двух предыдущих формул -- когда они обе выполнены:
		$$equals = \lambda\ m\ n.\ and\ (\texttt{\\ge}\ m\ n)\ (\texttt{\\le}\ m\ n)$$
		
		Для \texttt{\\lt} и \texttt{\\gt} нам нужно просто сказать, что разность двух чисел $\ge 1$:
		$$\texttt{\\gt} = \lambda\ m\ n.\ \texttt{\\ge}\ (\texttt{minus}\ m\ n)\ 1$$
		$$\texttt{\\lt} = \lambda\ m\ n.\ \texttt{\\ge}\ (\texttt{minus}\ n\ m)\ 1$$		
	\item
		$sum = \lambda\ l. l\ plus\ 0$
		Подставим в такую функцию произвольный список. Получим:
		$$sum (\lambda\ \texttt{c}\ n.\ \texttt{c}\ el_1\ (\texttt{c}\ el_2 \cdots (\texttt{c}\ el_k\ n) \cdots)) = 
		\texttt{plus}\ el_1\ (\texttt{plus}\ el_2 \cdots (\texttt{plus}\ el_k\ 0) \cdots )$$
		что нам и требовалось.

		$length = \lambda l. l (\lambda\ h\ t.\ \texttt{plus}\ 1\ t)\ 0$
		
		Работает так же, как и $sum$, с той лишь разницей, что теперь в операции $plus$ вместо 
		прибавления очередного элемента мы прибавляем единичку; начинаем так же с нуля(длина пустого списка).
	\item
		$tail = 
		\lambda\ l\ c\ n.\ l\ (\lambda\ h\ t.\ iif(\texttt{equals}\ (\texttt{length}\ t)\ (\texttt{pred}\ (\texttt{length}\ l)))\ t\ (\texttt{cons}\ h\ t))$

		То есть на каждом шаге мы будем смотреть на обозримую часть списка как на голову и весь оставшийся хвост
		(традиционное представление) и вернём хвост в тот момент, когда он станет нужной длины, то есть на голову
		короче исходного списка.
	\item
		\begin{itemize}
			\item
				$$
				FM = F
				$$
				$$
				F = \lambda\ m.\ F\ m
				$$
				$$
				F = (\lambda\ f\ m.\ f\ m)\ F
				$$
				$$
				F = Y(\lambda\ f\ m.\ f\ m)
				$$
			\item
				$$
				FM = MF
				$$
				$$
				F = (\lambda\ f\ m.\ m\ f\ m)F
				$$
				$$
				F = Y(\lambda\ f\ m.\ m\ f\ m)
				$$
			\item
				$$
				FMN = NF(NMF)
				$$
				$$
				F = (\lambda\ f\ m\ n.\ n\ f(n\ m\ f))F
				$$
				$$
				F = Y(\lambda\ f\ m\ n)
				$$
		\end{itemize}
	\item
		$$
		g = G\ f\ g
		$$
		$$
		g = Y(G\ f)
		$$

		$$
		f = F\ f\ (Y(G\ f))
		$$
		$$
		f = (\lambda\ x.\ F\ x(Y(G\ x)))f
		$$
		$$
		f = Y(\lambda\ x.\ F\ x(Y(G\ x)))
		$$
		
		Тогда $g$ можно переписать как
		$$
		g = Y(G\ (Y(\lambda\ x.\ F\ x(Y(G\ x)))))
		$$
\end{enumerate}

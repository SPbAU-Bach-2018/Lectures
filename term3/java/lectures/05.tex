\chapter{Внутренние(inner) и вложенные(nested) классы}
Nested class может добираться до статических методов и полей внешнего класса, не связан ни с каким его конкретным экземпляром.

Inner class'ы "--- это:
\begin{itemize}
	\item 
		basic(или "базовые классы")
	\item
		local(или "локальные классы")
	\item
	    	анонимные классы
\end{itemize}
Объект этого класса не может существовать без объекта внешнего класса.
(Например: VectorIterator не может существовать без конкретного Vector'a)
Создание происходит при участии объекта внешнего класса следующим образом: 
\java"objectOfOuterClass.new InnerClassName();".
Если "objectOfOuterClass" опущен, то подразумевается \java"this.new InnerClassName();".

\section{Basic классы}
Посмотрим на пример.
\begin{javacode}
public class Selector {//outer class
	private class SequenceSelected {...} 
	//private -- никто извне не может создать экземпляр этого класса
	//нет слова static, значит класс привязан к конкретному объекту, значит это inner класс
}
\end{javacode}
При этом, так как во внутреннем классе обязательно есть неявная ссылка на объект внешнего класса,
внутреннему классу доступны все поля и методы внешнего класса, в том числе приватные.
\begin{javacode}
public Selector selector() {//возвращаем Selector, так как SequenceSelector имеет модификатор доступа private
	return new SequenceSelector();
}
\end{javacode}

Во что это всё компилируется?
Под каждый класс создастся свой файлик:%c блекджеком и долларом посередине 
\java"Sequence.class", \java"Sequence$Selector.class"


\section{Nested классы}
Статический вложенный класс.
%example04
\begin{javacode}
public class Outer {
	public static class Inner {//лучше бы назвать его nested, ну да ладно
		public static void main(String[] args) {//то есть этот класс можно запустить
			foo();
			Example04 t = new Example04;
			t.boo();//и взяли вызвали приватный метод
		}
	}//это удобно для тестирования приложения:
	//запускать java Example04\$Inner для тестирования,
	//а при распространении(мы распространяем только class-файлы) можно просто удалить файл Example04\$Inner
	/*...*/
}
\end{javacode}
%и не делиться ни с кем, не делиться ничем

Обратите внимание на слово \java"static": вложенный класс не привязан в конкретному объекту.
Если бы не было \java"static", был бы это внутренний класс.
Вот и разница.


Интересный вопрос: а зачем так вообще писать? Сокрытие реализации.
(Многоугольники, сделаем точку nested "--- пользователю точки не понадобятся).

Также к таким классам принято писать модификаторы доступа \java"public"/\java"private" 
(\java"protected" "--- это какая-то вещь в себе).


\section{Анонимные классы}
%EventTest.java
\begin{javacode}
events[i] = new Event() {
	/*...*/
}//и хоть каждый раз -- разный класс
//и используется только здесь, больше не нужен нигде
\end{javacode}
Компилируется в EventTest\$1.class
Но воспользоваться этим файлом ручками мы не можем.
И создавать классы с именем-числом тоже нельзя.

\java"getEvent()" производит нам каждый раз новый класс.
Анонимные классы могут использовать поля-то любые, а вот переменные, которые внутри метода,
можно использовать, только если они \java"final". Почему?
Пусть можно было бы. Создали мы в \java"for"'e один класс с \java"i", потом другой. \java"i" изменилась. 
Надо её менять в первом классе или нет?
Непонятно, вопрос договорённости. Поэтому пусть лучше сразу \java"final" будет.

Анонимные классы используются много где.
Характерный пример "--- сортировка.
\begin{javacode}
class MyComparator implements Comparator<Integer> {
	public int compare(Integer a1, Integer a2) {
		return a2 - a1;
	}
}

public class ArrayTest {
	public static void main(String[] args) {
		Integer[] arr = new Integer[100];
		Random r = new Random();
		for(int i = 0; i < arr.length(); i++) {
			arr[i] = r.nextInt(10000);
		}
/*
		Arrays.sort(arr);
у этой сортировки всё очень хорошо, пока нам не потребовался нестандартный компаратор
		Arrays.sort(arr, new MyComparator);
это долго, нудно и отвратительно
		Arrays.sort(arr, new Comparator<Integer>() {
у анонимного класса нет конструктора(даже имени нет, чего уж там)
есть секция инициализации
			{
				System.out.println("Hello, world!");
			}

			public int compare(Integer a1, Integer a2) {
				return a2 - a1;
			}
		});
*/
		Arrays.sort(arr, (a1, a2) -> a2-a1);
		/*
		lambda -- это синтаксический сахар c Java8.
		у интерфейса компаратор всего один метод, и его мы описываем. 
		Даже типы не пишем, потому что ?компилятор? их сам знает
		*/
	}
}	
\end{javacode}


\section{Локальные(local) классы}
Пишутся внутри метода. Область видимости этим методом и ограничена.


\section{Пример: Singleton}
\begin{javacode}
public class Singleton {
	private Singleton() {}//приватный конструктор

	private static SingletonHolder {
		public static Singletin instance = new Singleton();//можно, так как внутренний класс
	}

	public static Singleton getInstance() {//статический метод, его от класса вызовем
		return SingletonHolder.instance;//поле можем взять, потому что оно статическое, одно на всех
	}
} //это Singleton с ленивой инициализацией

class Singleton2 {//внутри одного файла может быть только один public class
	private Singleton2() {

	}

	private static Singleton2 instance = new Singleton2();//это поле проинициализируется, когда будет первое упоминание Singleton2
	//оно может быть раньше нашей реальной необходимости
	//а SingletonHolder вызовется, только когда его из getInstance попросят, то есть экземпляр сохдастся только тогда, когда нам реально
	//понадобится

	public static Singleton2 getInstance() {
		return instance;
	}
	//можно ещё какой-нибудь foo() приписать
}
\end{javacode}


\section{Пример: FabricMethod}
У нас есть \t{Product}'ы и \t{Creator}.
\t{Creator} "--- абстрактный класс, у которого только один метод "--- \java"getProduct()". 
Почему он абстрактный?  Потому что \t{Product}'ы могут быть разные.

Для \t{ConcreteProduct}'ов "--- \t{ConcreteCreator}'ы.
В \t{ConcreteProductB} запихали его \t{ConcreteCreatorB}, который с помощью анонимного класса создаёт \t{Product}'ы. 
Причём этот \t{Creator} "--- \java"static",
так что он ровно один, и добираться до него легко, так что этот вриант предпочтительнее.


\section{Ещё один хитрый пример}
\begin{javacode}
public interface Node {
	public Node getNext();

	public final static Node NULL = new Node() {//final static -- константа
		//реализация по умолчанию
		public Node getNext() {
			return Node.NULL;//по определению NULL ссылается сам на себя в односвзяном списке
		}
	};
}
\end{javacode}

\section{CurrentSummary}
Типичный пример использования 
\begin{itemize}
	\item
		внутреннего класса "--- итератор
	\item
		анонимного класса "--- сортировка
	\item
		вложенного класса "--- спрятать часть кода спрятать + тестирование
\end{itemize}

\section{Множественное наследование}
Но вот вопрос: как в Java эмулировать множественное наследование?
Вот есть у нас класс \t{UpdateEvent}, в нём есть какая-то начинка:
\begin{itemize}
	\item
		Метод \t{fireEvent}(пробегает по списочку и применяет к каждому элементу какой-то \t{update}).
	\item
		Сам списочек \t{event}'ов, которые надо обновить.
	\item
		Конструктор.
\end{itemize}
А ещё у нас есть классс \t{DrawEvent}, у которого начинка, в сущности, такая же,
только он каждый элемент списочка рисует, а не обновляет.

Хотим себе класс, который умеет и то, и то. 
Технически нам нужен класс, который умеет приводиться и к первому, и ко воторму + 
все их фичи тоже умеет + полиморфизм.

\begin{javacode}
class X {
	private class A2 extends A {

	}

	private class B2 extends B {

	}

	private A a = new A2();//"привестись к типу" ---> вернуть, соответственно, a или b.
	private B b = new B2();//получаем через них полиморфизм: fierEvent в этих классах делает разное:

	public A toA() {return a;}
}
\end{javacode}

\section{Заметки на полях(планы на ближайшие лекции)}
В следующий раз кр на практике.
Базовые вещи, generic'и, коллекции

Презенташка может помочь с домашкой. Сдавать можно на Java8

После следующей лекции будет рассказ Миши Кринкина про приложение на android
На практике будет подразумеваться, что мы начнём своё писать и будем задавать свои вопросы.

На кр можно пользоваться всем, чем угодно -- ноутбуком.
Надо будет написать код.

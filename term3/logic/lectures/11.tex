\subsection{Доказательство теоремы о полноте}
\setauthor{Егор Суворов}
% Теорема о полноте и в:
% $\Gamma$ не противоречиво $\iff$ $\Gamma$ совместно ($\Gamma$ может быть бесконечным)
% Стрелка из непротиворечивости в совместность
\begin{Def}
	$\Gamma$ является полным, если для любой формулы $\phi$, использующей те же переменные, что и $\Gamma$
	либо $\Gamma \vdash \phi$, либо $\Gamma \vdash \lnot \phi$.
\end{Def}
\begin{lemma}\label{consistent_list_extend_to_full}
	Любой непротиворечивый список $\Gamma$ можно расширить до полного непротиворечивого.
\end{lemma}
\begin{proof}
	\begin{assertion}
		Пусть $\Gamma$ непротиворечив и $\Gamma \nvdash \lnot \phi$, тогда
		$\Gamma, \phi$ непротиворечив.
	\end{assertion}
	\begin{proof}
		От противного: пусть противоречиво, т.е. можно вывести и $A$, и $\lnot A$.
		Тогда отсюда знаем, что $\Gamma \vdash \lnot \phi$, противоречие.
	\end{proof}
	Теперь доказываем лемму в конечном/счётном случае.
	Переменных у нас тогда тоже конечно/счётно, соответственно, формул у нас тоже конечно/счётно.
	Добавляем по одной: берём формулу $\phi$, если нельзя вывести ни её, ни отрицание, то добавляем её в $\Gamma$.
	За каждый шаг получаем непротиворечивое.

	Пусть $\Gamma'$ "--- это объединение всех $\Gamma$ по шагам (этакий предел).
	Если оно противоречиво, то противоречие выводится за конечное число шагов, использует конечное число формул,
	т.е. противоречие выводится уже на каком-то конечном шаге алгоритма, чего быть не может.

	Доказательство для несчётного $\Gamma$ пока оставляется \hyperref[zorn_lemma_exmp_list_full]{на конец курса} "--- нам потребуется аксиома выбора/лемма Цорна.
\end{proof}
Теперь доказываем теорему \ref{correct_full_strong} (стрелку $\La$).
Был непротиворечивый $\Gamma$, расширили до полного непротиворечивого $\Gamma'$, теперь для любой переменной $x$ выводится либо $x$, либо $\lnot x$.
Теперь подставим значения: $x=1$, если выводится $x$, и $x=0$, если выводится $\lnot x$.
\begin{assertion}\label{interpret_eq_derivability_non_pred}
	Все выводимые формулы в такой подстановке истинны, а все невыводимые "--- ложны.
\end{assertion}
\begin{proof}
	Индукция по построению формулы.
	База очевидна: это просто переменные.
	Для перехода надо разобраться со связками:
	\begin{itemize}
		\item
			Отрицание.
			Пусть знаем для $\phi$ что её выводимость из $\Gamma'$ эквивалентна истинности в подстановке ($\Gamma' \vdash \phi \iff \phi = 1$).
			Теперь рассмотрим формулу $\lnot \phi$.
			Так как $\Gamma'$ непротиворечиво (обе быть выводимы не могут) и полно (хотя бы одно выводится): $\Gamma' \vdash \lnot \phi \iff \Gamma' \nvdash \phi \iff \phi = 0 \iff \lnot \phi = 1$.
		\item
			Конъюнкция.
			Для доказательства перехода надо доказать:
			$\Gamma' \vdash \phi \land \psi \iff (\Gamma' \vdash \phi) \land (\Gamma' \vdash \psi)$.
			\begin{description}
			\item[$\Ra$:]
				Очевидно, так как есть две аксиомы.
			\item[$\La$:]
				Было правило, переформулированное из аксиомы 3.
			\end{description}
		\item
			Дизъюнкция.
			Надо доказать, что:
			\[
				\Gamma' \vdash \phi \lor \psi \iff
				\left[
					\begin{array}{l}
					\Gamma' \vdash \phi \\
					\Gamma' \vdash \psi
					\end{array}
				\right.
			\]
			\begin{description}
			\item[$\La$:]
				Были аксиомы.
			\item[$\Ra$:]
				Доказательство от противного, интересный случай:
				$\phi \lor \psi$ выводится, а вот $\phi$ и $\psi$ "--- нет (т.е. по полноте $\Gamma$ выводятся $\lnot \phi$ и $\lnot \psi$).
				Давайте покажем противоречивость $\Gamma', \phi \lor \psi$: выведем из неё $\phi$ и $\lnot \phi$.
				Из них $\lnot \phi$ выводится по предположению.
				А дальше есть правило для дизъюнкции, т.е. надо проверить, что $\phi$ выводится в двух случаях:
				\begin{itemize}
					\item $\Gamma', \phi$ "--- берёт и выводится сразу
					\item $\Gamma', \psi$ "--- из $\Gamma'$ выводится $\lnot \psi$, т.е. $\Gamma', \psi$ противоречиво и можно вывести что угодно, в том числе $\phi$.
				\end{itemize}
				Таким образом, $\Gamma', \phi \lor \psi$ противоречиво, что и требовалось.
				Значит, $\phi \lor \psi$ выводится.
			\end{description}
		\item
			Импликация.
			\begin{Exercise}
				\[
				\Gamma' \vdash (\phi \to \psi) \iff
				\left[
					\begin{array}{l}
					\Gamma' \nvdash \phi \\
					\Gamma' \vdash \psi
					\end{array}
				\right.
				\]
			\end{Exercise}
	\end{itemize}
\end{proof}

\subsection{Исчисление предикатов}
Кажется, что-то похожее уже было на прошлой лекции, но всё равно повторим определение:
\begin{Def}
	Предикатная формула называется \textsl{общезначной} (или, реже, \textsl{тавтологией}),
	если она истина во всех интерпретациях и во всех оценках свободных переменных.
\end{Def}
Список аксиом в исчислении предикатов: 11 старых плюс две новые аксиомы
(для кванторов, иначе бы у нас кванторы навечно прилепились к атомарным формулам, что неинтересно,
вывелось бы не всё, что хочется):
\begin{enumerate}
	\setcounter{enumi}{11}
	\item $(\forall x \colon \phi(x)) \to \phi(t|x)$
	\item $\phi(t|x) \to (\exists x \colon \phi(x))$
\end{enumerate}
Здесь $\phi(t|x)$ "--- замена всех вхождений переменной $x$ (разумеется, если какой-то квантор $x$ не перекрыл)
на терм $t$, свободный для подстановки (то есть свободные переменные терма $t$ не попадают в область действия кванторов $\phi$,
куда бы мы не подставили $t$).
\begin{exmp}
	\[ \forall y \colon (P(x, y) \to Q(x, y)) \]
	Подстановка $y|x$ (вместо $x$ подставляем $y$) "--- не свободная, потому что $y$ "--- свободная переменная, а в выражении есть
	квантор по $y$.
\end{exmp}
\begin{exmp}
	\begin{gather*}
		R(x) \land (\forall x \colon (P(x, y) \to Q(x, y))) \\
		\downarrow (y\lor z)|x \\
		R(y \lor z) \land (\forall x \colon (P(x, y) \to Q(x, y))) \\
	\end{gather*}
	Внутрь квантора по $x$ мы не залезаем.
\end{exmp}
\begin{exmp}
	Всегда можно подставлять $t=x$, а также $t=c$, где $c$ "--- новая константа в смысле <<функциональный символ арности ноль>>.
\end{exmp}
Два новых правила вывода (modus ponens остаётся), правила Бернайса:
\begin{enumerate}
	\item
		$\frac{\phi \to \psi}{\phi \to \forall x \colon \psi}$, если $x$ не входит в свободные переменные $\phi$.

		Смысл такой: если у $\psi$ есть параметр $x$, а $\phi \to \phi$ верно независимо от этого параметра, то
		если $\phi$, то $\psi$ тоже верно независимо от этого параметра.
		\begin{exmp}
			Из формулы $(y \lor z) \to (x \land y)$ можно вывести $(y \lor z) \to (\forall x \colon (x \land y))$.
		\end{exmp}

	\item
		$\frac{\phi \to \psi}{(\exists x \colon \phi) \to \psi}$, если $x$ не входит в свободные переменные $\psi$.

		Смысл такой: если у $\phi$ есть параметр $x$, а у $\psi$ его нет и при этом из $\phi$ всегда следует $\psi$,
		то если $\phi$ верна хоть при каком-то параметре, то $\psi$ тоже верна.
		\begin{exmp}
			Из формулы $(x \land y) \to (y \lor z)$ можно вывести $(\exists x \colon (x \land y)) \to (y \lor z)$.
			То есть если хоть при каком-то $x$ будет верно $x \land y$, то $y \lor z$ тоже будет верно.
		\end{exmp}
\end{enumerate}

\begin{Def}
	Формула $\phi$ \textit{выводима}, если существует последовательность предикатных формул $A_1, \dots, A_k$ такая, что:
	\begin{itemize}
	\item $A_k = \phi$
	\item Формула $A_i$ получена одним из двух способов:	
		\begin{enumerate}
		\item Аксиома 1--13, в которую подставлены произвольные формулы вместо переменных
		\item Результат правил вывода для каких-то двух формул $A_a$ и $A_b$, где $a, b < i$.
		\end{enumerate}
	\end{itemize}
\end{Def}
\begin{theorem}[о корректности]
	$\phi$ выводима $\Ra$ $\phi$ общезначна.
\end{theorem}
\begin{proof}
	Для аксиом 1--11 общезначность мы знали, надо проверить для двух новых.
	Это упражнение.
	И еще надо проверить для правил Бернайса:
	\begin{enumerate}
		\item
			Так как $x$ не входит в левую часть и формула сверху общезначна,
			то независимо от значения $x$ верно $\phi \to \psi$, при этом $\phi$ от $x$ вообще не зависит,
			т.е. $\phi \to \forall x \colon \psi$.
		\item
			Аналогично: надо посмотреть на случай, когда $\phi \to \psi$ и $\exists x \colon \phi$.
			Берём этот конкретный $x$, подставляем в $\phi$, отсюда немедленно получаем корректность $\psi$.
	\end{enumerate}
\end{proof}

\begin{exmp}
	Примеры выводимых формул:
	\begin{enumerate}
		\item Все, полученные из пропозициональных тавтологий подстановкой предикатных формул вместо переменных
		\item Если выводима $\phi \to \psi$, то выводима \textsl{контрпозиция}: $\lnot \psi \to \lnot \phi$.
			\begin{proof}
				Такая штука просто является пропозициональной тавтологией: $(\phi \to \psi) \to (\lnot \psi \to \lnot \phi)$.
				Тогда применяем modus ponens и получаем, что надо.
			\end{proof}
		\item $\phi \to \psi$, $\psi \to \tau$ выводимы, тогда $\psi \to \tau$ выводима.
			\begin{proof}
				Была пропозициональная тавтология $(\phi \to \psi) \to ((\psi \to \tau) \to (\phi \to \tau))$,
				применяем два раза modus ponens
			\end{proof}
		\item $(\forall x \colon \phi(x)) \to (\exists x \colon \phi(x))$
			\begin{proof}
				Есть аксиома $(\forall x \colon \phi(x)) \to \phi(x)$,
				есть аксиома $\phi(x) \to (\exists x \colon \phi(x))$,
				дальше по пункту 3.
			\end{proof}
		\item $(\exists y \colon \forall x \colon \phi) \to (\forall x \colon \exists y \colon \phi)$
			\begin{proof}
				Есть аксиома $(\forall x \colon \phi) \to \phi$,
				есть аксиома $\phi \to (\exists y \colon \phi)$,
				по пункту 3 собираем $(\forall x \colon \phi) \to (\exists y \colon \phi)$.

				Теперь применяем правила Бернайса:
				\begin{gather*}
				    \underbrace{(\forall x \colon \phi)} \to \underbrace{(\exists y \colon \phi)} \\
				    \text{второе правило Бернайса для переменной $y$} \\
					(\exists y \colon \overbrace{\forall x \colon \phi}) \to \overbrace{(\exists y \colon \phi)} \\
					\underbrace{(\exists y \colon \forall x \colon \phi)} \to \underbrace{(\exists y \colon \phi)} \\
				    \text{первое правило Бернайса для переменной $x$} \\
					\overbrace{(\exists y \colon \forall x \colon \phi)} \to \forall x \colon \overbrace{(\exists y \colon \phi)} \\
				\end{gather*}
				Что и требовалось.
			\end{proof}
		\item $\lnot (\forall x \colon \phi(x)) \to (\exists \lnot \phi(x))$.
			\begin{proof}
				Воспользуемся контрпозицией.
				Достаточно вывести:
				\[
					\lnot (\exists x \colon \lnot \phi(x)) \to (\forall x \colon \phi(x))
				\]
				Для этого достаточно вывести без квантора $\forall x$ справа "--- его всегда можно добавить при
				помощи первого правила Бернайса (так как $x$ не входит в свободные переменные левой части).
				\begin{gather*}
					\lnot (\exists x \colon \lnot \phi(x)) \to \phi(x) \\
					\Updownarrow \text{применяем контрпозицию} \\
					\lnot \phi(x) \to \left(\exists x \colon \lnot \phi(x)\right)
				\end{gather*}
				А это уже аксиома, получили требуемое доказательство.
			\end{proof}
	\end{enumerate}
\end{exmp}

\begin{Def}
	Пусть $\Gamma$ "--- список замкнутых формул.
	Называем формулу $A$ \textit{выводимой} (и обозначаем $\Gamma \vdash A$),
	если её можно вывести, используя наравне с аксиомами формулы из $\Gamma$ (важно, что они замкнытые).
\end{Def}
\begin{Exercise}\label{a_x_forall_a_x}
	Показать, что если $A(x)$ выводимо, то и $\forall x \colon A(x)$ выводимо.
\end{Exercise}
\begin{Rem}
	Формула $A(x) \to \forall x \colon A(x)$ не является общезначимой: действительно, если положить $A(x) \iff x = 0$,
	то в оценке свободной переменной $x=0$ левая часть верна, а вот правая "--- ложна.
\end{Rem}

\begin{lemma}[о дедукции]
	Если $\Gamma$ "--- список замкнутых формул, $\phi$ замкнута, тогда:
	\[ \Gamma \vdash \phi \to \psi \iff \Gamma, \phi \vdash \psi \]
\end{lemma}
\begin{proof}
	\begin{description}
	\item[$\Ra$:]
		очевидно, так как можно вывести $\phi \to \psi$, а $\phi$ и modus ponens у нас уже  есть.
	\item[$\La$:]
		Берём какой-то вывод: $A_1, \dots, A_k = \psi$.
		Убираем $\phi$ из предпосылок, дописываем $\phi$ ко всем формулам $(\phi \to A_i)$, надо показать, что
		это можно достроить до вывода.
		В случае, когда мы для получения очередной формулы использовали modus ponens всё идёт аналогично
		случаю для пропозициональных формул.
		
		Покажем, что делать в случае, когда использовали первое правило Бернайса для вывода.
		Хотим показать, что вот такое верно (тут $x$ не входит в свободные переменные $A$):
		\[ \frac{\phi \to (A \to B)}{\phi \to (A \to \forall x \colon B)} \]
		То, что сверху, пропозиционально эквивалентно $(\phi \land A) \to B$, отсюда
		по правилу Бернайса $(\phi \land A) \to (\forall x \colon B)$ (так как $\phi$ вообще замкнута),
		что пропозиционально эквивалентно $\phi \to (A \to \forall x \colon B)$.
		\begin{Rem}
			<<Пропозиционально эквивалентно>> "--- взяли пропозициональную аксиому,
			подставили в неё нужные переменные, применили modus ponens.
		\end{Rem}

		Второе правило Бернайса: хотим показать, что вот такое можно достроить до вывода (тут $x$ не входит в свободные переменные $B$):
		\[ \frac{\phi \to (A \to B)}{\phi \to (\exists x \colon A) \to B} \]
		То, что сверху, пропозиционально эквивалентно $A \to (\phi \to B))$,
		применяем второе правило Бернайса:
		\begin{gather*}
			(\exists x \colon A) \to (\phi \to B) \\
			\text{пропозиционально эквивалентно} \\
			\phi \to ((\exists x \colon A) \to B) \\
		\end{gather*}
	\end{description}
\end{proof}

Свойства выводимости из списка:
\begin{enumerate}
	\item Если $\Gamma \vdash A$, $\Gamma' \supseteq \Gamma$, то $\Gamma' \vdash A$.
	\item Если $\Gamma \vdash A$, то существует конечное $\Gamma' \subseteq \Gamma$ такое, что $\Gamma' \vdash A$ (так как в выводе участвует лишь конечное число формул).
	\item $\gamma_1, \gamma_2, \dots, \gamma_n \vdash A$ (все $\gamma_i$ замкнуты), тогда выводимы формулы
		$\gamma_1 \to (\gamma_2 \to (\dots (\gamma_n \to A)\dots))$
		и
		$(\gamma_1 \land \gamma_2 \land \dots \land \gamma_n) \to A$.
\end{enumerate}

\subsection{Полнота исчисления предикатов в сильной форме}
\begin{lemma}[о свежих константах]\label{fresh_const_lemma}
	Пусть $\Gamma \vdash \phi(c|x)$, где $c$ "--- некоторая константа (нульместный функциональный символ), не входящая ни в $\phi$, ни в $\Gamma$.
	Тогда $\Gamma \vdash \phi$.
\end{lemma}
\begin{proof}
	Рассмотрим вывод $\phi(c|x)$.
	Заменим в выводе $c$ на $y$, где $y$ "--- какая-то новая свободная переменная, получили корректный вывод формулы $\phi(y|x)$.
	Потом по упражнению \ref{a_x_forall_a_x} выведем отсюда:
	\[ \forall y \colon \phi(y|x) \]
	Теперь берём аксиому 12:
	\[ (\forall y \colon \phi(y|x)) \to \phi(x|(y|x)) \]
	Теперь применяем modus ponens, получаем просто $\phi$, что и требовалось.
\end{proof}

\begin{lemma}[о добавлении константы]
	Пусть $\phi$ "--- формула сигнатуры $\sigma$, и она выводима в сигнатуре $\sigma'$, которая отличается от $\sigma$ добавлением константы $c$,
	не содержащейся ни в $\phi$, ни в $\sigma$.
	Тогда $\phi$ выводима и в $\sigma$.
\end{lemma}
\begin{proof}
	Возьмём эту новую константу $c$, она где-то в выводе встречается.
	Заменим её на какую-нибудь новую переменную $y$.
	Заметим, что вывод всё еще остался корректным, причём в конце мы всё еще получим $\phi$
	(так как эта константа в $\phi$ не содержалась).
\end{proof}

\begin{Def}
	Пусть $\Gamma$ "--- список замкнутых формул.
	Тогда интерпретация $I$ называется \textsl{моделью} для $\Gamma$, если
	она выполняет все формулы из $\Gamma$.

	Напоминаем, что интерпретация "--- это множество-носитель интепретации плюс
	задание всех операций из сигнатуры на этом носителе.
\end{Def}
\begin{theorem}[о полноте в сильной форме]
	Список формул $\Gamma$ непротиворечив $\iff$ $\Gamma$ имеет модель.
\end{theorem}
\begin{proof}
	\begin{description}
		\item[$\La$:]
			по ходу вывода модель сохраняется (то есть все новые формулы всё еще выполняются):
			надо это проверить для всех аксиом, потом для правил вывода.
		\item[$\Ra$:]
			\begin{Def}
				$\Gamma$ "--- \textsl{полный}, если для любой замкнутой формулы $\phi$
				либо $\Gamma \vdash \phi$, либо $\Gamma \vdash \lnot \phi$.
			\end{Def}
			\begin{lemma}
				Непротиворечивый $\Gamma$ можно непротиворечиво расширить до полного списка.
				Доказательство аналогично доказательству леммы \ref{consistent_list_extend_to_full} (у нас тогда
				еще не было предикатов, но они ничего не портят).
			\end{lemma}
			\begin{Def}
				Список $\Gamma$ "--- \textsl{экзистенциально полный}, если для любой замкнутой формулы $\phi$
				верно, что если $\Gamma \vdash (\exists x \colon \phi)$, то найдётся замкнутый терм $t$ такой,
				что $\Gamma \vdash \phi(t|x)$.
			\end{Def}
			\begin{lemma}
				Непротиворечивый $\Gamma$ можно непротиворечиво расширить до полного и экзистенциально полного списка.
			\end{lemma}
			\begin{proof}
				Сначала продолжим $\Gamma$ до полного $\Gamma'$.
				Теперь выводится любая формула или её отрицание, но экзистенциальной полноты может не быть.

				Давайте если $\Gamma' \vdash (\exists x \colon \phi)$, то добавим формулу $\phi(c)$, где $c$ "--- новая константа.
				Противоречивость не образуется: если образуется, то из некоторой формулы $\gamma$ (конечная конъюнкция формул из $\Gamma'$,
				участвующих в выводу) было выводимо $\lnot \phi(c)$,
				а по лемме \ref{fresh_const_lemma} (о свежих константах) получаем $\gamma \to \lnot \phi$.
				Делаем контрапозицию, получаем $\phi \to \lnot \gamma$.
				Применяем правило Бернайса для переменной $x$ (формула $\gamma$ замкнутая), получаем $(\exists x \colon \phi) \to \lnot \gamma$.
				Получаем, что множество $\Gamma'$ было противоречиво изначально.

				Дальше будем на одном шаге делать экзистенциальное пополнение, потом обычное, и так чередуем, получаем какую-то последовательность вложенных $\Gamma'_i$.
				Возьмём их объединение, получим искомый список.
				В самом деле: он непротиворечив (потому что противоречие выводится из конечного шагов и из конечного числа $\Gamma'_i$),
				он полон (так как каждая формула $\phi$ содержит конечное число констант, которые мы добавляем в $\Gamma'$ и тогда на каком-то
				шаге она или её отрицание станут выводимы), экзистенциально полно по той же причине.
			\end{proof}

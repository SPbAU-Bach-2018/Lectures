Свойства начальных отрезков:
\begin{enumerate}
	\item
		Начальный отрезок в.у.м. сам является в.у.м.
	\item
		Начальный отрезок начального отрезка "--- начальный отрезок исходного в.у.м.
	\item
		Объединение любого семейства начальных отрезков "--- начальный отрезок.
	\item
		\begin{Def}
			$[0, x) \eqDef \{ y \in M \mid y < x\}$
		
			$[0, x] \eqDef \{ y \in M \mid y \le x \}$
		\end{Def}
		Эти два множества являются начальными отрезками.
	\item
		Для любого начального отрезка $A$, не совпадающего со всем в.у.м., существует $x$ такой,
		что отрезок совпадает с $[0, x)$.
		\begin{proof}
			Возьмём $x = \min \{ y \notin A \}$
		\end{proof}
	\item
		Возьмём любые два начальных отрезка $A$ и $B$.
		Тогда один целиком лежит в другом.
\end{enumerate}

\section{Трансфинитная индукция}
Сейчас будем вести индукцию по произвольному вполне упорядоченному множеству.
Например:

\begin{theorem}
	Пусть $A$ "--- в.у.м. и $f \colon A \to A$ "--- строго монотонно возрастающая функция ($x < y \Ra f(x) < f(y)$).
	Тогда $f(x) \ge x$ для всех $x$.
\end{theorem}
\begin{proof}
	Докажем индукционных переход для элемента $y$.
	Мы предположили, что $\forall y < x \colon f(y) > y$.
	Тогда покажем, что $f(x) \ge x$ от противного: пусть $f(x) < x$.
	По индукционному предположению имеем $f(f(x)) \ge f(x)$.
	С другой стороны, по монотонности имеем $f(f(x)) < f(x)$.
	Противоречие.
\end{proof}

\begin{theorem}[о трансфинитной рекурсии]
	Пусть $A$ "--- в.у.м., $B$ "--- любое множество.
	Пусть $F$ "--- некоторая функция, которая по элементу $x \in A$ и
	фукции $g \colon [0, x) \to B$ выдаёт элемент $B$.
	Тогда существует единственная $f \colon A \to B$ такая, что:
	\[ \forall x \in A \colon f(x) = F(x, f\mid_{[0; x)}) \]
\end{theorem}
\begin{Rem}
	Индукция "--- это доказательство утверждений, а рекурсия "--- вычисление значений функции, зная
	значения на предыдущих элементах.
\end{Rem}
\begin{proof}
	Будем говорить, что функция $f$ \textit{корректна} на $[0, a]$, если:
	\[ \forall c \le a \colon f(c) = F(c, f\mid_{[0, c)}) \]
	Трансфинитной индукцией по $a$ будем доказывать, что существует единственная корректная функция на $[0, a]$.
	Заодно поймём, что они согласованы: если $f_{c_1}$ корректна на $[0; c_1]$, а $f_{c_2}$ корректная на $[0; c_2]$,
	а также $c_1 < c_2$, то $f_{c_2}\mid_{[0;c_1]} = f_{c_1}$ (следует из единственности).
	Тогда можно будет определить искомую $f$ так:
	\[ f(x) = f_x(x) \]
	(берём единственную $f_x$, корректную на $[0; x]$ и берём её значение в $x$).

	Итак, доказываем.
	Пусть доказали существование единственных корректных для всех меньших $a$, хотим показать единственную корректную $f_a$ на $[0; a]$.
	Давайте подберём $h \colon [0, a] \to B$, согласованную со всеми корректными $f_c$ ($c < a$):
	\begin{gather*}
		h(x) = f_x(x), \text{где $f_x$ корректна на~}[0; x] \\
		h(a) = F(a, h\mid_{[0;a)})
	\end{gather*}
	Значения в предыдущих точках определены однозначно.
	Значит, $h$ на $[0; a)$ определена однозначно.
	То есть $h(a)$ тоже однозначно из-за условия с $F$.
\end{proof}

\begin{theorem}[о трансфинитной рекурсии для частично определённых функций]
	Пусть $A$ "--- в.у.м., $B$ "--- любое множество.
	Пусть $F$ "--- некоторая \textit{частичная} функция (которая определена не везде),
	которая по элементу $x \in A$ и фукции $g \colon [0, x) \to B$ выдаёт элемент $B$.
	Тогда ровно одно из двух:
	\begin{enumerate}
		\item
			Существует единственная всюду определённая $f \colon A \to B$ такая, что:
			\[ \forall x \in A \colon f(x) = F(x, f\mid_{[0; x)}) \]
		\item
			Существует единственное $a \in A$ такое, что есть единственная всюду определённая $f \colon [0; a) \to B$ такая, что:
			\[ \forall x < a \colon f(x) = F(x, f\mid_{[0; x)}) \]
			а функция $F(a, f\mid_{[0; a)})$ не определена.
	\end{enumerate}
\end{theorem}
\begin{proof}
	Положим $B' = B \cup \{ \bot \}$ ($\bot$ "--- <<неопределённость>>).
	Скажем, что $F$ теперь определена везде, а где не была определена раньше "--- выдаёт $\bot$.
	Скажем, что если второй параметр $F$ где-то принимает $\bot$, то результат "--- тоже $\bot$.
	Теперь пользуемся старой теоремой.
	\begin{Rem}
		Тут, наверное, надо сказать, что введение такого специального значения $\bot$
		равносильно обычной неопределённости.
	\end{Rem}

	Если результатирующая $f$ везде определена, то мы получили вариант 1.
	А если где-то неопределена, то рассмотрим минимальное место, где она неопределена "--- точку $a$.
\end{proof}

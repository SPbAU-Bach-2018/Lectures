\setauthor{Надежда Бугакова}
\section{Вступление}
В этом семестре курс будет состоять из трёх частей:
\begin{enumerate}
	\item Вычислимость (теория алгоритмов)
	\item Вычисление предикатов
	\item Теория множеств
\end{enumerate}
Рекомендуется для прочтения: Н. К. Верещагин, А. Шень. "Лекции по математической логике и теории алгоритмов".
Читать в следующем порядка:
\begin{enumerate}
	\item
		Часть 3. Вычислимые функции.

		\url{http://www.mccme.ru/free-books/shen/shen-logic-part3-2.pdf}

	\item
		Часть 2. Языки и исчисления.

		\url{http://www.mccme.ru/free-books/shen/shen-logic-part2-2.pdf}

	\item
		Часть 1. Начала теории множеств

		\url{http://www.mccme.ru/free-books/shen/shen-logic-part1-2.pdf}
\end{enumerate}
Это не ошибка: от третьей части к первой.
Потому что гладиолус :)

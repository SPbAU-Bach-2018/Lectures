\setauthor{Егор Суворов}

\section{Начало лекции от 5 октября}
\subsection{Трюк Гёделя}

Давайте посмотрим на предикат $x=y^z$.
Наивное выражение этого предиката может выглядеть вот так:
\[ \exists t_1, t_2, \dots t_z (t_1 = y \land t_z = x \land \forall i \colon ((i < z) \Ra t_{i+1}=t_iy) \]
Это мы просто задали последовательность $t_i$, начинающуюся с $y$, заканчивающуюся на $x$ и
с фиксированным отношением между соседними членами.
Но, вообще говоря, так писать нельзя "--- это выражение не является строкой, так как его длина неопределена
и зависит от $z$.

Год назад мы боролись с этой проблемой, кодируя множества натуральными числами и вводя
предикат <<число принаджлежит множеству>>.
Сейчас рассмотрим другой подход, подход Гёделя.
Сначала нужно несколько лемм:

\begin{lemma}
	$\forall k \in \N \colon \exists$ сколь угодно большое натуральное $b$ такое,
	что $b+1$, $2b+1$, \dots, $kb+1$ являются попарно взаимно простыми.
\end{lemma}
\begin{proof}
	Пусть $i, j \in [1, k]$.
	Тогда пусть $p$ "--- такое простое, что $ib+1, jb+1 \vdots p$.
	Тогда $|i-j|b \vdots p$ и, так как $b \not\vdots p$, $|i-j| \vdots p$
	Возьмём $b \vdots k!$, тогда $b \vdots |i - kj| \Ra b \vdots p$, то есть получим противоречие.
	Значит, любое $b \vdots k!$ подходит в условие леммы.
\end{proof}

\begin{lemma}
	Для любой последовательности $x_0$, $x_1$, \dots, $x_n$ $\in \N$ найдутся такие
	$a, b \in \N$, что:
	\begin{align*}
		x_0 &\equiv a \mod (b + 1) \\
		x_1 &\equiv a \mod (2b + 1) \\
		\vdots & \vdots \\
		x_n &\equiv a \mod ((n+1)b + 1) \\
	\end{align*}
\end{lemma}
\begin{proof}
	По лемме 1 выберем такое $b > \max x_i$, что $b+1$, \dots $(n+1)b+1$ попарно взаимно просты.
	Тогда по китайской теореме об остатках существует решение $a$ системы:
	\begin{align*}
		a &\equiv x_0 \mod (b + 1) \\
		a &\equiv x_1 \mod (2b + 1) \\
		\vdots & \vdots \\
		a &\equiv x_n \mod ((n+1)b + 1) \\
	\end{align*}
	Оно и будет вторым требуемым числом.
\end{proof}

Возвращаясь к примеру $x=y^z$ можно закодировать последовательность $t_1, \dots, t_z$.
Вообще говоря, оператора $\mod$ у нас в строках-предикатах нет, но мы его выводили, можем подставить.
\TODO, не забыть указать длину
Трюк, кстати, называется, $\beta$-функцией Гёделя.

\subsection{Перечислимость и выразимость в арифметике}

\begin{Def}
	Предикат $P \colon \N \to \{0, 1\}$ является перечислимым, если $\{ x \mid P(x) = 1 \}$ является перечислимым.
\end{Def}
\begin{Def}
	Предикат $P \colon \N \to \{0, 1\}$ является разрешимым, если $\{ x \mid P(x) = 1 \}$ является разрешимым.
\end{Def}

\begin{theorem}
	Любой перечислимый предикат является арифметическим (выразим в арифметике).
\end{theorem}
\begin{proof}
	Сначала выберем удобную модель.
	Для машин Тьюринга можно, но гораздо удобнее будет оперировать программами с конечным числом переменных
	(они ведь уже оперируют с натуральными числами). % Да ладно? А не с целыми хотя бы?
	Мы знаем, что так как $P$ перечислим, то по одному из определний перечислимости существует
	полуразрешающий алгоритм, то есть программа с конечным числом
	переменных $S$ такая, что $P(x) = 1 \iff S(x)$ останавливается.

	Пусть имеется какая-то программа.
	Можно считать, что в ней ровно одна команда остановки программы (\t{stop}), потому что остальные
	можно заменить на \t{goto} к какому-нибудь \t{stop}.
	С концом программы тоже никаких проблем: \TODO

	Теперь давайте запишем следующее утверждение: существует некоторая последовательность
	состояний программы длины $T$, начинающаяся в изначальном состоянии и заканчивающаяся в единственном
	\t{stop}, причём переходы между соседними состояниями корректны.
	Последовательность состояний "--- это просто несколько последовательностей, соответствующих переменным
	и номеру текущей команды.
	Запишем в <<некорректной>> форме (с неопределённой длиной строки), по трюку Гёделя от этого легко избавиться:
	\begin{enumerate}
		\item Существует последовательность состояний:
			\[
			\exists T;
			\exists a_1^{(1)}, \dots, a_1^{(T)};
			\exists a_2^{(1)}, \dots, a_2^{(T)};
			\dots;
			\exists m^{(1)}, \dots, m^{(T)}
			\colon
			\]
		\item Начинаем где надо ($S_1$ "--- номер первой строчки):
			\[
			(a_1^{(1)} = x) \land (a_2^{(1)} = 0) \land \dots \land (a_m^{(1) = 0}) \land (m^{(1)} = S_1)
			\]
		\item И заканчиваем где надо:
			\[
			m^{(T)}=S_{stop}
			\]
		\item А также все шаги корректны:
			\[
			\forall i \colon (i < T) \land (i \ge 1) \Ra Step(
			a_1^{(i)}, a_2^{(i)}, \dots, a_k^{(i)}, m^{(i)},
			a_1^{(i+1)}, a_2^{(i+1)}, \dots, a_k^{(i+1)}, m^{(i+1)},
			)
			\]
	\end{enumerate}
	Осталось только написать предикат $Step$.
	Он является просто разбором случаев: берём логическое <<и>> условий для всех строк,
	для строки $S_x$ условие выглядит как $(m^{(i)} = S_x) \Ra (\dots)$, внутри скобок
	пишем условие на корректность перехода.
	\begin{enumerate}
		\item Если идёт изменение переменной, то оно выразимо в арифметике, остальные поменяться не должны,
			номер следующей строки тоже выразим.
		\item Если идёт переход, то переменные не поменялись, номер следующей строки тоже выразим.
		\item В случае с условным оператором вида \t{if (A) then B else C} можно записать $($
	\end{enumerate}
\end{proof}

\begin{conseq}
	График любой вычислимой функции $f \colon A \to \N$ выразим в арифметике.
	Напоминание: $\Gamma_f = \{ (x, f(x)) \mid x \in A \}$.
\end{conseq}
\begin{conseq}
	Любой разрешимый предикат выразим в арифметике (просто потому что он перечислим).
\end{conseq}

\section{Арифметическая иерархия}
\subsection{О предварённой форме}
\begin{Def}
	Формулы $\phi_1$ и $\phi_2$ эквивалентны, если у них одинаковое количество свободных переменных,
	одинаково именованные свободные переменные и во всех интерпретациях и во всех оценках
	свободных переменных их значения совпадают.

	Напоминание: оценка свободных переменных "--- это подстановка значений,
	а интерпретация "--- это когда мы задаём, какие у нас символы какие предикаты обозначают
	(например, что обозначает \t{<} и \t{>}).
\end{Def}

\begin{lemma}
	Для любой формулы можно построить эквивалентную ей, у которой все кванторы ($Q_i$) стоят в начале:
	\[
	Q_1 x_1 Q_2 x_2 \dots Q_n x_n \colon \phi(x_1, \dots, x_n, x_{n+1}, \dots x_m)
	\]
	(при этом в $\phi(\dots)$ кванторов нет).
\end{lemma}
\begin{proof}
	Давайте научимся выносить очередной квантор в начало формулы.
	\begin{itemize}
	\item $\lnot (\forall x \phi)$ эквивалентно $\exists x (\lnot \phi)$.
	\item $\lnot (\exists x \phi)$ эквивалентно $\forall x (\lnot \phi)$.
	\end{itemize}
	А вот дальше сложнее, потому что у нас некоторые связные переменные могут совпадать
	по именам с другими связными или свободными, заменять $(\exists x \colon \phi) \lor \psi$
	на $\exists x \colon (\phi \lor \psi)$ нельзя.
	Поэтому сначала надо переименовать все связные переменные так, чтобы имена вообще всех
	встречающихся переменных были попарно различны.
	А потом можно применять правила (первые два для отрицания уже были), вот еще одно:
	\[ (\exists x \phi) \lor \psi \sim \exists x (\phi \lor \psi) \]
	Для других операций и кванторов аналогично.
\end{proof}

\subsection{Иерархия Клини}
Еще называется арифметической иерархией.

Мы классифицируем предикаты по разным классам, а точнее "--- в классы $\Sigma_i$ и $\Pi_i$.
Начинаем с $\Sigma_0=\Pi_0$ "--- множество разрешимых предикатов.
\begin{Def}
	$P(x_1, \dots, x_m) \in \Sigma_1$, если существует разрешимый предикат $Q(y, x_1, \dots x_m)$
	(то есть предикат из $\Sigma_0 = \Pi_0$) такой, что
	\[
	\forall x_1, x_2, \dots x_m \colon P(x_1, \dots, x_m) = \exists y Q(y, x_1, \dots, x_m)
	\]
\end{Def}
\begin{Rem}
	$\Sigma_1$ есть множество перечислимых предикатов.
\end{Rem}
\begin{proof}
	В самом деле: для любого перечислимого есть алгоритм, который на шаге $y$ выдаёт какой-то
	вход, на котором предикат верен, тогда этот алгоритм можно <<обозначить>> за предикат $Q$,
	который уже разрешим (ведь алгоритм выполняет не более $y$ шагов).
	И в обратную сторону: если есть $Q$, то просто проверяем по очереди все $y$.
\end{proof}

\begin{Def}
  $P(x_1, \dots, x_m) \in \Pi_1$, если существует разрешимый предикат $Q(y, x_1, \dots x_m)$
  (то есть предикат из $\Sigma_0 = \Pi_0$) такой, что
  \[
  \forall x_1, x_2, \dots x_m \colon P(x_1, \dots, x_m) = \forall y Q(y, x_1, \dots, x_m)
  \]
  $\Sigma_1$ есть множество коперечислимых предикатов (у которых дополнение перечислимо).
\end{Def}

\TODO проверить определение по конспекту
\begin{Def}
	$P(x_1, \dots, x_m) \in \Sigma_{k+1}$ если $\exists Q(y, x_1, \dots, x_m) \in \Pi_k$ такой,
	что
	\[
	\forall x_1, x_2, \dots x_m \colon P(x_1, \dots, x_m) = \exists y Q(y, x_1, \dots, x_m)
	\]
\end{Def}
Аналогичное определение для $\Pi_{k+1}$:
\begin{Def}
	$P(x_1, \dots, x_m) \in \Pi_{k+1}$ если $\exists Q(y, x_1, \dots, x_m) \in \Sigma_k$ такой,
	что
	\[
	\forall x_1, x_2, \dots x_m \colon P(x_1, \dots, x_m) = \forall y Q(y, x_1, \dots, x_m)
	\]
\end{Def}
Можно определить нерекурсивно, там получится чередование кванторов.

Пока что это всё является непонятными формальными значками, это нормально.
Скоро разберёмся, какой у всего этого смысл.
Начнём с каких-нибудь свойств:

\begin{enumerate}
\item
	\[ \forall k \ge 0 (\Sigma_k \cup \Pi_k) \subseteq \Sigma_{k+1} \cap \Pi_{k+1} \]
	Надо показать, что каждое из множеств слева лежит в каждом из множеств справа.

	\begin{proof}
		\begin{itemize}
		\item $\Sigma_k \subseteq \Sigma_{k+1}$ "--- просто добавляем фиктивную переменную. \TODO
		\item $\Sigma_k \subseteq \Pi_{k+1}$ "--- добавляем фиктивную переменную в начало. \TODO
		\end{itemize}
	\end{proof}

\item
	$\Pi_k$ "--- это множество дополнений предикатов из $\Sigma_k$.
	\begin{proof}
		Можно по индукции и $k$, можно просто взять отрицание предиката, кванторы поменяются на противоположные.
	\end{proof}

\item
	Любой арифметический предикат содержится в некотором уровне иерархии.
	\begin{proof}
		Берём соответствующую ему арифметическую формулу, приводим в предварённую
		форму (выносим все кванторы вперёд).
		Формула, конечно, поменяется, но ничего страшного: у нас каждый предикат может иметь несколько
		форм записи, это нормально.

		Проверку истинности безкванторного остатка можно сделать алгоритмом (так как там только
		арифметические и логические операции).
		Получили почти определение класса, только без чередования кванторов.
		Можно либо добавить фиктивных кванторов над фиктивными неиспользуемыми переменными,
		либо <<запихнуть>> в одну переменную несколько.
	\end{proof}

\item
	Любой предикат из $\Sigma_k$ арифметический.
	\begin{proof}
		Тут лекция закончилась.
	\end{proof}
\end{enumerate}

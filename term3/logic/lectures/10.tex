\subsection{В комплексных числах}
\setauthor{Егор Суворов, Анастасия Старкова}
\begin{theorem}
	В $(\C, =, +, \cdot, 0, 1)$ допустима элиминация кванторов
\end{theorem}
\begin{proof}
	Атомарные формулы имеют вид $P(x_1, \dots, x_n)=0$.
	Назовём множество <<хорошим>>, если оно представляется в виде такого объединения:
	\[
		\bigcup_i
			\begin{cases}
				P_{i1} = 0, \\
				P_{i2} \neq 0, \\
				P_{i3} = 0, \\
				P_{i4} \neq 0, \\
				\vdots \\
			\end{cases}
	\]
	Будем действовать аналогично доказательству теоремы Тарского-Зайденберга.
	Нам потребуется доказать, что проекция хорошего множества на любую из осей координат также является хорошим,
	после этого мы сможем стандартными трюками (замена $\forall$ на $\exists$ и элиминация кванторов, начиная с самого вложенного)
	доказать текущую теорему.

	Также аналогичным образом вводим диаграммы, но в этот раз в диаграмму включаем только корни многочленов (потому что никаких <<промежуточных значений>> нет),
	состояния клеток "--- <<равно нулю>> и <<не равно нулю>>.
	Корни теперь можно записывать в любом порядке.
	Более того, мы для любого многочлена знаем, сколько корней нужно добавить: ровно его степень минус количество уже существующих.
\end{proof}

\section{Гильбертовское исчисление высказываний и предикатов}
Смысл происходящего: сейчас хотим чисто синтаксическими преобразованиями научиться выводить все истинные формулы.
Сначала "--- предикатные, но это при желании расширается.

\begin{Def}
	Предикатная формула называется \textsl{тавтологией}, если она истинна в любой интерпретации.
\end{Def}
\begin{Rem} % возможно, это всё-таки определение "выводится из аксиом"?
	Пусть есть набор аксиом (возможно, бесконечный): $A_1, \dots, A_i, \dots$ и формула $\phi$.
	Тогда естественно считать, что тавтологичность $(\Land A_i) \to \phi$ эквивалентна тому, что $\phi$ выводится из аксиом (здесь и далее $\to$ "--- следствие, являющееся частью предикатной формулы).
\end{Rem}

\begin{theorem}
	Если в сигнатуре достаточно много предикатных и функциональных символов, то множество тавтологий этой
	сигнатуры неразрешимо.
\end{theorem}
\begin{proof}
	\TODO на практике (что-то про ассоциативность)
\end{proof}

\begin{theorem}[Гёделя о полноте исчисления предикатов]
	Для любой перечислимой сигнатуры множество тавтологий перечислимо.
\end{theorem}
\begin{Rem}
	К этой теореме мы будем долго идти и в конце концов докажем более сильное утверждение.
\end{Rem}

\begin{Def}
	Правило вывода в Гильбертовском исчислении высказываний (modus ponens):
	\[ \frac{A \to B, A}{B} \]
	Читать так: если есть формулы $A \to B$ и $A$, то можно вывести формулу $B$.
\end{Def}
%\begin{Rem}
%	Какая-то шутка, которую мы когда-нибудь поймём: <<женский>> вариант правила:
%	\[ \frac{A \to B, \text{очень хочется }B}{A} \]
%\end{Rem}

Список аксиом (их много, но на экзамене можно будет сделать шпаргалку с ними):
\begin{enumerate}
\item $A \to (B \to A)$ "--- истину всегда можно вывести
\item $(A \to (B \to C)) \to ((A \to B) \to (A \to C))$ "--- некоторая транзитивность: если при условии $A$ из $B$ можно вывести $C$, то если выведем $B$ из $A$, то и $C$ из $A$ тоже выведем.
\item $(A \land B) \to A$
\item $(A \land B) \to B$
\item $A \to (B \to (A \land B))$ "--- при условии, что верно и $A$, и $B$, верно $A \land B$.
\item $A \to (A \lor B)$
\item $B \to (A \lor B)$
\item $(A \to C) \to ((B \to C) \to ((A \lor B) \to C))$ "--- если $C$ выводится и из $A$, и из $B$, то выводится из $A \lor B$.
\item $\lnot A \to (A \to B)$ "--- из лжи следует всё, что угодно
\item $(A \to B) \to ((A \to \lnot B) \to \lnot A)$ "--- из $A$ не могут одновременно следовать и $B$, и $\lnot B$
\item $A \lor \lnot A$
\end{enumerate}
Все эти формулы являются тавтологиями.

\begin{Def}
	Формула $\phi$ \textit{выводима}, если существует последовательность формул $C_1, \dots, C_k$ такая, что:
	\begin{itemize}
	\item $C_k = \phi$
	\item Формула $C_i$ получена одним из двух способов:	
		\begin{enumerate}
		\item Аксиома, в которую подставлены произвольные формулы вместо переменных
		\item Результат modus ponens для каких-то двух формул $C_a$ и $C_b$, где $a, b < i$.
		\end{enumerate}
	\end{itemize}
\end{Def}

\begin{theorem}[корректность]
	Если $\phi$ выводима, то $\phi$ "--- тавтология.
\end{theorem}
\begin{proof}
	Все аксиомы тавтологичны: надо просто рассмотреть таблицу истинности для каждой из них.
	Так как от интерпретации у нас зависят только изначальные значения переменных, то этого хватит.

	Использование modus ponens позволяет нам из двух тавтологий получить некоторую третью формулу, которая
	тоже является тавтологий (опять же, рассмотрим таблицу истинности), что и требовалось.
\end{proof}

\begin{lemma}\label{aToA}
	$\mcA \to \mcA$ выводима
\end{lemma}
\begin{proof}
	Рассмотрим вторую аксиому, подставим в неё $A=\mcA$, $C=\mcA$, $B=\mcA \to \mcA$:
	\[ (\underbrace{\mcA}_A \to (\underbrace{(\mcA \to \mcA)}_B \to \underbrace{\mcA}_C)) \to ((\underbrace{\mcA}_A \to \underbrace{(\mcA \to \mcA)}_B) \to (\underbrace{\mcA}_A \to \underbrace{\mcA}_C)) \]
	Теперь отдельно рассмотрим первую аксиому, подставим $A=\mcA$ и $B=\mcA \to \mcA$:
	\[ \underbrace{\mcA}_A \to (\underbrace{(\mcA \to \mcA)}_B \to \underbrace{\mcA}_A) \]
	Теперь по modus ponens соединим эти две формулы, получим:
	\[ (\mcA \to (\mcA \to \mcA)) \to (\mcA \to \mcA) \]
	Теперь заметим, что первая половина этой формулы "--- аксиома 1 с подставленными $A=B=\mcA$, то есть тавтология.
	Применяем modus ponens еще раз:
	\[ \mcA \to \mcA \]
\end{proof}

\subsection{Натуральный вывод}
Пусть $\Gamma$ "--- список пропозициональных формул.
\begin{Def}
	Говорим, что $\Gamma \vdash \phi$ (\textsl{$\phi$ выводима из $\Gamma$}), если
	есть конечная последовательность формул $C_1, \dots, C_k$ такая, что $C_k=\phi$ и каждое $C_i$ получено одним из способов:
	\begin{itemize}
		\item $C_i \in \Gamma$
		\item $C_i$ получается из аксиом произвольной подстановкой
		\item $C_i$ получается по modus ponens из каких-то предыдущих $C_j, C_k$ ($j, k < i$).
	\end{itemize}
\end{Def}

\begin{lemma}[о дедукции]
	$\Gamma \vdash (A \to B) \iff \Gamma, A \vdash B$
\end{lemma}
\begin{proof}
	\begin{description}
		\item[$\Ra$:]
			Пусть у нас есть $\Gamma, A$.
			Выведем $(A \to B)$, используя только $\Gamma$ (так можно сделать "--- это левая часть леммы).
			Теперь у нас есть $A$ и $A \to B$, применим modus ponens, получим $B$.
		\item[$\La$:]
			Рассмотрим вывод $B$ и $\Gamma, A$ "--- это какие-то $C_1$, $C_2$, \dots, $C_k=B$.
			Рассмотрим следующую последовательность:
			\[ (A \to C_1); (A \to C_2); \dots; (A \to C_k) = (A \to B) \]
			Это очень похоже на вывод формулы $A \to B$, в частности, последний элемент является тем, чем надо.
			Давайте разберёмся с остальными индукцией по $k$, нам надо для каждого элемента $A \to C_i$ показать, что надо добавить
			перед ним, чтобы получить корректный вывод в $\Gamma$.
			\begin{itemize}
				\item Если $C_i = A$, то формула $A \to A$ просто является тавтологией, её мы выводить умеем по лемме \ref{aToA}
				\item Если $C_i \in \Gamma$, то мы можем вывести $A \to C_i$:
					\[
					\begin{array}{cc}
						C_i \in \Gamma  & \\
						C_i \to (A \to (C_i)) & \text{аксиома 1} \\
						A \to C_i & \text{modus ponens} \\
					\end{array}
					\]
				\item Если $C_i$ "--- аксиома, то можно поступить аналогично предыдущему пункту
				\item
					Если $C_i$ "--- modus ponens от $C_x$ и $C_y$ ($x, y < i$).
					Переобозначим: $C_x = E \to C_i$, $C_y = E$ (вид именно такой, так как мы применили modus ponens).
					И теперь выведем:
					\[
					\begin{array}{cc}
						A \to (E \to C_i) & \text{выведено раньше} \\
						A \to E & \text{выведено раньше} \\
						(A \to (E \to C_i)) \to ((A \to E) \to (A \to C_i)) & \text{аксиома 2 с подстановкой} \\
						(A \to E) \to (A \to C_i) & \text{modus ponens} \\
						A \to C_i & \text{modus ponens} \\
					\end{array}
					\]
			\end{itemize}
	\end{description}
\end{proof}

Заметим, что мы в процессе доказательства леммы пользовались только аксиомами 1 и 2.
Вообще говоря, лемма о дедукции намного удобнее этих аксиом.
\begin{exmp}
	Аксиомы 1 и 2 можно вывести из леммы о дедукции.
\end{exmp}
\begin{proof}
	\begin{enumerate}
		\item
			Пусть есть конкретные формулы $A$ и $B$, докажем аксиому 1 через лемму о дедукции:
			\[ \Gamma \vdash (A \to (B \to A)) \iff \Gamma, A \vdash B \to A \iff \Gamma, A, B \vdash A\]
		\item
			Пусть есть конкретные формулы $A$, $B$ и $C$, докажем аксиому 1 через лемму о дедукции:
			\begin{gather*}
				\Gamma \vdash (A \to (B \to C)) \to ((A \to B) \to (A \to C)) \\
				\Updownarrow \\
				\Gamma, (A \to (B \to C)) \vdash ((A \to B) \to (A \to C)) \\
				\Updownarrow \\
				\Gamma, (A \to (B \to C)), (A \to B) \vdash (A \to C) \\
				\Updownarrow \\
				\Gamma, (A \to (B \to C)), (A \to B), A \vdash C
			\end{gather*}
			Применяем modus ponens в последней строчке (с посылкой $A$), получаем сначала $B \to C$, потом $B$, потом modus ponens к ним и получаем $C$.
	\end{enumerate}
\end{proof}

Теперь мы хотим аналогичным образом написать какие-то более удобные правила для доказательств, аналогичные оставшимся аксиомам.
\begin{center}
	\begin{tabular}{|c|cc|}
		\hline
		№ & Аксиома & Правило вывода \\\hline
		3 & $(A \land B) \to A$ & Если $\Gamma, (A \land B)$, то $\Gamma \vdash A$ \\
		4 & $(A \land B) \to B$ & Если $\Gamma, (A \land B)$, то $\Gamma \vdash B$ \\
		5 & $(A \to (B \to (A \land B)) \to B$ & Если $\Gamma \vdash A$ и $\Gamma \vdash B$, то $\Gamma \vdash (A \land B)$ \\
		\hline
		6 & $(A \to (B \lor A))$ & Если $\Gamma \vdash A$, то $\Gamma \vdash (B \lor A)$ \\
		7 & $(A \to (B \lor A))$ & Если $\Gamma \vdash B$, то $\Gamma \vdash (B \lor A)$ \\
		8 & $(A \to C) \to ((B \to C) \to ((A \lor B) \to C))$ & Если $\Gamma, A \vdash C$ и $\Gamma, B \vdash C$, то $\Gamma, (A \lor B) \vdash C$ \\
		\hline
		9 & $\lnot A \to (A \to B)$ & Если $\Gamma \vdash A$ и $\Gamma \vdash \lnot A$, то $\Gamma \vdash B$ \\
		10 & $(A \to B) \to ((A \to \lnot B) \to \lnot A)$ & Если $\Gamma, A \vdash B$ и $\Gamma, A \vdash \lnot B$, то $\Gamma \vdash \lnot A$ \\
		11 & $(A \lor \lnot A)$ & Если $\Gamma, A \vdash B$ и $\Gamma, \lnot A \vdash B$, то $\Gamma \vdash B$ \\
		\hline
	\end{tabular}
\end{center}
Теперь правило у аксиомы 10 показывает, что рассуждение <<от противного>> работает и в Гильбертовском исчислении: если хотим показать $\Gamma \vdash \lnot X$,
то надо предположить $X$ и вывести из $\Gamma, X$ два утверждения: $B$ и $\lnot B$, тогда по правилу 10 получим $X$,

А правило для аксиомы 11 требует некоторых пояснений:
\begin{proof}
	Пусть $\Gamma, A \vdash B$ и $\Gamma, \lnot A \vdash B$.
	Тогда по правилу для восьмой аксиомы получаем, что $\Gamma, (A \lor \lnot A) \vdash B$.
	А вот $A \lor \lnot A$ "--- это как раз аксиома, то есть её можно из левой части убрать, получить $\Gamma \vdash B$.
\end{proof}

\begin{lemma}[снятие двойного отрицания]
	$\vdash \lnot\lnot \mcA \to \mcA$ (следует из <<ничего>>, то есть из аксиом)
\end{lemma}
\begin{proof}
	Будем преобразовывать:
	\begin{gather*}
		\vdash \lnot\lnot \mcA \to \mcA
		\iff
		\lnot\lnot \mcA \vdash \mcA \\
		\Updownarrow \text{~добавили аксиому 11 с подстановкой} \\
		\lnot\lnot \mcA, \mcA \lor \lnot\mcA \vdash \mcA
		\iff
		\lnot\lnot \mcA \vdash (\mcA \lor \lnot\mcA) \to \mcA
		\iff
		\lnot\lnot \mcA \vdash (\mcA \to \mcA) \land (\lnot\mcA \to \mcA)
		\iff \\ \iff
		\begin{cases}
			\lnot\lnot \mcA, \mcA \vdash \mcA & \text{очевидно} \\
			\lnot\lnot \mcA, \lnot\mcA \vdash \mcA & \parbox{25em}{верно, так как слева есть и $\lnot\mcA$, и его отрицание, т.е. выводится что угодно} \\
		\end{cases}
	\end{gather*}
\end{proof}

\begin{Rem}
	Теперь рассуждение <<от противного>> действительно рассуждение <<от противного>>: предполагаем $\lnot X$, выводим противоречие,
	делаем вывод $X$ (нам для этого нужно снятие двоного отрицания).
\end{Rem}

\begin{exmp}\label{exmp_contradicto}
	Докажем $(A \to B) \to (\lnot B \to \lnot A)$.
	\begin{proof}
		Перепишем по лемме о дедукции:
		\begin{gather*}
			(A \to B) \stackrel{?}{\vdash} (\lnot B \to \lnot A) \\
			\Updownarrow \\
			A \to B, \lnot B \stackrel{?}{\vdash} \lnot A
		\end{gather*}
		Рассуждаем от противного (по переформулировке аксиомы 10).
		Добавляем $A$ к предпосылкам и выводим $B$ и $\lnot B$:
		\begin{gather*}
			A \to B, \lnot B, \lnot \lnot A \\
			\text{применяем снятие двойного отрицания} \\
			A \to B, \lnot B, \lnot \lnot A, A \\
			\text{применяем modus ponens} \\
			A \to B, \lnot B, \lnot \lnot A, B \\
		\end{gather*}
		Успех: вывели и $\lnot B$, и $B$.
	\end{proof}
\end{exmp}

\begin{Def}
	Список формул $\Gamma$ противоречив, если $\Gamma \vdash B, \lnot B$ для всех $B$.
\end{Def}

\begin{Def}
	Список формул $\Gamma$ совместен, если есть значения переменных, которые выполнит все формулы из $\Gamma$.
\end{Def}

\begin{theorem}[теорема о корректности и полноте в сильной форме]\label{correct_full_strong}
	$\Gamma$ совместно $\iff$ $\Gamma$ непротиворечиво.
\end{theorem}
\begin{Rem}
	Стрелка $\Ra$ очевидна: если это не так, что при некоторых значениях формул верны все формулы из $\Gamma$ и все, которые можно вывести,
	то есть $B$ и $\lnot B$, но они одновременно верны быть не могут.
\end{Rem}
\begin{conseq}
	Если $\phi$ "--- тавтология, то из пустого списка можно вывести $\phi$.
\end{conseq}
\begin{proof}
	Доказательство от противного (пример \ref{exmp_contradicto}).
	Предположим $\lnot \phi$ и выведем два утверждения: $B$ и $\lnot B$ (т.е. покажем $\lnot \phi \vdash B$ и $\lnot \phi \vdash \lnot B$.

	Так как $\phi$ "--- тавтология, то нет значений переменных, при которых $\lnot \phi$ верна.
	То есть список $\Gamma=\{ \lnot \phi \}$ не является совместным.
	Тогда по теореме $\Gamma$ противоречив, то есть можно вывести и $B$, и $\lnot B$.

	Таким образом получаем $\vdash \lnot \lnot \phi$, а снимать двойное отрицание мы уже умеем и получим $\vdash \phi$.
\end{proof}

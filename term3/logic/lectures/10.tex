\subsection{В комплексных числах}
\setauthor{Егор Суворов, Анастасия Старкова}
\begin{theorem}
	В $(\C, =, +, \cdot, 0, 1)$ допустима элиминация кванторов
\end{theorem}
\begin{proof}
	Атомарные формулы имеют вид $P(x_1, \dots, x_n)=0$.
	Назовём множество <<хорошим>>, если оно представляется в виде такого объединения:
	\[
		\bigcup_i
			\begin{cases}
				P_{i1} = 0, \\
				P_{i2} \neq 0, \\
				P_{i3} = 0, \\
				P_{i4} \neq 0, \\
				\vdots \\
			\end{cases}
	\]
	Будем действовать аналогично доказательству теоремы Тарского-Зайденберга.
	Нам потребуется доказать, что проекция хорошего множества на любую из осей координат также является хорошим,
	после этого мы сможем стандартными трюками (замена $\forall$ на $\exists$ и элиминация кванторов, начиная с самого вложенного)
	доказать текущую теорему.

	Также аналогичным образом вводим диаграммы, но в этот раз в диаграмму включаем только корни многочленов (потому что никаких <<промежуточных значений>> нет),
	состояния клеток "--- <<равно нулю>> и <<не равно нулю>>.
	Корни теперь можно записывать в любом порядке.
	\TODO
\end{proof}

\section{Гильбертовское исчисление высказываний и предикатов}
\begin{Def}
	Предикатная формула называется \textsl{тавтологией}, если она истинна в любой интерпретации.
\end{Def}
\begin{Rem} % возможно, это всё-таки определение "выводится из аксиом"?
	Пусть есть набор аксиом (возможно, бесконечный): $A_1, \dots, A_i, \dots$ и формула $\phi$.
	Тогда естественно считать, что тавтологичность $(\Land A_i) \to \phi$ эквивалентна тому, что $\phi$ выводится из аксиом (здесь и далее $\to$ "--- следствие, являющееся частью предикатной формулы).
\end{Rem}

\begin{theorem}
	Если в сигнатуре достаточно много предикатных и функциональных символов, то множество тавтологий этой
	сигнатуры неразрешимо.
\end{theorem}
\begin{proof}
	\TODO на практике (что-то про ассоциативность)
\end{proof}

\begin{theorem}[Гёделя о полноте исчисления предикатов]
	Для любой перечислимой сигнатуры множество тавтологий перечислимо.
\end{theorem}
\begin{Rem}
	К этой теореме мы будем долго идти и в конце концов докажем более сильное утверждение.
\end{Rem}

\begin{Def}
	Правило вывода в Гильбертовском исчислении высказываний (modus ponens):
	\[ \frac{A \to B, A}{B} \]
	Читать так: если есть формулы $A \to B$ и $A$, то можно вывести формулу $B$.
\end{Def}
%\begin{Rem}
%	Какая-то шутка, которую мы когда-нибудь поймём: <<женский>> вариант правила:
%	\[ \frac{A \to B, \text{очень хочется }B}{A} \]
%\end{Rem}

Список аксиом (их много, но на экзамене можно будет сделать шпаргалку с ними):
\begin{enumerate}
\item $A \to (B \to A)$ "--- истину всегда можно вывести
\item $(A \to (B \to C)) \to ((A \to B) \to (A \to C))$ "--- некоторая транзитивность: если при условии $A$ из $B$ можно вывести $C$, то если выведем $B$ из $A$, то и $C$ из $A$ тоже выведем.
\item $(A \land B) \to A$
\item $(A \land B) \to B$
\item $A \to (B \to (A \land B))$ "--- при условии, что верно и $A$, и $B$, верно $A \land B$.
\item $A \to (A \lor B)$
\item $B \to (A \lor B)$
\item $(A \to C) \to ((B \to C) \to ((A \lor B) \to C))$ "--- если $C$ выводится и из $A$, и из $B$, то выводится из $A \lor B$.
\item $\lnot A \to (A \to B)$ "--- из лжи следует всё, что угодно
\item $(A \to B) \to ((A \to \lnot B) \to \lnot A)$ "--- из $A$ не могут одновременно следовать и $B$, и $\lnot B$
\item $A \lor \lnot A$
\end{enumerate}
Все эти формулы являются тавтологиями.

\begin{Def}
	Формула $\phi$ \textit{выводима}, если существует последовательность формул $C_1, \dots, C_k$ такая, что:
	\begin{itemize}
	\item $C_k = \phi$
	\item Формула $C_i$ получена одним из двух способов:	
		\begin{enumerate}
		\item Аксиома, в которую подставлены произвольные формулы вместо переменных
		\item Результат modus ponens для каких-то двух формул $C_a$ и $C_b$, где $a, b < i$.
		\end{enumerate}
	\end{itemize}
\end{Def}

\begin{theorem}[корректность]
	Если $\phi$ выводима, то $\phi$ "--- тавтология.
\end{theorem}
\begin{proof}
	Все аксиомы тавтологичны: надо просто рассмотреть таблицу истинности для каждой из них.
	Так как от интерпретации у нас зависят только изначальные значения переменных, то этого хватит.

	Использование modus ponens позволяет нам из двух тавтологий получить некоторую третью формулу, которая
	тоже является тавтологий (опять же, рассмотрим таблицу истинности), что и требовалось.
\end{proof}

\begin{lemma}
	$\mcA \to \mcA$ выводима
\end{lemma}
\begin{proof}
	Рассмотрим вторую аксиому, подставим в неё $A=\mcA$, $C=\mcA$, $B=\mcA \to \mcA$:
	\[ (\underbrace{\mcA}_A \to (\underbrace{(\mcA \to \mcA)}_B \to \underbrace{\mcA}_C)) \to ((\underbrace{\mcA}_A \to \underbrace{(\mcA \to \mcA)}_B) \to (\underbrace{\mcA}_A \to \underbrace{\mcA}_C)) \]
	Теперь отдельно рассмотрим первую аксиому, подставим $A=\mcA$ и $B=\mcA \to \mcA$:
	\[ \underbrace{\mcA}_A \to (\underbrace{(\mcA \to \mcA)}_B \to \underbrace{\mcA}_A) \]
	Теперь по modus ponens соединим эти две формулы, получим:
	\[ (\mcA \to (\mcA \to \mcA)) \to (\mcA \to \mcA) \]
	Теперь заметим, что первая половина этой формулы "--- аксиома 1 с подставленными $A=B=\mcA$, то есть тавтология.
	Применяем modus ponens еще раз:
	\[ \mcA \to \mcA \]
\end{proof}
\TODO

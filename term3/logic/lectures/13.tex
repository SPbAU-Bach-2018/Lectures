\begin{Def}
	Теория $T$ \textit{разрешима}, если выводимость любой замкнутой формулы
	$\phi$ алгоритмически разрешима, то есть множество $\{ \phi \mid T \vdash \phi \}$ разрешимо.
\end{Def}

\begin{Def}
	Пусть $\sigma$ "--- сигнатура, $T$ "--- теория, $\phi$ "--- замкнутая формула.
	Тогда $\phi$ является \textit{теоремой}, если $T \vdash \phi$.
\end{Def}
\begin{Def}
	Если $\Gamma$ "--- множество теорем $T$.
	Тогда $\Gamma$ "--- \textit{аксиоматизация} $T$, если:
	\[ \forall \psi \in T \colon \Gamma \vdash \psi \]
	То есть мы выводим какое-то другое множество теорем, из которого можно
	вывести исходные аксиомы $\Ra$ можно вывести всё множество $T$ и ничего больше.
\end{Def}

\begin{theorem}
	Пусть $T$ полная, непротиворечивая и конечно аксиоматизируема (т.е. есть
	конечное множество аксиом).
	Тогда $T$ разрешимо.
\end{theorem}
\begin{proof}
	Так как $T$ полная, то из $T$ можно вывести либо $\phi$, либо $\lnot \phi$.
	Давайте запустим два алгоритма: один будет пытаться вывести $\phi$, а
	другой "--- вывести $\lnot \phi$ (просто перебирая все строки-доказательства и проверяя).
	Один из них обязательно завершится.
	Получили требуемый алгоритм.
\end{proof}
\begin{Exercise}
	Для перечислимого множества аксиом тоже верно.
\end{Exercise}

\begin{conseq}
	Множество $Th(\No, =, \times, +, 0, 1)$ полное.
	Однако оно неперечисислимо (по теореме Гёделя) $\Ra$ неразрешимо $\Ra$ нет перечислимой аксиоматизации.
\end{conseq}
\begin{Rem}
	Иногда натуральные числа определяют через аксиомы Пеано в сигнатуре $\{=, +, \times, 0, 1\}$:
	\begin{enumerate}
	\item Аксиомы равенства
	\item
		Для сложения и умножения (ассоциативность, дистрибутивность),
		Коммутативность "--- это сложная теорема.
	\item Связь сложения и умножения
	\item Аксиомы индукции (коих бесконечно): для любой формулы $\phi$ с одним параметром:
		\[ \phi(0) \land (\forall k \colon \phi(n) \to \phi(n+1)) \to (\forall x \colon \phi(x)) \]
	\end{enumerate}
	Натуральные числа являются моделью этих аксиом.
	Вообще, конечно, чтобы говорить <<модель>>, нам нужна непротиворечивая теория.
	Непротиворечива ли аксиоматика Пеано "--- вопрос философский: мы можем лишь вывести
	это из веры в непротиворечивость какой-нибудь другой теории (например, Цермелло-Френкеля).
	Бывает даже объяснение, почему нельзя ничего поточнее.

	Отсюда следует, что аксиоматика Пеано задаёт не все верные утверждения (потому что
	у всех верных не бывает перечислимой аксиоматизации, а аксиомы Пеано даже разрешимы).
	Об этом же, кстати, говорила первая теорема Гёделя о неполноте.
\end{Rem}
\begin{theorem}[вторая теорема Гёделя о неполноте]
	В арифметике Пеано невыразима формула $C$, которая выражает непротиворечивость
	арифметики Пеано.
	Без доказательства, это не самая строгая формулировка.
\end{theorem}

\subsection{Примеры теорий}
\subsubsection{Целые числа}
Знаем, что теория $Th(\Z, =, S, 0)$ разрешима (так как есть элиминация кванторов).
Когда мы доказывали существование элиминации кванторов, мы пользовались какими-то
свойствами модели с носителем $\Z$.
Можно посмотреть на то, чем мы пользовались, и построить список необходимых аксиом:
\begin{enumerate}
	\item $\forall x \colon \exists y \colon S(y)=x$ ($S$ сюръективна)
	\item $\forall x \colon \forall y \colon S(x) = S(y) \Ra x = y$ ($S$ инъективна)
	\item $\forall x \colon S(x) \ne x$
	\item $\forall x \colon S(S(x)) \ne x$
	\item $\cdots$
	\item $\forall x \colon S(\dots S(x) \dots) \ne x$
\end{enumerate}
Оказывается, что этого достаточно для существования элиминации кванторов.
\begin{Exercise}
	Проверить.
\end{Exercise}

\begin{theorem}
	У $Th(\Z, =, S, 0)$ нет конечной аксиоматизации.
\end{theorem}
\begin{proof}
	Пусть имеется конечная аксиоматизация $A_1, \dots, A_k$.
	Выведем каждую из $A_i$ из исходных аксиом.
	В каждом выводе используется конечное число исходных аксиом "--- $X_i$.
	Объединим все $X_i$, получим лишь конечное число исходных аксиом,
	достаточных для вывода любой новой аксиомы $A_1, \dots, A_k$.
	Теперь возьмём минимальный исходных аксиом префикс, содержащий все исходные аксиомы из $\bigcup X_i$,
	назовём его $L$.

	Заметим, что $L$ тоже является аксиоматизацией теории (так как из него можно вывести
	аксиоматизацию $A_i$).

	Пусть в $L$ входят все исходные аксиомы $S^{(n)}(x) \ne x$, а вот аксиому $S^{(n+1)}(x) \ne x$ не взяли.
	Покажем, что она невыводима из предыдущих.
	В самом деле: возьмём модель $\Z/(n+1)\Z$ (вычеты по модулю $n+1$), в ней получаем $S^{(n+1)}(x) = x$,
	а все предыдущие аксиомы верны.
\end{proof}

\subsubsection{Плотный линейный порядок}
\begin{Def}
	Теория плотного линейного порядка $T$ без первого и последнего элемента:
	есть сигнатура $\mathcal{P}=\{ =, \le \}$ (без функциональных символов) и следующие аксиомы:
	\begin{enumerate}
		\item Аксиомы равенства
		\item $\forall x, y \colon (x \le y) \lor (y \le x)$
		\item $\forall x \colon x \le x$
		\item $\forall x, y, z \colon (x \le y) \land (y \le z) \to x \le z$
		\item $\forall x, y \colon (x \le y) \land (y \le x) \to x = y$
		\item $\forall x \colon \exists y \colon x < y$ (тут за $x < y$ обозначали $(x \le y) \land \lnot (x = y)$
		\item $\forall x \colon \exists y \colon y < x$
		\item $\forall x, y \colon ((x < y) \to (\exists z \colon (x < z) \land (z < y)))$ "--- плотность порядка
	\end{enumerate}
\end{Def}
\begin{Rem}
	Интерпретация с носителем $\Q$ является моделью $T$, так как все аксиомы верны.
	Отсюда ясно, что $Th(\Q, =, \le)$ является надмножеством $T$, так как все аксиомы оно точно
	выполняет (пока непонятно, почему в него не попало что-то еще).
\end{Rem}
\begin{assertion}
	Для $T$ нет конечных моделей.
\end{assertion}
\begin{proof}
	У нас для каждого элемента есть строго больший, причём отношение <<строго больше>> замкнуто по циклу быть не может.
\end{proof}
\begin{assertion}
	Все счётные модели $T$ изоморфны.
\end{assertion}
\begin{proof}
	Пусть есть две счётные модели $M_1$ и $M_2$ с элементами $a_1, \dots$ и $b_1, \dots$.
	Берём по очереди по одному элементу из каждой модели: $a_1, b_1, a_2, b_2, \dots$.
	Взяли элемент, хотим найти ему соответствие в другой модели так, чтобы порядок был одинаков.
	Просто посмотрим, где в текущей модели стоит текущий элемент (между $x$ и $y$) и
	воткнём его в другой модели между образами $x$ и $y$.

	Каждый элемент получит пару когда-нибудь, построили биекцию.
\end{proof}
\begin{assertion}
	$T$ "--- полная теория.
\end{assertion}
\begin{Rem}
	Если это так, то множество теорем теории $T$ покрывает все верные утверждения,
	то есть совпадает с $Th(\Q, =, \le)$.
\end{Rem}
\begin{proof}
	От противного: пусть $T$ не является полной.
	Тогда возьмём какую-нибудь невыводимую формулу $\phi$:
	\[
		\begin{cases}
			T, \phi & \text{непротиворечива} \\
			T, \lnot\phi & \text{непротиворечива} \\
		\end{cases}
	\]
	Рассмотрим не более чем счётные модели для каждой из двух нвоых теорий: $M_1$, $M_2$
	(была теорема \hyperref[lowenheim]{Лёвенгейма-Сколема}).
	Они обе также являются моделями для $T$.
	Значит, они не могут быть конечными, т.е. они счётны.
	Следовательно, по предыдущему утверждению, они изоморфны.

	Получаем противоречие: модели изоморфны, но в одной утверждение $\phi$ верно,
	а в другой "--- ложно.
	Так не бывает "--- это следует просто из того, как мы считаем истинность формул.
\end{proof}

\begin{Rem}
	Наша теория $T$ конечно аксиоматизирована аксиомами $A_1, \dots, A_k$.
	Это хорошо, так как проверка выводимости формулы $\phi$ сводится к проверке
	общезначимости формулы такого вида:
	\[ (A_1 \land A_2 \land \dots \land A_k) \to \phi \]
\end{Rem}

\chapter{Упорядоченные множества}
\begin{Def}
	Есть множество $M$ и бинарное отношение $\le$.
	Оно называется \textit{отношением частичного порядка}, если верны три свойства:
	\begin{enumerate}
		\item Рефлексивность: $a \le a$ 
		\item Транзитивность: $a \le b \land b \le c \Ra a \le c$
		\item Антисимметричность: $a \le b \land b \le a \Ra a = b$
	\end{enumerate}
	Такое множество называется \textit{частично упорядоченным множеством} (Ч.У.М.).
\end{Def}
\begin{Def}
	Если $(M, \le)$ обладает свойством:
	\[ \forall x, y \colon (x \le y) \lor (y \le x) \]
	То множество называется \textit{линейно упорядоченным}
\end{Def}

\begin{Def}
	Пусть есть $(M_1, \le)$ и $(M_2, \le)$, то можно ввести ЧУМ $M_1+M_2$:
	нарисовали $M_1$ слева, $M_2$ справа (две совсем независимые копии).
	При сравнении элементов из одного множества
	порядок старый (частичный), а из разного "--- элемент из $M_1$ всегда меньше любого элемента $M_2$.
\end{Def}
\begin{Def}
	Можно ввести частичный порядок на декартовом произведении $M_1 \times M_2$:
	\[
		(x_1, y_1) < (x_2, y_2)
		\iff
		(x_1 < x_2) \lor (x_1 = x_2 \land y_1 < y_2)
	\]
	Например, если $x_1$ и $x_2$ не сравнимы и не равны, то $(x_1, y_1)$ и $(x_2, y_2)$ тоже не сравнимы.
\end{Def}

Напоминание:
\begin{Def}
	Элемент $y$ минимальный, если:
	\[ \not\exists x \colon x < y \]
\end{Def}
\begin{Def}
	Элемент $y$ наименьший, если:
	\[ \forall x \colon y \le x \]
\end{Def}

\begin{Def}
	$M$ "--- Ч.У.М., оно называется \textit{фундированным}, если любое его непустое подмножество
	имеет минимальный элемент.
\end{Def}
\begin{theorem}
	Следующие утверждения эквивалентны:
	\begin{enumerate}
		\item $M$ "--- фундированное
		\item В $M$ нет бесконечно убывающей цепи ($x_1 > x_2 > x_3 > \dots$)
		\item
			Есть индукция: для любого предиката $A(x)$ над $M$ верно:
			\begin{gather*}
				(\forall x \colon\\
					(\forall y\colon (y < x) \to A(y) \\
					) \to A(x) \\
				) \to \forall x \colon A(x)
			\end{gather*}
			Отдельное утверждение для базы не нужно: например, если у нас $M=\No$ (см. пример \ref{well_founded_no}),
			то элемент $x=0$ называется минимальным и левая часть при $x=0$ вырождается в $\t{true} \to A(x)$.
	\end{enumerate}
\end{theorem}
\begin{proof}
	\begin{description}
		\item[$1 \Ra 2$:]
			Пусть есть бесконечная цепь $x_1 > x_2 > \dots$.
			Рассмотрим множество элементов этой цепи.
			Так как множество фундированное, то есть минимальный элемент $x_k$.
			Тогда неравенство $x_k > x_{k+1}$ не может быть верным.

		\item[$2 \Ra 1$:]
			Возьмём какое-нибудь непустое множество $S$.
			Возьмём какой-нибудь элемент $x_1$.
			Дальше возьмём какой-нибудь элемент $x_2 < x_1$.
			Потом "--- $x_3 < x_2$, и так далее.
			Если в какой-то момент не сможем найти нужный элемент, то последний взятый и есть минимальный.
			Иначе построим бесконечную убывающую цепь.

		\item[$1 \Ra 3$:]
			Пусть $S=\{x \mid A(x) \text{"--- ложно}\}$.
			Пусть $y$ "--- минимальный элемент $S$.
			Тогда для любого $z<y$ верно, что $A(z)$ истинно.
			Следовательно, $A(y)$ истинно, т.е. $y \notin S$, противоречие.

		\item[$3 \Ra 1$:]
			Возьмём множество $S$, в котором нет минимального элемента.
			Возьмём характеристическую функцию $\chi_s(x)$ и предикат $P(x) \iff \chi_s(x)=0$
			(то есть предикат <<$x \notin S$>>).
			Покажем, что верно условие индукции (из пункта 3).

			Зафиксировали некоторый $x$.
			Пусть для всех меньших $y$ предикат уже верен, т.е. $y < x \Ra y \notin S$.
			Тогда если $x \in S$, то $x$ является минимальным элементом (так как все меньшие в $S$ не лежат).
			Значит, $x \notin S$.

			Тогда применяем принцип индукции, получаем, что $\forall x \colon P(x)$,
			то есть $S = \varnothing$.
	\end{description}
\end{proof}

\begin{exmp}\label{well_founded_no}
	$\No$ фундированное.
\end{exmp}
\begin{exmp}
	Если $A$ и $B$ фундированные, то $A \times B$ тоже финдированно.
\end{exmp}
\begin{proof}
	Тут проще всего проверять отсутствие бесконечно убывающих цепей:
	\[ (a_1, b_1) > (a_2, b_2) > \dots \]
	Посмотрим на первые элементы.
	Из них нельзя выбрать бесконечную убывающую цепь, то есть они рано или поздно стабилизируются.
	Теперь посмотрим на вторые элемента, начиная с момента стабилизации: они тоже рано или поздно
	стабилизируются.
\end{proof}
\begin{exmp}
	$\No^k$ тоже фундированное ($\No^k=(\No\times\No)\times\No\dots$).
\end{exmp}
\begin{exmp}
	Есть игра с бесконечным числом ходов: у нас бывают купюры $k$ достоинств.
	На первом шаге есть какие-то.
	На каждом шаге мы отдаём любую одну купюру и берём любое конечное количество
	любых купюр меньшего достоинства.

	Тогда в конце концов игрок останемся без денег.
\end{exmp}
\begin{proof}
\end{proof}

\begin{Def}
	$(M, \le)$ называется вполне упорядоченным, если оно является фундированным и
	порядок линеен (в частности, минимальный и наименьших элементы "--- одно и то же).
\end{Def}

Свойства в.у.м. $M$:
\begin{enumerate}
	\item
		Имеется наименьший элемент, мы его будем обозначать $0$ (ноль).
		\begin{proof}
			Возьмём подмножество $M$, в нём есть минимальный элемент, он же "--- наименьший.
		\end{proof}
	\item
		Для каждого элемента $x\in M$, не являющегося максимальным, есть непосредственно следующий, то есть
		минимальный элемент, больший данного: $\min \{ y \in M \mid y > x\}$.
		Давайте его обозначать $x+1$.
		\begin{exmp}
			Возьмём бесконечное множество $\No + \{ 1, 2, \dots, k \}$.
			Тут есть максимальный элемент.
		\end{exmp}
	\item
		Некоторые элементы могут не иметь непосредственного предшественника.
		\begin{Def}
			Таким элементы называются \textit{предельными}.
		\end{Def}
		\begin{exmp}
			В $\No + \No$ не имеют предшественника $0$ и его копия из второго множества.
		\end{exmp}
		\begin{Rem}
			Из-за линейности порядка непосредственный предшественник не более, чем один.
		\end{Rem}
	\item
		Любой элемент в.у.м. можно представить в виде $z+n$, где $z$ "--- некоторый предельный,
		а $n$ "--- натуральное число, то есть в виде $z\underbrace{+1+1+1+\dots+1}_{n\text{~штук}}$.
		\begin{proof}
			Возьмём элемент $x$ и начнём брать непосредственного предшественника.
			Либо упрёмся в предельный через конечное число шагов, либо получим бесконечную цепь.
		\end{proof}
	\item
		Если $S \subseteq M$ и есть такой $y \in M$, что:
		\[ \forall x \in S \colon x \le y \]
		То существует точная верхняя грань $S$: это наименьший из множества верхних границ $S$.
\end{enumerate}

\begin{Def}
	Пусть $M$ "--- в.у.м.
	Тогда $A$ называется \textit{начальным отрезком}, если:
	\[ \forall x \in A, y \in M \setminus A \colon y > x \]
\end{Def}

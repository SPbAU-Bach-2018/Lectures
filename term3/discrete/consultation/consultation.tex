\documentclass[12pt,a4paper]{report}
\usepackage{polyglossia}
\usepackage{fontspec}
\usepackage{amsmath, amssymb, extarrows}
\usepackage{xcolor}
\usepackage[russian]{hyperref}
\usepackage{indentfirst}
\usepackage{ifthen}
\usepackage[left=1cm,right=1cm,top=2cm,bottom=2cm]{geometry}
\usepackage{minted}
\usepackage[math-style=ISO,vargreek-shape=unicode]{unicode-math}

\setdefaultlanguage[spelling=modern,babelshorthands=true]{russian}
\setotherlanguage{english}

\defaultfontfeatures{Ligatures={TeX}}
\setmainfont{CMU Serif}
\setsansfont{CMU Sans Serif}
\setmonofont{CMU Typewriter Text}  
\setmathfont{Latin Modern Math}
\newcommand{\Setminus}{\mathbin{\backslash}}

\DeclareSymbolFont{cyrletters}{\encodingdefault}{\familydefault}{m}{it}
\newcommand{\makecyrmathletter}[1]{%
  \begingroup\lccode`a=#1\lowercase{\endgroup
  \Umathcode`a}="0 \csname symcyrletters\endcsname\space #1
}
\count255="409
\loop\ifnum\count255<"44F
  \advance\count255 by 1
  \makecyrmathletter{\count255}
\repeat
%% Simpy adds cyrillic to maths!

\frenchspacing

\def\la{\leftarrow}
\def\ra{\rightarrow}
\def\lra{\leftrightarrow}
\def\La{\Leftarrow}
\def\Ra{\Rightarrow}
\def\Lra{\Leftrightarrow}
\def\lrh{\leftrightharpoons}
\def\btu{\bigtriangleup}

\def\N{\mathbb{N}}
\def\Z{\mathbb{Z}}
\def\Q{\mathbb{Q}}
\def\R{\mathbb{R}}
\def\C{\mathbb{C}}
\def\d{\mathup{d}}

\def\LraDef{\stackrel{\mathrm{Def}}{\Lra}}
\def\eqDef{\stackrel{\mathrm{Def}}{=}}

\renewcommand{\le}{\leqslant}
\renewcommand{\ge}{\geqslant}

\DeclareMathOperator{\Int}{int}
\DeclareMathOperator{\cl}{cl}
\DeclareMathOperator{\diam}{diam}
\DeclareMathOperator{\Dom}{Dom}
\DeclareMathOperator{\coDom}{coDom}
\DeclareMathOperator{\Char}{char}
\DeclareMathOperator{\Arg}{Arg}
\AtBeginDocument{\let\Re\relax}
\AtBeginDocument{\newcommand{\Re}{\mathop{\mathrm{Re}}\nolimits}}
\AtBeginDocument{\let\Im\relax}
\AtBeginDocument{\newcommand{\Im}{\mathop{\mathrm{Im}}\nolimits}}
\newcommand{\emod}[1]{\mathop{\equiv}\limits_{#1}}
\newcommand{\Choose}[2]{{\left(#1 \atop #2\right)}}

\renewcommand{\thechapter}{\Roman{chapter}}
\renewcommand{\thesection}{\thechapter.\arabic{section}}

\newcounter{theorem}[section]
\renewcommand{\thetheorem}{\thesection.\arabic{theorem}}
\newcommand*{\theoremheader}[1]{%
	\par\refstepcounter{theorem}%
	\textbf{Теорема \thetheorem.\ifthenelse{\equal{#1}{}}{}{ #1.}}%
}
\newenvironment*{theorem}[1]{
	\theoremheader{#1}%
}{%
	\par%
}

\newcounter{conseq}[theorem]
\renewcommand{\theconseq}{\arabic{conseq}}
\newcommand*{\conseqheader}{%
	\par\refstepcounter{conseq}%
	\textit{Следствие \thetheorem.}%
}
\newenvironment*{conseq}{
	\conseqheader%
}{%
	\par
}

\newenvironment{assertion}{%
	\par%
	\textbf{Утверждение.}%
}{%
	\par%
}

\newenvironment{proof}{%
	\par%
	$\blacktriangleright$%
}{%
	\hfill$\blacktriangleleft$%
	\par%
}

\newenvironment{Def}{%
	\par%
	$\mathfrak{Def\colon}$%
}{%
	\par%
}

\newenvironment{Rem}{%
	\par%
	\textit{REM:}%
}{%
	\par%
}

\newenvironment{exmp}{%
	\par%
	\textbf{Пример:}%
}{%
	\par%
}

\setcounter{MaxMatrixCols}{40}

\newmintinline[cinl]{c}{} %\c is defined :(
\newmintinline[cpp]{cpp}{}
\newmintinline[python]{python}{}
\newmintinline[bash]{bash}{}
\newmintinline[make]{make}{}
\setminted{obeytabs,tabsize=4,linenos,texcomments}
\newminted{c}{}
\newminted{cpp}{}
\newminted{python}{}
\newminted{bash}{}
\newminted{make}{}

\renewcommand{\thesection}{\arabic{section}}
\rhead{}

\begin{document}
\gdef\LectureName{Консультация по ДМ, III семестр}

\section{}
Графы, следствие 7.4: когда мы определяем $L^*$, мы говорим, что в ней вычеркнуты
произвольные столбец и строка. А в следствии говорим, что она есть произведение
$M_i^*$ и $M_i^*$, т.е., по сути, удалили произвольную строку и соответствующий ей столбец.

Ответ: на самом деле, эти две $M_i^*$ независимы, т.е. мы в первой удаляем произвольную строку,
а во второй "--- тоже произвольную строку.

\section{}
Но дальше мы пользуемся формулой Бине-Коши, где мы неявно пользуемся тем, что $M_i^*$ выбраны
одинаковые.

Ответ: каждая из $M_i^*$ "--- это $M_i$ с выкинутой вершиной (строкой).
В формуле Бине-Коши мы в сумме берём не вообще все возможные подматрицы первой и возможные подматрицы второй,
а чтобы соответствовали наборы столбцов слева (рёбер) и строк справа (рёбер), см. конспект по комбе (последние страницы).
Т.е. у нас выбирается в каждом слагаемом один и тот же набор рёбер.
Значит, если одна имеет определитель ноль (так как набор рёбер образует цикл),
то другая тоже имеет определитель ноль (потому что тот же набор рёбер всё еще образует цикл).
Единственная проблема "--- если одна имеет определитель $+1$, а другая "--- $-1$.
Хочется понять, почему такого не бывает.
Омельченко придёт домой и подумает, напишет, почему не бывает.

\section{}
Конспект по комбинаторике: надо поправить $A\times A=A^{(2)}$ на $A \times A=A^2$.
То есть $A^2$ "--- декартово произведение, $A^{(2)}$ "--- подмножества размера два
(без учёта порядка).

\section{}
$P_n$, $C_n$ "--- $n$ есть число вершин.

\section{}
Вопрос: была теорема про ребро $e$ "--- мост в простом связном. Зачем связность в теореме?

Ответ: связность не нужна, так проще доказывать.
Мы во всех этих теоремах с мостами/точками сочленения всегда смотрим на связные графы, так как всегда можно распилить на компоненты.

\section{}
Графы, следствие 7.5: мы потом покажем, что кратность $\lambda=0$ равна единице.
Мы это показываем в главе про напряжения, там, где показываем, что ранг матрицы инцидентности
равен $n-1$.

\section{}
Спектр рассматриваем у неориентированных графов.
Вопрос: мы всё-таки про геометрическую кратность или алгебраическую говорим?
Петя считает, что если мы говорим про неориентированные, то матрица у нас симметрическая,
и из этого как-то наверняка можно показать, что алгебраические кратности совпадают с геометрическими.
Проблемы были только в задачах на спектры графов.
Омельченко посмотрит.

\section{}
Вопрос от Димы.
Теорема BEST: пояснить рассуждения (стр. 47), со слов <<Действительно, из любой вершины $y \neq x$>>.
Мы говорим, что если у нас правильным образом выбираются рёбра, то цикла не может быть.
Предполагаем, что цикл есть, мы как-то вошли в вершину, тогда мы как-то нумеровали рёбра так,
что мы должны выйти из этой вершины по ребру с ещё большим номером, а это невозможно
(мы так назначали нумерацию рёбер).
Еще комментарий: мы когда закончили цикл, вернувшись в вершину 1, мы не стартуем заново.

\section{}
Мультииндексных рекуррентных соотношений в билетах нет нарочно.

\end{document}

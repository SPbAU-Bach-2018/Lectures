\section{} % 01
Множества "--- различные элементы, мощность "--- число элементов ($|X|=n$, $X$ "--- $n$-множество),
можно объединять $\cup$, перескать $\cap$, симметрическую разность $\triangle$, дополнять $X \setminus A$.
Диаграммы "--- пересекающиеся кружочки (мн-ва), элементы "--- точки внутри (можно еще заштриховывать).
$X$ покрывается семейством $X_i$, если $\cup X_i=X$.
Разбиение "--- все $X_i$ непусты, попарно непересекаются, называются блоками разбиения.
Если порядок блоков важен, то упорядоченное разбиение.
Разделение "--- упорядоченное разбиение с разрешёнными пустыми $X_i$.
Декартово: $A \times B = \{(a,b) \mid a \in A, b \in B\}$, можно обобщить для конечного числа,
можно писать степень $X^{k}$.
$k$-мультимножество над $X$ "--- пара $(X, \phi \colon X \to \Z_{+})$,
$\phi(x)$ "--- число вхождений элемента.
Формально пишем $(X, \{(x, 2), (y, 1)\})$, но обычно пишут $\{x, x, y\}$.
$k$-сочетание без повторений "--- $k$-элементное подмножество $n$-множества.
С повторениями "--- мультимножество.
$k$-перестановка без повторений "--- упорядоченное $k$-подмножество.
С повторениями "--- любой элемент $X^{k}$.

Правило суммы: если выбираем элемент либо из $A$, либо из $B$ ($A \cap B=\varnothing$), то есть
$|A|+|B|$ способов сделать.
Общий вид "--- разбиение на блоки.
Правило произведения: выбираем один из $A$, один из $B$, $|A \times B| = |A| \cdot |B|$.

Обобщённое правило суммы: $|A\cup B|=|A|+|B|-|A\cap B|$.
Принцип включения-исключения в комбинаторике (если $A, B\subseteq X$): $|A'\cap B'|=|X|-|A|-|B|+|A\cap B|$ ($A'=X\setminus A$),
например <<сколько человек в комнате не знает ни один из двух языков>>.
Можно их обобщить на случай нескольких множеств.

\section{} % 02
Число $k$-сочетаний из $n$ элементов без повторений есть $\binom{n}{k}$ (<<из $n$ по $k$>>).
Можно формулой с факториалами, но она не обобщается.
Пусть $\Sigma_k$ "--- все $k$-подмножества $X$, тогда $|\Sigma_k|=\binom{n}{k}$.
Тогда по правилу суммы можно вывести $\binom{n}{k}=\binom{n-1}{k-1}+\binom{n-1}{k}$ (при $k, n \ge 1$).
Граничные условия: при $k>n$ ноль, при $k=0$ единица.
Есть треугольник Паскаля, он симметричен (комбинаторно докажем).
Можно нарисовать $\binom{n}{k}$ на плоскости в точке $(n, k)$, получим интерпретацию через количество путей.

Суммирование по верхнему индексу: $\binom{0}{k}+\dots+\binom{n}{k}=\binom{n+1}{k+1}$ (первые слагаемые "--- ноль).
Формальное док-во: $\binom{m}{k}=\binom{m+1}{k+1}-\binom{m}{k+1}$, просуммируем $m=k\dots n$, телескопически свернули.
Комбинаторный подход: взяли сочетания из $[n+1]$ на $k+1$, сгруппировали по последнему взятому элементу, просуммировали.
Бином Ньютона: $(x+y)^n=\sum_{k=0}^n \binom{n}{k}x^ky^{n-k}$ (док-во комбинаторно).
Подставим $x=y=1$, получили сумму всех коэффициентов.
Подставили $x=1, y=-1$, получили знакопеременную сумму, она ноль при $n\ge 1$,
т.е. число сочетаний с чётным числом элементов равно числу сочетаний с чётным.
Если продифференцировать по $x$ и поставить $x=y=1$, то $\sum_{k=1}^n k\binom{n}{k} = n2^{n-1}$ (выбрали в сочетании фиксированный элемент, комбинаторно тоже можно).

Принцип биекции: если $X$ и $Y$ конечные, есть биекция $|X|\to|Y|$ (т.е. для любого $y$ есть единственный прообраз), то $|X|=|Y|$.
Число $k$-мультимножеств над $n$ есть $\repbinom{n}{k}$.
Сортируем мультимножество, прибавляем к $i$-му элементу $i-1$, получили $k$-сочетание над $[n+k-1]$,
тогда $\repbinom{n}{k}=\binom{n+k-1}{k}$ (тут важен принцип биекции, так как мы в общем случае определяли сочетания для $X$, а не для $[|X|]$).

\section{} % 03
$k$-перестановки с повторениями (кортеж из $k$ элементов): количество равно $n^k$.
Теперь можно комбинаторно сосчитать количество подмножеств мн-ва $X$ (бинарная строка длины $|X|$).
Количество перестановок без повторений $P(n, k) = n(n-1)\dots(n-k+1) = (n)_k$, если $n=k$ имеем $P_n=P(n,k)=n!$.
А можно взять сочетание и упорядочить элементы: $(n)_k=k!\binom{n}{k}$, отсюда можно обобщить $\binom{n}{k}$ на
комплексные $n$ и целые $k$ (при $k<0$ ноль, при $k=0$ единица).
Убывающая факториальная степень: $(n)_k=n^{\underline k}$.
Возрастающая факториальная степень: $n^{(k)}=n^{\bar k}=q(q+1)\dots(q+k-1)$.
Тогда $\repbinom{n}{k}=\frac{n^{(k)}}{k!}$.

Урновая схема (к ней свяодятся задачи): есть урна с $n$ различными предметами,
вытаскиваем по очереди $k$ предментов; вариации: (не) возвращать предмет обратно (с повторениями/без повторений),
важен ли порядок (упорядоченные, неупорядоченные).
Ответы на задачи: $n^k$, $(n)_k$, $\repbinom{n}{k}$, $\binom{n}{k}$.
Раскладка предметов: есть $n$ различных ящиков, в них надо сложить $k$ различимых предметов (можно накладывать ограничение
на число предметов в ящике).
Если в ящике любое количество, то $n^k$.
Если не больше одного, то $(n)_k$.
Вариация: предметы теперь неразличимы, тогда если произвольное число, то $\binom{n}{k}$, если не более одного, то $\binom{n}{k}$.
Пример: разбиение натурального числа на $n$ неотрицательных слагаемых (кладём неразличимые единицы в $k$ ящиков)

Отображения $[n] \to [k]$: если все, то это различимые предметы по различимым ящикам ($k^n$),
инъекции "--- раскладка, но в ящике не больше одного ($(n)_k$), биекций $n!$, сюръекций $\hat S(n, k)$, считаем.
Любое отображение "--- сюръекция на образ, можно выразить $n^k$ через сумму $\hat S(n, k)$ по $k$
(с коэфф. $\binom{n}{k}$).
Формула обращения (докажем потом): если $f_k=\sum_{i=0}^k \binom{k}{i} g_i$, то $g_k=\sum_{i=0}^k (-1)^{k-i} \binom{k}{i} f_i$,
теперь выразили явно $\hat S(n, k)$.

\section{} % 04
Количество разделений на $k$ блоков (разделение разрешает пустые) "--- как отображений, т.е. $k^n$.
Сюръекции задают \textit{разбиения}, их $\hat S(n, k)$.
Рассмотрим разделения, где в $i$-м блоке $a_i$ элементов ($\sum a_i=n$),
тогда их кол-во равно мультиномиальному $\frac{n!}{\prod a_i!}$ ($P(n; a_1, \dots, a_k)$,
она же перестановка $n$-множества с повторениями).
Тогда всех разделений "--- сумма по всем наборам $a_i$ мультиномиальных, т.о. она равна $k^n$.
А если положить $a_i>0$, то получим кол-во разбиений, т.е. $\hat S(n, k)$.

Еще биекция: раскладка $k$ неразличимых по $n$ различимым ящикам, её можно смотреть
как упорядоченный набор из $k$ неразличимых и $n-1$ неразличимой перегородки между ящиками,
т.е. $\repbinom{n}{k}=P(k+n-1;k;n-1)=\binom{n+k-1}{k}$.
Пусть $S(n, k)$ "--- кол-во неупорядоченных разбиений на $k$ блоков, тогда $k!S(n, k)=\hat S(n, k)$
(блоки же непустые), получили явную формулу для $S(n, k)$, это числа Стирлинга второго рода.
Рекуррента для них: $S(n, k)=S(n-1, k-1)+kS(n-1, k)$ при $k>0$, $S(0,0)=1$, $S(n,0)=0$,
$S(n, >n)=0$.
Док-во: граничные условия понятные, рекуррента: выкинули элемент 1, либо убился ящик,
либо не убился (тогда осталось $n-1$ элементов, надо добавить элемент обратно в один из $k$ ящиков).
Можно смотреть как на взвешенные пути по решётке.
$k^n=\sum_{i=0}^n (k)_i \cdot S(n, i)$.
Числа Стирлинга "--- раскладки $n$ разных элементов по $k$ неразличимым ящикам, в любом хотя бы один предмет.
Если в ящике любое число предметов, то считаем $B(n, k)=\sum_{i=0}^k S(n, i)$.
Числа Белла: $B(n)=B(n, n)$ (число всех возможных разбиений $[n]$).
$B(n+1)=\sum_{k=0}^n \binom{n}{k} B_k$ (посмотрели, сколько чисел попали в группу к $(n+1)$).

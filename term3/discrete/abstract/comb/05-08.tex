\section{} % 05
Пусть $a_0, a_1, \dots$ "--- последовательность, причём для $n\ge 0$ и фикс.$m$
число $a_{n+m}$ зависит только от $m$ предыдущих функцией $f$, получаем рекуррентное соотношение $m$-го порядка.
Если $f$ линейна, то линейное соотношение.
Если свободный член ноль, то однородное.
Чтобы определить $a_i$, надо задать $f$ и первые $m$ членов.
Решение "--- явная формула для вычисления значения.
Решаем сначала линейное однородное первого порядка: предполагаем ответ вида $a_n=r^n$, получили частное решение
для случая $a_0=1$, а потом по однородности домножили, получили общее решение $a_n=a_0\cdot b^n$.
Теперь второй порядок: предположили степенное решение, получили характеристическое уравнение $r^2-b_1r-b_2=0$.
Есть два разных вещественных корня, то получили общее решение $c_1r_1^n+c_2r_2^n$ (надо показать, что это решение,
и что при любых начальных $a_0, a_1$ можно подобрать константы, считаем определитель системы).
Если один кратный корень, то $c_1\rho^n+c_2n\rho^n$ (показали то же самое).
Пусть есть два комплексно сопряжённых корня, тогда аналогично двум вещественным, но дораскрывали до
$c_1 \rho^n \cos (n \phi) + c_2\rho^n \sin(n \phi)$.
Для чисел Фибоначчи есть формула.

\section{} % 06
С постоянными коэффициентами "--- они не зависят от $n$.
Пусть есть однородное $m$-го порядка, тогда предположили вид $r^n$, получили характеристическое.
Если есть корень $r$ кратности $q$, то можно выписать частные решения $r^n, nr^n, \dots, n^{q-1}r^n$ (они решения,
так как кратный корень "--- корень производной, можно дифференцировать характеристический),
общее "--- их линейная комбинация (без доказательства, что всё покрыли?).
Аналогия с диффурами (диффуры сложно доказывать нигде не надо): $y(x_0)=y_0$, $y'(x_0)=\partd{y}{x}=b_1\cdot y(x)$, тогда разделим переменные и получим
$y(x)=y_0e^{b_1(x-x_0)}$, а можно было предположить характер решения ($e^{\lambda x}$) и потом заюзать однородность.
Для второго порядка: характер тот же, получаем характеристическое уравнение, в случае кратного корня
надо искать в $c_1e^{\lambda x}+c_2xe^{\lambda x}$, в случае вещественных можно распилить до синуса и косинуса.

Строим отображение между рекуррентами и диффурами: берём линейное однородное $a_{n+2}=b_1a_{n+1}+b_2a_n$,
домножаем на $\frac{x^n}{n!}$, суммируем $\sum_{n=0}^\infty$.
Полагаем $y(x)$ "--- ф-ция с рядом Тейлора в окр. ноль, тогда получили $y''(x)=b_1\cdot y'(x) + b_2 \cdot y(x)$
с начальными $y(0)=a_0$, $y'(0)=a_1$.
Обратно: разложили в Тейлора, получили рекурренту.
Иногда удобно от диффуров к рекуррентам (чтобы быстро вычислять Тейлора), а можно наоборот (использовать методы диффуров в рекуррентах,
например, метод неопределённых коэффициентов, см. дальше).
\TODO диффуры, неоднородности (5.4.4 и дальше)

\section{} % 07
Разбили множество $X$ на блоки (например, все графы по числу вершин), сопоставили блокам разные элементы
коммутативного кольца (а элементам $x$ "--- соотв. элементы кольца), теперь сумма по всем элементам $X$ "--- энумератор.
Стандартное кольцо "--- $\C^{\Z_+}$ всех счётных последовательностей комплексных чисел, сложение поэлементно,
умножение "--- свёртка (как многочлены) или биномиальная свёртка ($c_n=\sum \binom{n}{i}a_ib_{n-i}$), это коммутативная
обл. целостности (кольцо, произведение ненулевых "--- ноль).
На практике обозначаем их $\C[[z]]$ и $\C_e[[z]]$, вводим $z=(0,1,0,0,\dots)$, а потом замечаем, что $z^n$ есть единица в позиции $n$
(для биномиальной "--- $n!$ там же), тогда любая последовательность записывается как формальный степенной ряд.
А теперь элемент мн-ва $x$ отображаем в $z^k$ ($k$ "--- типа вес элемента), энумератор $\sum c_nz^n$ "--- обыкновенная производящая функция
мн-ва $X$ (для $\C_e[[z]]$ "--- экспоненциальная, $\sum c_n\frac{z^n}{n!}$).
Производящие функции можно складывать и умножать по правилам выше, по факту получаем как будто нормальные бесконечные многочлены-ряды.
Ряд $\sum z^n$ обратен к $1-z$.
Если свободный член не ноль $\iff$ есть обратный в $\C[[z]]$, т.к. выписали формулы.
Можно рассматривать поле частных вида $h(z)/g(z)$, где у $g(z)$ есть обратный.

\section{} % 08
Строим производящую функцию по рекурренте: взяли рекурренту порядка $m$ ($a_{n+m}=\dots$), домножили на $z^{n+m}$, просуммировали по $n=0\dots\infty$
(иногда надо чуть подвигать), подобавляли не хватающих членов, получили уравнение на $f(z)=\sum_{n=0}^\infty a_nz^n$, решили
относительно $f(z)$.
Если линейное неоднородное, по есть явная формула для $f$ в замкнутом виде (рац. функция).
Для однородных получаем взаимооднозначное соответствие между линейными однородными и рац.производящими.
После замкнутого вида разложили в простейшие, а потом $\frac{1}{(1-\alpha z)^m}=\sum \binom{m+n-1}{n} \alpha^m z^m$ (потому что домножение
на $\frac{1}{1-z}$ "--- это подсчёт сумм на префиксах, биномиальные вылезли из количества путей вниз-вправо по решётке).

Формальная производная ряда вводится понятно как из соображений матана, в $\C[[z]]$ элементы последовательности домножаются, а в $\C_e[[z]]$ просто сдвигаются
(не забываем, что матановая запись "--- лишь сокращение для последовательностей).
Можно доказать св-ва производных: $f'(z)=0 \Ra f(z)=a_0$, домножение на константу, $f'(z)=f(z) \Ra f=ce^z$ ($e^z$ просто так определяем).
Определение производной совпадает с производной функции комплексного аргумента в некоторой окр. $z=0$, это прикольно.
Пример: $a_n=\binom{2n}{n}$, вывели рекурренту $(n+1)a_{n+1}=4na_n+2a_n$, теперь можно вывести производящую функцию $(1-4z)f'z=2f(z)$, $f(0)=1$.
Теперь считаем, что $f(z)$ есть аналитическая функция, решаем диффур разделением переменных, $f(z)=\frac{1}{\sqrt{1-4z}}$, разложили по биному,
сошлось.
Важно, что $f(z)$ действительно аналитична около нуля (т.е. ряд около нуля ряд не раздойдётся).
Плохой пример: $a_{n+1}=(n+1)a_n$.
Но его можно экспоненциальными, тогда коэффициенты растут медленне, жить лучше.
Для чисел Белла экспоненциальные тоже затащят: $B'(z)=e^zB(z)$, $B(0)=1$, тогда $B(z)=e^{e^z-1}$.

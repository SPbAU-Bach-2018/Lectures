\section{} % 09
Дирихле: смотрим ряды $(a_1, a_2, \dots)$, для перемножения используем $c_n=\sum_{d|n} a_db_{n/d}$,
формальный степенной ряд "--- $\sum \frac{a_k}{n^z}$ (тут уже $z$ "--- тупо значок, а не сокращение для ряда),
это некая аналитическая функция, перемножение действительно работает как надо.
Любой элемент с ненулевым свободным членом обратим, нейтральный по умножению "--- $I(z)=\frac{1}{1^z}$.
Дзета-функция Римана: $\zeta(z) = \sum \frac{1}{n^z}$ (куча единиц), аналоги "--- $\frac{1}{1-z}$ и $e^z$.
Функция Мёбиуса $\mu(z)$ "--- обратная к $\zeta(z)$ с двух сторон.
Коэффициенты $\mu_n$: если $n$ свободно от квадратов, то $(-1)^k$ ($k$ "--- число простых в разложении), иначе ноль.
Формулы обращения: если $b_n=\sum_{d|n} a_d$, то $a_n=\sum_{d|n} \mu_d b_{n/d}$ (ввели $f(z)$ и $g(z)$, причём
$g(z)=\zeta(z) \cdot f(z)$ как ряды Дирихле).

ЧУМ: есть отношение $\preccurlyeq$, рефлексивно ($x \preccurlyeq x$), транзитивно, антисимметрично ($x\preccurlyeq y \land y \preccurlyeq x \Ra x=y$),
например, $\N$ с делимостью, $2^{[n]}$ с нестрогим вложением, $[0, n]$ с $\le$ (линейно упорядоченное мн-во).
Сравнимые элементы "--- $x \preccurlyeq y \lor y \preccurlyeq x$.
$y$ покрывает $x$, если $x \prec y$ и нет $z$ строго между ними.
Диаграмма Хассе: граф, вершины соединены, если одна покрывает другую, если $x \prec y$, то $y$ расположен выше.
Замкнутый интервал $[x, y]$: все $z$ с $x \preccurlyeq z \preccurlyeq y$, если любой интервал конечен, то $P$ локально конечно.
Пусть ЧУМ конечен, тогда $\Int(p)$ конечно (мн-во всех интервалов), сопоставим каждому $I=[x,y]$ комплексное (вещественное) число
$f(I)=f(x,y)$, причём $f(x,y)=0$ если они не сравнимы или $y \prec x$, тогда $f$ можно записать как матрицу.
Она получится верхнетреугольной (если отсортировать ЧУМ топологически).
Определим $\delta(X, Y)=(x==y)$ и $\zeta(X, Y)=(X \preccurlyeq Y)$ (единица везде, где можно).
Верхнетреугольные матрицы можно умножать на числа и между собой, определим так умножение функций на $\Int(p)$
($(fg)(x, y) = \sum_{x \preccurlyeq z \preccurlyeq y} f(x, z) \cdot g(z, y)$), мн-во функций замкнуто,
получили алгебру инцидентности (алгебра "--- коммутативное кольцо с единицей, в котором можно умножать элементы друг на друга, есть куча дистрибутивностей).

Можно ввести функцию Мёбиуса: $\mu(x,y)\cdot \zeta(x, y) = \delta(x, y)$, можно написать явную формулу (единица при $x=y$, иначе минус сумма Мёбиусов
по префиксам или суффиксам).
Для $B_n$ (ЧУМ на подмножествах $2^{[n]}$) $\mu(S, T)=(-1)^{|T\setminus S|}$.
Для линейно упорядоченного "--- одна 1, одна -1, остальные нули.
Для натуральных и делимости "--- $\mu(x,y)=\mu(y/x)$ (если делится, то функция Мёбиуса для натуральных от одного аргумента).
Пусть на элементах ЧУМ есть ф-ции $f$, $g$, причём $g(y)=\sum_{x\preccurlyeq y} f(x)$, тогда $f(y)=\sum_{x\preccurlyeq y} g(x)\mu(x, y)$
(док-во: вводим вектор-строки $f$ и $z$, переписываем в матричном виде, домножаем, сокращаем обратные матрицы).
Три примера (см. выше), все доказали.
Для производящих функций что-то похожее (но запись на языке матана делает всё очевидным): в обычных линейно упорядоченное множество $\zeta=\frac{1}{1-z}$, $\mu=1-z$;
в экспоненциальных $B_n$ имеем $\zeta=e^z$, $\mu=e^{-z}$; для Дирихле уже было.
Разные виды свёрток "--- разные способы частично упорядочить $[n]$, дальше не формализовывали.

\section{} % 10
Максимальный элемент "--- сравним со всеми, больше всех
Наибольший (по включению) "--- нет большего.
Верхняя грань мн-ва "--- элемент, больший всех из мн-ва.
Точная верхняя грань (супремум) "--- сравнима со всеми верхними гранями, меньше их всех.
Аналогично для минимумов.
Решётка "--- ЧУМ, любое конечное подмножество имеет точную верхнюю и нижнюю грань.
Достаточно этого требования для под-мн-в из двух элементов (индукция по размеру подмн-ва).
$x\vee y$ "--- точная верхняя (join), $x \wedge y$ "--- точная нижняя (meet).
Пример "--- $B_n$ ($2^{[n]}$).
Лемма: есть есть максимальный ($\hat 1$) и любые два имеют точнуюю нижнюю грань, то решётка
(показываем, что любые два имеют точную верхнюю: взяли все верхние, взяли их точную нижнюю).
Пример: разбиение $[n]$ на блоки, отношение измельчения.
Теорема Weisner: есть решётка, тогда $\mu(\hat 0, \hat 1)=-\sum \mu(x, \hat 1)$ (сумма по всем $x\wedge a = \hat 0, x \neq \hat 0$,
$a$ любой, если повезёт, надо очень мало точек смотреть).
Если надо считать от интервала, берут ЧУМ, изоморфный интервалу.
Для разбиения из $k$ блоков $\mu=(-1)^{n-1}(n-1)!$.
Док-во: так как $\hat 0 \wedge a = \hat 0$, то можно переписать как $\sum_{x\wedge a = \hat 0} \mu(x, \hat 1) = 0$,
вводим характеристическую ф-цию для хороших $x$, она равна $\sum_{y=\hat 0}^{x\wedge a} \mu(\hat 0, y)$,
подставили, поменяли порядок суммирования, внутренняя сумма занулилась.

Связные графы: умеем считать все графы, берём ЧУМ с разбиениеми $[m]$.
Для разбиения $\alpha$ легко считать $\sum_{\beta \preccurlyeq \alpha} f(\beta)$ (графы, у которых между компонентами
нет рёбер, внутри "--- как получится).
Берём формулу обращения, получили сумму по всем разбиениям, внутри "--- формулки от $k$.

\section{} % 11
Правильная скобочная "--- $n$ пар скобок, на префиксах открывающих $\ge$ закрывающих.
Их чисел Каталана $C_n$ (удобно считать $C_0=1$).
Слова Дика: как ПСП, но назвали скобки произвольными символами.
Пути Дика: из $(1,1)$ в $(2n,0)$, идут вправо-вверх и вправо-вниз, не касаются $y=0$.
Бинарные деревья: любая вершина не более двух детей (взяли ПСП для детей, завернули левого в скобку).
Расстановки скобок в выражении: дерево вычислений (тут либо ноль детей, либо два).
Триангуляции выпуклого $n$-угольника (взяли треуг. рядом с ребром $1--2$, разбились на два дальше получили бинарное дерево).
Обычные корневые деревья: Эйлеров обход.
Рекуррента: $C_n = \sum_{k=1}^n C_{k-1}C_{n-k}$ (взяли закрывающую, соотв. первой скобке, см. внутрь и наружу).
Берём обычную производящую функцию для $C_n$, $f=1+zf^2$, ищем аналитическую функцию, решение $\frac{1+\sqrt{1-4z}}{2z}$ расходится в нуле,
не очень аналитичное, а второе решение "--- ок.
Выведем формулу $C_n=\frac{1}{n+1}\binom{2n}{n}=\binom{2n}{n}-\binom{2n}{n+1}$.
Комбинаторно: берём все расстановки скобок, вычитаем неправильные.
Берём неправильную, находим первую скобку, где плохо, инвертируем префикс $2k+1$,
есть обратное, есть биекция, посчитали неправильные.

\section{} % 12
\TODO

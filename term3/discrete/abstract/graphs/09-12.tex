\section{} % 09
Всё для связного графа $G$.
Вводим пр-во циклов.
Рассмотрим мн-во подмножеств рёбер как линейное пр-во $\mathcal E = (\mathbb F_2)^m$
(плюс "--- симметрическая разность), можно ввести произведение подмножеств
(скалярно, это чётность размера пересечения).
Декомпозиция: разбили на попарно непересекающиеся по рёбрам подграфы на $V$
(можно декомпозировать на циклы/пути).
Veblen: есть декомпозиция на циклы $\iff$ граф чётен (все вершины чётной степени),
$\Ra$ очевидно, $\La$ индукция по числу рёбер.
Симметрическая разность чётных подграфов чётна
Множество чётных подграфов образует линейное подпр-во $C$ в $\mathcal E$,
оно порождается всеми циклами, называется пр-вом циклов.
Взяли остовное дерево $T$ (остаток рёбер "--- хорды, образуют кодерево).
Взяли хорду $e$, получили фундаментальный цикл, цикловый ранг $\gamma(G)$ "---
кол-во фундаментальных ($m-n+1$).
Фундаментальные линейно независимы (так как хорда лежит ровно в одном фундаментальном).

Вводим пр-во разрезов (окажется ортогонально $C$ и в нём найдётся $n-1$ линейно независимый вектор,
отсюда сразу будет следовать базисность всего на свете).
Подмножество $F\subseteq E$ рёберно разделяющее, если граф $G-F$ несвязен.
Рёберный разрез $[S_1, S_2]$ (тут $S_1, S_2 \subseteq V$) "--- ребра с одним концом
в $S_1$, с другим $S_2$, дальше интересуемся только $S_2=V\setminus S_1$.
Кограница мн-ва вершин $S$: $\delta(S)=[S,V\setminus S]$.
Очевидно $\delta(S)=\delta(\bar S)$, $\delta(S)=E \iff$ граф двудолен.
Тривиальный разрез: $\delta(\{x\})$ (размера $\deg x$).
Аналогично назовём $|\delta(S)|$ степенью $S$ (в случае мультиграфа без петель).
Любая кограница есть рёберно разделяющее (обратное неверно, см. треугольник с ручкой).
Минимальный рёберный разрез "--- рёберный разрез, любое строгое подмн-во рёберным разрезом не является.
Разрез $\delta(S)$ минимален $\iff$ граф $G - \delta(S)$ имеет ровно две компоненты связности.
$\La$: взяли две компоненты, взяли собственный рёберный разрез, упс, соединились.
$\Ra$: иначе либо в $S$, либо в $\bar S$ есть две компоненты, тогда можно взять одну из них,
получим собственный рёберный разрез.
Симметрическая разность разных разрезов есть разрез: нарисовали картинки.
Тогда все разрезы образуют линейное пространство рёберных разрезов $B$.
Теперь берём дерево $T$, удаление ребра $e$ бьёт вершины на $S$ и $\bar S$,
тогда фундаментальный разрез (связанный с деревом $T$) "--- $\delta(S)$,
все фундаментальные линейно независимы.

Доказываем $C=B^\bot$.
Взяли цикл и разрез, они пересеклись по чётному числу рёбер (зашли в $S$, вышли из $S$),
$C\subseteq B^\bot$.
Теперь возьмём $F\notin C$, пересекающийся с циклом по чётному числу рёбер.
Тогда есть вершина $x$, инцидентная нечётному числу рёбер из $F$ (иначе $F$ чётен).
Догда тривиальный разрез $\delta(\{x\})$ и $F$ пересекаются по нечётному числу рёбер,
т.е. $F$ не лежит в $B^\bot$, тогда $C=B^\bot$.
Осталось показать $B=(B^\bot)^\bot$.
Любой разрез ортогонален циклу, т.е. $B\subseteq C^\bot = (B^\bot)^\bot$,
дальше считаем размерности, $\dim B = \dim (B^\bot)^\bot$.
Следствие: мы в $B$ и $C$ знаем размерности и базисы (так как знаем нужные линейно независимые,
а сумма размерностей ровно $m$).

\section{} % 10
Гамильтонов цикл "--- надо пройти все вершины.
Критериев необходимости/достаточности еще нет, искать сложно.
Необходимость: число компонент связности в $G-S$ не больше $|S|$.
Если есть гамильтонов путь, то можно достроить до цикла, если $\deg x_1 + \deg x_n \ge n$
(нашли $x_i \to x_n$, $x_1 \to x_{i+1}$, развернули).
Следствие: есть максимальный по включению путь длины $k$, тогда его можно превратить в цикл,
если сумма степеней $\ge k$ (не забыть про случай смежности вершин).
Теорема Оре: если $n>2$ и для любых двух $\deg x + \deg y \ge n - 1$, то есть гам. путь.
Док-во: любые две вершины либо связны, либо имеют общего соседа.
Пусть нет гамильтонова пути, тогда есть цикл длины $k<n$, тогда по связности
есть вершина, смежная с вершиной цикла и в него не входящая, противоречие.
Следствие: если для всех $\deg x + \deg y \ge n$, то есть гам. цикл.
Следствие (Дирак): если степень каждой $\ge \frac{n-1}{2}$, то есть путь, если
$\ge \frac{n}{2}$, то есть цикл.

Пусть $G$ "--- простой граф, есть пара несмежных $x$, $y$, причём
$\deg x + \deg y \ge n$.
Тогда в $G$ есть гам. цикл $\iff$ в $G+\{x, y\}$ есть гам. цикл.
В одну сторону понятно, в другую сторону: был цикл, выкинули ребро,
остался путь, преобразовали в цикл.
Замыкание $C(G)$ "--- добавляем такие рёбра, пока можем (степени при этом меняются).
Порядок неважен: пусть добавляли в порядке $e_1, \dots, e_r$ и $f_1, \dots, f_s$ (во втором получили граф $H$),
предположим, что $e_i$ во втором списке нет ($i$ минимально).
Тогда в $G_1=G+\{e_1, \dots, e_{i-1}\}$ верно $\deg_{G_1} x + \deg_{G_1} y \ge n$,
при этом $G_1 \le H$, но $e_i \notin H$, значит, $e_i$ можно добавить в $H$.
Бонди-Хватала: простой граф имеет гамильтонов цикл $\iff$ $C(G)$ имеет гамильтонов цикл.
Следствие: если $C(G)=K_n$, то ок.
Дирак "--- частный случай, т.к. замыкается всё до $K_n$.

Хватала: пусть в простом $G$ степенная последовательность $(d_1, \dots, d_n)$ имеет для $i<n/2$:
$d_i>i\lor d_{n-i}\ge(n-i)$.
Тогда есть гам. цикл.
Док-во: замкнули, не получили полный, взяли пару $(x, y)$ с максимальной $s=\deg x + \deg y < n$.
Пусть $k=\deg x \le \deg y$, тогда $k < \frac{n}{2}$ и $\deg y < n-k$.
Более того, так как $s$ максимально, то все несмежные с $y$ имеют степень не больше $k$,
их $n-1-\deg y$, т.е. (из $s \le n-1$) их хотя бы $\deg x = k$.
Любая несмежная с $x$ имеет степень не больше $\deg y < n-k$,
их $n-1-k$.
Сама $x$ тоже имеет степень $< n-k$, т.е. всего таких вершин $n-k$.
Нашли в замыкании $k$ вершин степени не больше $k$, $n-k$ вершин степени $<n-k$,
т.е. в исходном они тоже были, противоречие (\TODO да ладно?)

\section{} % 11
Обобщение Дирака: сильносвязный орграф, входящие/исходящие степени каждой $\ge \frac{n}{2}$,
без доказательства.
Турнир "--- полный орграф.
В любом есть гамильтонов путь.
Индукция по числу вершин, удаляем вершину, берём старый путь, вклиниваем вершину в середину.

Циклическая строка длины $k$ над алфавитом из $n$ символов, содержащая все
$k$-меры (их $n^k$) имеет длину не менее $n^k$.
Можно искать гамильтонов цикл в графе над $n^k$ вершинами,
а можно построить граф де Брёйна: $n^{k-1}$ вершин, ребро $x \to y$,
если к $x$ дописали символ, откусили начало, получили $y$.
Рёбер $n^k$, надо найти Эйлеров цикл.
Он всегда есть.
Всего циклов де Брёйна $(n!)^{n^{k-1}}/n^k$, без доказательства.

\section{} % 12
\TODO

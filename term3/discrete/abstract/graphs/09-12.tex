\section{} % 09
Всё для связного графа $G$.
Вводим пр-во циклов.
Рассмотрим мн-во подмножеств рёбер как линейное пр-во $\mathcal E = (\mathbb F_2)^m$
(плюс "--- симметрическая разность), можно ввести произведение подмножеств
(скалярно, это чётность размера пересечения).
Декомпозиция: разбили на попарно непересекающиеся по рёбрам подграфы на $V$
(можно декомпозировать на циклы/пути).
Veblen: есть декомпозиция на циклы $\iff$ граф чётен (все вершины чётной степени),
$\Ra$ очевидно, $\La$ индукция по числу рёбер.
Симметрическая разность чётных подграфов чётна
Множество чётных подграфов образует линейное подпр-во $C$ в $\mathcal E$,
оно порождается всеми циклами, называется пр-вом циклов.
Взяли остовное дерево $T$ (остаток рёбер "--- хорды, образуют кодерево).
Взяли хорду $e$, получили фундаментальный цикл, цикловый ранг $\gamma(G)$ "---
кол-во фундаментальных ($m-n+1$).
Фундаментальные линейно независимы (так как хорда лежит ровно в одном фундаментальном).

Вводим пр-во разрезов (окажется ортогонально $C$ и в нём найдётся $n-1$ линейно независимый вектор,
отсюда сразу будет следовать базисность всего на свете).
Подмножество $F\subseteq E$ рёберно разделяющее, если граф $G-F$ несвязен.
Рёберный разрез $[S_1, S_2]$ (тут $S_1, S_2 \subseteq V$) "--- ребра с одним концом
в $S_1$, с другим $S_2$, дальше интересуемся только $S_2=V\setminus S_1$.
Кограница мн-ва вершин $S$: $\delta(S)=[S,V\setminus S]$.
Очевидно $\delta(S)=\delta(\bar S)$, $\delta(S)=E \iff$ граф двудолен.
Тривиальный разрез: $\delta(\{x\})$ (размера $\deg x$).
Аналогично назовём $|\delta(S)|$ степенью $S$ (в случае мультиграфа без петель).
Любая кограница есть рёберно разделяющее (обратное неверно, см. треугольник с ручкой).
Минимальный рёберный разрез "--- рёберный разрез, любое строгое подмн-во рёберным разрезом не является.
Разрез $\delta(S)$ минимален $\iff$ граф $G - \delta(S)$ имеет ровно две компоненты связности.
$\La$: взяли две компоненты, взяли собственный рёберный разрез, упс, соединились.
$\Ra$: иначе либо в $S$, либо в $\bar S$ есть две компоненты, тогда можно взять одну из них,
получим собственный рёберный разрез.
Симметрическая разность разных разрезов есть разрез: нарисовали картинки.
Тогда все разрезы образуют линейное пространство рёберных разрезов $B$.
Теперь берём дерево $T$, удаление ребра $e$ бьёт вершины на $S$ и $\bar S$,
тогда фундаментальный разрез (связанный с деревом $T$) "--- $\delta(S)$,
все фундаментальные линейно независимы.

Доказываем $C=B^\bot$.
Взяли цикл и разрез, они пересеклись по чётному числу рёбер (зашли в $S$, вышли из $S$),
$C\subseteq B^\bot$.
Теперь возьмём $F\notin C$, пересекающийся с циклом по чётному числу рёбер.
Тогда есть вершина $x$, инцидентная нечётному числу рёбер из $F$ (иначе $F$ чётен).
Догда тривиальный разрез $\delta(\{x\})$ и $F$ пересекаются по нечётному числу рёбер,
т.е. $F$ не лежит в $B^\bot$, тогда $C=B^\bot$.
Осталось показать $B=(B^\bot)^\bot$.
Любой разрез ортогонален циклу, т.е. $B\subseteq C^\bot = (B^\bot)^\bot$,
дальше считаем размерности, $\dim B = \dim (B^\bot)^\bot$.
Следствие: мы в $B$ и $C$ знаем размерности и базисы (так как знаем нужные линейно независимые,
а сумма размерностей ровно $m$).

\section{} % 10

\section{} % 11

\section{} % 12

\section{} % 01
Неориентированный мультиграф "--- упорядоченная тройка $G=(V, E, I)$ (номера вершин, рёбер, отображения из ребра в неупорядоченную пару вершин, возможно, одинаковых).
Ребро инцидентно концевым вершинам (вершины также инциденты ребру), если $x \to x$ "--- петля, если много рёбер $x \to y$ "--- мультирёбра,
если ровно одно "--- простое ребро.
Степень вершины (валентность) "--- количество ребёр, инцидентных (петля даёт вклад два).
Изолированная "--- $\deg = 0$, регулярный граф "--- все степени равны.
Первая теорема: $\sum_{v \in V(G)} \deg(x) = 2|E(G)|$.
Следствие: вершин нечётное степени нечётно.

Приём double counting: есть простые объекты ($C$), из можно сгруппировать двумя способами, образовать два мн-ва сложных объектов ($A$ и $B$),
а потом просуммировать число простых объектов в составе сложных: $\sum a_i = \sum b_i$.
Для первой теоремы взяли $C=\text{полурёбра}$, $A=\text{степень вершины}$, $B=\text{рёбра}$.
Формализация матрицей $M_i$ инцидентности: $n=|A|$ строк, $m=|B|$ столбцов, элемент $(M_i)_{i,j}$
равен числу простых, входящих и в $A_i$, и в $B_j$, тогда можно суммировать по строкам/столбцам
(объекты инцидентны $\iff$ $(M_i)_{i,j}>0$).
Для неориентированного графа строки "--- вершины, столбцы "--- рёбра.

Простой граф: нет петель ($\iff$ в $M_i$ только 0 и 1) и мультирёбер, не простой граф "--- мультиграф.
В них можно считать, что $E \subseteq V^{(2)}$ (не $V^2$) и не надо функцию $I$.
Полный граф $K_n$ (все со всеми), пустой граф (дополняет полный, дополнение простого графа "--- вершины совпадают, множества рёбер дополняют).
Двудольный, если вершины можно разбить на дво блока ($G[X, Y]$, $X, Y$ "--- блоки), полный двудольный $K_{n,m}$, звезда "--- $K_{n,1}$.
Путь $P_n$ из $n$ вершин, цикл $C_n$ из $n$ вершин.
$W_n$ "--- колесо с $n$ вершинами во внешнем цикле (соединили весь цикл с новой вершиной).
Всё еще есть регулярные простые графы.
Есть граф Петерсена (3-регулярен; звезда плюс цикл из пяти вершин, соединили края).
$Q_k$ "--- кубы, $k$-регулярные и двудольные, вершина "--- бинарная строка длины $k$.

Орграф: тройка $D=(V,E,I)$, отображаем в упорядоченные пары вершин ($(x, y)$ выходит из $x$, входит в $y$). 
Три степени: $\deg = \outdeg + \indeg$, есть аналог теоремы: $\sum \indeg = |E| = \sum \outdeg$.
Орграф простой, если нет петель и кратных упорядоченных рёбер ($x \to y \to x$ можно).
$x$ смежна с $y$ в неориентированном: есть ребро $\{x, y\}$, для ориентированного: есть ребро $y\to x$ (несимметрично).
Матрица смежности $M_a$: размер $n \times n$, в $m_{ij}$ "--- число рёбер $i\to j$ (для неориентированного симметрична $\Ra$
собственные числа $\in \R$), в простом графе на диагонали нули (след и сумма собственных нули) а в остальных "--- 0 и 1.
Список смежности: массив $L_a$ размера $n$, каждый элемент "--- список (мультимножество) вершин, смежных с $i$ (куда идут рёбра из $i$).
Графы по умолчанию неориентированные.

\section{} % 02
Маршрут (walk) в мультиграфе: последовательность из вершин и рёбер, длина "--- число рёбер, есть начальные и конечные вершины
(они \textit{связаны маршрутом $W$}), $W$ "--- $x_0x_k$-маршрут.
В простом графе достаточно вершин, без рёбер.
Путь (trail), если все рёбра различны; простой путь, если все вершины различны (path).
Замкнутый путь (closed trail) или составной цикл (circuit) "--- путь с $x_0=x_k$.
Можно простой цикл (иногда говорят без <<простой>>).
В простых графах нет циклов меньше треугольников.
Обхват графа "--- длина наименьшего цикла (или $+\infty$).
Петля "--- цикл длины один, кратные рёбра "--- длины два.
Изолированная вершины "--- замкнутый путь длины ноль, но циклом не считают.

Вершины связанны, если есть путь (отношение эквив.); связный граф;
расстояние (длина наименьше пути, он всегда простой, м.б. $+\infty$);
диаметр "--- $\max d(x, y)$;
эксцентриситет $\epsilon(x) = \max d(x, y)$;
радиус "--- $\min \epsilon(x)$.
В орграфах связны, если есть пути туда и обратно; орграф сильно связен (все вершины связны) и бьётся на компоненты
сильной связности; слабо связен (выкинули ориентацию, получили связный).
Если есть ребро из $H_1$ в $H_2$ (это компоненты сильной связности), то нет из $H_2$ в $H_1$.
Конденсация $C(D)$ "--- граф компонент сильной связности, циклов в нём нет (DAG "--- direct acyclic graph).

\section{} % 03
$H$ "--- подграф $G$ ($G$ "--- надграф/суперграф), если вкладываются $V$, $E$ и если ребро $e \in H$, то оно соединяет ту жа пару вершин в $G$.
Подграф можно получить, удаляя рёбра ($G - e$) или вершины (при удалении вершины нужно удалить все инцидентные рёбра; $G - v$),
можно удалять подмножества рёбер/вершин: $G-S$.
Если удаляли только рёбра, то получим остовный подграф (spanning).
Если только вершины, то получим подграф, индукцированный подмножеством оставшихся вершин.
Остовный путь "--- гамильтонов путь, остовный цикл "--- гамильтонов цикл.
Остовный 1-регулярный подграф "--- \textit{совершенное} паросочетание (только для неориентированных).
Остовный $k$-регулярный "--- $k$-фактор.
Из мультиграфа можно выкинуть петли и сократить мультирёбра, получим остовный подграф, простой.

Ребро "--- мост, если $G-e$ имеет на 1 больше компонент связности, чем $G$.
Вершина "--- точка сочленения, если $G-v$ имеет больше компонент связности (может, сильно больше).
$x$ в графе на $\ge 3$ вершинах "--- точка сочленения $\iff$ есть другие $y$ и $z$ такие, что любой путь $y\to z$ содержит $x$.
$e$ "--- мост в простом связном (связность не нужна, так проще доказывать и всегда можно распилить на компоненты) $\iff$ $e$ не принадлежит ни одному из циклов.
Это были аналоги теорем Менгера для $k$-связных графов.

Объединение графов $G\cup H$: формально объединили мн-ва вершин и рёбер (у вершин есть метки, у рёбер, видимо, нет? \TODO), если не пересекались по вершинам "--- получили $G+H$.
Пересечение "--- аналогично.
Симметрическая разность остовных подграфов графа $G$: $H_1 \Delta H_2$, взяли рёбра, лежащие ровно в одном из $H_1$, $H_2$.
Стягивание ребра: $G \backslash e$, удалили $e$, стянули концы в одну вершину.

\section{} % 04
Простых графов на $n$ вершинах $2^{\binom{n}{2}}$, для мультиграфов сложнее "--- надо ограничить число рёбер (не делали).
Различных простых орграфов $2^{n(n-1)}$
Два простых графы изоморфны, есть есть биекция между вершинами, сохраняющая смежность.
Класс эквивалентности по изоморфности называется непомеченным графом (а обычный граф "--- помеченным).
Для неориентированных мультиграфов надо биективно отображать еще и рёбра.
Для проверки изоморфность удобно смотреть на инварианты: степень вершины, самый длинный/короткий цикл.
Программно можно за $n!$.

Автоморфизм "--- перестановка вершин графа, дающая тот же самый \textit{помеченный} граф, с теми же списками смежности (т.е. не любой изоморфизм $G$ и $G$ есть автоморфизм $G$).
Получили группу автоморфизмов $\Aut(G)$, для $P_3$ это $\Z_2$.
Для непомеченных пометим как-нибудь, получим группу (\TODO независимо от исходных пометок группы совпадут?).
Количество помеченных графов из данного непомеченного есть $\frac{n!}{|\Aut(G)|}$.
Док-во с теорией групп (\TODO надо ли на простом языке?): берём $X$ "--- мн-во помеченных графов, группу $S_n$, действующую на нём, графы изоморфны $\iff$ лежат на одной орбите;
$\Aut(G)$ "--- в точности стабилизатор элемента $G$, применили Orbit-Stabilizer Theorem ($|G\sigma| \cdot |\St_G| = |S_n|$), получили что надо.
В конспекте эта теорема просто доказывается еще раз, как мы делали на алгебре (классы смежности, теорема Лагранжа и всё это).
Граф асимметричен, если $|\Aut(G)|=1$ (група тривиальна).
В конспекте оценивают число непомеченных графов за $\frac{2^{\binom{n}{2}}}{n!}$ (\TODO а не офигели ли?).

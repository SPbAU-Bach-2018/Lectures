\section{} % 01
Неориентированный мультиграф "--- упорядоченная тройка $G=(V, E, I)$ (номера вершин, рёбер, отображения из ребра в неупорядоченную пару вершин, возможно, одинаковых).
Ребро инцидентно концевым вершинам (вершины также инциденты ребру), если $x \to x$ "--- петля, если много рёбер $x \to y$ "--- мультирёбра,
если ровно одно "--- простое ребро.
Степень вершины (валентность) "--- количество ребёр, инцидентных (петля даёт вклад два).
Изолированная "--- $\deg = 0$, регулярный граф "--- все степени равны.
Первая теорема: $\sum_{v \in V(G)} \deg(x) = 2|E(G)|$.
Следствие: вершин нечётное степени нечётно.

Приём double counting: есть простые объекты ($C$), из можно сгруппировать двумя способами, образовать два мн-ва сложных объектов ($A$ и $B$),
а потом просуммировать число простых объектов в составе сложных: $\sum a_i = \sum b_i$.
Для первой теоремы взяли $C=\text{полурёбра}$, $A=\text{степень вершины}$, $B=\text{рёбра}$.
Формализация матрицей $M_i$ инцидентности: $n=|A|$ строк, $m=|B|$ столбцов, элемент $(M_i)_{i,j}$
равен числу простых, входящих и в $A_i$, и в $B_j$, тогда можно суммировать по строкам/столбцам
(объекты инцидентны $\iff$ $(M_i)_{i,j}>0$).
Для неориентированного графа строки "--- вершины, столбцы "--- рёбра.

Простой граф: нет петель ($\iff$ в $M_i$ только 0 и 1) и мультирёбер, не простой граф "--- мультиграф.
В них можно считать, что $E \subseteq V^{(2)}$ (не $V^2$) и не надо функцию $I$.
Полный граф $K_n$ (все со всеми), пустой граф (дополняет полный, дополнение простого графа "--- вершины совпадают, множества рёбер дополняют).
Двудольный, если вершины можно разбить на дво блока ($G[X, Y]$, $X, Y$ "--- блоки), полный двудольный $K_{n,m}$, звезда "--- $K_{n,1}$.
Путь $P_n$ из $n$ вершин, цикл $C_n$ из $n$ вершин (\TODO в конспекте рисунки не соответствуют определению).
$W_n$ "--- колесо с $n$ вершинами во внешнем цикле (соединили весь цикл с новой вершиной).
Всё еще есть регулярные графы.
Есть граф Петерсена (звезда плюс цикл из пяти вершин, соединили края).
$Q_k$ "--- кубы, $k$-регулярные и двудольные, вершина "--- бинарная строка длины $k$.

Орграф: тройка $D=(V,E,I)$, отображаем в упорядоченные пары вершин ($(x, y)$ выходит из $x$, входит в $y$). 
Три степени: $\deg = \outdeg + \indeg$, есть аналог теоремы: $\sum \indeg = |E| = \sum \outdeg$.
Орграф простой, если нет петель и кратных упорядоченных рёбер ($x \to y \to x$ можно).
$x$ смежна с $y$ в неориентированном: есть ребро $\{x, y\}$, для ориентированного: есть ребро $y\to x$ (несимметрично).
Матрица смежности $M_a$: размер $n \times n$, в $m_{ij}$ "--- число рёбер $i\to j$ (для неориентированного симметрична $\Ra$
собственные числа $\in \R$), в простом графе на диагонали нули (след и сумма собственных нули) а в остальных "--- 0 и 1.
Список смежности: массив $L_a$ размера $n$, каждый элемент "--- список (мультимножество) вершин, смежных с $i$ (куда идут рёбра из $i$).

\section{} % 02
Маршрут (walk) в мультиграфе: последовательность из вершин и рёбер, длина "--- число рёбер, есть начальные и конечные вершины
(они \textit{связаны маршрутом $W$}), $W$ "--- $x_0x_k$-маршрут.
В простом графе достаточно вершин, без рёбер.
Путь (trail), если все рёбра различны; простой путь, если все вершины различны (path).
Замкнутый путь (closed trail) или составной цикл (circuit) "--- путь с $x_0=x_k$.
Можно простой цикл (иногда говорят без <<простой>>).
В простых графах нет циклов меньше треугольников.
Обхват графа "--- длина наименьшего цикла (или $+\infty$).
Петля "--- цикл длины один, кратные рёбра "--- длины два.
Изолированная вершины "--- замкнутый путь длины ноль, но циклом не считают.

Вершины связанны, если есть путь (отношение эквив.); связный граф;
расстояние (длина наименьше пути, он всегда простой, м.б. $+\infty$);
диаметр "--- $\max d(x, y)$;
эксцентриситет $\epsilon(x) = \max d(x, y)$;
радиус "--- $\min \epsilon(x)$.
В орграфах связны, если есть пути туда и обратно; орграф сильно связен (все вершины связны) и бьётся на компоненты
сильной связности; слабо связен (выкинули ориентацию, получили связный).
Если есть ребро из $H_1$ в $H_2$ (это компоненты сильной связности), то нет из $H_2$ в $H_1$.
Конденсация $C(D)$ "--- граф компонент сильной связности, циклов в нём нет (DAG "--- direct acyclic graph).

\section{} % 03

\section{} % 04


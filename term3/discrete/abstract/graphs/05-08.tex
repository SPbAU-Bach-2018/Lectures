\section{} % 05
Дерево "--- простой связный без циклов, лес "--- простой без циклов.
Лист "--- степень один.
У дерева при $|V|\ge 2$ хотя бы два листа (см. путь максимальной длины).
В дереве $|E|=|V|-1$ (индукция по $n=|V|$, удаляем один лист).
Любой простой связный с $n-1$ ребром "--- дерево (обходим граф dfs'ом "--- тут он в первый раз введён "---, крася посещённые вершины и рёбра, циклов в покрашенных
рёбрах не появится, покрасим в конце всё).
Всякий связный имеет $\ge n-1$ ребра (взяли связный, удалили ребро из произвольного цикла, пока не дерево).
То есть у каждого связного есть остовное дерево (можно либо рёбра удалять, либо dfs).
Dfs'ом также можно искать компоненты связности.
Простой связный минимально связен (нельзя удалить ребро) $\iff$ дерево.
Любое ребро в дереве "--- мост.
Дерево $\iff$ любые две вершины соединяет ровно один простой путь.
Дерево плюс ребро содержит единственный цикл.
Если есть два остова, то можно выкинуть любое ребро из первого, добавить другое ребро из второго, снова получить остов.

Корневое дерево "--- выделили корень, в лесу "--- выделили в каждой компоненте.
В корневом дереве есть единственный путь из корня до $x$, его длина "--- уровень $x$;
есть высота дерева.
Удобно ориентировать от корня к листьям (входит не более одного ребра, листы "--- стоки \TODO кроме $n=2$?),
есть родители и дети, есть предки (прапрародители) и потомки.
$m$-арное дерево "--- максимальная исходящая степень равна $m$.
Можно ввести ЧУМ из вершин: $x \prec y$, если $x$ есть предок $y$.

Формула Кэли: количество разных помеченных деревьев равно $n^{n-2}$.
Сопоставляем дереву последовательность: выбрали лист с минимальным номером, удалили,
записали номер единственной смежной вершины.
Обратно: после $k$ шагов берём минимальную вершину, которая не встречается дальше в коде, она и была
удалена на этом шаге.
Есть отображение и обратное $\Ra$ биекция $\Ra$ посчитали последовательности, их $n^{n-2}$.

\section{} % 06
Берём ребро-не петлю $e$ в мультграфе, тогда число остовов $t(G)=t(G-e)+t(G\backslash e)$ (либо не лежит в остове, либо лежит и по нему стянули).
В процессе стягиваний появляются мультирёбра, петли (их надо игнорить), граф может стать несвязным.
Конечное условие "--- любое ребро есть петля, д.б. одна вершина.
Для связных графов без петель можно построить матрицу Лапласа $L=M_d=M_a$ ($M_a$ "--- смежности; $M_d$ "--- на диагонали степени вершин),
выкинуть произвольную строку и столбец, получить $L^*$, тогда $t(G)=\det L^*$.
Док-во:

Ориентируем рёбра в $G$ как угодно, получили орграф $D$ с матрицей инцидентности $M_i$, тогда $L=M_i \cdot M_i^T$.
Удалили из $M_i$ произвольную строку, получили $M_i^*$, $L^*=M_i^*\cdot (M_i^*)^\top$.
Пусть $B$ "--- квадратная подматрица $(n-1)\times(n-1)$ у $M_i^*$, строки $B$ "--- все вершины $D$ без одной, столбцы $B$ "--- набор из $n-1$
ориентированного ребра в $D$, а в $G$ образуют подграф $H$.
Если $H$ дерево, то $\det B = \pm 1$: есть хотя бы два листа, в $B$ есть строка для одного из них ($x$),
там единственный не-ноль, разложили по этой строке, получили матрицу для подграфа $H-x$, в конце получили остов на двух вершинах,
для него всё ок.
А если $H$ "--- не дерево, то в нём есть цикл, тогда есть нетривиальная линейная комбинация столбцов-рёбер, дающая ноль
(взяли цикл, обошли рёбра, возможно, домножили на $-1$, если ориентация не устроила).
Теперь формула Бине-Коши (без док-ва: $\det L^* = \det M_i^* \cdot (M_i^*)^T = \sum (\det B)^2$, сумма по всем возможным $B$.
А каждое возможное $B$ "--- разный набор из $n-1$ ребра, успех.

\section{} % 07
Напоминание про собственные: число, вектор, характеристический многочлен $\det(A-\lambda E)$, собственные вектора для числа $\lambda$ задают собственное подпространство $\Ker (A-\lambda E)$,
его размерность "--- геометрическая кратность $\lambda$ ($\gamma(\lambda)$), алгебраическая кратность (у многочлена, не меньше геометрической).
Спектр матрицы: $\begin{pmatrix}\lambda_1 & \lambda_2 & \lambda_3 \\ \gamma(\lambda_1) & \gamma(\lambda_2) & \gamma(\lambda_3) \end{pmatrix}$,
если $\rang A = k < n$, то есть $\gamma(0)=n-k$.
Если к диагонали добавить $c$, то все собственные увеличатся на $c$.

Пусть $\lambda_i$ "--- собственные числа $L(G)$ (матрица Лапласа) и $\lambda_1=0$ (так как сумма всех строк $L$ ноль), тогда
$t(G)=\frac{1}{n} \prod_2^k \lambda_i$.
Характеристический многочлен: $\lambda \cdot (\lambda-\lambda_2) \cdot \dots \dots (\lambda-\lambda_n)$, найдём коэффициент
при $\lambda^1$, он равен $(-1)^{n-1} \prod_2^k \lambda_i$, а также равен значению производной характеристического при $\lambda=0$ (младшему коэффициенту).
Запишем матрицу по столбцам $A(t)=(a_1(t), \dots, a_n(t))$, раскроем производную определителя: $\det(a_1'(t), \dots, a_n(t)) + \dots + \det(a_1(t), \dots, a_n'(t))$
(потому что каждое слагаемое определителя после дифференцирования распадётся на $n$ штук такого вида).
Дальше каждый из таких разложим по столбцу $j$ с производной: $\sum_{i=1}^n (-1)^{i+j}a'_{i,j}(t)\cdot M_{i,j}(t)$, тут $M$ "--- минор (вычеркнули строку+столбец, посчитали определитель).
Теперь записали матрицу $C$ из алгебраических дополнений (минор со знаком), получили, что производная определителя есть след $C^\top \cdot A'(t)$ (матрица $C^\top$ называется союзной
или присоединённой матрицей), при этом $A'(t)=E$.
Получили сумму миноров, при $\lambda=0$ это будет сумма определителей матриц $L^*$ (со знаком), получаем что надо.

\section{} % 08
Берём орграф, аналоги матрицы Лапласа: $L^{-}=M_d^{-}-M_a$ (на диагонали исходящие степени),
$L^{+}=M_d^{+}-M_a$ (на диагонали входящие степени), у $L^-$ сумма в строке ноль,
у $L^+$ сумма в столбце ноль.
Теперь берём корневые деревья внутри орграфа, где рёбра ориентированы к корню
(у них у всех вершин $\outdeg=1$).
Теорема Татта: количество $t^-(D, i)$ корневых деревьев с ориентацией к корню $i$
равно определителю $L^-$, из которой выкинули строку и столбец номер $i$.
Перенумеровали вершины так, чтобы $i=1$.
Док-во: если в орграфе $D$ есть вершина не-1 с $\outdeg=0$, то деревьев нет,
а соотв. строчка в $L^-$ будет ноль ($\det=0$).
Если есть вершина $x$ степени $>1$, то можно разложить строку $x$ матрицы $L^-$
на сумму $\deg x$ строк, разложить определитель, получим сумму определителей для графов,
где из $x$ выходит ровно одно ребро.
Так распилили по всем вершинам (не включая 1, строчка-то выкинута), получили матрицу $L^-$ с единицами на диагонали.
Теперь в каждом таком графе либо исходящие рёбра (из вершин, кроме 1) образуют дерево, либо нет.
Пусть есть цикл, не проходящий через вершину 1, тогда он проходит по некоторым вершинам,
соответствующие строчки матрицы дадут сумму ноль, т.е. определитель ноль, что и надо.
Если любой цикл проходит через 1, то в графе есть единственное дерево.
Аналогично, если циклов нет (т.е. из 1 не выходит рёбер), то тоже единственное дерево.
Осталось показать, что тогда $\det=1$.
Индукция по $n$, берём строчку вершины $x$ (есть ребро $x\to1$), в ней единственная
единица (причём на диагонали!), разложили определитель по строке, остатку соответствует старый граф с выкинутой
$x$, а её дети переподвешены к единице.
Следствие: матричная теорема о деревьях: интерпретируем неориентированный как орграф с рёбрами в две стороны,
применяем теорему, а корневые деревья с корнем в $x$ "--- в точности все неориентированные деревья.

В эйлеровом орграфе $D$ число эйлеровых циклов $e(D)$ равно $t^-(D, x)\cdot\prod(\outdeg y-1)!$, где $x$ любая
(следствие: $t^-(D, x)$ не зависит от $x$).
Берём $x$, фиксируем из неё ребро $x\to y$, каждом циклу по рёбрам $e_1, \dots, e_m$ сопоставляем следующий набор рёбер:
берём из каждой вершины (кроме $x$) ребро с максимальным номером $e_i$.
Это либо дерево, сходящееся к $x$, либо там есть цикл (причём не проходящий через $x$).
Но цикла быть не может: возьмём в таком цикле место, где мы взяли рёбра $e_a$ и $e_b$, причём
$a>b$ (такое место найдётся, т.к. цикл), тогда из промежуточной вершины выходит $e_{a+1}$,
тогда вместо $e_b$ надо было брать $e_{a+1}$ (так как промежуточная не $x$, $e_{a+1}\neq a_1$).
Каждому циклу сопоставили какое-то сходящееся дерево.
Теперь посмотрим, как можно было получить конкретное дерево.
В каждой вершине выбрали порядок исходящих рёбер, не лежащих в дереве;
в вершине $x$ одно ребро изначально фиксировано как стартовое.
Любой эйлеров цикл однозначно этими порядками характеризуется, т.е.
если для любого порядка есть цикл, то из дерева получим $\prod(\outdeg y - 1)!$ циклов.
Док-во: идём из $x$, на каждом шаге выбираем первое непосещённое исходящее ребро,
когда-то вернёмся в $x$.
Все исходящие из $x$ посетили $\Ra$ все входящие в $x$ посетили, в том числе
рёбра дерева.
Посмотрели на детей $x$, т.к. из каждого пошли по ребру дерева, то из каждого
уже посетили все исходящие рёбра $\Ra$ все входящие тоже посетили, в том числе
рёбра дерева.
И так далее по уровням, получили, что все рёбра посетили.

\setauthor{Бугакова Надежда}
Стек любой вычислительной системы:
\begin{itemize}
	\item Операционная система
	\item Архитектура процессора
	\item Микроархитектура
	\item Цифровые схемы
	\item Аналоговые цепи
	\item Элементы: транзисторы, конденсаторы
	\item Физические сигналы
\end{itemize}

\underline{Элементы аналоговой системы:}
\begin{enumerate}
	\item Резистор - ограничивает течение электрического тока. 
	\cimg{01_3}{0.3}

%	\begin{center}
%		\includegraphics{01_3}
%	\end{center}

	Чтобы ток тёк, нужна разность потенциалов. Сопротивление  - некая величина в Омах, которая обратно пропорциональна проводимости. 
	Медный проводник обладает очень маленьким сопротивлением.
	\item Конденсатор - накапливает заряд. 
	\cimg{01_1.png}{0.2}
	\item Диод - пускает ток только в одном направлении. \cimg{01_5}{0.2}
	\item Источник тока. У батарейки "+" там, где выступ.
\cimg{01_2.png}{0.2}
	\item Катушка индуктивности.\cimg{01_6}{0.2}
	\item Некоторые элементы обладают переменным потенциалом. \cimg{01_4}{0.1} Реостат - резистор с переменным номиналом.
	\item Транзистор. \cimg{01_7}{0.1} K - область с высоким потенциалом, Э - область с низким потенциалом. Если из базы(Б) тока нет, тогда
	переход(транзистор закрыт), иначе ток идёт из К в Э. 
	\underline{Пример:}
		\cimg{01_8}{0.5}
	Процессор может пропускать через себя мало тока, иначе сгорит. А у мотора очень маленькое сопротивление, поэтому ток будет большим. Если мы их соединим напрямую и
	подключим к нормальному напряжению, процессор может сгореть. Поэтому используем транзистор.
	может сгореть 
\end{enumerate}

На схемах уровни более высокого потенциала принято рисовать выше. Поэтому всё, что выше заземления обладает потенциалом.

В аналоговых цепях для нас важны уровни напряжения, но мы будем рассматривать цифровые, где сигнал дискретный(0,1). Для цифровых схем напряжение
и уровень сигнала не важны. Важно лишь наличие сигналов определённого уровня, которые мы называем 0 или 1.

Попробуем дискретизировать сигнал:

\begin{center}
----------------------------------------------------------- +5

1

----------------------------------------------------------- +2.5

0

------------------------------------------------------------ 0\end{center}

Такая дискретизация плоха тем, что мы не понимаем, как воспринимать сигнал в районе 2.5.

Поэтому делают так: \cimg{01_9}{0.3} Запас для скорости переключения сигнала:
\begin{itemize}
	\item Идеальный мир: сигнал \cimg{01_10}{0.5}
	\item Реальный мир: сигнал \cimg{01_11}{0.5}
\end{itemize}

При дискретизации сигнал будет запаздывать, значит, нужен запас.

Нам нужен аналоговый прибор, который будет давать 0 или 1 в зависимости от напряжения. Он называется операционный усилитель. 
Два входа, один выход. \cimg{01_12}{0.2}
Работает так: $\begin{cases} A > B \Rightarrow C \rightarrow V_{CC} \\
										 A < B \Rightarrow C \rightarrow 0\end{cases}$ $V_{CC}$ - напряжение '+' цифровой сети.

Аналоговый сигнал $\rightarrow$ цифровой сигнал:
\begin{itemize}
	\item Однобитный аналогово-цифровой преобразователь. \cimg{01_13}{0.25}
	\item Двухбитный аналогово-цифровой преобразователь. \cimg{01_14}{0.25}
\end{itemize}

Битность - точность представления чисел.

Всё про аналоговые, цифровые сигналы и их связь мы поняли. Теперь будем работать только с цифровыми сигналами.

Принцип управления сложностью:
\begin{itemize}
	\item Стандартизация(регулярность) - одни и те же технические решения должны обладать одинаковым интерфейсом.
	\item Модульность - можем заменять какие-то части схемы на другие с тем же интерфейсом.
	\item Иерархическая организация - соединение независимых модулей для чего-то более сложного.
\end{itemize}

Логическая схема - схема, состоящая из основных логических элементов: И, НЕ, ИЛИ, И-НЕ, ИЛИ-НЕ, XOR \dots и логических функций. 
Схемы бывают комбинаторные и последовательные.

Комбинаторные схема - схема, значение выходов которой определяется её входами. Такая схема описывается количеством входов, выходов + функциональной
спецификацией(таблицей истинности).

Пример: x ----- НЕ ----- y 

\[
\begin{tabular}{c | c}
\hline
x & y \\
\hline
0 & 1 \\
\hline
1 & 0 \\
\hline
\end{tabular}
\]
Последовательная схема - схема, значение выходов которой зависит от входов и предыдущих состояний.

Обозначения:
\cimg{01_15}{0.5}
\begin{itemize}
	\item ИЛИ
	\item И
	\item НЕ
	\item БУФЕР
	\item ИЛИ-НЕ
	\item Иногда встречаются многовходовые.
\end{itemize}



\[
	\begin{tabular}{cccc}
	\hline
	Data1&Data2&Select&Out \\
	\hline
	0&0&0&0 \\
	\hline 
	0&1&0&0 \\
	\hline
	1&0&0&1 \\
	\hline
	1&1&0&1 \\
	\hline
	0&0&1&0 \\
	\hline
	0&1&1&0 \\
	\hline
	1&0&1&0 \\
	\hline
	1&1&1&1 \\
	\hline
	\end{tabular}
\]    = if S D1 D2

\cimg{01_17}{0.4}

\underline{Дешифратор:}

\cimg{01_16}{0.2}

\[
	\begin{tabular}{cccccc}
	\hline
	A&B&$X_1$&$X_2$&$X_3$&$X_4$ \\
	\hline
	0&0&1&0&0&0 \\
	\hline
	0&1&0&1&0&0 \\
	\hline
	1&0&0&0&1&0 \\
	\hline
	1&1&0&0&1 \\
	\hline
	\end{tabular}
\]
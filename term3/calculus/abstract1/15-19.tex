\section{} % 15
Несобственные бывают с одним концом, надо непрерывность на $[a, b)$, несобственный "--- предел обычных на $[a, B]$ с $B \to b$.
Если функция непрерывна на $[a, b]$, то ничего нового.
Критерий Коши сходимости на $[a, b)$ (и симметрично для $(a, b]$): для любого $\epsilon$ есть хвост, на котором интегралы по любом отрезку ограничены $\iff$ сходится.
Док-во: определили $F(x)=\int_a^x$, взяли равносильность для $\exists \lim_{b\to B-} F(b)$.

Если есть последовательность $A_n, B_n \to b$ такая, что интегралы большие, то расходится.
Если есть первообразная, то несобственный "--- разность предела $\iff$ первообразная непрерывно продолжается.
Интеграл $\int_1^\infty x^{-p}$ сходится при $p>1$ и равен $\frac{1}{p-1}$ (посчитали через первообразную).
А $\int_0^1$ сходится при $p<1$ и равен $\frac1{1-p}$.

\section{} % 16
Если $c \in (a,b)$, то $\int_a^b$ и $\int_c^b$ сходятся/расходятся одновременно, причём если сходятся, то аддитивно.
Если $\int_a^b$ сходится, то устремив низ к $b$, значение интеграла устремится к нулю.
Если сходятся интегралы от $f$ и $g$, то сходится линейная комбинация.
Сумма сходящегося и расходящегося расходится, а вот два расходящихся могут друг друга компенсировать.
Монотонность: если $f \le g$, то $\int f \le \int g$ (если оба сходятся в $\bar \R$).
По частям: если $f$, $g$ непрерывны на $[a, b)$, то $\int fg' = fg - \int f'g$ (для существования любого слагаемого достаточно двух других).
Замена переменной: \TODO % страница 24

\section{} % 17
Знакопостоянство будем считать неотрицательностью.
Если $0 \le f \le g$ и непрерывны на $[a, b)$, то есть два утверждения вида <<(ра)сходится $X$ $\Ra$ (ра)сходится $B$>>.
Если $f=O(g)$ и $\int g$ сходится, то $\int f$ тоже.
Если $f(x) \sim g(x)$ при $x\to b-$, то ведут себя одинаково.
Если $f=O(\frac{1}{1+{1+\epsilon}})$ на бесконечности, то сходится.
Если $f\ge \frac cx$ на бесконечности, то расходится.
Пример: везде ноль, если галочки $[n-\frac1{2^n}, n+\frac1{2^n}$ до $n$, хотим ограничить первообразную (она есть сумма чего-то),
распилили знаменатель пополам, одну часть оценили константой, а интеграл от другой сходится.

\section{} % 18
Сходится абсолютно, если сходится модуль.
Отсюда следует обычная (док-во по критерию Коши, раскрываем модули).
Признак Дирихле: $\int_a^{\infty} f(x)g(x) \d x$ сходится, если первообразная $F$ ограничена, а $g$ монотонно стремится к нулю ($f$ непрерывна, $g$ дифференцируема).
Залезли под предел, раскрыли по частям (тут нужна $g'$), первое слагаемое сходится, а для второго докажем абсолютную сходимость ($g' \ge 0$, убили модуль, знаем первообразную для $g'$).
На самом деле непрерывная дифф. $g$ не нужна, но без неё сложно.

\section{} % 19
Если $\int f$ сходится, а $g$ монотонна и ограничена, то сходится $\int fg$.
Взяли первообразную $f$, она ограничена, для $g$ нашли предел, сделали её сходящейся к нулю, применили Дирихле.

Если $f$ непрерывна и $T$-периодична, а $g$ монотонно стремится к нулю (и дифф.), то если интеграл по периоду ноль, то $\int fg=0$.
Док-во: первообразная $F$ ограничена, применили Дирихле.
Если по периоду не ноль, то равносильна сходимость $\int_a fg$ и $\int_a^{a+T} g$. \TODO % wtf?
В одну сторону очевидно, в другую: вычли из $f$ что-то, получили интеграл по периоду ноль, воспользовались доказанной стрелочкой,
получили три слагаемых, сходимость двух равносильна (третье сходится).

Взяли $\int_1^{\infty} \frac{\sin x}{x^p} \d x$, при $p\le 0$ расходится (по критерию Коши).
При $p>0$ просто сходится. Для абсолютной сходимости надо $p>1$ (т.к. интеграл синуса по периоду не ноль и нужна сходимость $\int\frac{1}{x^p}$.

\section{} % 20
Есть открытое $G$, последовательность открытых $G_i$ его монотонно исчерпывает, если $G_k \supset \cl G_{k-1}$ и $G=\cup G_k$.
Любое открытое можно так исчерпать измеримыми: нарезали $\R^n$ на замкнутые кубики с единичной стороной, взяли влезшие в $G$, объединили, взяли внутренность,
потом уменьшили сторону в два раза и так далее.
Потом для каждого множества $G_i$ найдём следующее, покрывающее $\cl G_i$ (а между ними всё выкинем).
Взяли для каждой точки $\cl G_i$ расстояние до $\delta G$, выбрали минимум (компакт же), значит, при достаточно мелких кубиках у нас покроются вообще все точки из $\cl G_i$.
Отдельно по точкам нельзя: их много, а нам надо конкретное число сказать.

$f \colon G \to \R^n$ ($G$ открыто) интегрируема на любом измеримом $D \subset G$ таких, что $\cl D \subset G$ $\eqDef$
любая последовательность измеримых исчерпывающих $G_k$ такая, что интеграл по ней сходится к некоторому $A$.
В случае $n=1$ у нас были другие определения: там мы брали не какие угодны исчерпывающие $G_i$, а только вида <<луч в бесконечность>>,
поэтому там условие сходимости слабее.

\section{} % 21
Если $f \ge 0$, то для любой последовательности $G_k$ предел интегралов есть: конечный/бесконечные, причём предел не зависит от выбора последовательности.
Док-во: последовательность интегралов неубывает, значит, есть предел (возможно, бесконечный).
Единственность: пусть есть другая $G_k'$.
Замыкание каждого $G_k'$ есть компакт, он покрыт конечным числом $G_k$, но они вложены, значит, можно выбрать самый большой.
Тогда интегралы по $G_k'$ не больше интегралов по $G_k$, то есть предел не больше, из-за симметричности они равны.

Если $f \ge g \ge 0$, то: $f$ сходится $\Ra$ $g$ сходится; $g$ расходится $\Ra$ $f$ расходится.
Док-во: взяли любую последовательность, сделали на ней вывод, для остальных последовательностей всё будет ок.

\section{} % 22
Абсолютная сходимость равносильна обычной ($n>1$, при $n=1$ другое определение, послабее "--- там можно исчерпывать только определённым видом).
Из абсолютной в обычную: распилим $f$ на $f_+$ и $f_-$, если сходится абсолютно "--- сходятся и $f_\pm$, т.е. сходится разность.
Обратно от противного: пусть обычно сходится, а $\int |f|$ "--- нет.
\TODO % теорема 5.4.3

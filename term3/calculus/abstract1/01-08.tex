\section{} % 01
Сведение: $E$ измеримо, на $\Omega=E\times[a,b]$ есть \textit{интегрируемая} (надо знать заранее), можно взять $\int_a^b f$ в точках $E$ $\Ra$ тогда $g$ интегрируема на $E$.
Док-во: нам дали $\epsilon$, спросили у $f$, какой ей нужен $\delta$, взяли любое разбиение $E_i, \eta_i$ мелкости $\frac \delta 2$.
Теперь нашли разбиение $[a,b]$ мелкости $<\frac\delta 2$, чтобы во всех $\eta_i$ (их конечно) хорошо приблизить $g$ (на $\frac{\epsilon}{\mu E}$), перемножили
два разбиения, получили разбиение $\Omega$ мелкости $<\delta$, которое хорошо приближает $I$.
Т.е. хорошо приблизили интеграл $g$ разбиением $\Omega$, что успех.

Перестановка: если $f$ интегрируема на квадратике, можно переставлять (если не интегрируема "--- взяли $\frac{x^2-y^2}{(x^2+y^2)^2}$ $\Ra$ разный результат).
Элементарное множество: пририсовали к $E$ цилиндры между двумя графиками (графики непрерывны на замыкании);
если $f$ интегрируема и $\exists$ интегралы в точках $E$, то можно свести (док-во: продлили ограничивающие нулями, применили теорему).

\section{} % 02
% Лемма 3.6.1
Есть непрерывно дифференцируемая непрерывная функция на замыкании выпуклого компакта, $f \colon \R^{n-1} \to \R^n$, тогда мера образа ноль.
Док-во: разложили $f$ покоординатно, частные производные равномерно непрерывны (есть некое $M$) и ограничены.
Так как выпукло, можно ограничить разброс образов $M$ при помощи производных и т.Лагранжа.
Окружили компакт кубиком стороной $a$, покрошили на мелкие $\delta$-куски, диаметр образа каждого ограничен $M \delta$ (можно тоже вписать в куб), посчитали количество кубов в $\R^{n-1}$,
посчитали максимальную меру образов в $\R^n$, упс.
Вообще можно для просто связных, но нам не надо.
Без дифференцируемости неверно.

Условие (*) на функцию $\phi \colon \cl G \to D \subset \R^n$: она биективна, непрерывна, непрерывно дифференцируема.

% Лемма 3.6.2
Пусть есть функция из (*), тогда $\phi$ инвариантна относительно
разности множеств (т.к. биективна),
замыкания (два включения: $E \subset \cl E$, взяли $\phi$, взяли замыкание, а $\cl E$ компакт, чей образ тоже замкнут; взяли точку из $\cl E$, сказали про предел последовательности и непрерывность $\phi$),
включения (аналогичные два включения: $\phi(\Int E)$ точно открыто; взяли $a \in \Int \phi(E)$, взяли шарик вокруг, потом по биективности и открытости шарика, успех)
и взятия границы (следует из первых трёх).

\section{} % 03
Линейная составляющая ($\phi^{\mathup{lin}}_{t_0} (t)$) есть $\phi(t_0)$ плюс линейный член Тейлора (производная на $(t-t_0)$).
Лемма: если $G$ ограничено и $\phi \colon \cl G \to \R^n$ непрерывно дифференцируема, то ввели $S_\delta$ как максимум
по всем точкам из $B_\delta(t_0)$, соединёнными отрезками с $t_0$ штуки: норма отклонения функции от линейной части в $t$ на $\|t-t_0\|$.
Тогда $S_\delta \to 0$ при $\delta \to 0$ (то есть линейная часть действительно приближает функцию до первого порядка).

Пусть $g(t)$ "--- отклонение (в $t_0$ "--- ноль) и для каждого $t$ введём $h(\theta) \colon [0,1] \to \R^n$ (выдаёт значение $g$ на отрезке $[t_0, t]$),
нас интересует $g(t)=\|h(1)-h(0)\|$, оно по Лагранжу для векторнозначных \textit{не больше} $\|\h'(\theta_0)\|$ для нек. $\theta_0$.
Продифференцировали $h$ (с дифференциалами, она векторнозначна) как сложную функцию точке $u$ (соответствует $\theta_0$), оценили норму $\d_u g$
(она равен $\d_u \phi - \d_{t_0} \phi$, второе слагаемое вылезло из линейной части), подставили оценку на числитель, всё сократилось, получилась $\|\d_u \phi - \d_{t_0} \phi\|$.

Теперь её оценим как максимум по всем таким $u$ в небольшом шарике (ослабили исходное условие, нам лишь хуже).
Оператор есть разность частных производных в $t_0$ и в $u$, а так как они непрерывны, то при $\delta\to0$ разность $\to 0$ $\Ra$ норма стремится к нулю
(кстати, норма "--- это максимальная длина образа единичного вектора под действием оператора).

\section{} % 04
\TODO

\section{} % 05
\TODO

\section{} % 06
\TODO

\section{} % 07
\TODO

\section{} % 08
\TODO

\section{} % 01
Сведение: $E$ измеримо, на $\Omega=E\times[a,b]$ есть \textit{интегрируемая} (надо знать заранее), можно взять $\int_a^b f$ в точках $E$ $\Ra$ тогда $g$ интегрируема на $E$.
Док-во: нам дали $\epsilon$, спросили у $f$, какой ей нужен $\delta$, взяли любое разбиение $E_i, \eta_i$ мелкости $\frac \delta 2$.
Теперь нашли разбиение $[a,b]$ мелкости $<\frac\delta 2$, чтобы во всех $\eta_i$ (их конечно) хорошо приблизить $g$ (на $\frac{\epsilon}{\mu E}$), перемножили
два разбиения, получили разбиение $\Omega$ мелкости $<\delta$, которое хорошо приближает $I$.
Т.е. хорошо приблизили интеграл $g$ разбиением $\Omega$, что успех.

Перестановка: если $f$ интегрируема на квадратике, можно переставлять (если не интегрируема "--- взяли $\frac{x^2-y^2}{(x^2+y^2)^2}$ $\Ra$ разный результат).
Элементарное множество: пририсовали к $E$ цилиндры между двумя графиками (графики непрерывны на замыкании);
если $f$ интегрируема и $\exists$ интегралы в точках $E$, то можно свести (док-во: продлили ограничивающие нулями, применили теорему).

\section{} % 02
% Лемма 3.6.1
Есть непрерывно дифференцируемая непрерывная функция на замыкании выпуклого компакта, $f \colon \R^{n-1} \to \R^n$, тогда мера образа ноль.
Док-во: разложили $f$ покоординатно, частные производные равномерно непрерывны (есть некое $M$) и ограничены.
Так как выпукло, можно ограничить разброс образов $M$ при помощи производных и т.Лагранжа.
Окружили компакт кубиком стороной $a$, покрошили на мелкие $\delta$-куски, диаметр образа каждого ограничен $M \delta$ (можно тоже вписать в куб), посчитали количество кубов в $\R^{n-1}$,
посчитали максимальную меру образов в $\R^n$, упс.
Вообще можно для просто связных, но нам не надо.
Без дифференцируемости неверно.
\TODO

\section{} % 03
\TODO

\section{} % 04
\TODO

\section{} % 05
\TODO

\section{} % 06
\TODO

\section{} % 07
\TODO

\section{} % 08
\TODO

\section{} % 01
Сведение: $E$ измеримо, на $\Omega=E\times[a,b]$ есть \textit{интегрируемая} (надо знать заранее), можно взять $\int_a^b f$ в точках $E$ $\Ra$ тогда $g$ интегрируема на $E$.
Док-во: нам дали $\epsilon$, спросили у $f$, какой ей нужен $\delta$, взяли любое разбиение $E_i, \eta_i$ мелкости $\frac \delta 2$.
Теперь нашли разбиение $[a,b]$ мелкости $<\frac\delta 2$, чтобы во всех $\eta_i$ (их конечно) хорошо приблизить $g$ (на $\frac{\epsilon}{\mu E}$), перемножили
два разбиения, получили разбиение $\Omega$ мелкости $<\delta$, которое хорошо приближает $I$.
Т.е. хорошо приблизили интеграл $g$ разбиением $\Omega$, что успех.

Перестановка: если $f$ интегрируема на квадратике, можно переставлять (если не интегрируема "--- взяли $\frac{x^2-y^2}{(x^2+y^2)^2}$ $\Ra$ разный результат).
Элементарное множество: пририсовали к $E$ цилиндры между двумя графиками (графики непрерывны на замыкании);
если $f$ интегрируема и $\exists$ интегралы в точках $E$, то можно свести (док-во: продлили ограничивающие нулями, применили теорему).

\section{} % 02
% Лемма 3.6.1
Есть непрерывно дифференцируемая непрерывная функция на замыкании выпуклого компакта, $f \colon \R^{n-1} \to \R^n$, тогда мера образа ноль.
Док-во: разложили $f$ покоординатно, частные производные равномерно непрерывны (есть некое $M$) и ограничены.
Так как выпукло, можно ограничить разброс образов $M$ при помощи производных и т.Лагранжа.
Окружили компакт кубиком стороной $a$, покрошили на мелкие $\delta$-куски, диаметр образа каждого ограничен $M \delta$ (можно тоже вписать в куб), посчитали количество кубов в $\R^{n-1}$,
посчитали максимальную меру образов в $\R^n$, упс.
Вообще можно для просто связных, но нам не надо.
Без дифференцируемости неверно.

Условие (*) на функцию $\phi \colon \cl G \to D \subset \R^n$: она биективна, непрерывна, непрерывно дифференцируема, $G$ открыто.

% Лемма 3.6.2
Пусть есть функция из (*), тогда $\phi$ инвариантна относительно
разности множеств (т.к. биективна),
замыкания (два включения: $E \subset \cl E$, взяли $\phi$, взяли замыкание, а $\cl E$ компакт, чей образ тоже замкнут; взяли точку из $\cl E$, сказали про предел последовательности и непрерывность $\phi$),
включения (аналогичные два включения: $\phi(\Int E)$ точно открыто; взяли $a \in \Int \phi(E)$, взяли шарик вокруг, потом по биективности и открытости шарика, успех)
и взятия границы (следует из первых трёх).

\section{} % 03
Линейная составляющая ($\phi^{\mathup{lin}}_{t_0} (t)$) есть $\phi(t_0)$ плюс линейный член Тейлора (производная на $(t-t_0)$).
Лемма: если $G$ ограничено и $\phi \colon \cl G \to \R^n$ непрерывно дифференцируема, то ввели $S_\delta$ как максимум
по всем точкам из $B_\delta(t_0)$, соединёнными отрезками с $t_0$ такой штуки: норма отклонения функции от линейной части в $t$ на $\|t-t_0\|$.
Тогда $S_\delta \to 0$ при $\delta \to 0$ (то есть линейная часть действительно приближает функцию до первого порядка).

Пусть $g(t)$ "--- отклонение (в $t_0$ "--- ноль) и для каждого $t$ введём $h(\theta) \colon [0,1] \to \R^n$ (выдаёт значение $g$ на отрезке $[t_0, t]$),
нас интересует $g(t)=\|h(1)-h(0)\|$, оно по Лагранжу для векторнозначных \textit{не больше} $\|\h'(\theta_0)\|$ для нек. $\theta_0$.
Продифференцировали $h$ (с дифференциалами, она векторнозначна) как сложную функцию точке $u$ (соответствует $\theta_0$), оценили норму $\d_u g$
(она равен $\d_u \phi - \d_{t_0} \phi$, второе слагаемое вылезло из линейной части), подставили оценку на числитель, всё сократилось, получилась $\|\d_u \phi - \d_{t_0} \phi\|$.

Теперь её оценим как максимум по всем таким $u$ в небольшом шарике (ослабили исходное условие, нам лишь хуже).
Оператор есть разность частных производных в $t_0$ и в $u$, а так как они непрерывны, то при $\delta\to0$ разность $\to 0$ $\Ra$ норма стремится к нулю
(кстати, норма "--- это максимальная длина образа единичного вектора под действием оператора).

\section{} % 04
Есть $\phi$ с условием (*), есть точка $t^*$, есть куб $Q_h$ со стороной $h$ с углом в $t^*$, тогда $\phi(Q_h)$ измерима и предел отношения мер $\to J_{\phi}(t*)$
(объём вырос в якобиан); причём даже супремум разности объёмов по всем $t^*$ стремится к нулю.
Измеримость: граница есть объединение $2n$ кубиков (компакты) размерности $n-1$, их образ нулевой меры, т.о. образ кубика измерим.
Формула: пусть $R_h=\phi^{\mathup{lin}}_{t^*}(Q_h)$ (косоугольный параллелепипед, почти образ $Q_h$), тогда отношение размеров точно Якобиан (привет из алгебры).
Положили $\sigma(h)$ равным супремуму нормы отклонения $\phi(Q_h)$ от косоугольного, разделили на макс. расстояние внутри кубика (это диагональ), ограничили это сверху $S_h \to 0$,
т.е. $\sigma(h)=o(h)$.
Тогда образ $Q_h$ содержится в $R_h$, раздутом на $\sigma(h)$ и содержит $R_h$, сдутый на столько же (отличие-то от $R_h$ мелкое).
Покажем, что объём $Q_h$ и соответствующей модификации $R_h$ отличается на $o(h^n)$ (получим, что надо, потому что $\mu Q_h = h^n$).
Это просто: грань модификации увеличилась на $\sigma(h)$, объём увеличился на $O(\sigma^n(h))$, т.е. $o(h^n)$.

\section{} % 05
$G \subset \R^n$ открыто и измеримо, $\phi \colon \cl G \to D \subset \R^n$ биективна непрерывна и непрерывно дифф. на $\cl G$ (как в (*)), якобиан не ноль
(это определитель матрицы частных производных, в строке $i$ лежат $\partd{\phi_i}{t_j}$), образ измерим.
Тогда для непрерывной $f \colon \cl D \to \R$ имеем формулу замены переменной (с \textit{модулем} якобиана вместо производной $\phi$, поэтому важна биективность, в одномерном скачки друг друга компенсируют).
Возьмём $P \subset G$, будем доказывать для разных видов.

% Теорема 3.6.2
Пусть это куб со стороной $a$, тогда оба интеграла существуют (так как $\cl G$ измеримо $\Ra$ ограничено $\Ra$ компакт), взяли разбиение $P$ на одинаковые кубики, оснащение "--- минимальные углы кубиков,
тогда такая сумма при мелких кубиках стремится к $\int_P$.
Подействуем на разбиение и оснащение $\phi$; так как $\phi$ равномерно непрерывна (на компакте же), то у нас мелкость разбиения в образе тоже будет $\to 0$.
Теперь сравниваем интегралы через суммы (одна сумма $f\cdot \dots$ по разбиению, другая "--- сумма $f$ по образу разбиения), куча всего вынесется,
останется отклонение отношения мер от якобиана (а супремум этой штуки по всем точкам стремится к нулю).

Конечное объединение кубов: интересуют только дизъюнктные объединения, а интегралы аддитивны.

\section{} % 06
Лемма: если в $G_1\subset G_2 \subset \dots \subset G$ все ограничены и измеримы, а $f \colon \cl G \to \R$ непрерывна, причём мера непокрытого $\to 0$, то предел интегралов равен $\int G$ (интегралу предела).
Док-во: $\cl G$ "--- компакт, оценили интеграл по $G \setminus G_n$ как максимум модуля на меру, что $\to 0$.

Случай теоремы: пусть $P$ "--- ограниченное и измеримое.
Тогда поджимаем множество снизу кубами по шагам (стороны уменьшаются на шаге в два раза).
Оно стремится к $P$, так как можно зажать границу $P$ клеточным множеством меры $\epsilon$, а мера пересечения кубиков (мы тут смотрим на кубики, входящие в $P$ плюс клеточное),
с каждым параллелепипедом $X$ из клеточного будет $\to \mu X$, то есть сумма мер кубиков, покрывающих границу, стремится к нулю.
А остальные "--- не покрывают $\Ra$ лежат в $P$, что и требовалось.

Следствие теоремы: если можно из $G$ и $D$ выкинуть что-то меры ноль и получить биекцию, то всё равно можно преобразовывать интеграл.

\section{} % 07
\TODO

\section{} % 08
\TODO

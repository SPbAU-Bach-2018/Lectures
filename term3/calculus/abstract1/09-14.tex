\section{} % 09
Есть гладкий путь, на нём непрерывная функция, тогда формула "--- как в замене переменной (с модулем скорости).
Другая параметризация (есть монотонная $r(x)$ для перехода), подставили и заменили переменную, не забыть проверить знак, если направление изменилось.
Значит, от направления не зависит.
Линеен по функции.
Аддитивен по кускам кривых.
Если путь спрямляем (длина конечна), то есть натуральная параметризация и интеграл будет из одного множителя.

\section{} % 10
$\int_{\gamma} \d s = l(\gamma)$, можно еще ограничить модуль сверху максимумом $f$ на длину (натуральная параметризация).
Если $f \le g$, то интегралы тоже меньше.
Сумма: разбили $[a,b]$ на куски, оснастили, посчитали сумму, тогда существование хорошего предела сумм $\iff$ интегралу.
Док-во: \TODO %начала интеграл по определению, страница 15

\section{} % 11
Еще называются <<криволинейный второго рода>>.
По определению: слева несколько дифференциалов, справа их независимо расписали как в первом роде (но без модулей "--- это же координаты).
Если сохранили ориентацию, то можно заменить параметризацию кривой (как замену переменной) и всё останется.
Смена направления сменит знак.
Если взять единичный касательный вектор $\vec \sigma$ (по направлению обхода), то интеграл по форме меняем на интеграл скалярного произведения $f$ и $\vec\sigma$.
Линейность по функции.
Аддитивность по кривой при сохранении ориентации.

\section{} % 12
Модуль интеграла $f$ по форме $\le$ интеграл модуля $f$ $le$ максимум модуля на длину.
Заменили интеграл по форме на скалярное произведение, снесли модуль внутрь, КБШ, модуль $\vec \sigma$ равен единице.
Интегральная сумма: разбили и оснастили путь, взяли сумму по кускам, в каждом куске "--- сумму по координатам.
Дальше сказали, что есть они хорошо сходятся (при любом мелком оснащённом разбиении) $\iff$ есть интеграл.
Док-во: \TODO % страница 17

\section{} % 13
Первообразная: такая дифференцируемая $F$, что её дифференциал (который есть линейное отображение) равен чему надо,
т.е. $f_1=\partd{F}{x_1}, \dots$.
Если есть первообразная, то интеграл по кривой $A \to B$ равен $F(B)-F(A)$
(расписали интеграл формы, заменили кучу частных производных на $(F \circ \gamma)'$, ура).
Следствие: от пути не зависит, в физике <<потенциальное поле>>.

Если $G$ "--- область (открыто и \textit{линейно} связно), $f_i \colon G \to \R$ непрерывны, то есть первообразная $\iff$ интеграл по любому замкнутому равен нулю.
$\Ra$ очевидно, для $\La$ строим $F$ как интеграл по какому-нибудь пути от фиксированной $x$ до любой $a$ (однозначно), проверим: берём частную производную $\partd{F}{x_1}$ по определению,
заменили разность на интеграл по пути, на обычный интеграл, на значение в точке на длину, сократилось, получили $f_1(x)$, что и надо.

\section{} % 14
Есть область $G \subset \R^2$, граница состоит из конечного числа не(само)пересекающихся замкнутых кусочно-гладких кривых.
Если $P$, $G$ непрерывно дифференцируемы, то $\int_{\delta G} P\d x + G\d y = \int_G \partd{Q}{x}-\partd{P}{y}$ (почти векторное произведение в координатах),
направление обхода границы важно.
Док-во: сначала разобрались с <<треугольником>> (справа и снизу отрезок, гипотенуза "--- функция), посчитали интегралы $\partd{P}{y}$ и другой по треугольнику,
один интеграл ушёл с частной производной, получили криволинейные, они посокращались с чем надо (надо еще один нулевой интеграл добавить по вертикальному отрезку, чтобы красиво).
Потом показали, что при разрезании сложной фигуры всё ок, потом нашинковали криволинейную с конечным числом экстремумов на такие треугольники, не забыть про дырки.
Площадь (чтобы справа в формуле получили единицу): $Q=x, P=0$; $P=-y, Q=0$; $Q=\frac x 2, P=-y 2$ (в последнем еще коэффициент при интеграле будет)

\setauthor{Дима Лапшин}
Планы:
\begin{enumerate}
	\item Кратные интегралы
	\item Криволинейные интегралы
	\item Ряды
	\item Функциональные ряды
	\item Несобственные интегралы (их могут попросить пораньше)
\end{enumerate}

\setcounter{chapter}{2}
\chapter{Кратные интегралы (продолжение)}
\setcounter{section}{4}
\section[Сведение крат. интеграла к повторному]{Сведение кратного интеграла к повторному}

\begin{theorem}
	$E \subset \R^n$ измеримо, $\Omega = E \times [a, b] \subset \R^{n+1}$ (координаты: $(x_1, \dots, x_n, y)$).
	$f\colon \Omega \ra \R$ интегрируема на $\Omega$, $\forall x \in E, \exists \int_a^b f(x,y) \d y \lrh g(x)$.
	Тогда $g$ интегрируема на $E$ и
	\[ \int\limits_E g = \int\limits_\Omega f \]
\end{theorem}
\begin{proof}
	$f$ интегрируема на $\Omega$ и интеграл равен $I$, значит
	\[
		\forall \epsilon > 0, \exists \delta > 0\colon \forall \tau\colon |\tau| < \delta,
		\forall \xi, \left|S(f,\tau,\xi) - I\right| < \epsilon
	\]
	Тут $\tau = \{\Omega_1, \dots, \Omega_m\}$, $\diam \Omega_i < \delta$.
	Дальше мы будем интересоваться только некоторыми разбиениями $\Omega$.
	Возьмём разбиение $E$ мелькостью менее $\frac\delta2$ и разбиение $[a, b]$ такой же мелкости.
	Перемножим эти два разбиения, получим разбиение мелкостью менее $\delta$.
	\begin{Rem}
		Каждое измеримое множество--фрагмент $E$ декартого умножается на \\отрезок--фрагмент $[a, b]$, получаем измеримое.
	\end{Rem}

	Возьмём это разбиение в  $\Omega_i$.
	Для доказательства теоремы надо показать, что $g$ интегрируема.
	Зафиксируем $\epsilon$, возьмём $\delta$ из интегрируемости $f$ по $\Omega$.
	Осталось показать, что любое достаточно мелкое разбиение $E$ даст нам сумму $g$ Римана со значением $I$ с небольшой погрешностью.
	Возьмём произвольное разбиение $\tau = \{E_1, E_2, \dots\}$ мелкости менее $\frac\delta2$ и его оснащение $\{\eta_1, \eta_2, \dots\}$.
	Покажем из него, что
	\[
		\left| S(g, \tau, \eta) - I \right| < 2\epsilon
	\]
	Мы знаем, что $g(\eta_i)$ существуют, значит, для каждого $\eta_i$ можно расписать существование интеграла $g(\eta_i)=\int_a^b f(\eta_x, y)\d y$:
	\[
		\forall \tilde\epsilon>0, \exists \tilde\delta_{\eta_i} > 0\colon \forall \underbrace{\tilde\tau\colon |\tilde\tau| < \delta_{\eta_i}}_{\text{разбиение }[a, b]},
		\forall \{y_k\}, \left| g(\eta_i) - S(f, \tilde\tau, y_k) \right| < \tilde\epsilon
	\]
	Возьмём в качестве $\tilde\epsilon = \frac\epsilon{\mu E}$.
	Мы хотим найти такую мелкость $\tilde\delta < \frac\delta2$ разбиения $[a, b]$, что будет выполнено
	\[
		\left| S(g, \tau, \eta) - I \right| < 2\epsilon
	\]

	Делаем мы это так.
	Взяли разбиение $E_i$.
	У него есть оснащение $\eta_i$.
	Для каждого такого есть $\tilde\delta_{\eta_i}$.
	Их конечное число, значит есть такое $\tilde\delta$, которое меньше их всех и даже $\frac\delta2$.
	Теперь возмьмём разбиение $E$ и умножаем его на дробление $[a, b]$ мелкости меньше $\tilde\delta$:
	$\{I_1, I_2, \dots\}$ с оснащением ${y_1, y_2, \dots}$.

	Получим:
	\begin{gather*}
		\tau\colon \Omega_{ij} = E_i \times I_j \quad \xi\colon \xi_{ij} = (\eta_i, y_j) \\
		S(f, \tau, \xi) = \sum_{i,j} f(\eta_i, y_j) \mu(E_i \times I_j) = \sum_{i,j} f(\eta_i, y_j) \mu E_i \mu I_j
			= \sum_i \underbrace{\left(\sum_j f(\eta_i, y_j) \mu I_j\right)}_{\text{сумма для $\int_a^b f(\eta_i,y) \d x$}} \mu E_i > \cdots\\
		\sum_j f(\eta_i, y_j) \mu I_j > \int\limits_a^b f(x, y) \d x - \frac\epsilon{\mu E} \\
		\cdots > \sum_i \left(\int\limits_a^b f(x, y) \d x - \frac\epsilon{\mu E}\right) \mu E_i
			= \sum_i g(\eta_i) \mu E_i - \epsilon \sum_{i} \frac1{\mu E} \mu E_i = \sum_i g(\eta_i) \mu E_i - \epsilon \\
		\sum_i g(\eta_i) \mu E_i < I + 2\epsilon
	\end{gather*}
	Аналогично
	\begin{gather*}
		\sum_i g(\eta_i) \mu E_i > I - 2\epsilon \\
		\left| \sum_i g(\eta_i) \mu E_i - I \right| < 2\epsilon \Ra \int\limits_E g = I
	\end{gather*}
\end{proof}
\begin{Rem}
	Пусть мы знаем, что $f$ непрерывна на компакте, а значит и равномерно непрерывна.
	Тогда из определения равномерной непрерывности можно взять $\tilde\delta$ для $\tilde\epsilon = \frac\epsilon{\mu E}$.
	Оно без игр с $\tilde\delta_{\eta_i}$ подойдёт везде.
\end{Rem}

\begin{conseq}
	Пусть есть $f\colon [a, b] \times [c, d] \ra \R$, $f$ интегрируема, $\forall x, \text{$f(x)$ интегрируема по $y$}$.
	Тогда
	\[ \int\limits_{[a, b] \times [c, d]} f = \int\limits_{[a, b]} \left( \int\limits_{[c, d]} f \d y \right) \d x \]
\end{conseq}
\begin{conseq}
	Пусть есть $f\colon [a, b] \times [c, d] \ra \R$, $f$ интегрируема,
	$\forall x, \text{$f(x)$ интегрируема по $y$}$, $\forall x, \text{$f(y)$ интегрируема по $x$}$.
	Тогда
	\[
		\int\limits_{[a, b]} \left( \int\limits_{[c, d]} f \d y \right) \d x
		= \int\limits_{[c, d]} \left( \int\limits_{[a, b]} f \d x \right) \d y
	\]
\end{conseq}
\begin{Rem}
	Без интегрируемости всей $f$ на $[a, b] \times [c, d]$ неверно.
	\begin{exmp}
		\begin{gather*}
			f(x, y) = \begin{cases} \frac{x^2 - y^2}{(x^2 + y^2)^2} & (x, y) \ne 0 \\ 0 & (x, y) = 0 \end{cases} \\
			\int\limits_0^1 \int\limits_0^1 f(x, y) \d x \d y = \frac\pi4 \quad \int\limits_0^1 \int\limits_0^1 f(x, y) \d y \d x = \frac{-\pi}4
		\end{gather*}
	\end{exmp}
\end{Rem}

\begin{Rem}
	Вспоминаем, что такое элементарное множество: взяли $\phi, \psi\colon \cl E \subset \R^n \ra \R$, $\phi \le \psi$.
	\[ X = \{ (x, y) \in \R^{n+1} \mid x \in E \land \phi(x) \le y \le \psi(x)\} \]
\end{Rem}
\begin{theorem}
	Пусть $X$ "--- элементарное множество, $f\colon X \ra \R$, $f$ интегрируема, $\int_{\phi(x)}^{\psi(x)} f(x, y) \d y$ существует.
	Тогда
	\[ \int\limits_X f = \int\limits_E \left( \int\limits_{\phi(x)}^{\psi(x)} f(x, y) \d y \right) \d x \]
\end{theorem}
\begin{proof}
	$a = \min \phi(x)$, $b = \max \psi(x)$.
	Продлим функцию $f$ нулями, применим предыдущую теорему:
	\begin{gather*}
		h\colon E \times [a, b] \ra \R \\
		h(x, y) = \begin{cases} f(x, y) & (x, y) \in X \\ 0 & (x, y) \notin X \end{cases}
	\end{gather*}
\end{proof}

\section{Замена переменной в интеграле}

\begin{Def}
	$f\colon G \ra \R$, $f$ непрерывна.
	$f$ непрерывно продолжаемая на замыкании, если существует непрерывная $g\colon \cl G \ra \R$, что
	\[ g \biggr|_G = f \]
\end{Def}

\begin{Def}
	$f\colon \cl G \ra \R$, $G$ открыто.
	$f$ называется непрерывно дифференцируемой $k$ раз на $\cl G$,
	если $f$ непрерывно дифференцируема на $G$ и все происзодные непрерывно продолжаются на $\cl G$.
\end{Def}

Глобальная цель: есть открытое измеримое $G \subset \R^n$, $\phi\colon \cl G \ra \R^n$,
$\phi(G) = D$ "--- открытое, $\phi$ является биекцией между $G$ и $D$, $\phi$ непрерывно дифференцируема на $\cl G$,
Якобиан
\[	\mathcal J_\phi(t) = \det \begin{pmatrix}
		\partd{\phi_1}{t_1} & \cdots & \partd{\phi_1}{t_n} \\
		\vdots & \ddots & \vdots \\
		\partd{\phi_n}{t_1} & \cdots & \partd{\phi_n}{t_n}
	\end{pmatrix} \ne 0
\]
Тогда можно заменить переменную:
\[
	\int\limits_D f = \int\limits_G f(\phi(t)) \left| J_\phi(t) \right| \d t
\]
Всё похоже на одномерный случай, но тогда нам не нужна была биекция: если $\phi$ убежит вверх, то она так же убежит вниз, и уберёт избыток.
Также, нам не нужен там был модуль, потому что важно было направление.

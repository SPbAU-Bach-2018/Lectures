\begin{Rem}\textbf{
	Тут была другая лемма, из прошлого семестра, про то, что мера графика ноль.
	Нам на следующей лекции уточнили, что нужна другая.
	Вот она.
}\end{Rem}
% \begin{lemma}
% 	$f\colon K \ra \R$, $K \subset \R^n$ "--- компакт, $f$ непрерывна.
% 	Тогда $\Gamma_f$ измеримо, и мера его "--- ноль.
% \end{lemma}
% \begin{proof}
% 	$K$ ограниченно, значит $K$ лежит в кубе $Q$.
% 	$f$ непрерывна на $K$, значит равномерно непрерывна, значит
% 	\[ \forall \epsilon > 0, \exists \delta > 0\colon \forall x, y \in K, |x-y| < \delta \Ra |f(x)-f(y)| < \epsilon \]
% 	Нарежем $Q$ на маленькие кубики со стороной менее $\delta / \sqrt{n}$, тогда там разброс функции меньше $\epsilon$.
% 	Тогда мера графика менее $\epsilon \mu K$.
% \end{proof}

% \begin{lemma}
% 	$A$ линейное отображение и $P$ "--- косоугольный параллелепипед (натянут на $n$ линейно независимых векторов).
% 	Тогда
% 	\[ \mu(AP) = |\det A| \mu P \]
% \end{lemma}
% \begin{proof}
% 	Из линейной алгебры: $P$ задан набором векторов $u_1, u_2, \dots, u_n$.
% 	Его объём "--- определитель матрицы векторов.
% 	\[ \mu P = \left|\det \begin{pmatrix} u_1 \\ u_2 \\ \vdots \\ u_n \end{pmatrix}\right| \]
% 	Теперь
% 	\begin{gather*}
% 		\mu(AP) = \left|\det \left(A \begin{pmatrix} u_1 \\ u_2 \\ \vdots \\ u_n \end{pmatrix} \right)\right|
% 			= |\det A| \cdot \left|\det \begin{pmatrix} u_1 \\ u_2 \\ \vdots \\ u_n \end{pmatrix} \right| = |\det A| \mu P
% 	\end{gather*}
% \end{proof}
\begin{lemma}
	$K \subset \R^{n-1}$ выпуклый компакт, $f\colon K \ra \R^n$ непрерывна и непрерывно дифференцируема на замыкании.
	Тогда $\mu f(K) = 0$.
\end{lemma}
\begin{proof}
	\[ f = \begin{pmatrix}f_1 \\ f_2 \\ \vdots \\ f_n\end{pmatrix} \]
	Все координаты равномерно непрерывны, значит все частные производные непрерывны и ограничены.
	Тогда
	\[ \forall x, y, \|f(x) - f(y)\| \le M \|x - y\| \]

	Берём $K$, окружаем кубиком $Q$ и режем на куски диаметра не более $\delta$.
	Хотим, чтобы диаметр разбиения был не более $\delta$, то есть сторона не более $\delta / \sqrt{n-1}$.
	Тогда образ каждого кусочка имеет диаметр не более $M\delta$.
	Значит $f(K_i)$ можно вписать в куб $C_i$ с ребром не более $2M\delta$.
	\begin{gather*}
		f(K) = \bigcup f(K_i) \subset \bigcup C_i \\
		\mu f(K) \le \left( \frac{a}{\delta / \sqrt{n-1}} \right)^{n-1} \cdot \left(M\delta\right)^n = const \cdot \delta
	\end{gather*}
	Получили, что можем оценить $\mu f(K)$ сколь угодно мало.
\end{proof}

\begin{Rem}
	Выпуклость не так нужна, можно связные множества, а потом на связные куски разбить, но нам нужно только для куба.
\end{Rem}

\begin{Rem}
	Без дифференцируемости неверно, можно привести какой-то пример.
	\textit{Какой именно на лекции честно написать не получилось.}
\end{Rem}

Ещё раз формула замены переменной:
$G \subset \R^n$ "--- открыто и измеримо, \\
$\phi\colon \cl G \ra D \subset \R^n$ биективна, непрерывна и непрерывно дифференцируема на $\cl G$ (*), а также имеет нулевой Якобиан:
\[
	J_\phi(t) = \det
	\begin{pmatrix}
		\partd{\phi_1}{t_1} & \partd{\phi_1}{t_2} & \cdots & \partd{\phi_1}{t_n} \\
		\partd{\phi_2}{t_1} & \partd{\phi_2}{t_2} & \cdots & \partd{\phi_2}{t_n} \\
		\vdots              & \vdots              & \ddots & \vdots              \\
		\partd{\phi_n}{t_1} & \partd{\phi_n}{t_2} & \cdots & \partd{\phi_n}{t_n} \\
	\end{pmatrix} \ne 0
\]
$\phi(G) = D$ измеримо.
Тогда для непрерывной $f\colon \cl D \ra \R$. 
\[
	\int\limits_D f(x) \d x = \int\limits_G f(\phi(t)) |J_\phi(t)| \d t
\]

\begin{lemma}
	$E_1 \subset E \subset G$ ограниченны, $\phi\colon \cl G \ra D \subset \R^n$ удовлетворяет условию (*).
	Тогда
	\begin{enumerate}
		\item $\phi(E \setminus E_1) = \phi(E) \setminus \phi(E_1)$
		\item $\phi(\cl E) = \cl \phi(E)$
		\item $\phi(\Int E) = \Int \phi(E)$
		\item $\phi(\delta E) = \delta \phi(E)$
	\end{enumerate}
\end{lemma}
\begin{proof}\begin{enumerate}
\item
	$\phi$ биективна.

\item
	\begin{description}
	\item[$\supset$:]
		$E \subset G$ ограниченно, значит $\cl E$ ограничено, значит $\cl E$ компакт,
		значит $\phi(\cl E)$ непрерывный образ компакта, то есть компакт, то есть $\phi(\cl E)$ замкнуто.
		\[
			E \subset \cl E \Ra \phi(E) \subset \phi(\cl E) \Ra \cl \phi(E) \subset \cl \phi(\cl E) = \phi(\cl E)
		\]

	\item[$\subset$:]
		Возьмём $a \in \cl E$. Есть последовательность $\{a_n\} \subset E$, что $\lim a_n = a$.
		По непрерывности $\phi$
		\[
			\lim \phi(a_n) = \phi(\lim a_n) = \phi(a)
		\]
		откуда $\phi(a) \in \cl \phi(E)$.
	\end{description}

\item
	\begin{description}
	\item[$\subset$:]
		$\Int E$ открыто, значит $\phi(\Int E)$ открыто по теореме об обратной функции.
		\[
			\Int E \subset E \Ra \phi(\Int E) \subset \phi(E) \Ra \underbrace{\Int \phi(\Int E)}_{= \phi(\Int E)} \subset \Int(\phi E)
		\]

	\item[$\supset$:]
		Возьмём $a \in \Int \phi(E)$.
		Есть шарик $B_r(a) \subset \phi(E)$.
		Тогда можно взять шарик\\ $B_\epsilon(\phi^{-1}(a)) \subset E$, образ которого помещается в $B_r(a) \subset \phi(E)$.
		По биекции можно снять $\phi$, и получили, что $\phi^{-1}(a) \in \Int E$.
	\end{description}

\item
	\begin{gather*}
		\delta E = \cl E \setminus \Int E \\
		\phi(\delta E) = \phi(\cl E \setminus \Int E) = \phi(\cl E) \setminus \phi (\Int E)
		= \cl \phi(E) \setminus \Int \phi(E) = \delta \phi(E)
	\end{gather*}
\end{enumerate}\end{proof}

\begin{Def}
	$\phi\colon G \ra \R^n$ дифферецируема в $t_0$.
	Напишем Тейлора:
	\[
		\phi(t) = \phi(t_0) + \d_{t_0} \phi(t - t_0) + \dots
	\]
	Назовём эти первые два слагаемых линейной составляющей $\phi$:
	\[
		\phi^{\mathup{lin}}_{t_0} (t) \eqDef \phi(t_0) + \left( d_{t_0} \phi \right) \cdot (t - t_0)
	\]
\end{Def}

\begin{lemma}
	Если $G$ ограничено и $\phi\colon \cl G \ra \R^n$ непрерывно дифферецируема на $\cl G$, то
	\[
		S_\delta \lrh \sup_{\substack{\|t-t_0\| \le \delta \\ [t, t_0] \subset G}} \frac{\|\phi(t) - \linf{\phi}_{t_0} (t)\|}{\|t-t_0\|} \xlongrightarrow{\delta \ra 0+} 0
	\]
\end{lemma}
\begin{proof}
	$g(t) = \phi(t) - \linf{\phi}_{t_0} (t)$, $g(t_0) = 0$.
	Переберём точки на $[t_0, t]$ вида $\theta t_0 + (1-\theta) t$.

	$h(\theta) = g(\theta t_0 + (1-\theta) t)$ "--- векторная функция одной переменной, дифференцируема.
	Тогда по теореме Лагранжа для векторозначных функций
	\begin{gather*}
		\|h(1) - h(0)\| \le \|h'(\theta_0)\| \cdot (1-0) \\
		\|h(1) - h(0)\| = \|g(t_0) - g(t)\| = \|g(t)\|
	\end{gather*}
	Мы только что оценили числитель дроби.
	\begin{gather*}
		\d_{\theta_0}h = \left(\d_{\theta_0 t_0 + (1-\theta_0) t} g\right) \cdot \left(\d_{\theta_0} (\theta t_0 + (1-\theta) t)\right)
			= \left(\d_{\theta_0 t_0 + (1-\theta_0) t} g\right) \cdot (t_0 - t) \\
		u \lrh \theta_0 t_0 + (1-\theta_0) t \\
		\d_u g = \d_u \phi - \d_u \linf{\phi}_{t_0} = \d_u \phi - \d_{t_0} \phi \\
		\|d_{\theta_0} h\| = \|(\d_u \phi - \d_{t_0} \phi)(t-t_0)\| \le \|\d_u \phi - \d_{t_0} \phi\| \|t-t_0\| \\
		\frac{\|\phi(t) - \linf{\phi}_{t_0} (t)\|}{\|t-t_0\|} = \frac{\|g(t)\|}{\|t-t_0\|} \le \frac{\|\d_{\theta_0} h\|}{\|t-t_0\|}
			\le \frac{\|\d_u \phi - \d_{t_0} \phi\| \|t-t_0\|}{\|t-t_0\|} = \|\d_u \phi - \d_{t_0} \phi\| \\
		S_\delta \le \sup_{\substack{\|t-t_0\| \le \delta \\ [t, t_0] \subset G}} \|\d_u \phi - \d_{t_0} \phi\|
			\le \sup_{\|u-t_0\|\le \delta} \|\d_u \phi - \d_{t_0} \phi\|
	\end{gather*}
	В конце написана матрица из разностей производных функции $\phi$ по разным координатам.
	Эта вещь целиком непрерывна, поэтому норма матрицы при стремящейся к нулю $\delta$ стремится к нулю.
\end{proof}

\begin{theorem}[Геометрический смысл модуля Якобиана]
	$\phi\colon \cl G \ra D \subset \R^n$ удовлетворяет (*).
	$t^* \in G$, $Q_h \lrh \{t \mid t^*_i \le t_i \le t^*_i + h\}$ "--- куб на $t^*$.
	Тогда $\phi(Q_h)$ измерима и
	\[
		\frac{\mu \phi(Q_h)}{\mu Q_h} \xlongrightarrow{h \ra 0+} |J_\phi(t^*)|
	\]
	Более того,
	\[
		\sup_{t^* \in G} \left| \frac{\mu \phi(Q_h)}{\mu Q_h} - |J_\phi(t^*)|\right| \xlongrightarrow{h \ra 0+} 0
	\]
\end{theorem}
\begin{Rem}
	Геометрический смысл тут такой: взяли маленький кубик, от воздействия на него $\phi$-шкой его объём вырос примерно в Якобиан.
\end{Rem}

Свойства:
\begin{enumerate}
\item
	Интеграл не зависит от ориентации, так как ориентации в определении просто нет.

\item
	Линейность по функции.
	\[ \int\limits_M (\alpha f + \beta g) \d S = \alpha \int\limits_M f \d S + \beta \int\limits_M g \d S \]

\item
	Линейность по поверхности: если в $\R^2$ имеем $G = G_1 \sqcup G_2$ (пересекаются по мн-ву меры ноль),
	то тогда соответствующие поверхности в $\R^3$: $M = M_1 \sqcup M_2$ и верно следующее:
	\[ \int\limits_M f \d S = \int\limits_{M_1} f \d S + \int\limits_{M_2} f \d S \]

\item
	$f \ge 0$
	\[ \int\limits_M f \d S \ge 0 \]
\end{enumerate}

\begin{exmp}[Сферические координаты]
	\begin{gather*}
		x = R \cos u \cos v \quad y = R \sin u \cos v \quad z = R \sin v \\
		u \in [0, 2\pi] \quad v \in \left[-\frac\pi2,\frac\pi2\right]
	\end{gather*}
	Биекция сходу не получится, разрежем сферу на две полусферы.
	Теперь проблема в полюсах, мы выкинем полярные шапки, и устремим их размер к нулю, интеграл там будет нулевой, так что в конце всё хорошо.
	Это рукомахательство, но нас устроит.

	Посчитаем что-нибдуь:
	\begin{gather*}
		\vec r_u' = (-R \sin u \cos v, R \cos u \cos v, 0) \\
		\vec r_v' = (-R \cos u \sin v, -R \sin u \sin v, R \cos v) \\
		E = \left<\vec r_u', \vec r_u'\right> = R^2 \cos^2 v \\
		F = \left<\vec r_u', \vec r_v'\right> = 0 \\
		G = \left<\vec r_v', \vec r_v'\right> = R^2 \\
		\sqrt{EG-F^2} = R^2 \cos v \\
		\int\limits_{x^2+y^2+z^2=R^2} f(x, y, z) \d S
		= \int\limits_{\substack{u \in [0, 2\pi] \\ v \in \left[-\frac\pi2,\frac\pi2\right]}} f(R \cos u \cos v, R \sin u \cos v, R \sin v) R^2 \cos v \d u \d v
	\end{gather*}
\end{exmp}

\section{Дифференциальные формы}

\begin{Def}
	Внешняя форма $\omega\colon \underbrace{\R^n \times \R^n \times \dots \times \R^n}_{k} \ra \R$
	линейна по каждой координате и антисимметричка:
	\[ \omega(\xi_1, \dots, \xi_i, \dots, \xi_j \dots, \xi_n) = -\omega(\xi_1, \dots, \xi_j, \dots, \xi_i \dots, \xi_n) \]
\end{Def}

\begin{exmp}\begin{description}
	\item[0-форма:] константа
	\item[1-форма:] линейное отображение
	\item[2-форма:] билинейная антисимметричная форма
\end{description}\end{exmp}

\begin{Rem}
	\[ \omega(\xi, \xi) = 0 \]
\end{Rem}

Пусть $n=k=2$, $\omega\colon \R^2 \times \R^2 \ra \R$.
\begin{gather*}
	A = (\xi_1, \xi_2) = \begin{pmatrix} a_{11} & a_{12} \\ a_{21} & a_{22} \end{pmatrix} \\
	\omega(\xi_1, \xi_2)
	= \omega(a_{11} e_1 + a_{21} e_2, a_{12} e_1 + a_{22} e_2)
	= a_{11} a_{12} \omega(e_1, e_1)
	+ a_{11} a_{22} \omega(e_1, e_2) + \\
	+ a_{21} a_{12} \omega(e_2, e_1)
	+ a_{21} a_{22} \omega(e_2, e_2)
	= (a_{11} a_{22} - a_{21} a_{12}) \omega(e_1, e_2) \\
	\omega(\xi_1, \xi_2) = \det A \omega(e_1, e_2)
\end{gather*}

\begin{Def}
	У открытого $G \subset \R^n$ дифференциальная форма "--- сопоставление каждой точке $G$ некоторой внешей формы.
	$\Omega(G)$ "--- множество дифференциальных форм в $G$.
\end{Def}
\begin{exmp}\begin{description}
\item[0-форма:]
	У каждой точки константа, то есть целиком "--- функция,

\item[1-форма:]
	У каждой точки линейное отображение, то есть целиком "--- дифференциалы скалярозначной функций.
	Например, посмотрим на:
	\begin{gather*}
		f_i(x) = x_i \\
		x_i + h_i = f_i(x + h) = f_i(x) + \d f_x(h) + o(h) \\
		\d f_x = f = (x \mapsto x_i)
	\end{gather*}
	Обозначим $\d x_i$ "--- отображение, оставляющее от вектора заданную координату.
	Тогда можно, оказывается, записать:
	\[ \d f = \partd{f}{x_1} \d x_1 + \dots + \partd{f}{x_n} \d x_n \]
	Общий вид 1-формы:
	\[ \omega = P_1 \d x_1 + \dots + P_n \d x_n \]
	Это даже точно соотвествует тому смыслу, что означали $\d x_i$ в криволинейных интегралах II рода!
\end{description}\end{exmp}

\begin{Def}
	Внешнее произведение двух внешних 1-форм (обозначается $\wedge$) "--- внешняя 2-форма с свойствами:
	\begin{enumerate}
	\item
		\[ \omega_1 \wedge \omega_2 = - \omega_2 \wedge \omega_1 \]
	
	\item
		Если есть две функции $\omega_1$, $\omega_2$ и две точки $\xi_1$, $\xi_2$ со следующими свойствами:
		\begin{align*}
			\omega_1(\xi_1) &= 1 \\
			\omega_1(\xi_2) &= 0 \\
			\omega_2(\xi_1) &= 0 \\
			\omega_2(\xi_2) &= 1
		\end{align*}
		То $(\omega_1 \wedge \omega_2)(\xi_1, \xi_2) = 1$ (про остальные точки ничего не говорим).
	\end{enumerate}
	Операция внешнего произведения, к тому же, линейна по каждому аргументу-функции (это можно вывести, но мы возьмём в определение).
\end{Def}

\begin{theorem}
	\[ (\omega_1 \wedge \omega_2)(\xi_1, \xi_2) = \begin{vmatrix} a_{11} & a_{12} \\ a_{21} & a_{22} \end{vmatrix} \]
	где $a_{ij} = \omega_i(\xi_j)$
\end{theorem}
\begin{proof}
	Если $\omega \ne 0$ и $\xi$ линейно независимы, то найдутся 1-формы $\beta_i(\xi_j) = \delta_{ij}$ (как в условии внешнего произведения).
	Откуда?
	Давайте построем $\beta_1$: возьмём функцию, равную проекции вектора на нормаль к $\xi_2$.
	Теперь в $\xi_2$ эта функция точно ноль, а в $\xi_1$ не ноль, можно нормировать.

	Тогда
	\begin{gather*}
		\omega_1 = a_{11} \beta_1 + a_{12} \beta_2 \quad \omega_2 = a_{21} \beta_1 + a_{22} \beta_2 \\
		\omega_1 \wedge \omega_2
		= a_{11} a_{12} (\beta_1 \wedge \beta_1)
		+ a_{11} a_{22} (\beta_1 \wedge \beta_2)
		+ a_{21} a_{12} (\beta_2 \wedge \beta_1)
		+ a_{21} a_{22} (\beta_2 \wedge \beta_2) = \\
		= (a_{11} a_{22} - a_{21} a_{12}) (\beta_1 \wedge \beta_2) \\
		(\omega_1 \wedge \omega_2) (\xi_1, \xi_2)
		= \begin{vmatrix} a_{11} & a_{12} \\ a_{21} & a_{22} \end{vmatrix} \underbrace{(\beta_1 \wedge \beta_2) (\xi_1, \xi_2)}_{=1}
	\end{gather*}

	Если форма ноль или $\xi_i$ линейно зависимы, то слева и справа написан ноль.
\end{proof}

\begin{Rem}
	Возьмём 2-формы.
	Вообще говоря, размерность их пространства $n^2$ (надо выбрать по значению на базисе $(e_i, e_j)$).
	Но из-за антисимметричности, размерность их пространства $\frac{n(n-1)}2$.
\end{Rem}

\begin{theorem}
	2-формы в $\R^n$ "--- линейные комбинации $\d x_i \wedge \d x_j$ (коэффициенты зависят от точки).
\end{theorem}
\begin{proof}
	$\d x_i \wedge \d x_j$ линейно независимы, так как подставляем $(e_i, e_j)$, у каждого есть пара, на котором только он единичен.
	А их ещё $\frac{n(n-1)}2$.
	Значит это базис.
\end{proof}

\begin{conseq}
	Все 2-формы в $\R^3$ выглядят
	\[ P \d y \d z + Q \d z \d x + R \d x \d y \]
\end{conseq}

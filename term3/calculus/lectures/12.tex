\begin{theorem}
	$f_n \in C^1[a, b]$, $c \in [a, b]$, $f_n' \rra g$ на $[a, b]$, $\lim_{n \ra \infty} f_n(c)$ существует и конечен.
	Тогда
	\begin{enumerate}
		\item $f_n \rra h$ на $[a, b]$.
		\item $h$ "--- дифференцируемая функция, и $h' = g$
	\end{enumerate}
	То есть
	\[ \left(\underbrace{\lim_{n \ra \infty} f_n(x)}_{h(x)}\right)' = \underbrace{\lim_{n \ra \infty} f_n'(x)}_{g(x)} \]
\end{theorem}
\begin{proof}
	\begin{gather*}
		\int\limits_c^x g(t) \d t = \int\limits_c^x \lim_{n \ra \infty} f_n'(t) \d t = \lim_{n \ra \infty} \int\limits_c^x f_n'(x) \d x = \\
		= \lim_{n \ra \infty} (f_n(x) - f_n(c)) = \lim_{n \ra \infty} f_n(x) - \lim_{n \ra \infty} f_n(c)
	\end{gather*}
	Отсюда $\lim_{n \ra \infty} f_n(x)$ существует и конечен
	(на самом деле, этим фактом мы дальше не пользуемся "--- сразу доказываем равномерную сходимость).
	Но из предыдущей теоремы мы также знаем, что
	\[ f_n(x) - f_n(c) = \int\limits_c^x f_n'(t) \d t \rra \int\limits_c^x g(t) \d t \]
	откуда $f_n(x) - f_n(c)$ равномерно сходится на $[a, b]$, а так как $f_n(c)$ равномерно сходится (потому что от $x$ не зависит), то
	$f_n(x)$ равномерно сходится на $[a, b]$.

	Назовём этот предел $h$.
	Тогда посмотрим на равенства выше:
	\begin{gather*}
		h(x) - h(c) = \lim_{n \ra \infty} f_n(x) - \lim_{n \ra \infty} f_n(c) = \int\limits_c^x g(t) \d t \Ra \\
		\Ra h(x) = h(c) + \int\limits_c^x g(t) \d t
	\end{gather*}
	$g$ непрерывна как равномереный предел непрерывных функций, значит $\int\limits_c^x g(t) \d t$ дифференцируема и
	\[ h'(x) = g(x) \]
\end{proof}

\begin{conseq}
	$u_n \in C^1[a, b]$, $c \in [a, b]$, $\sum_{k=1}^\infty u_k'(x)$ равномерно сходится на $[a, b]$, $\sum_{n=1}^\infty u_n(c)$ сходится.
	Тогда
	\begin{enumerate}
		\item $\sum_{k=1}^n u_k(x)$ равномерно сходится к дифференцируемой функции.
		\item \[ \left(\sum_{n=1}^\infty u_n(x)\right)' = \sum_{n=1}^\infty u_n'(x) \]
	\end{enumerate}
\end{conseq}
\begin{proof}
	\begin{gather*}
		f_n(x) = \sum_{k=1}^n u_k(x) \\
		f_n'(x) = \sum_{k=1}^n u_k'(x) \rra \sum_{n=1}^\infty u_n'(x) \lra g(x) \Ra f_n' \rra g
	\end{gather*}
	$f_n(c)$ имеет конечный предел, теперь все условия теоремы выполнены:
	\begin{gather*}
		\left( \sum_{n=1}^\infty u_n(x) \right)'
		= \left( \lim_{n \ra \infty} f_n(x) \right)'
		= \lim_{n \ra \infty} f_n'(x)
		= \lim_{n \ra \infty} \sum_{k=1}^n u_k'(x)
		= \sum_{n=1}^\infty u_k'(x)
	\end{gather*}
\end{proof}

\begin{exmp}
	Условие на равномерную сходимость ряда производных нужно:
	\begin{gather*}
		\sum_{n=1}^\infty \frac{\sin nx}{n^2} \text{ "--- равномерно сходится по Вейерштрассу} \\
		\left|\frac{\sin nx}{n^2}\right| \le \frac1{n^2} \quad \sum_{n=1}^\infty \frac1{n^2} \text{ сходится} \\
		\sum_{n=1}^\infty \left(\frac{\sin nx}{n^2}\right)' = \sum_{n=1}^\infty \frac{\cos nx}{n} \text{ "--- расходится при $x=2\pi k$}
	\end{gather*}
\end{exmp}

\section{Степеннные ряды}

\begin{Def}
	$a_n \in \C$, $z, z_0 \in C$.
	Степенной ряд "--- это
	\[ \sum_{n=0}^\infty a_n (z - z_0)^n \]
\end{Def}

\begin{Rem}
	Далее считаем, что $z_0 = 0$.
\end{Rem}

\begin{theorem}[Абеля]
	Если ряд $\sum_{n=0}^\infty a_n z^n$ сходится при $z = z_0$, то он абсолютно сойдется и при всех $z$, что $|z| < |z_0|$.
\end{theorem}
\begin{proof}
	$\sum_{n=0}^\infty a_n z_0^n$ сходится, значит $a_n z_0^n$ сходятся и ограниченны, откуда
	\begin{gather*}
		|a_n z_0^n| \le M
		\Ra |a_n z^n| = |a_n z_0^n| \cdot \left| \frac{z}{z_0} \right|^n \le M \left| \frac{z}{z_0} \right|^n
	\end{gather*}
	Но ряд $\sum_{n=0}^\infty M \left| \frac{z}{z_0} \right|^n$ сходится, так как это геометрическая прогрессия с основанием, меньшим 1.
	Отсюда $\sum_{n=0}^\infty |a_n z^n|$ сходится, а $\sum_{n=0}^\infty a_n z^n$ сходится абсолютно.
\end{proof}

\begin{conseq}
	Если ряд $\sum_{n=0}^\infty a_n z^n$ расходится при $z = z_0$, то он разойдется и при всех $z$, что $|z| > |z_0|$.
\end{conseq}

\begin{exmp}
	Важно строгое равенство:
	\[ \sum_{n=0}^\infty \frac{z^n}n \]
	При $z = 1$ расходится, а при $z = -1$ сходится.
\end{exmp}

\begin{Def}
	Радиус сходимости степенного ряда "--- такое $R \in [0, \infty]$, что этот ряд сходится при $|z| < R$ и расходится при $|z| > R$.
\end{Def}

\begin{Def}
	Круг сходимости "--- открытый круг радиуса $R$.
\end{Def}

\begin{theorem}
	\begin{enumerate}
	\item
		Радиус сходимости всегда существует.

	\item
		Если $R$ "--- радиус сходимости и $0 \le r < R$,
		то в круге $|z| \le r$ ряд равномерно абсолютно сходится (как функциональный ряд от $z$).
	\end{enumerate}
\end{theorem}
\begin{proof}
	\begin{enumerate}
	\item
		Рассмотрим ряд $\sum_{n=0}^\infty a_n x^n$, и множество $A$ таких $x \ge 0$, на которых есть сходимость.
		Оно не пусто, так как там есть ноль.
		Значит $\sup A \in [0, \infty]$.

		Поймём, что это и будет радиусом сходимости.
		Пусть $|z| < \sup A$.
		Тогда есть $x \in A$, что $|z| < |x|$ и $\sum_{n=0}^\infty a_n x^n$ сходится.
		Отсюда по теореме Абеля $\sum_{n=0}^\infty a_n z^n$ сходится.

		Аналогично, пусть $|z| > \sup A$.
		Тогда есть $x \in \bar A$, что $|z| > |x|$ и $\sum_{n=0}^\infty a_n x^n$ расходится.
		Отсюда по следствию $\sum_{n=0}^\infty a_n z^n$ расходится.

	\item
		Пусть $r < R$.
		Тогда есть $x \in A$, что $r < x \le R$, значит ряд $\sum_{n=0}^\infty a_n x^n$ сходится,
		откуда $a_n x^n$ ограниченно.

		Возьмём $z\colon |z| \le r$.
		\begin{gather*}
			|a_nz^n| = |a_nx^n| \cdot \left| \frac zx \right|^n \le M \left( \frac rx \right)^n
		\end{gather*}
		Мы знаем, что $\sum_{n=0}^\infty M \left( \frac rx \right)^n$ сходится,
		откуда $\sum_{n=0}^\infty |a_n z^n|$ равномерно сходится по сравнению.
	\end{enumerate}
\end{proof}

\begin{exmp}
	\[ \sum_{n=0}^\infty z^n \]
	Радиус "--- 1.
\end{exmp}

\begin{exmp}
	\[ \sum_{n=0}^\infty \frac{z^n}{n!} \]
	Радиус "--- $+\infty$ по признаку Даламбера.
\end{exmp}

\begin{exmp}
	\[ \sum_{n=0}^\infty n! z^n \]
	Радиус "--- $0$ по признаку Даламбера.
\end{exmp}

\begin{theorem}[Абеля]
	Пусть ряд $\sum_{n=0}^\infty a_n z^n$ сходится при $z = R \ge 0$.
	Тогда ряд сходится равномерно на $[0, R]$.
\end{theorem}
\begin{proof}
	$x \in [0, R]$
	\begin{gather*}
		\sum_{n=0}^\infty a_n x^n
		= \sum_{n=0}^\infty a_n R^n \left(\frac xR\right)^n
	\end{gather*}
	Ряд $\sum_{n=0}^\infty a_n R^n$ равномерно сходится, поскольку от $x$ не зависит,
	а $\left(\frac xR\right)^n$ равномерно ограниченна единицей и монотонна.
	По признаку Абеля $\sum_{n=0}^\infty a_n x^n$ равномерно сходится на $[0, R)$.
\end{proof}

\begin{conseq}
	Пусть ряд $\sum_{n=0}^\infty a_n z^n$ сходится при $z = R \ge 0$.
	Тогда $f(x) = \sum_{n=0}^\infty a_n x^n$ непрерывна на $[0, R]$
	и в частности
	\[ \lim_{x \ra R-0} \sum_{n=0}^\infty a_n x^n = \sum_{n=0}^\infty a_n R^n \]
	(непрерывность в $R$)
\end{conseq}
\begin{proof}
	$S_n(x) = \sum_{k=0}^n a_k x^k$ непрерывна, $S_n \rra f$ на $[0, R]$, отсюда $f$ непрерывна на $[0, R]$.
\end{proof}

\begin{theorem}[Формула Коши-Адамара]
	\[ R = \frac1{\varlimsup\limits_{n \ra \infty} \sqrt[n]{|a_n|}} \]
\end{theorem}
\begin{proof}
	\[ \rho \lrh \varlimsup\limits_{n \ra \infty} \sqrt[n]{|a_n|} \in [0, \infty] \]
	\begin{description}
	\item[$\rho = 0$:]
		Покажем, что $R = +\infty$ и при всех $z$ ряд сходится.
		Так как $\sqrt[n]{|a_n|} \ge 0$ и $\rho = 0$, то
		\[ \lim_{n \ra \infty} \sqrt[n]{|a_n|} = 0 \]
		И тогда
		\[ \exists N\colon \forall n > N, \sqrt[n]{|a_n|} < \frac1{2|z|} \Ra |a_n| < \frac1{2^n|z|^n} \Ra |a_nz^n| < \frac1{2^n} \]
		\begin{Rem}
			Можно было бы применить признак Коши.
		\end{Rem}

	\item[$\rho = +\infty$:]
		Покажем, что $R = 0$ и при всех $z \ne 0$ ряд расходится.
		\begin{gather*}
			\varlimsup\limits_{n \ra \infty} \sqrt[n]{|a_n|} = +\infty
			\Ra \exists \{n_k\}\colon \lim_{n \ra \infty} \sqrt[n_k]{\left|a_{n_k}\right|} = +\infty \Ra \\
			\Ra \exists N\colon \forall k > N, \sqrt[n_k]{\left|a_{n_k}\right|} > \frac1z \Ra \left|a_{n_k} z^{n_k}\right| > 1
		\end{gather*}
		Отсюда $a_n z^n \nra 0$ и ряд расходится.

	\item[$\rho \in (0, \infty)$:]
		Покажем, что $R = \frac1\rho$.
		Пусть $|z| < \frac1\rho$.
		Возьмём $\epsilon > 0$, что $|z| < \frac1{\rho + \epsilon}$.
		\begin{gather*}
			\varlimsup\limits_{n \ra \infty} \sqrt[n]{|a_n|} = \rho
			\Ra \lim_{n \ra \infty} \sup_{k \ge n} \sqrt[k]{|a_k|} = \rho \Ra \\
			\Ra \exists N\colon \forall n > N, \sup_{k \ge N} \sqrt[k]{|a_k|} < \rho + \epsilon
			\Ra \forall k > N, \sqrt[k]{|a_k|} < \rho + \epsilon \Ra \\
			\Ra \forall k > N, |a_k z^k| < (\rho + \epsilon)^k |z|^k
		\end{gather*}
		Значит ряд с некоторого места мажорируется геометрической прогрессией с знаменателем $(\rho+\epsilon)|z| < 1$.
		Пусть $|z| > \frac1\rho$.
		Возьмём $\epsilon > 0$, что $|z| > \frac1{\rho - \epsilon}$.
		\begin{gather*}
			\varlimsup\limits_{n \ra \infty} \sqrt[n]{|a_n|} = \rho
			\Ra \exists \{n_k\}\colon \lim_{n \ra \infty} \sqrt[n_k]{\left|a_{n_k}\right|} = \rho \Ra \\
			\Ra \exists N\colon \forall k > N, \sqrt[n_k]{\left|a_{n_k}\right|} > \rho - \epsilon
			\Ra \left|a_{n_k}\right| > (\rho - \epsilon)^{n_k} \Ra \\
			\Ra \left|a_{n_k} z^{n_k}\right| > ((\rho - \epsilon)|z|)^{n_k} > 1
		\end{gather*}
	\end{description}
\end{proof}

Пусть есть функции $f \in C[a, +\infty)$, $g \in C^1[a, +\infty)$, хотим исследовать сходимость интеграла:

\[ \int_a^{+\infty} f(x) g(x) \d x \]

Два признака сходимости:
\begin{theorem}[Признак Дирихле]
	\begin{enumerate}
		\item $F(x) = \int_a^x f(t) \d t$ ограниченная функция
		\item $g(x)$ монотонно стремится к нулю при $x \ra +\infty$.
	\end{enumerate}
\end{theorem}

\begin{theorem}[Признак Абеля]
	\begin{enumerate}
		\item $\int_a^{+\infty} f(x) \d x$ сходится
		\item $g(x)$ монотонна и ограничена
	\end{enumerate}
\end{theorem}

\begin{proof}
	\[ \int\limits_a^A f(x) g(x) \d x = F(x) g(x) \biggr|_a^A - \int\limits_a^A F(x) g'(x) \d x \]
	Знаем, что $F(x)$ ограниченная, а $g(x)$ стремится к нулю, значит первое слагаемое сходится.
	Для второго покажем, что сходится абсолютно:
	\[ \int\limits_a^{+\infty} |F(x)|\cdot|g'(x)| \d x \]
	$|F(x) \le M|$, $g$ монотонна, значит $g'$ знакопостоянна.
	Не умаляя общности, пусть $g' \ge 0$.
	\begin{gather*}
		|F(x)| \cdot |g'(x)| \le Mg'(x) \\
		\int\limits_a^{+\infty} Mg'(x) \d x = Mg(x) \biggr|_a^{+\infty} = -Mg(a)
	\end{gather*}
	Таким образом, всё сходится.
\end{proof}

\begin{proof}
	\[ F(x) \xlongrightarrow{x \ra +\infty} \int\limits_a^{+\infty} f(t) \d t\]
	Значит $F(x)$ ограничена.
	$g(x)$ монотонна и ограничена, значит существует конечный предел $\lim_{x\ra+\infty} g(x) = K$.
	$\tilde g(x) = g(x) - K$.
	Применим признак Дирихле для интеграла
	\[ \int\limits_a^{+\infty} f(x) \tilde g(x) \d x \]
	Он сойдётся.
	\[ \int\limits_a^{+\infty} f(x) g(x) \d x = \int\limits_a^{+\infty} f(x) \tilde g(x) \d x + K \int\limits_a^{+\infty} f(x) \d x \]
	Второй сходится по условию.
\end{proof}

\begin{Rem}
	Непрерывная дифференцируемость $g$ не нужна, но без неё сложно.
\end{Rem}

\begin{conseq}
	$f$ непрерывна и $T$-периодична, $g \in C^1[a, +\infty)$ и монотонно стремится к нулю.
	Тогда
	\[ \int\limits_a^{+\infty} f(x) g(x) \d x < \infty \La \int\limits_a^{a+T} f(x) \d x = 0 \]
	а если второй интеграл не 0, то
	\[ \int\limits_a^{+\infty} f(x) g(x) \d x < \infty \Lra \int\limits_a^{a+T} g(x) < \infty \]
\end{conseq}

\begin{proof}
	Хотим сказать, что если интеграл по периоду ноль, то
	\[ \int\limits_a^x f(t) \d t = \int\limits_{a+kT}^x f(t) \d t = \int\limits_0^{x-kT} f(t) \d t = F(x - kT) \]
	где $k = \left\lfloor \frac{x - a}T \right\rfloor$.
	Отсюда $\max |F(x)| = \max_{a \le x < a + T} |F(x)| \le M$.

	Обратно: пусть интеграл периода не нулевой и равен $c$.
	Рассмотрим $\tilde f(x) = f(x) - \frac{c}T$.
	Воспользуемся уже доказанным направлением следствия.
	\[ \int\limits_a^{+\infty} f(x) g(x) \d x = \int\limits_a^{+\infty} \tilde f(x) g(x) \d x + \frac{c}T \int\limits_a^{+\infty} g(x) \d x \]
	Средний интеграл сходится, значит, сходимость первого и последнего равносильны.
\end{proof}

\begin{exmp}\label{exmpIntFracSin}
	$p > 0$:
	\[ \int\limits_1^{+\infty} \frac{\sin x}{x^p} \d x \]
	$g(x) = 1 / x^p$, $g \downarrow 0$, $\sin$ периодическая и интеграл по периоду есть 0.
	Значит интеграл сходится.

	Теперь рассмотрим абсолютную сходимость при $p>0$.
	Интеграл модуля синуса по периоду не ноль, и интеграл сойдётся тогда и только тогда, когда сойдётся
	\[ \int\limits_1^{+\infty} \frac{\d x}{x^p} \]
	то есть при $p > 1$.
	А при $0 < p \le 1$ абсолютной сходимости нет.

	При $p \le 0$ не сходится:
	\[ \int\limits_{\pi/6+2\pi k}^{5 \pi/6 + 2\pi k} \frac{\sin x}{x^p} \ge \frac12 \left(\frac{5\pi}6 - \frac\pi6\right) = \frac\pi3 \]
	Не сходится по критерию Коши.
\end{exmp}

\section{Несобственные кратные интегралы}

\begin{Def}
	$G \subset \R^n$ открыто.
	${G_i}$ монотонно исчерпывет $G$, если $G_n$ открыто, $G_k \supset \cl G_{k-1}$ и $G = \bigcup G_k$.
\end{Def}

\begin{theorem}
	$G \subset \R^n$ открыто.
	Тогда существует последовательность $G_i$ открытых измеримых монотонно исчерпывающих $G$ множеств.
\end{theorem}

\begin{proof}
	Нарежем $\R^n$ на замкнутые кубики с единичной стороной.
	Беру те, что целиком влезли в $G$, беру внутренность выбранного множества как $G_1$.
	Продолжим алгоритм с кубиками со стрононами $1/2$, $1/4$ и так далее.
	Получим множества, открытые, измеримые, также покрывающие в конце $G$.

	Покажем, что для каждого $k$ есть $m$, что $\cl G_k \subset G_m$.
	Возьмём граничную точку $G_k$. Его граничная точка лежит в открытом $G$, при каком-то $m_1$ эту точку накроет.
	Теперь покажем, что можно выбрать $m$ сразу для всех точек $\cl G_k$.
	Рассмотрим фнукцию расстояния от точки $\cl G_k$ до $\partial G$.
	Она непрерывна на компакте и больше нуля, значит есть минимум, больший нуля.
	Подгоним $m$ под него.
\end{proof}

\begin{Def}
	$f\colon G \ra \R^n$, интегрируема на любом измеримом $D \subset G$, таких что $\cl D \subset G$, $G$ открыто.
	\[ \int\limits_G f(x) \d x = A \]
	если для любой последовательности измеримых $G_k$ монотонно исчерпывающих $G$
	\[ \lim_{n\ra+\infty} \int\limits_{G_k} f(x) \d x = A \]
\end{Def}
\begin{Rem}
	Для $n = 1$ определение \underline{\bf совсем} другое.
	Оно слабее: там мы брали интегралы только из начальной точки до некоторой, а тут можем брать интегралы по вообще любым объединениям отрезков.
	Тогда, например, в примере \href{exmpIntFracSin} у нас сходимости при $0<p<1$ не получилось: можем в $G_k$ брать $2k$ <<пиков>> синусов и $k$ <<провалов>>,
	тогда интегралы будут увеличиваться, но мы, тем не менее, всё покроем.
\end{Rem}

\begin{theorem}
	$f \ge 0$, $f\colon G \ra \R$, $G$ открыто.
	Тогда для каждой последовательности $G_k$ предел существует: конечный или бесконечный, и не зависит от выбора $G_k$.
\end{theorem}
\begin{proof}
	Так как $G_k \subset G_{k+1}$, то $\int_{G_k} f \le \int_{G_{k+1}} f$.
	Эта последовательность монотонно растёт, значит есть предел (возможно, не конечный).

	Едиснтвенность: пусть есть другая последовательность $G'_k$.
	Замыкание конечного измеримого множества есть компакт.
	\[ \cl G'_m \subset \bigcup G_k = G \]
	Компакт $\cl G'_i$ покрыт конечным числом открытых, можно выбрать конечное подпокрытие, но они все вложенны, значит лежит в самом большом:
	\[ G'_m \subset \cl G'_m \subset G_l \]
	Значит множества $G'_k$ зажаты $G_k$-ыми, значит интералы зажаты, значит у $\int_{G'_i} f$ предел такой же, как у $\int_{G_i} f$.
\end{proof}

\begin{conseq}[Признак сравнения]
	$f \ge g \ge 0$, $f,g \colon G \ra \R$, $G$ открыто.
	\begin{enumerate}
		\item Если $\int_G f$ сходится, то и $\int_G g$ сходится.
		\item Если $\int_G g$ расходится, то и $\int_G f$ расходится.
	\end{enumerate}
\end{conseq}
\begin{proof}
	Перейдём к последовательностям, там знаки сохранятся.
\end{proof}

\begin{Def}
	Интеграл функции абсолютно сходится, если сходится интеграл модуля.
\end{Def}

\begin{theorem}
	Для $n > 1$ абсолютная сходимость равносильна сходимости.
\end{theorem}
% \begin{proof}
% 	\begin{description}
% 	\item[$\Ra:$]
% 		Возьмём последовательность $G_k$ монотонно исчерпывающих $G$.
% 		\begin{gather*}
% 			\int\limits_{G_k} f \lrh a_k \quad \int\limits_{G_{k+1}} |f| \lrh b_k \\
% 			b_{k+1} - b_k \ge |a_{k+1} - a_k| \\
% 			b_m - b_l = \int\limits_{G_m \setminus G_l} f \ge \left| \int\limits_{G_m \setminus G_l} f \right| = |a_m - a_l|
% 		\end{gather*}
% 		Отсюда $a_i$ фундаментальна.
% 	\end{description}
% \end{proof}
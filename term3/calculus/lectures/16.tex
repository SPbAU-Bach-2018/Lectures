\begin{Rem}
	Поговорим о внешних 3-формах в $\R^3$. Значение формы однозначно задано её значением $\omega(e_1, e_2, e_3)$
	(так как отсюда оно однозначно задаётся на всех перестановках базисных и на всех полилинейных комбинациях).
	Тогда размерность пространства таких форм один, и главного представителя обозначим $\d x \wedge \d y \wedge \d z$.
\end{Rem}

\begin{Def}[Перенос формы]
	Есть множества $U$ и $V$, а также гладкое отображение $\phi\colon U \ra V$.
	На $V$ есть дифференциальная форма из $\Sigma(V)$ "--- $\omega(x; \xi_1, \dots, \xi_k)$ (взяли в точке $x \in V$).
	Тогда перенос формы $\phi^*$ из $V$ в $U$:
	\[ \phi^*(\omega)(t;\eta_1,\dots,\eta_k) \eqDef \omega(\phi(t); (\d_t \phi)(\eta_1), \dots, (\d_t \phi)(\eta_k)) \]
\end{Def}

Свойства:
\begin{enumerate}
\item
	$\phi^*$ линейно (как оператор над формой):
	\[ \phi^*(a\alpha + b\beta) = a\phi^*(\alpha) + b\phi^*(\beta) \]

\item
	Пусть есть множества $U$, $V$ и $W$ с переносами $\phi^*\colon \Omega(V) \ra \Omega(U)$ и $\psi^*\colon \Omega(W) \ra \Omega(V)$.
	Тогда
	\[ \phi^* \circ \psi^* = (\psi \circ \phi)^* \]

\item
	Для произвольной $f\colon V \ra \R$
	\[ \phi^*(f \cdot \omega) = (f \circ \phi) \cdot \phi^*(\omega) \]
	То есть мы домножили старую форму на константу $f(x)$ (зависяющую от точки), а после переноса эта константа вылезла наружу (уже в виде $f(\phi(u))$).
	\begin{Rem}
		В реальной жизни это означает, что при переносе формы вида $P \d x + Q \d y + R \d z$ надо почленно домножить коэффициенты на $f$.
	\end{Rem}
	\begin{proof}
		\begin{gather*}
			\phi^* (f \cdot \omega)(t;\eta_1, \dots, \eta_n)
			= (f \cdot \omega)(\phi(t); \d \phi_t(\eta_1), \dots, \d \phi_t(\eta_k)) = \\
			= f(\phi(t)) \cdot \omega (\phi(t); \d \phi_t(\eta_1), \dots, \d \phi_t(\eta_k))
			= (f \circ \phi) \cdot \phi^*(\omega)
		\end{gather*}
		Если расписать то же самое подробнее и в программистском стиле, получится так:
		\begin{gather*}
			(\phi^* (f \cdot \omega))(t)(\eta_1, \dots, \eta_n)
			= [(f \cdot \omega)(\phi(t))](\d \phi_t(\eta_1), \dots, \d \phi_t(\eta_k)) = \\
			= [\underbrace{f(\phi(t))}_{\substack{\text{конкретное число,}\\\text{от $\eta$ не зависит}}} \cdot \underbrace{\omega(\phi(t))}_{\text{внешняя $k$-форма}}](\d \phi_t(\eta_1), \dots, \d \phi_t(\eta_k)) = \\
			= f(\phi(t)) \cdot [\omega(\phi(t))](\d \phi_t(\eta_1), \dots, \d \phi_t(\eta_k)) = \\
			= (f \circ \phi)(t) \cdot [\omega(\phi(t))](\d \phi_t(\eta_1), \dots, \d \phi_t(\eta_k)) = \\
			= (f \circ \phi)(t) \cdot (\phi^*(\omega))(t)(\eta_1, \dots, \eta_n)
		\end{gather*}
	\end{proof}

\item
	$\omega_1$, $\omega_2$ "--- внешние 1-формы.
	Их можно естественно преобразовать в дифференциальную форму, не зависяющую от точки $t$.
	Тогда:
	\[ \phi^*(\omega_1 \wedge \omega_2) = (\phi^* \omega_1) \wedge (\phi^* \omega_2) \]
	\begin{proof}
		\begin{align*}
			(\phi^* \omega_1) (t; \xi) &= \omega_1(\phi(t); \d \phi_t(\xi)) \\
			(\phi^* \omega_2) (t; \xi) &= \omega_2(\phi(t); \d \phi_t(\xi)) \\
			((\phi^* \omega_1) \wedge (\phi^* \omega_2)) (t; \xi_1, \xi_2)
			&= ((\phi^* \omega_1)(t) \wedge (\phi^* \omega_2)(t)) (\xi_1, \xi_2) = \\
			&= \begin{vmatrix}
				(\phi^* \omega_1)(t; \xi_1) & (\phi^* \omega_1)(t; \xi_2) \\
				(\phi^* \omega_2)(t; \xi_1) & (\phi^* \omega_2)(t; \xi_2) \\
			\end{vmatrix}
			= \begin{vmatrix}
				\omega_1(\d \phi_t(\xi_1)) & \omega_1(\d \phi_t(\xi_2)) \\
				\omega_2(\d \phi_t(\xi_1)) & \omega_2(\d \phi_t(\xi_2)) \\
			\end{vmatrix} \\
			(\phi^* (\omega_1 \wedge \omega_2)) (t; \xi_1, \xi_2)
			&= (\omega_1 \wedge \omega_2)(\phi(t); \d \phi_t(\xi_1), \d \phi_t(\xi_2)) = \\
			&= \begin{vmatrix}
				\omega_1(\d \phi_t(\xi_1)) & \omega_1(\d \phi_t(\xi_2)) \\
				\omega_2(\d \phi_t(\xi_1)) & \omega_2(\d \phi_t(\xi_2)) \\
			\end{vmatrix}
		\end{align*}
	\end{proof}
\end{enumerate}

\begin{exmp}
	Возьмём понятную форму $P \d x + Q \d y$, живущую в $V$ по переменных $(x,y)$,
	возьмём отображение $\phi$ из $U$ на переменных $(u,v)$.
	\begin{gather*}
		\phi \begin{pmatrix} u \\ v \end{pmatrix} \coloneq \begin{pmatrix} x(u, v) \\ y(u, v) \end{pmatrix} \\
		\d \phi = \begin{pmatrix} x_u' & x_v' \\ y_u' & y_v' \end{pmatrix} \\
		\eta \coloneq \begin{pmatrix} a \\ b \end{pmatrix} \\
		(\d \phi)(\eta) = \begin{pmatrix} x_u'a + x_v'b \\ y_u'a + y_v'b \end{pmatrix} \\
		\d x ((\d \phi)(\eta)) = x_u'a + x_v'b \\
		\d y ((\d \phi)(\eta)) = y_u'a + y_v'b \\
		(\phi^* \omega) (u, v; \eta)
		= \omega(\phi(u, v); \d \phi(\eta))
		= P(x(u, v), y(u, v)) (x_u' a + x_v' b)
		+ Q(x(u, v), y(u, v)) (y_u' a + y_v' b) = \\
		= P(x(u, v), y(u, v)) (x_u' \d u(\eta) + x_v' \d v(\eta))
		+ Q(x(u, v), y(u, v)) (y_u' \d u(\eta) + y_v' \d v(\eta)) \\
		\phi^* (P \d x + Q \d y) = P(x_u' \d u + x_v' \d v) + Q(y_u' \d u + y_v' \d v)
	\end{gather*}
	Но мы уже знаем эту запись, это замена переменной!
	По сути, мы как бы записали дифференциалы $\d x$ и $\d y$ с учётом того, что
	$x$ и $y$ зависят от $u$ и $v$.

	Еще пример: можно брать 1-форму на прямой и переносить прямую в плоскость (получать кривую),
	получим что-то похожее на криволинейный интеграл II рода:
	\begin{gather*}
		\gamma\colon [a, b] \ra \R^2 \quad \d x = \gamma_1'(t) \d t \quad \d y = \gamma_2'(t) \d t
	\end{gather*}

	\TODO см. детали в конспекте Нади за консультацию от 21.01.2016, видимо, страница 9
\end{exmp}

\begin{Def}[Поверхностный интеграл от формы (поверхностный интеграл II рода)]
	Возьмём 2-форму $\omega$, простую поверхность $S = \vec r(\cl G)$.
	\begin{description}
	\item[$S \subset \R^2$:]
		$\omega = f \cdot \d x \wedge \d y$, и тогда
		\[ \int\limits_S \omega = \pm\int\limits_S f \d x \d y \]
		Знак определяется направлением нормали: если нормаль вверх, то плюс.

	\item[$S \subset \R^3$:]
		\[ \int\limits_S \omega = \int\limits_G \vec r^* \omega \]
		При этом ориентация переносится согласованно:
		возьмём тройку из базисных векторов и нормали в $S$, перенесём её по прообразам в $G$, и теперь смотрим на знак интеграла в $\R^2$.
	\end{description}
\end{Def}

\begin{Rem}
	Теперь нам придётся проверять корректность "--- мы могли брать разные допустимые $\vec r$ и получать очень разные знаки и значения.
\end{Rem}

\begin{theorem}[Корректность определения]
	\begin{enumerate}
	\item
		$\phi\colon G \subset \R^2 \ra \Omega \subset \R^2$.
		То есть мы взяли кусок плоскости как повехность.
		Хотим показать, что
		\[ \int\limits_\Omega \omega = \int\limits_G \phi^* \omega\]
		Для этого
		\begin{gather*}
			\int\limits_\Omega f \d x \d y = \int\limits_\Omega \omega \quad
			\int\limits_G \phi^* \omega = \pm^1 \int\limits_G (f \circ \phi) J_\phi \d u \d v \\
			\phi^* \omega
			= f \circ \phi \cdot \phi^*(\d x) \wedge \phi^*(\d y)
			= f \circ \phi \cdot (x_u' \d u + x_v' \d v) \wedge (y_u' \d u + y_v' \d v) = \\
			= f \circ \phi \cdot (x_u' y_v' - x_v' y_u') \d u \d v
			= f \circ \phi \cdot J_\phi \d u \wedge \d v
		\end{gather*}
		Мы в $1$ написали как будто якобиан положителен.
		Если он отрицателен, то вылезет минус, но тогда нормаль не в ту сторону, и мы поставим минус второй раз.

	\item
		Теперь хотим показать, что для допустимой параметризации $\vec R = \vec r \circ w$ всё хорошо:
		\[ \int\limits_G \vec r^* \omega \stackrel{?}{=} \int\limits_\Omega \vec R^* \omega \]
		\begin{gather*}
			\vec R^* = (\vec r \circ w)^* = w^* \vec r^* \\
			\tilde \omega \lrh \vec r^* \omega \\
			\int\limits_G \vec r^* \omega = \int\limits_\Omega \vec R^* \omega \Lra
			\int\limits_G \tilde \omega = \int\limits_\Omega w^* \tilde \omega
		\end{gather*}
		А это уже предыдущий пункт.
	\end{enumerate}
\end{theorem}

\begin{Def}[Интеграл дифференциальной формы по кусочно-гладкой поверхности]
	Взяли простые куски $S_i$ (их конечное число), задали согласованную с $S$ ориентацию, взяли сумму интегралов.
\end{Def}

\begin{Def}
	$\vec F$ "--- векторное поле, если в каждой точке свой вектор той же размерности, что и пространство.
\end{Def}

\begin{Def}
	$\omega = P \d y \wedge \d z + Q \d z \wedge \d x + R \d x \wedge \d y$ "--- форма потока поля $\vec F$.
\end{Def}

\begin{Def}
	Поток поля через поверхность "--- интеграл
	\[ \int\limits_S \omega \]
\end{Def}

\begin{theorem}
	$\vec F = (P, Q, R)$ "--- поле.
	\[ \int\limits_S P \d y \wedge \d z + Q \d z \wedge \d x + R \d x \wedge \d y = \int\limits_S \left<\vec F, \vec n\right> \d S \]
\end{theorem}
\begin{proof}
	Достаточно доказать для простой поверхности $S = \vec r(\cl G)$.
	\begin{gather*}
		\left<\vec F, \vec h\right>
		= \frac1{\|\vec r_u' \times \vec r_v'\|} \left<\vec F, \vec r_u' \times \vec r_v'\right>
		= \frac1{\|\vec r_u' \times \vec r_v'\|} \begin{vmatrix} P & Q & R \\ x_u' & y_u' & z_u' \\ x_v' & y_v' & z_v' \end{vmatrix} \\
		\int\limits_S \left<\vec F, \vec h\right>
		= \int\limits_G \frac1{\|\vec r_u' \times \vec r_v'\|} \begin{vmatrix} P & Q & R \\ x_u' & y_u' & z_u' \\ x_v' & y_v' & z_v' \end{vmatrix} \sqrt{EG-F^2} \d u \d v
		= \int\limits_G \begin{vmatrix} P & Q & R \\ x_u' & y_u' & z_u' \\ x_v' & y_v' & z_v' \end{vmatrix} \d u \d v = \\
		= \int\limits_G P\underbrace{(y_u' z_v' - y_v' z_u')}_{\d y \wedge \d x} + Q\underbrace{(z_u' x_v' - z_v' x_u')}_{\d z \wedge \d x} + R\underbrace{(x_u' y_v' - x_v' y_u')}_{\d x \wedge \d y} \d u \d v = \\
		\begin{aligned}
			\d x &= x_u' \d u + x_v' \d v \\
			\d y &= y_u' \d u + y_v' \d v \\
			\d z &= z_u' \d u + z_v' \d v
		\end{aligned} \\
		= \int\limits_S R \d y \wedge \d z + Q \d z \wedge \d x + R \d x \wedge \d y
	\end{gather*}
\end{proof}

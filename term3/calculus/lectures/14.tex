\begin{Def}
	$S = \vec r(\cl G)$. Тогда край $S$
	\[ \partial S = \vec r(\partial G) = \vec r(\cl G \setminus G) \]
\end{Def}

\begin{Def}
	Кусочно-гладкая поверхность "--- объединение простых гладких поверхностей, пересекающихся только по краям.
\end{Def}

\begin{exmp}
	Поверхность куба.
\end{exmp}

\begin{Def}
	Край кусочно-гладкой поверхности "--- объединение тех краёв, которые принадлежали ровно простой поверхности.
\end{Def}

\begin{Def}
	Пусть есть простая градкая поверхность $S = \vec r(\cl G)$ и точка на ней $\vec r(u, v)$.
	Касательная поверхность "--- поверхность, натянутая на вектора производной
	\[ \mathcal{Lin}(\vec r_u', \vec r_v') \]
\end{Def}

\begin{Def}
	Нормаль "--- единичный вектор, ортогональный касательной плоскости.
	\[ \vec n = \pm\frac{\vec r_u' \times \vec r_v'}{\|\vec r_u' \times \vec r_v'\|} \]
	Нам неважно пока, в какую сторону нормаль, поэтому плюс-минус.
\end{Def}

\begin{Def}
	Простая градкая поверхность $S = \vec r(\cl G)$.
	Допустимая параметризация "--- такая параметризация $S = \vec R(\cl \Sigma)$,
	что $\vec R = \vec r \circ w$, а $w\colon \cl \Sigma \ra \cl G$ непрерывно дифференцируема, и $J_w \ne 0$.
\end{Def}

\begin{theorem}
	Касательная плоскость и нормаль к простой гладкой поверхности не зависят от параметризации.
\end{theorem}
\begin{proof}
	\begin{gather*}
		\vec R_\xi'
		= (\vec r(u(\xi,\eta),v(\xi,\eta)))_\xi'
		= \vec r_u' \circ \vec u_\xi'  + \vec r_v' \circ v_\xi'  \\
		\vec R_\eta'
		= (\vec r(u(\xi,\eta),v(\xi,\eta)))_\eta'
		= \vec r_u' \circ \vec u_\eta' + \vec r_v' \circ v_\eta' \\
		\vec R_\xi' \times \vec R_\eta'
		= (\vec r_u' \circ \vec u_\xi'  + \vec r_v' \circ v_\xi' ) \times (\vec r_u' \circ \vec u_\eta' + \vec r_v' \circ v_\eta')
		= (\vec r_u' \times \vec r_v') \cdot (u_\xi' \cdot v_\eta' - u_\eta' \cdot v_\xi')
		= (\vec r_u' \times \vec r_v') \cdot J_w \ne 0
	\end{gather*}
\end{proof}

\begin{Def}
	Гладкая поверхность называется ориентируемой, если в каждой точке сущесвует нормаль, непрерывно зависящая от точки.
\end{Def}

\begin{Rem}
	Если есть одна ориентация, то есть две (возьмём с минусом).
\end{Rem}

\begin{Rem}
	Простая гладкая поверхность всегда ориентируема:
	\[ \vec n = \pm \frac{\vec r_v' \times \vec r_u'}{\|\vec r_v' \times \vec r_u'\|} \]
\end{Rem}

\begin{Rem}
	Лента Мёбиуса не ориентируема.
	Берём нормаль в какой-то точке и потащим её по непрерывности вдоль ленты.
	Когда вернёмся в исходную точку, мы перейдём на другую сторону ленты, а нормаль не должна была поменяться.

	На самом деле можно показать, что если поверхность неориентируема, то из неё можно вырезать ленту Мёбиуса.
\end{Rem}

\begin{Def}
	Возьмём гладкую простую поверхность $S = \vec r(\cl G)$.
	Обойдём $\partial G$ как у нас принято (чтобы от $u$ к $v$ угол рос).
	Ориентируем поверхность
	\[ \vec n = \frac{\vec r_v' \times \vec r_u'}{\|\vec r_v' \times \vec r_u'\|} \]
	А также возьмём обход края, перетащенного с $\partial G$.
	Такая ориентация называется согласованной.

	Проще говоря, направление ориентации и обход края связаны правилом правой руки.
\end{Def}

Теперь ориентируем кусочно-гладкие поверхности.

\begin{Def}
	Ориентация кусочно-гладкое поверхности согасована, если на каждом куске ориентация согласована,
	а также пересечения кусков, происходящих только по кривым, имеют встречные направления обхода кусков.
\end{Def}

Тут не очень ясно, что делать с плохими пересечениями (например, по точкам).
Можно просто потребовать, чтобы пересечения были только по кривым, и там всё хорошо.

\section{Поверхностный интеграл I рода}

\begin{Def}
	$S = \vec r(\cl G)$ "--- простая гладкая поверхность, $G = \bigcup_{i=1}^n G_i$, пересекаются только по множествам нулевой меры.
	В каждом $G_i$ выберем $\xi_i = (u_i, v_i)$.
	Возьмём линейное приближение
	\[ \vec r_i(u, v) = \vec r(u, v) + \vec r_u' (u_i, v_i)(u - u_i) + \vec r_v'(u_i, v_i)(v - v_i) \]
	Теперь возьмём $S_i = \vec r_i(\cl G_i)$ "--- кусок касательной плоскости, <<чешуйка>>.
	Далее, мы автоматически получим возможность считать меру $\mu S_i$.
	Далее, посчитаем отношение площадей (мы считаем, что эта формула известна):
	\[ \frac{\mu S_i}{\mu G_i} = \|\vec r_u' \times \vec r_v' \| \]
	Тогда $\sigma$ "--- площадь поверхности $S$, если
	\[
		\forall \epsilon > 0, \exists \delta > 0\colon \forall \{G_i\}_{i=1}^n\colon \bigcup_{i=1}^n G_i = G\colon
		\diam G_i < \delta, \left| \sum_{i=1}^n \mu S_i - \sigma \right| < \epsilon
	\]
\end{Def}

\begin{theorem}
	$S = \vec r(\cl G)$ "--- простая гладкая поверхность.
	Тогда
	\[ \sigma(S) = \int\limits_{G} \|\vec r_u'(u, v) \times \vec r_v'(u, v)\| \d u \d v \]
\end{theorem}
\begin{proof}
	$g(u, v) = \|\vec r_u'(u, v) \times \vec r_v'(u, v)\|$ "--- непрерына на $\cl G$.
	Сумма Римана
	\[
		\sigma(g, \tau, \xi) = \sum_{i=1}^n g(u_i, v_i) \mu G_i = \sum_{i=1}^n \mu S_i
	\]
	Тогда по определению интеграла
	\[ \sigma(S) = \int\limits_{G} \|\vec r_u'(u, v) \times \vec r_v'(u, v)\| \d u \d v \]
\end{proof}

\begin{Def}
	Первая квадратичная форма поверхности.
	Возьмём какой-нибудь вектор в касательной плоскости:
	\[ \alpha \vec r_u' + \beta \vec r_v' \]
	Возьмём квадрат его длины:
	\begin{gather*}
		I = \|\alpha \vec r_u' + \beta \vec r_v'\|^2
		= \left<\alpha \vec r_u' + \beta \vec r_v',\alpha \vec r_u' + \beta \vec r_v'\right> = \\
		= \alpha^2 \left<\vec r_u', \vec r_u'\right> + 2\alpha\beta \left<\vec r_u', \vec r_v'\right> + \beta^2 \left<\vec r_v', \vec r_v'\right>
		= \alpha^2 E + 2\alpha\beta F + \beta^2 G \\
		I(\alpha, \beta)
		= \begin{pmatrix} \alpha & \beta \end{pmatrix} \begin{pmatrix} E & F \\ F & G \end{pmatrix} \begin{pmatrix} \alpha\\\beta \end{pmatrix}
	\end{gather*}
\end{Def}

Свойства:
\begin{enumerate}
\item
	Это положительно определённая квадратичная форма.

\item
	\[ EG - F^2 > 0 \]

\item
	\[ \|\vec r_u' \times \vec r_v' \|^2 = EG - F^2 \]
	\begin{proof}
		\begin{gather*}
			\vec r_u' \times \vec r_v'
		= \begin{vmatrix} \vec i & \vec j & \vec k \\ x_u' & y_u' & z_u' \\ x_v' & y_v' & z_v' \end{vmatrix}
			= (y_u' z_v' - z_u' y_v', z_u' x_v' - x_u' z_v', x_u' y_v' - y_u' x_v') \\
			 \|\vec r_u' \times \vec r_v' \|^2
			= (y_u' z_v' - z_u' y_v')^2 + (z_u' x_v' - x_u' z_v')^2 + (x_u' y_v' - y_u' x_v')^2 \\
			EG - F^2 = (x_u'^2 + y_u'^2 + z_u'^2)(x_v'^2 + y_v'^2 + z_v'^2) - (x_u'x_v' + y_u'y_v' + z_u'z_v')^2
		\end{gather*}
		Они равны, проверьте.
	\end{proof}

\item
	Если $S$ "--- график функции $z(x,y)$, то
	\[ EG - F^2 = 1 + z_u'^2 + z_v'^2 \]
	\begin{proof}
		\begin{gather*}
			\vec r(x, y) = \begin{pmatrix} x \\ y \\ z(x, y) \end{pmatrix} \quad
			\vec r_x'(x, y) = \begin{pmatrix} 1 \\ 0 \\ z_x' \end{pmatrix} \quad
			\vec r_y'(x, y) = \begin{pmatrix} 0 \\ 1 \\ z_y' \end{pmatrix} \\
			E = \left< r_x', r_x' \right> = 1 + z_x'^2 \quad
			F = \left< r_x', r_y' \right> = z_x'z_y' \quad
			G = \left< r_y', r_y' \right> = 1 + z_y'^2
		\end{gather*}
	\end{proof}
\end{enumerate}

\begin{conseq}
	\[
		\sigma(S) = \int\limits_G \sqrt{EG - F^2} \d u \d v
	\]
\end{conseq}

\begin{conseq}
	$S$ "--- график функции $z(x, y)$, то
	\[
		\sigma(S) = \int\limits_G \sqrt{1 + z_x'^2(x, y) + z_y'^2(x, y)} \d x \d y
	\]
\end{conseq}

\begin{Def}
	$S = \vec r(\cl G)$ "--- простая гладкая поверхность, $f\colon S \ra \R$ непрерывна.
	Тогда
	\[
		\int\limits_S f \d S \eqDef \int\limits_G f(\vec r(u, v)) \| \vec r_u' \times \vec r_v' \| \d u \d v
	\]
\end{Def}

\begin{theorem}
	Интеграл не зависит от параметризации.
\end{theorem}
\begin{proof}
	Возьмём допустимую параметризацию. Тогда мы уже знаем, что
	\begin{gather*}
		\|\vec R_\xi' \times \vec R_\eta'\| = \|\vec r_u' \times \vec r_v'\| \cdot |J_w| \\
		\int\limits_\Omega f(\vec R(\xi, \eta)) \| \vec R_\xi' \times \vec R_\eta' \| \d\xi \d \eta
		= \int\limits_G f(\vec R(\xi, \eta)) \|\vec r_u' \times \vec r_v'\| \cdot |J_w| \d u \d v
		= \int\limits_G f(\vec r(u, v)) \| \vec r_u' \times \vec r_v' \| \d u \d v
	\end{gather*}
	Мы специально так определяли допустимую параметризацию, чтобы формула замены переменной работала.
\end{proof}

\begin{Rem}
	$S$ "--- график функции, то
	\[ \int\limits_S f \d S = \int\limits_G f(x, y, z(x, y)) \sqrt{1 + z_x'^2(x, y) + z_y'^2(x, y)} \d x \d y \]
\end{Rem}

\begin{Def}
	Пусть $S$ "--- кусочно-гладкая поверхность.
	\[ \int\limits_S f \d s = \sum_{i=1}^n \int\limits_{S_i} f \d S \]
	где $S_i$ "--- простые куски.
\end{Def}

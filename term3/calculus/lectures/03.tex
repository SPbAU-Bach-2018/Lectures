\begin{proof}\begin{description}
\item[Покажем измеримость $\phi(Q_h)$.]
	Покажем, что мера его границы ноль.
	\[
		\delta \phi(Q_h) = \phi(\delta Q_h)
	\]
	$\delta Q_h$ "--- это $2n$ кубиков меньшей размерности.
	Образ таких кубиков "--- сужение графика $f$, то есть множество нулевой меры.

\item[Доказательство формулы.]
	$R_h = \linf{\phi}_{t_0} (Q_h)$.
	\[ \frac{\mu R_h}{\mu Q_h} = |\det \d_{t^*} \phi| = |J_\phi(t^*)| \]
	Теперь перейдём к $\phi(Q_h)$.
	\begin{gather*}
		\sigma(h) \lrh \sup_{t^*\colon Q_h \subset G} \sup_{t \in Q_h} \|\phi(t) - \linf{\phi}_{t^*}(t)\| \\
		t, t^* \in Q_h \Ra \|t - t^*\| \le \sqrt{n} h \\
		\frac{\sigma(h)}{\sqrt{n} h} \le S_n \ra 0
	\end{gather*}
	Значит $\sigma(h) = o(h)$.
	Тогда посмотирм на $\phi(Q_h)$ и $R_h$.
	\[
		\forall t \in Q_h, \rho( \phi(t), \linf{\phi}_{t^*}(h) ) \le \sigma(h)
	\]
	Тогда $\phi(Q_h)$ содержится в $R_h$, раздутом на $\sigma(h)$ (обозначим $\tilde R_h$),
	а также содержит $R_h$, сдутый на $\sigma(h)$ (обозначим $\tilde{\tilde R}_h$).

	Хотим получить следующее:
	\begin{gather*}
		\tilde {\tilde R_h} \subset \phi(Q_h) \subset \tilde R_h \\
		\mu \tilde R_h = \mu R_h + o(h^n) \quad \mu \tilde{\tilde R}_h = \mu R_h + o(h^n) \\
		\Ra \frac{\mu \phi(Q_h)}{\mu Q_h} = \frac{\mu \phi(Q_h)}{h^n} = \frac{\mu R_h + o(h^n)}{h^n}
		= \frac{\mu R_h}{\mu Q_h} + o(1) = |J_\phi(t^*)| + o(1)
	\end{gather*}

	Теперь смотрим на $R_h$. Если мы его раздули на $\sigma(h)$.
	\begin{gather*}
		0 \le \mu R_h - \mu \tilde R_h \le 2n \sigma(h) \mu (\text{грань кубика}) \\
		(\text{сторона грани}) \le \|\d_{t^*} \phi\| h \le Mh \\
		(\text{сторона увелчившейся грани}) \le Mh + 2\sigma(h) \\
		(\text{объём грани кубика}) \le \sigma(h) (Mg + 2\sigma(h))^{n-1} = o(h^n)
	\end{gather*}
\end{description}\end{proof}

\begin{lemma}
	$G_1 \subset G_2 \subset G_3 \subset \dots \subset G_n \subset \dots \subset G$ "--- ограниченны и все измеримы.
	$f\colon \cl G \ra \R$ непрерывна, $\mu (G \setminus G_n) \ra 0$.
	Тогда
	\[ \lim_{i \ra \infty} \int\limits_{G_n} = \int\limits_G f \]
\end{lemma}
\begin{proof}
	$G$ ограничена, значит на замыкании тоже, значит $f$ непрерывна на комакте.
	\begin{gather*}
		\int\limits_G f - \int\limits_{G_n} f = \int\limits_{G \setminus G_n} f \\
		\left| \int\limits_G f - \int\limits_{G_n} f \right| = \left| \int\limits_{G \setminus G_n} f \right|
		\le \int\limits_{G \setminus G_n} |f| \le M \mu(G \setminus G_n) \ra 0
	\end{gather*}
\end{proof}

\begin{theorem}[Замена переменной в кратном интеграле]
	$G \subset \R^n$ "--- открыто и измеримо,\\
	$\phi\colon \cl G \ra \R^n$ биективна, непрерывна и непрерывно дифференцируема на $\cl G$,
	якобиан
	\[
		J_\phi(t) = \det
		\begin{pmatrix}
			\partd{\phi_1}{t_1} & \partd{\phi_1}{t_2} & \cdots & \partd{\phi_1}{t_n} \\
			\partd{\phi_2}{t_1} & \partd{\phi_2}{t_2} & \cdots & \partd{\phi_2}{t_n} \\
			\vdots              & \vdots              & \ddots & \vdots              \\
			\partd{\phi_n}{t_1} & \partd{\phi_n}{t_2} & \cdots & \partd{\phi_n}{t_n} \\
		\end{pmatrix} \ne 0
	\]
	$\phi(G) = D$ измеримо.
	Тогда для непрерывной $f\colon \cl D \ra \R$.
	\[
		\int\limits_D f(x) \d x = \int\limits_G f(\phi(t)) |J_\phi(t)| \d t
	\]
\end{theorem}
\begin{proof}
	$P \subset G$.
	\[ \int\limits_{\phi(P)} f(x) \d x = \int\limits_P f(\phi(t)) |J_\phi(t)| \d t \]
	Будем доказывать для разных $P$.
	\begin{description}
	\item[$P$ "--- куб со стороной $a$:]
		Оба интеграла существуют.
		$\tau_k$ "--- разбиение куба $P$ на кубики со стороной $\frac{a}k$.
		Оснащение $\xi_k$ возьмём точки, минимальные по каждой координате.
		$g(t) = f(\phi(t)) |J_\phi(t)|$.
		\[
			S(g, \tau_k, \xi_k) \xlongrightarrow{k \ra \infty} \int\limits_P g(t) \d t \\
		\]
		$\tilde \tau_k$ "--- разбиение $\phi(P)$  на $\phi(P_i)$, на оснащение тоже подействуем.
		Надо бы понять, что $|\tilde \tau_k| \ra 0$.
		Для этого хотим, что для любого $\epsilon > 0$ с какого-то $k$ мелкость $\tau_k$ меньше $\epsilon$.
		Всопмним, что $\phi$ равномерно непрерывна:
		\[
			\forall \epsilon > 0, \exists \delta > 0\colon \forall t,s, \|t-s\| \le \delta \Ra \|\phi(t) - \phi(s)\| \le \epsilon
		\]
		Значит, можно выбрать такое $k$, что $P_i$ будут настолько малы, что и $\phi(P_i)$ будут малы.

		Осталось сравнить интегралы через суммы.
		\begin{gather*}
			\sigma(g, \tau_k, \xi_k) - \sigma(f, \tilde \tau_k, \tilde \xi_k)
			= \sum g(\xi_{ki}) \mu P_{ki} - \sum f(\tilde\xi_{ki}) \mu \phi(P_{ki}) = \\
			= \sum \biggl( f(\phi(\xi_{ki})) |J_\phi (\xi_{ki})| \mu P_{ki} - f(\phi(\xi_{ki})) \mu \phi(P_{ki}) \biggr)
			= \sum f(\phi(\xi_{ki})) \mu P_{ki} \left( |J_\phi (\xi_{ki})| - \frac{\mu \phi(P_{ki})}{\mu P_{ki}} \right) \\
			\left| \sigma(g, \tau_k, \xi_k) - \sigma(f, \tilde \tau_k, \tilde \xi_k) \right|
			\le \sum\left|f(\phi(\xi_{ki}))\right|\frac{a^n}{k^n} \left||J_\phi (\xi_{ki})| - \frac{\mu\phi(P_{ki})}{\mu P_{ki}} \right| \le \\ 
			\le M \sum \frac{a^n}{k^n} \left| |J_\phi (\xi_{ki})| - \frac{\mu \phi(P_{ki})}{\mu P_{ki}} \right|
			\le M a^n \sup_{t^*} \left| |J_\phi (t^*)| - \frac{\mu \phi\left(Q_{\frac{a}k}\right)}{\mu Q_{\frac{a}k}} \right| \ra 0
		\end{gather*}

	\item[$P$ "--- конечное объединение кубов:]
		$P = \bigsqcup_{i=1}^m P_i$ "--- дизъюнктное объединение кубиков.
		Разложили на слагаемые и победа.

	\item[$P$ "--- измеримое и ограниченное:]
		Подгоняем множество кубами: всё множество лежит в кубе, режем его по серединам рёбер, и так далее.
		На каждом шаге есть объединение кубов $G_n$, целиком лежащих внутри $P$.
		Покажем, что оно стремится к $P$.
		Возьмём границу $P$ и приблизим её множеством параллелепипедов (клеточным множеством) меры меньше $\epsilon$.
		Возьмём любой параллелепипед, смотрим, сколько кубиков, его задевающих целиком.
		По мере измельчения объём задевающих стремится к объёму пареллелепипеда.
		Значит с какого-то момента кубики, покрывающие границу, будут иметь суммарную меру не более $2\epsilon$.
	\end{description}
\end{proof}

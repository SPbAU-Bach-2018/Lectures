\section{} % 37
Есть последовательность функций $f_n \colon E \to \R$.
Поточечно сх-ся к $f \colon E \to \R$, если в каждой точке $\lim_{n\to\infty} f_n(x) = f(x)$.
Равномерно сходится ($f \rra f$), если $\forall \epsilon > 0$ с некоторого $N$
в любом $x$ ошибка $|f_n(x)-f(x)|<\epsilon$.
Разница: в первом случае $\forall x$ внутри, во втором снаружи.
Из равномерной следует поточечная; если есть обе, то к одному и тому же $f$.
Пример: $f_n(x)=x^n$ сх-ся к 0 поточечно, но не равномерно.
Определение эквивалентно тому, что супремум ошибки стремится к нулю при $n\to\infty$.
Следствие: если для всех $x$ верно $|f_n(x)-f(x)|\le a_n$ и $a_n \to 0$, то равномерно сх-ся.
Следствие: если есть последовательность $x_n$ такая, что $f(x_n) - f(x) \nra 0$, то не сх-ся равномерно
(так как сколь угодно далеко есть большой супремум ошибки).

\section{} % 38
$f_n$ равномерно ограниченые константой $M$, если $|f_n(x)| \le M$ при всех $n$, $x$.
Если это так и $g_n \rra 0$, то $f_ng_n \rra 0$ (через супремум).
Критерий Коши: $f_n \rra f$ на $E$ $\iff$ для любого $\epsilon > 0$ для любых достаточно больших $n, m$
ошибка $|f_n(x) - f_m(x)|$ равномерно ограничена (типа равномерная фундаментальность).
$\Ra$ понятно, $\La$: сначала взяли поточечный предел $f(x)$ (он есть), потом в условии Коши устремили $m \to \infty$
(формальнее: взяли $m$ такое большое, чтобы разница $f(x)-f_m(x)$ была меньше $\frac{\epsilon}{2}$).

\section{} % 39
Пусть $f_n$ непрерывны в т. $a$ и $f_n \rra f$, тогда $f$ непрерывна в $a$.
Док-во: нам дали $\epsilon$, взяли большое $n$ (чтобы супремум отклонения мелкий),
расписали разницы между $f(x)$, $f_n(x)$, $f_n(a)$, $f(a)$.
Всё можно сломать примером $f_n(x)=x^n$, они непрерывны на $[0,1]$, а предел разрывен в $1$.
Следствие: пусть $f_n \rra f$ на $E$, $a$ "--- предельная точка $E$ и в ней у каждой $f_n$ есть предел.
Тогда можно менять пределы $x\to a$, $n \to \infty$ местами.
Вводим $g_n$ на $E \cup \{a\}$ (как $f_n$, только в точке $a$ берётся предел), каждая $g_n$ непрерывна в $a$,
они равномерно сходятся (почему? \TODO), тогда их равномерный предел $g$ непрерывен в $a$,
тогда оба порядка пределов свернутся до $f(a)$.

\section{} % 40
Пусть $K$ "--- компакт, $C(K)$ "--- пр-во непрерывных функций на $K$ с нормой
$\|f\|_{C(K)}=\max |f(x)|$ (для этого и компакт).
Это норма, проверить: тривиальность нуля, $\|af\|=a\|f\|$ (линейность), $\|f+g|\le\|f\|+\|g\|$ (треугольник).
Теорема: оно еще и полно (любая фундаментальная последовательность имеет предел, т.е. если отличия где угодно в хвосте мелкие).
Можно найти поточечный предел (так как $\|f-g\| \ge |f(x)-g(x)|$ и $\R$ полно), осталось показать $\|f_n-f\| \to 0$, что понятно.

\section{} % 41
Ряд $\sum u_n(x)$ сходится поточечно/равномерно, если частичные суммы $S_n(x)$ сх-ся так же (они уже функции).
Если сходится поточечно, то хвост (остаток) "--- набор функций $r_n(x)=\sum_n^\infty u_k(x)$.
Сходится равномерно тогда и только тогда, когда его хвост равномерно стремится к нулю.
Если ряд сходится равномерно, то члены равномерно стремятся к нулю (так как член "--- разность хвостов).

\section{} % 42
Критерий Коши: $u_n(x)$ равномерно сх-ся $\iff$ для любого $\epsilon > 0$ при достаточно больших $n, m$
в любой точке сумма $\sum_n^m u_k(x)$ не больше $\epsilon$ (из Коши для функций очевидно).
Следствие: если есть последовательность $x_n$ и $u_n(x_n) \nra 0$, то равномерной сходимости нет (из супремума члены равномерно стремиться к нулю не будут, упс).
А вот из расходимости $u_n(x_n)$ ничего не следует (вообще бред так суммировать): пусть $u_n = \frac 1 n$ при $x \in (\frac{1}{n-1}, \frac1n]$ и ноль иначе,
смотрим $x_n=\frac 1 n$, сумма разошлась, но равномерно сх-ся (так как хвост равномерно не больше $\frac 1 n$).

Признак сравнения: если везде в $E$ имеем $|u_n(x)| \le v_n(x)$, то при равномерном схождении $v_n(x)$ равномерно сходится и $u_n(x)$.
Док-во по Коши: модуль куска суммы $u_n$ не больше суммы модулей, не больше $v_n(x)$, что равно модулю суммы $v_n(x)$, а оно достаточно маленькое
из равномерной сходимости.
Признак Вейерштрасса: если ограничили $|u_n(x)| \le c_n$ и $c_n$ сх-то, то $u_n(x)$ равномерно сходится.
Следствие: если $|u_n(x)|$ сходится равномерно, то без модулей тоже равномерно ($v_n(x)=|u_n(x)|$).

\section{} % 43
Дирихле: если частичные суммы $a_n(x)$ равномерно ограничены константой $M$, для каждого $x$ ф-ция $b_n(x)$ монотонна и $b_n \rra 0$, то ряд $a_n(x)b_n(x)$ равномерно сходится.
Преобразуем по Абелю в каждой точке, первое слагаемое равномерно стремится к нулю, второе ограничим $v_n(x)=M|b_k(x)-b_{k+1}(x)|$ и покажем равномерную сходимость $v_n(x)$,
частичная сумма $v_n(x)$ в точке равна $M|b_1(x)-b_{n+1}(x)|$, так как $b_n \rra 0$ успех.
Абеля: ряд $a_n(x)$ равномерно сходится, для каждого $x$ ф-ция $b_n(x)$ монотонна и равномерно ограничена, $\Ra$ ряд $\sum a_n(x)b_n(x)$ равномерно сходится.
Док-во (аналогично числовому): найдём у $b_n(x)$ в каждой точке предел $b(x)$ (возможно, неравномерный), перепишем члены ряда: $a_n(x)b_n(x) = a_n(x)(b_n(x)-b(x)) + a_n(x)b(x)$,
первое слагаемое по Дирихле равномерно сходится (есть равномерный предел $\Ra$ равномерно ограничено), а частичные суммы второго есть частичные суммы $a_n(x)$ на $b(x)$ (которое равномерно ограничено),
т.е. они тоже равномерно ограничены.
Лейбниц: $b_n(x) \ge 0$, в каждой точке убывают, и $b_n(x) \rra 0$, тогда ряд $(-1)^nb_n(x)$ равномерно сходится.
Берём Дирихле и $a_n(x)=(-1)^n$.

\section{} % 44
Пусть $u_n(x)$ непрерывны в точке $a$, ряд равномерно сходится, тогда сумма ряда $S(x)$ непрерывна в $a$.
Применяем теорему про функции, т.к. частичные суммы непрерывны (как суммы конечного числа непрерывных).
Пусть функция $f_n \in C[a, b]$, $c \in [a,b]$ и $f_n \rra f$.
Тогда можно ввести функциональный ряд $F_n(x)=\int_c^x f_n(t) \d t$, тогда он сходится к $F(x)=\int_c^x f(t)\d t$ (можно менять предел и интеграл).
Корректно, так как $f_n$ и $f$ непрерывны.
Док-во: оценили $|F_n(x)-F(x)|$ как интеграл разности, его оценили как длину $(b-a)$ на супремум ошибки $f_n$, сошлось.
Важна равномерная непрерывность: пусть $f_n(x)=nxe^{-nx^2}$, везде пределы равны нулю (но неравномерно).
Предел интегралов на $[0,1]$ равен $\frac 1 2$, а вот интеграл пределов равен нулю.
Следствие: пусть $u_n \in C[a, b]$, $c \in [a,b]$, ряд равномерно сходится, тогда $\int_c^x \sum_1^\infty u_n(t) \d t = \sum_1^\infty \int_c^x u_n(t) \d t$ (можно менять сумму и интеграл).
Взяли сумму интегралов, записали как предел $\int_c^x f_n$ (интеграл аддитивен), поменяли предел и интеграл, получили интеграл суммы ряда, успех.

\section{} % 45
\TODO перепроверить условия теоремы и следствия на консультации

Пусть $f_n$ имеет непрерывную производную на $[a,b]$, $c\in[a,b]$, их производные равномерно стремятся к $g(x)$, предел $f_n(с)$ (в одной точке) есть и конечен.
Тогда $f_n \rra h$ на $[a,b]$ и $h'=g$.
Док-во: $g$ непрерывна (как равномерный предел непрерывных $f'_n$), взяли $\int_c^x g(t)$, переписали, он равен $\lim f_n(x) - \lim f_n(c)$, отсюда есть предел $f_n(x)$ в любой точке.
Также из теоремы про интегрирование знаем, что последовательность интегралов $f_n'(t)$ равномерно сходится к $\int f(t)$, отсюда $f_n(x)$ равномерно сх-ся,
пусть к $h(x)$.
Тогда считаем $h(x)-h(c)$, получаем интеграл от $g$, он дифференцируем (так как $g$ непрерывна), т.е. $h(x)$ дифференцируема.

Следствие: если $u_n \in C^1[a, b]$, $c\in[a,b]$, ряд $\sum u_k'(x)$ равномерно сх-ся, ряд $\sum u_k(c)$ сходится $\Ra$ $u_k(x)$ равномерно сходится к дифференцируемой функции
и её производная есть $\sum_1^\infty u_n'(x)$.
Док-во: ввели $f_n(x)$ как частичную сумму ряда, применили теорему.
Необходимость равномерной сходимости ряда производных (равномерной сходимости ряда не хватит): берём ряд $\sum_{n=1}^\infty \frac{\sin nx}{n^2}$, он равномерно сходится по Вейерштрассу,
а ряд из производных разойдётся при $x=2\pi k$.

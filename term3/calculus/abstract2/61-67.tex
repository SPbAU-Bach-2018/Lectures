\section{} % 61
Внешнаяя $k$-форма: $\omega \colon \underbrace{\R^n \times \dots \R^n}_{k} \to \R$, полилинейна, если поменять два аргумента
местами, то сменит знак $\Ra$ если два аргумента равны, то ноль.
0-форма: константа, 1-форма: линейное отображение (так как двух разных аргументов нет), 2-форма: билинейная кососимметрическая форма.
\TODO кусок текста наверху страницы 64
У открытого $G \subseteq $ дифференциальная $k$-форма: сопоставляет каждой точке $G$ какую-то внешнюю $k$-форму ($k$ одно на дифф.форму, аргументы внешней формы "--- из $\R^n$, того же, что и $G$),
никакой связи между ними не требуем.
По-программистски: $F(x)(\xi_1, \dots, \xi_k)$, по-математически: $F(x; x_1, \dots, x_k) \in \R$.
$\Omega(G)$ "--- мн-во дифференциальных форм в $G$.
Пример: 0-дифф. форма "--- функция от точки.

Разложение 1-дифф. формы на базис: мы в каждой точке строим линейное отображение $\R^n \to \R$,
т.о. пр-во 1-дифф. форм имеет размерность $n$ (задали линейное на базисных $\R^n$, дальше однозначно).
Рассмотрим $f_i(x)=x_i$, посмотрим на её $\d_x f_i$ (оно равно $f_i$ независимо от $x)$, обозначим
это линейное отображение за $\d x_i$.
Все $\d x_i$ линейно независимы, их $n$, они базис, теперь умеем раскладывать 1-дифф. формы в точке по базису.

\section{} % 62
Внешнее произведение двух внешних 1-форм $\omega_1 \wedge \omega_2$ "--- внешняя 2-форма такая, что
$\omega_1 \wedge \omega_2 = -\omega_2 \wedge \omega_1$ (при всех параметрах) и если есть две точки $\xi_1$, $\xi_2$
такие, что $\omega_i(\xi_j) = \delta_{ij}$ (верно ли, что $i = j$), то их произведение в $(\xi_1, \xi_2)$ есть единица.
Еще в определение пихаем, что эта операция линейна по $\omega_1$ и $\omega_2$ (можно вывести, но пихаем, так легче).
Теорема: $(\omega_1 \wedge \omega_2)(x_1, x_2) = \det \begin{pmatrix} \omega_1(x_1) & \omega_1(x_2) \\ \omega_2(x_1) & \omega_2(x_2) \end{pmatrix}$.
Док-во: если одна из форм ноль, то с двух сторон нули (столбец определителя занулился, внешнее произведение ноль по линейности).
Если $x_i$ линейно зависимы, то строки определителя линейно зависимы, внешнее произведение ноль потому что аргументы 2-формы линейно зависимы.
Иначе фиксируем $x_i$, строим 1-формы $\beta_i$ (в $x_i$ единица, в $x_{3-i}$ ноль) методом <<взять перпендикуляр к $x_i$ и проецировать аргумент на него>>,
теперь в двух точках $\omega_i=a_{i1}\beta_1+a_{i2}\beta_2$, выписали формулу для $\xi_1, \xi_2$, она от них зависит только как надо, успех.

Для внешних 2-форм: размерность их пр-ва равна $\frac{n(n-1)}{2}$ (так как посмотрели на значения в $(v_i, v_j)$, написали матрицу, не забыли про кососимметричность).
Тогда берём $\d x_i \wedge \d x_j$ как базис.
Линейная независимость, потому что подставили $(e_i, e_j)$ (и по формуле посчитали значение) "--- $\pm 1$ только один.

\section{} % 63
Есть гладкое отображение $\phi \colon U \to V$, на $V$ есть дифф. $n$-форма $\omega$, тогда перенос формы $\phi^*$ (из $V$ в $U$) есть:
$\phi^*(\omega)(t;\eta_1, \dots, \eta_k)=\omega(\phi(t);(\d_t\phi)(\eta_i))$ (а что другое мы можем написать? нам как вторые аргументы нужно что-то линейное от исходных вторых).
Свойства: линейно по форме, есть композиция $\phi^* \circ \psi^* = (\psi \circ \phi)^*$ (дифференциал композиции есть композиция дифференциалов),
если $f \colon V \to R$ (домножили исходную форму на разные константы в точках), то $\phi^*(f \cdot \omega) = (f \circ \phi) \cdot \phi^*(\omega)$
(после переноса константа вылезла из под $\phi^*$), честно и очень аккуратно расписать.
Свойство: пусть $\omega_1$, $\omega_2$ "--- 1-формы, тогда определено $\omega_1 \wedge \omega_2$ и перенос дистрибутивен относительно $\wedge$ (в каком-нибудь
естественном понимании), честно расписываем через определитель.

Пример: есть форма $P\d x + Q\d y$, возьмём отображение $\phi(u, v) = (x(u, v), y(u, v))$ из $U$ в $V$, тогда можно перенести форму,
получим $\phi^*(P\d x + Q\d y) = P(x_u' \d u + x_v' \d v) + Q(y_u' \d u + y_v' \d v)$, как бы выразили дифференциалы $\d x$ и $\d y$
через $\d u$ и $\d v$.

\section{} % 64
Поверхностный интеграл II рода (интеграл от формы): берём 2-форму $\omega$, простую поверхность $S$, тогда если
$S \subset \R^2$, то 2-форма имеет вид $f\cdot \d x \wedge \d y$, тогда определяем просто как кратный интеграл этой функции.
Если $S \subset \R^3$, то переносим форму в $G$ и считаем кратный интеграл там, возможно, надо добавить знак, если у нас
направление нормы в $S$ не совпало с тем, что хотим иметь в $G$.
Корректность: \TODO
Интеграл по кусочно-гладкой: \TODO

\section{} % 65
$\vec F$ "--- векторное поле, если в каждой точке $\R^n$ есть вектор из $\R^n$.
Форма поля "--- $\omega = P\d y \wedge \d z + Q\d z \wedge \d x + R\d x \wedge \d y$.
Поток поля через поверхность $S$ "--- интеграл этой формы по $S$.
Теорема: если поле задано формой, то $\int_S \omega = \int_S \left<\vec F, \vec n\right> \d S$ (что форму интегрировать, что скалярное произведение нормали с потоком;
слева интеграл II рода, справа "--- первого).
Док-во: сначала научимся перетаскивать 2-форму из $\R^3$ в $\R^2$ (сначала перетащили $\d x$, выразили его через $\d u$, $\d v$,
потом перетащили $\d x \wedge \d y$ по дистрибутивности умножения и перетаскивания).
Теперь взяли $\left<\vec F, \vec n\right>$, расписали норму по формуле, вынесли константу, скаларное произведение с векторным записали как определитель
(очень похож на просто векторное), интеграл в кратный по определению I рода (появился множитель и сразу убился об константу), разложили
определитель по первой строке, получили в точности оределение интеграла II рода в $\R^3$ (коэффициенты при $P$, $Q$, $S$ "--- результат перетаскивания формы).

\section{} % 66
\TODO

\section{} % 67
\TODO

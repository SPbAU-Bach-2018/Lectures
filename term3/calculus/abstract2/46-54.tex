\section{} % 46
Есть последовательность $a_n \in \C$, есть $z, z_0 \in \C$ ($z$ "--- типа переменная)
Степенной ряд "--- $\sum_{n=0}^\infty a_n(z-z_0)^n$, далее считаем $z_0=0$ и символ $z_0$ больше не используем.
Первая теорема Абеля: если ряд сходится при $z=\alpha$, то абсолютно сойдётся при всех $|z|<|\alpha|$.
Док-во: $a_n\alpha^n$ сходится $\Ra$ члены ряда ограничены $M$ по модулю.
Тогда оценим $|a_nz^n|$ как $M\cdot|\frac{z}{\alpha}|^n$, такой ряд сойдётся (как геом.прогрессия), т.е.
$a_nz^n$ сойдётся абсолютно.
Следствие: если ряд расходится при $z=\alpha$, то расходится и при $|z|>|\alpha|$ (так как иначе применим теорему).
Важно строгое равенство: ряд $\sum \frac{z^n}{n}$ разойдётся при $z=-1$ и сойдётся при $z=1$.
Радиус сходимости "--- такое $R \in [0, +\infty]$, что при $|z| < R$ ряд сходится и при $|z| > R$ расходится
(а при $|z|=R$ "--- неизвестно).
Круг сходимости "--- открытый круг вокруг нуля (или где там степенной ряд раскладываем) радиуса $R$.

\section{} % 47
Теорема: радиус сходимости всегда есть.
Док-во: возьмём мн-во $A$ таких вещественных $x \ge 0$, что ряд сходится (там точно есть ноль), положим
$R=\sup A \in [0, +\infty]$.
Тогда при меньших сходится по теореме, а при больших можно найти кого-то между супремумом и $|z|$ на вещественной, где разойдётся, ура.
Продолжение теоремы: в круге $|z| \le r < R$ ряд равномерно абсолютно сходится (как ряд от $z$).
Док-во: пусть $r<R$, тогда есть $x$: $r<x<R$, тогда элементы $a_nx^n$ ограничены.
Оценили ряды внутри $r$ геометрической прогрессией $M\cdot (\frac{r}{x})^n$.
Примеры: у $z^n$ радиус 1; у $\frac{z^n}{n!}$ радиус $+\infty$ по Даламберу, у $n!z^n$ радиус 0 по Даламберу.

\section{} % 48
Вторая теорема Абеля: пусть $a_nz^n$ сходится при $z=R\ge0$, тогда сходится равномерно на $[0,R]$ (а не просто на $[0, r<R]$).
Док-во: взяли $0 \le x \le R$.
Ряд $a_nR^n$ равномерно сходится, $(\frac x R)^n$ равномерно ограничено единицей и монотонно, по признаку Абеля получили что надо.
Следствие: если $a_nz^n$ сходится при $z=R \ge 0$, то $f(x) = \sum a_nx^n$ непрерывна на $[0, R]$ и, в частности
(непрерывность в $R$): предел $f$ слева в точке $R$ равен $\sum a_nR^n$.
Док-во: частичные суммы непрерывны, ряд сходится равномерно, тогда предел непрерывен.

\section{} % 49
Формула Коши-Адамара: пусть $\rho=\varlimsup\sqrt[n]{|a_n|}$ (верхний предел aka точная верхняя грань мн-ва пределов подпоследовательностей), тогда радиус сходимости есть $\frac1\rho$.
Если $\rho=0$, то получаем, что просто предел тоже равен нулю (так как корень неотрицателен).
Взяли $z$, тогда с некоторого места $\sqrt[n]{|a_n|} < \frac{1}{2|z|}$, возвлеи в степень, перенесли $|z|^n$, ряд оценился геом. прогрессией и везде сошёлся, что и надо.
Если $\rho=+\infty$, то покажем, что при $z \neq 0$ разойдётся: взяли последовательность $n_k$, при которой $\sqrt[n]{|a_n|} \to +\infty$, тогда для любого
$z$ найдётся такое $k$, что начиная с него $\sqrt[n]{a_n} > 1/z$, а тогда члены ряда $a_nz^n$ не стремятся к нулю и ряд разойдётся.
Пусть $0<\rho < \infty$.
Для схождения: берём $|z|<\frac{1}{\rho+\epsilon}$, отделяем $\sqrt[n]{|a_n|}<p+\epsilon$, ряд с некоторого места мажорируется $(\rho + \epsilon)|z|<1$ (так как верхний предел).
Для расхождения: отделяем $|z|>\frac{1}{\rho-\epsilon$, находим последовательность $n_k$ с пределом $\sqrt[n]{|a_n|}$ равным $\rho$, нашли член ряда $>1$ сколь угодно далеко, не сх-ся.

\section{} % 50
$f$ аналитическая в точке $x_0$, если в некоторой окрестности $f(x)=\sum_{n=0}^\infty a_n(x-x_0)^n$ (раскладывается в степенной ряд,
коэффициенты зависят только от $x_0$).
Линейная комбинация и произведение аналитических "--- аналитические.
Лемма: у рядов $a_nx^n$, $na_nx^{n-1}$ и $a_nx^{n+1}/(n+1)$ один и тот же радиус сходимости.
По формуле Коши-Адамара получаем, что верхний предел у рядов один и тот же, успех.
Теорема: пусть $f$ аналитична в вещественной $x_0$, $R$ "--- радиус сходимости, тогда $f$ бесконечно дифференцируема на $(x_0-R, x_0+R)$
(в понятно какие ряды), и если $x$ оттуда, то $\int_{x_0}^x f(t) \d t$ есть понятно какой степенной ряд, радиусы сходимости рядов равны $R$.
\TODO уточнить формулировку, <<радиус сходимости>> ряда или всё-таки той области, где функция совпадает с рядом?

\section{} % 51
\TODO

\section{} % 52
Определяем как степенные ряды (по Тейлору): $e^z = \sum_{n=0}^\infty \frac{z^n}{n!}$,
$\sin z = \sum_{n=0}^\infty \frac{(-1)^nz^{2n+1}}{(2n+1)!}$, сходятся везде (по Даламберу).
Свойства: $e^{iz}=\cos z + i \sin z$, формула Эйлера сразу отсюда.
$\cos$ и $\sin$ через полусумму/полуразность $e^{iz}$ и $e^{-iz}$.
$e^{a+b}=e^ae^b$ (расписали, вспомнили бином Ньютона), отсюда вся тригонометрия.

\section{} % 53
\TODO

\section{} % 54
\TODO

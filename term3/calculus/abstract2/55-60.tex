\section{} % 55
Пусть $G \subset \R^2$ "--- измеримая область (открытое, линейно связное, ограниченное).
Есть непрерывная параметризация поверхности $S$: $\vec r \colon \cl G \to \R^3$.
Гладкая поверхность "--- если $\vec r$ непрерывно дифф. и нет особых точек (в которых
$\vec r_u'$ и $\vec r_v'$ коллинеарны, т.е. нельзя плоскость проверсти).
Простая гладкая "--- если биекция.
$\delta S = \vec r(\delta G)$ (для простой).
Кусочно-гладкая "--- объединяем простые, пересекаются только по краям.
Край кусочной-гладкой "--- края, по которым не сшили.
Касательная плоскость "--- натянули на $\vec r_u'$ и $\vec r_v'$, нормаль "--- вектор ей ортогональный
($\pm$ векторное на модуль векторного).
Другая параметризация допустима, если перевести $\cl \Sigma \to \cl G$ непрерывно дифференцируемой
с ненулевым якобианом.
Касательная плоскость не меняется при переходе к допустимой параметризации: расписали производные от сложных
функций перехода из допустимой в $\R^3$, перемножили векторно, раскрыли скобки, $v \times v = 0$,
осталась старая нормаль на ненулевой якобиан.

\section{} % 56
Гладкая поверхность ориентируема, если в каждой точке есть нормаль, непрерывная.
Если ориентируема, есть две ориентации.
Простая гладкая всегда ориентируема по формуле.
Лента Мёбиуса (кусочно-гладкая) "--- нет.
Согласованная ориентация "--- по правилу правой руки идём вдоль границы, смотрим влево, норма вверх (обошли границу в $\R^2$ против часовой, а потом сказали, что нормаль ровно такая).
Ориентация кусочно-гладкой: ориентировали каждый кусок, на стыках ненулевой длины должны идти в разные стороны (пересечения по точкам здесь и далее запретим).

\section{} % 57
Есть простая гладкая $S$, пусть разбили $G=\cup_1^n G_i$ (пересекаются по мн-вам нулевой меры), выбрали оснащение $\xi_i$.
Вместо $\vec r$ в каждом куске приблизили его касательной плоскостью через $\vec r(\xi_i)$, площадь чешуйки умеем считать
как $\mu S_i = \|r_u' \times r_v'\| \mu G_i$ (дано свыше, в якобиан раз растянулась).
Тогда $\sigma$ "--- площадь $S$, если сумма Римана: любое оснащённое разбиение выдаёт чешуйки, которые в сумме сколь угодно хорошо приближают $\sigma$.
Теорема: $\sigma = \int_G \|r_u' \times r_v'\| \d u \d v$, обозначаем под интегралом за $g(u, v)$, пишем суммы Римана для обычного интеграла $\int_G$,
$g\cdot \sigma G_i = \sigma S_i$ (было выше), получили что надо.

\section{} % 58
Первая квадратичная форма "--- штука, как-то (хз как) характеризующая форму поверхности в конкретной точке.
Берём вектор из касательной плоскости, раскладываем по базису (частные производные $\vec r$) на $\alpha \beta$,
берём квадрат его длины, раскрыли всё на свете, получили квадратичную форму $I(\alpha, \beta)=\begin{pmatrix}\alpha & \beta\end{pmatrix} \begin{pmatrix} E & F \\ F & G\end{pmatrix} \begin{pmatrix} \alpha \\ \beta \end{pmatrix}$.
Она положительна определена (по построению) $\Ra$ $EG-F^2 > 0$ (определитель; привет, критерий Сильвестра).
Также можно в лоб проверить $\|r_u' \times r_v'\| = EG-F^2$ (отсюда и вся прелесть первой квадратичной).
Частный случай: $S=(x, y, z(x, y))$ ($S$ "--- график функции $z$), тогда посчитали $EG-F^2$.
Теперь умеем считать площадь понятным интегралом (через $EG-F^2=\|r_u' \times r_v'\|$), частный случай для графика.

\section{} % 59
Есть простая гладкая $S$ и непрерывная $f \colon S \to \R$, тогда определяем интеграл I рода как
$\int_S f \d S = \int_G f(\vec r(u, v))\|r_u'\times r_v'\| \d u \d v$.
Если взять допустимую параметризацию, то не поменяется: взяли две параметризации (одна через другую),
честно расписали по определению (да, надо считать сложные частные производные $R_\xi'$, уже было где-то),
получили интегралы, отличающиеся заменой переменных, они равны.

\section{} % 60
Для кусочно-гладкой поверхности: сумма интегралов по всем простым кускам (разбиение на них неважно, см. дальше про нарезку кусков, а отсюда уже очевидно).
Свойства: не зависит от ориентации (её нет в определении), линеен по функции (т.к. кратный линеен),
аддитивен по поверхности (если было $G$, то можно распилить его на несколько кусков), от неотрицательной неотрицателен,
не зависит от нарезки.

Сферические координаты: пусть $u \in [0, 2\pi]$ (долгота), $v \in [-\sfrac\pi 2, +\sfrac\pi 2]$ (широта),
тогда $x=R \cos u \cos v$, $y = R\sin u \cos v$, $z = R \sin v$.
Сфера простой параметризацией так не задастся, надо сначала распилить на восток/запад, а потом играть с полюсами (там тоже биективность нарушится):
устремляем размер вырезанного полюса к нулю, в случае хорошей функции мы потеряем тоже что-то стремящееся к нулю.
Дальше считаем $EG-F^2$, ($E=R^2\cos^2 v$, $F=0$, $G=R^2$, берём корень, можно положить $f=1$ и посчитать площадь, будет $4\pi R^2$.

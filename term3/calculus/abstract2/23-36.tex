\section{} % 23
Ряд "--- конструкция $\sum_{n=1}^\infty a_n$, начинается с любого места.
Частичная сумма $S_k$ "--- $\sum_{n=1}^k$.
Если есть предел $\lim S_n \in \bar \R = X$ (может быть $\pm\infty$), то есть сумма ряда: $\sum a_n = X$.
Сходится, если сумма конечна, расходится, когда бесконечна или не существует (например, $1, -1, 1, -1, \dots$).
Геометрическая прогрессия: $\sum_{n=0}^\infty q_n$.
Телескопическое: $\sum \frac{1}{n(n+1)} = \sum \left(\frac 1 n - \frac 1 {n+1}\right)$.
Гармонический: $\sum \frac 1 n$, $S_n = H_n = \ln n + \gamma + o(1)$, т.е. расходится.
Отбрасывание начала ряда не меняет сходимость.
Хвост ряда $\sigma_m$: сумма с элемента $m$.
Если сходятися, то сумма хвоста $\to 0$.
Если ряды сходятся, то сходится и линейная комбинация (из св-в пределов).
Если сходятся и $a_n \le b_n$, то $\sum a_n \le \sum b_n$.
Сходимость в $\C$: есть предел частичных сумм в $\C$ (бесконечностей уже нет).
Равносильна сходимости вещественной и мнимой частей.

\section{} % 24
Необходимое условие: если сходится, то $a_n \to 0$.
Критерий Коши (аналогия с пределами): сходится $\iff$ для любого $\epsilon > 0$ с некоторого места все суммы кусков ряда маленькие
(условие на фундаментальность $S_n$)

\section{} % 25
Группировка членов: группируем конечные куски ряда в члены нового ряда.
Если ряд имел сумму в $\bar \R$/$\C$, то группировка имеет ту же сумму (обратное неверно: $(1-1)+(1-1)+\dots$).
А вот если в каждой группе не более $M$ членов и $a_n \to 0$, то сходится группа $\Ra$ сходится исходный
Или есть в каждой группе все члены одного знака, то сходится группа $\Ra$ сходится исходный.

\section{} % 26
Если $a_n \ge 0$, то $\sum a_n$ сх-ся $\iff$ частичные суммы ограничены (т.к. $S_n$ монотонна).
Признак сравнения: если $0 \le a_n \le b_n$, то $b_n$ сх-ся $\Ra$ $a_n$ сх-ся; $a_n$ расх-ся $\Ra$ $b_n$ расх-ся.
Следствие: если $a_n \sim b_n$, то ведут себя одинаково (подпёрли ряды с двух сторон).

\section{} % 27
$a_n \ge 0$, Коши:
Если $\sqrt[n]{a_n} \le q < 1$, то сх-ся (зажато геом. прогрессией). 
Если $\sqrt[n]{q_n} \ge 1$, то расх-ся (так как не $\to 0$).
Если $\lim \sqrt[n]{a_n} = q^*$, то при $q^* < 1$ сходится, при $q^* > 1$ расходится, при $q^*=1$ хз (с некоторого места применили первый из двух пунктов).
Плохие примеры: $a_n=1$ и $a_n=\frac{1}{n(n+1)}$.

\section{} % 28
Даламбер: $a_n > 0$.
Если $\frac{a_{n+1}}{a_n} \le d < 1$, то сходится (телескопически перемножили, применили Коши), если $\ge 1$, то расходится (ряд неубывает),
аналогичный Коши предельный случай, примеры те же.
Еще пример: $e^x=\sum {x^n}{n!}$, отношение соседей считать проще корня.
Связь: если $a_n > 0$ и $\frac{a_{n+1}}{a_n} \to d^*$, то $\lim \sqrt[n]{a_n} = d^*$ (взяли логарифмы от двух частей,
применили к правой Штольца, aka дискретный Лопиталь: если существует, то предел дробей равен пределу дробей из разностей).

\section{} % 29
Пусть вещественная $f$ на целочисленном отрезке $[a,b]$ неотрицательна и монотонна, тогда разность суммы и интеграла не больше
максимума значений на концах.
Док-во: пусть убывает, тогда нарисовали прямоугольники и зажали интеграл суммами на $a+1\dots b$ и $a\dots b-1$,
вычли неравенство из суммы на $a\dots b$, зажали разность интеграла и суммы между $f(b)$ и $f(a)$.
Если нет неотрицательности, то надо брать максимум из $|f(a)|, |f(b)|$.
Снова неоттрицательность: пусть убывает и $c_b = \sum_a^{b_1} - \int_a^b$ (ошибка из-за кривизны функции), она ограничена сверху $f(a)$ и растёт,
есть предел $C$, тогда при увеличении $b$: $\sum_a^{b-1} = \int_a^b + C + o(1)$.
Например, в $\sum_1^n k^{-p}$ можно вынести последнее слагаемое и интеграл, получить оценку $\frac{n^{1-p}}{1-p} + C + o(1)$.

Интегральный признак: $f \ge 0$ и монотонно убывает, тогда сумма и интеграл ведут себя одинаково.
Взяли частичные суммы/интеграли, ошибка не больше $f(1)$.
Пример: ряд $\sum \frac{1}{n^p}$ при $p\le0$ не сходится (не стремится к нулю), а при $p>0$ убывает, про интеграл всё знаем (сходится только при $p>1$.
Следствие: если оценили ряд как $\frac{c}{n^{>1}}$, то сх-ся.
Пример: $f(x)=\frac{1}{x \ln x}$ с некоторого места убывает, ведёт себя как интеграл, а он берётся и расходится (аккуратно с границей).

\section{} % 30
$\sum a_n$ сх-ся абсолютно, если сх-ся $\sum |a_n|$.
Свойства (оцениваем модули): если абс. сх-ся, то лин. комб. абс. сходится.
Абс. сх-ся $\iff$ просто/абсолютно сходятся $(a_n)_+ = \sum \max(0, a_n)$ и $(a_n)_- = \sum \max(0, -a_n)$
В комплексных: абс. сх-ся $\iff$ абс. сх-ся вещественные и мнимые части.
Если абс. сх-ся, то и просто сходится и $|\sum a_n| \le \sum |a_n|$.
В вещественном: сходятся положительные и отрицательные $\Ra$ сх-ся их разность.
В комплексном: абс. сх-ся вещественные мнимые $\Ra$ сх-ся просто $\Ra$ успех.
Неравенство: переходим к пределу частичных суммы (всё уже существует).
Если сходится, но не абсолютно, то сх-ся \textit{условно}.

\section{} % 31
Преобразование Абеля "--- аналог интегрирования по частям, пусть $A_n = \sum_1^n a_k$, $A_0=0$,
тогда $\sum_{k=1}^n a_kb_k = A_nb_n + \sum_{k=1}^{n-1} A_k(b_k-b_{k+1})$ (выразили $a_k=A_k-A_{k-1}$).
Пример: $\sum_{k=1}^n 1\cdot H_k = (n+1)H_n-n$ (применили Абеля в лоб, потом вычли из слагаемых внутри единицу).
Признак Дирихле: если частичные суммы ограничены и $b_k$ монотонно $\to 0$, то $\sum a_nb_n$ сходится.
Расписали $S_n$ по Абелю, первое слагаемое занулилось (ограниченное $\cdot$ ноль), навесили модулей на слагаемые
внутри суммы, вынесли $M$, ограничилось $Mb_1$.

\section{} % 32
Пусть $p>0$, тогда $\sum_{n=1}^n \frac{\sin n}{n^p}$ сх-ся всегда (для $nx$ у нас не было и его выкинули).
По Абелю: посчитали сумму синусов (как геометрическую прогрессию ешки), получилось поделить,
если $e^{ix}\neq 1$ (а если равно, то синус ноль), ограничено.
Для косинусов аналогично.
Абсолютная сходимость: при $p>1$ очевидно, иначе не сходится (так как $|\sin|\ge \sin^2 = 1-\cos 2n$, первое
разошлось, второе сошлось по Дирихле, проверить отдельно).

Признак Абеля: $\sum_{k=1}^n a_n$ сходится, $b_k$ монотонна и ограничена, тогда $\sum a_nb_n$ сх-ся.
Док-во: есть предел $b$, тогда заменили $a_nb_n = a_n(b_n - b) + ba_n$, первое сошлось по Дирихле,
второе сошлось.

\section{} % 33
Признак Лейбница: $b_k \ge 0$ монотонны, тогда $\sum (-1)^{n-1} b_n$ сходится $\iff b_n \to 0$.
$\Ra$ очевидно, для $\La$ либо Дирихле, либо смотрим на цепочку вложенных отрезков $S_{2k}, S_{2k+1}]$, длины к нулю, границы сходятся,
еще имеем $S_{2n} \le S \le S_{2n+1}$.
Пример: $\frac{(-1)^{n-1}}{n^p}$.
Ряд Лейбница: $\frac{(-1)^{n-1}}{n} = \ln 2$ ($S_{2n}=H_{2n}-H_n$).

\section{} % 34
Пусть $\phi \colon \N \to \N$ "--- биекция, тогда перестановка $\sum a_n$ есть $\sum a_{\phi(n)}$.
Если абсолютно сходился, а сумма равна $S \in \bar \R$, то любая перестановка просто сойдётся к $S$.
В случае $a_n \ge 0$: 
Док-во: покажем $S' \le S$, берём частичную сумму $S'_n$, ограничиваем её сверху $S_m$, берём предел,
для обратной перестновки имеем $S \le S'$, успех.
Общий случай: раз $a_n$ абс. сх-ся, то $(a_n)_{\pm}$ сходятся, их можно переставлять, продолжают сходится к тому же,
потом вычитаем, успех.
Для комплексных разбиваем на вещественную и мнимую.

\section{} % 35
Замечание: если сходится лишь условно, то $(a_n)_+$ и $(a_n)_-$ расходятся (к $+\infty)$,
так как $(a_n)_+ + (a_n)_- = |a_n|$ расходится, а разность сходится, выразили.
Теорема Римана: пусть ряд вещественных условно сходится, тогда можно перестановкой получить любую сумму из $\R$,
а также можно перестановкой получить расходящийся.
Док-во: распилили ряд на положительные и отрицательные (оба результата стремятся к бесконечности).
Если желаемое $S \ge 0$ конечно, то набираем сначала положительных, пока не перевалим за $S$, потом
отрицательных, пока не перевалим обратно, и так далее.
Сгруппируем так, чтобы в группе был один знак, покажем сумму.
Для $S=+\infty$ на шаге $i$ переваливаем через $i$.
Расхождение: стремимся к $1, -1, 2, -2, 3, -3, \dots$.

\section{} % 36
Пусть $\sum a_n = A$ и $\sum b_n = B$, причём оба ряда сходятся еще и абсолютно.
Тогда ряд из всех возможных попарных произведений $a_ib_j$ (в любом порядке) абсолютно сходится,
а его сумма есть $AB$.
Покажем для одного порядка, а для остальных будет следовать из абс. сходимости.
На шаге $k$ берём $a_{\le k}b_{\le k}$, еще не взятые.
Пусть $S_n$ "--- сумма модулей после шага $k$, тогда она равна произведению частичных сумм модулей исходных, $S_n$ сходятся.
Без модулей тоже всё ок и куда надо.
Надо понять, что промежуточные частичные (которые по слоям квадратиков) несильно выбиваются, это верно т.к. члены рядов стремятся к нулю,
а сумму начала можно оценить как сумму ряда.

Произведение рядов: $c_n = \sum_{k=1}^n a_k b_{n-k+1}$ (сумма по диагонали).
Удобно для перемножения рядов Тейлора.
Теорема Мертенса (без док-ва): если $a_n$ и абс. сх-ся, а $b_n$ сх-ся, то $c_n$ сх-ся к $AB$, порядок менять нельзя, абсолютной сходимости не обещаем.
Необходимость абс. сходимости: $\frac{(-1)^{n-1}}{\sqrt n}$ условно сх-ся, но его квадрат "--- нет, т.к. расписали честно, получили $c_n=1$.

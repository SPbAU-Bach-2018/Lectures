\section{Евклидовы и унитарные пространства}

\begin{Def}
n-мерное Евклидово пространство:
$$\R^n$$
$$(x, y) = \sum_{i = 1}^{n}x_iy_i$$
                                

n-мерное унитарное пространство:
$$\C^{n}$$
$$(x, y) = \sum_{i = 1}^{n}x_i \overline{y_i}$$
                                 

В обоих случаях: $\Gamma = E$
\end{Def}


\begin{Def}

V "--- евклидово (унитарное) пространство.

 
Расстояние между двумя векторами:
$$d(x, y) = \sqrt{(x - y, x - y)}$$
$$|x| = \sqrt{(x, x)}$$
$$d(x, y) = |x - y|$$
\end{Def}

\begin{proof}

Докажем, что это метрика. 

$d(x, y) = d(y, x)$ в силу симметричности (эрмитовой)

$d(x, y) = 0 \Lra x = y$ в силу положительной определенности. 

$$d(x, y) \le d(x, z) + d(y, z)$$
$$|x - y| \le |x - z| + |y - z|$$            
$$u = x - z$$
$$v = z - y$$
$$|u + v| \le |u| + |v|$$
$$|u + v|^2 \le (|u| + |v|)^2$$
$$(u, u) + (u, v) + (v, u) + (v, v) \le (u, u) + 2|u||v| + (v, v)$$
$$(u, v) + (v, u) \le 2|u||v|$$
Сумма двух комплексно сопряжённых чисел, из-за этого мнимая часть убьётся. 
$$2Re(u, v) \le 2|u||v|$$
$$|(u, v)|^2 \le (u, u)(v, v)$$
$$|u|^2 |v^2| \ge (Re(u, v))^2 + (Im(u, v))^2 \ge (Re(u,v))^2$$



$$0 \le (u + \lambda v, u + \lambda v) = $$
$$= (u, u) + \overline{\lambda}(u, v) + \lambda(v, u) + |\lambda|^2(v, v)$$

$$\lambda = (u, v)t, t \in \R$$

$(u, v) = 0$ "--- очевидно. 
$(u, v \ne 0)$

$$0 \le (u, u) + 2|(u, v)|^2t + |(u, v)|^2t^2(v, v)$$
$$\lambda\overline{\lambda} = |(u, v)|^2$$
$$0 \ge \frac{1}{4}D = |(u,v)|^4 - (u, u)(v, v)|(u, v)|^2 \Ra$$
$$0 \ge |(u, v)|^2 - (u, u)(v, v)$$
\end{proof}

\begin{Def}
Угол между векторами. 

$$|(u, v)| \le |u||v|$$
$$-1 \le \frac{(u, v)}{|u||v|}$$
$$\exists \Phi \in \R  \colon \frac{(u,v)}{|u||v|} = \cos \Phi$$
$$0 \le \Phi \le \pi$$

$\Phi$ "--- угол между u и v.
\end{Def} 

\section{Евклидовы и унитарные пространства}

\begin{Def}
	$n$-мерное Евклидово пространство:
	\[ \R^n \quad (x, y) = \sum_{i=1}^n x_iy_i \]
	$n$-мерное унитарное пространство:
	\[ \C^n \quad (x, y) = \sum_{i=1}^n x_i \bar y_i \]
	В обоих случаях (если берём естественный базис)
	\[ \Gamma = E \]
\end{Def}

\begin{Def}
	$V$ "--- евклидово (унитарное) пространство.
	Расстояние между двумя векторами:
	\begin{gather*}
		d(x, y) = \sqrt{(x - y, x - y)} \\
		|x| = \sqrt{(x, x)} \\
		d(x, y) = |x - y|
	\end{gather*}
\end{Def}

Докажем, что это метрика.
\begin{proof}
	$d(x, y) = d(y, x)$ в силу симметричности (эрмитовой). \\
	$d(x, y) = 0 \Lra x = y$ в силу положительной определенности. \\
	Осталось показать:
	\begin{gather*}
		d(x, y) \le d(x, z) + d(y, z) \\
		|x - y| \le |x - z| + |y - z|
	\end{gather*}
	$u = x - z$, $v = z - y$.
	\begin{gather*}
		|u + v| \le |u| + |v| \\
		|u + v|^2 \le (|u| + |v|)^2 \\
		(u, u) + (u, v) + (v, u) + (v, v) \le (u, u) + 2|u||v| + (v, v) \\
		(u, v) + (v, u) \le 2|u||v|
	\end{gather*}
	Сумма двух комплексно сопряжённых чисел, из-за этого мнимая часть убьётся.
	\begin{gather*}
		2\Re(u, v) \le 2|u||v| \\
		|(u, v)|^2 \le (u, u)(v, v) \\
		|u|^2 |v^2| \ge (\Re(u, v))^2 + (\Im(u, v))^2 \ge (\Re(u,v))^2 \\
		0 \le (u + \lambda v, u + \lambda v) =  \\
		= (u, u) + \overline{\lambda}(u, v) + \lambda(v, u) + |\lambda|^2(v, v)
	\end{gather*}
	$\lambda = (u, v)t$, $t \in \R$.
	$(u, v) = 0$ "--- очевидно.
	Пусть $(u, v) \ne 0$:
	\[
		0 \le (u, u) + 2|(u, v)|^2t + |(u, v)|^2t^2(v, v)
	\]
	$\lambda\overline{\lambda} = |(u, v)|^2$
	\begin{gather*}
		0 \ge \frac{1}{4}D = |(u,v)|^4 - (u, u)(v, v)|(u, v)|^2 \Ra \\
		0 \ge |(u, v)|^2 - (u, u)(v, v)
	\end{gather*}
\end{proof}

\begin{Def}
	Угол между векторами:
	\begin{gather*}
		|(u, v)| \le |u||v| \\
		-1 \le \frac{(u, v)}{|u||v|} \le 1 \\
		\exists \Phi \in \R\colon \frac{(u,v)}{|u||v|} = \cos \Phi \\
		0 \le \Phi \le \pi
	\end{gather*}
	$\Phi$ "--- угол между $u$ и $v$.
\end{Def}

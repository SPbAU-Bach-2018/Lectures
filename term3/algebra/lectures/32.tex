\section{Центр p-группы}
\begin{Def}
$G$-группа\\
$Z(G) = \{a \in G \colon ag = ga \forall g \in G \}$\\
"--- центр группы.

В честности $Z(G) \unlhd G$\\
$(\forall a \in Z(G) \forall g \in G, gag^{-1} = a)$ \\
\end{Def}
\begin{Def}
p-простое число\\
Кононическая группа G порядок которой 
есть $p^{k}$ называется p-группой $(k \in \N)$\\
\end{Def}
\begin{exmp}
$\Z/{p^{k}\Z}$  "--- циклическая p-группа.
$\Z/(p^{k_1}\Z) \times \cdots \times \Z/(p^{k_s}\Z)$   абелева $p$-группа 
(но не циклическая, если $s \ge 2$)

ToDo
\end{exmp}
\begin{theorem}
$G$ "--- p-группа $\Ra |Z(G)| > 1$\\
\end{theorem}
\begin{proof}
Рассмотри действие $G$ на $G$ сопряжением.\\
$G \times G \to G$\\
$g, x \to g \times g^{-1}$\\

Орбита x состоит из 1 элемента $\Lra \forall g \: g x g^{-1} = x$\\
$\Lra \forall g, gx = xg \Lra x \in Z(G)$\\
$p^{k} = |G| = \sum_{A - \text{класс сопряженности}}|A| = \sum_{|A| = 1}|A| + \sum_{|A| > 1}|A| = \\
 = |Z(G)| + \sum_{|A| > 1}|A|$\\

Класс сопряженности = орбита в действие $G$ на $G$ сопряжением. 
$|A| | |G|$  \\
$|G| = p^{k}$\\
$|A| > 1$ \\
$p | |A|$\\
$|Z(G)| = p^{k} - \sum_{|A| > 1}|A|$\\
$|Z(G)| \vdots p$\\
$1_{G} \in Z(G)$ \\
$|Z(G)| > 0$\\
$\Ra |Z(G)| > 1$\\
\end{proof}

Упражнение toDo.

Можно показать, что группа $|G| = p^2$ абелева группа. \\
$G \cong \Z/(p^2\Z)$ или $\Z/(pZ) \times (\Z/(p\Z))$ \\

\chapter{Полилинейная алгебра и тензоры}
\section{Факторпространство}
$K$ "--- поле, $V$ "--- векторное поле над $K$, $U \subset V$.

$v_1, v_2 \in U \colon ,v_1 \sim v_2$, если $v_1 - v_2 \in U$. 

Упражнение. Это отношение эквивалентности. 
$[v] = v + U = \{v + u| u \in U\}$"--- линейное многообразие. 

$(v_1 + U) + (v_2 + U) = (v_1 + v_2) + U$
$\alpha(v_1 + U) = \alpha v_1 + U$

Упражнение: докажите что определение корректно(независимость от выбора представителей в классаx)

$V/\sim$

Упражнение: операции сложения классов и умножение классов на скаляр удовлетворяет аксиомам векторного пространства.

$V/\sim$ с операциями + и $\cdot$ называется факторпространством $V$ по подпространству $U$. $V/U$ 

Что значит, что $\{v_i + U\}_{i \in I}$ базис $V/U$.
$$\forall v + U \in V/U$$
$\exists$ и единственный набор $\alpha_i \in K$, почти все $\alpha_i = 0$

$$v + U = \sum_{i \in I}\alpha_i(v_i + U) = (\sum_{i \in I}\alpha_i v_i) + U $$


Упражнение:
$\{v_i + U\}_{i \in I}$  "--- базис $V/U$
$\Lra \{v_i\}_{i \in I}$ "--- относительный базис V относительно U.

\subsection{Отображение, индуцированное на факторкольце}
    $V, W$ "--- векторные пространства над $K$.
    $$U \subset V$$
    $ f \colon V \to W$ линейная. 
    
    Вопроc: когда можно построить естественное линейное индуцированное отображение $\bar f \colon  V / U \to W$

    Предложение: Если $U \subset \ker f$, то такое индуцированное отображение можно построить. 

    $$\bar f \colon V/U \to W$$
    $$v + U \in V/U$$
    $$\bar f(v + U) = f(v)$$

    корректность: $v + U = v' + U \Ra \bar f(v + U) = \bar f (v' + U)$

    $$v - v' \in U, f(v) = f(v')$$
    $$f(v) - f(v') = f(v - v' ) = 0$$
    $$v - v' \in U \subset \ker f $$

    $$\bar f(\alpha_1(v_1 + U) + \alpha_2(v_2 + U)) = \bar f(\alpha_1 v_1 + \alpha_2 v_2 + U) = $$
    $$= f(\alpha_1 v_1 + \alpha_2 v_2) = \alpha_1 f(v_1) + \alpha_2 f(v_2) = \alpha_1 \bar f(v_1 + U) + \alpha_2\bar f(v_2 + U) $$

    $$\bar f \in \mathcal{L}(V/U, W)$$



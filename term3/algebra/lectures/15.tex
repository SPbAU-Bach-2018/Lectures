\chapter{Полилинейная алгебра и тензоры}

\section{Факторпространство}

\begin{Def}
	$K$ "--- поле, $V$ "--- векторное поле над $K$, $U < V$.
	Введём отношение для $v_1, v_2 \in V$:
	\[ v_1 \sim v_2 \Lra v_1 - v_2 \in U \]
\end{Def}
\textbf{Упражнение:} Это отношение эквивалентности. То есть выполняется транзитивность, рефлексивность и симметричность.

\begin{Def}
	$[v] = v + U = \{v + u\colon u \in U\}$ "--- линейное многообразие.
	Действие над классами:
	\begin{gather*}
		(v_1 + U) + (v_2 + U) = (v_1 + v_2) + U \\
		\alpha(v_1 + U) = \alpha v_1 + U
	\end{gather*}
\end{Def}
\textbf{Упражнение:} докажите что определение корректно(независимость от выбора представителей в классаx)

Рассмотрим фактормножество $V/\sim$.
\textbf{Упражнение:} операции сложения классов и умножение классов на скаляр удовлетворяет аксиомам векторного пространства. То есть
такой фактор "--- это векторное пространство над полем $K$.

\begin{Def}
	$V/\sim$ с операциями + и $\cdot$ называется факторпространством $V$ по подпространству $U$.
	Обозначение: $V/U$.
\end{Def}
Что значит, что $\{v_i + U\}_{i \in I}$ "--- базис $V/U$?
\begin{gather*}
	\forall v + U \in V/U, \exists! \{\alpha_i\} \subset K\colon \text{почти все $\alpha_i$ равны 0}\colon \\
	v + U = \sum_{i \in I}\alpha_i(v_i + U) = \left(\sum_{i \in I}\alpha_i v_i\right) + U
\end{gather*}
\textbf{Упражнение:}
$\{v_i + U\}_{i \in I}$  "--- базис $V/U$ тогда и только тогда, когда
$\{v_i\}_{i \in I}$ "--- относительный базис $V$ относительно $U$.

\subsection{Отображение, индуцированное на факторпространстве}

$V, W$ "--- векторные пространства над $K$, $U < V$.
$f \colon V \to W$ линейная.
\textbf{Вопроc:} когда можно построить естественное линейное индуцированное отображение $\bar f\colon V/U \to W$?

\textbf{Предложение:} Если $U \le \ker f$, то такое индуцированное отображение можно построить.
\begin{gather*}
	\bar f \colon V/U \to W \\
	v + U \in V/U\colon \bar f(v + U) = f(v)
\end{gather*}
Корректность:
\begin{gather*}
	v + U = v' + U \overset{?}{\Ra} \bar f(v + U) = \bar f (v' + U) \\
	v - v' \in U \le \ker f \\
	v - v' \in U \Ra f(v) = f(v')\colon f(v) - f(v') = f(v - v') = 0 \\
\end{gather*}
Теперь проверим линейность:
\begin{gather*}
	\bar f\left(\alpha_1(v_1 + U) + \alpha_2(v_2 + U)\right) = \bar f\left(\alpha_1 v_1 + \alpha_2 v_2 + U\right) = \\
	= f(\alpha_1 v_1 + \alpha_2 v_2) = \alpha_1 f(v_1) + \alpha_2 f(v_2) = \alpha_1 \bar f(v_1 + U) + \alpha_2\bar f(v_2 + U) \\
	\bar f \in \mathcal{L}(V/U, W)
\end{gather*}
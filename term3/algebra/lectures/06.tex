\section{Самосопряженный оператор}

\begin{Def}
	$V$, $(,)$, $\mathcal A \in End(V)$.\\
	$\mathcal A $ называется самосопряженным, если
	\[ \mathcal A^* = \mathcal A \]
\end{Def}
Обозначение:
\begin{gather*}
	V, (,), \mathcal A \in End(V) \\
	\left<, \right>\colon V \times V \to K \\
	\left<x, y\right> = (\mathcal A x, y)
\end{gather*}
$\left<,\right>$ "--- билинейная или полуторалинейная форма. То есть построили новую форму на том же пространстве. 
Невырожденность в общем случае может пропадать, так как не требуем обратимости $\mathcal A$.

\begin{theorem}
	Пусть $(,)$ "--- симметричная (эрмитово симметричная) невырожденная, у $\mathcal A $ есть сопряжение.
	Тогда cледующие условия равносильны:
	\begin{enumerate}
		\item $\left<,\right>$  симметричная (эрмитово симметричная).
		\item $\mathcal A$  "--- самосопряженная.
	\end{enumerate}
\end{theorem}
\begin{proof}
	\begin{description}
	\item[$2 \Ra 1$:]
		\[
			\left<y, x\right> = (\mathcal A y, x) = (y, \mathcal A^*x)
			= (y, \mathcal A x) = \overline{(\mathcal A x, y)} = \overline{\left<x, y\right>}
		\]

	\item[$1 \Ra 2$:]
		\begin{gather*}
			\forall y, x, \left<y, x\right> = \overline{\left<x, y\right>} \Ra\\
			\forall y, x, (\mathcal A y, x) = \overline{(\mathcal A x, y)} \\
			(y, \mathcal A^*x) = (\mathcal A y, x) = \overline{(\mathcal A x, y)} = (y, \mathcal A x) \\
			x \in V \colon \forall y, (y, \mathcal A^*x - \mathcal A x) = 0
		\end{gather*}
		В силу невырожденности $(,)$
		\[ \mathcal A^*x - \mathcal A x = 0 \]
		В силу произвольности $x$
		\[ \mathcal A^* = \mathcal A \]
	\end{description}
\end{proof}

\begin{exmp}
	$\dim V = n < \infty$, $\{v_i\}$ "--- базис, $(,)$ "--- невырожденная (эрмитово)симметричная форма, $\Gamma$, $\mathcal A$, $[\mathcal A]$.
	\[ \left<x, y\right> = (\mathcal A x, y) \]
	$\Gamma'$ "--- матрица Грамма $\left<,\right>$.
	\[ \Gamma' = [\mathcal A]^T \cdot \Gamma \]
	Если $\mathcal A$ "--- самосопряжён, то $\Gamma' = \overline{(\Gamma')^T}$:
	\begin{gather*}
		[\mathcal A]
		= [\mathcal A^*]
		= \bar \Gamma^{-1} \overline{[\mathcal A]^T} \bar \Gamma \\
		\overline{[\mathcal A]} = \Gamma^{-1} [\mathcal A]^T \Gamma \\
		\overline{(\Gamma')^T}
		= \overline{(\Gamma)^T} [\overline{\mathcal A}]
		= \Gamma[\overline{\mathcal A}] = \Gamma \cdot \Gamma^{-1}[\mathcal A]^T\Gamma = \Gamma'
	\end{gather*}
\end{exmp}

\begin{conseq}
	$\dim V < \infty$.
	Тогда $\left<,\right>$ невырожденная тогда и только тогда, когда $\mathcal A$ обратима.
\end{conseq}
\begin{proof}
$$\Gamma' = [\mathcal A]^T\Gamma$$
$\left <, \right>$ невырожденная $\Lra$ $\det \Gamma' \ne 0$
$$\Lra \det [\mathcal A]^T \ne 0$$
$$\Lra \mathcal A \text{"--- обратим}$$ 
\end{proof}

\begin{conseq}
	$\dim V < \infty$,
	$(,)$ "--- невырожденная симметричная (эрмитово симметричная)/
	Тогда всякое симметричное (эрмитово симметричное) скалярное произведение на $V$ имеет вид
	\[ \left<x, y\right> = (\mathcal A x, y) \]
	для некоторого самосопряженного оператора $\mathcal A$.
\end{conseq}
\begin{proof}
	$\{v_i\}$ "--- зафиксируем базис $V$, $(,) = \Gamma$ "--- исходной скалярное произведение.\\
	Все $\left<,\right>$ однозначно определяются $\Gamma'$ "--- своей матрицей Грамма.
	Хотим построить самосопряженный оператор у этой матрицы Грамма.
	\begin{gather*}
		[\mathcal A] = (\Gamma' \Gamma^{-1})^T = (\Gamma^T)^{-1} (\Gamma')^T \\
		\Gamma' = \overline(\Gamma')^T \\
		[\mathcal A^*] = \overline{\Gamma^{-1}} \overline{[\mathcal A]^T]} \overline{\Gamma} = \\
		= \overline{\Gamma^{-1}} (\overline{\Gamma' \Gamma^{-1}}) \overline{\Gamma} = \\
		= \overline{\Gamma^{-1}} \overline{\Gamma'} = (\Gamma^T)^{-1}(\Gamma')^T \\
		\overline{\Gamma^T} = \Gamma \quad \overline{\Gamma'^T} = \Gamma' \\
		[\mathcal A] = [\mathcal A^*] \Ra \mathcal A = \mathcal A^*
	\end{gather*}
\end{proof}

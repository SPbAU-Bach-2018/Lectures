\section{Факторгруппы}

$G, H \unlhd G$.
$X$ "--- множество (левых) смежных классов по $H$.
Определим операцию умножения. 
\[ g_1H g_2H = g_1g_2H \]
\begin{theorem}
	Множество $X$ с так введенной операцией является группой. 
\end{theorem}
\begin{proof}\begin{description}
\item[Шаг 1: Корректность.]
	Пусть $g_1H = g_1'H, g_2H = g_2'H$
	Нужно проверить, что $g_1g_2H = g_1'g_2'H$.
	\begin{gather*}
		(g_1g_2)^{-1}g_1'g_2' \in H \\
		(g_1g_2)^{-1}g_1'g_2'
		= g_2^{-1}g_1^{-1}g_1'g_2'
		= \underbrace{g_2^{-1}g_2'}_{\in H} \underbrace{(g_2')^{-1} \underbrace{g_1^{-1}g_1'}_{\in H}g_2'}_{\in H, H \unlhd G} \in H
	\end{gather*}

\item[Шаг 2: Проверим ассоциативность.]
	Еще одна интерпритация умножения:
	\[ g_1Hg_2H = g_1(Hg_2)H = g_1(g_2H)H = g_1g_2(HH) = g_1g_2H \]
	Ассоциативность умножения в $X$ следует из ассоциативности умножения по Минковскому.

\item[Шаг 3: Нейтральный элемент.]
	\begin{gather*}
		1_X = H = 1_GH \\
		HgH = gHH = gH \\
		gHH = gH
	\end{gather*}

\item [Шаг 4: Для любого смежного класса найдется обратный.]
	\begin{gather*}
		(gH)^{-1} = g^{-1}H \\
		gHg^{-1}H = gg^{-1}HH = 1H = H \\
		g^{-1}HgH = g^{-1}gHH = 1H = H
	\end{gather*}
\end{description}\end{proof}

\begin{Def}
	Группа, построенная в предыдущей теореме называется факторгруппой $G$ по нормальной подгруппе $H$.
	Обозначение: $G / H$.
\end{Def}

\begin{theorem}
	$G$ "--- группа, $H \unlhd G$,\\
	$\phi \colon G \to G / H$, \\
	$\phi(g) = gH$.\\
	Тогда $\phi$ сюръективный гомоморфизм групп и $\ker \phi = H$.
\end{theorem}
\begin{proof}
        \begin{description}
        \item [Гомоморфизм:] \hfill \\
	\[ \phi(g_1)\phi(g_2) = g_1Hg_2H = g_1g_2H = \phi(g_1g_2) \]
	\item [Сюръективность] \hfill \\
	 очевидна. 

	\item [Проверяем $\ker \phi = H$.] \hfill \\
	\begin{description}
	\item[$\supset$:] \hfill \\
		$h \in H$
		\[ \phi(h) = hH =  H = 1_{G/H} \Ra H \subset \ker \phi \]

	\item[$\subset$:] \hfill \\
		$h \in \ker \phi$
		\begin{gather*}
			hH = \phi(h) = 1_{G/H} = H \\
			\Ra h\cdot1 \in H \Ra h \in H \\
			\Ra \ker \phi \subset H
		\end{gather*}
	\end{description}
\end{proof}

\begin{conseq}
	Нормальные подгруппы "--- это в точности ядра гомоморфизмов.
\end{conseq}

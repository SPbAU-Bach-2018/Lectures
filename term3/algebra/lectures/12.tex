\section{\texorpdfstring{$SU(2)$}{SU(2)}, кватернионы и \texorpdfstring{$SO(3)$}{SO(3)}}

\begin{Def}
    \[ SU(n) = \{g \in U(n) \mid \det g = 1\} \]
\end{Def}
$g \in SU(2)$
\begin{gather*}
    g = \begin{pmatrix}
        w_1 & w_2 \\
        w_3 & w_4
    \end{pmatrix} \\
    g^T\bar g = E_2 \\
    \begin{pmatrix}
        w_1 & w_3 \\
        w_2 & w_4
    \end{pmatrix} \cdot \begin{pmatrix}
        \bar w_1 & \bar w_2 \\
        \bar w_3 & \bar w_4
    \end{pmatrix} = E = \begin{pmatrix}
        \bar w_1 & \bar w_2 \\
        \bar w_3 & \bar w_4
    \end{pmatrix} \cdot \begin{pmatrix}
        w_1 & w_3 \\
        w_2 & w_4
    \end{pmatrix} \\
    g^T\bar g = E \Ra \bar g g^T = E \\
    \bar g = (g^T)^{-1} \\
    \left\{
    \begin{aligned}
    |w_1|^2 + |w_2|^2 = 1 \\
    |w_3|^2 + |w_4|^2 = 1 \\
    \bar w_1w_3 + \bar w_2 w_4 = 0 \\
    \bar w_3w_1 + \bar w_4 w_2 = 0 \\
    \end{aligned}
    \right.
    \det g = 1 \Ra w_1w_4 - w_2w_3 = 1 \\
    \begin{pmatrix}
        \bar w_1 & \bar w_2 \\
        -w_2 & w_1
    \end{pmatrix}\begin{pmatrix}
        w_3 \\
        w_4
    \end{pmatrix} = \begin{pmatrix}
        0 \\
        1
    \end{pmatrix} \\
    \det = |w_1|^2+|w_2|^2 = 1 \\
    \begin{pmatrix}
        w_3 \\
        w_4
    \end{pmatrix} = \begin{pmatrix}
        w_1 &-\bar w_2 \\
        w_2 & \bar w_1
    \end{pmatrix}\begin{pmatrix}
        0 \\
        1
    \end{pmatrix} \\
    w_3 = -\bar w_2 \quad w_4 =  \bar w_1
\end{gather*}
% OH MY GOD
\begin{gather*}
    g \in SU(2) \Lra g = \begin{pmatrix}
        w_1 & w_2 \\
        -\bar w_2 & \bar w_1
    \end{pmatrix} \\
    |w_1|^2 + |w_2|^2 = 1 \\
    \La\\
    g^T\bar g = \begin{pmatrix}
        w_1 & -\bar w_2 \\
        w_2 &  \bar w_1
    \end{pmatrix}\begin{pmatrix}
        \bar w_1 & \bar w_2 \\
        -w_2 & w_1
    \end{pmatrix} = \begin{pmatrix}
        1 & 0 \\
        0 & 1
    \end{pmatrix} = E\\
    g = \begin{pmatrix}
         a + bi & c + di \\
        -c + di & a - bi
    \end{pmatrix} \\
    a, b, c, d \in \R, a^2 + b^2 + c^2 + d^2 = 1
\end{gather*}

Унитарные матрицы $SU(2)$, матрицы Паули:
\begin{gather*}
    1 = \begin{pmatrix}
        1 & 0 \\
        0 & 1
    \end{pmatrix},\i = \begin{pmatrix}
        i,  0 \\
        0, -i
    \end{pmatrix},\j = \begin{pmatrix}
         0 & 1 \\
        -1 & 0
    \end{pmatrix},\k = \begin{pmatrix}
        0, i \\
        i, 0
    \end{pmatrix} \\
    \i^2 = \j^2 = \k^2 = -E \\
    \i\j = \k = -\j\i \\
    g = aE + b\i + c\j + d\k \\
    a^2 + b^2 + c^2 + d^2 = 1\\
    \H_1 = \{z \in \H \mid |z| = 1\} \text{"--- кватернионы нормы 1}
\end{gather*}
$\H_1 \to SU(2)$ "--- изоморфизм групп.
\begin{gather*}
    a + b\i + c\j + d\k \to \begin{pmatrix}
         a + bi & c + di \\
        -c + di & a - bi
    \end{pmatrix}\text{(биекция)} \\
\end{gather*}
Сохраняется ли групповая операция(то есть гоморфизм)?
\begin{gather*}    
    \begin{pmatrix}
        w_1 & w_2 \\
        -\bar w_2& \bar w_1
    \end{pmatrix}\begin{pmatrix}
        u_1 & u_2 \\
        -\bar u_2 & \bar u_1
    \end{pmatrix} = \begin{pmatrix}
        w_1u_1 - w_2\bar u_2 & w_1u_2 + w_2 \bar u_1 \\
        -u_1\bar w_2 - \bar u_2 \bar w_1 & -u_2\bar w_2 + \bar w_1 \bar u_1
    \end{pmatrix}
\end{gather*}
$$
\left\{
\begin{aligned}
g = a + b\i + c\j + d\k\\
g_1 = a_1 + b_1\i + c_1\j = d_1\k\\
\end{aligned}
\right.
$$

$gg_1 = $ то же, что при перемножение кватернионов. 

Было:
\begin{gather*}
    \H \sim \left\{ \begin{pmatrix}
         a + bi & c + di \\
        -c + di & a - bi
    \end{pmatrix} \middle| a, b, c, d \in \R \right\} \\
    a + b\i + c\j + d\k \to \begin{pmatrix}
         a + bi & c + di \\
        -c + di & a - bi
    \end{pmatrix}
\end{gather*}
Гомоморфизм колец, значит сужение на $\H_1$ сохраняет операцию умножения,
то есть является гомоморфизмом $\H_1$ в $SU(2)$, значит есть изоморфизм.
\[ \dim_{\R}(\H) = 4 \]
% ВЫТАЩИТЕ МЕНЯ ОТСЮДА
\begin{enumerate}
\item
    \begin{gather*}
        \H = \{a + b\i + c\j + d\k \} \\
        \H_1 = \{z \in \H \colon |z| = 1\} \\
        (z_1, z_2) = Re(z_1, \bar z_2) \text{скалярное произведение}\\
        (z_1, z_1) = Re(z_1, \bar z_1) = |z_1|^2 = a_1^2 + b_1^2 + c_1^2 + d_1^2\
        (z_1 + z_1', z_2) = (z_1, z_2) + (z_1', z_2) \Ra \text{билинейная форма}\\
        z \to \bar{z}
    \end{gather*}

\item
    \[
        g = \begin{pmatrix}
             a + bi & c + di \\
            -c + di & a - bi
        \end{pmatrix}
    \]
    $g \in SU(2)$, $\det g = 1$.
    Вещественная часть "--- это $\frac12 \Tr(g)$
    \begin{gather*}
        (g_1, g_2) = \frac12 \Tr(g_1, \bar g_2^T) \\
        (g_1, g_1) = \det g_1 \\
        (g_1, g_1) = \frac12 \Tr(g_1, \bar g_1^T) = \frac12(a^2 + b^2 + c^2 + d^2 + a^2 + b^2 + c^2 + d^2) = \det g_1
    \end{gather*}
    $$g \to \bar g^T$$
    $$\bar g = 
    \begin{pmatrix}
     a - bi & -c-di\\
     c - di & a + bi\\
     \end{pmatrix}$$
\end{enumerate}

Действие $\H_1$ на $\H \sim \R^4$

$\H_1 \times \H \to \H$

$B(x, y)$ "--- симметричная, билинейная

$Q(x) = B(x, x)$  "--- квадратичная форма.

По $Q(x)$ можно востановить $B(x, y):$
$$B(x, y) = \frac{1}{2}(B(x + y, x + y) - B(x, x) - B(y, y)) = \frac{1}{2}(Q(x + y) - Q(x) - Q(y))$$

Достаточно проверить, что сохраняется соответствующая квадратичная форма
$$\H_1\times \H \to \H $$
$(h, z) \to hzh^{-1}  = hz\bar h$  "--- это отображение "--- изометрия.

$Q(z) = |z|^2$

$Q(hzh^{-1}) = |hzh^{-1}|^2 = |h|^2|z|^2|h|^{-2} = |z|^2 = Q(z)$

$z \to hzh^{-1}$ "--- изометрия $\H \sim \R^4$

$\{a\colon a \in \R\} = <1>_{\R}$ "--- инвариантно относительно этого действия.

$hzh$

$$hah^{-1} = a$$

$<1>^\bot = <\i, \j, \k>$

$h \in \H_1$ действует на $\R^4 \sim \H$.
$z \to hzh^{-1} = hz\bar h$ (так как $|z| = 1$, то $z^{-1} = \bar z$)
$$\phi_h \colon \H \to \H$$
изометрия 4х мерного пространства.
  
Утеверждение: $(z_1, z_2) = (hz_1h^{-1}, hz_2h^{-1})$

\begin{Rem}
Доказательство с помощью матричного представления.

$h = \begin{pmatrix}
a+bi & c + di\\
-c + di & a - bi\\
\end{pmatrix},
z =
\begin{pmatrix}
x_0 + x_1i & x_2 + x_3i\\
-x_2 + x_3i & x_0 - x_1i\\
\end{pmatrix}
$

$$(z_1, z_2) = \frac{1}{2}Tr(z_1, \bar z_2^T)$$
$$(hzh^{-1}, hz_2h^{-1}) = \frac{1}{2}Tr(hzh^{-1}\cdot \overline{(h z_2 h^{-1})}^T) = $$
$$= \frac{1}{2}Tr(hzh^{-1}\bar h^{-1})^Tz_2^Th^T) = \frac{1}{2}Tr(hz\bar z_2^T \bar h^T) = \frac{1}{2}Tr(z\bar z_2^T\bar h^Th) = \frac{1}{2}Tr(z\bar z_2^T)$$

\end{Rem}

\begin{conseq}
 $\{z \in \H\colon Re z = 0\} = <1>^\bot$ "--- инвариатно
 относительно действия h сопряжением.
\end{conseq}

\begin{proof}
$z \in <1>^\bot$

$$(1, z) = 0$$
$$(h1h^{-1}, hzh^{-1}) = 0 \Ra hzh^{-1} \in <1>^\bot$$
\end{proof}

Можно рассмотреть сужение действия сопряжения при помощи h на пространство линейных кватернионов.

$V = <1>^\bot$

$\phi_h|_v \colon V \to V$

Сохраняет скалярное произведение (так как $\phi_2$ изометрия на всем $\H$)

$V = <\i, \j, \k> \sim \R^3$

$\phi_h|_v \in O(3)$

$$h \in \H_1 \sim SU(2)$$
$$h \to \phi_h|_V$$
$$SU(2) \to O(3)$$
$$\phi_{h_1h_2}(z) = h_1h_2zh_2^{-1}h_1^{-1} = \phi_{h_1}(\phi_{h_2}(z))$$
$$\phi_{h_1h_2} = \phi_{h_1}\circ \phi_{h_2}$$
$$\phi_{h_1h_2}|_v = \phi_{h_1}|_v \circ \phi_{h_2}|_v $$

Отображение $h \to \phi_h|_v$ является гомоморфизмом.

\begin{theorem}
    Отображение из $\H_1 \sim SU(2)$  h ставится
    в соответствие $\phi_h|_V$  является гомоморфизмом из
    $SU(2)$ на $SO(3) = \{g \in O(3) \colon \det g = 1\}$
    причем ядро этого отображение есть $\{\pm1\}$

    $h \to \phi_h|_v$
\end{theorem}

\begin{Rem}
С топологической точки зрения
$SU(2)$  гомеоморфна $S^3$

$$
\begin{pmatrix}
a + bi & c + di\\
-c + di & a - bi\\
\end{pmatrix}
$$

$$a^2 + b^2 + c^2 + d^2 = 1$$

C топологической точки зрения $SO(3)$  устроено как $RP^3$.

$RP^3$  можно вложить в $\R^9$

$SO(3)$ "--- зависит от 9 параметров удовлетворяющее 7 условиям.

$$f \in SO(3)$$
$$
\begin{pmatrix}
a_1 & a_2 & a_3\\
a_4 & a_5 & a_6\\
a_7 & a_8 & a_9\\
\end{pmatrix}
$$

$$a_1^2 + a_4^2 + a_7^2 = 1$$
$$a_2^2 + a_5^2 + a_8^2 = 1$$
$$a_3^2 + a_6^2 + a_9^2 = 1$$
$$a_1a_2 + a_4a_5 + a_7a_8 = 0$$
$$\cdots$$
$$\det(\cdots) = 1$$
\end{Rem}
\begin{proof}
$h \to [\phi_h|_v]$ в базисе из матриц Пауля.

$$
h = \begin{pmatrix}
a + b_i & c + d_i\\
-c = d_i & a - b_i\\
\end{pmatrix}
$$

$$
h \begin{pmatrix}
i & 0\\
0 & -i\\
\end{pmatrix}
h^{-1}
=
\begin{pmatrix}
(a^2 + b^2 - c^2 -d^2)i & 2ad + 2bc + (2bd - 2ac)i\\
-2ad - 2bc + (2bd - 2ac)i & (-a^2 -b^2 + c^2 + d^2)i\\
\end{pmatrix}
$$
$$
h
\begin{pmatrix}
0 & 1\\
-1 & 0\\
\end{pmatrix}
h^{-1} =
\begin{pmatrix}
(-2ad + 2bc)i & a^2 - b^2 + c^2 - d^2 + (2ab + 2cd)i\\
*&*\\
\end{pmatrix}
$$
$$
h
\begin{pmatrix}
0 & i\\
i & 0\\
\end{pmatrix}
h^{-1} =
\begin{pmatrix}
(2ac + 2bd)i & (2cd - 2ab) + (a^2 - b^2 - c^2 - d^2)i\\
*&*\\
\end{pmatrix}
$$
$$
[\phi_h|_V] =
\begin{pmatrix}
a^2 + b^2 - c^2 - d^2 & -2ad + 2bc & 2ac + 2bd\\
2ad + 2bc & a^2 - b^2 + c^2 - d^2 & -2ab + 2cd\\
2bd - 2ac & 2ab + 2cd & a^2 - b^2 - c^2 + d^2\\
\end{pmatrix}
$$
$$\det [\phi|_v] = (a^2 + b^2 + c^2 + d^2)^3 = 1$$
$$[\phi][\phi]^t = (a^2 + b^2 + c^2 + d^2)^2E = E$$
\end{proof}

Отступление: кватернионы и трехмерная геометрия.
$$z = x_0 + x_1i + x_2j + x_3k = [x_0, \bar w]$$
$$z' = x_0' + x_1'i + x_2'j + x_3'k = [x_0', \bar w]$$
$$zz' = [x_0x_0' - (\bar w, \bar w'), x_0\bar w' + x_0'\bar w + (x_1x_2' - x_2 x_1')k + (x_3x_1' - x_1x_3')j + (x_2x_3' - x_3x_2')i]  =$$
$$= [x_0x_0' - (\bar w, \bar w'), x_0\bar w + x_0'\bar w + \begin{vmatrix}
i & j & k\\
x_1 & x_2 & x_3\\
x_1' & x_2' & x_3'\\
\end{vmatrix}] =$$
$$= [x_0x_0' - (\bar w, \bar w'), x_0\bar w' + x_0'\bar w + \bar w \times \bar w'] $$

\begin{proof}
Доказательство, что отображение $h \to \phi_h|_v$  "--- сюръекция.
$A \in SO(3)$, если $A = E$, то $h = 1$

Ищем h, чтобы $A = [\phi_h|_V]$

Если $A \ne E$, то существует ортонормированный базис V в котором оператор усножения на A имеет вид
$$
\begin{pmatrix}
\cos \alpha & - \sin \alpha & 0\\
\sin \alpha & \cos\alpha & 0\\
0 & 0 & 1\\
\end{pmatrix}
$$
Есть $w_1 \ne 0, A\bar w_1 = \bar w_1, |w_1| = 1$

A "--- Поворот оси $\bar w_1$  на угол $\phi$

Выберем какой-то $\bar w_2 \bot w_1$, $|w_2| = 1$

$\bar w_1, \bar w_2$ и $\bar w_1 \times \bar w_2$ "--- базис V.
$\bar w_2$ и $\bar w_1 \times \bar w_2$ "--- ортонормированный базис $<w_1>^\bot$

$$h = ?$$
$$h - [x_0, \bar w]$$
$$h \bar w h^{-1} = (x_0 + \bar w)\bar w(x_0 - \bar w) = \bar w(x_0 + \bar w)(x_0 - \bar w) = \bar w h h^{-1} = \bar w$$
$$|h| = 1$$
Если хотим, чтобы ось вращения совпадала с осью натянотой на $w_1$, то $w = \lambda w_1$

$h = [x_0, \lambda w_1]$
$1 = |h|^2 = x_0^2 + \lambda^2|w_1|^2 = x_0^2 + \lambda^2  \Ra \exists \in \R x_0 = \cos \psi, \lambda = \sin \psi$
$$h = \cos \psi + \sin \psi \bar w_1, hw_1h^{-1} = w_1$$
$$hw_2h^{-1} = (\cos \psi  + \sin \psi \bar w_1 )(0 + \bar w_2)(\cos \psi - \sin \psi \bar w_1) = $$
$$= (\cos \psi 0 - \sin \psi (\bar w_1 \bar w_2) + \cos \psi \bar w_2 + 0 + $$
$$\sin \psi \cdot \bar w_1 \times \bar w_2)(\cos \psi - \sin \psi \bar w_1)$$
$$= (\cos \psi \bar w_2 + \sin \psi \bar w_1 \times \bar w_2)(\cos \psi - \sin \psi \bar w_1) = $$
$$= 0 + \cos^2 \psi \bar w_2 + \cos \psi \sin \psi \bar w_1 \times \bar w_2 - \cos \psi \sin \psi \bar w_2 \times \bar w_1 - $$
$$- \sin^2 \psi(\bar w_1 \times \bar w_2) \bar w_1 = $$

$(w_1 \times w_2) \times w_1$ "--- нормированный $\bot w_1 \times w_2$, $\bot w_1$

$w_1, w_2, w_1 \times w_2$ "--- ортонормированный базис.

$(w_1 \times w_2) \times w_1 = \pm w_2$
$(w_1 \times w_2, w_1, \pm w_2)$ "--- правая тройка. $\Ra$ знак +.

$$= (\cos^2 \psi - \sin^2 \psi)\bar w_2 + 2 \cos \psi \sin \psi \bar w_1 \bar w_2 =  $$
$$= \cos 2 \psi \bar w_2 + \sin 2 \psi \bar w_1 \times \bar w_2$$
$$2 \psi = \alpha$$

\end{proof}

\begin{proof}
$h = \pm$

$1 z 1^{-1} = z, \phi_1 = id$

$(-1)z(-1)^{-1} = z, \phi_2 = id$

Когда образ есть $E_3$?
    $$4ab = 0$$
    $$4ad = 0$$
    $$4ac = 0$$
    \begin{enumerate}
    \item a = 0
    $$2bc = 0$$
    $$2bd = 0 $$
    $$2cd =0 $$
    $\Ra $ хотя бы 2 из трех $b, c, d$ равны 0.
    $1 ненулевой$, остальные 0. На диагонали найдем противоречие.
    $$\Ra a \ne 0 \Ra b = c = d = 0$$
    $$a^2 = 1$$
    $$a = \pm 1$$
    \end{enumerate}
\end{proof}


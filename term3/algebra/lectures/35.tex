\section{Группы, заданные образующими и соотношениями}
$|A| < \infty$, $A \cup \bar{A}$\\
Есть какой-то набор слов в алфавите из этих символов $w_1, \cdots, w_n \in W$\\
Рассмотрим $F_A$ и рассмотрим $H$ "--- наименьшая нормальная подгруппа, содержащая классы слов $w_1, \cdots, w_n$, не умоляя общности 
можем считать, что эти слова несократимые.

Почему такая подгруппа существует? Мы ее можем определить так $H = \cap_{G \unlhd F_A} G$.
Пересечние любого количества нормальных групп вновь нормально, с другой стороны, пересекаемое множество не пусто пересечение содержит 
нужные нам элементы. 

Раз $H$ нормальная подгруппа, то мы можем рассмотреть факторгруппу.
$F_A/H = <a_1, \cdots, a_k|w_1 = 1, \cdots, w_n = 1>$\\
Вторая половина это обозначение, перечисляются образующие, перечисляются соотношения. То есть, 
после факторизации все наши слова будут нейтральными. 

У нас есть гоморфизм
$\phi \colon F_A \to F_A/H$\\
$w_i \to 1$ \\
Эта группа еще называется: группа --- образующими $a_1, \cdots, a_k$ и соотношениями $w_1, \cdots, w_n$\\


Такую группу мы можем задать другим способом. Начинаем со слов $W$,
теперь мы разрешаем вставки/вычеркивание $aa^{-1}, a^{-1}a, w_1, \cdots, w_n$. То есть 
завили новые классы эквивалентности.\\
$W/\sim$ изоморфно $F_A/H$. Это мы оставим без доказательства.

\begin{exmp}
$<a, b| aba^{-1}b^{-1} = 1> \cong \Z \times \Z$\\
$aa^{-1}, bb^{-1}, \cdots, aba^{-1}b^{-1}$

В каждом классе ровно 1 представитель вида $a^nb^m$\\
$u_1bau_2 = u_1aba^{-1}b^{-1}bau_2 = u_1abu_2$\\

$a^nb^m = a^{n'}b^{m'} \Ra n = n', m = m'$\\
$w$ посчитаем суммарное вхождения a и суммарное вхождения b.
При данных преобразованиях количество вхождений не меняется.  
\end{exmp}
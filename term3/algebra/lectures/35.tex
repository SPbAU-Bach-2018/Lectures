\section{Группы и соотношения}
$|A| < \infty$\\
$A \cup \bar{A}$\\
$w_1, \cdots,w_n \in W$\\
$F_A$\\
$H$ "--- наименьшая нормальная подгруппа, содержащая классы слов. 
$w_1, \cdots, w_n$.
$H = \cap_{G \unlhd F_A} G$ пересечние любого количества нормальных
групп вновь нормально. 
С другой стороны, пересекаемое множество не пусто. 

$F_A/H = <a_1, \cdots, a_k|w_1 = 1, \cdots, w_n = 1>$\\
$F_A \to F_A/H$\\
$w_i \to 1$ \\
группа --- образующими $a_1, \cdots, a_k$ и соотношениями $w_1, \cdots, w_n$\\

$W$, вставки/вычеркивание $aa^{-1}, a^{-1}a, w_1, \cdots, w_n$ 
$W/\sim$ изоморфно $F_A/H$.

\begin{exmp}
$,a, b| aba^{-1}b^{-1} = 1> \cong \Z \times \Z$\\
$aa^{-1}, bb^{-1}, \cdots, aba^{-1}b^{-1}$

В каждом классе ровно 1 представитель вида $a^nb^m$\\
$u_1bau_2 = u_1abu_2$\\
$u_1aba^{-1}b^{-1}bau_2 = u_1abu_2$\\

$a^nb^m = a^{n'}b^{m'} \Ra n = n', m = m'$\\
$w$ посчитаем суммарное вхождения a и суммарное вхождения b.
При данных преобразованиях количество вхождений не меняется.  
\end{exmp}
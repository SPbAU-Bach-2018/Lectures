\section{Нормальные операторы в евклидовых и унитарных пространствах.}

\begin{Def}
$V$ "--- евклидово или унитарное (, ) "--- стандартное скалярное произведение.

$\mathscr{A} \in End(V)$ 

$\mathscr{A}$ "--- нормальный, если $\mathscr{A} \mathscr{A}^* = \mathscr{A}^*\mathscr{A}$.
\end{Def}

\begin{exmp}
\begin{enumerate}
    \item  самосопряженный "--- нормальный. 
    \item Ортогональный и унитарный операторы так же нормальный. 

    \begin{proof}
    $(\mathscr{A}x, \mathscr{A}y) = (x, y)$

    Изометрия "--- обратимое преобразование. 

    $y = \mathscr{A}^{-1}z$
    
    $$(x, \mathscr{A}^*z) = (\mathscr{A}x, z) = (x, \mathscr{A}^{-1}z)$$

    Если $\mathscr{A}$ "--- унитарное или ортогональное, тогда $\mathscr{A}^{*} = \mathscr{A}^{-1}$.

    $$\forall x (x, \mathscr{A}^{-1}z - \mathscr{A}^{*}z) = 0 \Ra \mathscr{A}^{-1} = \mathscr{A}^*$$

    $\mathscr{A}$ и $\mathscr{A}^{-1}$ коммутируют. 
    \end{proof}    
\end{enumerate}
\end{exmp}

\begin{theorem}
$V$ "--- унитарное пространство. 

Следующие условия равносильны:
    \begin{enumerate}
    \item $\mathscr{A}$ "--- нормальна. 
    \item $\exists$  ортонормированный базис $V$, в котором $[\mathscr{A}]$ диагональная. 
    \end{enumerate}
\end{theorem}

\begin{proof}
$$2 \Ra 1$$
$\{v_i\}$ "--- ортонормированный базис $\Gamma = E$
$[\mathscr{A}] $ "--- диагональная. 

$[\mathscr{A}^*] = \overline{[\mathscr{A}]^{T}}$ "--- диагональная. 

$$[\mathscr{A}\mathscr{A}^*] = [\mathscr{A}][\mathscr{A}^*] = [\mathscr{A}^*][\mathscr{A}] = [\mathscr{A}^*\mathscr{A}] \Ra \mathscr{A}\mathscr{A}^* = \mathscr{A}^*\mathscr{A}$$
$$1 \Ra 2$$

$\mathscr{A}$ "--- нормальный оператор. 

$\lambda$ "--- собственное число $\mathscr{A}$.

$U(\lambda) = U_1(\lambda)$ "--- пространство собственных векторов отвечающих $\lambda$.

$U$ "--- $\mathscr{A}$-инвариантное подпространство.

$\mathscr{A}^*$ "--- инвариантно?

Возьмем $u \in U$ и применим оператор $\mathscr{A}^*$, нас интересует, почему $\mathscr{A}^*u \in U$ 

$$\mathscr{A}(\mathscr{A}^*u) = \mathscr{A}^*(\mathscr{A}u) = \mathscr{A}^*(\lambda u) = \lambda \mathscr{A}^*(u)$$

$\mathscr{A}^*u$ "--- собственный вектор $\mathscr{A}$ отвечающий $\lambda$

$$\Ra \mathscr{A}^*u \in U$$ 

Мы нашли подпространство, которое одновременно $\mathscr{A}$ и $\mathscr{A}^*$ инвариантно. 

$U^{\bot}$ "--- $\mathscr{A}^*$ и $\mathscr{A}^{**} = \mathscr{A}$  "--- инвариантно. 

$V = U \oplus U^{\bot}$ и оба подпространства инвариантны $\mathscr{A}$ и $\mathscr{A}^*$.

Значит определены. 
$$\mathscr{A}|_{u}, \mathscr{A}^*|_{u} \in End(U)$$
$$ \mathscr{A}|_{u^{\bot}}, \mathscr{A}'|_{u^{\bot}} \in End(U^{\bot})$$

$\mathscr{A}|_{u}$ и $\mathscr{A}^*|_{u}$ "--- нормальные операторы на $U$.

$\mathscr{A}|_{u^{\bot}}$ и $\mathscr{A}^*|_{u^{\bot}}$ "--- нормальные операторы на $U^{\bot}$.

Выберем ортонормированный базис U.

$[\mathscr{A}|_{u}] = \lambda E_{\dim U}$

$[\mathscr{A}^{*}|_{u}] = \overline{[\mathscr{A}|_{u}]^{T}} = \overline{\lambda}E$

Далее: доказательство по индукции (применить к $U^{\bot}$)
\end{proof}

\begin{conseq}
    \begin{enumerate}
    \item 
    $\mathscr{A}$ "--- нормальный оператор. 
    $u$ "--- собственный вектор A, отвечающй  $\lambda$
    $\Ra$  $u$ "--- собственный вектор $\mathscr{A}^*$, отвечающий $\overline{\lambda}$
    \item 
    $\mathscr{A}$ "--- норма в унитарном пространстве. 

    собственный вектор, отвечающий разным собственным числам ортогонального. 
    $$\mathscr{A}u = \lambda u$$
    $$\mathscr{A}v = \beta v$$
    $$\beta(u, v) = (u, \overline{\beta v}) = (u, \mathscr{A}^*v) = (\mathscr{A}u, v) = (\lambda u, v) = \lambda (u, v)$$
    $$\Ra (u, v) = 0$$
    \item 

    Всякий нормальный оператор в унитарном пространстве диагонализируемы. 
    \end{enumerate}
\end{conseq}



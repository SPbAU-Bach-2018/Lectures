\section{Нормальные операторы в евклидовых и унитарных пространствах.}

\begin{Def}
	$V$ "--- евклидово или унитарное, $(,)$ "--- стандартное скалярное произведение, $\mathcal A \in End(V)$.
	$\mathcal A$ "--- нормальный, если $\mathcal A \mathcal{A}^* = \mathcal{A}^*\mathcal A$.
\end{Def}

\begin{exmp}
	\begin{enumerate}
		\item Самосопряженный "--- нормальный.
		\item Ортогональный и унитарный операторы также нормальные.
		\begin{proof}
			\[ (\mathcal Ax, \mathcal Ay) = (x, y) \]
			Изометрия "--- обратимое преобразование.
			$y = \mathcal{A}^{-1}z$.
			\[ (x, \mathcal{A}^*z) = (\mathcal Ax, z) = (x, \mathcal{A}^{-1}z) \]
			Если $\mathcal A$ "--- унитарное или ортогональное, тогда $\mathcal{A}^* = \mathcal{A}^{-1}$.
			\[ \forall x (x, \mathcal{A}^{-1}z - \mathcal{A}^*z) = 0 \Ra \mathcal{A}^{-1} = \mathcal{A}^* \]
			$\mathcal A$ и $\mathcal{A}^{-1}$ коммутируют.
		\end{proof}
	\end{enumerate}
\end{exmp}

\begin{theorem}
	$V$ "--- унитарное пространство.
	Следующие условия равносильны:
	\begin{enumerate}
		\item $\mathcal A$ "--- нормальна.
		\item Существует ортонормированный базис $V$, в котором $[\mathcal A]$ диагональная.
	\end{enumerate}
\end{theorem}

\begin{proof}
	\begin{description}
	\item[$2 \Ra 1$:]
		$\{v_i\}$ "--- ортонормированный базис, $\Gamma = E$.
		$[\mathcal A] $ "--- диагональная.
		$[\mathcal{A}^*] = \overline{[\mathcal A]^T}$ "--- диагональная.
		Две диагональные матрицы всегда коммутируют.
		\[
			[\mathcal A\mathcal{A}^*]
			= [\mathcal A][\mathcal{A}^*]
			= [\mathcal{A}^*][\mathcal A]
			= [\mathcal{A}^*\mathcal A]
			\Ra \mathcal A\mathcal{A}^*
			= \mathcal{A}^*\mathcal A
		\]

	\item[$1 \Ra 2$:]
		$\mathcal A$ "--- нормальный оператор.
		$\lambda$ "--- собственное число $\mathcal A$.
		$U(\lambda) = U_1(\lambda)$ "--- пространство собственных векторов отвечающих $\lambda$.
		$U$ "--- $\mathcal A$-инвариантное подпространство.
		$\mathcal{A}^*$ "--- инвариантно?

		Возьмем $u \in U$ и применим оператор $\mathcal{A}^*$, нас интересует, почему $\mathcal{A}^*u \in U$.
		\[ \mathcal A(\mathcal{A}^*u) = \mathcal{A}^*(\mathcal Au) = \mathcal{A}^*(\lambda u) = \lambda \mathcal{A}^*(u) \]
		$\mathcal{A}^*u$ "--- собственный вектор $\mathcal A$, отвечающий $\lambda$.
		Тогда $\mathcal{A}^*u \in U$.

		Мы нашли подпространство, которое одновременно $\mathcal A$- и $\mathcal{A}^*$-инвариантно.
		$U^{\bot}$ "--- $\mathcal{A}^*$- и $\mathcal{A}^{**} = \mathcal A$-инвариантно.

		$V = U \oplus U^{\bot}$ и оба подпространства инвариантны $\mathcal A$ и $\mathcal{A}^*$.
		Значит определены.
		\begin{gather*}
			\mathcal A|_U, \mathcal{A}^*|_U \in End(U) \\
			\mathcal A|_{U^{\bot}}, \mathcal{A}^*|_{U^{\bot}} \in End(U^{\bot})
		\end{gather*}
		$\mathcal A|_U$ и $\mathcal{A}^*|_U$ "--- нормальные операторы на $U$.
		$\mathcal A|_{U^{\bot}}$ и $\mathcal{A}^*|_{U^{\bot}}$ "--- нормальные операторы на $U^{\bot}$.

		Выберем ортонормированный базис U.
		\begin{gather*}
			[\mathcal A|_U] = \lambda E_{\dim U} \\
			[\mathcal{A}^*|_U] = \overline{[\mathcal A|_u]^T} = \overline{\lambda}E
		\end{gather*}
		Далее: доказательство по индукции (применить к $U^{\bot}$).
	\end{description}
\end{proof}

\begin{conseq}
	\begin{enumerate}
	\item
		$\mathcal A$ "--- нормальный оператор, $u$ "--- собственный вектор $\mathcal A$, отвечающй  $\lambda$.
		Тогда $u$ "--- собственный вектор $\mathcal{A}^*$, отвечающий $\overline{\lambda}$.
	
	\item
		$\mathcal A$ "--- нормальный в унитарном пространстве.
		Cобственный вектора, отвечающие разным собственным числам, ортогональны.
		\begin{gather*}
			\mathcal Au = \lambda u \quad \mathcal Av = \beta v \\
			\beta(u, v) = (u, \overline{\beta v}) = (u, \mathcal{A}^*v) = (\mathcal Au, v) = (\lambda u, v) = \lambda (u, v) \\
			\Ra (u, v) = 0
		\end{gather*}
	
	\item
		Всякий нормальный оператор в унитарном пространстве диагонализируемы.
	\end{enumerate}
\end{conseq}



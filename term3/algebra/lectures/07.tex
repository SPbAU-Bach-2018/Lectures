\section{Нормальные операторы в евклидовых и унитарных пространствах.}

\begin{Def}
	$V$ "--- евклидово или унитарное, $(,)$ "--- стандартное скалярное произведение, $\mathscr A \in End(V)$.
	$\mathscr A$ "--- нормальный, если $\mathscr A \mathscr{A}^* = \mathscr{A}^*\mathscr A$.
\end{Def}

\begin{exmp}
	\begin{enumerate}
		\item Самосопряженный "--- нормальный.
		\item Ортогональный и унитарный операторы также нормальные.
		\begin{proof}
			\[ (\mathscr Ax, \mathscr Ay) = (x, y) \]
			Изометрия "--- обратимое преобразование.
			$y = \mathscr{A}^{-1}z$.
			\[ (x, \mathscr{A}^*z) = (\mathscr Ax, z) = (x, \mathscr{A}^{-1}z) \]
			Если $\mathscr A$ "--- унитарное или ортогональное, тогда $\mathscr{A}^* = \mathscr{A}^{-1}$.
			\[ \forall x (x, \mathscr{A}^{-1}z - \mathscr{A}^*z) = 0 \Ra \mathscr{A}^{-1} = \mathscr{A}^* \]
			$\mathscr A$ и $\mathscr{A}^{-1}$ коммутируют.
		\end{proof}
	\end{enumerate}
\end{exmp}

\begin{theorem}
	$V$ "--- унитарное пространство.
	Следующие условия равносильны:
	\begin{enumerate}
		\item $\mathscr A$ "--- нормальна.
		\item Существует ортонормированный базис $V$, в котором $[\mathscr A]$ диагональная.
	\end{enumerate}
\end{theorem}

\begin{proof}
	\begin{description}
	\item[$2 \Ra 1$:]
		$\{v_i\}$ "--- ортонормированный базис, $\Gamma = E$.
		$[\mathscr A] $ "--- диагональная.
		$[\mathscr{A}^*] = \overline{[\mathscr A]^T}$ "--- диагональная.
		\[
			[\mathscr A\mathscr{A}^*]
			= [\mathscr A][\mathscr{A}^*]
			= [\mathscr{A}^*][\mathscr A]
			= [\mathscr{A}^*\mathscr A]
			\Ra \mathscr A\mathscr{A}^*
			= \mathscr{A}^*\mathscr A
		\]

	\item[$1 \Ra 2$:]
		$\mathscr A$ "--- нормальный оператор.
		$\lambda$ "--- собственное число $\mathscr A$.
		$U(\lambda) = U_1(\lambda)$ "--- пространство собственных векторов отвечающих $\lambda$.
		$U$ "--- $\mathscr A$-инвариантное подпространство.
		$\mathscr{A}^*$ "--- инвариантно?

		Возьмем $u \in U$ и применим оператор $\mathscr{A}^*$, нас интересует, почему $\mathscr{A}^*u \in U$.
		\[ \mathscr A(\mathscr{A}^*u) = \mathscr{A}^*(\mathscr Au) = \mathscr{A}^*(\lambda u) = \lambda \mathscr{A}^*(u) \]
		$\mathscr{A}^*u$ "--- собственный вектор $\mathscr A$, отвечающий $\lambda$.
		Тогда $\mathscr{A}^*u \in U$.

		Мы нашли подпространство, которое одновременно $\mathscr A$- и $\mathscr{A}^*$-инвариантно.
		$U^{\bot}$ "--- $\mathscr{A}^*$- и $\mathscr{A}^{**} = \mathscr A$-инвариантно.

		$V = U \oplus U^{\bot}$ и оба подпространства инвариантны $\mathscr A$ и $\mathscr{A}^*$.
		Значит определены.
		\begin{gather*}
			\mathscr A|_U, \mathscr{A}^*|_U \in End(U) \\
			\mathscr A|_{U^{\bot}}, \mathscr{A}^*|_{U^{\bot}} \in End(U^{\bot})
		\end{gather*}
		$\mathscr A|_U$ и $\mathscr{A}^*|_U$ "--- нормальные операторы на $U$.
		$\mathscr A|_{U^{\bot}}$ и $\mathscr{A}^*|_{U^{\bot}}$ "--- нормальные операторы на $U^{\bot}$.

		Выберем ортонормированный базис U.
		\begin{gather*}
			[\mathscr A|_u] = \lambda E_{\dim U} \\
			[\mathscr{A}^*|_u] = \overline{[\mathscr A|_u]^T} = \overline{\lambda}E \\
		\end{gather*}
		Далее: доказательство по индукции (применить к $U^{\bot}$).
	\end{description}
\end{proof}

\begin{conseq}
	\begin{enumerate}
	\item
		$\mathscr A$ "--- нормальный оператор, $u$ "--- собственный вектор $\mathscr A$, отвечающй  $\lambda$.
		Тогда $u$ "--- собственный вектор $\mathscr{A}^*$, отвечающий $\overline{\lambda}$.
	
	\item
		$\mathscr A$ "--- нормальный в унитарном пространстве.
		Cобственный вектора, отвечающие разным собственным числам, ортогональны.
		\begin{gather*}
			\mathscr Au = \lambda u \quad \mathscr Av = \beta v \\
			\beta(u, v) = (u, \overline{\beta v}) = (u, \mathscr{A}^*v) = (\mathscr Au, v) = (\lambda u, v) = \lambda (u, v) \\
			\Ra (u, v) = 0
		\end{gather*}
	
	\item
		Всякий нормальный оператор в унитарном пространстве диагонализируемы.
	\end{enumerate}
\end{conseq}



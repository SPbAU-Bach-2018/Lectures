\section{Матрица Грама}
\begin{Def}
$\dim_k V < \infty$

$B \colon V \times V \to K$ (полуторолинейная(билинейная форма))
$v_1, \cdots, v_n$ "--- базис V.

$$\Gamma \in M(n \times n,  K)$$
$\Gamma = (B(v_i, v_j))_{i,j = 1,n}$ "--- матрица Грама формы B, отвечающая базису $v_1, \cdots, v_n$.
\end{Def}


Почему нам достаточно $\Gamma$, чтобы востановить значение билинейной формы на всем пространстве.
$$x, y \in V$$
$$x = \sum_{i = 1}^{n}x_i v_i$$
$$y = \sum_{i = 1}^{n}y_i v_i$$
$$B(x, y) = B(\sum_{i = 1}^{n}x_iv_i, \sum_{j = 1}^{n} y_{j}v_j) = $$
$$= \sum_{i = 1}^{n}\sum_{j = 1}^{n}x_i \overline{y_j} B(v_i, v_j) = $$
$$= \sum_{i = 1}^{n}\sum_{j = 1}^{n}x_i B(v_i, v_j)\overline{y_j} = $$
$$= (x_1, \cdots, x_n) B(v_i, v_j) \begin{pmatrix}\overline{y_1}\\ \vdots \\ \overline{y_n} \end{pmatrix} $$

$$B(x, y) = x^{T}\Gamma\bar y$$

\begin{Rem}
Замечание о бесконечномерном случае.
$x = \sum x_i v_i$, где почти все $x_i = 0$
$y = \sum y_i v_j$, где почти все $y_i = 0$ 

$B(x, y) = \sum_i \sum_j x_i B(v_i, v_j) \overline{y_j}$, почти все $x_i, y_j = 0$
\end{Rem}

\begin{lemma}{}
Матрица Грама эрмитово симметрична $\Lra$ форма эрмитово симметрична.
\end{lemma}
\begin{proof}
Эрмитово симметрична $B(x, y) = \overline{B(y, x)}$
$$\Gamma = B(v_i, v_j)$$
$\Gamma^T = \overline{\Gamma}$ "--- эрмитовски симметричная матрица.

Верно и обратное.
$$x \to \begin{pmatrix}
x_1\\
\vdots\\
x_n\\
\end{pmatrix}$$
$$y \to \begin{pmatrix}
y_1\\
\vdots\\
y_n\\
\end{pmatrix}$$
$$B(x, y) = x^{T}\Gamma \overline{y}$$
$$\overline{B(x, y)} = \overline{(x^T \Gamma \overline{y})} = (\overline{x^T \Gamma \overline{y}})^T =$$
$$(x^{-T}\overline{\Gamma}y)^T = y^{T}\overline{\Gamma}^{T}(\overline{x}^T)^T = y^T \Gamma \overline{x} = B(y, x)$$
$$B(x, y) = \overline{B(y, x)}$$

\end{proof}
\begin{lemma}{}
\begin{enumerate}
\item $B$ "--- эрмитово симметрична полуторолинейная $\Lra$ ее матрица Грама эрмитово симметрична.
\item $B$ "--- симметрична билинейная  форма $\Lra$ ее матрица Грама (в любом базисе) симметричная матрица.
\item $B$ "--- кососимметричная билинейная форма $\Lra$ ее матрица Грама "--- 
кососимметричная матрица с нулевой диагональю.
\end{enumerate}

\end{lemma}
\begin{proof}
1,2 смотрите выше.
3 $$\forall x,y B(x, y) = -B(y,x)$$
$$\forall x B(x, x) = 0$$
$$\Gamma^T = -\Gamma \Ra B(x, y) = -B(y, x) \forall (x, y)$$
$$B(x, y) = x^{T}\Gamma y = (x^T \Gamma y)^T = y^T\Gamma^T x = -y^T\Gamma x = -B(y, x)$$

В обратную сторону, если верно для $\forall x,y$, то, в частности, верно и 
для базисных векторов.
Если $B(x, x) = 0 \forall x$
$B(v_i, v_j) = 0 \Ra$ диагональ $\Gamma$ "--- нулевая.

$\Gamma^{T} = -\Gamma $ + нулевая диагональ.
$$B(x, x) = x^{T}\Gamma x = \sum_{i, j}x_i\Gamma_{ij}x_j = 
\sum_{i}\Gamma_{ii}x_i^2 + \sum_{i < j}(\Gamma_{ij}+ \Gamma_{ji})x_ix_j = 0$$
\end{proof}

\begin{lemma}{}
$B$ "--- невырожденная $\Lra \Gamma$ невырожденная. 
\end{lemma}
\begin{proof}
$$\forall x \ne 0 \exists y B(x, y) \ne 0$$
$$\forall x \ne 0 \exists \overline{y} x^{T}\Gamma\overline{y} \ne 0$$
$ \Lra$ для строки $x^{T}\Gamma$ найдется столбец $\overline{y}$, такой что
$x^{T}\Gamma\overline{y} \ne 0$
$$(c_1, \cdots, c_n)$$
$$\exists y c_1\overline{y_1} + \cdots + c_n\overline{y_n} \ne 0 \Lra \forall x (c_1, \cdots, c_n) \ne (0, \cdots, 0) \Lra \forall x \ne 0 x^T\Gamma \ne 0$$

Отображение из пространства строк в пространство строк.
$$x^{T} \to x^T \Gamma$$
Невырожденно $\Lra$ ядро отображения тривиально $\Lra \det \Gamma \ne 0$
$$x \to \Gamma^{T} x $$
Невырожденно $\Lra \det(\Gamma^T) \ne 0 \Lra \det(P \ne 0)$

Аналогично, $\forall \overline{y} \exists x B(x, y) \ne 0 \Lra \det(\Gamma) \ne 0$
\end{proof}
\begin{Def}
$K = \R, B$ "--- симмитричная положительно определенная.
$\forall x \ne 0, x^{T}\Gamma x > 0$
Матрица $\Gamma$ с таким свойством называется положительно определенной матрицей.
\end{Def} 
\begin{Def}
$K = \C B$ "--- полуторалинейная эрмитово симметричная положительно определенная

$$\forall x \ne 0, x^T \Gamma \overline{x} > 0$$
$$\Gamma^T = \overline{\Gamma}$$

Такая матрица грамма "--- положительно определенная эрмитова матрица. 
\end{Def}

\begin{lemma}{}
$v_1, \cdots, v_n; v_1', \cdots, v_n'$ "--- базисы V

B "--- полуторолинейная(билинейная) форма на V.

$\Gamma, \Gamma'$ "--- матрица Грама в соответствующих базисах. 

C "--- матрица переходв от $v_1, \cdots, v_n$ к  $v'_1, \cdots, v'_n$,
тогда
$$\Gamma' = C^{T}\Gamma \overline{C}$$
\end{lemma}

\begin{proof}
$x, y$ "--- столбцы координат в старом базисе. 

$x', y'$ "--- столбцы координат в новом базисе.
$$u = \sum x_i v_i = \sum x_i' v_i'$$
$$v = \sum y_i v_i = \sum y_i' v_i'$$
$$x = C x', y = Cy'$$
$$x'^{T}C^T\Gamma\overline{C}\overline{y} = x^T\Gamma \overline{y} = B(u, v) = x'^{T}\Gamma'\overline{y'} $$

$\forall$ столбцов $x', y'$
$$x'^T \Gamma'\overline{y'} = x'^TC^T \Gamma \overline{C}\overline{y'}$$
$$\Gamma' = C^T\Gamma\overline{C}$$
$$\forall i, j (\Gamma')_{ij} = (C^{T}\Gamma \overline{C})_{ij} \Ra \Gamma = C^{T}\Gamma\overline{C}$$
\end{proof}
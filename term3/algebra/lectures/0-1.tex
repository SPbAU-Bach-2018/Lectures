\section{Матрица Грама}

\begin{Def}
	$\dim_K V < \infty$.
	$B \colon V \times V \to K$ "--- полуторолинейная(билинейная) форма.
	$v_1, \dots, v_n$ "--- базис $V$.
	\begin{gather*}
		\Gamma \in M(n \times n, K) \\
		\Gamma = (B(v_i, v_j))_{i,j=1..n}
	\end{gather*}
	$\Gamma$ "--- матрица Грама формы $B$, отвечающая базису $v_1, \cdots, v_n$.
\end{Def}


Почему нам достаточно $\Gamma$, чтобы востановить значение билинейной формы на всем пространстве.
\begin{gather*}
	x, y \in V \\
	x = \sum_{i=1}^n x_i v_i \quad y = \sum_{i=1}^n y_i v_i \\
	B(x, y)
	= B \left( \sum_{i=1}^n x_iv_i, \sum_{j=1}^n y_jv_j \right)
	= \sum_{i=1}^n \sum_{j=1}^n x_i \bar y_j B(v_i, v_j) = \\
	= \sum_{i=1}^n \sum_{j=1}^n x_i B(v_i, v_j) \bar y_j
	= \begin{pmatrix} x_1 & \cdots & x_n \end{pmatrix} B(v_i, v_j) \begin{pmatrix}\bar y_1 \\ \vdots \\ \bar y_n \end{pmatrix} \\
	B(x, y) = x^T\Gamma\bar y
\end{gather*}

\begin{Rem}[Замечание о бесконечномерном случае]
	$x = \sum x_i v_i$, где почти все $x_i = 0$.
	$y = \sum y_i v_j$, где почти все $y_i = 0$.
	\[ B(x, y) = \sum_i \sum_j x_i B(v_i, v_j) \bar y_j \]
	почти все $x_i, y_j = 0$.
\end{Rem}

\begin{lemma}
	Матрица Грама эрмитово симметрична $\Lra$ форма эрмитово симметрична.
\end{lemma}
\begin{proof}
	Эрмитово симметрична $B(x, y) = \overline{B(y, x)}$.
	\[ \Gamma = B(v_i, v_j) \]
	$\Gamma^T = \overline{\Gamma}$ "--- эрмитово симметричная матрица.

	Верно и обратное.
	\begin{gather*}
		x \mapsto \begin{pmatrix}
			x_1 \\
			\vdots \\
			x_n
		\end{pmatrix} \quad
		y \mapsto \begin{pmatrix}
			y_1 \\
			\vdots \\
			y_n
		\end{pmatrix} \\
		B(x, y) = x^T\Gamma \bar y \\
		\overline{B(x, y)}
		= \overline{(x^T \Gamma \bar y)}
		= \left( \overline{x^T \Gamma \overline{y}} \right)^T = \\
		= \left( \bar x^T \bar \Gamma \bar y \right)^T
		= y^T \bar \Gamma^T (\bar x^T)^T = y^T \Gamma \bar x
		= B(y, x) \\
		B(x, y) = \overline{B(y, x)}
	\end{gather*}
\end{proof}
\begin{lemma}\hfill
	\begin{enumerate}
		\item $B$ "--- эрмитово симметрична полуторолинейная $\Lra$ ее матрица Грама эрмитово симметрична.
		\item $B$ "--- симметрична билинейная  форма $\Lra$ ее матрица Грама (в любом базисе) симметричная матрица.
		\item $B$ "--- кососимметричная билинейная форма $\Lra$ ее матрица Грама "--- кососимметричная матрица с нулевой диагональю.
	\end{enumerate}
\end{lemma}
\begin{proof}
	Пункты 1 и 2 смотрите выше.
	Докажем пункт 3:
	\begin{gather*}
		\forall x,y, B(x, y) = -B(y,x) \\
		\forall x B(x, x) = 0 \\
		\Gamma^T = -\Gamma \Ra B(x, y) = -B(y, x) \forall (x, y) \\
		B(x, y) = x^T \Gamma y = (x^T \Gamma y)^T = y^T\Gamma^T x = -y^T\Gamma x = -B(y, x)
	\end{gather*}

	В обратную сторону, если верно для $\forall x,y$,
	то, в частности, верно и для базисных векторов.
	Если $B(x, x) = 0 \forall x$, $B(v_i, v_j) = 0 \Ra$ диагональ $\Gamma$ "--- нулевая.

	$\Gamma^{T} = -\Gamma $ + нулевая диагональ.
	\[
		B(x, x)
		= x^T \Gamma x
		= \sum_{i, j} x_i\Gamma_{ij}x_j
		= \sum_i\Gamma_{ii}x_i^2 + \sum_{i<j} (\Gamma_{ij}+ \Gamma_{ji})x_ix_j = 0
	\]
\end{proof}

\begin{lemma}
	$B$ "--- невырожденная $\Lra \Gamma$ невырожденная.
\end{lemma}
\begin{proof}
	\begin{gather*}
		\forall x \ne 0, \exists y\colon B(x, y) \ne 0 \\
		\forall x \ne 0, \exists \bar y\colon x^T\Gamma\bar y \ne 0
	\end{gather*}
	$\Lra$ для строки $x^{T}\Gamma$ найдется столбец $\bar y$, такой что
	$x^T \Gamma \bar y \ne 0$.
	\begin{gather*}
		(c_1, \dots, c_n) \\
		\exists y c_1\bar y_1 + \dots + c_n \bar y_n \ne 0
		\Lra \forall x, (c_1, \dots, c_n) \ne (0, \dots, 0)
		\Lra \forall x \ne 0, x^T\Gamma \ne 0
	\end{gather*}
	Отображение из пространства строк в пространство строк.
	\[ x^{T} \to x^T \Gamma \]
	невырожденно $\Lra$ ядро отображения тривиально $\Lra \det \Gamma \ne 0$.
	\[ x \to \Gamma^{T} x \]
	Невырожденно $\Lra \det(\Gamma^T) \ne 0 \Lra \det(P \ne 0)$.

	Аналогично,
	\[ \forall \bar y, \exists x\colon B(x, y) \ne 0 \Lra \det \Gamma \ne 0 \]
\end{proof}

\begin{Def}
	$K = \R$, $B$ "--- симмитричная положительно определенная.
	\[ \forall x \ne 0, x^{T}\Gamma x > 0 \]
	Матрица $\Gamma$ с таким свойством называется положительно определенной матрицей.
\end{Def}
\begin{Def}
	$K = \C$, $B$ "--- полуторалинейная, эрмитово симметричная, положительно определенная.
	\begin{gather*}
		\forall x \ne 0, x^T \Gamma \bar x > 0 \\
		\Gamma^T = \bar \Gamma
	\end{gather*}
	Такая матрица Грамма "--- положительно определенная эрмитова матрица.
\end{Def}

\begin{lemma}
	$v_1, \dots, v_n$, $v_1', \dots, v_n'$ "--- базисы $V$.
	$B$ "--- полуторолинейная (билинейная) форма на $V$.
	$\Gamma, \Gamma'$ "--- матрицы Грама в соответствующих базисах.
	$C$ "--- матрица перехода от $v_1, \dots, v_n$ к  $v'_1, \dots, v'_n$.
	Тогда
	\[ \Gamma' = C^T \Gamma \bar C \]
\end{lemma}

\begin{proof}
	$x$, $y$ "--- столбцы координат в старом базисе.
	$x'$, $y'$ "--- столбцы координат в новом базисе.
	\begin{gather*}
		u = \sum x_i v_i = \sum x_i' v_i' \\
		v = \sum y_i v_i = \sum y_i' v_i' \\
		x = C x' \quad y = Cy' \\
		x'^T C^T\Gamma\bar C\bar y = x^T\Gamma \bar y = B(u, v) = x'^T \Gamma'\bar y'
	\end{gather*}
	Для всех столбцов $x'$, $y'$:
	\begin{gather*}
		x'^T \Gamma' \bar y' = x'^TC^T \Gamma \bar C\bar y' \\
		\Gamma' = C^T\Gamma\bar C \\
		\forall i, j, (\Gamma')_{ij} = \left(C^T\Gamma \bar C\right)_{ij} \Ra \Gamma = C^T\Gamma\bar C
	\end{gather*}
\end{proof}

\section{Некоторые стандартные изоморфизмы для тензорных произведений.}
В этом параграфе все векторные пространства конечномерные.

\begin{theorem}{Ассоциативность}\hfill
\begin{enumerate}
    \item $(V_1 \otimes V_2) \otimes V_3 \cong V_1 \otimes V_2 \otimes V_3$
    \item Более общий случай, для любой правильной расстановки скобок. 
    $$(V_1 \otimes V_2)(\cdots \otimes V_n) \cong  V_1 \otimes \cdots \otimes V_n$$
\end{enumerate}
\end{theorem}
\begin{proof}
Для любых пространств ограничемся первым случаем. 
$$
\begin{matrix}
V_1 \times V_2 \times V_3& \overset{\phi}{\rightarrow}& V_1 \otimes V_2 \otimes V_3\\
&\overset{\alpha}{\searrow}&\overset{\beta}{\downarrow}\\
&&(V_1 \otimes V_2) \otimes V_3\\
\end{matrix}
$$

$\alpha = \beta \circ \phi$ \\
Необходимо показать, что $\alpha$ полилинейное отображение.\\ 

$$V_1 \times V_2 \to V_1 \otimes V_2$$
$$(v_1, v_2) \to v_1 \otimes v_2$$
Данное отоброжение билинейное. 
$$\alpha \colon V_1 \times V_2 \times V_3 \to (V_1 \otimes V_2) \otimes V_3$$
$$(v_1, v_2, v_3) \to  (v_1 \otimes v_2) \otimes v_3$$
Значит отображение $\alpha$ полилинейное.

Из этого следует, что существует единственное полилинейное отображение $\beta$, которое по самой
конструкции  действует следующим образом:
$$\beta(v_1 \otimes v_2 \otimes v_3) = \alpha(v_1, v_2, v_3) = (v_1 \otimes v_2) \otimes v_3$$

Теперь выберем базис у двух конструкций. Проверим, что базис переходит 
в базис и из этого будет следовать, что наше отображение изоморфизм.\\ 
$\{e_i\}$ "--- базис $V_1$\\             
$\{f_j\}$ "--- базис $V_2$\\
$\{h_k\}$ "--- базис $V_3$\\
$$\beta(e_i \otimes f_j \otimes h_k) = (e_i \otimes f_j) \otimes h_k$$
$\beta$ переводит базис в базис $\Ra$ изоморфизм.

Второй способ:

$(e_i \otimes f_j) \otimes h_k$ "--- базис $(V_1 \otimes V_2) \otimes V_3$
и лежит в образе $\beta \Ra \beta$ сюръективно.
\begin{gather*}
\dim(V_1 \otimes V_2 \otimes V_3) = \dim V_1 \cdot \dim V_2 \cdot \dim V_3 =\\
=\dim(V_1 \otimes V_2) \cdot \dim V_3 = \dim((V_1 \otimes V_2) \otimes V_3)
\end{gather*}
$\Ra \beta$ "--- инъективно.
$\Ra$так как $\beta$ биекция и полилинейно, значит $\beta$ изоморфизм.
\end{proof}

\begin{theorem}{Коммутативность} \hfill
$V_1, \cdots, V_n$ "--- векторные пространства над полем $K$\\
$\sigma \in S_n$ "--- любая перестановка чисел от 1 до $n$\\
Существует изоморфизм $f_{\sigma}\colon V_1 \otimes \cdots \otimes V_n \to V_{\sigma(1)} \otimes \cdots V_{\sigma(n)}$
и при этом  $f_{\sigma \tau} = f_{\sigma} \circ f_{\tau}$
\end{theorem}
\begin{proof}
$$
\begin{matrix}
V_1 \times \cdots \times V_n& \overset{\phi}{\rightarrow}& V_1 \otimes \cdots \otimes V_n\\
&\overset{\alpha}{\searrow}&\overset{\beta}{\downarrow}\\
&&V_{\sigma(1)} \otimes \cdots \otimes V_{\sigma(n)}\\
\end{matrix}
$$

$\alpha$ "--- полилинейная \\
$\beta(v_1 \otimes \cdots \otimes v_n) = v_{\sigma(1)} \otimes \cdots \otimes v_{\sigma(n)}$\\
$\beta$ переводит тензорный базис в тензорный базис $\Ra$ изоморфизм.\\

$$f_{\sigma \tau}(v_1 \otimes \cdots \otimes v_n) = v_{\sigma\tau(1)} \otimes \cdots \otimes v_{\sigma \tau(n)}$$
$$f_{\tau}(v_1 \otimes \cdots \otimes v_n) = v_{\tau(1)} \otimes \cdots \otimes v_{\tau(n)} $$
$$f_{\sigma}\cdot f_{\tau}(v_1 \otimes \cdots \otimes v_n) = f_{\sigma}(v_{\tau(1) \otimes \cdots \otimes v_{\tau(n)}}) =$$
$$= v_{\sigma\tau(1)} \otimes \cdots \otimes v_{\sigma \tau(n)}$$
\end{proof}

\begin{theorem}{Двойственность}\hfill
$$\dim V_i < \infty \Ra V_1^* \otimes \cdots \otimes V_n^* \cong (V_1 \otimes \cdots \otimes V_n)^* $$ 
\end{theorem}
\begin{proof}
То, что справа, изоморфно $\mathcal L(V_1, \dots, V_n, K)$.
Давайте сначала построим $\alpha$:
$$
\begin{matrix}
V_1^* \times \cdots \times V_n^* & \overset{\phi}{\rightarrow}& V_1^* \otimes \cdots \otimes V_n^*\\
&\overset{\alpha}{\searrow}&\overset{\beta}{\downarrow}\\
&&\mathcal{L}(V_1, \cdots, V_n, K) \\
\end{matrix}
$$

$\alpha$ будет отображать набор $v_i$ в следующий функционал:
$$g_{v_1^* \cdots v_n^*} \colon (u_1, \cdots, u_n) \to v_1^*(u_1) \cdot \cdots \cdot v_n^*(u_n), u_i \in V_i$$
При фиксации всех элементов в кортеже, кроме одного получается линейная функция, 
значит $g_{v_1^* \cdots v_n^*} \in \mathcal{L}(v_1, \cdots, V_n, K)$\\
$$\alpha \colon V_1^* \times \dots \times V_n^* \to \mathcal{L}(V_1, \cdots, V_n, K) $$
$$\alpha(v_1^*, \cdots, v_n^*) = g_{v_1^* \cdots v_n^*} $$
$\alpha$ "--- каноническое отображение из $V_1^* \times \cdots \times V_n^*$, $\alpha$ является полилинейным.
Тогда можем достроить линейное отображение $\beta$ (по универсальности тензорного произведения).
\begin{gather*}
	\beta: V_1^* \otimes \cdots \otimes V_n^* \ra \mathcal{L}(V_1, \cdots, V_n, K) \cong \mathcal{L}(V_1 \otimes \cdots \otimes V_n, K) = (V_1 \otimes \cdots \otimes V_n)^* \\
	\beta(v^*_1 \otimes \cdots \otimes v^*_n)(u_1, \cdots, u_n) = v_1^*(u_1) \cdots v_n^*(u_n)
\end{gather*}
  
Осталось проверить, что $\beta$ изоморфизм, то есть биективность $\beta$.
Так как мы ограничиваемся только конечномерным случаем, то достаточно
проверить только сюръективность и убедится, что размерности в обоих частях 
совпадают. 
\begin{gather*}
        \dim (V_1^* \otimes \cdots \otimes V_n^*) = \Pi(\dim V_i^*) = \Pi (\dim V_i)\\
        \dim (V_1 \otimes \cdots \otimes V_n)^* = \dim (V_1 \otimes \cdots \otimes V_n) = \Pi (\dim V_i)
\end{gather*}

Теперь проверим сюръективность.	
    \begin{gather*}
        f \in \mathcal{L}(V_1, \cdots, V_n, K) \\
        \text{$\{e_i^{(j)}\}$ "--- базис $V_i$} \Ra \\
        \text{$\{e_i^{(j)*}\}$ "--- базис $V_i^*$}\Ra\\
        e_{i_1}^{(1)*} \cdot \cdots \cdot e_{i_n}^{(n)*} \text{"--- базис $\mathcal{L}(V_1, \cdots, V_n, K)$}\\ 
        f(.,\cdots,.) = \sum f(e_{i_1}, \cdots, e_{i_n})e_{i_1}^{(1)*}(.)\cdot \cdots \cdot e_{i_n}^{(n*)}(.)\\
        \beta(e_{i_1}^{(i)*} \otimes \cdots \otimes e_{i_n}^{(n)*}) = e_{i_1}^{(1)*} \cdot \cdots \cdot e_{i_n}^{(n)*} \\
        \Ra \mathcal{L}(V_1, \cdots, V_n, K) \subseteq \Im \beta
    \end{gather*}
    То есть образующие лежат в образе, значит в образе лежит все пространство,
    значит, отображение сюръективно. 

 Значит отображение $\beta$ изоморфизм, что и требовалось.
\end{proof}


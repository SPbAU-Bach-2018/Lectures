\section{Некоторые стандартные изоморфизмы для тензорных произведений.}
В этом параграфе все векторные пространства конечномерные.

\begin{theorem}{Ассоциативность}
\begin{enumerate}
    \item $(V_1 \otimes V_2) \otimes V_3 \cong V_1 \otimes V_2 \otimes V_3$
    \item Более общий случай, для любой правильной расстановки скобок. 
    $$(V_1 \otimes V_2)(\cdots \otimes V_n) \cong  V_1 \otimes \cdots \otimes V_n$$
\end{enumerate}
\end{theorem}
\begin{proof}
Для любых пространств ограничемся первым случаем. 
$$\phi \colon V_1 \times V_2 \times V_3 \to V_1 \otimes V_2 \otimes V_3$$
$$\alpha \colon V_1 \times V_2 \times V_3 \to (V_1 \otimes V_2) \otimes V_3$$
$$\beta\colon V_1 \otimes V_2 \otimes V_3 \to (V_1 \otimes V_2)\otimes V_3$$
$\beta$ "--- линейное отображение. $\alpha = \beta \circ \phi$

$$V_1 \times V_2 \to V_1 \otimes V_2$$
$$(v_1, v_2) \to V_1 \otimes V_2$$
$$\alpha \colon V_1 \times V_2 \times V_3 \to (V_1 \otimes V_2) \otimes V_3$$
$$(v_1, v_2, v_3) \to  (V_1 \otimes V_2) \otimes V_3$$
$$\beta(V_1 \otimes V_2 \otimes V_3) = \alpha(V_1, V_2, V_3) = (V_1 \otimes V_2) \otimes V_3$$

$\{e_i\}$ "--- базис $V_1$\\
$\{f_j\}$ "--- базис $V_2$\\
$\{h_k\}$ "--- базис $V_3$\\
$$\beta(e_i \otimes f_j \otimes h_k) = (e_i \otimes f_j) \otimes h_k$$
$\beta$ переводит базис в базис $\Ra$ изоморфизм.

Второй способ:

$(e_i \otimes f_j) \otimes h_k$ "--- базис $(V_1 \otimes V_2) \otimes V_3$
и лежит в базисе $\beta \Ra \beta$ сюръективно.
\begin{gather*}
\dim(V_1 \otimes V_2 \otimes V_3) = \dim V_1 \cdot \dim V_2 \cdot \dim V_3 =\\
=\dim(V_1 \otimes V_2) \cdot \dim V_3 = \dim((V_1 \otimes V_2) \otimes V_3)
\end{gather*}
$\Ra \beta$ "--- инъективно. 
\end{proof}
\begin{theorem}
$V_1, \cdots, V_n$\\
$\sigma \in S_n$\\
Существует изоморфизм $f_{\sigma}\colon V_1 \otimes \cdots \otimes V_n \to V_{\sigma(1)} \otimes \cdots V_{\sigma(n)}$
и при этом  $f_{\sigma \tau} = f_{\sigma} \circ f_{\tau}$
\end{theorem}
\begin{proof}
$$\phi \colon V_1 \times \cdots V_n \to V_1 \otimes \cdots \otimes V_n$$
$$\alpha \colon V_1 \times \cdots \times V_n \to V_{\sigma(1)} \otimes \cdots \otimes V_{\sigma(n)}$$
$$\beta \colon V_1 \otimes \cdots \otimes V_n \to V_{\sigma(1)} \otimes \cdots \otimes V_{\sigma(n)}$$

$\alpha$ "--- полилинейная \\
$\beta(V_1 \otimes \cdots \otimes V_n) = V_{\sigma(1)} \otimes \cdots \otimes V_{\sigma(n)}$\\
$\beta$ переводит тензорный базис в тензорный базис $\Ra$ изоморфизм.
\end{proof}

$$f_{\sigma \tau}(V_1 \otimes \cdots \otimes V_n) = V_{\sigma\tau(1)} \otimes \cdots \otimes V_{\sigma \tau(n)}$$
$$f_{\tau}(V_1 \otimes \cdots \otimes V_n) = V_{\tau(1)} \otimes \cdots \otimes V_{\tau(n)} $$
$$f_{\sigma}\cdot f_{\tau}(V_1 \otimes \cdots \otimes V_n) = f_{\sigma}(V_{\tau(1) \otimes \cdots \otimes V_{\tau(n)}}) =$$
$$= V_{\sigma\tau(1)} \otimes \cdots \otimes V_{\sigma \tau(n)}$$

\begin{theorem}{двойственность}

$$(V_1 \otimes \cdots \otimes V_n)^* \cong V_1^* \times \cdots \times V_n^*$$

\end{theorem}
\begin{proof}
$$g_{v_1^* \cdots v_n^*} \colon (u_1, \cdots, u_n) \to v_1^*(u_1) \cdot \cdots \cdot V_n^*(u_n)$$
$$u_i \in V_i$$
$$g_{v_1^* \cdots v_n^*} \in \mathcal{L}(v_1, \cdots, V_n, K) $$
$$\alpha \colon V_1^* \times V_n^* \to \mathcal{L}(V_1, \cdots, V_n, K) $$
$$\alpha(V_1^*, \cdots, V_n^*) = g_{v_1^* \cdots v_n^*} $$
\end{proof}


\section{Вторая теорема о гомоморфизме}
$G$ "--- группа, $H \unlhd G, K \le G$\\
$HK = KH \le G$\\
$H = H1 = 1H \le HK, \le KH$\\

$H \le HK \le G$\\
$\forall g \in G, gHg^{-1} \in H$\\
$\forall g \in HK, gHg^{-1} = H$\\
$H \unlhd KH$\\
\begin{theorem}
$H \unlhd G, K \le G \Ra H \le KH$ и $KH\setminus H \cong K \setminus (K \cap H)$\\
\end{theorem}
\begin{proof}
$\psi \colon K \to KH \setminus H$\\
$\psi(k) = kH$\\
$\psi(k_1)\psi(k_2) = k_1Hk_2H = k_1k_2H = \psi(k_1k_2)$\\

Класс смежности в $KH$ по $H$.\\
$khH = k(hH) = kH$\\
$khH = \psi(k), \psi$ "--- сюръективный гомоморфизм.\\

$k \in ker \psi$\\
$\psi(k) = 1_{KH \setminus H} = H$\\
$kH = H \Ra k \in H$\\
$k \in K$\\
$\Ra k \in  K \cap H$\\

$\ker \psi \subset K \cap H$\\
$K \cap H, \psi(k) = kH = H1 \Ra k \in \ker \psi$\\
$\Ra k \cap H \subset \ker \psi$\\
$\ker \psi = K \cap H$\\

$K \setminus K \cap H = K \setminus \ker \psi \cong \psi(K) = KH\setminus H$
\end{proof}

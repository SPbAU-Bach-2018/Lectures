\section{Внешнее прямое произведение}
$H_1, \cdots, H_n$ "--- группы.\\
$G = H_1 \times \cdots \times H_n$ (как множество).
Зададим умножение на $G$.
$(h_1,\cdots, h_n)(g_1, \cdots, g_n) = (h_1g_1, h_2g_2, \cdots, h_ng_n)$\\
Упражнение $G$ с такой операцией "--- это группа.
$1_G = (1_{H_1}, \cdots, 1_{H_n})$\\
$(h_1, \cdots, h_n)^{-1} = (h_1^{-1}, \cdots, h_n^{-1})$\\

\begin{Def}
 $G$ с так введенной операцией называется прямым произведением групп $H_1, \cdots, H_n$\\
 $H_1 \times \cdots \times H_n$\\
 $\sigma \colon H_i \to G$\\
 $h_i \to (1, \cdots,  h, \cdots, 1)$\\
 $\sigma$ "--- инъективный гомоморфизм из $H_i$ в $G$\\
 $\widetilde{H_i} = \sigma_i(H_i)$
 $(1, \cdot, H_i, 1, \cdots, 1)$\\
\end{Def}
\begin{conseq}
\begin{enumerate}
    \item $i \ne j \Ra \widetilde{H_i}, \widetilde{H_j}$ коммутируют поэлементно\\
    \item $\widetilde{H_i} \unlhd G$\\
    \item $\widetilde{H_i} \cap (\widetilde{H_1}, \cdots \widetilde{H_{i - 1}} \widetilde{H_{i + 1}} \cdots \widetilde{H_n}) = \{1_G\}$
\end{enumerate}
\end{conseq}
\section{Внешнее прямое произведение}
$H_1, \cdots, H_n$ "--- группы.\\
$G = H_1 \times \cdots \times H_n$ (как множество).
Зададим умножение на $G$.
$(h_1,\cdots, h_n)(g_1, \cdots, g_n) = (h_1g_1, h_2g_2, \cdots, h_ng_n)$\\
Упражнение $G$ с такой операцией "--- это группа.
$1_G = (1_{H_1}, \cdots, 1_{H_n})$\\
$(h_1, \cdots, h_n)^{-1} = (h_1^{-1}, \cdots, h_n^{-1})$\\

\begin{Def}
 $G$ с так введенной операцией называется прямым произведением групп $H_1, \cdots, H_n$\\
 $H_1 \times \cdots \times H_n$\\
 $\sigma \colon H_i \to G$\\
 $h_i \to (1, \cdots,  h, \cdots, 1)$\\
 $\sigma$ "--- инъективный гомоморфизм из $H_i$ в $G$\\
 $\widetilde{H_i} = \sigma_i(H_i) = $
 $(1, \cdot, H_i, 1, \cdots, 1)$\\
\end{Def}
\begin{conseq}
\begin{enumerate}
    \item $i \ne j \Ra \widetilde{H_i}, \widetilde{H_j}$ коммутируют поэлементно\\
    \begin{proof}
    $(1, \cdots, h_i, 1, \cdots)(1, \cdots, h_j, \cdots, 1) = (1, \cdots, h_i, 1 \cdots, h_j, \cdots, 1)$\\
    $(1, \cdots, h_j, 1, \cdots)(1, \cdots, h_i, \cdots, 1) = (1, \cdots, h_i, 1 \cdots, h_j, \cdots, 1)$\\

    В частности, раз они коммутируют поэлментно, то $\widetilde{H_i}\widetilde{H_j} = \widetilde{H_j}\widetilde{H_i}$
    \end{proof}
    \item $\widetilde{H_i} \unlhd G$\\
    \begin{proof}
    $g \in G$\\
    $g = (h_1, \cdots, h_n)$\\
    $f \in \widetilde{H_i} $\\
    $f = (1, \cdots, f_i,\cdots, 1) \in \widetilde{H_i}$\\
    $gfg^{-1} = (h_11h_1^{-1}, \cdots, h_if_ih_i^{-1}, \cdots, h_n1h_n^{-1}) = (1, \cdots, h_if_ih_i^{-1}, \cdots, 1)$\\
    $h_if_ih_i^{-1} \in H_i$\\
    Это доказывает нормальность. 
    \end{proof}
    \item $\widetilde{H_i} \cap (\widetilde{H_1} \cdots \widetilde{H_{i - 1}} \widetilde{H_{i + 1}} \cdots \widetilde{H_n}) = \{1_G\}$
    \begin{proof}
    $g = (1, \cdots, h_i, 1, \cdots, 1) \in \widetilde{H_i}$\\
    $g = (h_1, \cdots, h_{i - 1}, 1, \cdots, h_n) \in (\widetilde{H_1} \cdots \widetilde{H_{i - 1}} \widetilde{H_{i + 1}} \cdots \widetilde{H_n})$\\
    Если g попал в обе группы, то это возможно только, когда $h_i = 1_H$, то есть пересечение тривиально. 
    \end{proof}
\end{enumerate}
\end{conseq}
\begin{Rem}
Справедлив следующий результат. 
$G, H_1, \cdots, H_n \le G$\\
Причем   
\begin{enumerate}
\item $\forall i \ne j H_i$ и $H_j$ поэлментно коммутируют. 
\item $\forall i \colon H_i \unlhd G$\\
\item $\forall i \colon H_i \cap (H_1 \cdots H_{i - 1}H_{i + 1} \cdots H_n) = \{1_G\}$\\
\item $H_1 \cdots H_n = G$\\
\end{enumerate}
Тогда $G \cong H_1 \times \cdots \times H_n$

Упражнение на практику. 
\end{Rem}

\begin{exmp}
Пример использования замечания.\\
Рассмотрим цикличискую группу $(\Z / 6 \Z)^{+}$  порядка 6.
${\bar 0, \bar 1, \bar 2, \bar 3, \bar 4, \bar 5}$\\
$H_1 = \{\bar 0, \bar 3\} \cong (\Z/2\Z)^{+}$\\
$H_2 = \{\bar 0, \bar 2, \bar 4\} \cong (\Z/3\Z)^{+}$\\
Это сумма по Минковскому, так как $H_1$ и $H_2$ подгруппы $G$ и операция была записана аддетивно. 
$G = H_1 + H_2$ \\
$\Ra (\Z / 6\Z)^{+} \cong (\Z/2\Z)^{+} \times (\Z/3\Z)^{+}$

Более того, можем обобщить для любых взаимно простых n и m\\
$(\Z/mn\Z)^{+} \cong (\Z/m\Z)^{+} \times (\Z/n\Z)^{+}$
\end{exmp}


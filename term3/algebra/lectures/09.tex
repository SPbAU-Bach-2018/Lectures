\section{Канонический вид унитарного оператора.}

$U^{-1} = U^* (\Lra [U^{T}][\overline{U}] = E)$  

Унитарный оператор "--- нормальный. 

Существует ортонормированный базис состоящий из собственных векторов, в котором $[U]$ "--- диагональная(на диагонали собственные числа). 

$[U]^T[\overline{U}] = E$

$$U = \begin{pmatrix}
\lambda_1 & \cdot & \cdot \\
\cdot & \cdot &\cdot\\
\cdot & \cdot &\lambda_n\\
\end{pmatrix}$$

$\lambda_i \overline{\lambda_i} = 1 \Lra |\lambda_i|^2 = 1$

\begin{theorem}{}
 Собственные числа унитарного оператора по модулю равны 1 и 
 существует базис из собственных векторов, в котором матрица $[U]$ диагональная и 
 на диагонали собственные числа $U$.
\end{theorem} 

\begin{proof}
$U$ "--- унитарный $\Ra$ существует ортонормированный базис $[U]^T[\overline{U}]$ "--- диагональные и диагональные элементы 
по модулю равны 1.

$$[U]^T[\overline{U}] = E$$
$$\lambda_i \overline{\lambda_i} = 1$$

U унитарный оператор $\Lra [U]^T\Gamma[\overline{U}] = \Gamma$
\end{proof}

Переформулируем на языке матриц. 

$U$ "--- унитарная матрица, то есть $U^T\overline{U} \equiv E$ 
$\Ra \exists V$ "--- унитарная $V^*UV = V^{-1}UV$ "--- диагональная матрица и на диагонали собственные числа по модулю 1.



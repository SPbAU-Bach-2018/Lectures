\section{Канонический вид унитарного оператора.}

\[ U^{-1} = U^* (\Lra [U^T][\bar U] = E) \]
Унитарный оператор "--- нормальный.

Существует ортонормированный базис состоящий из собственных векторов, в котором $[U]$ "--- диагональная (на диагонали собственные числа).
\begin{gather*}
	\,[U]^T[\bar U] = E \\
	U = \begin{pmatrix}
		\lambda_1 & \cdots & 0 \\
		\vdots & \ddots & \vdots \\
		0 & \cdots & \lambda_n
	\end{pmatrix}
\end{gather*}
$\lambda_i \bar \lambda_i = 1 \Lra |\lambda_i|^2 = 1$.

\begin{theorem}
	Собственные числа унитарного оператора по модулю равны 1 и
	существует базис из собственных векторов, в котором матрица $[U]$ диагональная и
	на диагонали собственные числа $U$.
\end{theorem}

\begin{proof}
	$U$ "--- унитарный, значит существует ортонормированный базис, в котором $[U]^T[\overline{U}]$ "--- диагональные и диагональные элементы
	по модулю равны 1.
	\begin{gather*}
		[U]^T[\overline{U}] = E \\
		\lambda_i \overline{\lambda_i} = 1
	\end{gather*}
	$U$ унитарный оператор тогда и только тогда, когда $[U]^T\Gamma[\bar U] = \Gamma$.
\end{proof}

Переформулируем на языке матриц.

$U$ "--- унитарная матрица, то есть $U^T\overline{U} \equiv E$
$\Ra \exists V$ "--- унитарная $V^*UV = V^{-1}UV$ "--- диагональная матрица и на диагонали собственные числа по модулю 1.

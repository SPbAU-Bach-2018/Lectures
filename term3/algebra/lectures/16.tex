\section{Тензорное произведение пространств}

$V_1, \cdots, V_n, W$ "--- векторное пространство над K.

Полилинейные отображения $V_1 \times \cdots \times V_n$ в $W$. 
то есть отображение линейное по каждому из аргументов. 

Построим новое пространство (тензорное пространство $V_1, \cdots, V_n$ $V_1 \otimes \cdots \otimes V_n$)  
и каноническое отображение из $V_1 \times \cdots V_n$ в $V_1 \otimes \cdots \otimes V_n$

\begin{description}
    \item[Шаг 1:]
    Построим $V_1 \otimes \cdots \otimes V_n$

    $M$

    $V_1 \times \cdots \times V_n = \{(v_1, \cdots, v_n)| v_i \in V_i\}$

    $M$ "--- пространство всех финитных отображений из $V_1 \times \cdot \times V_n$ в $K$.
    (каждое отображение почти всюду равно 0).

    $$f, g \in M$$
    $$(f + g)(v_1, \cdots, v_n) = f(v_1, \cdots, v_n) + g(v_1, \cdots, v_n)$$
    $$(\alpha f)(v_1 \cdots v_n) = \alpha f(v_1, \cdots, v_n)$$

    Упражнение: $M$ "--- векторное пространство над $K$.

    $\delta_{(v_1,\cdots, v_n)}(u_1, \cdots, u_n) = \left\{
    \begin{aligned}
    1, u_1 = v_1, \cdots, u_n = v_n
    0, 
    \end{aligned}
    \right.$

    $f = \sum_{(v_1, \cdots, v_n)} f(v_1, \cdots, v_n)\delta_{(v_1, \cdots, v_n)}$ 
    (так как f финитная, то сумма конечна). 

    $\delta$ "--- функция порождает $M$.

    $\sum \alpha_{(v_1, \cdots, v_n)} \delta_{(v_1, \cdots, v_n)} = 0 \Lra$  все $\alpha_{(v_1, \cdots, v_n)} = 0$ 

    $\alpha_{v_1, \cdots, v_n} \in K \Ra \delta$-функция линейно независима. 

    $\delta$-функция "--- базис $M$.
    Если поле $K$ бесконечно и хотя бы одно из $V_i$ не нульмерно, то 
    $M$ "--- бесконечномерное. 
    
    Далее символ $\delta$ будем опускать $\sum \alpha_{v_1, \cdots, v_n}(v_1, \cdots, v_n)$ "--- такие формальные суммы образуют $M$.

    Формальные финитные линейные комбинации элементов $V_1 \times \cdots \times V_n$.

    $$\sum \alpha_{v_1, \cdots, v_n}(v_1, \cdots, v_n) = \sum \beta_{v_1, \cdots, v_n}(v_1, \cdots, v_n)$$ 
    $$\Lra \forall v_1, \cdots, v_n \alpha_{(v_1, \cdots,  v_n)} = \beta_{(v_1, \cdots, v_n)}$$

    $$M_0 = <(v_1, \cdots, v_i + v_i', \cdots, v_n) - (v_1, \cdots, v_i, \cdots, v_n) - (v_1, \cdots, v_i', \cdots, v_n)$$
    $$       (v_1, \cdots, \alpha v_i, \cdots, v_n) - \alpha(v_1,\cdots, v_i, \cdots, v_n)\colon  $$
    $$ i = 1, \cdots, n \colon v_i v_i' \in V_i, \alpha \in K>$$

    $$V_1 \otimes \cdots V_n := M/M_0$$
    
    $$(v_1, \cdots, v_i + v_i', \cdots, v_n) + M_0 = (v_1, \cdots, v_i, \cdots, v_n) + (v_1, \cdots, v_i',\cdots, v_n) + M_0$$
    
    \begin{Def}
    $V_1 \otimes \cdots V_n$ "--- тензорное произведение пространств. 

    элементы тензорного произведения "--- тензоры. 

    так как  $(v_1, \cdots, v_n)$ порождают $M$

    то $(v_1,\cdots, v_n) + M_0$ порождают $M/M_0 = V_1 \otimes \cdots \otimes V_n$
    $V_1 \otimes \cdots \otimes V_n$ "--- разложимый тензор. 
    \end{Def}
   
    $(v_1, \cdots, v_n) + (u_1, \cdots, u_n) + M_0$ "--- не разложимый тензор. 

    Предостережение: разложимые тензоры порождают $V_1 \otimes \cdots V_n$ 
    но не являются базисом.

    $$v_1 \otimes \cdots \otimes (v_i + v'_i) \cdots \otimes v_n - v_1 \otimes \cdots v_i \cdots \otimes v_n - v_1 \otimes \cdots \otimes v_n = 0$$
    \item[Шаг 2:] Построение полилинейного отображения из $V_1 \times \cdots V_n$ в $V_1 \otimes \cdots \otimes V_n$

    $$(v_1, \cdots, v_n) \to v_1 \otimes \cdots \otimes v_n = (v_1, \cdots, v_n) + M_0$$

    $$\phi((v_1, \cdots, v_i + v_i', \cdots, v_n)) = v_1 \otimes \cdots \otimes (v_i + v_i') \otimes \cdots v_n =$$
    $$= v_1 \otimes \cdots v_i \otimes \cdots v_n + v_1 \otimes \cdots v_i' \otimes \cdots \otimes v_n  = $$
    $$ = \phi ((v_1, \cdots, v_n)) + \phi((v_1, \cdots, v_i', \cdots, v_n))$$

    $$\phi((v_1, \cdots, \alpha v_i, \cdots, v_n)) = v_1 \otimes \cdots \alpha v_i \otimes \cdots v_n = \alpha v_1 \otimes \cdots \otimes v_n = $$
    $$= \alpha \phi((v_1, \cdots, v_n))$$
    $\Ra \phi$ "--- полилинейно. 

    $$(v_1, \cdots, v_i + v_i', \cdots, v_n) + M_0 - ((v_1, \cdots, v_i, \cdots, v_n) + M_0) - ((v_1,\cdots, v_n) + M_0) =$$
    $$= (v_1, \cdots, v_i + v_i',  \cdots, v_n) - (v_1, \cdots, v_i, \cdots, v_n) -  (v_1, \cdots, v_i', \cdots, v_0) + M_0 = 0 + M_0$$
    \item[Шаг 3:] Универсальность тензорного произведения. 
    $$V_1 \times \cdots \times V_n \to  V_1 \otimes \cdots \otimes V_n$$
    $$V_1 \times \cdots \times V_n \to W $$
    \begin{theorem}
    $\forall W$ и $\forall$ полилинейного отображения $\alpha \colon V_1 \times \cdots \times V_n \to W$
    $\exists ! \beta \colon V_1 \otimes \cdots \otimes V_n \to W$, полилинейное такое,что $\alpha = \beta \circ \phi$
    \end{theorem}
    \begin{proof}
    \begin{enumerate}
    \item Единственность. 
    Пусть $\beta \in mathscr{L}(V_1 \otimes \cdots \otimes V_n, W)$  существует. 
                              
    Достаточно проверить, что $\beta$ однозначно определено на семействе образующих. 

    $$\beta((v_1\otimes \cdots \otimes v_n)) = \beta (\phi(v_1, \cdots, v_n)) = \alpha (v_1, \cdots, v_n)$$
    $\Ra$ существует не более одного отображения $\beta$.
    \item Существование. 
         $$\bar \beta \colon M \to W$$
         $$\bar \beta ((v_1, \cdots, v_n)) = \alpha(v_1, \cdots, v_n)$$

         по линейности достраиваем до линейного отображения из $M$ в $W$.
    \end{enumerate}
    \end{proof}
\end{description}

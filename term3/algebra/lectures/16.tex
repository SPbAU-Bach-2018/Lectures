\section{Тензорное произведение пространств}

$V_1, \cdots, V_n, W$ "--- векторное пространство над $K$.

Полилинейные отображения $V_1 \times \cdots \times V_n$ в $W$,
то есть отображение линейное по каждому из аргументов. 

Построим новое пространство (тензорное пространство $V_1, \cdots, V_n \colon V_1 \otimes \cdots \otimes V_n$)  
и каноническое отображение из $V_1 \times \cdots \times V_n$ в $V_1 \otimes \cdots \otimes V_n$

\begin{description}
    \item[Шаг 1:]
    Построим $V_1 \otimes \cdots \otimes V_n$
    
    $$V_1 \times \cdots \times V_n = \{(v_1, \cdots, v_n)\colon v_i \in V_i\}$$

    $M$ "--- пространство всех линейных финитных отображений из $V_1 \times \cdots \times V_n$ в $K$
    (каждое отображение почти всюду равно 0).

    $$f, g \in M$$
    $$(f + g)(v_1, \cdots, v_n) = f(v_1, \cdots, v_n) + g(v_1, \cdots, v_n)$$
    $$(\alpha f)(v_1 \cdots v_n) = \alpha f(v_1, \cdots, v_n)$$

   \textbf{Упражнение:} $M$ "--- векторное пространство над $K$.

    $\delta_{(v_1,\cdots, v_n)}(u_1, \cdots, u_n) = \left\{
    \begin{aligned}
    1,  u_1 = v_1, \cdots, u_n = v_n\\
    0, \texttt{otherwise}\\ 
    \end{aligned}
    \right.$

    $f = \sum_{(v_1, \cdots, v_n)} f(v_1, \cdots, v_n)\delta_{(v_1, \cdots, v_n)}$ 
    (так как f финитная, то сумма конечна). 

    $\delta$ "--- функция порождает $M$.

    $\sum \alpha_{(v_1, \cdots, v_n)} \delta_{(v_1, \cdots, v_n)} = 0 \Lra$  все $\alpha_{(v_1, \cdots, v_n)} = 0$ 

    $\alpha_{v_1, \cdots, v_n} \in K \Ra \delta$-функция линейно независима. 

    $\delta$-функция "--- базис $M$.
    Если поле $K$ бесконечно и хотя бы одно из $V_i$ не нульмерно, то 
    $M$ "--- бесконечномерное. 
    
    Далее символ $\delta$ будем опускать $\sum \alpha_{v_1, \cdots, v_n}(v_1, \cdots, v_n)$ "--- такие формальные суммы образуют $M$.

    Формальные финитные линейные комбинации элементов $V_1 \times \cdots \times V_n$.

    $$\sum \alpha_{v_1, \cdots, v_n}(v_1, \cdots, v_n) = \sum \beta_{v_1, \cdots, v_n}(v_1, \cdots, v_n)$$ 
    $$\Lra \forall v_1, \cdots, v_n \colon \alpha_{(v_1, \cdots,  v_n)} = \beta_{(v_1, \cdots, v_n)}$$

    Возьмем подпространство, порожденное следующим набором векторов. 
    $$M_0 = <(v_1, \cdots, v_i + v_i', \cdots, v_n) - (v_1, \cdots, v_i, \cdots, v_n) - (v_1, \cdots, v_i', \cdots, v_n), $$
    $$       (v_1, \cdots, \alpha v_i, \cdots, v_n) - \alpha(v_1,\cdots, v_i, \cdots, v_n)\colon  $$
    $$ i = 1, \cdots, n \colon v_i, v_i' \in V_i, \alpha \in K>$$

    $$V_1 \otimes \cdots \otimes V_n := M/M_0$$
    
    \begin{Rem}
    $$(v_1, \cdots, v_i + v_i', \cdots, v_n) + M_0 = (v_1, \cdots, v_i, \cdots, v_n) + (v_1, \cdots, v_i',\cdots, v_n) + M_0$$
    $$(v_1, \cdots, \alpha v_i, \cdots, v_n) + M_0 = \alpha(v_1,\cdots, v_i, \cdots, v_n) + M_0$$
    \end{Rem}
    \begin{Def}
    $V_1 \otimes \cdots \otimes V_n$ "--- тензорное произведение пространств. 

    Элементы тензорного произведения "--- тензоры. 

    Так как  $(v_1, \cdots, v_n)$ порождают $M$,
    то $(v_1,\cdots, v_n) + M_0$ порождают $M/M_0 = V_1 \otimes \cdots \otimes V_n$
    
    $v_1 \otimes \cdots \otimes v_n$ "--- разложимый тензор. 
    \end{Def}
   
    $(v_1, \cdots, v_n) + (u_1, \cdots, u_n) + M_0$ "--- не разложимый тензор. 

    Предостережение: разложимые тензоры порождают $V_1 \otimes \cdots \otimes V_n$ 
    но не являются базисом.

    $$v_1 \otimes \cdots \otimes (v_i + v'_i) \cdots \otimes v_n - v_1 \otimes \cdots v_i \cdots \otimes v_n - v_1 \otimes \cdots \otimes v_n = 0$$
    \item[Шаг 2:] Построение полилинейного отображения из $V_1 \times \cdots \times V_n$ в $V_1 \otimes \cdots \otimes V_n$

    $$(v_1, \cdots, v_n) \to v_1 \otimes \cdots \otimes v_n = (v_1, \cdots, v_n) + M_0$$

    $$\phi((v_1, \cdots, v_i + v_i', \cdots, v_n)) = v_1 \otimes \cdots \otimes (v_i + v_i') \otimes \cdots v_n =$$
    $$= v_1 \otimes \cdots v_i \otimes \cdots v_n + v_1 \otimes \cdots v_i' \otimes \cdots \otimes v_n  = $$
    $$ = \phi ((v_1, \cdots, v_n)) + \phi((v_1, \cdots, v_i', \cdots, v_n))$$

    $$\phi((v_1, \cdots, \alpha v_i, \cdots, v_n)) = v_1 \otimes \cdots \alpha v_i \otimes \cdots v_n = \alpha v_1 \otimes \cdots \otimes v_n = $$
    $$= \alpha \phi((v_1, \cdots, v_n))$$
    $\Ra \phi$ "--- полилинейно. 

    $$(v_1, \cdots, v_i + v_i', \cdots, v_n) + M_0 - ((v_1, \cdots, v_i, \cdots, v_n) + M_0) - ((v_1,\cdots, v_n) + M_0) =$$
    $$= (v_1, \cdots, v_i + v_i',  \cdots, v_n) - (v_1, \cdots, v_i, \cdots, v_n) -  (v_1, \cdots, v_i', \cdots, v_0) + M_0 = 0 + M_0$$
    \item[Шаг 3:] Универсальность тензорного произведения. 
    $$V_1 \times \cdots \times V_n \to  V_1 \otimes \cdots \otimes V_n$$
    $$V_1 \times \cdots \times V_n \to W $$
    \begin{theorem}
    $\forall W$ и $\forall$ полилинейного отображения $\alpha \colon V_1 \times \cdots \times V_n \to W$
    $\exists ! \beta \colon V_1 \otimes \cdots \otimes V_n \to W$, полилинейное такое,что $\alpha = \beta \circ \phi$
    \end{theorem}
    \begin{proof}
    \begin{enumerate}
    \item Единственность. 
    Пусть $\beta \in \mathscr{L}(V_1 \otimes \cdots \otimes V_n, W)$  существует. 
                              
    Достаточно проверить, что $\beta$ однозначно определено на семействе образующих. 

    $$\beta((v_1\otimes \cdots \otimes v_n)) = \beta (\phi(v_1, \cdots, v_n)) = \alpha (v_1, \cdots, v_n)$$
    $\Ra$ существует не более одного отображения $\beta$.
    \item Существование. 
         $$\tilde{\beta} \colon M \to W$$
         $$\tilde{\beta} ((v_1, \cdots, v_n)) = \alpha(v_1, \cdots, v_n)$$

         по линейности достраиваем до линейного отображения из $M$ в $W$.

         $$\tilde{\beta}(\sum a_{v_1 \cdots, v_n}(v_1, \cdots, v_n)) = \sum a_{v_1, \cdots, v_n} \tilde{\beta}(v_1, \cdots, v_n) = \sum a_{v_1, \cdots, v_n}\alpha(v_1, \cdots, v_n)$$

         $\tilde{\beta}$ "--- линейное отображение. 
         $$\beta\colon M/M_0 \to W$$
         Это можно сделать, если $M_0 \subset \ker \tilde{\beta}$

         $$\tilde{\beta}((v_1, \cdots, v_i + v_i', \cdots, v_n)) - (v_1, \cdots, v_i, \cdots, v_n) - (v_1, \cdots, v_i', \cdots, v_n)) =$$
         $$= \tilde{\beta}((v_1, \cdots, v_i + v_i', \cdots, v_n)) - \tilde{\beta}(v_1, \cdots, v_i, \cdots, v_n) - 
         \tilde{\beta}((v_1, \cdots,v_i', \cdots v_n))$$
         $$= \alpha(v_1, \cdots, v_i + v_i', \cdots, v_n) - \alpha(v_1, \cdots, v_i, \cdots, v_n) - \alpha(v_1, \cdots,v_i', \cdots,  v_n) = 0$$

         $$\tilde{\beta}((v_1, \cdots, a v_i,\cdots, v_n)) - a(v_1, \cdots, v_n) = $$
         $$= \tilde{\beta}((v_1, \cdots, a v_i, \cdots, v_n)) - a \tilde{\beta}((v_1, \cdots, v_i,\cdots, v_n)) = $$
         $$= \alpha(v_1, \cdots, a v_i, \cdots v_n) - a\alpha(v_1, \cdots, v_i, \cdots, v_n) = 0$$

         $\tilde{\beta}$ принимет значение 0 на порождающих подпространствах $M_0$
         $$M_0 \subset \ker \tilde{\beta}$$

         $$\beta(m + M_0) = \tilde{\beta}(m)$$
         $$\beta \colon V_1 \otimes \cdots \otimes V_n \to W$$

         Последнее, что осталось проверить: $\alpha = \beta \circ \phi$

         $$(v_1, \cdots, v_n) \in V_1 \times \cdots \times V_n$$
         $$\phi((v_1, \cdots, v_n)) = v_1 \otimes \cdots \otimes v_n = (v_1, \cdots, v_n) + M_0$$

         $$(\beta \circ \phi)(v_1, \cdots, v_n) = \beta((v_1, \cdots, v_n) + M_0) = $$
         $$= \title{\beta}(v_1, \cdots, v_n) = \alpha(v_1, \cdots, v_n)$$

         $$\beta \circ \phi = \alpha$$
    \end{enumerate}
    \end{proof}
        $\mathcal{L}(V, W)$  "--- множество линейных отображений. 
       
        $V_1, \cdots, V_n, W$   "--- векторное пространство над K.
        $\mathcal{L}(V_1, \cdots, V_n, W)$ "--- пространство полилинейных отображений из
        $V_1 \times \cdots \times V_n$ в $W$.  
        
        $\mathcal{L}(V_1, \cdots, V_n, W)$ "--- векторное пространство над K.
        Сложение и умножение на скаляр определяется поточечно. 
       
        Надо проверить, что $f+g, \alpha f$ вновь полилинейные и 
        что $\mathcal{L}(V_1, \cdots, V_n, W)$ является векторным пространством над $K$.

        $$\alpha = \beta \circ \phi$$
        $$\alpha \in \mathcal{L}(V_1, \cdots, V_n, W)$$
        $$\beta \in \mathcal{L}(V_1 \otimes \cdots,\otimes V_n, W)$$
        $$\alpha \to \beta$$
        $$\mathcal{L}(V_1, \cdots, V_n, W) \to \mathcal{L}(V_1 \otimes \cdots \otimes V_n, W)$$
        $$\alpha + \alpha' = \beta + \beta'$$
        $$\tilde{\beta}((v_1, \cdots, v_n)) = \alpha(v_1, \cdots, v_n)$$
        $$\tilde{\beta'}(()) = \alpha'(\cdots)$$
        $$\tilde{\beta} + \tilde{\beta'} = (\alpha + \alpha')(\cdots)$$
        Аналогично для умножения на скаляр. 

        Отображение $(\alpha \to \beta')$ построенное в теореме является
        линейным отображением из $\mathcal{L}(V_1, \cdots, V_n, W)$  в $\mathcal{L}(V_1 \otimes \cdots \otimes V_n, W)$.

        Это линейное отображение сюръективно. 
        
        Возьмем $\alpha = \beta \circ \phi$. $\beta$ "--- линейное, $\phi$ "--- полилинейное
        $\Ra \alpha$ "--- полилинейное. 

        В силу чатси  единственности в теореме отображение действительно переводит в $\beta$.
        $$\alpha \to \beta_1 \colon \alpha = \beta_1 \circ \phi \Ra \beta = \beta_1$$

        Это линейное отображение инъективно. 
        
        Достаточно поверить, что только нулевое переходит в 0.
        (так как для линейных отображений инъективность равносильно тривиальности ядра).
        $$\alpha \to \beta = 0$$  
        $$\alpha = \beta \circ \phi = 0 \Ra \alpha = 0$$ 

    \begin{conseq}
    Пространства $\mathcal{L}(V_1, \cdots, V_n, W)$ и 
    $\mathcal{\alpha}(V_1 \otimes \cdots \otimes V_n, W)$ "--- изоморфны. 
    \end{conseq}
\end{description}

\begin{exmp}
    \begin{enumerate}
    \item Многочлены
     $$\N \cup \{0\}$$

     Последовательности = отображения из $\N \cup \{0\}$ в K, K "--- поле. 

     Нам нужны только финитные последовательности. 
     
     \textbf{Базис} пространства финитных последовательностей $\{(0,\cdots, 0, 1, 0, \cdots)\} = \delta_i$
     
     $$\delta_i\colon \N\cup\{0\} \to K$$
     $$\delta_i(i) = 1, \delta_i(j) = 0(j \ne i)$$

     $$x^i = (0, \cdots, 1, \cdots, 0)$$

     f "--- многочлен. 
     $$f = a_nx^n + \cdots + a_0x^0$$
     $$g = b_nx^n + \cdots + b_0x^0$$

     $$f + g = (a_n + b_n)x^n + \cdots +(a_0 + b_0)x^0$$
     $f = 0 \Lra$ все коэффициенты = 0. 

    \item Конструкция пространства M.
    $$v_1 \times \cdots \times V_n$$

    Финитные отображения из $V_1 \cdots V_n$ в $K$.

    $\delta_{(v_1, \cdots, v_n)}$ "--- базис пространства. 
    
    $\delta_{(v_1, \cdots, v_n)}(v_1, \cdots, v_n) = 1$
    
    $\delta_{(v_1, \cdots, v_n)}(u_1, \cdots, u_n) = 0$ если $\exists v_i \ne u_i$.

    $$(v_1, \cdots, v_n) = \delta_{(v_1, \cdots, v_n)}$$
    
    $$f(.) = \sum_{(v_1, \cdots, v_n), f(v_1, \cdots, v_n) \ne 0} f(v_1, \cdots, v_n)\delta_{v_1, \cdots, v_n}(.)$$ 

    $$f = \sum_{v_1, \cdots, v_n}\alpha_{v_1, \cdots, v_n}(v_1, \cdots, v_n)$$
    $$g = \sum_{v_1, \cdots, v_n}\beta_{v_1, \cdots, v_n}(v_1, \cdots, v_n)$$
    $$(f + g) = (\alpha_{v_1 \cdots, v_n} + \beta_{v_1 \cdots v_n})(v_1, \cdots, v_n)$$
    $f = 0 \Lra$  все $\alpha_{v_1, \cdots, v_n}$  равны 0.
    \end{enumerate}
\end{exmp}

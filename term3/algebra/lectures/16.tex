\section{Тензорное произведение пространств}

Возьмём $V_1, \dots, V_n$, $W$ "--- векторные пространства над $K$.
Рассмотрим полилинейные отображения $V_1 \times \dots \times V_n$ в $W$,
то есть отображения, линейные по каждому из аргументов.

Построим новое, тензорное пространство $V_1 \otimes \dots \otimes V_n$
и каноническое отображение из $V_1 \times \dots \times V_n$ в $V_1 \otimes \dots \otimes V_n$

\begin{description}
\item[Шаг 1:]
	Построим $V_1 \otimes \dots \otimes V_n$
	\[ V_1 \times \dots \times V_n = \{(v_1, \dots, v_n)\colon v_i \in V_i\} \]
	Возьмём $M$ "--- пространство всех финитных отображений из $V_1 \times \dots \times V_n$ в $K$
	(финитное значит, что отображение почти всюду равно 0).
	\begin{gather*}
		f, g \in M \\
		(f + g)(v_1, \dots, v_n) = f(v_1, \dots, v_n) + g(v_1, \dots, v_n) \\
		(\alpha f)(v_1 \dots v_n) = \alpha f(v_1, \dots, v_n)
	\end{gather*}
	\textbf{Упражнение:} $M$ "--- векторное пространство над $K$.
	Теперь рассмотрим набор финитных отображений $\delta_{(v_1,\dots, v_n)}$:
	\[
		\delta_{(v_1,\dots, v_n)}(u_1, \dots, u_n) = \begin{cases}
			1 & u_1 = v_1 \land \dots \land u_n = v_n \\
			0 & \text{иначе}
		\end{cases}
	\]
	Для любой $f$:
	\[ f = \sum_{(v_1, \dots, v_n)} f(v_1, \dots, v_n)\delta_{(v_1, \dots, v_n)} \]
	Если $f$ финитная, то сумма конечна.
	Тогда набор $\delta$-функций порождает $M$.
	\[ \sum \alpha_{(v_1, \dots, v_n)} \delta_{(v_1, \dots, v_n)} = 0 \Lra \alpha_{(v_1, \dots, v_n)} = 0 \]
	$\alpha_{v_1, \dots, v_n} \in K$, значит $\delta$-функции линейно независимы.

	$\delta$-функции "--- базис $M$.
	Если поле $K$ бесконечно и хотя бы одно из $V_i$ не нульмерно, то $M$ "--- бесконечномерное.

	Далее символ $\delta$ будем опускать, будем просто писать $(v_1, \dots, v_n)$.
	\TODO в конспекте ещё не везде хватает скобок, чтобы это обозначение было видно.

	$\sum \alpha_{v_1, \dots, v_n}(v_1, \dots, v_n)$ "--- такие формальные суммы образуют $M$.

	Теперь рассмотрим формальные финитные линейные комбинации элементов $V_1 \times \dots \times V_n$:
	\begin{gather*}
		\sum \alpha_{v_1, \dots, v_n}(v_1, \dots, v_n) = \sum \beta_{v_1, \dots, v_n}(v_1, \dots, v_n) \Lra \\
		\Lra \forall v_1, \dots, v_n \colon \alpha_{(v_1, \dots,  v_n)} = \beta_{(v_1, \dots, v_n)}
	\end{gather*}

	Возьмем подпространство, порожденное следующим набором векторов.
	\begin{align*}
		M_0=& \left< (v_1, \dots, v_i + v_i', \dots, v_n) - (v_1, \dots, v_i, \dots, v_n) - (v_1, \dots, v_i', \dots, v_n), \right.\\
			& (v_1, \dots, \alpha v_i, \dots, v_n) - \alpha(v_1,\dots, v_i, \dots, v_n) \\
			& \mid \left. \forall i = 1..n, v_i, v_i' \in V_i, \alpha \in K \right>
	\end{align*}
	Что же такое $M_0$? Это те самые разности, которые не позволяют нам работать с $\delta$-ми линейно по их параметрам.
	Нам бы хотелось, чтобы все $M_0$ были на самом деле нулями в новом пространстве.
	Давайте возьмём факторпространство:
	\[ V_1 \otimes \dots \otimes V_n \coloneqq M/M_0 \]

	\begin{Rem}
	$$(v_1, \dots, v_i + v_i', \dots, v_n) + M_0 = (v_1, \dots, v_i, \dots, v_n) + (v_1, \dots, v_i',\dots, v_n) + M_0$$
	$$(v_1, \dots, \alpha v_i, \dots, v_n) + M_0 = \alpha(v_1,\dots, v_i, \dots, v_n) + M_0$$
	\end{Rem}

	\begin{Def}
		$V_1 \otimes \dots \otimes V_n$ "--- тензорное произведение пространств.
		Элементы тензорного произведения "--- тензоры.
		Так как  $(v_1, \dots, v_n)$ порождают $M$,
		то $(v_1,\dots, v_n) + M_0$ порождают $M/M_0 = V_1 \otimes \dots \otimes V_n$.
		$v_1 \otimes \dots \otimes v_n$ "--- разложимый тензор.
	\end{Def}
	\begin{exmp}
		$(v_1, \dots, v_n) + (u_1, \dots, u_n) + M_0$ "--- не разложимый тензор.
	\end{exmp}
	Предостережение: разложимые тензоры порождают $V_1 \otimes \dots \otimes V_n$, но не являются базисом:
	\[ v_1 \otimes \dots \otimes (v_i + v'_i) \dots \otimes v_n - v_1 \otimes \dots v_i \dots \otimes v_n - v_1 \otimes \dots \otimes v_n = 0 \]

\item[Шаг 2:]
	Построение полилинейного отображения из $V_1 \times \dots \times V_n$ в $V_1 \otimes \dots \otimes V_n$.
	\begin{gather*}
		\underbrace{(v_1, \dots, v_n)}_{\text{это не $\delta$}} \to v_1 \otimes \dots \otimes v_n = \underbrace{(v_1, \dots, v_n)}_{\text{а это $\delta$}} + M_0 \\
		\phi((v_1, \dots, v_i + v_i', \dots, v_n))
		= v_1 \otimes \dots \otimes (v_i + v_i') \otimes \dots v_n = \\
		= v_1 \otimes \dots v_i \otimes \dots v_n + v_1 \otimes \dots v_i' \otimes \dots \otimes v_n
		= \phi ((v_1, \dots, v_n)) + \phi((v_1, \dots, v_i', \dots, v_n)) \\
		\phi((v_1, \dots, \alpha v_i, \dots, v_n))
		= v_1 \otimes \dots \alpha v_i \otimes \dots v_n
		= \alpha v_1 \otimes \dots \otimes v_n
		= \alpha \phi((v_1, \dots, v_n))
	\end{gather*}
	Получили, что $\phi$ "--- полилинейно.
	\begin{gather*}
		((v_1, \dots, v_i + v_i', \dots, v_n) + M_0) - ((v_1, \dots, v_i, \dots, v_n) + M_0) - ((v_1,\dots, v_n) + M_0) = \\
		= (v_1, \dots, v_i + v_i',  \dots, v_n) - (v_1, \dots, v_i, \dots, v_n) -  (v_1, \dots, v_i', \dots, v_0) + M_0 = 0 + M_0
	\end{gather*}

\item[Шаг 3:] Универсальность тензорного произведения.
	У нас уже есть $V_1 \times \dots \times V_n \to V_1 \otimes \dots \otimes V_n$, а также $V_1 \times \dots \times V_n \to W$.
	\begin{theorem}
		Для любых $W$ и полилинейного отображения $\alpha \colon V_1 \times \dots \times V_n \to W$,
		существует единственное полилинейное отображение $\beta \colon V_1 \otimes \dots \otimes V_n \to W$,
		что $\alpha = \beta \circ \phi$.
	\end{theorem}
	\begin{proof}\begin{description}
	\item[Единственность:]
		Пусть $\beta \in \mathscr{L}(V_1 \otimes \dots \otimes V_n, W)$  существует.
		Достаточно проверить, что $\beta$ однозначно определено на семействе образующих, то есть на разложимых тензорах:
		\[ \beta((v_1\otimes \dots \otimes v_n)) = \beta (\phi(v_1, \dots, v_n)) = \alpha (v_1, \dots, v_n) \]
		Значит действительно существует не более одного отображения $\beta$.

	\item[Существование:]
		\begin{gather*}
			\tilde \beta\colon M \to W \\
			\tilde \beta((v_1, \dots, v_n)) = \alpha(v_1, \dots, v_n)
		\end{gather*}
		по линейности достраиваем до линейного отображения из $M$ в $W$.
		\[ \tilde \beta\left(\sum a_{v_1 \dots, v_n}(v_1, \dots, v_n)\right) = \sum a_{v_1, \dots, v_n} \tilde \beta((v_1, \dots, v_n)) = \sum a_{v_1, \dots, v_n} \alpha(v_1, \dots, v_n) \]
		$\tilde \beta$ "--- линейное отображение.
		Теперь хотим линейное $\beta\colon M/M_0 \to W$.
		Это можно сделать, если $M_0 \subset \ker \tilde \beta$:
		\begin{gather*}
			\tilde \beta((v_1, \dots, v_i + v_i', \dots, v_n)) - (v_1, \dots, v_i, \dots, v_n) - (v_1, \dots, v_i', \dots, v_n)) = \\
			= \tilde \beta((v_1, \dots, v_i + v_i', \dots, v_n)) - \tilde \beta(v_1, \dots, v_i, \dots, v_n)
			- \tilde \beta((v_1, \dots,v_i', \dots v_n)) = \\
			= \alpha(v_1, \dots, v_i + v_i', \dots, v_n) - \alpha(v_1, \dots, v_i, \dots, v_n) - \alpha(v_1, \dots,v_i', \dots,  v_n)
			= 0 \\
			\tilde \beta((v_1, \dots, a v_i,\dots, v_n)) - a(v_1, \dots, v_n)
			= \tilde \beta((v_1, \dots, a v_i, \dots, v_n)) - a \tilde \beta((v_1, \dots, v_i,\dots, v_n)) = \\
			= \alpha(v_1, \dots, a v_i, \dots v_n) - a\alpha(v_1, \dots, v_i, \dots, v_n)
			= 0
		\end{gather*}
		$\tilde \beta$ принимет значение 0 на порождающих подпространствах $M_0$.
		Таким образом, $M_0 \subset \ker \tilde \beta$.
		\begin{gather*}
			\beta(m + M_0) = \tilde \beta(m) \\
			\beta \colon V_1 \otimes \dots \otimes V_n \to W
		\end{gather*}
		Последнее, что осталось проверить: $\alpha = \beta \circ \phi$
		\begin{gather*}
			(v_1, \dots, v_n) \in V_1 \times \dots \times V_n \\
			\phi((v_1, \dots, v_n)) = v_1 \otimes \dots \otimes v_n = (v_1, \dots, v_n) + M_0 \\
			(\beta \circ \phi)(v_1, \dots, v_n) = \beta((v_1, \dots, v_n) + M_0)
			= \tilde \beta(v_1, \dots, v_n) = \alpha(v_1, \dots, v_n) \\
			\beta \circ \phi = \alpha
		\end{gather*}
	\end{description}\end{proof}
	$\mathcal L(V, W)$ "--- множество линейных отображений.
	$V_1, \dots, V_n$, $W$ "--- векторные пространства над $K$.
	$\mathcal L(V_1, \dots, V_n, W)$ "--- пространство полилинейных отображений из $V_1 \times \dots \times V_n$ в $W$.

	$\mathcal L(V_1, \dots, V_n, W)$ "--- векторное пространство над $K$.
	Сложение и умножение на скаляр определяется поточечно.
	Надо проверить, что $f+g, \alpha f$ вновь полилинейные и
	что $\mathcal L(V_1, \dots, V_n, W)$ является векторным пространством над $K$.
	\begin{gather*}
		\alpha = \beta \circ \phi \\
		\alpha \in \mathcal L(V_1, \dots, V_n, W) \quad
		\beta \in \mathcal L(V_1 \otimes \dots,\otimes V_n, W) \\
		\alpha \to \beta \\
		\mathcal L(V_1, \dots, V_n, W) \to \mathcal L(V_1 \otimes \dots \otimes V_n, W) \\
		\alpha + \alpha' = \beta + \beta' \\
		\tilde \beta((v_1, \dots, v_n)) = \alpha(v_1, \dots, v_n) \\
		\tilde \beta'((\dots)) = \alpha'(\dots) \\
		\tilde \beta + \tilde{\beta'} = (\alpha + \alpha')(\dots)
	\end{gather*}
	Аналогично для умножения на скаляр.

	Отображение $(\alpha \to \beta)$ построенное в теореме является линейным отображением из
	$\mathcal L(V_1, \dots, V_n, W)$  в $\mathcal L(V_1 \otimes \dots \otimes V_n, W)$.

	Это линейное отображение сюръективно:
	возьмем произвольное линейное $\beta$, $\alpha = \beta \circ \phi$.
	$\beta$ "--- линейное, $\phi$ "--- полилинейное, значит $\alpha$ "--- полилинейное.
	В силу части единственности в теореме отображение действительно переводит в $\beta$.
	\[ \alpha \mapsto \beta_1 \colon \alpha = \beta_1 \circ \phi \Ra \beta = \beta_1 \]

	Это линейное отображение инъективно:
	достаточно поверить, что только нулевое переходит в 0.
	(так как для линейных отображений инъективность равносильно тривиальности ядра).
	\begin{gather*}
		\alpha \mapsto \beta = 0 \\
		\alpha = \beta \circ \phi = 0 \Ra \alpha = 0
	\end{gather*}

	\begin{conseq}
		Пространства $\mathcal L(V_1, \dots, V_n, W)$ и $\mathcal L(V_1 \otimes \dots \otimes V_n, W)$ изоморфны.
		\[ \mathcal L(V_1, \dots, V_n, W) \cong \mathcal L(V_1 \otimes \dots \otimes V_n, W) \]
	\end{conseq}
\end{description}

\begin{exmp}
	\begin{enumerate}
	\item Многочлены
	 $$\N \cup \{0\}$$

	 Последовательности = отображения из $\N \cup \{0\}$ в K, K "--- поле.

	 Нам нужны только финитные последовательности.

	 \textbf{Базис} пространства финитных последовательностей $\{(0,\dots, 0, 1, 0, \dots)\} = \delta_i$

	 $$\delta_i\colon \N\cup\{0\} \to K$$
	 $$\delta_i(i) = 1, \delta_i(j) = 0(j \ne i)$$

	 $$x^i = (0, \dots, 1, \dots, 0)$$

	 f "--- многочлен.
	 $$f = a_nx^n + \dots + a_0x^0$$
	 $$g = b_nx^n + \dots + b_0x^0$$

	 $$f + g = (a_n + b_n)x^n + \dots +(a_0 + b_0)x^0$$
	 $f = 0 \Lra$ все коэффициенты = 0.

	\item Конструкция пространства M.
	$$v_1 \times \dots \times V_n$$

	Финитные отображения из $V_1 \dots V_n$ в $K$.

	$\delta_{(v_1, \dots, v_n)}$ "--- базис пространства.

	$\delta_{(v_1, \dots, v_n)}(v_1, \dots, v_n) = 1$

	$\delta_{(v_1, \dots, v_n)}(u_1, \dots, u_n) = 0$ если $\exists v_i \ne u_i$.

	$$(v_1, \dots, v_n) = \delta_{(v_1, \dots, v_n)}$$

	$$f(.) = \sum_{(v_1, \dots, v_n), f(v_1, \dots, v_n) \ne 0} f(v_1, \dots, v_n)\delta_{v_1, \dots, v_n}(.)$$

	$$f = \sum_{v_1, \dots, v_n}\alpha_{v_1, \dots, v_n}(v_1, \dots, v_n)$$
	$$g = \sum_{v_1, \dots, v_n}\beta_{v_1, \dots, v_n}(v_1, \dots, v_n)$$
	$$(f + g) = (\alpha_{v_1 \dots, v_n} + \beta_{v_1 \dots v_n})(v_1, \dots, v_n)$$
	$f = 0 \Lra$  все $\alpha_{v_1, \dots, v_n}$  равны 0.
	\end{enumerate}
\end{exmp}

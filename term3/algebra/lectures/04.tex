\section{Изометрии}
$$V, (, )$$
Интересующие нас ситуации. 

\begin{enumerate}
    \item симметрическая, билинейная. 
    \item кососимметрическая, билинейная. 
    \item эрмитовосимметричная, полуторалинейная. 
\end{enumerate}

\begin{Def}
$\mathscr{A} \in \End(V)$ "--- изометрия V, (, )
если $\forall x, y \in V \colon (\mathscr{A}x, \mathscr{A}y) = (x, y)$
\end{Def}

\begin{exmp}

$\R^2$, поворот "--- изометрия. 

\end{exmp}

\begin{theorem}
$$\dim V = n < \infty$$
$\{e_i\} $ "--- базис V.

$\Gamma$ "--- матрица Грама относительно базиса $\{e_i\}$

$A = [\mathscr{A}]_{<e_i>}$ "--- матрица линейного отображения $\mathscr{A}$.

Следующие условия равносильны:
\begin{enumerate}
    \item $\mathscr{A}$ "--- изометрия.
    \item $A^T\Gamma\overline{A} = \Gamma$
\end{enumerate}
\end{theorem}

\begin{proof}
$$x \lra [x] = \begin{pmatrix} x_1\\ \cdots\\ x_n\\ \end{pmatrix} $$
$$y \lra [y] = \begin{pmatrix} y_1\\ \cdots\\ y_n\\ \end{pmatrix} $$

$$[\mathscr{A}x] = A[x]$$
$$[\mathscr{A}y] = A[y]$$

$$(x, y) = [x]^T \Gamma \overline{[y]}$$

$$(\mathscr{A}x, \mathscr{A}y) = [x]^TA^T\Gamma \overline{A} \overline{[y]}$$
$$(x, y) = [x]^T\Gamma\overline{[y]}$$

\begin{enumerate}
\item $1 \to 2$ очевидно. 
\item $2 \to 1$
$$\forall [x][y] \colon [x]^TA^T\Gamma\overline{A}\overline{[y]} = [x]^{T} \Gamma \overline{[y]}$$

$$[x] = \begin{pmatrix} 0 \\ 0 \\ 1\\ 0 \\ \end{pmatrix}$$
$$[y] = \begin{pmatrix} 0\\ 1\\ 0\\ 0\\ \end{pmatrix}$$

$$(A^T \Gamma \overline{A})_{ij} = \Gamma_{ij} \Ra A^T \Gamma \overline{A} = \Gamma$$
\end{enumerate}
\end{proof}

\begin{conseq}
Так как форма невырожденная, то изометрия обратимое преобразование. 
\end{conseq}
\begin{proof}
$$A^{T}\Gamma \overline{A} = \Gamma, \det \Gamma \ne 0 \Ra \det A \ne 0$$

$A = [\mathscr{A}]$ обратима $\Ra$ A "--- обратим.  
\end{proof}

\begin{conseq}
Изометрия пространства V, (,)
образуют группу (относительно композиции).
\end{conseq}

\begin{proof}
$id_{V}$  "--- изометрия. 

$$(\mathscr{A}\mathscr{B}x, \mathscr{A}\mathscr{B}y) = (\mathscr{B}x, \mathscr{B}y) = (x, y)$$

$\mathscr{A}, \mathscr{A}^{-1}$
$$E = [\mathscr{A}\mathscr{A}^{-1}] = [\mathscr{A}][\mathscr{A}^{-1}]$$
$$[\mathscr{A}] = A$$
$$[\mathscr{A}^{-1}] = \mathscr{A}^{-1}$$

$$A^{T}\Gamma \overline{A} = \Gamma$$
$$\Gamma = (A^T)^{-1} \Gamma (\overline{A})^{-1} = (A^{-1})^{T} \Gamma \overline{(A^{-1})}$$

$\mathscr{A}^{-1}$ "--- изометрия.  
\end{proof}

\begin{Def}
V "--- векторное конечномерное пространство над K. $\dim V = n < \infty$ 

$B = (, )$ "--- невырожденная, симметричная билинейная форма. 

Группа изометрии $O(V, B)$ называется ортогональной группы связанной с формой $B$.

В частности, если $V = \R^n$  евклидово пространство. $(x, y) = \sum x_i y_i$
$O(n)$ "--- вещественная ортогональная группа. 
\end{Def}

В $\R^n$ могут быть билинейные формы $\sum_{i = 1}^{p}x_iy_i - \sum_{i = p + 1}^{p + q} x_i y_i$, $O(p, q), p + q = n$

$O(1, 3)$ "--- группа изометрий пространства Минковского. 

Любая билинейная форма будет эквивалента одной из этой. Это будет показано позже.

\begin{Def}
$B$ "--- кососимметрическая, билинейная

Группа изометрий (V, B) "--- симплектическая группа.
$Sp(V, B)$ 
\end{Def}

\begin{exmp}
$$B \Gamma = \begin{pmatrix} 0& -E_n\\ E_n & 0\\ \end{pmatrix}$$
$$K^{2n} Sp(K^{2n}, B)$$

Симплектические группы исчерпываются группами заданными такой матрицей Грамма. 
\end{exmp}

\begin{Def}
K "---- поле с инволюцией. 

B "--- эрмитовосимметричная, полуторалинейная, невырожденная. 

Группа изометрий пространства $(V, B)$   называется унитарной группой, отвечающий форме B.

$U(V, B)$. 

\end{Def}

\begin{exmp}
$V = \C^n, U(n)$
$\sum_{i = 1}^{n} x_i\overline{y_i}$


$U(p, q):$
$$\sum_{i = 1}^{p} x_i\overline{y_i}  - \sum_{i = p + 1}^{p + q} x_i\overline{y_i}$$
\end{exmp}

\subsection{Матричная интерпретация}
$O(n)$

$\R^{n}$ "--- евклидово пространство. 
$e_1, \cdots, e_n$ "--- стандартный базис $\Gamma = E$
$A = [\mathscr{A}]$

изометрии евклидова пространства = ортогональные операторы. 

$A^TA = E \Lra A^{-1} = A^{T}$

\begin{Def}
$A \in M(n, \R)$ называется ортогональной, если $A^{-1} = A^{T}$
\end{Def}

Ортогональная матрица "--- это матрица ортогонального оператора в евклидовом пространстве. 

\begin{conseq}{Ортогональных матриц}
A "--- ортогональная матрица. 

$A = (A_1| \cdots |A_n)$
 
A "--- ортогональна $\Lra$ $A_i$ попарно ортогональны и длины 1.

$A^TA = E$ означает, что произведение разных столбцов дает 0, а самого на себя дает 1.

То есть это просто переформулировка условия ортогональности. 

Но, оказывается, тоже самое верно и для строк. 

$$A = \begin{pmatrix} A_1\\ \vdots\\ A_n \\ \end{pmatrix}$$

$A$ "--- ортогональны $\Lra$ строки попарно ортогональны и длиня 1.

$$A^{T}A = E \Ra A^{T} = A^{-1}$$
$$AA^{T} = AA^{-1} = E$$
$$AA^T = E$$

$\mathscr{A}$ "--- ортогональный оператор в евклидовом пространстве $\Lra$ $\mathscr{A}$ переводит ортонормированный базис в ортонормированный. 

$\{e_i\}$ "--- ортонормированный базис $\Gamma_{\{e_i\}} = E$
$$(\mathscr{A}e_i, \mathscr{A}e_j)_{ij} = (e_i,e_j)_{i, j} = E$$
$$(\mathscr{A}(\sum a_ie_i), \mathscr{A}(\sum b_i e_i)) = (a_1, \cdots, a_n) (\mathscr{A}e_i, \mathscr{A}e_j)_{i, j}
\begin{pmatrix} b_1\\ \vdots\\ b_n\\ \end{pmatrix} = (a_1, \cdots, a_n) \begin{pmatrix} b_1 \\ \vdots \\ b_n\\ \end{pmatrix}$$
$$(a_1e_1 + \cdots + a_n e_n,b_1e_1 + \cdots + b_ne_n)$$
\end{conseq}

\begin{exmp}
\begin{enumerate}
\item 
$$O(1)$$
$$a^{T}a = 1, a \in \R$$
$$a = \pm 1 $$
$$O(1) = \{\pm 1\} $$
\item 
$$O(2)$$
$$\begin{pmatrix} a&c\\ b&d\\ \end{pmatrix}$$
$$\begin{pmatrix} a&b\\ c&d\\ \end{pmatrix} \begin{pmatrix} a&c\\ b&d\\ \end{pmatrix} = E$$
$$a^2 + b^2 = 1$$
$$c^2 + d^2 = 1$$
$$ac + bd = 0$$

$A$ "--- ортогональный.
$\Ra \det A = \pm 1$


$$SO(n) = \{ A \in O(n), \det A = 1\} $$
$$O(2) = SO(2) \cup \begin{pmatrix} 1 & 0 \\  0& -1\\ \end{pmatrix} SO(2)$$


Матрицы с определителем 1:
$$a^2 + b^2 = 1$$
$$c^2 + d^2 = 1$$
$$ac + bd = 0$$
$$ad  - bc = 1$$

Запишем тоже самое по строкам. 
$$a^2 + c^2 = 1$$
$$b^2 + d^2 = 1$$
$$ab + cd = 0$$
$$ad  - bc = 1$$

Делаем выводы, что 
$$a = \cos(\Phi), c = \sin(\Phi)$$
$$c = -\sin(\Phi), d = \cos(\Phi)$$
$c \ne \sin(\Phi)$ так как не подойдет под одно из условий. 


Тогда матрица имеет вид:
$$
\begin{pmatrix}
\cos \Phi & -\sin \Phi\\
\sin \Phi & \cos \Phi\\
\end{pmatrix}
$$

\item 
$V = \C^n$ "--- унитарное пространство.

Изометрия A называется  унитарным оператором, A его матрица в стандартном базисе "--- называется унитарной матрицей. 

$A \in M(n, \C)$, $\Gamma = E$

$A^{T}\overline{A} = E$
$$\overline{A}^TA = E$$
$$A^{-1} = \overline{A^T}$$

\textbf{Упражнение:} A "--- унитарный $\Lra$ ортонормированный базис переводит в ортонормированный.

\item 
$U(1)$

$$a^{T}\overline{a} = 1$$
$$|a| = 1$$

$$U(1) = S_1 \sim SO(2)$$
$$SU(2) \to SO(3) $$

\end{enumerate}
\end{exmp} 

\setauthor{Глеб Валин}

\begin{gather*}
    f_{\sigma} : V_1 \otimes \cdots \oplus V_n \ra V_\sigma(1) \otimes \cdots \otimes V_{\sigma(n)} \\
    f_{\sigma, \lambda}: V_{\lambda(i)} \otimes \cdots \otimes V_{\lambda(n)} \ra V_{\sigma\lambda(1)} \otimes \cdots \otimes V_{\sigma\
    lambda(1)}
\end{gather*}

\begin{theorem} (О двойственности)
    $$    \dim V_i < \infty \Ra V_1^* \otimes \cdots \otimes V_n^* \cong (V_1 \otimes \cdots \otimes V_n)^* $$ % заменить на знак изоморфности
\end{theorem}

\begin{proof}
    \begin{gather*} 
        \text{коммутативная диаграмма универсального свойства:} \\
        V_1^* \times \cdots \times V_n^* \ra^{\phi} V_1^* \otimes \cdots \otimes V_n^* \\
        V_1^* \times \cdots \times V_n^* \ra^{\alpha} \mathcal{L}(V_1, \cdots, V_n, K) \\
        (v^*_1, \cdots, v^*_n) \mapsto \alpha(v_1^*, \cdots, v^*_n) \\
        \alpha(v_1^*, \cdots, v_n^*)(u_1, \cdots, u_n) = v_1^*(u_1)  \cdots v_n^*(u_n) \\
        \text{$\alpha$ каноническое отображение из $V_1^* \times \cdots \times V_n^*$} \\
        \beta: V_1^* \otimes \cdots \otimes V_n^* \ra \mathcal{L}(V_1, \cdots, V_n, K) \cong \mathcal{L}(V_1 \otimes \cdots \otimes V_n, K) = (V_1 \otimes \cdots \otimes V_n)^* \\
        \beta(v^*_1 \otimes \cdots \otimes v^*_n)(u_1, \cdots, u_n) = v_1^*(u_1) \cdots v_n^*(u_n) \\
        \text{$\beta$ "--- является ли изоморфизмом? } \\
        \text{т.к. всё конечномерно, то можем написать:} \\
        \dim (V_1^* \otimes \cdots \otimes V_n^*) = \Pi(\dim V_i^*) = \Pi (\dim V_i) \\
        \dim (V_1 \otimes \cdots \otimes V_n)^* = \dim (V_1 \otimes \cdots \otimes V_n) = \Pi (\dim V_i)
	\end{gather*}
    т.к. размерности совпадают, то для доказательства биективности достаточно проверить сюрективность (по теореме о размерности ядра будет следовать инъективность, т.к. размерность ядра равно нулю)
	\begin{gather*}
        f \in \mathcal{L}(V_1, \cdots, V_n, K) \\
        \text{$\{e_i^{(j)}\}$ "--- базис $V_i$} \Ra \\
		\TODO
        ДОПИСАТЬ
		\TODO \\
        \beta(e_{i_1}^{(i)*} \otimes \cdots \otimes e_{i_n}^{(n)*}) = e_{i_1}^{(1)*} \cdot \cdots \cdot e_{i_n}^{(n)*} \\
        \Ra \mathcal{L}(V_1, \cdots, V_n, K) \subseteq \Im \beta
    \end{gather*}
\end{proof}

\section{Тензорное произведение и линейные отображения}
Рассмотрим ещё одну интерпретацию тензорного произведения:

\begin{theorem}
    $U$, $V$ "--- конечномерные пр-ва над $K$, тогда
	\[ U^* \otimes_{K} V \cong \mathcal{L}(U, V) \cong M(\dim V, \dim U, K) \]
\end{theorem}
\begin{proof}
    Рисуем коммутатвную диаграмму универсального свойства:
    \begin{gather*}
        U^* \times V \ra^{\phi} U^* \otimes V \\
        U^* \times V \ra^{\alpha} \mathcal{L}(U, V) \\
        U^* \otimes V \ra^{\beta} \mathcal{L}(U, V) \\
    \end{gather*}

    Теперь расписываем: $\alpha(u^*, v)^{\in \mathcal{L}(U,V)}(x \in U) = u^*(x)\cdot v$
    \begin{enumerate}
        \item $\alpha(u^*, v)$  линейно по $x$
        \item $\alpha$ "--- полилинейное отображение из $U^* \times V$
    \end{enumerate}

    \begin{gather*}
        \beta: U^* \otimes V \ra \mathcal{L}(U, V) \\
        \beta (u^* \otimes v)(\cdot) = u^*(\cdot)\cdot v
    \end{gather*}

    Размерность области определения $\beta$ и области назначения равны

    Достаточно проверить, что $\beta$ сюрьективно:

    $\epsilon_1, \cdots, \epsilon_n$ "--- базис $V$

    $e_1 \cdots, em$ "--- базис $U$

    $e_1^*, \cdots, e_m^*$ "--- базис $U^*$ двойственный к $e_1 \cdots, e_m$

    \begin{gather*}
        \beta(e_i^* \otimes \epsilon_j)(e_i) = 1 \cdot \epsilon_j = \epsilon_j \\
        \beta(e_1^* \otimes \epsilon_j)(e_k) = 0 \cdot \epsilon_j = 0, k \neq j \\
        U_{e_1, \cdots, e_n} \ra V_{\epsilon_1, \cdots, \epsilon_n} \\
        [\beta(e_i^* \otimes \epsilon_j)] = j
        \begin{pmatrix}
        0 & \cdots & 0 \\
        \vdots & 1 & \vdots \\
        0 & \cdots & 0 \\
        \end{pmatrix} \text{в позиции $(j, j$ стоит единичка}
    \end{gather*}

    Что то ещё что я не успел записать
\end{proof}

\begin{Rem} про обозначения:

    $\beta(e^i \otimes \epsilon_j) = j \begin{pmatrix}
        0 & \cdots & 0 \\
        \vdots & 1 & \vdots \\
        0 & \cdots & 0 \\
        \end{pmatrix} = e_j^i = e_{ji}$
\end{Rem}

\begin{Rem}
    (замечане про отличие разложимых тензоров от неразложимых)

    $U^* \otimes V$, где 

    $V = \sum^n_{j=1} a^j \epsilon_j$ (индекс вверху значит, сумирование по нему (уточнить???))

    $U^* = \sum^n_{i=1}b_ie^i$

	\begin{gather*}\beta(u^*\otimes v) = 
            \beta((\sum_i b_i e^i)\otimes(\sum_j a^j \epsilon_j)) = 
            \sum_i\sum_j b_i a^j \beta(e^i\otimes\epsilon_j) =
            \sum_i\sum_j b_i a^j e_j^i = 
            \begin{pmatrix}
            a^1 \\
            \vdots \\
            a^n
			\end{pmatrix} \cdot (b_1, \cdots, b_m),\\ \text{ если $u^* \neq 0$ и $v \neq 0$}
	\end{gather*}

    $\rang(\beta(u^* \otimes v)) = 1$

    и наоборот, всякая матрица ранга $1$ есть столбец на строку

    $\Ra$ всякая матрица ранга $1$ "--- образ разложимого тензора

    $U^* \otimes V \cong M(n,m, K)$
\end{Rem}

\begin{Def}
    $v \in V_1 \otimes \cdots \otimes V_n$

    Тензорный ранг $v$ "--- наименьшее $m$, что $v$ сумма $m$ разложимых тензоров
\end{Def}

$n = 2$: $U^* \otimes V \cong M_2(\dim U, K)$

Тензорный ранг матрицы равен $r$, если у отвечающего ей при этом изоморфизме тензора тензорный ранг равен $r$

Тензорный ранг матрицы $A$ равен наименьшему $r$, что $A$ сумма $r$ матриц ранга $1$

\begin{theorem}
    Тензорный ранг матрицы совпадает с обычным рангом
\end{theorem}

\begin{proof}
    \begin{enumerate}
        \item Тензорный ранг не больше обычного ранга
        \item $\rang A = r$ $\Ra$ $\exists$ обратимые матрицы $C$ и $D$, что:
            $C^-1AD = 
            \begin{pmatrix}
                 E_r & 0 \\
                 0 & 0 
            \end{pmatrix} == \sum_{i = 1}^r e_i^i$

            тогда $A = \sum_{i=1}^n C(e_i^i)D{-1}$, где ранг каждого слагаемого $1$

            $\rang (F G) \leq \min (\rang F, \rang G) \Ra$ тензорный ранг $\leq r$

            $\rang A \leq $ тензорный ранг $A$

            $\rang(F + G) \leq \rang(F) + \rang(G)$

            $A = A_1 + \cdots + A_r$, тогда $\rang A \leq 1 + \cdot + 1 = r$
    \end{enumerate}
\end{proof}

\section{Тензорный ранг и алгоритм Штрассена}

в этом параграфе будут нетензорные обозначение $a_i^j = a_{ij}$ (уточнить)

Будем пытаться что-то сказать про умножение матриц

Рассмотрим отображение:

$ M_n(K) \times M_n(K) \ra M_n(K)$
$ (A, B) \mapsto AB$ "--- билинейное отображение

\begin{itemize}
    \item $\{e_{ij}\}$ "--- базис $M_n(K)$
    \item $\epsilon_{ij}$ "--- базис $(M_n(K))^*$
    \item $A = (a_{ij}), \epsilon_{ij}(A) = a_{ij}$
\end{itemize}

\begin{gather*}
    A = \sum_{i,j} \epsilon_{i,j}(A)e_{ij} \\
    B = \sum_{k,l} \epsilon_{k,l}(B)e_{kl} \\
    e_{ij}e_{kl} = \left\{
    \begin{matrix} 
    0 & i \neq k \\
    e_{il} & j = k
    \end{matrix}
    \right. \\
    AB = ()() = \sum_{i,j,l} \epsilon_{ij}(A)\epsilon_{jl}(B)e_{il} = 
    \sum_{i,j,l} \left(\epsilon_{ij}(A) \epsilon_{jl}\right)(B)e_{il}, 
    \text{здесть $n^3$ умножений и $n^2(n-1)$ сложений} \\
\end{gather*}

теперь что значит это умножение с точки зрения тензоров

\begin{gather*}
    T \in M^*_n(K) \otimes M^*_n(K) \otimes M_n(K) \\
    T = \sum_{i,j,l} \epsilon_{ij} \otimes \epsilon_{jl} \otimes e_{il} = 
        \sum_{i,l} (\sum_{j} \epsilon_{ij} \otimes \epsilon_{jl}) \otimes e_{il}
\end{gather*}
$T$ задаёт билинейное отображение $M_n(K)\times M_n(K)$ в $M_n(K)$
\begin{gather*}
    T(A, B) = \sum_{i,j,l} \epsilon_{ij}(A) \epsilon_{jl}(B) e_{il} = AB
\end{gather*}

Заметим, что тензорный ранг $T \leq n^3$, из этого можно извлечь выгоду, ведь на самом деле
ранг может быть меньше

Предположим, что 
$$(*): T = \sum^N_{p=1}    \underbrace{f_p}_{\in M_n^*(K)} \otimes 
                    \underbrace{g_p}_{\in M_n^*(K)} \otimes
                    \underbrace{H_p}_{\in M_n(K)} $$
пусть все $H_p$ матрицы с целыми коэффицентами, т.е. каждый коэффицент $H_p$ это $1 + \cdots + 1$ или $-(1 + \cdots + 1)$ или $0$
$$ AB = T(A, B) = \sum_{p=1}^N f_p(A)g_p(B)H_p $$
Если такое разложение $(*)$, то для перемножения матриц нужно $N$ умножений

Вопрос: верно ли, что тензорный ранг $T$ может быть меньше, чем $n^3$ ?

Ответ: оказывается для матриц $2\times2$ ранг не $8$ а $7$

Собственно умножений для матриц $2\times2$ нужно делать $7$ (Штрассен, 1969), сейчас построим формулу

\begin{gather*}
    \begin{matrix}
        e_{11} & a & & \epsilon_{11} & \alpha
        e_{12} & b & & \epsilon_{12} & \beta
        e_{21} & c & & \epsilon_{21} & \gamma
        e_{22} & d & & \epsilon_{22} & \delta
    \end{matrix}\\
    alpha \begin{matrix}
        a_{11} & a_{12} \\
        a_{21} & a_{22}
    \end{matrix} = a_{11} \\
    e_{11} = \begin{matrix}
    1 & 0 \\
    0 & 0
    \end{matrix}\\
    \text{обычное умножение:} \\
    \begin{matrix}
        a_{11} & a_{12} \\
        a_{21} & a_{22}
    \end{matrix}
    \begin{matrix}
        b_{11} & b_{12} \\
        b_{21} & b_{22}
    \end{matrix} \\
    T = (\alpha \otimes \alpha + \beta  \otimes \gamma) \otimes a + \\
        (\alpha \otimes \beta  + \beta  \otimes \delta) \otimes b + \\
        (\gamma \otimes \alpha + \delta \otimes \gamma) \otimes c + \\
        (\gamma \otimes \beta  + \delta \otimes \delta) \otimes d = \\
        (\alpha - \delta) \otimes (\alpha - \delta) \otimes (a + d) + \\
        \delta \otimes (\alpha + \gamma) \otimes (a + c) + \\
        (\alpha - \beta) \otimes \delta \otimes (a - b) + \\
        \alpha \otimes (\beta + \delta) \otimes (b + d)
\end{gather*}

Итого тут мы делаем 7 умножений и 18 сложений

Мы докажем (т.к. чуть проще), что 7 умножений и 22 сложения

\begin{gather*}
    \begin{matrix}
        a_{11} & a_{12} \\
        a_{21} & a_{22}
    \end{matrix}
    \begin{matrix}
        b_{11} & b_{12} \\
        b_{21} & b_{22}
    \end{matrix} = 
    \begin{matrix}
        c_{11} & c_{12} \\
        c_{21} & c_{22}
    \end{matrix}\\
    \text{суммы:} \\
    \text{смотрим на первую строчку: $(\alpha - \delta) \otimes (\alpha - \delta) \otimes (a + d)$} \\
    S_1 = a_{11} - a_{22}, S_2 = b_{11} - b_{22} \\
    \text{теперь на вторую строчку, далее аналогично} \\
    S_3 = b_{11} - b_{21}, S_4 = b_{11} - a_{12} \\
    S_5 = b_{12} + b_{22}, S_6 = a_{21} - a_{22} \\
    S_7 = a_{12} - a_{22}, S_8 = b_{21} - b_{22} \\
    S_9 = a_{21} - a_{11}, S_{10} = b_{11} + b_{12}
    \text{произведения:} \\
    P_1 = S_1 S_2    \\
    P_2 = a_{22} S_3 \\
    P_3 = S_4 b_{22} \\
    P_4 = a_{11} S_5 \\
    P_5 = S_6 b_{11} \\
    P_6 = S_7 b_{11} \\
    P_7 = S_7 S_8    \\
    P_8 = S_9 S_{10} \\
    \text{результат:} \\
    c_{11} = P_1 + P_2 + P_3 + P_6, c_{12} = P_4 - P_3 \\
    c_{12} = P_2 + P_5, c_{22} = P_1 + P_4 - P_5 + P_7
\end{gather*}

Заметим, что данное умножение справедливо и для матриц где значения "--- некоммутирующие 
переменные, т.к. порядок множителей здесь сохраняется 
(можно раскрыть скобки и увидеть, или понять, что $a_i$ всегда раньше $b_i$)

Как можно перемножать блочные матрицы?

$A_{11}, A_{12}, \cdots, B_{22}$  "--- квадратные матрицы одного размера

$$\begin{matrix}
        A_{11} & A_{12} \\
        A_{21} & A_{22}
    \end{matrix}
    \begin{matrix}
        B_{11} & B_{12} \\
        B_{21} & B_{22}
    \end{matrix} = 
    \begin{matrix}
        A_{11}B_{11} + A_{12}B_{21} & A_{11}B_{12} + A_{12}B_{21} \\
        A_{21}B_{11} + A_{22}B_{21} & A_{21}B_{12} + A_{22}B_{22}
    \end{matrix}$$
и теперь умножаем Штрассеном, где у нас будет 7 матричных умножений

Не трудно заметить, что мы научились быстро умножать матрицы размера $n = 2^k$
В таком случае для $n = 2^k$, число умножений:

    \begin{gather*}
        2^k \leq n \leq 2^{k+1} \Ra (2^{k+1})^{\log_2 7} \leq (n^{1 + \frac{1}{k}})^{\log_2 7} \sim n^{log_2 7 + O(1)} \\
        7^k = (2^{\log_2 7})^k = (2^k)^{\log_2 7} = n^{\log_2 7} \sim n^{2.88}
    \end{gather*}

Есть ещё один результат: Coppersmith Vilwgrad 1981

$O(n^{2.376})$ "--- но неприменимо из-за слишком большой константы

Есть ещё гипотеза Штрассена, но её уже стёрли с доски

\section{Тензорное произведение операторов}

$V_1, \cdots, V_n, W_1, \cdots, W_n$ "--- все конечномерные

$A_1: V_1 \ra W_1, \cdots, A_n: V_n \ra W_n$

$A_1 \otimes \cdots A_n: V_1 \otimes \cdots \otimes V_n \ra W_1 \otimes \cdots \otimes W_n$

$(A_1 \otimes \cdots A_n)(v_1 \otimes \cdots \otimes v_n) = A_1 v_1 \otimes \cdots \otimes A_n v_n$

\begin{theorem}
    Такой оператор существует и единственный
\end{theorem}

\begin{proof}
    Доказываем, используя снова свойство универсальности, рисуем диаграмму:
    \begin{enumerate}
        \item $V_1 \times \cdots \times V_n \ra^\phi V_1 \otimes \cdots \otimes V_n$
        \item $V_1 \times \cdots \times V_n \ra^\alpha W_1 \otimes \cdots \otimes W_n$
        \item $V_1 \otimes \cdots \otimes V_n \ra^{\beta =  A_1 \otimes \cdots A_n} W_1 \otimes \cdots \otimes W_n$
    \end{enumerate}

    Хотим найти матрицу $[A_1 \otimes \cdots A_n]$ в тензорном базисе

    $A_1 \otimes A_2: V_1 \otimes V_2 \ra W_1 \otimes W_2$

    $\{e_i\}_{i=1\cdots n}$ "--- базис $V_1$

    $\{f_i\}_{i=1\cdots n}$ "--- базис $V_2$

    $\{\epsilon_i\}_{i=1\cdots n}$ "--- базис $W_1$

    $\{\chi_i\}_{i=1\cdots n}$ "--- базис $W_w$

    тогда $e_i \otimes f_j$ "--- базис $V_1 \otimes V_2$ и $\epsilon_r \otimes \ch_s$ "--- базис $W_1 \otimes W_2$

    $[A_1]_{e, \epsilon} = (a_{ij})$

    $[A_2]_{f, \chi} = (b_{rs})$

    $[A_1 \otimes A_2]_{e \otimes f, \epsilon \otimes \chi}$ строки нумеруем парами $(r, s)$, столбцы нумеруем парами $(i, j)$

	$[(A_1 \otimes A_2))(e_i \otimes f_j)]_{(r, s)} = [A_1e_i \otimes A_2f_j]_{(r,s)} = a_{(r, i)}b_{(s, j)}$

    $r = 1, i = 1, \dim V_1$

    $r = 2, i = 1, \dim V_2$

    $\begin{pmatrix}
        a_{11}[A_2] & a_{12}[A_2]\\
        a_{21}[A_2] & 
    \end{pmatrix}$

    кронекерово произведение
    
\end{proof}

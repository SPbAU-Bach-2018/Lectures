\section{Свободная группа}
$A \ne 0$ "--- множество (алфавит).\\
$A = \{a_i\}_{i \in I}$\\
$\bar {A} \cap A = 0$\\
$\bar{A} = \{\bar{a_i}\}_{i \in I}$ биекция между $A$ и $\bar{A}$\\
$a_i \to \bar{a_i}$\\
$A \cup \bar{A}$\\
$W = $ множество всех слов в алфавите $A \cup \bar{A}$\\
(то есть множество конечных носителей букв из $A \cup \bar{A}$)\\
$\Lambda$ "--- пустое слово. \\
На $W$ задаем операцию умножения слов. 
$w_1, w_2 \to w_1w_2$\\

Ассоциативность $w_1(w_2w_3) = (w_1w_2)w_3$\\
Нейтральный элемент $\Lambda w_1 = w_1 = w_1 \Lambda$\\

$\sim \colon w_1$ и $w_2$ эквивалентны, если одно из них можно получитьь из другого
применением (возможно, неоднократного).
\begin{enumerate}
\item вставка. В любое место можно вставить $a\bar{a}$ или $\bar{a}a$\\
$w_1w_2 \sim w_1a\bar{a}w_2$\\
$w_1w_2 \sim w_1\bar{a}aw_2$\\
\item вычеркивания. Если в слове есть $a \bar{a}$ или $\bar{a}a$, то их можно вычеркнуть. 
$w_1a\bar{a}w_2 \sim w_1w_2$\\
$w_1\bar{a}aw_2 \sim w_1w_2$\\
\end{enumerate}

Упражнение: $\sim$ отношение эквивалентности.\\
$u_1 \to u_2$\\
$u_1 \to u_2 \to u_3$\\

$W/\sim$ "--- множество классов эквивалентности.\\
$[w_1][w_2] = [w_1w_2]$ \\
$w_1 \sim u_1$\\
$w_2 \sim u_2$\\
$w_1w_2 \sim u_1u_2$\\

Умножение классов "--- ассоциативно.
$\Lambda w = w$\\
$[\Lambda w] = [w]$\\
$[\Lambda][w] = [w]$\\

$a \in A$ \\
$a^1 = a$\\
$a^{-1} = \bar{a}$\\
$w = \prod_{i = 1}^{n} a_i^{\epsilon_i}$\\
$\epsilon_i = \pm 1$\\
$[w]$\\
$\bar{w} = \prod_{i = 1}^{n} a_{n + 1 - i}^{-\epsilon_i}$\\
$w\bar{w} = a_1^{\epsilon_1} \cdots a_n^{\epsilon_n} \cdot a_1^{-\epsilon_1} \sim a_1^{\epsilon_1}a_1^{-\epsilon_1} \sim \Lambda$\\
$[w][\bar{w}] = [\Lambda]$\\
$[\bar{w}][w] = [\Lambda]$\\
$W/\sim$ "--- группа\\
$F_A$ "--- свободная группа с алфавитом $A$.
$F$--- free group.

\begin{exmp}
$A = \{a\}$ \\
$a$, $\bar{a}$\\
Классы эквивалентности 
$[\Lambda], [a]^n, [\bar{a}]^n = [a]^{-n}$\\
$F_{\{a\}} \cong (\Z,+)$\\
\end{exmp}
\begin{Def}
Слово w называется несократимым, если в нем
нет вхождений $aa^{-1}$ или $a^{-1}a$, $a \in A$\\
\end{Def}   
\begin{theorem}
В W в каждом классе эквивалентности есть ровно одно несократимое слово.
\end{theorem}
\begin{Rem}
Свободная группа = множество несократимых слов.
$w_1 \cdot w_2$ перемножаем и сокращаем.
\end{Rem}
\begin{proof}
 \begin{enumerate}
 \item
 Существование.
 $w_1 = \prod_{i = 1}^{w}a_i^{\epsilon_i}$\\
 $n$ "--- длина слова.

 Если $w$ не является несократимым есть $aa^{-1}$  либо $a^{-1}a$ длина уменьшится.
 Так как длина конечна, процесс когда-нибудь оборвется.
 \item
 Единственность. \\
 $w_1$ и $w_2$ "--- несократимые слова.\\
 $w_1 \sim w_2 \Ra w_1 = w_2$\\
 $A_0 = w_1 \sim A_1 \sim \cdots \sim A_k \sim w_2 = A_{k + 1}$\\
 На каждом шаге одна операция либо вставки либо вычеркивания. 
 $l$ "--- суммарная длина промежуточных слов.
 Множество суммарных длин не пусто. Найдем цепочку, с наименьшим значением $l$.

 Если $l = 0$, то $w_1 \sim w_2 \Ra w_1 = w_2$
 операция не может быть ни вставкой, ни сокращением.
 Если $l \ne 0$, то можно найти цепочку преобразований 
 с меньшей суммарной длинной промежуточных слов.

 $w_1 = A_0 \sim A_1 \sim \cdots \sim A_{i - 1} \sim A_i \sim A_{i + 1} \sim \cdots \sim A_n = w_2$\\
 Пусть $A_i$ слово наибольшей длины. Значит перед этим была вставка, а после этого будет вычеркивание.

 \begin{enumerate}
 \item Вставим $a\bar{a}$ и на следующем шаге это же вхождение вычеркивание. \\
 $w_1 = A_0 \cdots \sim \cdots A_{i - 1} \sim A_{i + 2}$\\
 \item
 $\bar{a}a$ и вычеркнули $\bar{a}{a}$\\
 \item
 вставили $a\bar{a}$, сократили $\bar{a} a$\\
 $a \to a\bar{a}{a} \to a$\\
 $A_{i - 1} = A_{i + 1}$
 \item аналогичный случай, только наоборот.
 \item
 Вставили $a\bar{a}$ сократили $b \bar{b}$
 Вхождения $a$ и $b$ не совпадают. 

 Можем сначала сократить $b\bar{b}$ а потом вставим $a\bar{a}$. Суммарная длина
 промежуточных слов уменьшилась на 4.
 \end{enumerate}
 \end{enumerate}
\end{proof}
\begin{exmp}
$|A| \ge 2, F_A$ "--- неабелева.\\
$a, b\in A$\\
$ab(ba)^{-1} = aba^{-1}b^{1} \ne 1\Ra ab \ne ba$ \\
$A \subset B \Ra F_A \subset F_B$\\
Упражнение: $F_{\{a, b\}}$ содержит подгруппу, изоморфную $F_\{a, b, c\}$ как
следствие, и с любым конечным числом образующих.
\end{exmp}
\begin{exmp}
$\forall n \exists H \subset F_{\{a, b\}}$ \\
$H \cong F_{\{x_1, \cdots, x_n\}}$\\
\end{exmp}
\begin{theorem}
$G = <x_1, \cdots, x_n>$\\
$\Ra \exists$ гомоморфизм из $F_{a_1, \cdots, a_n}$ на $G$, 
такое что $a_1 \to x_1, \cdots, a_n \to x_n$\\
\end{theorem}
\begin{proof}
$A = \{a_1, \cdots, a_n\}$ \\
$\bar{A}$\\
$W$ слова в алфавите $A \cup \bar{A}$\\
$W \to G$\\
$a_i \to x$\\
$\prod a_i^{\epsilon_i} \to \prod_{i = 1}^{n} x_i^{\epsilon_i}$\\
$\psi(w_1w_2) = \psi(w_1)\psi(w_2)$\\
Если $w_1 \sim w_2$, то $\psi(w_1) = \psi(w_2)$\\
$u_1u_2 \sim u_1a_i\bar{a_i}u_2$\\
$\psi(u_1)\psi(u_2) = \psi(u_1)x_ix_i^{-1}\psi(u_2)$\\
$\psi$ индуцирует отображение из $W/\sim$ в $G$.

То есть получили отображение из $F_A$ в $G$ и это
гомоморфизм.
\end{proof}
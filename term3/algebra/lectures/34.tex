\section{Свободная группа}
$A \ne 0$ "--- множество (алфавит).\\
$A = \{a_i\}_{i \in I}$\\
$\bar {A} \cap A = 0$\\
$\bar{A} = \{\bar{a_i}\}_{i \in I}$ биекция между $A$ и $\bar{A}$\\
Каждой букве сопоставляем обратную $a_i \to \bar{a_i}$\\
Нащ алфавит это объединение двух частей $A \cup \bar{A}$\\
$W = $ множество всех слов в алфавите $A \cup \bar{A}$\\
(то есть множество конечных носителей букв из $A \cup \bar{A}$)\\
$\Lambda$ "--- пустое слово. \\
На $W$ задаем операцию умножения слов 
$w_1, w_2 \to w_1w_2$(конкатинация двух слов)\\
Ассоциативность $w_1(w_2w_3) = (w_1w_2)w_3$\\
Роль нейтрального элемента играет пустое слово $\Lambda w_1 = w_1 = w_1 \Lambda$\\

Проблема, для создания группы нехватает обратных элементов. Поэтому вместо множетсва $W$ рассмотрим
некоторое фактор множество.

$\sim \colon w_1$ и $w_2$ эквивалентны, если одно из них можно получить из другого
применением (возможно, неоднократного) операций следующего вида:
\begin{enumerate}
\item Операция вставки. В любое место можно вставить $a\bar{a}$ или $\bar{a}a$\\
$w_1w_2 \sim w_1a\bar{a}w_2$\\
$w_1w_2 \sim w_1\bar{a}aw_2$\\
\item Операция вычеркивания. Если в слове есть $a \bar{a}$ или $\bar{a}a$, то их можно вычеркнуть. 
$w_1a\bar{a}w_2 \sim w_1w_2$\\
$w_1\bar{a}aw_2 \sim w_1w_2$\\
То есть это обратное действие к вставке.
\end{enumerate}

Упражнение: $\sim$ отношение эквивалентности.\\
Для доказательства рефликсивности вообще ничего не надо делать. 
Что бы доказать симметричность, надо проделать все операции в обратном порядке заменяя операции на противоположные. 
Транзитивность, тоже понятно. Просто нужно проделать две цепочки преобразований.

$W/\sim$ "--- множество классов эквивалентности.\\

Можно ввести умножение класов. 
$[w_1][w_2] = [w_1w_2]$ \\

Теперь необходимо проверить, что введеная операция корректна, то есть не зависит от выбора представителя. 
$w_1 \sim u_1$\\
$w_2 \sim u_2$\\
$w_1w_2 \sim u_1u_2$\\
Если есть цепочка преобразований из $w_1$ в $u_1$, то можем ее проделать и после конкатинации. Так же со вторым, 
значит, после перемножения оказались в одном классе.


Умножение классов "--- ассоциативно.\\

После факторизации класс пустого слова будет нейтральным. 
$\Lambda w = w$\\
$[\Lambda w] = [w]$\\
$[\Lambda][w] = [w]$\\

Как быть с обратными?\\
Для букв введем обозначение:
$a \in A$\\
$a^1 = a$\\
$a^{-1} = \bar{a}$\\
Тогда произвольное слово в нашем алфавите может быть записано как:
$w = \prod_{i = 1}^{n} a_i^{\epsilon_i}$\\
$\epsilon_i = \pm 1$\\
Что логично предложить в качестве обратного элемента $[w]$?\\
$\bar{w} = \prod_{i = 1}^{n} a_{n + 1 - i}^{-\epsilon_i}$ "--- перевернем слово и заменим элементы на обратные.\\
$w\bar{w} = a_1^{\epsilon_1} \cdots a_n^{\epsilon_n} \cdot a_1^{-\epsilon_1} \sim a_1^{\epsilon_1}a_1^{-\epsilon_1} \sim \Lambda$\\
$[w][\bar{w}] = [\Lambda]$\\
$[\bar{w}][w] = [\Lambda]$\\

Таким образом множество классов эквивалентностей $W/\sim$ превратилась группу.\\
Эта группа обозначается $F_A$ и называется свободная группа с алфавитом $A$.
$F$ от слова free group.

\begin{exmp}
$A = \{a\}$ \\
$a$, $\bar{a}$\\
Классы эквивалентности 
$[\Lambda], [a]^n, [\bar{a}]^n = [a]^{-n}$\\
$F_{\{a\}} \cong (\Z,+)$\\
\end{exmp}

С целым классом работать сложно, класс большой. Нам удобнее выбрать в каждом классе какого-нибудь 
канонического представителя.
\begin{Def}
Слово w называется несократимым, если в нем
нет вхождений $aa^{-1}$ или $a^{-1}a$, $a \in A$\\
\end{Def}   
\begin{theorem}
В W в каждом классе эквивалентности есть ровно одно несократимое слово.
\end{theorem}
\begin{Rem}
О свободной группе можно думать как о множестве несократимых слов.
Но теперь после перемножения слов нужно сократить результат.
Если последний символ первого слова обратен в каком-то смысле первому символу второму, то 
сокращаем и повторяем действие пока можем.
\end{Rem}
\begin{proof}
 \begin{description}
 \item[Существование.]
 $w_1 = \prod_{i = 1}^{w}a_i^{\epsilon_i}$\\
 $n$ "--- длина слова.

 Если $w$ не является несократимым есть $aa^{-1}$, либо $a^{-1}a$, значит можем провести операцию 
 сокращения, но при этом длина уменьшится. Продолжаем операцию пока можем.

 Так как длина конечна, процесс когда-нибудь оборвется.
 
 \item[Единственность.]
 $w_1$ и $w_2$ "--- несократимые слова из одного класса тогда нужно проверить, что это просто одно слово.\\
 Так как слова эквивалентны, то существует цепочка преобразований,которая переводит одно слово в другое.
 $A_0 = w_1 \sim A_1 \sim \cdots \sim A_k \sim w_2 = A_{k + 1}$\\
 На каждом шаге одна операция либо вставки либо вычеркивания. В частности первая операция может быть
 только вставки, последняя только вычеркивания.  
 
 Введем параметр $l$ "--- суммарная длина промежуточных слов по которому будем вести индукцию.

 Так как слова эквивалентны, то хотя бы одна цепочка нашлась, значит 
 множество суммарных длин не пусто. Тогда найдем цепочку, с наименьшим значением $l$.

 Если $l = 0$, то $w_1 \sim w_2 \Ra w_1 = w_2$
 операция не может быть ни вставкой, ни сокращением.

 Если $l \ne 0$, то можно найти цепочку преобразований 
 с меньшей суммарной длинной промежуточных слов.

 $w_1 = A_0 \sim A_1 \sim \cdots \sim A_{i - 1} \sim A_i \sim A_{i + 1} \sim \cdots \sim A_n = w_2$\\
 Пусть $A_i$ слово наибольшей длины. Значит перед этим была вставка, а после этого будет вычеркивание.

 Дальше все зависит от того, какие именно вставки и вычеркивания были. Придется разобрать несколько случаев.
 \begin{enumerate}
 \item Вставим $a\bar{a}$ и на следующем шаге это же вхождение вычеркивание. \\
 $w_1 = A_0 \cdots \sim \cdots A_{i - 1} \sim A_{i + 2} \cdots \sim A_n$ \\
 Но тогда мы вообще могли обойтись последовательностью более короткой длины без $A_i$, даже 
 после этого придется выкинуть $A_{i - 1}$ или $A_{i + 2}$, ну или разрешить ничего не делать, кроме того, 
 что бы вставлять и вычеркивать. 
 \item
 $\bar{a}a$ и вычеркнули $\bar{a}{a}$. Здесь рассуждения совершенно 
 аналогичные.
 \item
 Вставили $a\bar{a}$, сократили $\bar{a} a$, где $\bar{a}$ тот же самый символ. 
 $a \to a\bar{a}{a} \to a$\\
 $A_{i - 1} = A_{i + 1}$
 Но опять, одно промежуточное слово можем выкинуть.
 \item аналогичный случай, когда вставляли $\bar{a}a$, сокращали $a\bar{a}$, с тем же самым 
 вхождением $a$.
 \item
 Вставили $a\bar{a}$ сократили $b \bar{b}$
 Вхождения $a$ и $b$ не совпадают и никак не пересекаются. 

 Можем сначала сократить $b\bar{b}$ а потом вставим $a\bar{a}$. Суммарная длина
 промежуточных слов уменьшилась на 4.
 \end{enumerate}
 \end{description}
\end{proof}
\begin{Rem}
$|A| \ge 2, F_A$ "--- неабелева.\\
$a, b\in A$\\
$ab(ba)^{-1} = aba^{-1}b^{1} \ne 1\Ra ab \ne ba$ \\

$A \subset B \Ra F_A \subset F_B$\\
Упражнение: $F_{\{a, b\}}$ содержит подгруппу, изоморфную $F_{\{a, b, c\}}$ как
следствие, и с любым конечным числом образующих.

$\forall n \colon \exists H \subset F_{\{a, b\}} \colon H \cong F_{\{x_1, \cdots, x_n\}}$\\
\end{Rem}

\begin{theorem}
$G = <x_1, \cdots, x_n>$ "--- группа с конечным числом образующих\\
$\Ra \exists$ гомоморфизм из $F_{a_1, \cdots, a_n}$ на $G$, 
такое что $a_1 \to x_1, \cdots, a_n \to x_n$\\

Другими славами, всякая группа с конечным числом образующих может быть реализована, как
фактор группа свободной группы.
\end{theorem}
\begin{proof}
$A = \{a_1, \cdots, a_n\}$, $\bar{A}$\\
$W$ слова в алфавите $A \cup \bar{A}$\\
Сначала постороим отображение из $W$ в $G$.\\
$\psi \colon W \to G$\\
$a_i \to x_i$\\
$\prod a_i^{\epsilon_i} \to \prod_{i = 1}^{n} x_i^{\epsilon_i}$\\
$\psi(w_1w_2) = \psi(w_1)\psi(w_2)$\\

Теперь нужно проверить, что отображение на эквивалентных словах принимает одинаковое значение.
Если $w_1 \sim w_2$, то $\psi(w_1) = \psi(w_2)$\\
$u_1u_2 \sim u_1a_i\bar{a_i}u_2$\\
$\psi(u_1)\psi(u_2) = \psi(u_1)x_ix_i^{-1}\psi(u_2)$\\
$\psi$ индуцирует отображение из $W/\sim$ в $G$.

То есть получили отображение из $F_A$ в $G$ и это
гомоморфизм.
\end{proof}
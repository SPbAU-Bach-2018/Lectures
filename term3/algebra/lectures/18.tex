\section{Примеры}
\begin{enumerate}
    \item Пространства функций на декартовых произведениях.\\ 
    $S_1, \cdots S_n$ "--- конечные множества, $K$ "--- поле.\\ 
    $F(S_i)$ множества отображений из $S_i$ в $K$.\\
    
    Рассмотрим множество функций $F(S_1 \times \cdots \times S_n)$ со значениями в поле $K$.
    На этом множестве вводится естественное векторное пространство c + и $\cdot$:
    \begin{enumerate}
    \item $(f + g)(x) = f(x) + g(x)$\\
    \item $(\alpha g)(x) = \alpha g(x)$\\ 
    \end{enumerate}

    Существует каноническое отождествление 
    $F(S_1 \times \cdots \times S_n) = F(S_1) \otimes \cdots \otimes F(S_n)$

    $\dim_k F(S_i) = |S_i|$ "--- базис $\delta$ функций (характеристические функции одноэлементных подмножеств).
    $$x \in S_i, \delta_{x}^{(i)} \colon S_i \to K$$
    $$\delta_{x}(x) = 1$$
    $$\delta_{x}(y) = 0, y \ne x$$    
    $$f(x) = \sum_{i = 1}^{n}f(x_i)\delta_i(x)$$
    Так же дельта функции линейно независимые $\sum \alpha_i \delta_i(x) = 0$.
    Подставим вместо x все возможные $x_i$, получаем, что каждая конкретная $\alpha_i = 0$.                                         

     Размерности обоих пространств совпадают: 
    \begin{gather*}
    \dim(F(S_1 \times \cdots \times S_n)) = |S_1 \times \cdots \times S_n| = 
    |S_1| \times \cdots \times |S_n| = \\
    = \dim F(S_1) \times \cdots \times \dim F(S_n) = 
    = \dim (F(S_1) \otimes \cdots \otimes F(S_n))
    \end{gather*}

    Базис $F(S_1 \times \cdots \times S_n)$:\\
    $\delta_{x_1, \cdots, x_n} = \delta_{x_1} \cdot \cdots \cdot \delta_{x_n}$ \\
    $\delta_{x_1, \cdots, x_n}(y_1 \cdots, y_n) = \delta_{x_1}(y_1) \cdot \cdots \cdot \delta_{x_n}(y_n)$

    Базис $F(S_1 \otimes \cdots \otimes S_n)$:\\
    $\{\delta_{x_1}^1, \cdots, \delta_{x_n}^n| x_i \in S_i\}$, 
    $\delta_{x_1}^{1} \otimes \cdots \otimes \delta_{x_n}^n$.

    Изоморфизм сопоставляет $\delta_{x_1} \otimes \cdots \otimes \delta_{x_n}$, $\delta_{x_1, \cdots, x_n}$. 
    $(\delta_{x_1} \otimes \cdots \otimes \delta_{x_n})(x_{j_1}, \cdots, x_{j_n}) := \delta_{x_1}(x_{j_1}) \cdot \cdots \cdot \delta_{x_n}(x_{j_n}) \\
    = \delta_{x_1, \cdots, x_n}(x_{j_1}, \cdots, x_{j_n})$\\

    Продолжаем по линейности на все тензоры. 
    
    Вопрос: что будет образом разложимых тензоров при таком изоморфизме?
    $f_1, \cdots, f_n, f_i \in F(S_i)$\\
    $f_1 \otimes \cdots \otimes f_n \to f_1(.) \cdot f_2(.) \cdots f_n(.)$\\
    Образом является функция с разделяющимися переменными
    \begin{Def}
    $g \in F(S_1 \times \cdots \times S_n)$ "--- функция с разделяющемися переменными, если
    $\exists f_1, \cdots, f_n| f_i \in F(S_i) \colon \forall x_1, \cdots x_n g(x_1, \cdots, x_n)  = f_1(x_1)\cdot \cdots \cdot f_n(x_n)$.
    \end{Def}
    \begin{proof}
    $f_i = \sum_{j = 1}^{|S_i|}\alpha_j^{i}\delta_j^{i}$\\
    $f_1 \otimes \cdots \otimes f_n = \otimes(\sum_{j = 1}^{|S_i|}\alpha_j^{i}\delta_{j}^i) = \\
    = \sum_{j_1, \cdots j_n}_{1\le j_i \le |S_i|}\alpha_{j_1}\cdtos \alpha_{j_n}\delta_{j_1} \otimes \cdots \otimes \delta_{j_n}^n \to \\
    \sum \alpha__{j_1}\cdots \alpha_{j_n} \delta_{j_1}^1 \cdots \delta_{j_n}^n = \prod_{i = 1^n}(\sum_{j = 1}^{|S_i|}\alpha_j^i\delta_j^i) =\\
    \prod_{i = 1}^{n}f_i(.)$
    Вот и получилась функция с разделяющимеся переменными.

    И наоборот, можено просто прочитать цепочку в другую сторону.
    \end{proof}
\item
Подъем поля скаляров.\\
Пусть $K \subset L$ "--- поле\\
$V_K$ "--- векторное пространство над $K$\\

Хотим построим новое векторное пространство $V_L$ над $L$\\
Хотим, что бы $V_L$ было как-то связано с $V_K$.\\
Хотим, что бы выполнялись следующие свойства:
\begin{enumerate}
\item $V_K \subset V_L$, точнее $V_K$ можно отождествлить 
с некоторым подмножеством $V_L$ и это подмножество изоморфны, как 
векторные пространства над $K$.
\item $\dim_L V_L = \dim_K V_K$
\end{enumerate}

$\{e_i\}$ "--- базис $V_K$ над $K$\\
Любой вектор можем записать как линейная комбинация $v \in V_K$, 
$v = \sum \alpha_i e_i$ и если посмотреть на теже вектора, но домножать на 
коэфиценты из $L$, то получится ровно наше новое веторное пространство. 
$V_L  = \{\sum \beta_ie_i\}, \beta_i \in L$\\

Сейчас будет описано конструкция, которая будет удовлетворять всем 
описанным условиям.

Рассмотрим $L$ как векторное пространтсво над $K$ и зафиксируем базис $l_j$.\\

Рассмотрим тензорное произведение над полем $K$ и обоазначим это множество как $V_L$:
$V_L = L \otimes_{K} V_K = < l \otimes v | l \in L, v \in V >$\\

Тогда базис $V_L$:$\{l_j \otimes e_i\}$\\

Проверим выполнение первого условия: $V_K \cong K \otimes V_K \subset L \otimes V_L$\\
Объясним, почему выполняется первый изоморфизм:\\
$\{1\}$ "--- базис $K$ над $K$.\\
$\{1 \otimes e_i\}$ "--- базис $K \otimes V_K$ над K.\\
$\{e_i\}$ "--- базис $V_k$\\
Понятно, как отождествить базисные вектора, дальше продолжаем по линейности и понятно, почему это 
изоморфизм.\\

Зададим на $V_L$ структуру векторного пространства над $L$:
Нам нужно задать отображение $L \times V_L \to V_L$, которое удовлетворяет свойствам умножения на сколяр в 
векторном пространстве и сужение первого аргумента на $K$ довало уже имеющиеся у нас умножение. Сложение у нас уже есть.
$$a(l \otimes v) = (al) \otimes v | a \in L $$
Мы определили умножение на семействе образующих, умножение для всего остального сейчас будем доопределять.
Сейчас этим и займемся.

Доопределение:
$v \in L \otimes_K V_k \Ra v = \sum_{i = 1}^{S}\alpha_i (l_i \otimes v_i)$\\
Теперь умножим произвольный вектор на $\beta:$\\
$\beta (\sum_{i = 1}^{S}\alpha_i(l_i \otimes v_i)) = \sum \alpha_i \beta(l_i \otimes v_i) = $
$= \sum \alpha_i(\beta l_i \otimes v_i)$.

Теперь нужно проверить корректность, то есть, 
что наше определение не зависит от разложения.

Проверим корректность определения.\\
$\sum_{i = 1}^{S}\alpha_i(l_i \otimes v_i) = \sum_{j = 1}^{r}\gamma_j(l'_j \otimes v'_j)$\\
Нужно проверить, что после домножения на $\beta$ получится одно и тоже.\\

Вспомним, как мы вообще определяли тензоры. Если два тензора равны, то мы можем из 
одного попасть в другой за конечное число преобразований вида($l \in L, k \in K, v_i \in V$):
$(l_1 + l_2)\otimes v = (l_1 \otimes v) + (l_2\otimes v)$ \\
$l\otimes(v_1 + v_2) = (l\otimes v_1) + (l \otimes v_2)$ \\
$kl_1 \otimes v =  k(l\otimes v)$\\ 
$l\otimes kv = k(l\otimes v)$\\

И нам осталось проверить, что умножение слева и справа в смысле нашего умножения
сохраняет тождества.

Проверяем:
$\beta(\alpha l \otimes v) = (\beta \alpha l \otimes v)$\\
$= (\alpha \beta l \otimes v) = \alpha (\beta l \otimes v)$\\
$= \alpha \beta(l \otimes v) = (\alpha \beta)(l \otimes v) = \beta \alpha (l \otimes v)$

$\beta(l_1 + l_2)\otimes v = (\beta(l_1 + l_2) \otimes v) = (\beta l_1 + \beta l_2)\otimes v = \\
(\beta l_1) \otimes v + (\beta l_2) \otimes v = \beta(l_1 \otimes v)+ \beta(l_2 \otimes v) = \beta((l_1 \otimes v) + (l_2 \otimes v))$                                       

Теперь нужно проверить,что то, что мы определили дает структуру векторного пространства.\\ 
$L \times V_L \to V_L$\\
Нужно проверить 4 аксиомы, давай-те проверим какую-нибудь одну. 
$\forall v \in V, 1_L V = V$\\
$1_L = 1_K$\\
Для $1_K$ это верно, значит и для $1_L$\\

Ну или можем расписать это подробнее.
$1v = 1\sum \alpha_i(l_i \otimes v_i) = \\
1(\sum (\alpha_i l_i) \otimes v_i) = \sum(\alpha_i l_i)\otimes v_i  =\\
= \sum(\alpha_i l_i \otimes v_i) = \sum \alpha_i(l_i \otimes v_i) = v$


Если $\{e_i\}$ "--- базис $V_K$ над $K$, то
$\{1_L \otimes e_i\}$ "--- базис $L \otimes V$ над $L$.

Как это увидеть. 
$v \in V_L$\\
$v = \sum \alpha_i(l_i \otimes v_i) = \sum \alpha_i(l_i \otimes \sum_{j}\beta_{ij}e_j) = \\
 = \sum_{i}\sum_{j}\alpha_i \beta_{ij} (l_i \otimes e_j) = \sum_{i}\sum_{j}\alpha_i \beta_{ij}l_i(1 \otimes e_i)$\\

Теперь надо понять, почему они линейно независимы над $L$.
$\sum \beta_{j}(1 \otimes e_j) = 0$\\
$\{l_i\}$ "--- базис $L$ над $K$\\
$\beta_j = \sum \gamma_{ij}l_i$\\
$\sum_j \sum_i \gamma_{ij}l_i(1 \otimes e_j) =\\ 
\sum_j \sum_i \gamma_{ij}(l_i \otimes e_j)$  "--- линейно независимы над K, так как образуют тензорный базис.
$\Ra \gamma_{ij} = 0 \Ra \beta_j = 0$\\
 

Есть оператор $\mathcal{A} \in End_K(V_K)$, хотим построить $\mathcal{A} \in End_L(V_L)$, 
которые обладает следующими свойствами:
\begin{enumerate}
\item
$\mathcal{A}_L$ "--- L-линейный оператор.
\item 
$\mathcal{A}_L|_{V_K} = \mathcal{A}$
\end{enumerate}

Зададим его на базисе следующим образом $\mathcal{A}_L(l \otimes v) = (l \otimes \mathcal{A}v)$.
Давай-те распространим это определение по линейности и вспомним, что $l(1 \otimes e_i) = (l \otimes e_i)$\\ 

Но мы не хотим что бы наше построение зависило от выбора базиса, но мы на это уже забили, но если вы все-таки
не хотите, то следующие 50 строчек для вас.

$$\tilde{\mathcal{A}} \colon M \to V_L$$
Расспространяем по $K$-линейности на все $M$.
$$\tilde{\mathcal{A}}((l, v)) = (l, \mathcal{A}v)$$ 
$\tilde A$ - L линейное отображение. 
$a \in L$
$$\tilde{\mathcal{A}}(a(l, v)) = \tilde{A}((al, v)) = (al,  \mathcal{A}v) = a(l,  \mathcal A v)$$

$$M_0 \subset ker \tilde \mathcal{A}$$
\begin{gather*}
\tilde \mathcal{A}((l_1 + l_2, v) - (l_1,  v) - (l_2, v)) = \\
 =\tilde \mathcal{A}((l_1 + l_2, v)) - \tilde \mathcal{A}((l_1, v)) - \tilde \mathcal{A}((l_2, v)) =\\
 =(l_1 + l_2, \mathcal{A}v) - (l_1, \mathcal{A}v) - (l_2, \mathcal{A}v) =\\
 =(l_1 + l_2)(1, \mathcal{A}v) - l_1(1, \mathcal{A}v) - l_2(1, \mathcal{A}v) = 0
\end{gather*}

\begin{gather*}
\mathcal{A}((l, v_1 + v_2) - (l, v_1) - (l, v_2)) =\\
= (l, \mathcal{A}v_1 + \mathcal{A}v_2) - (l, \mathcal{A}v_1) - (l, \mathcal{A}v_2) =\\
= l((1, \mathcal{A}v_1 + \mathcal{A}v_2) - (1, \mathcal{A}v_1) - (1, \mathcal{A}v_2)) \in M_0 
\end{gather*}

$$\tilde \mathcal{A}(M_0) \subset M_0$$

$\tilde \mathcal{A}$ индуцирует L-линейный оператор на $M/M_0$, то есть на $L \otimes_K V$

$$\mathcal{A}_L \colon L \otimes_{K}V \to L \otimes V$$
$$\mathcal{A}_L|_{k \otimes v}\colon K \otimes V \to K \otimes V$$
$$\mathcal{A}_L((l, v) + M_0) = \tilde \mathcal{A}(l, v)$$
Независимость от выбора представителя $\Lra$ $\tilde \mathcal{A}(M_0)\subset M_0$

$$\mathcal{A}_L(x + M_0) = (\tilde \mathcal{A}x) + M_0, x \in M$$
$$x + M_0 = x' + M_0 \Lra \tilde \mathcal{A}(x) = \tilde \mathcal{A}(x') + M_0$$
$$x - x' \in M_0 \Lra \mathcal{A}(x - x') \in M_0$$

Упражнение: $[\mathcal{A}]_{\{e_i\}} = [\mathcal{A}_L]_{\{1 \otimes e_i\}}$

Комплексификация:
$$V_{\R} \to V_{\C} = \C \otimes_{\R} V$$

  
\end{enumerate}

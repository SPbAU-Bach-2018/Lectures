\section{Примеры}
\begin{enumerate}
    \item Пространства функций на декартовых произведениях. 
    $S_1, \cdots S_n$ "--- конечные множества, $K$ "--- поле.\\ 
    $F(S_i)$ множества отображений из $S_i$ в $K$.\\
    
    Существует каноническое отождествление 
    $F(S_1 \times \cdots \times S_n) = F(S_1) \otimes \cdots \otimes F(S_n)$

    $\dim_k F(S_i) = |S_i|$ "--- базис $\delta$ функций (характеристические функции одноэлементных подмножеств).
    $$x \in S_i, \delta_{x}^{(i)} \colon S_i \to K$$
    $$\delta_{x}(x) = 1$$
    $$\delta_{x}(y) = 0, y \ne x$$

     Размерности обоих пространств совпадают: 
    \begin{gather*}
    \dim(F(S_1 \times \cdots \times S_n)) = |S_1 \times \cdots \times S_n| = 
    |S_1| \times \cdots \times |S_n| = \\
    = \dim F(S_1) \times \cdots \times \dim F(S_n) = 
    = \dim (F(S_1) \otimes \cdots \otimes F(S_n))
    \end{gather*}

    Базис $F(S_1 \times \cdots \times S_n)$:\\
    $\delta_{x_1, \cdots, x_n} = \delta_{x_1} \cdot \cdots \cdot \delta_{x_n}$ \\
    $\delta_{x_1, \cdots, x_n}(y_1 \cdots, y_n) = \delta_{x_1}(y_1) \cdot \cdots \cdot \delta_{x_n}(y_n)$

    Базис $F(S_1 \otimes \cdots \otimes S_n)$:\\
    $\{\delta_{x_1}, \cdots, \delta_{x_n}| x_i \in S_i\}$, 
    $\delta_{x_1} \otimes \cdots \otimes \delta_{x_n}$.

    Изоморфизм сопоставляет $\delta_{x_1, \cdots, x_n}$ $\delta_{x_1} \otimes \cdots \otimes \delta_{x_n}$, 
\item
Подъем поля скаляров.\\
Пусть $K \subset L$ "--- поле\\
$V$ "--- векторное пространство над $K$\\
Рассмотрим $L$ как веуторное пространтсво над $K$.\\
$\{e_i\}$ "--- базис $V$ над $K$\\
Построим новое векторное пространство $V_L$ над $L$\\
$$V_L = L \otimes_{K} V = < l \otimes v | l \in L, v \in V >$$
Зададим на $V_L$ структуру векторного пространства над $L$.
$$a(l \otimes v) = (al) \otimes v \colon a \in L $$

Проверим корректность определения.
$$M = <(l, v)>$$
$$a \in L, a(l,v) = (al, v)$$
$M$ приобретает структуру векторного пространства над $L$.
$\delta(kl, v) - k\delta(l, v) \in M_0$ при $k \in K$ $\Lra$ новое умножение 
по модулю $M_0$ согласовано с умножением на скаляр из $K$.

\begin{gather*}
M_0 = <\delta(l_1 + l_2, v) - \delta(l_1, v) - \delta(l_2, v), \\
         \delta(l, v_1 + v_2) - \delta(l, v_1) - \delta(l, v_2), \\
         \delta(kl_1, v) - k\delta(l, v),\\ 
         \delta(l, kv) - k\delta(l, v) | l \in L, k \in K, v_i \in V>
\end{gather*}

$$a \in L$$
\begin{gather*}
a(l_1 + l_2, v) - a(l_1, v) - a(l_2, v) =\\ 
=(a(l_1 + l_2), v) - (al_1, v) -  (al_2, v) \in M_0
\end{gather*}

Аналогично для других соотношений $\Ra M_0$ подпространство над $L$.

$M/M_0$ "--- векторное пространство над $L$.
$$L \times (L \times V) \to L \times V$$
$$a(l, v) \to (al, v)$$ 

Индуцирует 
$$L \times (L \otimes V) \to L \otimes V$$
$$(a, (l \otimes v)) \to (al \otimes v)$$

По линейности продолжаем на все $L \otimes V$

Если $\{e_i\}$ "--- базис $V$ над $K$.
$\{1_L \otimes e_i\}$ "--- базис $L \otimes V$ над $L$.

$$\mathcal{A} \in End_K(V)$$
$$\mathcal{A} \in End_L(V_L)$$
Хотим проверить, что $\mathcal{A}_L(l \otimes v) = (l \otimes \mathcal{A}v)$

$$\tilde{\mathcal{A}} \colon M \to V_L$$
Расспространяем по $K$-линейности на все $M$.
$$\tilde{\mathcal{A}}((l, v)) = (l, \mathcal{A}v)$$ 
$\tilde A$ - L линейное отображение. 
$a \in L$
$$\tilde{\mathcal{A}}(a(l, v)) = \tilde{A}((al, v)) = (al,  \mathcal{A}v) = a(l,  \mathcal A v)$$

$$M_0 \subset ker \tilde \mathcal{A}$$
\begin{gather*}
\tilde \mathcal{A}((l_1 + l_2, v) - (l_1,  v) - (l_2, v)) = \\
 =\tilde \mathcal{A}((l_1 + l_2, v)) - \tilde \mathcal{A}((l_1, v)) - \tilde \mathcal{A}((l_2, v)) =\\
 =(l_1 + l_2, \mathcal{A}v) - (l_1, \mathcal{A}v) - (l_2, \mathcal{A}v) =\\
 =(l_1 + l_2)(1, \mathcal{A}v) - l_1(1, \mathcal{A}v) - l_2(1, \mathcal{A}v) = 0
\end{gather*}

\begin{gather*}
\mathcal{A}((l, v_1 + v_2) - (l, v_1) - (l, v_2)) =\\
= (l, \mathcal{A}v_1 + \mathcal{A}v_2) - (l, \mathcal{A}v_1) - (l, \mathcal{A}v_2) =\\
= l((1, \mathcal{A}v_1 + \mathcal{A}v_2) - (1, \mathcal{A}v_1) - (1, \mathcal{A}v_2)) \in M_0 
\end{gather*}

$$\tilde \mathcal{A}(M_0) \subset M_0$$

$\tilde \mathcal{A}$ индуцирует L-линейный оператор на $M/M_0$, то есть на $L \otimes_K V$

$$\mathcal{A}_L \colon L \otimes_{K}V \to L \otimes V$$
$$\mathcal{A}_L|_{k \otimes v}\colon K \otimes V \to K \otimes V$$
$$\mathcal{A}_L((l, v) + M_0) = \tilde \mathcal{A}(l, v)$$
Независимость от выбора представителя $\Lra$ $\tilde \mathcal{A}(M_0)\subset M_0$

$$\mathcal{A}_L(x + M_0) = (\tilde \mathcal{A}x) + M_0, x \in M$$
$$x + M_0 = x' + M_0 \Lra \tilde \mathcal{A}(x) = \tilde \mathcal{A}(x') + M_0$$
$$x - x' \in M_0 \Lra \mathcal{A}(x - x') \in M_0$$

Упражнение: $[\mathcal{A}]_{\{e_i\}} = [\mathcal{A}_L]_{\{1 \otimes e_i\}}$

Комплексификация:
$$V_{\R} \to V_{\C} = \C \otimes_{\R} V$$

  
\end{enumerate}

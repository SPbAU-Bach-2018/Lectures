\section{Взаимный базис}

\begin{Def}
	$K$ "--- поле, $V$ "---  векторное пространство, $(,)$ "--- форма (билинейная или полуторалинейная).
	$\{e_i\}$ "--- базис $V$.
	$\{e_i^*\}$  взаимный базис к базису $\{e_i\}$, если
	\[ (e_i, e_j^*) = \begin{cases} 1 & i = j \\ 0 & i \ne j \end{cases} \]
\end{Def}

\begin{exmp}
	$\{e_i\}$ "--- ортонормированный базис.
	\[ \{e_i^*\} = \{e_i\} \]
\end{exmp}

\begin{theorem}
	$\dim V = n < \infty$, $\{e_i\}$ "--- базис, $(, )$ "--- невырожденная.
	Тогда существует единственный взаимный базис $\{e_i^*\}$.
\end{theorem}

\begin{proof}
	$\Gamma$ "--- матрица Грамма базиса $\{e_i\}$.
	Разложим $e_j^*$ по базису $e_i$, получим матрицу $C = (c_{kj})$:
	\[ e_j^* = \sum_{k=1}^n c_{kj} e_k \]
	Теперь рассмотрим матрицу из всевозможных скалярных произведений между $e_i$ и $e_j^*$.
	Эта матрица единичная из условия, что базис взаимный:
	\begin{gather*}
		E_n = ((e_i, e_j^*))_{i, j = 1..n} = \left(\left(e_i, \sum_{k = 1}^n c_{kj}e_k\right)\right) = \\
		= \left(\sum_{k=1}^n \bar c_{kj} (e_i, e_k)\right) = \left(\sum_{k} (e_i, e_k) \bar c_{kj} \right) = \Gamma \bar C \\
		E = \Gamma \bar C
	\end{gather*}
	$\Gamma$ "--- невырожденная.
	\[ C = \overline{\Gamma^{-1}} = (\bar \Gamma)^{-1} \]
	$C$ "--- тоже невырожденная, значит $\{e_i^*\}$ "--- базис.
\end{proof}
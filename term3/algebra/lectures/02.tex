\section{Взаимный базис}
$K$ "--- поле.
$V$ "---  векторное пространство. 

(,) "--- форма (билинейная или полуторалинейная). 

$\{e_i\}$ "--- базис V.

\begin{Def}
  
    $\{e_i^*\}$  взаимный базис к базису $\{e_i\}$, если 
    $$(e_i, e_j^*) = \left\{ \begin{array}{cc} 1& (i = j) \\ 0 &(i \ne j) \end{array} \right.$$
\end{Def}

\begin{exmp}

$\{e_i\}$ "--- ортонормированный базис

$\{e_i^*\} = \{e_i\}$
\end{exmp}

\begin{theorem}{}

$\dim V = n < \infty$

$\{e_i\}$ "--- базис   

$(, )$ "--- невырожденная. 

Тогда $\exists$  и единственный взаимный базис $\{e_i^*\}$
\end{theorem}

\begin{proof}

$\Gamma$ "--- матрица Грамма базиса $\{e_i\}$.

$$e_j^* = \sum_{k = 1}^{n} c_{kj}e_k$$
$$C = (c_{kj})$$

Рассмотрим матрицу из всевозможных скалярных произведений.  Эта матрица единичная из условия, что базис взаимный. 

$$E_{n} = ((e_i, e_j^*))_{i, j = 1 \cdots n} = ((e_i, \sum_{k = 1}^{n}c_{kj}e_k)) = $$
$$= (\sum_{k = 1}^{n}\overline{c_{kj}}(e_i, e_k)) = (\sum_{k}(e_i, e_k)\overline{c_{kj}}) = \Gamma \overline{C}$$

$$E = \Gamma \overline{C}$$
$\Gamma$ "--- невырожденная.

$$C = \overline{\Gamma^{-1}} = (\overline{\Gamma})^{-1} $$

$C$ "--- тоже невырожденная, значит $\{e_i^*\}$ "--- базис. 
\end{proof}
\section{Классическое определение тензоров}
$$V\colon e_1, \cdots, e_n$$
$$V^* \colon e^1, \cdots, e^n$$ 

$$T \in T_{p}^q(V)$$

$$T = \sum_{i_1, \cdots i_p, j_1, \cdots, j_q} T_{i_1, \cdots, i_p}^{j_1, \cdots, j_q} e^{i_1} \otimes \cdots \otimes e^{j_p} \otimes e_{j_1} \otimes \cdots \otimes e_{j_q}$$ 

$e_i'$ "--- базис $V$\\
$(e^i)'$ "--- базис $V^*$\\

Матрица перехода от базиса к базису: $e_j' = \sum(A_j^i)e_i$

$$\begin{pmatrix}
A^1_j\\
A^2_j\\
\vdots\\
A^n_j\\
\end{pmatrix}$$
"--- $j$-й столбец матрицы перехода от $\{e\}$ к $\{e'\}$

$$(e^j)' = \sum_{i}B^{j}_ie^i$$

Как связаны матрицы $A, B$.

$$(e^j)'(e_i') = (\sum_k B_{k}^je^k)(\sum_{l}A_{i}^le_l) = \sum_k B_{k}^j A_{i}^k = \sum_k A_i^kB_k^j = \left\{
\begin{aligned} 1, i = j\\ 
0, i \ne j\\
\end{aligned}
\right.$$

$$(A_i^k)(B_k^j) = E$$

Закон преобразования тензоров: 
$$T = \sum T_{i_1 \cdots i_p}^{j_1, \cdots, j_q}e^{i_1} \otimes \cdots \otimes e^{i_p} \otimes e_{j_1} \cdots \otimes e_{j_q}$$
$$T = \sum {T_{k_1, \cdots, k_p}^{l_1, \cdots, l_q}}'{e^{k_1}}' \otimes \cdots \otimes {e^{k_p}}' \otimes {e_{l_1}}' \cdots \otimes {e_{l_q}}' = $$
$$= \sum_{k,l}T'^{l_1, \cdots, l_q}_{k_1, \cdots, k_p}((\sum_{i_1}B_{i_1}^{k_1}e^{i_1}) \otimes \cdots \otimes (\sum A_{l_q}^{j}e_{j_q}))$$

$$\sum_{i, \cdots, i_p, j,\cdots, j_n} (\sum_{k_1,\cdots, k_p, l_1, \cdots, l_p}T'_{k_1, \cdots, k_p, l_1, \cdots, l_n}B_{i_1}^{k_1} \cdots B_{i_p}^{k_p}A_{l_1}^{j_1} \cdots A_{l_q}^{j_q})e^{i_1} \otimes \cdots \otimes e_{j_q}$$


\subsection{Свертка тензоров}
$$T_{p}^{q}(V) \to T^{q - 1}_{p - 1}(V), p, q \ge 1$$
$$T_{i_1, \cdots, i_p}^{j_1, \cdots, j_q}$$
Фиксируем $i_k, j_l$\\
\begin{Def}
$\widetilde{T_{i_1, \cdots, i_{k - 1}, i_{k + 1}, \cdots, i_p}^{j_1, \cdots, j_{l - 1}, j_{l + 1}, \cdots, j_q}} = \sum_{i = 1}^{n}T_{i_1, \cdots, i_{k - 1}, i, i_{k + 1}, \cdots, i_p}^{j_1\cdots, j_{l - 1}, i, \cdots, j_{q}}$
свертка тензора по паре индексов $i_k, j_l$\\
\end{Def}

\begin{exmp}\hfill
\begin{enumerate}
\item
$T \in T_{1}^{1}$\\
$\widetilde{T} = \sum_{i}T_{i}^{i} = tr(T)$  
\item
$T, S \in T_{1}^{1}(V)$\\
$A = T \otimes S \in T^2_2$
$A^{jl}_{ik} = T_i^jS_k^l$
Свертка по $j$ и $k$: $\sum_{r = 1}^{n}T_i^rS^l_r \in T^{1}_1(V)$
Произведение матриц $T$ и $S$.

\end{enumerate}
\end{exmp}
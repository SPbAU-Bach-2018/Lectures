\section{Канонический вид ортогонального оператора в евклидовом пространстве.}

$$\R^n$$
$A$ "--- ортогональная матрица(матрица ортогонального оператора) $A^TA = E$
$$A \in M_n(\R) \subset M_n(\C)$$
$$A^{T}\overline{A} = E$$
$$\overline{A} = A$$

$\Ra A$ "--- унитарная матрица (как комплексная матрица).

Собственные числа A по модулю равны 1. 
$\chi_A \in \R[t]$
Корни $\chi_A$ "--- либо вещественные, либо
распадаются на пары $\lambda, \overline{\lambda}$(причем кратности $\lambda, \overline{\lambda}$ одинаковые).

Значит, вещественные собственные числа A это $\pm 1$.

Комплексные имеют вид $\cos \phi + i \sin \phi$ и $\cos \phi - i \sin \phi$ для некоторых $\phi$.

Существует ортонормированный базис $\C^n$ состоящий из комплексных
собственных векторов $A$, такой что
$C^{-1}AC$ "--- диагональная, на диагонали $\pm 1$ и числа $\cos \phi_k \pm i \sin \phi_k$, где
С "--- матрица перехода от стандартного базиса к этому ортонормированному базису.

Пусть $\lambda = \cos \phi + i \sin \phi, \phi \ne \pi k$и $\overline{\lambda}$ 
это пара комплексных собственных чисел A.

$$\exists v = u + iw, u, w\in \R^n$$
$$Av = \lambda v$$
$$A(u + iw) = (\cos \phi + i \sin \phi)(u + iw)$$


$$\left\{
\begin{aligned}
Au &= \cos \phi u - \sin \phi w \\
Aw &= \cos \phi w + \sin \phi u =  \sin \phi u + \cos \phi w
\end{aligned}
\right.$$


$$Av = \lambda v$$
$$\overline{Av} = \overline{\lambda} \overline{v}$$
$A \overline{v} = \overline{\lambda} \overline{v}$, $\overline{v}$ "--- собственный вектор, отвечающий $\overline{\lambda}(\ne \lambda)$

$v$ и $\overline{v}$ "--- линейно независимые над $\C$ (собственные вектора, отвечающие разным собственным числам.) 

$\frac{v + \overline{v}}{2}$ и $\frac{v - \overline{v}}{2i}$ "--- линейно независимые над $\C$ так как порождают то же пространство, что и $v$, $\overline{v}$.

$u$ и $w$ линейно независимые над $\C \Ra$ линейно независимые над $\R$

Сужение оператора умножения на A на подпространство <u, w> в базисе u w 
имеет вид матрица поворота на угол $\phi$. 
$\begin{pmatrix}
 \cos \phi& -\sin \phi\\
 \sin \phi & \cos \phi\\
\end{pmatrix}$

\begin{theorem}{}
$\mathscr{A}$ "---  ортогональный оператор в $\R^n \exists$ ортонормированный базис пространства $\R^n$, в котором 
матрица $[\mathscr{A}]$ имеет вид: 
$$\begin{pmatrix}
 \cos \phi & -\sin \phi & \cdots\\
 \sin \phi & \cos \phi & \cdots\\
 \vdots & \ddots & \vdots\\
 \cdots & \cdots & \pm 1\\
 \end{pmatrix}$$
\end{theorem}
\begin{proof}
$v, \bar v$ можно дополнить до ортонормированного базиса $\C^n$ в котором матрица оператора умножения на A  диагонализуема
остальные вектора, кроме $v$ и $\bar v$ порождают ортогональное дополнение к <$v, \bar v$> в $\C$. 
\end{proof}

$A$ в стандартном базисе $x \to Ax$ ортогональный оператор $\mathscr{A}$, $[\mathscr{A}] = A$

$\lambda$  и $\bar \lambda$ "--- собственные числа $\mathscr{A}$(как унитарного оператора) $v \in \C^n$

$$|v| = 1$$
$$\mathscr{A}v = \lambda v$$
$$\mathscr{A}\overline{v} = \bar \lambda \bar v$$
$$|\bar v| = |v| = 1$$
Собственные вектора, отвечающие разным собственным числам "--- ортогональные. 

$<v, \bar v>$

$u, w$

$$|v|^2 = |u|^2 + |w|^2$$
$$v = u + iw$$
$$0 = (v, \bar v) = (u + iw, u - iw) = |u|^2 + i(u, w) + i(w, u) - |w|^2 = |u|^2 - |w|^2 + 2i(u, w) $$
$$\left\{
\begin{aligned}
|u|^2 + |w|^2 = 1\\
|u|^2 - |w|^2 = 0\\
(u, w) = 0\\
\end{aligned}
\right.$$

$$|u| = |w| = \frac{1}{\sqrt{2}}$$
$\sqrt{2}u, \sqrt{2} w$ "--- ортонормированный базис в $<u, w>$

$$Au = \cos \phi u - \sin \phi w$$
$$Aw = \sin \phi u + \cos \phi w$$


$$A(\sqrt 2 u) = \cos \phi (\sqrt 2 u) - \sin \phi (\sqrt 2 w)$$
$$A(\sqrt 2 w) = \sin \phi (\sqrt 2 u) + \cos \phi (\sqrt 2 w)$$

$$\mathscr{A}|_{<\sqrt2 u, \sqrt2w>} = \begin{pmatrix}
 \cos \phi & \sin \phi\\
 - \sin \phi & \cos \phi\\
 \end{pmatrix}$$

$\mathscr{A}|_{<u, w>^{\bot}}$ "--- вновь ортогональный оператор. 

$$\mathscr{A}|_{<u, w>^{\bot}} \colon <u, w>^{\bot}\to <u,w>^{\bot}$$
$$x \bot u, w $$
$$Ax \bot u, w$$
$$(\mathscr{A}x, w) = (x, \mathscr{A}^*u) = (x, \mathscr{A}^{-1}u) = 0$$
$\mathscr{A}^{-1}$ на $<u, w>$ действует как поворот в обратную сторону. 

Описан шаг индукции, если есть пара комплексных собственных чисел, 
если остались только вещественные, то они должны быть равны $\pm 1$ 
В каждом из собственных подпространств $U(1), U(-1)$ выбираем по ортонормированному базису. 

\begin{exmp}
$\R^3$

$$O(3) = \{ A \in GL_3(\R) \colon A^{T}A = E\} \Ra \det A^2 = 1, \det A = \pm 1$$
$SO(3) = \{A \in O(3)\colon \det A = 1\} $ "--- специальная ортогональная группа. 
\end{exmp}

\textbf{Упражнение:} Докажите, что ортогональное преобразование $\R^n$ есть 
композиция не более, чем n отображений. 


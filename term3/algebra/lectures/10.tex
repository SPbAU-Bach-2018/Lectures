\section{Канонический вид ортогонального оператора в евклидовом пространстве.}

$\R^n$.
$A$ "--- ортогональная матрица (матрица ортогонального оператора), $A^TA = E$.
\begin{gather*}
	A \in M_n(\R) \subset M_n(\C) \\
	A^{T}\bar A = E \\
	\bar A = A
\end{gather*}
Отсюда $A$ "--- унитарная матрица (как комплексная матрица).

Собственные числа $A$ по модулю равны 1.
\[ \chi_A \in \R[t] \]
$$\chi_A = \det(A - tE)$$
Корни $\chi_A$ "--- либо вещественные, либо
распадаются на пары $\lambda, \bar \lambda$ (причем кратности $\lambda, \bar \lambda$ одинаковые).

Значит, вещественные собственные числа $A$ это $\pm 1$.
Комплексные имеют вид $\cos \phi + i \sin \phi$ и $\cos \phi - i \sin \phi$ для некоторых $\phi$.

Существует ортонормированный базис $\C^n$, состоящий из комплексных собственных векторов $A$,
такой что $C^{-1}AC$ "--- диагональная, на диагонали $\pm 1$ и числа $\cos \phi_k \pm i \sin \phi_k$,
где $С$ "--- матрица перехода от стандартного базиса к этому ортонормированному базису.

Пусть $\lambda = \cos \phi + i \sin \phi$ ($\phi \ne \pi k$) и $\bar \lambda$ это пара комплексных собственных чисел $A$.
\begin{gather*}
	\exists v = u + iw, u, w\in \R^n \\
	Av = \lambda v \\
	A(u + iw) = (\cos \phi + i \sin \phi)(u + iw) \\
	\left\{\begin{aligned}
		Au &= \cos \phi u - \sin \phi w \\
		Aw &= \cos \phi w + \sin \phi u = \sin \phi u + \cos \phi w
	\end{aligned}\right.\\
\end{gather*}
u и w "--- линейно независимые, пространство, поражденное u и w инвариантно относительно A.
\begin{gather*}	
	Av = \lambda v \\
	\overline{Av} = \bar\lambda \bar v
\end{gather*}
$A \bar v = \bar\lambda \bar v$, $\bar v$ "--- собственный вектор, отвечающий $\bar \lambda \ne \lambda$.

$v$ и $\bar v$ "--- линейно независимые над $\C$ (собственные вектора, отвечающие разным собственным числам).
$(v + \bar v)/2$ и $(v - \bar v)/2i$ "--- линейно независимые над $\C$, так как порождают то же пространство, что и $v, \bar v$.
$u$ и $w$ линейно независимые над $\C$, значит они линейно независимые над $\R$.

Сужение оператора умножения на A на подпространство $\left<u, w\right>$ в базисе $(u, w)$
имеет вид матрица поворота на угол $\phi$. 
\[ \begin{pmatrix}
	\cos \phi & -\sin \phi \\
 	\sin \phi &  \cos \phi
\end{pmatrix} \]                    

\begin{theorem}
	$\mathcal A$ "--- ортогональный оператор в $\R^n$.
	Существует ортонормированный базис пространства $\R^n$, в котором матрица $[\mathcal A]$ имеет вид:
	\[ \begin{pmatrix}
 		\cos \phi & -\sin \phi & \cdots\\
 		\sin \phi & \cos \phi & \cdots\\
 		\vdots & \ddots & \vdots\\
 		\cdots & \cdots & \pm 1\\
 	\end{pmatrix} \]
\end{theorem}
\begin{proof}
	$v, \bar v$ можно дополнить до ортонормированного базиса $\C^n$, в котором матрица оператора умножения на $A$ диагонализуема.
	Остальные вектора, кроме $v$ и $\bar v$, порождают ортогональное дополнение к $\left<v, \bar v\right>$ в $\C$.
\end{proof}

$A$ в стандартном базисе, $x \to Ax$ "--- ортогональный оператор $\mathcal A$, $[\mathcal A] = A$.
$\lambda$ и $\bar \lambda$ "--- собственные числа $\mathcal A$ (как унитарного оператора), $v \in \C^n$.
\begin{gather*}
	|v| = 1 \\
	\mathcal Av = \lambda v \\
	\mathcal A\bar v = \bar \lambda \bar v \\
	|\bar v| = |v| = 1
\end{gather*}
Собственные вектора, отвечающие разным собственным числам "--- ортогональные. 

Рассмотрим $\left<v, \bar v\right>$, $(u, w)$.
\begin{gather*}
	|v|^2 = |u|^2 + |w|^2 \\
	v = u + iw \\
	0 = (v, \bar v) = (u + iw, u - iw) = |u|^2 + i(u, w) + i(w, u) - |w|^2 = |u|^2 - |w|^2 + 2i(u, w) \\
	\left\{ \begin{aligned}
		|u|^2 + |w|^2 &= 1 \\
		|u|^2 - |w|^2 &= 0 \\
		(u, w) &= 0
	\end{aligned}\right. \\
	|u| = |w| = \frac1{\sqrt2}
\end{gather*}
$\sqrt2u, \sqrt2 w$ "--- ортонормированный базис в $\left<u, w\right>$
\begin{gather*}
	Au = \cos \phi u - \sin \phi w \\
	Aw = \sin \phi u + \cos \phi w \\
	A(\sqrt 2 u) = \cos \phi (\sqrt 2 u) - \sin \phi (\sqrt 2 w) \\
	A(\sqrt 2 w) = \sin \phi (\sqrt 2 u) + \cos \phi (\sqrt 2 w) \\
	\mathcal A|_{\left<\sqrt2 u, \sqrt2w\right>} = \begin{pmatrix}
 		 \cos \phi & \sin \phi \\
		-\sin \phi & \cos \phi
 	\end{pmatrix}
\end{gather*}
$\mathcal A|_{\left<u, w\right>^{\bot}}$ "--- вновь ортогональный оператор. 
\begin{gather*}
	\mathcal A|_{\left<u, w\right>^{\bot}} \colon \left<u, w\right>^{\bot}\to \left<u,w\right>^{\bot} \\
	x \bot u, w \\
	Ax \bot u, w \\
	(\mathcal Ax, w) = (x, \mathcal A^*u) = (x, \mathcal A^{-1}u) = 0
\end{gather*}
$\mathcal A^{-1}$ на $\left<u, w\right>$ действует как поворот в обратную сторону. 

Описан шаг индукции, если есть пара комплексных собственных чисел,
если остались только вещественные, то они должны быть равны $\pm 1$.
В каждом из собственных подпространств $U(1), U(-1)$ выбираем по ортонормированному базису. 

\begin{exmp}
	$\R^3$
	\[ O(3) = \{A \in GL_3(\R) \mid A^{T}A = E\} \Ra \det A^2 = 1, \det A = \pm 1 \]
	$O(3) = \{A \in O(3) \mid \det A = 1\} $ "--- специальная ортогональная группа. 
\end{exmp}

\textbf{Упражнение:}
Докажите, что ортогональное преобразование $\R^n$ есть композиция не более, чем $n$ отображений.

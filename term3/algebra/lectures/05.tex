\section{Сопряженный оператор}

\begin{Def}
	$V$ "--- пространство с $(,)$ билинейной или полуторалинейной невырожденной формой, $\mathcal A \in End(V)$.\\
	$\mathcal A^* \in End(V)$ называется сопряжённым оператором к $A$, если
$$\forall x, y \in V \colon (\mathcal Ax, y) = (x, \mathcal A^*y)$$
\end{Def}

\begin{theorem}
	$\dim V < \infty$, тогда $\mathcal A^*$ существует и единственный.
\end{theorem}
\begin{proof}
	\begin{description}
	\item[Единственность:]
		Предположим, что $\mathcal A^*$ существует.
		Убедимся, что существует всего один.

		$v_1, \dots, v_n$ "--- базис $V$,
		$\Gamma$ "--- матрица Грамма.
		\begin{gather*}
			(\mathcal Ax, y) = [\mathcal Ax]^T \Gamma \overline{[y]}
			= ([\mathcal A][x])^T \Gamma \overline{[y]}
			= [x]^T [\mathcal A]^T \Gamma \overline{[y]} \\
			(x, \mathcal A^* y) = [x]^T \Gamma \overline{[\mathcal A^*]} \overline{[y]}
		\end{gather*}
		То есть для всех $x$ и $y$ верно, что $[\mathcal A]^T \Gamma = \Gamma \overline{[\mathcal A^*]}$.

		$(,)$ невырожденная, откуда $\Gamma$ "--- невырожденная.
		\[ \overline{[\mathcal A^*]} = \Gamma^{-1} [\mathcal A]^T \Gamma \]
		\begin{equation}\label{conj}
			[\mathcal A^*] = \overline{\Gamma^{-1}} \overline{[\mathcal A]^T}\overline{\Gamma}
		\end{equation}
		Значит $\mathcal A^*$ определен однозначно.

	\item[Существование:]
		Возьмем $\mathcal A^*$ так, чтобы матрица $\mathcal A^*$ в базисе $v_1, \dots, v_n$ задавалась равенством~(\ref{conj}).
		\begin{gather*}
			[\mathcal A^*] = \overline{\Gamma^{-1}} \overline{[\mathcal A]^T} \overline{\Gamma} \\
			\overline{[\mathcal A^*]} = \Gamma^{-1} [\mathcal A]^T \Gamma \\
			[\mathcal A]^T \Gamma = \Gamma \overline{[\mathcal A^*]} \\
			(\mathcal Ax, y) = [x]^T [\mathcal A]^T \Gamma \overline{[y]}
			= [x]^T \Gamma \overline{[\mathcal A^*]} \overline{[y]} = (x, \mathcal A^*y)
			\Ra (\mathcal Ax, y) = (x,  \mathcal A^*y)
		\end{gather*}
		В силу произвольности $x$ и $y$ получичи, что $\mathcal A^*$ "--- сопряженный оператор.
	\end{description}
\end{proof}

\begin{conseq}\hfill
	\begin{enumerate}
	\item
		$[\mathcal A^*] = \overline{\Gamma^{-1}} \overline{[\mathcal A]^T} \overline{\Gamma}$.

	\item
		Если есть ортонормированный базис, то в этом базисе
		$[\mathcal A^*] = \overline{[\mathcal A]^T}$ "--- матрица, эрмитово сопряженная с $[\mathcal A]$.\\
		В случае билинейной формы в ортонормированном базисе $[\mathcal A^*] = [\mathcal A]^T$.
	\end{enumerate}
\end{conseq}

\begin{theorem}
	$V, (,)$ "--- невырожденная форма, $U \le V$, $\mathcal A \in End(V)$.
	$U$ "--- $\mathcal A$-инвариантно.
	Тогда
	$U^\bot$ "--- $\mathcal A^*$-инвариантно.
\end{theorem}
\begin{proof}
	$w \in U^\bot$.
	Хотим проверить, что $\mathcal A^*w \in U^\bot$.

	$u \in U$
	\[ (u, \mathcal A^* w) = (\mathcal A u, w) = 0 \]
	$\Ra \mathcal A^* w \in U^\bot$.
	В силу произвольности w получили, что $U^\bot$ "--- $\mathcal A^*$-инвариантно.
\end{proof}

\begin{exmp}
	$U \le V$, $\dim V < \infty$.
	Пусть есть ортонормированный базис $U$, дополненный до ортонормированного базиса $V$.
	\[
		[\mathcal A] = \begin{pmatrix}
			\overbrace{*}^U & \overbrace{*}^{U^\bot} \\
			0 & *
			\end{pmatrix}
	\]
	$U$ "--- $\mathcal A$ инвариантно.
	\[
		[\mathcal A^*] = \overline{[\mathcal A]^T} = \begin{pmatrix}
			\overbrace{*}^U & \overbrace{0}^{U^\bot} \\
			* & *
		\end{pmatrix}
	\]
\end{exmp}

\begin{theorem}
	$\dim V < \infty$,
	$(,)$ "--- билинейная (симметрическая или кососимметрическая) или полуторалинейная и эрмитовосмметричная невырожденная форма.
	Тогда
	\[ (\mathcal A^*)^* = \mathcal A\]
\end{theorem}
\begin{proof}
	\begin{gather*}
		\forall x, y, (\mathcal Ax, y) = (x, \mathcal A^* y) \\
		\epsilon \overline{(y, \mathcal A x)} = (\mathcal Ax, y) = (x, \mathcal A^* y) = \epsilon \overline{(\mathcal A^*y, x)}
	\end{gather*}
	$\epsilon = -1$ для кососимметричной, $\epsilon = +1$ для симметричной и эрмитовосимметричной. \\
	$\bar{\vphantom{x}}$ "--- инволюция для полуторалинейной формы, $id$ "--- для билинейной формы.
	\[ \forall x, y, (\mathcal A^*y, x) = (y, \mathcal Ax) \Ra \mathcal A = (\mathcal A^*)^* \]
\end{proof}

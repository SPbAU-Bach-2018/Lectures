\section{Сопряженный оператор}
\begin{Def}
$V$ "--- пространство с $(, )$ билинейной или полуторалинейной невырожденной формой. 

$\mathscr{A} \in End(V)$

$\mathscr{A}^* \in End(V)$ называется сопряжённым оператором к $A$, если 
$$\forall x, y \in V \colon (\mathscr{A}x, y) = (x, \mathscr{A}^*y)$$
\end{Def}

\begin{theorem}{}
$\dim V < \infty$, тогда $\mathscr{A}^*$ существует и единственный. 
\end{theorem}
\begin{proof}
    \begin{enumerate}
    \item Предположим, что $\mathscr{A}^{*}$ существует. Убедимся, что существует всего одна. 
    
    $v_1, \cdots, v_n$ "--- базис V.
    
    $\Gamma$  "--- матрица Грамма.

    $$(\mathscr{A}x, y) = [\mathscr{A}x]^{T}\Gamma \overline{[y]} = ([\mathscr{A}][x])^{T}\Gamma \overline{[y]} = 
    [x]^{T}[\mathscr{A}]^{T}\Gamma \overline{[y]}$$ 
    $$(x, \mathscr{A}^{*}y) = [x]^{T}\Gamma \overline{[\mathscr{A}^*]}\overline{[y]}$$ 

    То есть для всех x и y верно, что $[\mathscr{A}]^{T} \Gamma = \Gamma \overline{[\mathscr{A}^*]}$

    $(,)$ "--- невырожденная $\Ra \Gamma$ "--- невырожденная

    $$\overline{[\mathscr{A}^*]} = \Gamma^{-1}[\mathscr{A}]^{T}\Gamma$$
    
    \begin{equation}
    \label{conj}
    [\mathscr{A}^{*}] = \overline{\Gamma^{-1}}\overline{[\mathscr{A}]^{T}}\overline{\Gamma}   
    \end{equation}
   
    $\Ra \mathscr{A}^{*}$ определен однозначно.

    \item Существование.

    Возьмем $\mathscr{A}^{*}$ так, чтобы матрица $\mathscr{A}^*$ в базисе $v_1,\cdots, v_n$ задавалась равенством ~(\ref{conj})

    $$[\mathscr{A}^{*}] = \overline{\Gamma^{-1}}\overline{[\mathscr{A}]^{T}}\overline{\Gamma}$$
    $$\overline{[\mathscr{A}^*]} = \Gamma^{-1}[\mathscr{A}]^{T}\Gamma$$
    $$[\mathscr{A}]^{T} \Gamma = \Gamma \overline{[\mathscr{A}^*]}$$

    $$(\mathscr{A}x, y) = [x]^{T}[\mathscr{A}]^{T}\Gamma \overline{[y]} = [x]^{T}\Gamma \overline{[\mathscr{A}^*]}\overline{[y]} = (x, \mathscr{A}^{*}y)$$ 
    
    $$\Ra (\mathscr{A}x, y) = (x,  \mathscr{A}^*y)$$

    В силу произвольности x и y $\mathscr{A}^*$ "--- сопряженный оператор.  
    \end{enumerate}
\end{proof}

\begin{conseq}
    \begin{enumerate}
    \item $[\mathscr{A}^*] = \overline{\Gamma^{-1}}\overline{[\mathscr{A}]^{T}}\overline{\Gamma}$
    \item Если есть ортонормированный базис,то в этом базисе $[\mathscr{A}^*] = \overline{[\mathscr{A}]^T}$ "--- матрица, эрмитовосопряженная с $[\mathscr{A}]$

    В случае билинейной формы в ортонормированном базисе $[\mathscr{A}^*] = [\mathscr{A}]^{T}$
    \end{enumerate}
\end{conseq}

\begin{theorem}{}
$V, (,)$ "--- невырожденная форма. 

$U \le V, \mathscr{A} \in End(V)$

$U$ "--- $\mathscr{A}$-инвариантно. 

$\Ra U^{\bot}$ "--- $\mathscr{A}^{*}$ "--- инвариантно. 
\end{theorem}

\begin{proof}
$$w \in U^{\bot}$$ 
Хотим проверить, что $\mathscr{A}^{*}w \in U^{\bot}$

$$u \in U$$
$$(u, \mathscr{A}^*w) = (\mathscr{A}u, w) = 0$$
$\Ra \mathscr{A}^{*}w \in U^{\bot} \Ra U^{\bot}$ "--- $\mathscr{A}^{*}$ "--- инвариантно. 
\end{proof}

\begin{exmp}
$U \le V, dim V < \infty$

Пусть есть ортонормированный базис $U$, дополненный до ортонормированного базиса $V$.


$$[\mathscr{A}] = \begin{pmatrix}
 \overbrace{*}^U & \overbrace{*}^{U^{\bot}} \\
 0 & * \\
 \end{pmatrix}$$

$U$ "--- $\mathscr{A}$ инвариантно.  

$$ 
[\mathscr{A}^{*}] = \overline{[\mathscr{A}]^T} = 
\begin{pmatrix}
\overbrace{*}^U& \overbrace{0}^{U^{\bot}}\\
*& *\\
\end{pmatrix}
$$
\end{exmp}

\begin{theorem}{}
$\dim V < \infty$

$(,)$ "--- билинейная (симметрическая или кососимметрическая) или полуторалинейная и эрмитовосмметричная невырожденная форма. 

$\Ra (\mathscr{A}^*)^* = \mathscr{A}$
\end{theorem}

\begin{proof}
$$\forall x, y \colon (\mathscr{A}x, y) = (x, \mathscr{A}^* y) $$
$$\epsilon \overline(y, \mathscr{A} x) = (\mathscr{A}x, y) = (x, \mathscr{A}^* y) = \epsilon \overline{\mathscr{A}^*y, x}$$

$\epsilon = -1$ для кососимметричной. 

$\epsilon = +1$ для симметричной и эрмитовосимметричной. 

$-$ "--- инволюция для полуторалинейной формы. 

$- = id$ "--- для билинейной формы. 

$\forall x, y \colon (\mathscr{A}^{*}y, x) = (y, \mathscr{A}x) \Ra \mathscr{A} = (\mathscr{A}^*)^*$
\end{proof}

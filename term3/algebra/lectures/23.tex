\chapter{Теория групп (продолжение)}
\begin{Rem}
$G$ "--- группа, $H \le G$, $H$ "--- подгруппа G\\

$0 \ne A, B \le G$\\
Произведение по Минковскому:\\
$A \cdot B = \{ab| a \in A, b \in B\}$

Если $A = \{a\}$, будем писать $aB$ или $Ba$ вместо $\{a\}B$, $B\{a\}$

В частности, если $B \le G$\\
    $aB$ "--- левый класс смежности по B\\
    $Ba$ "--- правый класс смежности по B\\
\end{Rem}

\begin{conseq}\hfill
\begin{enumerate}
\item$(AB)C = A(BC) = \{abc|a \in A, b \in B, c \in C\}$\\
\item
Если $H \le G$, то $H \cdot H = H$\\
   $H \cdot H \le H$(замкнуто относительно умножения)\\
   $H = 1 H \le H \cdot H$\\
\end{enumerate}
\end{conseq}
\begin{Def}
$0 \ne A \le G$\\
$A^{-1} = \{a^{-1}|a \in A\}$\\
$H \le G, H^{-1} = H$\\
\end{Def}
\begin{Rem}
Чаще всего будем использвать мультипликативную запись.\\
Нейтральный элемент $G$ "--- $1_{G} = 1$
Если будет  встречаться аддитивная запись, то нейтральный элемент $0$.
\end{Rem}

\begin{Def}
$G, G_1$ "--- группа\\
$\phi \colon G \to G_1$ называется гомоморфизмом групп. 

если $\forall a, b \in G, \phi(ab) = \phi(a) \phi(b)$

В частности $\phi(1_G) = 1_{G_1}, \phi(a^{-1}) = \phi(a)^{-1}$\\

Изоморфизм $\phi \colon G \to G_1$ это биекция и гомоморфизм \\
\end{Def}
    

\section{Нормальные подгруппы}
\begin{theorem}
 Пусть $H \le G$, тогда следующие условия равносильны:
 \begin{enumerate}
 \item $\forall g \in G gHg^{-1} \le H$\\
 \item $\forall g \in G gHg^{-1} = H$ \\
 \item $\forall g \in G gH = Hg$\\
 \end{enumerate}
\end{theorem}   
\begin{proof}
$2 \Ra 1$ очевидно\\
$1 \Ra 2$ В силу произвольности g:\\
$gHg^{-1} \le H$\\
$g^{-1}Hg \le H \Ra H \le gHg^{-1} \Ra H = gHg^{-1}$\\
$2 \Ra 3$ домножаем $gHg^{-1} = H$ на g справа\\
$3 \Ra 2$ домножаем $gH = Hg$ домножаем на $g^{-1}$ слева\\
\end{proof}

\begin{Def}
Подгруппа $H \le G$, удовлетворяющий любому из свойств, 
перечисленных в теореме 1, называется нормальной подгруппой.

Обозначение: $H \unlhd G$\\
\end{Def}

\begin{conseq}
$H \unlhd G$, то левые классы смежности по $H$ совпадают с правыми классами смежности
\end{conseq}
\begin{exmp}\hfill
\begin{enumerate}
    \item $\{1\} \unlhd G, G \unlhd G$ \\
    $g\{1\} = \{g\} = \{1\}g$\\
    \item $G = S_3$\\
    $<(123)> \unlhd S_3$\\
    $<(12)> \ntrianglelefteq S_3$\\
    \item $SL(n, K) \unlhd GL(n, K)$\\
    $GL(n, K)$ все матрицы $n \times n$ над $K$.
    $SL(n, K) = \{g \in GL(n, K)\colon\det(g) = 1\}$
    \item 
    $[G:H] = 2 \Ra H \unlhd G$
    индекс подгруппы H в G.

    Классы смежности $H, G \setminus H$

    $G = \cup gH$\\
    Если $g \in H$, то $gH = H = Hg$\\
    Если $g \ne H$, то $gH = G\setminus H = Hg$\\

    $g_1H = g_2H \Lra g_2^{-1}g_1 \in H$ \\
    Возьмите $g_2 = 1$\\
    $g_1H = H \Lra g_1 \in H$\\
\end{enumerate}
\end{exmp}

\begin{theorem}
Пусть $\phi\colon G \to G_1$, $\phi$ "--- гомоморфизм\\
$\ker \phi = \phi^{-1}(\{1_{G_1}\}) = \{g \in G \colon \phi(g) = 1_{G_1}\}$ \\
Тогда $\ker \phi \unlhd G$\\
\end{theorem}
\begin{proof}
$\ker \phi \le G$ "--- упражнение\\
$g \in G, h \in \ker \phi$\\
$ghg^{-1} \in \ker \phi$\\
$\phi(ghg^{-1}) = \phi(g)\phi(h)\phi(g^{-1}) = \phi(g)1_{G_1}\phi(g^{-1}) = \phi(g)\phi(g^{-1}) = \phi(gg^{-1}) = \phi(1) = 1_{G_1}$\\
\end{proof}
\begin{exmp}\hfill
\begin{enumerate} 
\item
$A_n \unlhd S_n$"--- группа четных перестановок\\
$\phi \colon S_n \to \{\pm 1\}$\\
$\phi(\sigma) = (-1)^{\#\sigma}$ \\
Ядро это четные перестановки.
$\ker \phi  = A_n$

\item 
$\phi \colon GL(n,K) \to K^*$\\
$\phi(g) = \det g$\\
$\ker(\phi) = SL(n, K) \unlhd GL(n, K)$\\
\end{enumerate}
\end{exmp}

\begin{lemma}
$\phi\colon G \to G_1$ "--- гомоморфизм\\
$H = \ker \phi \Ra$ смежные классы по $H$ "--- это в точности $\phi^{-1}(\{b\})$, где $b \in \phi(G)$\\
\end{lemma}
\begin{proof}
$gH = \phi^{-1}(\{\phi(g)\})$\\
$\subset$\\
$g_1 \in gH$\\
$g_1 = gh$\\
$\phi(g_1) = \phi(g)\phi(h) = \phi(g)1$ \\
$\Ra g_1 \in \phi^{-1}(\{\phi(g)\})$\\
$\supset$\\
$g_1 \in \phi^{-1}(\{\phi(g)\})$ \\
$\phi(g_1) = \phi(g)$\\
$\phi(g^{-1}g_1) = 1 \Ra g^{-1}g_1 \in \ker \phi = H$\\
$g_1 = gg^{-1}g_1 \in gH$
\end{proof}



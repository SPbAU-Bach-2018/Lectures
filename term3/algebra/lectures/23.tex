\chapter{Теория групп (продолжение)}

\section{Вступление}

\begin{Def} \hfill \\
	$G$ "--- группа, $\emptyset \ne A, B \subset G$.
	Произведение по Минковскому:
	\[ A \cdot B = \{ab \mid a \in A, b \in B\} \]
\end{Def}                                                       
\begin{Def} \hfill \\
        Если $A = \{a\}$, будем писать $aB$ или $Ba$ вместо $\{a\}B$, $B\{a\}$.\\
	В частности, если $B \le G$,\\
	$aB$ "--- левый класс смежности по $B$,\\
	$Ba$ "--- правый класс смежности по $B$.\\
\end{Def}

\begin{conseq}\hfill
	\begin{enumerate}
	\item
	$(AB)C = A(BC) = \{abc \mid a \in A, b \in B, c \in C\}$
	\item
		Если $H \le G$, то $H \cdot H = H$:\\
		$H \cdot H \le H$ (замкнуто относительно умножения),\\
		$H = 1 \cdot H \le H \cdot H$.
	\end{enumerate}
\end{conseq}

\begin{Def}
	$\emptyset \ne A \le G$.  \\
	$A^{-1} = \{a^{-1} \mid a \in A\}$ \\
	Если $H \le G \Ra H^{-1} = H$.
\end{Def}

\begin{Rem}
	Чаще всего будем использвать мультипликативную запись.
	Нейтральный элемент $G$ "--- $1_{G} = 1$.
	Если будет встречаться аддитивная запись, то нейтральный элемент "--- $0$.
\end{Rem}

\begin{Def}
	$G$, $G_1$ "--- группы.
	$\phi \colon G \to G_1$ называется гомоморфизмом групп, если
	\[ \forall a, b \in G, \phi(ab) = \phi(a) \phi(b) \]
	В частности $\phi(1_G) = 1_{G_1}, \phi(a^{-1}) = \phi(a)^{-1}$.
\end{Def}
\begin{Def}
	Изоморфизм $\phi\colon G \to G_1$ "--- это биекция и гомоморфизм.
\end{Def}

\section{Нормальные подгруппы}

\begin{theorem}
	Пусть $H \le G$, тогда следующие условия равносильны:
	\begin{enumerate}
		\item $\forall g \in G \colon gHg^{-1} \le H$
		\item $\forall g \in G \colon gHg^{-1} = H$
		\item $\forall g \in G \colon gH = Hg$
	\end{enumerate}
\end{theorem}
\begin{proof}\begin{description}
\item[$2 \Ra 1$:]
	Очевидно.

\item[$1 \Ra 2$:]
	В силу произвольности $g$:
	\begin{gather*}
		gHg^{-1} \le H \\
		g^{-1}Hg \le H \Ra H \le gHg^{-1} \Ra H = gHg^{-1}
	\end{gather*}

\item[$2 \Ra 3$:]
	Домножаем $gHg^{-1} = H$ на $g$ справа.

\item[$3 \Ra 2$:]
	Домножаем $gH = Hg$ домножаем на $g^{-1}$ слева.
\end{description}\end{proof}

\begin{Def}
	Подгруппа $H \le G$, удовлетворяющий любому из свойств, перечисленных в теореме 1, называется нормальной подгруппой.
	Обозначение: $H \unlhd G$.
\end{Def}

\begin{conseq}
	Если $H \unlhd G$, то левые классы смежности по $H$ совпадают с правыми классами смежности.
\end{conseq}
\begin{exmp}\hfill\begin{enumerate}
\item
	$\{1\} \unlhd G$, $G \unlhd G$:
	\[ g\{1\} = \{g\} = \{1\}g \]

\item
	$G = S_3$:
	\begin{align*}
		\left<(123)\right> &\unlhd S_3 \\
		\left<(12)\right> &\ntrianglelefteq S_3
	\end{align*}

\item
	\[ SL(n, K) \unlhd GL(n, K) \]
	$GL(n, K)$ "--- все обратимые матрицы $n \times n$ над $K$,
	$SL(n, K) = \{g \in GL(n, K) \mid \det g = 1 \}$.

\item
	\[ [G:H] = 2 \Ra H \unlhd G \]
	$[G:H]$ "--- индекс подгруппы $H$ в $G$.\\
	индекс подгруппы "--- количество классов смежности.\\
	Классов смежности два, $H$ и $G \setminus H$:
	\[ G = \bigcup gH \]
	\begin{itemize}
		\item Если $g \in H$, то $gH = H = Hg$.
		\item Если $g \ne H$, то $gH = G \setminus H = Hg$.
	\end{itemize}
	\[ g_1H = g_2H \Lra g_2^{-1}g_1 \in H \]
	Возьмите $g_2 = 1$:
	\[ g_1H = H \Lra g_1 \in H \]
\end{enumerate}\end{exmp}

\begin{Def}
\[ \ker \phi = \phi^{-1}(\{1_{G_1}\}) = \{g \in G \mid \phi(g) = 1_{G_1}\} \]
\end{Def}

\begin{theorem}
	Пусть $\phi\colon G \to G_1$, $\phi$ "--- гомоморфизм,
	тогда $\ker \phi \unlhd G$.
\end{theorem}
\begin{proof}
	$\ker \phi \le G$ "--- упражнение.

	$g \in G$, $h \in \ker \phi$.
	Покажем, что $ghg^{-1} \in \ker \phi$:
	\[
		\phi(ghg^{-1}) = \phi(g)\phi(h)\phi(g^{-1}) = \phi(g)1_{G_1}\phi(g^{-1}) = \phi(g)\phi(g^{-1}) = \phi(gg^{-1}) = \phi(1) = 1_{G_1}
	\]
\end{proof}

\begin{exmp}\hfill\begin{enumerate}
\item
	$A_n \unlhd S_n$"--- группа чётных перестановок.
	\[ \phi\colon S_n \to \{\pm 1\} \quad \phi(\sigma) = (-1)^{\#\sigma} \]
	Ядро "--- это чётные перестановки.
	\[ \ker \phi = A_n \]

\item
	\begin{gather*}
		\phi\colon GL(n,K) \to K^* \quad \phi(g) = \det g \\
		\ker \phi = SL(n, K) \unlhd GL(n, K)
	\end{gather*}
	$K^*$ "--- обратимые элементы, то естьв данном случае K без 0.
\end{enumerate}\end{exmp}

\begin{lemma}
	$\phi\colon G \to G_1$ "--- гомоморфизм.\\
	Если $H = \ker \phi$, то смежные классы по $H$ "--- это в точности $\phi^{-1}(\{b\})$, где $b \in \phi(G)$.
        
        Другими словами,$\forall g \in G \colon gH = \{\phi^{-1}(\phi(g))\}$\\
\end{lemma}
\begin{proof}
	Покажем, что $gH = \phi^{-1}(\{\phi(g)\})$ ($b = \phi(g)$):
	\begin{description}
	\item[$\subset$:] \hfill \\
		$g_1 \in gH$, $g_1 = gh$:
		\[ \phi(g_1) = \phi(g)\phi(h) = \phi(g) \Ra \\ g_1 \in \phi^{-1}(\{\phi(g)\}) \]

	\item[$\supset$:]\hfill \\
		$g_1 \in \phi^{-1}(\{\phi(g)\})$:
		\begin{gather*}
			\phi(g_1) = \phi(g) \\
			\phi(g^{-1}g_1) = 1 \Ra g^{-1}g_1 \in \ker \phi = H \\
			g_1 = gg^{-1}g_1 \in gH
		\end{gather*}
	\end{description}
\end{proof}


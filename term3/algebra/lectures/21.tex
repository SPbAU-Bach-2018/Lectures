\setauthor{Черникова Ольга}
\section{Тензорная алгебра векторного пространства}
$V$ "--- векторное пространство над $K$. $(\dim_{K}V < \infty)$\\
Рассмотрим $T_p^q = \underbrace{V^*\otimes \cdots \otimes V^*}_{p} \otimes \underbrace{V \otimes \cdots \otimes V}_{q}$\\
Элементы $T_{p}^q(V)$ "--- тензоры валентности $(p, q)$. ($p$ раз ковариантные, $q$ раз контрвариантные)

$T_0^1(V) = V$ (элементы $T_0^1$ "--- векторы)\\
$T_1^0(V) = V^*$ (элементы $T_1^0$ "--- ковекторы)\\
$$v^* \in T_1^0(V), v^* \colon V \to K$$
$T_1^1(V) = V^* \otimes V \cong \mathcal{L}(V, V) = End(V) \cong M_n(K)$
Можно понимать как матрица линейных отображений. 

$T_2^0(V) = V^*\otimes V^* \cong (V \otimes V)^* = \mathcal{L}(V \otimes V, K) \cong \mathcal{L}(V, V, K)$ 

Элементы $T_2^0(V)$ "--- билинейные формы на $V$.

\subsection{Алгебра и их структурные константы}
\begin{Def}
Алгебра($A$) над полем $K$ "--- это алгебраическая  структура, которая является векторным пространством
над $K$ и на котором задано <<умножение>>, то есть билинейное отображение $\colon A \times A \to A$

$$+ \colon A \times A \to A$$
$$\cdot \colon K \times A \to A$$
$$\cdot \colon A \times A \to A$$

\end{Def}
\begin{Def}
    Ассоциативная алгеьра: $\cdot$ "--- ассоциативна и $(A, +, \cdot)$ "--- кольцо.\\ 
    $\forall a_1, a_2 \in A, \forall k \in K \colon (ka_1)a_2 = a_1(ka_2) = k(a_1 \cdot a_2)$\\
\end{Def}
\begin{exmp}   
    \begin{enumerate}
    \item
    $K[t]$ "--- алгебра многочленов.\\
    \item
    $M_n(K)$ "--- матричная алгебра.\\
    \item
    $\H$ "--- 4-мерная алгебра над $\R$\\
    \item
    $\C$ "--- 2-мерная алгебра над $\C$\\
    \begin{Rem}
    $\H$ "--- веторное пространство над $\C$, но не алгебра. 

    $i(jk) = (ij) k = -1$\\
    $j(ik) = j(-j) = 1 \ne i(jk)$
     то есть форма не билинейная. 
    \end{Rem}
    \item
    $\O$ "--- восьмимерная неассоциативная алгебра октав. 
    $1, e_1, \cdots, e_7$ "--- базис $V$

	\cimg{im1.png}{0.25}
        
    Идем по стрелке. Получается число, в которое мы пришли. Знак зависит от того, шли мы по стрелке или нет. 
    \item Алгебра Ли \\
    $[]\colon A \times A \to A $ "--- умножение\\
    $[X, X] = 0 \Ra [X, Y] = -[Y, X]$ "--- кососимметричная\\
    $$\forall X, Y, Z\in A \colon [X, [Y, Z]] + [Y, [Z, X]] + [Z, [X, Y]] = 0$$
    \end{enumerate}
\end{exmp}

$A$  "--- алгебра.\\
$\circ \colon A \times A \to A$(билинейное отображение)\\
$$\mathcal{L}(A, A, A) \cong \mathcal{L}(A \otimes A, A) \cong (A \otimes A)^* \otimes A \cong A^* \otimes A^* \otimes A$$

Билинейное умножение на алгебре $A$  соответствует тензорам $T_2^1(A)$.

$\{e_i\}$ "--- базис $A$
$$a = \sum_{i} a^i e_i$$
$$b = \sum_{i} b^i e_j$$

$a^i$ "--- коэффицент $a$ в базисе $e_i$

$$e_ie_j = \sum \gamma_{ij}^ke_k$$
$$ab = \sum_{k}(\sum_{ij}a^ib^j\gamma_{ij}^k)e_k$$

\begin{Def}
$\{\gamma_{ij}^k\}$ "--- структурные константы алгебры $A$.
\end{Def}

Если умножение ассоциативно:
    $$(e_ie_j)e_l = e_i(e_je_l)$$
    $$(e_ie_j)e_l = \sum_k\gamma_{ij}^ke_ke_l = \sum_{k}\sum_{s}\gamma_{ij}^k \gamma_{kl}^s e_s = \sum_s(\sum_k \gamma_{ij}^k \gamma_{kl}^s)e_s$$
    $$e_i(e_je_l) = e_i(\sum_r \gamma_{jl}^re_r) = \sum_{r}\gamma_{jl}^{r}e_ie_r = \sum_s(\sum_r \gamma_{jl}^r\gamma_{ir}^s)e_s$$

    $$\Ra \forall i, j, l, s \colon \sum_{k}\gamma_{ij}^k\gamma_{kl}^s = \sum_r\gamma_{jl}^r\gamma_{ir}^s$$

В случае коммутативного умножения:
    $$\forall i, j \colon e_ie_j = e_je_i$$
    $$\forall i, j, k \colon \gamma_{ij}^k = \gamma_{ji}^k$$


Вернемся  к общему случаю $T_p^q(V)$
$$T_0^0 \cong K$$
$$T(V) = \oplus_{p,q}T_{p}^{q}(V)$$

$$\otimes \colon T(V) \times T(V) \to T(V)$$
$$\otimes T_{p}^{q}(V) \times T_{p'}^{q'}(V) = T_{p + p'}^{q + q'}(V)$$

$$T_{p}^{q} = V^* \otimes \cdots \otimes V^* \otimes V \cdots \otimes V \cong \mathcal{L}(V^*, \cdots, V^*, V, \cdots, V, K)$$
$$\alpha\colon V^* \times \cdots \times V^* \times V \times \cdots \times V \to K$$

$$\beta \in T_{p'}^{q'}$$
$$\beta \colon V^* \times \cdots V^* \times V \times \cdots \times V \to K$$ 

Хотим построить полилинейное отображение 
$$(\alpha, \beta)\colon V^* \times \cdots V^* \times V \cdots \times V \to K$$ 
Первых элемнетов $p + p'$, вторых $q + q'$

На первых p действует как $\alpha$, на вторых p' как $\beta$, для q аналогично. 

$$(\alpha, \beta) \colon (v_1^*, \cdots, v_{p + q}^*, v_1, \cdots, v_{q + q'}) = \alpha(v_1^*, \cdots, v_p^*, v_1, \cdots, v_q)\beta(v_{p + 1}^*, \cdots, v_{p + p'}^*, v_{q + 1}, \cdots, v^*_{q + q'})$$

На самом деле можно все это получить пользуясь уже известными изоморфизмами:
    $$T_{p}^q(V) \otimes T_{p'}^{q'}(V) \cong T_{p + p'}^{q + q'}(V)$$
    Воспользуемся тем, что мы можем раскрывать скобки и переставлять элементы получим в точности $T_{p + p'}^{q + q'}$
   

\section{Самосопряженные операторы в евклидовом и унитарном пространстве}

\begin{theorem}{}
$\mathscr{A}$ в унитарном пространстве самосопряжен $\Lra$
$\exists$ ортонормированный базис, в котором $[\mathscr{A}]$ диагональная и вещественная. 
\end{theorem}

\begin{proof}

Унитарное пространство:
 $$\mathscr{A} = \mathscr{A}^*$$
$\exists$ ортонормированный базис, в котором 
$[\mathscr{A}] = \begin{pmatrix}
\lambda_1&0\\
0&\lambda_n\\
\end{pmatrix}$

$$[\mathscr{A}] = [\mathscr{A}^*] = \overline{[\mathscr{A}]^{T}} = [\mathscr{A}] = \begin{pmatrix}
\overline{\lambda_1}&0\\
0&\overline{\lambda_n}\\
\end{pmatrix} $$

И наоборот если 
$\exists$ ортонормированный базис в котором $[\mathscr{A}] = \begin{pmatrix}
\lambda_1&0\\
0&\lambda_n\\
\end{pmatrix}$ 

$[\mathscr{A}^T] = [\mathscr{A}] \Ra \mathscr{A} = \mathscr{A}^*$ 

\end{proof}

\begin{conseq}
Собственный числа самосопряженных операторов "--- вещественные. 
\end{conseq}

$\{e_{i}\}$ стандартный базис. 

$\{v_i\}$ "--- ортонормированный базис. 
В этом базисе $[\mathscr{A}] = \begin{pmatrix}
\lambda_1&0\\
0&\lambda_n\\
\end{pmatrix}$ 

$$\mathscr{A} = \mathscr{A}^*$$

$$[\mathscr{A}] = \overline{[\mathscr{A}]^T}$$

$$U^{T}\overline{U} = E$$

Унитарная матрица  = в точности матрицы, которые ортонормированный переводят в ортонормированные. 

Пусть U матрица перехода от $\{e_i\}$ к $\{v_i\}$.

$U$  "--- унитарная матрица. 

$U^{-1}[\mathscr{A}]U = \begin{pmatrix}
\lambda_1&0\\
0&\lambda_n\\
\end{pmatrix}, \lambda_i \in \R$ 

\begin{proof}
Для всякой эрмитовосимметрической матрицы  $[\mathscr{A}] \exists$ унитарная U
$$U^{-1}[\mathscr{A}]U = \begin{pmatrix}
\lambda_1&0\\
0&\lambda_n\\
\end{pmatrix}, \lambda_i \in \R$$
\end{proof}

\begin{theorem}{}
$\C^n < >$ "--- полуторалинейная эрмитовосимметричная форма на $\C^n$,
тогда $\exists$  базис $\C^n$ в котором 
$\Gamma = \begin{pmatrix}
1&0&\cdots&0&0\\
0&-1&\cdots&0&0\\
\cdot &\cdot & \cdot& \cdot& \cdot\\
0&0&\cdots &1&0\\
0&0&\cdots &0&1\\
\end{pmatrix}$
\end{theorem}

\begin{proof}
$<x, y> = (\mathscr{A}x, y)$ для некоторого самосопряженного $\mathscr{A}$.

$\Gamma'_{<>} = [\mathscr{A}]^{T}$

$\mathscr{A}$ "--- эрмитовосимметричная матрица. 
$[\mathscr{A}]^T$ "---эрмитовосимметричная. 
$\exists$ ортонормированный базис(относительно стандартного скалярного произведения). 

$[\mathscr{A}]^{T}$ "--- диагональ с вещественными элементами. 

$\Gamma' = \begin{pmatrix}
\lambda_1&0\\
0&\lambda_n\\
\end{pmatrix}$ 

Не умаляя общности могут сначала идти положительные, потом отрицательные, потом нулю. 

$\{v_i\}$ "--- ортонормированный базис. 

$w_i:$
$\frac{1}{\sqrt{\lambda}v_i}$

$\frac{1}{\sqrt{-\mu}v_i}$

$v_i$ 

$$<w_i, w_i> = \frac{1}{\lambda_i} <v_i, v_i> = 1$$
$$<w_i, w_i> = \frac{1}{\mu_i} <v_i, v_i> = -1$$
$$<w_i,w_i> = 0 $$

$$\Gamma'' = \begin{pmatrix}
1&0&\cdots&0&0\\
0&-1&\cdots&0&0\\
\cdot &\cdot & \cdot& \cdot& \cdot\\
0&0&\cdots &1&0\\
0&0&\cdots &0&1\\
\end{pmatrix}$$

\end{proof}

Евклидово пространство. $\R^n$
$$\mathscr{A} = \mathscr{A}^*$$
$[\mathscr{A}] = [\mathscr{A}]^{T}$  "--- симметрическая матрица. 

\begin{lemma}{}
Если $\mathscr{A} = [\mathscr{A}]$ "--- симметрическая матрица, то ее собственный числа вещественные. 
\end{lemma}

\begin{proof}
$\lambda$ "--- собственное число $\mathscr{A}$, $\lambda \in \C$

$$\mathscr{A}x = \lambda x, x \in \C^n$$
$$x \ne 0$$
$$\overline{x^T} \mathscr{A}x  = \overline{x^T} \cdot \lambda x = \lambda \overline{x^t}x = \lambda(|x_1|^2 + \cdots + |x_n|^2)$$

$$\overline{x^T} \mathscr{A}x = \overline{x^{T}} \overline{\mathscr{A}^{T}}x = \overline{(\mathscr{A}x)^T} \cdot x  = \overline{(\lambda x)^T} x = $$
$$= \overline{\lambda}\overline{x}^Tx = \overline{\lambda}(|x_1|^2 + \cdots + |x_n|^2)$$

$\lambda = \overline{\lambda} \Ra \lambda$ "--- вещественная. 
\end{proof}


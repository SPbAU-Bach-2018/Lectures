\section{Самосопряженные операторы в евклидовом и унитарном пространстве}

\begin{theorem}
	$\mathcal A$ в унитарном пространстве самосопряжен тогда и только тогда, когда
	существует ортонормированный базис, в котором $[\mathcal A]$ диагональная и вещественная.
\end{theorem}
\begin{proof}
	Унитарное пространство:
	\[ \mathcal A = \mathcal{A}^* \]
	Существует ортонормированный базис, в котором
	\begin{gather*}
		[\mathcal A]
		= \begin{pmatrix}
			\lambda_1 & \cdots & 0 \\
			\vdots & \ddots & \vdots \\
			0 & \cdots & \lambda_n
		\end{pmatrix} \\
		[\mathcal A]
		= [\mathcal{A}^*]
		= \overline{[\mathcal A]^T}
		= \overline{[\mathcal A]}
		= \begin{pmatrix}
			\bar\lambda_1 & \cdots & 0 \\
			\vdots & \ddots & \vdots \\
			0 & \cdots & \bar\lambda_n
		\end{pmatrix}
	\end{gather*}
	И наоборот, если найдётся ортонормированный базис, в котором
	\[
		[\mathcal A]
		= \begin{pmatrix}
			\lambda_1 & \cdots & 0 \\
			\vdots & \ddots & \vdots \\
			0 & \cdots & \lambda_n
		\end{pmatrix}
	\]
	то
	\[
		[\mathcal{A}^T] = [\mathcal A] \Ra \mathcal A = \mathcal{A}^*
	\]
\end{proof}

\begin{conseq}
	Собственный числа самосопряженных операторов "--- вещественные.
\end{conseq}
$\{e_i\}$ стандартный базис, $\{v_i\}$ "--- ортонормированный базис.
В этом базисе
$
	[\mathcal A] = \begin{pmatrix}
		\lambda_1 & \cdots & 0 \\
		\vdots & \ddots & \vdots \\
		0 & \cdots & \lambda_n
	\end{pmatrix}
$
\begin{gather*}
	\mathcal A = \mathcal{A}^* \\
	[\mathcal A] = \overline{[\mathcal A]^T} \\
	U^T\bar U = E
\end{gather*}
Унитарные матрицы "--- в точности матрицы, которые ортонормированный переводят в ортонормированные.
Пусть $U$ "--- матрица перехода от $\{e_i\}$ к $\{v_i\}$.
$U$  "--- унитарная матрица.
$
	U^{-1} [\mathcal A] U = \begin{pmatrix}
		\lambda_1 & \cdots & 0 \\
		\vdots & \ddots & \vdots \\
		0 & \cdots & \lambda_n
	\end{pmatrix}
$, $\lambda_i \in \R$.

\begin{proof}
	Для всякой эрмитовосимметрической матрицы  $[\mathcal A]$ существует унитарная $U$
	\[
		U^{-1}[\mathcal A]U = \begin{pmatrix}
			\lambda_1 & \cdots & 0 \\
			\vdots & \ddots & \vdots \\
			0 & \cdots & \lambda_n
		\end{pmatrix}
	\]
	$\lambda_i \in \R$.
\end{proof}

\begin{theorem}
	$\C^n$, $\left<,\right>$ "--- полуторалинейная эрмитовосимметричная форма на $\C^n$.
	Тогда существутет базис $\C^n$, в котором
	\[
		\Gamma = \begin{pmatrix}
			1 & 0 &\cdots & 0 & 0 \\
			0 &-1 &\cdots & 0 & 0 \\
			\vdots&\vdots &\ddots&\vdots&\vdots\\
			0 & 0 &\cdots & 1 & 0 \\
			0 & 0 &\cdots & 0 & 1
		\end{pmatrix}
	\]
\end{theorem}

\begin{proof}
	$\left<x, y\right> = (\mathcal Ax, y)$ для некоторого самосопряженного $\mathcal A$.
	$\Gamma'_{\left<\right>} = [\mathcal A]^T$.
	$\mathcal A$ "--- эрмитовосимметричная матрица.
	$[\mathcal A]^T$ "---эрмитовосимметричная.
	Существует ортонормированный базис (относительно стандартного скалярного произведения).

	$[\mathcal A]^T$ "--- диагональ с вещественными элементами.
	\[
		\Gamma' = \begin{pmatrix}
			\lambda_1 & \cdots & 0 \\
			\vdots & \ddots & \vdots \\
			0 & \cdots & \lambda_n
		\end{pmatrix}
	\]
	Не умаляя общности могут сначала идти положительные, потом отрицательные, потом нулю.

	$\{v_i\}$ "--- ортонормированный базис.
	$w_i:$
	$\frac{1}{\sqrt{\lambda}v_i}$\\
	$\frac{1}{\sqrt{-\mu}v_i}$\\
	$v_i$ \TODO
	\begin{gather*}
		\left<w_i, w_i\right> = \frac1{\lambda_i} \left<v_i, v_i\right> = 1 \\
		\left<w_i, w_i\right> = \frac1{\mu_i} \left<v_i, v_i\right> = -1 \\
		\left<w_i, w_i\right> = 0 \\
		\Gamma'' = \begin{pmatrix}
			1 & 0 &\cdots & 0 & 0 \\
			0 &-1 &\cdots & 0 & 0 \\
			\vdots&\vdots &\ddots&\vdots&\vdots\\
			0 & 0 &\cdots & 1 & 0 \\
			0 & 0 &\cdots & 0 & 1
		\end{pmatrix}
	\end{gather*}
\end{proof}

Евклидово пространство $\R^n$:
\[ \mathcal A = \mathcal{A}^* \]
$[\mathcal A] = [\mathcal A]^T$  "--- симметрическая матрица.

\begin{lemma}
	Если $\mathcal A = [\mathcal A]$ "--- симметрическая матрица, то ее собственный числа вещественные.
\end{lemma}

\begin{proof}
	$\lambda$ "--- собственное число $\mathcal A$, $\lambda \in \C$.
	\begin{gather*}
		\mathcal Ax = \lambda x, x \in \C^n, x \ne 0 \\
		\overline{x^T} \mathcal Ax
		= \overline{x^T} \cdot \lambda x
		= \lambda \overline{x^t}x
		= \lambda(|x_1|^2 + \cdots + |x_n|^2) \\
		\overline{x^T} \mathcal Ax
		= \overline{x^T} \overline{\mathcal{A}^T}x
		= \overline{(\mathcal Ax)^T} \cdot x
		= \overline{(\lambda x)^T} x = \\
		= \overline{\lambda}\bar x^Tx = \bar \lambda(|x_1|^2 + \cdots + |x_n|^2)
	\end{gather*}
	$\lambda = \overline{\lambda} \Ra \lambda \in \R$
\end{proof}

\section{Самосопряженные операторы в унитарном пространстве}

\begin{enumerate}
\item
	$\mathcal A = \mathcal{A}^*$.
	Существует ортонормированный базис из собственных векторов $\mathcal A$,
	в котором матрица $[\mathcal A]$ диагональная и на диагонали стоят собственные числа $\mathcal A$.
	Причем собственные числа вещественные.

\item
	$A \in M(n, \C)$, $A = \overline{A^T}$.
	Существует ортонормированный базис из собственных векторов $A$ и матрица перехода $U$ из стандартного базиса к этому базису.
	$U^{-1} A U$ есть диагональная матрица, на диагонали собственные числа $A$.

\item
	$A \in M(n, \C)$, $A = \overline{A^T}$.
	Существует унитарная матрица $U$, что $U^{-1}AU$ "--- диагональная, на диагонали собственные числа $A$.
\end{enumerate}

\begin{Rem}
   $U$ из пункта 2 "--- унитарная матрица (и она же из пункта 3).
   $U$ переводит ортонормированный базис в ортонормированный (то есть изометрия унитарного пространства).

   \[ U^T \bar U = E \]
   $\bar U^T U = E \Ra \bar U\text{ "--- унитарная}$.
   $U^{-1} = \bar U^T$ ($U^{-1} = U^*$).
   \[ U^{-1}AU = \bar U^T A U = U_1^T A \bar U_1 \]
   Где $U_1 = \bar U$ "--- унитарная.
\end{Rem}

Всякая эрмитовосимметричная, полуторалинейная форма имеет вид $\left<x, y\right> = (\mathcal Ax, y)$, где $\mathcal A = \mathcal{A}^*$.

\begin{conseq}
	Существует ортонормированный (относительно стандартного скалярного произведения) базис,
	в котором $\mathcal A$ "--- диагонален,
	и значит, в этом базисе $\Gamma$ формы $\left<,\right>$ диагональная с вещественными числами на диагонали.
\end{conseq}
\begin{conseq}
	Существует базис, в котором матрица Грама формы $\left<,\right>$ диагональная и на диагонали $+1, -1, 0$ (<<Закон инерции>>).
\end{conseq}

\subsection{Самосопряжённые операторы в Евклидовом пространстве}

\begin{enumerate}
\item
	$\mathcal A = \mathcal{A}^*$.
	Собственные числа вещественные и существует ортонормированный базис (в $\R^n$) из собственных векторов $\mathcal A$,
	в котором матрица $[\mathcal A]$ диагональная и на диагонали собственные числа.

\item
	$A \in M(n, \R)$, $A = A^T$.
	Существует ортонормированный базис из собственных векторов $A$ и матрица перехода $O$ от старого базиса к этому.
	$O^{-1}AO$ диагональная и на диагонали собственные числа $A$ (и все они вещественные).

	$O$ переводит стандартный базис в ортонормированный, а значит изометрия $\R^n$:
	\begin{align*}
		O^T E O &= E \\
		O^{-1}  &= O^T
	\end{align*}

\item
	$A \in M(n,\R)$, $A = A^T$.
	Существует $O$ "--- ортогональная, что $O^TAO = O^{-1}AO$ "--- диагональная и на диагональной собственные числа $A$.

\item
	$\left<,\right>$ "--- билинейная форма симметричная на $\R^n$.
	$<x, y> = (\mathcal Ax, y)$, $\mathcal A = \mathcal{A}^*$.

	Существует ортонормированный базис, в котором матрица Грама скалярного произведения диагональная.
	На диагонали сначала сколько-то положительных чисел, потом сколько-то отрицательных и нули.
\end{enumerate}

\begin{conseq}
	Существует базис, в котором матрица Грама имеет вид: на диагонали сколько-то единиц, сколько-то -1 и 0, остальные 0.
	$k$ "--- количество 1, $l$ "--- количество -1.
	(<<Закон инерции квадратичных форм>>).
\end{conseq}

\textbf{Комментарий к закону инерции:}
$k$, $l$ и $(n - k - l)$ определены формой $\left<,\right>$ однозначно.
$k + l$ "--- задают ранг матрицы Грама, не меняется при замени базиса.

\textbf{Упражнение:}
Подумайте, почему $k$ тоже задается однозначно.

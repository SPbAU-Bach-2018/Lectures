\section{Третья теорема о гомоморфизме}
\begin{theorem}
$\psi_1 \colon G \twoheadrightarrow G_1$ \\
$\psi_2 \colon G \twoheadrightarrow G_2$ \\
$\psi_1, \psi_2$ "--- гомоморфизмы\\
$\ker \psi_1 \subset \ker \psi_2$\\
$\exists \sigma \colon G_1 \twoheadrightarrow G_2$, такое, что $\sigma \circ \psi_1 = \psi_2$\\
\end{theorem}
\begin{proof}
$b \in G_1, \exists g \colon \psi_1(g) = b$\\
$\sigma (b) = \psi_2(g)$\\

\begin{enumerate}
\item Корректность.

Пусть $g, g_1 \colon \psi_1(g) = \psi(g_1) = b$\\
$\psi_1(g_1^{-1}g) = b^{-1}b = 1_G$\\
$g_1^{-1}g \in \ker \psi_1 \subset \ker \psi_2$\\
$\psi_2(g_1^{-1}g) = 1_G \Ra \psi_2(g) = \psi_2(g_1)$\\

$\psi_2(g) = \sigma(\psi_1(g)) = \psi_2(g)$\\
для всех $g \in G_1$\\
$\sigma \circ \psi_1 = \psi_2$\\
$\psi_2$ "--- сюръективный $\Ra \sigma$ "--- сюръективный\\

$\sigma$ "--- гомоморфизм\\
$b_1 = \psi_1(g_1)$\\
$b_2 = \psi_1(g_2)$\\
$b_1b_2 = \psi_1(g_1g_2)$\\

$\sigma(b_1) = \psi_2(g_1), \sigma(b_2) = \psi_2(g_2)$\\
$\sigma(b_1)\sigma(b_2) = \psi_2(g_1) \psi_2(g_2) = \psi_2(g_1g_2) = \sigma(b1b2)$\\
\end{enumerate}
\end{proof}
\begin{Rem}
$H\subset K \subset G$\\
$\{1_{G\setminus H}\subset L \subset G \setminus H$ \\
$L = K \setminus H$\\
$L = \{kH\}, k \in K$\\
$L \to k = \{g \in G \colon gH \in L\}$\\
\end{Rem}

\begin{theorem}
$G, H_1 \unlhd G, H_2 \unlhd G$\\
$H_1 \le H_2$\\
$G\setminus H_1, H_2 \setminus H_1$\\
$H_2 \setminus H_1$  "--- нормальная в $G \setminus H_1$\\
$(G \setminus H_1)\setminus(H_2 \setminus H_1) \cong G \setminus H_2$\\
\end{theorem}
\begin{proof}
$\phi_1 \colon G \to G \setminus H_1$\\
$\phi_2 \colon G \to G \setminus H_2$\\
$\ker \phi_1 = H_1$\\
$\ker \phi_2 = H_2$\\
$\ker \phi_1 \subset \ker \phi_2$\\
$\exists \sigma \colon G \setminus H_1 \to G \setminus H_2$(по третей теореме о гомоморфизме)
$\sigma \circ \phi_1 = \phi_2$\\
$\sigma(gH_1) = 1_{G \setminus H_2}$\\
$gH_2 = \sigma(gH_1) = H_2 \Ra g \in H_2$\\

$gH_1 \in \ker \sigma \Ra g \in H_2$\\
$gH_1 \in H_2 \setminus H_1$\\
$\ker \sigma \setminus H_2 \setminus H_1$\\

$g \in H_2, gH_1$\\
$\sigma(gH_1) = \phi_2(g) = gH_2 = H_2 = 1_{G\setminus H_2}$\\
$gH_1$, где $g \in H_2$, лежит в $\ker \sigma$\\
$H_2 \setminus H_1 \subset \ker \sigma$\\
$H_2 \setminus H_1 = \ker \sigma$\\

По первой теореме о гомоморфизме 
$G \setminus H_2 = \sigma (G \setminus H_1) \cong (G \setminus H_1)\setminus \ker \sigma = (G\setminus H_1)\setminus(H_2 \setminus H_1)$

\end{proof}
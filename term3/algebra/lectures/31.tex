\section{Теорема об орбитах и стабилизаторах}
$G$ действует на X (то есть $X$ "--- G-операторное множество)\\
\begin{Def}
$G$ "--- группа, X, X' "--- два G-операторных множества. 
$X$ изоморфно $X'$ как G-операторное множество, если 
$\exists$ биекция $\psi \colon X \to X'$\\
$\forall g \in G \forall x \in X \colon \psi(gx) = g \psi(x)$\\
$G$ действует на $X$\\
\end{Def}
\begin{theorem}
Орбита $Ga$ как $G$-операторное множество изоморфна
множеству левых классов смежности группы $G$ на подгруппе $St_a$
\end{theorem}
\begin{proof}
$St_a = \{g \in G \colon ga = a\}$\\
$Ga$ орбита точки из $X$\\
$X' = \{gSt_a | g \in G\}$\\
$\psi \colon Ga \to X'$\\
$ga \to g St_a$\\

Коррестность $g_1a  = g_2 a \Ra g_2^{-1}g_1 = a \Ra g_2^{-1}g_1 \in St_a$\\
$\Ra g_2^{-1}g_1 St_a = St_a \Ra g_1 St_a = g_2 St_a$\\

Инъективность:\\
$\psi (g_1 a) = \psi (g_2a)$\\
$g_1St_a = g_2St_a$\\
$g_2^{-1}g_1St_a = St_a$\\
$g_2^{-1}g_1 \in St_a$\\
$g_2^{-1}g_1a = a$\\
$g_1a = g_2a$\\

Сюръективность:\\
класс $gSt_a$ есть $\psi(ga)$\\
$h \in G$ $ga \in Ga$\\
$\psi(hga) = h \psi(ga)$\\
$\psi(hga) = \psi((hg)a) = hgSt_a$\\
$ = h(gSt_a) = h \psi(ga)$\\
\end{proof}
\begin{conseq}
\begin{enumerate}
\item (Orbit Stabilizer theorem)\\
$G$ "--- как группа действует на $X$\\
$|Ga| = \frac{|G|}{|St_a|}$\\
$|Ga||St_a| = |G|$\\
\begin{proof}
$Ga$ "--- множество левых смежных классов на $St_a$ изоморыно, как $G$-оператор
множества.
$|Ga|$ = число левых классов смежности по $St_a$ на $H$
$= [G \colon St_a] = \frac{|G|}{|St_a|}$ по теореме Лагранжа.
\end{proof}
\begin{Rem}
для бесконечных $G$\\
$|Ga| = [G \colon St_a]$\\
\end{Rem}
\item
$G$ "--- как группа $|St_a|$ делит $|G|$\\
                    $|Ga|$ делит $|G|$\\
\end{enumerate}
\end{conseq}
\begin{exmp}
$G$ "--- группа самосовмещений куба и $H$ "--- подгруппа собственных самосовмещений куба.

$|H| = 24$, $|G| = 48$\\

$x$ "--- вершина куба $G$ и $H$ действуют транзитивно на множестве вершин.\\
$|G| = |Gx||St_x|$\\
$|Gx| = 8$\\
$St_x$ "--- стабилизатор в $G$.\\
\\
$St_x$ действует на $(a, b, c)$\\
$|St_x^{G}| = |St_x^{G}\cdot a||St_{x, a}^{G}|$\\
$|St_x \cdot a| = 3$\\
$St_{x, a}^{G}$ действует транзитивно на $\{b, c\}$\\
$|St_{x, a}^{G}| = |St_{x, a}^{G}\cdot b||St_{x, a, b}| = 2$\\

$G = 8 \cdot 3 \cdot 2 = 48$\\
$H = |Hx||St_x^{H}| = |Hx||St_x^{H}\cdot a||St_{x,a}^{H}| = 24$\\
\end{exmp}

\section{Теорема об орбитах и стабилизаторах}
Напоминание, когда $G$ действует на X, мы говорили, что $X$ "--- G-операторное множество\\
\begin{Def}
$G$ "--- группа, X, X' "--- два G-операторных множества.\\
$X$ изоморфно $X'$ как G-операторное множество, если 
существует биекция $\psi \colon X \to X'$, такая что
$\forall g \in G, \forall x \in X \colon \psi(gx) = g \psi(x)$\\
\end{Def}

Группа $G$ действует на множество $X$, тогда множество $X$ распадается на орбиты и 
в прошлый раз мы выяснили, что что бы изучить действие группы на множестве достаточно изучить
действия групппы на орбиты. 

\begin{theorem}
Орбита $Ga$, как $G$-операторное множество изоморфна
множеству левых классов смежности группы $G$ на подгруппе $St_a$
\end{theorem}
\begin{proof}
$St_a = \{g \in G \mid ga = a\}$\\
$Ga$ орбита точки из $X$\\
$X' = \{gSt_a \mid g \in G\}$ "--- множество смежных классов по подгруппе стабилизатора.\\
$\psi \colon Ga \to X'$\\
$ga \to g St_a$\\
\begin{description}
\item[Корректность:]
Нужно проверить, что если $g_1a = g_2a$, то результат отображения будет один и тот же.\\
$g_1a  = g_2 a \Ra g_2^{-1}g_1a = a \Ra g_2^{-1}g_1 \in St_a$\\
$\Ra g_2^{-1}g_1 St_a = St_a \Ra g_1 St_a = g_2 St_a$\\

\item[Инъективность:]
$\psi (g_1 a) = \psi (g_2a)$\\
$g_1St_a = g_2St_a$\\
$g_2^{-1}g_1St_a = St_a$\\
$g_2^{-1}g_1 \in St_a$\\
$g_2^{-1}g_1a = a$\\
$g_1a = g_2a$\\
Мы взяли два элемента орбиты и предположили, что $\psi$ переводит их
в один и тот же класс и из этого сделали вывод, что это один и тот же элемент. 
\\
\item[Сюръективность:]
Класс $gSt_a$ есть $\psi(ga)$\\

\item[Гомоморфизм:] 
$h \in G$, $ga \in Ga$\\
Хотим проверить, что $\psi(hga) = h \psi(ga)$\\
$\psi(hga) = \psi((hg)a) = hgSt_a = h(gSt_a) = h \psi(ga)$\\
\end{description}
\end{proof}
\begin{conseq}\hfill
(Orbit Stabilizer theorem)\\
$G$ конечная группа и как группа действует на $X$\\
$|Ga| = \frac{|G|}{|St_a|}$\\
$|Ga||St_a| = |G|$\\
\end{conseq}
\begin{proof}
$Ga$ и множество левых смежных классов на $St_a$ изоморфно, как $G$-оператор
множества.\\
$|Ga|$ = число левых классов смежности на $St_a$ = по определению индекс подгруппы
$= [G \colon St_a] = \frac{|G|}{|St_a|}$ по теореме Лагранжа.
\end{proof}
\begin{Rem}
для бесконечных $G$\\
$|Ga| = [G \colon St_a]$\\
\end{Rem}
\begin{conseq}
$|St_a|$ делит $|G|$\\
$|Ga|$ делит $|G|$\\
\end{conseq}
\begin{exmp}
$G$ "--- группа самосовмещений куба и $H$ "--- подгруппа собственных самосовмещений куба(сохраняющие ориентацию).\\
Рассмотрим действие на множестве вершин.\\
$x$ "--- вершина куба.\\
$G$ и $H$ действуют транзитивно на множестве вершин.\\
То есть каждую наперед заданую вершину можем перевести в другую наперед заданую вершину.\\
По теореме о стабилизаторах $|G| = |Gx||St_x|$\\
Длина орбиты из-за транзитивности действия это 8 ($|Gx| = 8$)\\
$St_x$ "--- стабилизатор в $G$. Мы зафиксировали одну вершину и она неподвижна, тогда
$St_x$ действует на множестве смежных с ней вершинах $(a, b, c)$\\
$|St_x^{G}| = |St_x^{G}a||St_{x, a}^{G}|$\\
$|St_x a| = 3$ "--- размер орбиты вершины a в группе стабилизатора x\\
$St_{x, a}^{G}$(фиксируем одновременно точки x и a) действует транзитивно на $\{b, c\}$\\
$|St_{x, a}^{G}| = |St_{x, a}^{G}\cdot b||St_{x, a, b}| = 2$\\
$G = 8 \cdot 3 \cdot 2 = 48$\\
Что изменится если мы группу $G$ заменим на группу $H$?
$H = |Hx||St_x^{H}| = |Hx||St_x^{H}\cdot a||St_{x,a}^{H}| = 24$\\
$|St_{x,a}^{H}| = 1$ не можем поменять местами последнее ребро в силу жесткости конструкции. 
\end{exmp}

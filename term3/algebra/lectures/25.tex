\section{Первая теорема о гомоморфизме}
\begin{theorem}
$\psi \colon G \to G_1$, $H = \ker \psi$\\
Тогда $G \setminus H \cong \psi(G)$\\
\end{theorem}
\begin{proof}
$\psi(G)$"--- подгруппа в $G_1$ (группа)\\
$\psi(G) \le G_1$\\
$\phi(G) = G \setminus H$\\
Хотим построить изоморфизм $\sigma\colon \psi(G) \to G \setminus H$\\

$b \in \psi(G) \Ra \exists g \in G, \psi(g) = b$\\
$\sigma(b) = \phi(g) = gH$\\
\begin{enumerate}
\item корректность\\
$g_1, g_2 \in \psi^{-1}(\{b\})$\\
$\psi(g_1) = \psi(g_2) \Ra \psi(g_2^{-1}g_1) = 1_{G_1}$\\
$g_2^{-1}g_1 \in \ker\psi = H$\\
$\Ra g_1H = g_2H \Ra \sigma$ не зависит от выбора конкретного элемента в $\psi^{-1}(\{b\})$\\
\item $\sigma$ "--- гомоморфизм
$b_1, b_2 \in \psi(G)$ \\
$b_i = \psi(g_i), i = 1,2$\\
$b_1b_2 = \psi(g_1g_2)$\\
$\sigma(b_1b_2) = g_1g_2H$\\
$\sigma(b_1)\sigma(b_2) = g_1Hg_2H = g_1g_2HH = g_1g_2H$\\
$\Ra \sigma(b_1b_2) = \sigma(b_1)\sigma(b_2)$\\
$\sigma$ "--- сюръективно.\\
$gH = \sigma(\psi(g))$\\
$\sigma$"--- инъективно.\\
$b_1, b_2 \in \psi(G), \sigma(b_1) = \sigma(b_2)$\\
$b_1 = \psi(g_1), b_2 = \psi(g_2)$\\
$\sigma(b_1) = g_1H$\\
$\sigma(b_2) = g_2H$\\
$\Ra g_2^{-1}g_1 \in H = \ker \psi$\\
$\psi(g_2^{-1}g_1 = 1) \Ra \psi(g_2) = \psi(g_1) \Ra b_2 = b_1$\\
\end{enumerate}
\end{proof}
\begin{Rem}
$\sigma(\psi(g)) = Gh = \phi(g)$ \\
$\sigma \circ \psi = \phi$\\
\end{Rem}
\begin{exmp}\hfill
\begin{enumerate}
\item
$(\R, +)$\\
$(\Z, +)$\\
$\R \setminus \Z \cong S = \{z \in \C, |z| = 1\}$ \\
$\psi\colon \R \to S$\\
$x \to e^{2\pi i x} = \cos(2\pi x) + i \sin(2\pi x)$\\
$\psi(\R = S)$\\
$\ker \psi = \Z$\\
$S = \psi(\R) \cong \R \setminus \ker \psi = \R \setminus \Z$\\
\item
$S = \{z \in \C \colon |z| = 1\}$\\
$U_n = \{z \in S, z^n = 1\} = \{\cos \frac{2\pi k}{n} + i \sin\frac{2\pi k}{n}| k = 0, \cdots, n - 1\}$\\
$S \setminus U_n \cong S$\\
$\psi\colon S \to S$\\
$z \to z^{n}$\\
$\psi(S) = S$\\
$\ker \psi = U_n$\\
$S = \psi(S) \cong S \setminus \ker \psi = S \setminus U_n$\\
\item
$SL_2(\Z), SL_n(\Z)$\\
$n \ge 5$
Какая из двух групп больше? 
С одной стороны можем всегда $SL_2$ вложить в $SL_n$.
$\phi \colon SL_2(\Z) \to SL_n(\Z)$\\
При $n \ge 5 \exists H \unlhd SL_2(\Z), SL_2(\Z) \setminus \H \cong SL_n(\Z)$
\end{enumerate}
\end{exmp}
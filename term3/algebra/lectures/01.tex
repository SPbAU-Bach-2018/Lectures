\section{Ортогональное дополнение}
\begin{Def}
    $$S \subset V$$
    $$S^{\bot} = \{v \in V \colon \forall s \in S \colon (v, s) = 0\}$$
    $S^{\bot}$ "--- ортогональное дополнение к S.
\end{Def}

\begin{conseq}
    \begin{enumerate}
    \item $$S^{\bot} \le V$$
    \item $$S_1 \subset S_2 \Ra S_1^{\bot} \ge S_2^{\bot}$$
    \item $S^{\bot} = <S>^{\bot}$ ортогональное дополнение к S совпадает 
    с ортогональным дополнением к подпространству порожденным S.
    \item $\{0\}^{\bot} = V$
    \item Если $(,)$ невырожденная, то $V^{\bot} = \{0\}$
    \item Если $(,)$ симметричная, кососимметричная, эрмитовосимметричная, то 
    $$S \subset (S^{\bot})^{\bot} (\Ra <S> \le (S^{\bot})^{\bot})$$
    \item $U_1, U_2 \le V$
    $$(U_1 + U_2)^{\bot} = U_1^{\bot} \cap U_2^{\bot}$$
    \item $$(U_1 \cap U_2)^{\bot} \ge U_1^{\bot} + U_2^{\bot}$$
    
    \begin{Rem}
    Далее $K = \R$ либо $\C$ и $(,)$ "--- положительно определенная, симметричная(или эрмитово симметричная)
    \end{Rem}
    \item Если $\dim V = n < \infty$, и форма положительно определенная 
    $$U \le V, \dim U = k \Ra \dim U^{\bot} = n - k$$
    В частности $V = U \oplus U^{\bot}$
    \item $<S> = (S^{\bot})^{\bot}$
    $\dim V < \infty$

    \item $\dim V < \infty$
    $$(U_1 \cap U_2)^{\bot} = U_1^{\bot} + U_2^{\bot}$$
    \end{enumerate}
\end{conseq}
\begin{proof}
    \begin{enumerate}
    \item $$0 \in S^{\bot}$$
    $$v_1, v_2 \in S^{\bot}$$
    $$s \in S$$
    $$(\alpha_1 v_1 + \alpha_2 v_2, s) = \alpha_1(v_1, s) + \alpha_2(v_2, s) = 0$$
    В силу произвольности $s, \alpha_1 v_1 + \alpha_2 v_2 \in S^{\bot}$.

    \item $$v \in S_2^{\bot}$$ 
    $$\forall s \in S_2, (v, s) = 0$$
    $$\Ra \forall s \in S_1, (v, s) = 0$$
    $$\Ra v \in v \in S_1^{\bot}$$
    \item 
    $$S \subset <S> \Ra S^{\bot} \ge <S>^{\bot} $$
    $$v \in S^{\bot}, \forall s \in S, (v, s) = 0$$
    $\sum_i \alpha_i s_i \in <S>$ ($s_i \in S$, почти все $\alpha_i = 0$)
    $$(v, \sum_i \alpha_i s_i) = \sum_i \overline{\alpha_i}(v, s_i) = 0$$
    $$\Ra v \in <s>^{\bot}$$
    Проверили обратное включение $\Ra S^{\bot} = <S>^{\bot}$
    \item очевидно.
    \item $$V^{\bot} = \{u \colon \forall v \in V \colon (u, v) = 0\}$$ 
    Если $u \ne 0$, то $\exists v \in V \colon (u, v) \ne 0$(в силу не вырожденности)
    $\Ra V^{\bot} = \{0\}$
    \item Предположение о форме влечет, что отношение ортогональности между векторами симметрично
    $$(u, v) = 0 \Lra (v, u) = 0$$
    $$S^{\bot} = \{v\colon \forall s \in S \colon (v, S) = 0\}$$
    $$s \in S, v \in S^{\bot}, (s, v) = (v, s) = 0 \Ra s \in (S^{\bot})^{\bot}$$
    \begin{exmp}
    $$V = l^2 = \{(a_1, a_2, \cdots)\colon a_i \in \R, \sum_{i = 1}^{\infty} a_i^2 < \infty\}$$
    $$A = (a_1, \cdots)$$
    $$B = (b_1, \cdots)$$
    $(A, B) = \sum_{i = 1}^{\infty}a_i b_i$ "--- почему ряд сходится?
    $$\sum_{i = 1}^{\infty} |a_i b_i|\le \sum_{i = 1}^{\infty} \frac{a_i^2 + b_i^2}{2} = \frac{1}{2}\sum_{i = 1}^{\infty}a_i^2 + \frac{1}{2}\sum_{i = 1}^{\infty}b_i^2  < \infty$$
    $$|a_i b_i| \le \frac{a_i^2 + b_i^2}{2} \Lra (|a_i| - |b_i|)^2 \ge 0$$
    $$(A_1 + A_2, B) = (A_1, B) + (A_2, B)$$
    $$A_1 = (a_1', a_2', \cdots)$$
    $$A_2 = (a_1'', a_2'', \cdots)$$
    $$\sum(a_i' + a_i'')b_i = \sum a_i'b_i + \sum a_i''b_i$$
    $$(1, \frac{1}{2}, \frac{1}{3}, \cdots) \in l^2$$
    $$\sum_{n = 1}^{\infty}\frac{1}{n^2} = \frac{\pi^2}{6}$$
    $$(1, \frac{1}{\sqrt{2}}, \frac{1}{\sqrt{3}}, \cdots) \notin l^2$$
    $l^2_0 = \{(a_1, \cdots):$ почти все $a_i = 0\}$
    $$l_0^2 \le l^2$$
    $$A \in (l_0^2)^{\bot}$$
    $$\forall i \colon A \bot (0, 0, \cdots, 1, 0, \cdots) = e_i$$
    $$0 = (A, e_i) = a_i$$
    $$(l_0^2)^{\bot} = \{0\}, ((l_0^2)^{\bot})^{\bot} = \{0\}^{\bot} = l^2 \ne l_0^2$$
    \end{exmp}
    \item 
    $$U_1 \le U_1 + U_2, U_2 \le U_1 + U_2 \Ra $$
    $$\Ra U_1^{\bot} \ge (U_1 + U_2)^{\bot}, U_2^{\bot} \ge (U_1 + U_2)^{\bot} \Ra$$
    $$\Ra U_1^{\bot} \cap U_2^{\bot} \ge (U_1 + U_2)^{\bot}$$
    $$\overline{v} \in  U_1^{\bot} \cap U_2^{\bot}$$
    $$v \in (U_1 \cup U_2)^{\bot} = <U_1 \cup U_2>^{\bot} = (U_1 + U_2)^{\bot}$$
    \item $$(U_1 \cap U_2)^{\bot} \supset U_1^{\bot} + U_2^{\bot}$$
    $$w = w_1 + w_2, (w_1 \in U_1^{\bot}, w_2 \in U_2^{\bot})$$
    $$u \in U_1 \cap U_2$$
    $$(w, u) = (w_1 + w_2, u) = (w_1, u) + (w_2, u) = 0$$
    В силу произвольности $u$ имеем $w \in (U_1 \cap U_2)^{\bot}$
    \item $$U = <S>, S^{\bot} = U^{\bot}$$
    $$U \subset V, U \cap U^{\bot} = \{0\}$$ 
    $$v \in U \cap U^{\bot} \Ra (v, v) = 0$$
    Так как форма положительно определенная, то $v = 0$.
    \begin{exmp}
    $$((x_1, x_2), (y_1, y_2)) = x_1 y_1 - x_2 y_1$$
    $$v = (1, 1) \in \R^{2}, (v, v) = 0$$
    $$v \in <v>^{\bot}, s = <(1, 1)>$$
    \end{exmp}
    \begin{exmp}
    $$\R^4$$
    $$(x, y) = x_1 y_1 - x_2 y_2 - x_3 y_3 - x_4 y_4$$
    $(1, 1, 0, 0)$  "--- ортогонален сам себе.
    Если $(u, v) = 0$, то $v$ "--- изотропный вектор. 
    \end{exmp}
    \begin{exmp}
    $$\mathbb{F}_2, V = \mathbb{F}_2^{n}$$
    $$(x, y) = \sum_{i = 1}^{n} x_i y_i$$
    $(x, x) = 0 \Lra$ среди его координат четное число 1.
    \end{exmp}
    \item $$U \le V$$
    $u_1, \cdots, u_k$ базис U дополним до базиса V.
    Ортогонализуем
    $e_1, \cdots, e_k, e_{k + 1}, \cdots, e_n$ "--- ортонормированный базис V.
    $e_1, \cdots, e_k$ "--- ортонормированный базис u по построению в процессе ортогонализации Грамма-Шмита.
    $$U^{\bot} \ge <e_{k + 1}, \cdots, e_{n}>$$
    $$v \in U^{\bot}, v = \sum \alpha_i e_i$$
    $$j = 1 \cdots k, (v, e_j) = 0 \Ra \alpha_1 = \cdots = \alpha_k = 0$$
    $$\Ra U^{\bot} \le <e_{k + 1}, \cdots, e_{n}> $$  
    $$U + U^{\bot} \le V$$
    $$\dim (U + U^{\bot}) = k + n - k - \dim (U \cap U^{\bot}) = n$$
    $$\Ra U + U^{\bot} = V \Ra U \oplus U^{\bot} = V$$
    \item 
    $$\dim <S> = k$$
    $$\dim S^{\bot} = \dim <S>^{\bot} = n - k$$
    $$\dim (S^{\bot})^{\bot} = n - (n - k) = k$$
    $$<S> \le (S^{\bot})^{\bot}$$
    Размерности совпадают и одно подпространство другого$\Ra$ пространства совпадают.
    \item $(U_1 + U_2)^{\bot} = U_1^{\bot} \cap U_2^{\bot}$  "--- верно для любого подпространства. 

    В частности 
    $$(U_1^{\bot} + U_2^{\bot})^{\bot} = (U_1^{\bot})^{\bot} = (U_1^{\bot})^{\bot} \cap (U_2^{\bot})^{\bot} = U_1 \cap U_2$$
    $$(U_1 \cap U_2)^{\bot} = ((U_1^{\bot} + U_2^{\bot})^{\bot})^{\bot} = U_1^{\bot} + U_2^{\bot}$$
    $$ $$
    \end{enumerate}
\end{proof}

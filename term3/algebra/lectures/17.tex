\section{Размерность тензорного произведения и базисы.}

\begin{lemma}
Если среди $V_i,  i = 1 \cdots n$ есть нулевое пространство, 
то $V_1 \otimes \cdots \otimes V_n$  также нульмерно. 
\end{lemma}
\begin{proof}
    $$V_i = \{0\}$$
    $$V_1 \times \cdots \times V_i \cdots \times V_n \to K$$
    $$\alpha \in \mathcal{L}(V_1, \cdots, V_i, \cdots, V_n, K)$$
    Зафиксируем все элементы кроме i-ого: $\alpha(V_1, \cdots, ., \cdots, V_n) \colon V_i \to K$
    $V_i = \{0\}$ линейные отображения 0-мерного пространства только 0-мерное отображение. 
    $$\Ra \alpha(V_1, \cdots, 0, \cdots, V_n) = 0 \Ra \alpha = 0$$
    То есть, если $V_i = \{0\}$, то 
    $$\mathcal{L}(V_1, \cdots, V_n, K) = \{0\}$$
    $= \mathcal{L}(V_1 \otimes \cdots \otimes V_n, K) = (V_1 \otimes \cdots \otimes V_n)^*$ "--- двойствнное пространство. 
    $$\Ra V_1 \otimes \cdots \otimes V_n = \{0\}$$

    Если бы $V_1 \otimes \cdots \otimes V_n$ содеражло бы ненулевой элемент, скажем e, 
    то тогда бы мы могли дополнить e до базиса, тензорного произведения  и определим функционал $e^*$
    $e^*(e) = 1$, e(остальных базисных) = 0.

    Противоресие с тем, что $(V_1 \otimes \cdots \otimes V_n)^* = \{0\}$.
\end{proof}
\begin{theorem}
    $V_1, \cdots, V_n$ "--- векторное пространство над $K$, $\dim V_i = d_i < \infty$, 
    тогда $\dim(V_1 \otimes \cdots \otimes V_n) = d_1 \cdot \cdots \cdot d_n$
\end{theorem}
\begin{proof}
    \begin{enumerate}
    \item $\exists i, \dim V_i = 0$ очевидно по лемме.
    \item Все $d_i \ne 0$
    $$\mathcal{L}(V_1, \cdots, V_n, K)$$
    $e_1^{j}, \cdots, e_{d_j}^j$  "--- базис $V_j$
    $(e_i^j)^*$ "--- элементы двойственного базиса (базиса в $V_j^*$)
    $$(e_i^j)*(e_{k}^j) =  1, i = k; 0, i \ne k$$
    Рассмотрим все такие произведения:
    $$(e^{(1)}_{i_1})^* \cdots (e_{i_n}^{(n)})^* \colon V_1 \times \cdots \times V_n \to K$$
    $$(e_{i_1}^{(1)})^* \cdots (e^{(n)}_{i_n})^*((v_1, \cdots, v_n)) = (e_{i_1}^{(1)})^*(v_1) \cdots (e_{i_n}^{(n)})^*(v_n)$$
    $$(e_{i_1}^{(1)})^* \cdots (e_{i_n}^{(n)})^* \in \mathcal{L}(V_1, \cdots, V_n, K)$$

    Таких произведений $d_1, \cdots, d_n$ 
    $$\{ (e_{i_1}^{(1)})^* \cdots (e_{i_n}^{(n)})^* \colon 1 \le i_1 \le d_1, \cdots, 1 \le i_n \le d_n$$ "--- базис
    пространства полилинейного отображени из $V_1 \times \cdots \times V_n$ в K.

    $f \colon V_1 \times \cdots \times V_n \to K$, полилинейная. 
    $$f(v_1, \cdots, v_n) = \sum_{i_1, \cdots, i_n}f(e_{i_1}, \cdots, e_{i_n})(e_{i_1}^{(1)} \cdots (e_{i_n}^{(n)})^{*})$$
    $$f(v_1, \cdots, v_n) = f(\sum \alpha_i^{(1)}e_i^{(1)}, \cdots, \sum\alpha_i^{(n)}e_i^{(n)}) = \sum_{i_1, \cdots, i_n}\alpha_{i_1}^{1} \cdots \alpha_{i_n}^{(n)}f(e_1, \cdots, e_n) = $$
    $$= \sum_{i_1, \cdots, i_n} \alpha_i^{(1)} \cdots \alpha_{i_n}^{(n)} \cdot f(e_{i}^{1}, \cdots, e_{i_n}^{(n)})(e_{i_1}^{(1)})^*(e_{i_1}^{(1)}) \cdots (e_{i_n}^{(n)})^*(e_{i_n}^{(n)})$$
    $f$ однозначно определяется совими значениями на $f(e_{i_1}^{(1)}, \cdots, e_{i_n}^{(n)})$
    f и правая часть формулы (1) принимают одинаковые значения на всех наборах $\{(e_{j_1}^{(1)}, \cdots, e_{j_n}^{(n)})\}$
    В силу полилинейности f и правая часть(1) совпадают всюду.
   
    Теперь нужно доказать, что наш предполагаемый базис линейно независимый. 

    $$0 = \sum_{i_1, \cdots, i_n}a_{i_1, \cdots, i_n}(e_{i_1}^{(1)})^* \cdots (e_{i_n}^{(1)})^*$$
    Переберем в качестве аргументов все наборы $(e_{j_1}^{(1)}, \cdots, e_{j_n}^(1))$
    Тогда спарава всегда 0, а слева останется одина аргумент
    $0 = a_{j_1, \cdots, j_n} \Ra$  линейно независимые. 

    Доказали: $\dim \mathcal{L}(V_1, \cdots, V_n, K) = d_1 \cdots, d_n$
    $$\dim \mathcal{L}(V_1 \otimes \cdots \otimes V_n, K) = d_1 \cdots d_n$$
    $$\dim (V_1, \otimes \cdots \otimes V_n)^* = d_1 \cdots d_n$$
    $$\dim (V_1, \otimes \cdots \otimes V_n) = d_1 \cdots d_n$$
    $$(V_1 \otimes \cdots \otimes V_n) \subset (V_1, \otimes \cdots V_n)^{**} = $$
    $$= ((V_1 \otimes \cdots \otimes V_n)^*)^*$$
    Все конечномерное 
    $$(V_1, \otimes \cdots \otimes V_n) = (V_1, \otimes \cdots \otimes V_n)^*$$
    и размерности совпадают. 
    \end{enumerate}
\end{proof}

\begin{theorem}
$$V_1, \cdots V_n$$ в.п над K, $\dim V_i = d_i < \infty$
$\{e_{i}^{(j)}\}_{i = 1}^{d_j}$ "--- базисы $V_1, \cdots, V_n$.
Тогда $\{e_{i_1}^{(1)} \otimes \cdots \otimes e_{i_n}^{(n)}\}$ "--- базиc $V_1 \otimes \cdots \otimes V_n$.
\end{theorem}
\begin{proof}
    Семейство образующих:
    $V_1 \otimes \cdots \otimes V_n$ пораждаются разложимами тензорами $V_1 \otimes \cdots \otimes V_n$
    $$= (\sum \alpha_{i_1} e_1^{(1)}) \otimes \cdots \otimes (\sum_{i_k}\alpha_ie_i^{(n)}) =$$
    $$= \sum_{i_1 \cdots i_n}\alpha_{i_1} \cdots \alpha_{i_n} e_{i_1}^{(1)} \otimes\cdots \otimes e_{i_n}^{(n)}$$
    $$\cdots \otimes (u + u') \otimes \cdots = \cdots \otimes u \otimes \cdots + \cdots \otimes u' \otimes \cdots$$
    $$\cdots \otimes (au) \otimes \cdots = a(\cdots \otimes u \otimes \cdots) $$

    Произвеление $e_{i_1}^{(1)} \otimes \cdots \otimes e_{i_n}^{(n)}$ в точности $d_1 \cdots d_n$ то есть 
    в точности размерность пространства $\Ra$ базис. 
\end{proof}
\begin{Rem}
Теорема остается верной и для бесконечномерных пространств. 
\end{Rem}
\begin{proof}
    Доказательство того, что это семейство образующих дословно повторяет рассуждение в конечномерном случае. 

    Линейная независимость в общем случае. 
    $$\sum_{i_1,  \cdots i_n}\alpha_{i_1,\cdots, i_n}e_{i_1}^{(1)} \otimes \cdots \otimes e_{i_n}^{(n)} = 0$$
    Лишь конечное число ненулевых коэффицентов. 
    $\Ra$ из каждого пространства присутствует лишь конечное число базисных векторв. 

    Рассмотрим соответсвующие (конечномерные) пространства $U_i \le V_i$ порожденные этими базисными векторами. 

    Тогда лежить не только в $V_1 \otimes \cdots \otimes V_n$, а в $U_1 \otimes \cdots \otimes U_n$(конечномерные)
    $\Ra$ по теореме эти $e_{i_1}^{(1)} \otimes \cdots e_{i_n}^{(n)}$ "--- линейно независимые в $U_1 \otimes \cdots \otimes U_n \Ra$
    $\alpha_{i_1 \cdots i_n} = 0$ для всех $i_1 \cdots i_n$. 
\end{proof}

\section{Действия групп на множествах}
$G$ "--- группа, $X$ "--- множество.
\begin{Def}
$G$ действует на $X$ если задано отображение $\cdot \colon G \times X \to X$\\
со следующими свойствами:
  \begin{enumerate}
  \item аналог ассоциотивности $\forall g_1, g_2 \in G \forall x \in X \colon g_1(g_2x) = (g_1g_2)x$\\
  Умножения разные в $g_1g_2$. ToDo
  \item $\forall x \in X \colon 1_G x = x$
  \end{enumerate}
\end{Def} 
\begin{exmp}
\begin{enumerate}
\item $G = GL_n(K), X = K^{n}$ 
действие групп матриц на вектор.
\item 
$S_n$ "--- перестановка действует на $\{1, 2, \cdots, n\}$
\end{enumerate}
\end{exmp}
\begin{Def}
$G$ действует на $X$, то $X$ называют $G$-оператором множества(левым G-оператором множества)\\
\end{Def}
$G \times X \to X$\\
Фиксируем g
$\phi_{g}\colon X \to X$\\
$\phi_{g}(x) = gx$\\

$Sym(X)$ "--- множество всех биекций на $X$(группа относительно композиций)\\
\begin{lemma}
\begin{enumerate}
\item
В этих обозначениях $\phi_g \in Sym(X)$.
\item
$\phi_g \circ \phi_h = \phi_{gh}$\\
\end{enumerate}
\end{lemma}
\begin{proof}
\begin{enumerate}
     \item
     Что бы показать, что это биекция, нужно найти обратное отображение. 
     $\phi_g$\\
     По второму пункту
     $\phi_g \circ \phi_{g^{-1}} = \phi_{gg^{-1}} = \phi_{1_G} = id_X$\\
     $1_G(x) = x$ по второму свойству определения действия\\
     $\phi_{g^{-1}}\phi_{g} = id_{x}$ \\
     $\phi_{g^{-1}} = (\phi_g)^{-1}$\\
     $\phi_g$ "--- обратимо, значит, биекция\\
     \item
     Возьмем $x \in X$ хотим проверить что оба отображения на него действуют одинаково\\
     $(\phi_g \circ \phi_h)(x) = \phi_g(\phi_h(x)) = g(h(x)) = (gh)(x) = \phi_{gh}(x)$\\
\end{enumerate}
\end{proof}

Подведем итоги
$\Phi \colon G \to Sym(X)$\\
$g \to \phi_g$\\

$\Phi(g_1g_2) = \phi_{g1g2} = \phi_{g_1}\phi_{g_2} = \Phi(g_1) \circ \Phi(g_2)$\\
$\Phi$ "--- гомоморфизм из $G$ в $Sym(X)$\\

И наоборот, всякий гомоморфизм $\Phi \colon G \to Sym(X)$ задает действие группы $G$ на множество $X$\\
$G \times X \to X$\\
$gx = \Phi(g)(x)$\\


Докажем, что то, что мы определили действие:
первое свойство:\\
$(gh)(x) = \Phi(gh)(x) = (\Phi(g)\Phi(h))(x) = \Phi(g)(\Phi(h)(x)) = ghx$\\
так как $\Phi$ "--- гомоморфизм\\ 
Второе свойство:\\
$\Phi(1_G) = 1_{Sym(X)} = id_X$\\
$1_{G}x = id_X(x) = x$

То есть мы можем смотреть на действие как на перестановку на множестве\\

\begin{Def}
Действие $G$ на множестве $x$ называется точным, если 
$\forall g \in G ((\forall x \in X\colon gx = x) \Ra g = 1_{G})$\\
точность $\equiv \ker \Phi = \{1_G\}$\\
в частности $\Phi(G) \cong G$ \\
\end{Def}

\begin{Def}
Действие $G$ на множестве $X$ называется транзитивным, если 
$\forall x, x' \in X \exists g \in G \colon gx = x'$
\end{Def}
\begin{Def}
Пусть $a \in X$, тогда орбитой $a$ называется
множество $Ga = \{ga| g \in G\}$ \\

НА орбиты можно смотреть с двух сторон
На $X$ введем $\sim$\\
    $x \sim x'$ если $\exists g \in G \colon gx = x'$\\

$\sim$  "--- отношение эквивалентности\\
\end{Def}
\begin{proof}
$x \sim x, 1x = x$\\
$x \sim x' \Ra x' \sim x$\\
$\exists g \colon gx = x'$\\
$g^{-1}x' = g^{-1}(gx) = (g^{-1}g)x = 1x = x$\\

$x \sim x', x \sim x''$\\
$\exists g$ и $g'$
$x' = gx$\\
$x'' = g'x$\\
$\Ra x'' = g'gx \Ra x'' \sim x$\\
\end{proof}
$\sim$ отношение эквивалентности\\
Рассмотрим множества классов эквивалентности $X / \sim = X / G$\\
$a \in X$\\
$[a] = \{x \colon x \sim a\} = \{x \colon \exists g \colon x = ga\} = Ga$ \\

Таким образом второе определение орбиты это классы эквивалентности $X/G$.
Будем обозначать множество всех орбит как $X/G$.

Можем предстваить $X$ как
$\cup_{A \in X/G} A$ "--- объединение орбит\\
$A = Ga$ "--- орбита\\

Рассмотрим сужение действия на орбиту
$G \times A \to A$\\

Такое сужение задает действие $G$ на $A$. Почему это будет действие? 
\begin{enumerate}
\item первая при сужение сохраняется
\item ...
\end{enumerate}

Или что то же самое, орбита сама является $G$-операторным множеством.

При сужение на орбиту действие всегда становится транзитивным, поэтому ими удобно пользоваться.

\begin{Def}
$G$ действует на $X$, $a \in X$\\
Стабилизатор $a$ $St_{a} = \{g \in G | ga = a\}$  \\

\end{Def}
\begin{lemma}
$St_a \le G$
\end{lemma}
\begin{proof}
Докажем, что $St_a$ является подгруппой 
\begin{enumerate}
\item 
$1_G \in St_a$ то есть множество не пусто.
\item 
Коммутирует
$g_1a = a, g_2a = a$\\
$(g_1g_2)a = g_1(g_2(a)) = g_1a = a \Ra g_1g_2 \in St_a$ \\
\item 
обратоные элемент
$ga = a$\\
$g^{-1}a = g^{-1}(ga) = (g^{-1}g)a = 1a = a \Ra g^{-1} \in St_a$\\
\end{enumerate}
\end{proof}

\begin{exmp}
\begin{enumerate}
\item $GL_n(K)$ на $K^n$
точное только единичная матрица сохраняет все множетсво, не транзитивное
$A_1 = \{0\}$\\
$A_2 = K^n\setminus \{0\}$

Докажем, что $A_2$ орбита.
$0 \ne x \in K^{n}, 0 \ne x' \in K^n$\\
Дополним $x$ до базиса $K^n$\\
Дополним $x'$ до другого базиса $K^n$\\
$\exists g\in GL_n(K)$ которая первый базис переводит во второй
В частности $gx = x'$\\
\item 
$G = GL_n(K)$ \\
$X = M_n(K)$ \\
$G \times X \to X$ \\
$g \cdot m \to gmg^{-1}$ "--- действие\\
Докажем, что выполнены аксиомы действия\\
$(gh)m = ghm(gh)^{-1} = g(hmh^{-1})g^{-1} = g(hm)$\\
$1m = m$\\

Рассмотрим, является ли это действие точным 
Для каких $g$ $gmg^{-1} = m$ для всех m, то есть $\forall m gm = mg$\\
то есть $g$ коммутирует с любой матрицей, значит $g$ имеет вид 
$g = \lambda E, \lambda \in K^{*}$\\
И всякая матрица такого вида действует сопряжением тождественно

Итог: это действие $GL_n(K)$  "--- точное $\Lra K^* = \{1\} \Lra K = F_2$ \\
\item
$G$ "--- группа \\
$H \le G$ "--- подгруппа\\
Рассмотрим множество левых классов $X = \{gH\}$\\
$G \times X \to X$\\
$f gH \to (fg)H$\\
Упражнение: проверить две аксиомы действия и корректность. 

Проверяем, что действие будет транзитивным, то есть что 
$(g'g^{-1})gH = g'H$ действие транзитивно.

ToDo
\item
$G, H = \{1_G\}$\\
$X = \{g\}$\\
Каждый элемент задает свой класс смежности. 
То есть действие задается как 
$G \times G \to G$\\
Проверим транзитивность
$\forall h gh = h$(возьмем $h = 1_G$, получаем, что $g = 1_G$)\\
$\Ra$ действие точное

\begin{conseq}
Всякая конечная группа изоморфна некотрой подгруппе группы перестановок.

$|G| \le \infty$ \\
$\Phi \colon G \to Sym(G)\cong S_{|G|}$\\
$S_{|G|}$ "--- перестановка.
Так как действие точно, $\Phi(G) \cong G$\\
$\Phi(G) \le Sym(G) \cong S_{|G|}$\\

Исследование подгрупп групп можно свести к иследованию подгрупп перестановок.
\begin{Rem}
  В одну орбиту входят матрицы с одинаковой жардановой формой
\end{Rem}
\end{conseq}
\item
$G$ "--- группа, $X = G$

Действие сопряжением:
$G \times G \to G$\\
$g \cdot h = ghg^{-1}$\\
Когда такое действие точное:
$g \colon g \forall h \colon ghg^{-1} = h$\\
$\forall h \colon gh = hg$ \\
$\{g \colon \forall h gh = hg\} $ "--- центр группы $G$ обозначается $Z(G)$.

Действие точное $\Lra Z(G) = \{1_{G}\}$.

Будет ли это действие транзитивным?

У нас есть единичный элемент, значит если в центр группы входит что-то еще (то есть $|G| > 1$ то есть больше одной орбиты), то 
действие не транзитивно, так  как есть орбита из $\{1_G\}$\\

Орбита в этом действие называется классами сопряженности.

Например, в $S_n$ кассы сопряженности состоят из перестановое с одинм цикловым типом.
\item
Рассмотрим единичный куб и рассмотрим две группы 
$G$ "--- группа изометрий $\R^3$ переводящая куб в себя.(группа самосовмещения куба).
$H \le G$ подгруппа собственных(то есть сохраняющих ориентацию) самосовмещений куба.

$G$ и $H$ действует на вершинах куба.
$G \to S_8$.

На гранях $G \to S_6$\\
на ребра $G \to S_{12}$\\
На диагоналях $G \to S_4$\\

Аналогично для $H$.


Упражнения. Какие из указанных действий для $G$ (для H) транзитивные, 
какие точные.
\end{enumerate}
\end{exmp}
                       

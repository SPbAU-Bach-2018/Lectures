\section{Лемма Бернсайда}
$G$ действует на $X$\\
$g \in G$\\
$X^{g} = \{x \in X \mid gx = x\}$\\

$X/G$ "--- множество орбит.
\begin{theorem} (лемма Бернсайда)
$|G| < \infty, |X| < \infty$, $G$ действует на $X$.

$|X/G| = \frac{\sum_{g \in G}|X^g|}{|G|}$\\

\end{theorem}
\begin{proof}
$\sum_{g \in G}|X^g| = \sum_{g \in G} \sum_{a \in X} \{(g, a) \mid ga = a\} 1 = \\
 = \sum_{a \in X}(\sum_{g \colon ga = a}1) = \sum_{a \in X} |St_a| = \sum_{a \in X}\sum\frac{|G|}{|G_a|} = \\
= \sum_{A \in X/G}\sum_{a \in A}\frac{|G|}{|G_a|} = \sum_{A \in X/G}\sum_{a \in A}\frac{|G|}{|A|} = \sum_{A \in X/G}\frac{|G|}{|A|} \sum_{a \in A}1 =\\
= \sum_{A \in X/G} \frac{|G|}{|A|}|A| = 
\sum_{A \in X/G}|G| = |G|\sum_{A \in X/G}1 = |G||X/G|$\\
\\
$|X/G| = \frac{\sum_{g \in G}|X^g|}{|G|}$\\
\end{proof}
\begin{exmp}
Подсчет количества цветных ожерелий.
Сколько различных ожерелий можно составить из
$n$ бусин $k$ цветов.

Различных с точностью до поворотов и отрожений. 

$n = 6, k = 3$\\
$|X| = k^n$\\
$D_{2n}$ "--- группа совмещений правильного n-угольника, 
диэдральная группа порядка 2n.\\
Число ожерелий = число орбит в действие $D_{2n}$ на $X$.

$D_{12}$ действует на $X$\\
$|X/D_{12}| = \frac{\sum_{g \in D_{12}}|X^g|}{12}$\\
$g = id$\\
$X^{id} = X$\\
$|X^g| = 3^6$\\
Поворот на $\pi$ нужно, что бы противоположные бусины имели одинаковый цвет. 
Поворот на $\pm \frac{2\pi}{3}$  $|X^g| = 3^2$\\
Поворот на $\pm \frac{\pi}{3}, |X^g| = 3$\\

Теперь отражения.
Два случая. Если через вершинку проходит ось симметрии. 
$|X^g| = 3^4$ \\
Если для стороны, то 
$|X^g| = 3^3$\\

$|X/G| = \frac{3^6 + 3^3 + 2\cdot 3^2 + 2\cdot 3 + 3 \cdot 3^4 + 3 \cdot 3^3}{12}$
\end{exmp}
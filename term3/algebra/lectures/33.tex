\section{Лемма Бернсайда}
\begin{Def}
$G$ действует на $X$\\
$g \in G$\\
$X^{g} = \{x \in X \mid gx = x\}$ "--- множеством неподвижных точек.
\end{Def}

\begin{theorem} (лемма Бернсайда)

$X/G$ "--- множество орбит.\\
$|G| < \infty, |X| < \infty$, $G$ действует на $X$.\\
$|X/G| = \frac{\sum_{g \in G}|X^g|}{|G|}$\\
\end{theorem}
\begin{proof}
Посчитаем число стоящие в числителе двумя разными способами. 

$\sum_{g \in G}|X^g| = \sum_{g \in G} \sum_{a \in X, \{(g, a) \mid ga = a\}} 1 = \\
 = \sum_{a \in X}(\sum_{g \colon ga = a}1) = \sum_{a \in X} |\St_a| = \sum_{a \in X}\frac{|G|}{|Ga|} = \\
= \sum_{A \in X/G}\sum_{a \in A}\frac{|G|}{|Ga|} = \sum_{A \in X/G}\sum_{a \in A}\frac{|G|}{|A|} = \sum_{A \in X/G}\frac{|G|}{|A|} \sum_{a \in A}1 =\\
= \sum_{A \in X/G} \frac{|G|}{|A|}|A| = 
\sum_{A \in X/G}|G| = |G|\sum_{A \in X/G}1 = |G||X/G|$\\
\\
$|X/G| = \frac{\sum_{g \in G}|X^g|}{|G|}$\\
\end{proof}
\begin{exmp}
Подсчет количества цветных ожерелий.\\
Сколько различных ожерелий можно составить из
$n$ бусин $k$ цветов.

Различных с точностью до поворотов и отрожений.\\ 
$n = 6, k = 3$\\

$X$ "--- n бусин, нумеруем и ставим по кругу, бусины раскрашены в $k$ цветов.\\
$|X| = k^n$\\

$D_{2n}$ "--- группа совмещений правильного n-угольника, 
диэдральная группа порядка 2n.\\
В одну орбиту попадают не различимые ожирелья, поэтому 
число ожерелий = число орбит в действие $D_{2n}$ на $X$.

$D_{12}$ действует на $X$\\
по лемме Бернсайда $|X/D_{12}| = \frac{\sum_{g \in D_{12}}|X^g|}{12}$\\

Давай-то посмотрем,какие бывают преобразования. 
\begin{enumerate}
\item $g = id$\\
$X^{id} = X$\\
$|X^g| = 3^6$\\
\item 
Поворот на $\pi$ нужно, что бы противоположные бусины имели одинаковый цвет. $|X^g| = 3^3$
\item
Поворот на $\pm \frac{2\pi}{3}$  $|X^g| = 3^2$\\
\item
Поворот на $\pm \frac{\pi}{3}, |X^g| = 3$
\item
Теперь отражения.
Два случая. Если через вершинку проходит ось симметрии. 
$|X^g| = 3^4$ \\
\item
Если для стороны, то 
$|X^g| = 3^3$\\
\end{enumerate}

$|X/G| = \frac{3^6 + 3^3 + 2\cdot 3^2 + 2\cdot 3 + 3 \cdot 3^4 + 3 \cdot 3^3}{12} = 92$
\end{exmp}
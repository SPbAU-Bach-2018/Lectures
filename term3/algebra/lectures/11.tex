\section{Овеществление и комплексификация}
\subsection{Овеществление}
$\C^n \sim \R^{2n}$

$\mathscr{A} \in End(\C^n)$

Рассмотрим $\mathscr{A}$ как оператор на $\R^{2n}$

$e_1, \cdots, e_n$ "--- стандартный базис $\C^n$.

$e_1, \cdots, e_n$, $ie_1, \cdots, e_n$ "--- базис в $\R^{2n}$

$$v = \sum \lambda_j e_j = \sum (\Re \lambda_j)e_j + (\Im \lambda_j)(i e_j)$$

$$\mathscr{A}(\sum a_j e_j + \sum b_j(i e_j)) = \mathscr{A}(\sum(a_j + i b_j)e_j) = \sum(a_j + ib_j)\mathscr{A}(e_j) =$$
$$\sum a_j \mathscr{A}(e_j) + \sum b_j i \mathscr{A}(e_j) = \sum a_j \mathscr{A}(e_j) + \sum b_j \mathscr{A}(ie_j)$$
$\Ra \mathscr{A}$ "--- линейный оператор в 2n "--- мерном вещественном пространстве.

Над $\C\colon [\mathscr{A}] = (\alpha_{kj})_{k, j = 1, \cdots, n}$ 

$[\mathscr{A}]$ "--- матрица $\mathscr{A}$ в стандартном базисе. 

$\alpha_{kj} = \beta_{kj} + i \gamma_{kj}$

$\mathscr{A} \in End(\C^n), e_1, ie_1, \cdots, e_n, ie_n$

$\mathscr{A}e_j = \sum_k \alpha_{kj}e_j = \sum (\beta_{kj}e_j + \gamma_{kj}ie_j) = 
\sum_k(-\gamma_{kj}e_j + \beta_{kj}(ie_j))$ 

$\alpha_{kj} \to \begin{pmatrix}
\beta_{kj} & -\gamma_{kj} \\
\gamma_{kj} & \beta_{kj}\\
\end{pmatrix}$

Получили матрицу оператора $\mathscr{A}$  рассматриваемого как $\R$- оператор. 

Дословно повторяется дла $V, \dim_{\C}V = n, \mathscr{A} \in \End_{\C}(V)$

$\dim_{\R}V = 2n, \mathscr{A} \in End_{\R}(V)$

$$U(1), SO(2)$$
$$(\cos\phi + i \sin \phi) \to 
\begin{pmatrix}
\cos \phi & -\sin \phi\\
\sin \phi & \cos \phi\\
\end{pmatrix}$$

\subsection{Комплексификация} 

$V$ "--- пространство над $\R$  $\dim V = n$

строим новое пространство $V_{\C}$ "--- над  $\R$ должно иметь размерность $2n$.

$V_{\C} = V \oplus V'$ 
$V' = $ "копия $V$"

$V_{\C} = V \times V = \{(v, u)| v, u \in V\}$

$V \sim V \times \{0\}$

$V' \sim \{0\} \times V$ 

$(v, u) = (v, 0) + (0, u)$

$$\alpha = \beta = i \gamma$$
$$\alpha(u, v) = (\beta + i \gamma)(u, v) = (\beta u - \gamma v, \beta v + \gamma u)$$

Упражнение $V_{\C}$ "--- векторное пространство над $\C$.

Отождествляем V с $V \times \{0\}$

$i(u, 0) = (0, u)$

$v_1, \cdots, v_n$ "--- базис $V_{\C}$ над $\C$.
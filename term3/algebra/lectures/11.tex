\section{Овеществление и комплексификация}
\subsection{Овеществление}

\[ \C^n \sim \R^{2n} \]

$\mathcal A \in End(\C^n)$.
Рассмотрим $\mathcal A$ как оператор на $\R^{2n}$.\\
$e_1, \dots, e_n$ "--- стандартный базис $\C^n$.
$e_1, \dots, e_n, ie_1, \dots, e_n$ "--- базис в $\R^{2n}$.
\begin{gather*}
	v
	= \sum \lambda_j e_j
	= \sum (\Re \lambda_j) e_j + (\Im \lambda_j) (i e_j) \\
	\mathcal A \left(\sum a_j e_j + \sum b_j(i e_j)\right)
	= \mathcal A \left(\sum(a_j + i b_j)e_j \right)
	= \sum (a_j + ib_j) \mathcal A(e_j) = \\
	= \sum a_j \mathcal A(e_j) + \sum b_j i \mathcal A(e_j)
	= \sum a_j \mathcal A(e_j) + \sum b_j \mathcal A(ie_j)
\end{gather*}
Получили, что$\mathcal A$ "--- линейный оператор в $2n$-мерном вещественном пространстве.

Над $\C$: $[\mathcal A] = (\alpha_{kj})_{k, j = 1..n}$.
$[\mathcal A]$ "--- матрица $\mathcal A$ в стандартном базисе.
\begin{gather*}
	\alpha_{kj} = \beta_{kj} + i \gamma_{kj} \\
	\mathcal A \in End(\C^n) \quad e_1, ie_1, \cdots, e_n, ie_n \\
	\mathcal Ae_j
	= \sum_k \alpha_{kj}e_j
	= \sum (\beta_{kj}e_j + \gamma_{kj}ie_j)
	\mathcal A(ie_j) = \cdots 
	= \sum_k(-\gamma_{kj}e_j + \beta_{kj}(ie_j)) \\
	\alpha_{kj} \to \begin{pmatrix}
		\beta_{kj}  & -\gamma_{kj} \\
		\gamma_{kj} & \beta_{kj}
	\end{pmatrix}
\end{gather*}
Получили матрицу оператора $\mathcal A$ рассматриваемого как $\R^{2n}$-оператор.
Дословно повторяется дла $V$, $\dim_{\C}V = n$, $\mathcal A \in \End_{\C}(V)$.

    $$\dim_{\R}V = 2n, \mathcal A \in End_{\R}(V)$$
\begin{exmp}
\begin{gather*}
	U(1) \quad SO(2) \\
	(\cos\phi + i \sin \phi) \to \begin{pmatrix}
		\cos \phi & -\sin \phi \\
		\sin \phi &  \cos \phi
	\end{pmatrix}
\end{gather*}
\end{exmp}

\subsection{Комплексификация}

$V$ "--- векторное пространство над $\R$, $\dim V = n$.
Cтроим новое пространство $V_{\C}$ векторное пространчтво над $\C$ "--- над $\R$ должно иметь размерность $2n$.
\[ V_{\C} = V \oplus V' \]
$V'$ "--- <<копия>> $V$.
\begin{gather*}
	V_{\C} = V \times V = \{(v, u)| v, u \in V\} \\
	\begin{aligned}
		V  &\sim V \times \{0\} \\
		V' &\sim \{0\} \times V
	\end{aligned} \\
	(v, u) = (v, 0) + (0, u) \\
	\alpha = \beta + i \gamma \\
	\alpha(u, v) = (\beta + i \gamma)(u, v) = (\beta u - \gamma v, \beta v + \gamma u)
\end{gather*}

\textbf{Упражнение: }
$V_{\C}$ "--- векторное пространство над $\C$.
$$V_{\C} = \{(v, 0)\} \oplus \{(0, u)\}$$
$$i(v, 0) = (0, u)$$
Отождествляем $V$ с вещественным подпространством $V \times \{0\}$.
$i(u, 0) = (0, u)$.
$v_1, \dots, v_n$ "--- базис $V_{\C}$ над $\C$.
\begin{gather*}
	V_{\C} = \{(v, 0)\} \oplus \{(0, u)\} \\
	V_{\C} = V \oplus iV
\end{gather*}
$\{v_i\}$ "--- базис V над $\R$.
$\{(v_i, 0)\}$ "--- базис $V_{\C}$ над $\C$.

Отождествление $v_i$ и $(v_i, 0)$.
По $V$ строим $V_{\C}$ (по $\R$ строим $\C$ соответствует $\dim V = 1$).
\[ v_{i\C} = (v_i,0) \]
\begin{Def}
	$V$, $V_{\C}$, $\dim_{\R}V = n < \infty$, $\mathcal A \in End(V)$.
	$\mathcal A_{\C} \in End(V_{\C})$ "--- комплексикация $\mathcal A$.
\end{Def}
Положим $\mathcal A_{\C} ((v, u)) = (\mathcal Av, \mathcal Au)$.
Адитивность $\mathcal A_{\C}$ очевидна.
\begin{gather*}
	\mathcal A_{\C}((\alpha + \beta i)(v, u))
	= \mathcal A_{\C}(\alpha v - \beta u, \beta v + \alpha u) = \\
	= (\mathcal A(\alpha v - \beta u), \mathcal A(\beta v + u))
	= (\alpha \mathcal A(v) - \beta \mathcal A(u), \beta \mathcal A(v) + \alpha \mathcal A(u)) = \\
	= (\alpha + \beta i)(\mathcal A(v), \mathcal A(u)) = (\alpha + \beta i)\alpha_{\C}(v, u)
\end{gather*}
$\mathcal A_{\C} \to \C$ "--- линейный оператор.
\begin{gather*}
	\mathcal A_{\C}|_{V} = \mathcal A|_{v \times \{0\}} \\
	\mathcal A_{\C}|_{V} = \mathcal A \\
	\mathcal A_{\C}((v, 0)) = (\mathcal Av, 0) \\
	[\mathcal A]_{\{v_i\}} = (a_{ij}) \\
	\mathcal A v_j = \sum_{i = 1}^n a_{ij}v_i \\
	\mathcal A_{\C}(v_{j_{\C}})
	= \mathcal A_{\C}((v_j, 0))
	= (\mathcal Av_j, 0)
	= \left(\sum_{i = 1}^{n}a_{ij}v_i, 0\right)
	= \sum_{i = 1}^{n}a_{ij}(v_i, 0)
	= \sum_{i = 1}^{n}a_{ij}v_{i_{\C}} \\
	[\mathcal A_{\C}]_{v_{i\C}} = [\mathcal A]_{\{v_i\}}
\end{gather*}
$\{v_i\}$ "--- базис V, над $\R$.
$\{(v_i, 0)\}$ "--- базис $V_{\C}$ над $\C$
\begin{gather*}
	(v, v)
	= \left(\sum \alpha v_j, \sum \beta_{j}v_j\right)
	= \sum \alpha_j(v_j, 0) + \sum \beta_j(0, v_j)
	= \sum \alpha(v_j, 0) + \sum i \beta_j(v_j, 0)
	= \sum(\alpha_j + i \beta_i)(v_j, 0)
\end{gather*}

\begin{exmp}
	\begin{enumerate}
	\item
		Всякий самосопряженный оператор в евклидовом пространстве можно рассматривать как
		cамосопряженный оператор в унитарном пространстве той же размерности.
	\item
		Всякий ортогональный оператор в $\R^n$ можно рассматривать как унитарный оператор той же размерности в $\C^n$.
	\end{enumerate}
\end{exmp}
\begin{conseq}
   \[ O(n) \le U(n) \]
\end{conseq}

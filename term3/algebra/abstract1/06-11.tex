\section{} % 06
Есть пространство с невырожденной формой (полуторалинейная/билинейная), есть эндоморфизм $\mathcal{A}$.
Тогда $\mathcal{A}^*$ "--- \textit{сопряжённый} к $\mathcal{A}$, если $(\mathcal{A}x, y) = (x, \mathcal{A}^*y)$.
В конечномерном сопряжённый существует и единственен.
Единственен, так как можно выписать формулу в конкретном базисе по матрице Грама и получить условие $A^\top \Gamma = \Gamma \overline{A^*}$,
а $\Gamma$ невырожденная, т.е. $A$ однозначно.
Существование просто подстановкой формулы для $A=[\mathcal{A}]$.
Следствие: в ортонормированном базисе матрица сопряжённого эрмитово сопряжённа с транспозицией матрицы (если инволюции нет "--- просто равна ей).

Есть невырожденная форма, эндоморфизм $\mathcal{A}$ и подпространство $U$, которое $\mathcal{A}$-инвариантно.
Тогда $U^\bot$ "--- $\mathcal{A}^*$-инвариантно (просто расписали по определениям, \TODO не нужна ли какая-нибудь симметричность формы?).
Пример: если $[\mathcal{A}] = \begin{matrix}*&*\\0&*\end{matrix}$ в базисе, где первая половина "--- вектора $U$,
тогда сопряжённый оператор по формуле имеет красивую матрицу, из которой очевидна $\mathcal{A}^*$-инвариантность $U^\bot$.

В конечномерном (чтобы сопряжённый был и единственный) $V$: если невырожденная форма (чтобы можно было говорить про сопряжённость), (просто/косо/эрмитово) симметричная форма,
тогда $(A^*)^*=A$ (порасписывали, начиная с $\overline{(y, \mathcal{A}x)}$).

\section{} % 07
Если есть оператор $\mathcal A$, можно ввести новую билинейную/полуторалинейную форму на пространстве $V$: $\langle x, y \rangle = (\mathcal{A}x, y)$.
Она тоже билинейная/полуторалинейная, но невырожденность может пропадать ($\mathcal A$ может быть необратим).

Оператор самосопряжён, если $\mathcal{A}^* = \mathcal{A}$.
Если форма $(,)$ (эрмитово) симметричная и невырожденная, есть $\mathcal A^*$, то (эрмитова) симметричность $\langle , \rangle$ $\iff$ самосопряжённость $\mathcal A$.
$\La$ очевидно, для $\Ra$ надо расписать симметричность, получить $(y, \mathcal A^* x) = (y, \mathcal A x)$, вычесть, по невырожденности $\mathcal A^* x = \mathcal A x$, по произвольности $x$ что надо.
В конечномерном можно расписать матрицу Грама для $(,)$ и $\langle , \rangle$ ($\Gamma' = A^\top \Gamma$), получить условие на матрицу самосопряжённого оператора $\mathcal A$.
Отсюда (в конечномерном $V$): невырожденность $\langle , \rangle$ $\iff$ обратимость $\mathcal A$, т.к. $\det \Gamma \ne 0$.

Пусть есть невырожденная (эрмитово) симметричная форма в конечномерном.
Тогда всякая другая (эрмитово) симметричная имеет вид $\langle x, y \rangle = (\mathcal A x, y)$ (для некоторого самосопряжённого $\mathcal A$).
Док-во: \TODO  % стр. 22

\section{} % 08
Пусть $V$ "--- евклидово или унитарное со стандартным скалярным произведением, тогда $\mathcal A$ "--- \textit{нормальный}, если он коммутирует
со своим сопряжённым ($\mathcal A \mathcal A^* = \mathcal A^* \mathcal A$.
Примеры нормальных: самосопряжённый, ортогональный/унитарный (обратили изомерию, получили $y=\mathcal A^{-1}z$, отсюда $(x, \mathcal A^*z)=(x,\mathcal A^{-1}z)$,
вычли, по невырожденности $\mathcal A^*=\mathcal A^{-1}$, а они коммутируют).

В унитарном: $\mathcal A$ нормален $\iff$ есть ортонормированный базис $V$, в котором $[\mathcal A]$ диагональна.
$\La$: посчитали $[\mathcal A^*]$ по формуле, она тоже диагональна, такие коммутируют.
$\Ra$: взяли собственное число $\lambda$ оператора $\mathcal{A}$ (такое всегда есть, комплексное, например) и рассмотрели $U=U_1(\lambda)$ (пространство собственных векторов, отвечающих $\lambda$).
Знаем, что $U$ "--- $\mathcal A$-инвариантно, хотим показать, что оно же $\mathcal A^*$-инвариантно (взяли $u \in U$, применили $\mathcal A^*$, проверили,
что собственный вектор для $\mathcal A$ с числом $\lambda$).
Тогда $U^\bot$ "--- $\mathcal A^*$ и $\mathcal A^{**}=A$-инвариантно, а $V=U \oplus U^\bot$ (причём оба слагаемых инвариантны),
то есть можно сузить $\mathcal A$ и $\mathcal A^*$ и получить на $U, U^\bot$ эндоморфизмы (причём сопряжённость всё еще останется, и нормальность $\mathcal A$ тоже).
Индукцией доказали, что для них всё ок, а потом взяли соответствующие базисы в них и матрица оператора получилась блочно-диагональной из двух блоков, то есть просто диагональной.

%Если $\mathcal A$ "--- нормальный оператор, а $u$ "--- собственный вектор $\mathcal$, отвечающий $\lambda$, то $u$ "--- собственный для $\mathcal A^*$ и отвечает $\bar\lambda$.
\TODO  % стр 23, следствие 1.10.1.1

\section{} % 09
В унитарном простренства оператор самоспоряжён $\iff$ есть ортонормированный базис, в котором $[ \mathcal A ]$ диагональная \textit{и вещественная}.
Док-во: возьмём базис, где матрица диагональна, $A = A^* = \overline{A^\top} = \overline{A}$, т.о. все числа вещественные.
И наоборот: если есть базис с хорошей матрицей, то по формуле показали равенство.
Отсюда: собственные числа у самоспоряжённых $\in \R$.

Теорема 1.11.2? \TODO

\section{} % 10
Если есть $\C^n$ и на нём полуторалинейная эрмитово симметричная форма $\langle , \rangle$, то есть
базис, в котором матрица Грама диагональна, а на диагонали стоят (в порядке): единицы, минус единицы, нули.
Док-во: мы знаем, что $\langle x, y \rangle = (\mathcal Ax, y)$ для некоторого самосопряжённого $\mathcal A$,
а $\Gamma' = [\mathcal]^\top$. \TODO

Лемма 1.11.1? \TODO

\section{} % 11

\TODO

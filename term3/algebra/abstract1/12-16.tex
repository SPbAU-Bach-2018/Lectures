\section{} % 12

\TODO

\section{} % 13

\TODO

\section{} % 14
Овеществление: взяли эндоморфизм $\mathcal A$ для $\C^n$, взяли базис в $\R^{2n}$ вида $e_k, ie_k$, $\mathcal A$ в таком базисе линейный оператор (т.е. вообще линейный оператор).
Можно записать матрицу, она получится красивая из четырёх блоков (например, в комплексных есть поворот, и в вещественных есть поворот).

Комплексификация: было пространство $V=\R^n$, построили новое пространство вида $V\oplus V'$ (где $V'$ "--- копия $V$, сюда пойдут <<мнимые>> части комплексных чисел),
$V$ в него просто вложили по первым координатам, а комплексификация оператора $\mathcal A$ будет естественно выглядеть как $\mathcal A$, применённый к комплексным
числам: $\mathcal A (u, v) = (\mathcal Au, \mathcal Av)$.
Надо проверить аддитивность и линейность этого оператора, успех.

Следствие: можно самосопряжённые в евклидовом рассматривать как самосопряжённые в унитарном (той же размерности),
а всякий ортогональный рассматривать как унитарный той же размерности.
Следствие: $O(n) \le U(n)$.

\section{} % 15
Пусть $SU(n)$ "--- все унитарные матрицы с определителем $1$ (матрицы операторов-изомерий $\C^n$, для них $A^{-1}=\overline{A^\top})$).

\TODO

\section{} % 16

\TODO

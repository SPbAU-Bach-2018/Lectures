\section{} % 17
Пусть $V=\R[x]$ (или можно ограничить степень, тогда будет $V_0 \subset V_1 \subset \dots$), базис "--- $x^k$.
Форма: $(f, g) = \int_{-1}^1 f(x)g(x) \d x$, положительно определённая.
Можно ортогонализировать базис, например, до многочленов Лежандра: $P_n(x) = \frac{1}{2^nn!}\frac{\d^n}{\d x^n}(x^2-1)^n$ (это формула Родрига).
У них степени равны $n$, а в единице они равны единице.
Док-во: они линейно независимы, первые $k$ порождают $V_{k-1}$, надо проверить ортогональность.
Взяли любой полином $f(x)$ степени меньше $n$, подставили в скалярное произведение,
по частям, подстановка убилась (так как $(n-1)$-я производная штуки делится на $x^2-1$),
повторили еще $(n-1)$ раз, в конце имеем $n$-ю производную $f$, а она ноль.
В единице значение единица, так как расписали зашитую производную по биному для производных, все слагаемые (кроме одного) убились, осталось ровно $n!(1+1)^n$ в числителе.
Еще есть странная теорема 1.17.2 без доказательства \TODO

Чебышева: произведение вида $\int_{-1}^1 \frac{f(x)g(x)}{\sqrt{1 - x^2}} \d x$, при этом считаем $x=\cos \phi$ (и $\d x$ соответственно),
можно преобразовать до $\int_0^{\pi} f(\cos \phi)g(\cos \phi) \d \phi$.
Скажем, что $\cos(n\phi)=T_n(\cos\phi)$ (многочлен Чебышева), тогда понятно, что $(T_n, T_m)=0$ при $n \ne m$ (формула для произведения синусов через синус разности/суммы).
Рекурсивная формула: $T_n(x)=2xT_{n-1}(x)-T_{n-2}(x)$.
Еще можно ввести некие <<ортогональные многочлены с весом $\sqrt{1-x^2}$>>: $\sin(n+1)\phi = U_n(\cos \phi)\sin \phi$.

Лагерра: $(f,g)=\int_0^\infty f(x)g(x)e^{-x}\d x$, тогда $L_n(x)=\frac{e^x}{n!}\frac{\d^n}{\d x^n}(x^ne^{-x})$.

Эрмита: $(f, g)=\int_{-\infty}^\infty f(x)g(x)e^{-x^2}\d x$, тогда $H_n(x) = (-1)^n e^x \frac{\d^n}{\d x^n}(e^{-x^2})$.

Якоби (без явной формулы), обозначаются $J_n^{\alpha,\beta}(x)$: $(f, g)=\int_{-1}^1 (x-1)^\alpha(x+1)^\beta f(x) g(x) \d x$ при $\alpha, \beta > -1$.

\section{} % 18
Пусть у нас многочлены Лежандра и $\int_{-1}^1 f(x)g(x) \d x$.
Можно рассмотреть линейный оператор $D(f)=(x^2-1)f''(x)+2xf'(x)$, он, внезапно, самосопряжён $(Df, g)=(f, D^*g)$.
Можно сужать на $V_n$, тогда можно строить ортонормированный базис из собственных многочленов этого оператора.
Оказывается, многочлены Лежандра подойдут "--- у $P_m$ собственное число равно $m(m+1)$ (проверяется в лоб).
\TODO

Для Чебышева, Лагерра, Эрмита (без Якоби?) тоже есть какие-то дифференциальные операторы, для которых они "--- собственные функции.

\section{} % 19
Тригонометрические многочлены: $a_0+a_1\cos x + a_2\cos 2x + \dots + a_n\cos nx + b_1\sin x + \dots + b_n\sin nx$, размерность равна $2n+1$.
Произведение: $(f, g)=\int_0^{2\pi} f(x)g(x) \d x$.
Можно ввести оператор $D=\frac{\d}{\d x}$ (просто дифференцирование), найти сопряжённый, проинтегрировав по частям (получим $-D$).
А вот $\left(\frac{\d^2}{\d x^2}\right)$ самосопряжён.
И еще у него есть собственные числа: $0$ (для $f(x)=1$), $-n^2$ (для $f(x)=\cos nx$ и $f(x)=\sin nx$).

Можно найти базис: это $1$, $\cos nx$, $\sin nx$: размерность не больше $2n+1$, всё порождают.
Но они еще и попарно ортогональны, что прикольно.

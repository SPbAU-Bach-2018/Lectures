\section{} % 01
Ортогональное дополнение: $S^\bot = \{ v \mid (v, s) = 0 \}$, порядок произведения важен (сначала вектор из $S^\bot$, потом из $S$).
Это подпространство (так как форма хотя бы полуторалинейная)/
Если $S_1 \subset S_2$, то $S_1^\bot \ge S_2^\bot$/
Орт. дополнение к порождаемому пространству ($\langle S \rangle$) равно $S^\bot$ (есть одно вложение, проверили оставшееся).
$\{0\}^\bot=V$ (очевидно).
Для невырожденной формы $V^\bot = \{0\}$ (очевидно).
Для (просто/косо/эрмитово) симметричной формы имеем $S \subset (S^\bot)^\bot$ (откуда $\langle S \rangle \le (S^\bot)^\bot$),
док-во: отношение ортогональности симметрично, взяли $s \in S$, показали $s \in (S^\bot)^\bot$.
Два подпространства $U_1, U_2$, тогда $(U_1+U_2)^\bot = U_1^\bot \cap U_2^\bot$, одно включение очевидно, показали второе.
Два подпространства $U_1, U_2$, тогда $(U_1\cap U_2)^\bot \ge U_1^\bot + U_2^\bot$, одно включение очевидно, показали второе.

С положительно определённой формой: $\langle S \rangle \cap S^\bot = \{0\}$ (контрпример: векторное произведение, оно только невырождено).
В конечномерных положительноопределённых: если $U \le V$ то $\dim U + \dim U^\bot = \dim V$ и прямая сумма даёт $V$,
док-во: взяли базис $U$, дополнили до базиса $V$, ортогонализовали, вторая половина векторов ортогональна $U$ и лежит в $U^\bot$
(т.о. $U^\bot$ хотя бы размерности второй половины), покажем, что это базис $U^\bot$ (разложили $v\in U^\bot$ по базису $V$, он $\bot$ всем базисным
$U$ в частности, т.е. первая половина векторов в разложении не участвует), отсюда знаем $\dim U^\bot$, знаем $\dim (U\cap U^\bot)$, отсюда
знаем размерность $U+U^\bot$, т.о. они в прямой сумме дают $V$.

В конечномерных положительноопределённых: $\langle S \rangle = (S^\bot)^\bot$
(предыдущий пункт $\Ra$ размерности совпадают, одно "--- подпространство второго, т.е. равны. \TODO контрпример).
В конечномерных положительноопределённых: $(U_1\cap U_2)^\bot = U_1^\bot + U_2^\bot$ (предыдущий пункт плюс пункт про дополнение к сумме).

\section{} % 02
Базис $\{e_j^*\}$ "--- взаимный базис к $\{e_i\}$, если $(e_i, e_j^*) \in \{0, 1\}$ (в зависимости от $i\stackrel{?}{=}j$),
т.е. каждый новый вектор перпендикулярен всем старым, кроме одного и <<параллелен>> оставшемуся.
Например, ортонормированный взаимен сам себе.

В конечномерных с невырожденной формой есть единственный взаимный.
Док-во: взяли матрицу Грама для $e_i$, взяли матрицу перехода от $e_j^*$ к $e_i$ ($e^*=Ce$), рассмотрели
матрицу скалярных произведений между $e_j^*$ и $e_i$ (она единична), она равна $\Gamma \bar C$ (расписали элементы матрицы, расписали, не забыв про полуторалинейность),
так как $\Gamma$ невырождена, можно найти обратную, получим $C$ (у которой тоже есть обратный $\Ra$ невырожденная $\Ra$ действительно базис).

\section{} % 03
Евклидово пространство: есть $\R^n$ и $(x, y) = \sum x_iy_i$ (в естественном базисе $\Gamma=E$).
Унитарное пространство: есть $\C^n$ и $(x, y) = \sum x_i\bar y_i$ (в естественном базисе $\Gamma=E$).
Метрика положительно определена (и невырождена), (эрмитово) симметрична.

Норма: $|x|=\sqrt{(x,x)}$, $d(x, y)=|x-y|$.
Метрика, так как симметрично по (эрмитовой) симметричности формы, тождество по положительноопределённости,
неравенство треугольника (переобозначили, покажем $|u+v|\le|u|+|v|$, \TODO).
Косинус угла "--- $\frac{(u,v)}{|u||v|}$, он от $-1$ до $1$ (так как $|(u, v)| \le |u||v|$, \TODO).

\section{} % 04
Есть пространство и невырожденная форма: либо (косо)симметричная билинейная, либо эрмитово симметричная полуторалинейная.
$\mathcal{A} \in \End(V)$ "--- изомерия, если $(\mathcal{A}x, \mathcal{A}y) = (x, y)$ (а $\mathcal{A}$ тогда "--- ортогональный оператор).
Например, поворот плоскости.

В конечномерном, если есть матрица Грама и матрица линейного отображения $A$, то $A^\top \Gamma \bar A = \Gamma$ $\iff$ изомеричность $A$ (разумеется, форма тут невырожденная).
Док-во: $\mathcal{A} x = Ax$, раскрыли скалярное произведение, получили точно условие.
В одну сторону очевидно, в другую: взяли вектора вида <<куча нулей и одна единица>>, и мы можем <<выбрать>> из матриц с каждой стороны равенства нужный элемент.
Следствие: так как форма невырожденаая, то изомерия обратима.
Следствие: изомерии образуют группу (нейтральный есть, обратный есть, ассоциативность есть).

Группа изометрий $O(V, B)$ называется \t{ортогональной группой, связанной с формой $B$} (у нас конечномерное пространствол, форма невырожденная, симметричная и билинейная).
В евклидовом пространстве над $\R^n$ есть \t{вещественная ортогональная группа} $O(n)$.
А еще в $\R^n$ можно ввести формы $\sum_1^p x_iy_i - \sum_{p+1}^n x_iy_i$ "--- это $O(p, q)$ ($p+q=n$).
В $\R^n$ все ортогональные группы эквивалентны одной из этих, будет показано позже.
$O(1, 3)$ есть группой изметрий пространства Минковского.

Аналогично бывают: симплектические группы $Sp(V, B)$, для кососимметрических форм; \TODO чем исчерпываются).
Еще бывают унитарные группы $U(V, B)$, для эрмитовосимметричных; можно аналогично ортогональным ввести $U(n)$ и $U(p, q)$.

\section{} % 05
Рассмотрим ортогональную группу $O(n)$.
Так как в евклидовом пространстве $\Gamma=E$, то условие на ортогональность оператора (по его матрице $A$) есть $A^{-1}=A^\top$ (такая матрица называется ортогональной).
Строки и столбцы такой матрицы попарно ортогональны и единичной длины, это даже равносильное условие (для столбцов "--- просто переформулировка условия; для строк "--- надо переставить местами $A$ и $A^{-1}$).
$\mathcal{A}$ "--- ортогональный оператор в евклидовом $\iff$ переводит ортонормированный базис в ортонормированный ($\Ra$ очевидно, для $\La$ взяли преобразование и проверили, что он "--- изометрия).
Пример: $O(1)=\{ \pm 1 \}$.
Пример с $O(2)$: записываем систему на ячейки матрицы, $\det A = \pm 1$ (так как $\det A^\top A = 1$), положили $SO(n)$ как множество матриц из $O(n)$ с определителем $1$,
тогда $O(2)$ есть $SO(2)$ плюс $SO(2)$ с отражением пространства (\TODO почему?), а из системы уравнений $SO(2)$ есть матрицы поворотов (\TODO почему?)

В унитарном пространстве бывают унитарные операторы.
Матрицы таких операторов "--- унитарные (условие получается $A^{-1}=\overline{A^\top}$.
Оператор унитарен $\iff$ переводит ортонормированный в ортонормированный.
Пример: $U(1)$  \TODO пример 1.7.7

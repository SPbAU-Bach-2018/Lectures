\section{} % 01
Ортогональное дополнение: $S^\bot = \{ v \mid (v, s) = 0 \}$, порядок произведения важен (сначала вектор из $S^\bot$, потом из $S$).
Это подпространство (так как форма хотя бы полуторалинейная)/
Tсли $S_1 \subset S_2$, то $S_1^\bot \ge S_2^\bot$/
Орт. дополнение к порождаемому пространству ($\langle S \rangle$) равно $S^\bot$ (есть одно вложение, проверили оставшееся).
$\{0\}^\bot=V$ (очевидно).
Для невырожденной формы $V^\bot = \{0\}$ (очевидно).
Для (просто/косо/эрмитово) симметричной формы имеем $S \subset (S^\bot)^\bot$ (откуда $\langle S \rangle \le (S^\bot)^\bot$),
док-во: отношение ортогональности симметрично, взяли $s \in S$, показали $s \in (S^\bot)^\bot$.
Два подпространства $U_1, U_2$, тогда $(U_1+U_2)^\bot = U_1^\bot \cap U_2^\bot$, одно включение очевидно, показали второе.
Два подпространства $U_1, U_2$, тогда $(U_1\cap U_2)^\bot \ge U_1^\bot + U_2^\bot$, одно включение очевидно, показали второе.

С положительно определённой формой: $\langle S \rangle \cap S^\bot = \{0\}$ (контрпример: векторное произведение, оно только невырождено).
В конечномерных положительноопределённых: если $U \le V$ то $\dim U + \dim U^\bot = \dim V$ и прямая сумма даёт $V$,
док-во: взяли базис $U$, дополнили до базиса $V$, ортогонализовали, вторая половина векторов ортогональна $U$ и лежит в $U^\bot$
(т.о. $U^\bot$ хотя бы размерности второй половины), покажем, что это базис $U^\bot$ (разложили $v\in U^\bot$ по базису $V$, он $\bot$ всем базисным
$U$ в частности, т.е. первая половина векторов в разложении не участвует), отсюда знаем $\dim U^\bot$, знаем $\dim (U\cap U^\top)$, отсюда
знаем размерность $U+U^\top$, т.о. они в прямой сумме дают $V$.

В конечномерных положительноопределённых: $\langle S \rangle = (S^\bot)^\bot$
(предыдущий пункт $\Ra$ размерности совпадают, одно "--- подпространство второго, т.е. равны. \TODO контрпример).
В конечномерных положительноопределённых: $(U_1\cap U_2)^\bot = U_1^\bot + U_2^\bot$ (предыдущий пункт плюс пункт про дополнение к сумме).

\section{} % 02


\section{} % 03


\section{} % 04


\section{} % 05



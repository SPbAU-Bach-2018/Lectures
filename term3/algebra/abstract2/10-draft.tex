Как комплексификация: было поле $K$, в него вложено $L$.
И еще было $V_K$ над $K$.
Хотим очень естественно расширить его до $V_L$ над $L$
(чтобы в нём было подмножество, изоморфное $V_K$: если смотреть на
вектора, у которых все координаты из $K$, то ведут себя как в $V_K$).

Замечание: $K \otimes V_K \cong V_K$.

$L$ "--- векторное пр-во над $K$ (так как можно складывать и домножать на элементы $K$ слева).
$V_K$ "--- тоже векторное пр-во над $K$.
(1) Можно их тензорно перемножить, потому что можем: $L \otimes V_K$, скажем, что это и есть $V_L$,
это какое-то векторное пространство над $K$.
Замечание: $V_K \cong K \otimes V_K \subseteq L \otimes V_K = V_L$, т.е. в $V_L$ можно
выбрать кусок, изоморфный $V_K$.

Осталось показать, что $V_L$ на самом деле векторное пространство над $L$.
Складывать мы уже умеем (это от поля, над которым мы строим пространство, не зависит).
Осталось определить умножение <<векторов>> на элементы $L$, т.е. задать операцию $L \times V_L \to V_L$.
Мы хотим, в частности, чтобы эта операция, суженная на $K \times V_L$ давало старое умножение (1).

Давайте введём операцию так (пусть $\beta \in L$): $\beta \cdot (l \otimes v) \eqDef (\beta l) \otimes v$.
Это мы ввели умножение на мн-ве разложимых тензорах.
Оно согласовано с умножением (1), если мы умножаем на элемент $K$.
Теперь пробуем распространить это на все тензоры: если у нас есть разложение тензора,
то определяем естественно; Надо проверить корректность: разложений тензора может быть несколько.

Что значит, что два разложения дают один и тот же тензор?
Если мы вычтем два разложения, то получим что-то из $M_0$, то есть
что-то представимое в виде конечной линейно суммы образующих $M_0$,
то есть одно разложение можно перевести в другое за конечное число шагов
<<преобразовали один разложимый тензор из суммы по линейности>>.

То есть нам надо проверить, что умножение элементов из $M_0$ на $\beta$ тождество всё
еще сохраняет.
Это просто проверяем.

Осталось понять, что происходит с базисом.

Надо еще что-то с операторами проверить: что их тоже можно как-то перевести в V_L так, чтобы
они на подмножестве V_K работали так же.

\section{} % 01
$V$ "--- векторное пр-во над полем $K$, есть $U < V$, говорим $u_1 \sim u_2 \iff v_1-v_2 \in U$,
определяем сложение и умножение классов эквив. на $c$ (как будто без $U$),
проверяем корректность.
Фактормножество $V/~$ "--- векторное пр-во (абелева группа по сложению, умножение
ассоциативно и дистрибутивно двумя способами, есть единица).
Обозначается $V/U$.

Из прошлого семестра: относительная линейная независимость набора
(линейная комбинация векторов плюс кто-то из $U$ ноль $\iff$ коэффициенты нолЬ),
отн. семейство образующих (любой из $V$ представляется как $u$ плюс комбинация),
отн. базис (и отн. лин. нез. и отн. сем. обр.).
Базис у $V/U$ $\iff$ отн. базис $U$ (док-во: \TODO).

\section{} % 02
Пусть $V, W$ "--- век. пр-ва над $K$, есть $U < V$ и линейная $f \colon V \to W$.
Будем считать, что если $U \subseteq \ker f$, то можно построить естественное
$\bar f \colon V/U \to W$: $\bar f(v+U) = \bar f(v)$, проверить корректность, линейность.

\section{} % 03
Сначала строим $M$ "--- финитные функции из $V_1 \times \dots \times V_n$ в $K$
(почти всюду равные нулю), это векторное пр-во над $K$, его базис "---
$\delta_{v_1, \dots, v_n}$ (равны единице ровно в одной точке); если $K$ бесконечно
и хотя бы одно не нульмерно, то $M$ бесконечномерно.
$\delta_{(v_1, \dots, v_n)}$ часто обозначается просто как $(v_1, \dots, v_k)$ (HATE IT, но это очень базисный элемент,
лень писать "--- привет из многочленов, в которых за $x^k$ обозначается финитная ф-ция $\mathcal N \to K$).
Строим $M_0$ "--- это элементы $M$ вида $\delta_{(a+b,c)}-\delta_{(a,c)}-\delta_{(b,c)}$ и
$\delta_{(\alpha a, \alpha b)} - \alpha\delta_{(a, b)}$.
$M_0$ "--- подпространство (проверить два свойства).

Теперь берём фактор $M/M_0=V_1\otimes \dots \otimes V_n$ (тензорное произведение, элементы "--- тензоры).
Обозначаем некоторые элементы: $v_1\otimes \dots \otimes v_n = \delta_{(v_1, \dots, v_n)} + M_0$
(т.н. разложимый тензор).
Они порождают все тензоры (так как элемент $M$ "--- конечная сумма $\delta$)
но базисом не являются: между ними куча линейных зависимостей.

\section{} % 04
Построим каноническое отображение из $V_1 \times \dots \times V_n$ в $V_1 \otimes \dots \otimes V_n$:
переводим вектор $(v_1, \dots, v_n)$ в элемент $v_1\otimes\dots\otimes v_n=\delta_{(v_1, \dots, v_n)}$.
Оно полилинейно (линейно по каждому аргументу, проверить два свойства).

\section{} % 05
Пусть есть полилинейное отображение: $\alpha \colon V_1 \times \dots \times V_n \to W$.
Тогда есть единственное линейное $\beta \colon V_1 \otimes \dots \otimes V_n \to W$ такое,
что $\alpha = \beta \circ \phi$ ($\phi$ "--- каноническое $V_1 \times \dots \times V_n \to  V_1 \otimes \dots \otimes V_n$).
Станет хорошо: было полилинейное из чего-то простого в $W$, стало линейное из чего-то сложного в $W$,
перенесли сложность из отображений в пространство.

Единственность очевидна: значение на образующих (т.е. разложимых тензорах) однозначно (т.к. каждый
разложимый "--- образ понятно чего в $\phi$), а из них однозначно вычисляется значение на остальных
тензорах (тензор "--- конечная сумма разложимых).
Существование: сначала зададим $\tilde\beta \colon M \to W$, в образующих понятно как ($\alpha$),
на остальных элементах (конечные наборы векторов) "--- по линейности.
Теперь покажем, что $M_0 \subseteq \ker \tilde\beta$, тогда сможем построить $\beta$.
Действительно, берём $x \in M_0$, раскрываем до образующих, используем полилинейность
$\alpha$ (доказали два нужных свойства из $M_0$).
Т.о. есть $\beta$.
Проверим, что $\alpha = \beta \circ \phi$, успех.

\section{} % 06
Пусть $\mathcal L(A_1, \dots, A_n, B)$ "--- пространство полилинейных отображений $A_1 \times \dots \times A_n \to B$
(если $n=1$ "--- просто линейных).
Тогда $\mathcal L(V_1, \dots, V_n, W) \cong \mathcal L (V_1 \otimes \dots \otimes V_n, W)$ "--- свели полилинейные к линейным.
Док-во: слева стоит векторное пр-во (сложение и умножение на скаляр поточечное), проверили 8 св-в;
справа тоже векторное пр-во; а построенное в теореме преобразование линейно (сложили две функции $\Ra$
получили сложение функций от тензоров; аналогично с константами).
Проверим сюръективность (взяли линейное $\beta$, взяли $\alpha=\beta \circ \phi$, оно получилось
полилинейным из-за полилинейности $\phi$, его наше преобразование могло перевести только в $\beta$ по единственности
$\beta$.
И инъективность: для линейных $\iff$ тривиальность ядра (а ядро очевидно тривиально),
т.к. можно вычитать параметры.
Итого изоморфзим.

\section{} % 07
Размерность тензорного пр-ия равна размерности двойственного к тензорному (т.е. линейным из тензорного в $K$),
а оно $\cong$ полилинейным из $V_i$ в $K$.

Лемма: если в тензорном произведении есть нулевое пр-во $V_i$, то само произведение нульмерно.
Док-во: любое полилинейное отображение линейно по $V_i$, но оно м.б. только в ноль (т.к. домножили этот вектор на ноль,
он не изменился, значение ф-ции стало нулём), ч.т.д.

Теорема: размерность тензорного есть произведение размерностей.
Если хотя бы одно нулевое, то см. лемму
Иначе см. на полилинейные, пусть базис в $V_j$ есть $e_1^j, e_2^j, \dots, e_{d_j}^j$, тогда любое полилинейное
задаётся значениями на всех $e_i^j$, а их сколько надо.
При этом для любых значений на $e_i^j$ получается полилинейное, все разные.
\TODO а в Кострикине что-то проще конспекта!

\section{} % 08
Пусть базис $V_j$ есть $e_1^j, e_2^j, \dots, e_{d_j}^j$,
тогда базис тензорного есть $e_{i_1}^1 \otimes \dots \otimes e_{i_n}^n$.
Это семейство образующих: каждый разложимый тензор порождается такими (разложили по очереди аргументы $\otimes$ по базису и линейности),
а разложимые порождают всё.
Их сколько размерность $\Ra$ они базис.
В бесконечномерном случае: это всё еще семейство образующих, про линейную независимость от противного: берём конечную лин.комб.
базисных, из каждого пр-ва есть конечное число базисных, сузили пр-ва до конечномерных, применили предыдущий пункт.

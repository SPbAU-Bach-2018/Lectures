\section{} % 21
$B$ "--- подгруппа $A$ ($B \le A$, не $B \subseteq A$).
Произведение мн-в по Минковскому: $A\cdot B=\{a \cdot b \mid a \in A, b \in B \}$, ассоциативно, некоммутативно.
Сокращение: $\{a\}B = aB$ (это \textit{левый} смежный класс эл-та $a$ по мн-ву $B$, есть правый).
Можно ввести $A^{-1}=\{a^{-1}\mid a \in A\}$.
Если $H \le G$, то $H\cdot H = H$ и $H^{-1}=H$.
Бывает аддитивная запись ($+$ и 0 вместо $\cdot$ и 1).
Гомоморфизм $f \colon G_1 \to G_2$: $f(a\cdot b) = f(a) \cdot f(b)$ ($a, b \in G_1$).
Изоморфизм "--- гомоморфизм и биекция.
Мономорфизм/эпиморфизм "--- инъекция/сюръекция.

Нормальная подгруппа $H$ ($H \unlhd G$), равносильные определения нормальности начинаются на $\forall g \in G$,
выглядят так: $gHg^{-1} \subseteq H$ (сопряжение $H$ элементом $g$ есть подмножество $H$), $gHg^{-1} = H$, $gH=Hg$.
$2\Ra 1$ очевидно, $1\Ra 2$ (взяли условие, сопрягли $g^{-1}$, получили второе включение),
$2 \Ra 3$ (домножили на $g$ справа), $3 \Ra 2$ (домножили на $g^{-1}$ справа).

В нормальной левый класс смежности по элементу совпадает с правым, (например, транспозиция не образует
нормальную в $S_3$, а цикл длины 3 "--- образует).
Индекс подгруппы: $[G:H]$ (число классов смежности, либо левых, либо правых, они одинаковы),
если $[G:H]=2$, то нормальна.

Ядро гомоморфизма ($f(H=\ker f)=\{1\}$) есть нормальная подгруппа (подгруппа "--- <<в лоб>>,
нормальность по $gHg^{-1} \subseteq H$).
Более того, $gH=Hg=f^{-1}(x)$ для $x=f(g)$ (два включения: $\subseteq$ очевидно, для $\supseteq$
поделили).

\section{} % 22
Пусть $H \unlhd G$, $X$ "--- множество левых смежных классов по $H$,
определяем на $X$ умножение: $(g_1H)\cdot_X(g_2H)=(g_1g_2)H$.
Это корректно (результат не зависит от представителей $g_1$ и $g_2$ в классах
смежности), ассоциативно ($g_1H\cdot_X g_2H = g_1(Hg_2)H=g_1g_2HH=g_1g_2H$,
следует из ассоциативности умножения по Минковскому), нейтральный и обратный есть.
$X$ есть факторгруппа "--- $G / H$ ($H$ нормальна).

\section{} % 23
Можно построить гомоморфизм $G \to G / H$ ($f(g)=gH$, проверить корректность),
его ядро есть $H$ (два включения, трюк с $f(a)=f(b) \iff f(b^{-1}a) = 1H$).
Т.о. нормальные подгруппы есть в точности ядра каких-то гомоморфизмов.

\section{} % 24
Есть гомоморфизм $f \colon G \to G_1$ и $H=\ker f$, тогда $G/H \cong f = \Im G \le G_1$
(то есть поведение образа гомоморфизма задаётся ядром), $f$ может не быть сюръекцией.
Строим изоморфизм $\phi \colon \Im G \to G / H$ естественным образом: $\phi(x)=(f^{-1}(x))H$.
Проверим корректность (несколько прообразов).
Проверим гомоморфность.
Сюръективно, т.к. $\phi(f(g))=gH$.
Инъективно: трюк с $f(b^{-1}a)=1H$.
Замечание: $\phi \circ f$ даёт естественный гомоморфизм $G \to G/H$.
Примеры: $\R/\Z$ с $x\to e^{2\pi ix}$; $S/U_n$, где $S=\{x\mid |x|=1\}$ и $U_n=\{z^n=1\}$

\section{} % 25
Произведение по Минковскому.
Равносильно: $AB \le G$ (произведение есть подгруппа) и $AB=BA$ (последнее не значит, что $A$ и $B$ коммутируют по элементам).
$2 \Ra 1$: замкнуто по умножению ($(AB)(AB)=ABAB=A(BA)B=AABB=AB$),
есть $1=1_A\cdot 1_B$, есть обратные (так как $AB=BA$).
$1 \Ra 2$: два включения (первое: взяли обратные; второе через первое, взяв обратные в множествах).

Тогда если $A \unlhd G$ и $B \le G$, то $AB \le G$ (так как $aB=Ba$, а $AB=\cup_a aB$).

\section{} % 26
Пусть $H \unlhd G$ и $K \le G$.
Тогда автоматически $HK=KH \le G$, и $1H=H1=H\le HK, KH$;
зажали с двух сторон: $H \le HK \le G$.

Теорема: тогда $H \unlhd KH$ и $KH / H \cong K / (K \cap H)$.
Первое понятно (так как $H$ нормальна в $G$, воспользовались сопряжением $gHg^{-1}$).
Построим специальный $\phi \colon K \to KH / H$: $\phi(k_{K})=k_{KH}H_{KH}$ (проверим корректность).
Это сюръекция: взяли элемент $KH/H$, это $khH=k(hH)=kH=\phi(k)$.

Заметим, что $\ker \phi=K \cap H$ ($\subseteq$: взяли $k \in \ker \phi$, тогда
он $\in H$, и по определению $\in K$; $\supseteq$: взяли $k \in K \cap H$,
тогда $\phi(k)=kH=H$), отсюда же следует нормальность $K\cap H$.
Тогда знаем, что $K/(K\cap H) = K/(\ker \phi) \cong \phi(K) = KH / H$.

\section{} % 27
Теорема раз:
пусть $\phi_1 \colon G \twoheadrightarrow G_1$ и
$\phi_2 \colon G \twoheadrightarrow G_2$ (эпиморфизмы),
причём $\ker \phi_1 \subseteq \ker \phi_2$.
Тогда есть $\sigma \colon G_1 \twoheadrightarrow G_2$ такое, что $\sigma \circ \phi_1 = \phi_2$
(т.е. можно образ <<менее сжимающего>> гомоморфизма гомоморфно перевести в образ второго).

Положим $\sigma(g) = \phi_2(\phi_1^{-1}(g))$.
Корректно (независимо от выбора прообраза), так как если у $g$ есть прообразы $x$, $y$,
то $\phi_1(y^{-1}x)=1$, т.е. $y^{-1}x \in \ker \phi_1 \Ra y^{-1}x \in \ker \phi_2$,
т.е. $\phi_2(y^{-1}x)=1$, что и надо.
Нужное свойство есть, сюръективность тогда тоже есть, так как $\phi_2$ сюръективен.
Проверим гомоморфность в лоб.

Теорема два: если $H_1, H_2 \unlhd G$ и $H_1 \subseteq H_2$, то	$H_1$ нормальна в $H_2$,
$H_2 / H_1$ нормальна в $G/H_1$, и $(G/H_1)/(H_2/H_1)\cong G/H_2$ (типа сокращаем).
Док-во: подгруппность и нормальность $H_1$ в $H_2$ очевидна (св-во $ghg^{-1}$),
взяли $\phi_1 \colon G \to G/H_1$ и $\phi \colon G \to G/H_2$ с ядрами $H_1$ и $H_2$,
ядра вложены, применили теорему, нашли $\sigma \colon G/H_1 \to G/H_2$.

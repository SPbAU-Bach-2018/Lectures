\section{} % 14
Для конечномерных: $U^* \otimes_K V \cong \mathcal L(U, V) \cong M(\dim V, \dim U, K)$.
Как обычно, строим полилинейное $\alpha(u^*, v)(x) = u^*(x) \cdot v$ ($U^* \times V \to \mathcal L(U, V)$), как обычно получили единственное
линейное $\beta$ из тензорного в $\mathcal L (U, V)$.
Размерности тензорного и $\mathcal L$ равны, проверим сюръективность $\beta$, получим нужный изоморфизм.
Для этого пусть $e_i$ "--- базис $U$ (и двойственный ему $e_i^*$ базис $U^*$), $\epsilon_i$ "--- базис $V$.
Тогда смотрим значения $\beta$ на тензорном базисе, получили, что матрица $\beta(e_i^*\otimes \epsilon_j)$ есть куча нулей и одна единичка,
такие матрицы образуют базис $\mathcal L (U, V)$, вообще успех.

\section{} % 15
Еще базис $U^*$ обозначают за $e^i$, а преобразования так: $\beta(e^i\otimes e_j)=e^i_j$ (вверху "--- откуда, внизу "--- куда).
Матрица разложимого тензора имеет вид $a\cdot b^\top$, т.е. ранга 1.
Любая матрица такого ранга есть столбец на строку $\Ra$ образ разложимого тензора.
Тензорный ранг "--- наименьшее число разложимых, дающих в сумме данный.
Тензорный ранг матрицы "--- посмотрели на тензорный прообраз (минимальное $r$, что представимо в виде
суммы $r$ матриц ранга 1).

Тензорный не больше обычного: если $\rang A = r$, то есть обратимые $C$ и $D$ такие, что $C^{-1}AD$ есть $r$ единиц на диагонали,
представили $A$ как сумму $C[e^i_i]D^{-1}$, ранг слагаемого $\le 1$, т.к. ранг не больше минимума рангов.
Обычный не больше тензорного: ранг суммы не больше суммы рангов, а если $A=\sum A_i$, то $\rang A \le r$.

\section{} % 16
Есть линейные операторы из/в конечномерные: $A_i \colon V_i \to W_i$,
тогда можно построить линейное $\otimes A_i \colon (\otimes V_i) \to (\otimes W_i)$ такое,
что $(A_1 \otimes A_n)(v_1\otimes v_n)=A_1v_1 \otimes A_n v_n$, это тензорное пр-ие $A_i$,
такое есть и единственно.
Док-во через универсальность: строим полилинейное $\alpha \colon V_1 \times V_n \to W_1 \otimes W_n$ понятно как,
получили линейное $\beta$, что было надо.
Кстати, если изначальные были изоморфизмами, то в результате тоже изоморфизм (потому что нашли семейство образующих).

Теперь ищем $[A_1 \otimes A_n]$ в тензорном базисе.
Рассмотрим для $n=2$, выписываем четыре базиса, получаем блочную матрицу, каждый блок "--- это один элемент $[A_1]$ умноженный
на матрицу $[A_2]$, итого строна матрицы "--- произведение сторон $[A_1]$ и $[A_2]$.

\section{} % 17
Пусть $V$ "--- конечномерное пр-во над $K$, тогда $T_p^q = V^*\otimes V^*\otimes V \otimes V$ ($p$ раз ковариантное $V^*$, $q$ раз контрвариантное $V$; $p+q$ "--- ранг/валентность).
Говорим, что $T_0^0(V)=K$ (скаляры), $T_0^1(V)=V$ (вектора), $T_1^0(V)=V^*$ (ковектора, лин. отображения $V \to K$), $T_1^1(V) \cong \mathcal L (V, V)$ (лин. отображения),
$T_2^0(V) \cong \mathcal L(V, V, K)$ (билинейные формы).

Можно построить алгебру (см. опр. дальше) как прямую сумму пространств $T_p^q(V)$ (все элементы, кроме нуля, разные),
её элементы "--- линейные комбинации тензоров разных типов.
Складывать и умножать на элемент $K$ понятно как.
Теперь научимся умножать две линейные комбинации: просто берём все пары тензоров, конкатенируем их как строки через $\otimes$ ($T_p^q \otimes T_{p'}^{q'} \to T_{p+p'}^{q+q'}$),
приводим подобные, получили алгебру тензоров всех типов.

\section{} % 18
Алгебра "--- это векторное пр-во с умножением векторов (по нему кольцо, умножение билинейно), м.б. ассоциативной;
например, многочлены, матрицы, $\H$, $\C$, октавы (неассоциативная).
Пусть $A$ "--- алгебра, тогда $\mathcal L(A, A, A) \cong A^* \otimes A^* \otimes A$, т.е. тензорам $T_2^1(A)$.
Можно взять базис $A$, разложить входные и выходной вектора по базису, получить задание произведения структурными
константами $\gamma_{ij}^k$ (внизу "--- номера входных базисных; наверху "--- выходного).
Если умножение ассоциативно, то можно расписать условие на $\gamma$ (какая-то сумма равно какой-то для любой четвёрки индексов).
Для коммутативности получаем симметричность $\gamma$.
Еще $\gamma$ можно записать как тензор, если домножить $\gamma$ на $e_i\otimes e_j \otimes e^k$.

\section{} % 19
Выберем в $V$ базис $e_i$, в двойственном выберем базис $e^j$ (причём $e^i(e_j)=int(i \neq j)$), тензор из $T_p^q(V)$ "--- это линейная
комбинация формальных конструкций $e^{i_1}\otimes\dots\otimes e^{i_p}\otimes e_{j_1} \otimes\dots\otimes e_{j_q}$ с коэффициентами $T_{i_1,\dots,i_p}^{j_1,\dots,j_p}$
(тут $i_*$ и $j_*$ "--- индексы в базисах, мы же много пр-в перемножаем, но они почти все одинаковы).
При изменении базиса коэффициенты как-то меняются, получили обобщение вектора.

Пусть изменили базис: $e_j' = \sum(A_j^i)e_i$ (тут $A$ "--- матрица,нижний индекс "--- столбец), а $(e^j)'=\sum (B_i^j) e^i$ ($B$ "--- тоже матрица, нижний индекс "--- столбец),
тогда подстановкой проверяем, что $AB=1$.
Преобразование тензоров: расписываем, во что переходит тензорное произведение новых базисных в терминах старых, раскрываем по линейности,
получаем ответ: новый коэффициент есть сумма по двум переменным $x$, $y$: $T_{x}^y B_{i}^x A_{y}^{j}$.

\section{} % 20
Хотим построить линейное $T_p^q(V) \to T_{p-1}^{q-1}(V)$ (для $p,q\ge 1$) в классическом виде.
Если был тензор $T$, то его свёртка $\tilde T$ по паре индексов $i_k, j_l$ есть: задаём очередной коэффициент как сумму из $n=\dim V$ коэффициентов старого,
$i$-е слагаемое "--- заменили в исходном коэфф. индексы $i_k$ и $j_l$ на просто $i$ (у нас соотв. вектора в базисах хорошо относятся).

Если $T \in T_1^1$, то свертка единственна: $\tilde T = \sum_i T_i^i$ (типа след матрицы преобразования).

Если $T, S \in T_1^1$ (как бы квадратные матрицы), то можно перемножить тензоры (конкатенацией), получить $T_2^2$, а потом свёртка по $j$ и $k$ "--- это коэффициент
произведения матриц в строке $j$, столбце $k$.

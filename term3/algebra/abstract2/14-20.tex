\section{} % 14
Для конечномерных: $U^* \otimes_K V \cong \mathcal L(U, V) \cong M(\dim V, \dim U, K)$.
Как обычно, строим полилинейное $\alpha(u^*, v)(x) = u^*(x) \cdot v$ ($U^* \times V \to \mathcal L(U, V)$), как обычно получили единственное
линейное $\beta$ из тензорного в $\mathcal L (U, V)$.
Размерности тензорного и $\mathcal L$ равны, проверим сюръективность $\beta$, получим нужный изоморфизм.
Для этого пусть $e_i$ "--- базис $U$ (и двойственный ему $e_i^*$ базис $U^*$), $\epsilon_i$ "--- базис $V$.
Тогда смотрим значения $\beta$ на тензорном базисе, получили, что матрица $\beta(e_i^*\otimes \epsilon_j)$ есть куча нулей и одна единичка,
такие матрицы образуют базис $\mathcal (U, V)$, вообще успех.

Еще базис $U^*$ обозначают за $e^i$, а преобразования так: $\beta(e^i\otimes e_j)=e^i_j$ (вверху "--- откуда, внизу "--- куда).
Матрица разложимого тензора имеет вид $a\cdot b^\top$, т.е. ранга 1.
Любая матрица такого ранга есть столбец на строку $\Ra$ образ разложимого тензора.
Тензорный ранг "--- наименьшее число разложимых, дающих в сумме данный.
Тензорный ранг матрицы "--- посмотрели на тензорный прообраз (минимальное $r$, что представимо в виде
суммы $r$ матриц ранга 1).

Тензорный не больше обычного: если $\rang A = r$, то есть обратимые $C$ и $D$ такие, что $C^{-1}AD$ есть $r$ единиц на диагонали,
представили $A$ как сумму $C[e^i_i]D^{-1}$, ранг слагаемого $\le 1$, т.к. ранг не больше минимума рангов.
Обычный не больше тензорного: ранг суммы не больше суммы рангов, а если $A=\sum A_i$, то $\rang A \le r$.

\section{} % 15

\section{} % 16

\section{} % 17

\section{} % 18

\section{} % 19

\section{} % 20

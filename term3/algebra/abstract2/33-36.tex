\section{} % 33
Берём непустой алфавит $A$, строим алфавит из букв с чёрточками ($\bar A$, отличаются от $A$),
берём строки из $A\cup \bar A$, умножение "--- конкатенация (ассоциативно).
Нейтральный элемент (пустая строка) "--- $\Lambda$.
Слова эквивалентны, если из одного можно получить другое вставкой/удалением $x\bar x$ в произвольных местах.
Это отношение эквивалентности.
$W/\sim$ "--- мн-во классов, умножение классов корректно (не зависит от представителей),
есть нейтральный ($\Lambda$), ассоциативно, есть нейтральный, есть обратный (перевернули, дорисовали чёрточки).
Группа $W/\sim$ обозначается $F_A$ (свободная группа).
Пример: $F_{\{a\}} \cong (\Z, +)$ (циклическая группа).

\section{} % 34
Слово несократимо, если нет $a\bar a$ или $\bar a a$.
В каждом классе $F_A$ есть ровно один несократимый.
Существует: взяли слово минимальной длины, оно несократимо.
Единственность: взяли два разных нескоратимых, найдём цепочку преобразований с минимальной суммарной длиной промежуточных ($L$).
Посмотрели в ней на слово максимальной длины, перед ним "--- вставка, после "--- удаление.
Если операции не пересекаются как отрезки, то можно их переставить, уменьшим $L$ на четыре.
Если пересекаются, то обязательно пересекаются по букве, тогда либо вставили и удалили одно и то же,
либо $a\bar a a$, в любом случае можно на эти две операции забить.

Следствие: $ab\neq ba$.
Если $A \subseteq B$, то $F_a \subseteq F_b$.
$F_{a, b}$ содержит подгруппу, изоморфную $F_{a,b,c}$: \TODO
Отсюда сразу следует, что есть подгруппа, изоморфная любой $F_{x_1, \dots, x_n}$: \TODO

\section{} % 35
Если есть группа с конечным числом образующих $x_1, \dots, x_n$ (возможно, с дополнительными соотношениями),
то есть эпиморфизм из $F_{a_1, \dots, a_n}$ на $G$, переводящий $a_i \to x_i$ (а можно сделать фактор по его ядру).
Строим естественным образом: выполняем операции в строке в группе $G$.
Это гомоморфизм, сюръекция, надо проверить корректность (выбор представителя в классе $W/\sim$).
\TODO и всё?

\section{} % 36

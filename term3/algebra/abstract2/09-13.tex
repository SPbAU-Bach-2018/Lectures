\section{} % 09
$F(S)$ "--- пр-во любых функций из конечного $S$ в поле $K$.
Можно отождествить $F(S_1 \times \dots \times S_n)$ и $F(S_1) \otimes \dots \otimes F(S_n)$.
Берём базис из $\delta$-функций (единица в ровно одной точке) для каждого $F(S_i)$, размерности слева и справа совпали.
Изоморфизм сопоставит $\delta_{x_1, \dots, x_n} \lra \delta_{x_1} \otimes \dots \otimes \delta_{x_n}$,
остальное по аддитивности, получим что надо.
Разложимые тензоры отвечают функциями вида $f_1(x_1) \cdot f_2(x_2)$ (но не $f_1(x_1)\cdot f_2(x_2) + f_3(x_2)$).

\section{} % 10
Как комплексификация: было поле $K$, в него вложено $L$.
И еще было $V_K$ над $K$.
Хотим очень естественно расширить его до $V_L$ над $L$
(чтобы в нём было подмножество, изоморфное $V_K$: если смотреть на
вектора, у которых все координаты из $K$, то они ведут себя как в $V_K$;
а также хорошо себя ведут все такие линейные операторы).

$L$ "--- векторное пр-во над $K$ (проверить, базис не умеем), положим $V_L=L \otimes V_K \supseteq K \otimes V_K \cong V_K$ (последнее из базисов).
Покажем, что $V_L$ есть векторное пр-во над $L$, складывать умеем, умножаем на скаляр $c$ внутри $M$: $c(l\otimes v)=(cl)\otimes v$, остальные по билинейности,
все вектора из $M_0$ этим умножением остаются в $M_0$, тогда можем индуцировать умножение на скаляр $c$ в $M/M_0=V_L$, оно корректно, надо
проверить св-ва умножения в векторном поле (две дистриб., $\cdot 1$ и ассоц.), а также что если умножать на $K$, то всё как в $V_K$.

Базис у $V_L$ над $L$: $1_L \otimes e_i$, это семейство образующих, причём линейно независимых (расписать от противного,
получить что тензорный базис $V_L$ над $K$ линейно зависим, упс).

Если есть оператор $A$ над $V_K$, то построим его над $V_L$, чтобы на сужении вёл себя как раньше:
$A(l \otimes v) = l \otimes A(v)$, распространяем по линейности, доказывать независимость от базиса мы забили.

\section{} % 11
Определяли $V_1 \otimes \dots \otimes V_n$,
хотим показать, что оно изоморфно $(V_1 \otimes V_2) \otimes (\dots)$.
Тогда вместе с коммутативностью можно будет раскрывать скобки как угодно.
Док-во для $n=3$ (для больших надо, т.к. у нас произведение от $n$ аргументов, а не двух,
но мы не доказываем, аналогично): берём $V_1 \times V_2 \times V_3$, строим полилинейное в
векторное пр-во $(V_1 \otimes V_2) \otimes V_3$ ясно как, тогда есть единственный линейный гомоморфизм
между ним им $V_1 \otimes V_2 \otimes V_3$.
Выбрали базисы и там, и там, базис перешёл в базис $\iff$ изоморфизм.

\section{} % 12
Есть изоморфизм между тензорным произведением ($A$) и им же с переставленными множетелями ($B$).
Как обычно: построили полилинейное из декартова произведение в $B$, получили единственный линейный
гомоморфизм $\beta$, посмотрели на тензорный базис в $A$, оно переводится им в тензорный базис в $B$,
значит изоморфизм.

\section{} % 13
Для конечномерных: $A=V_1^* \otimes \dots \otimes V_n^* \cong (V_1 \otimes V_n)^*=B$.
Второе изоморфно полилинейным из $V_i$ в $K$ (назовём $C$).
Строим полилинейное $\alpha$ из $D=V_1^* \times \dots \times V_n^*$ в $C$ на кортежах базисных отображений понятно как.
Получаем, что есть единственное линейное из $A$ в $C$ ($\beta$).
Это что надо, если изоморфизм.
Показали, что размерность $A$ и $C$ совпадают, показали сюръективность $\beta$ (любое полилинейное
представляется в каком надо виде), успех.

\section{} % 28
Прямое произведение групп $H_1, \dots, H_n$ есть группа на носителе $G=H_1 \times \dots \times H_n$
(декартово), операции почленные (обозначается как декартово: $\times$).
Есть инъективный гомоморфизм $\sigma \colon H_i \to G$ (остальные элементы "--- единицы).
Можно ввести множество $\widetilde{H_i}=\sigma(H_i)$.

При $i \neq j$ мн-ва $\widetilde{H_i}$ и $\widetilde{H_j}$ коммутируют поэлементно ($ab=ba$),
и как множества: $\widetilde{H_i}\widetilde{H_j}=\widetilde{H_j}\widetilde{H_i}$.
Также $\widetilde{H_i} \unlhd G$ (через $ghg^{-1}$).
Также $\widetilde{H_i} \cap (\widetilde{H_1}\cdot \dots \cdot \overbar{\widetilde{H_i}} \cdot \dots \cdot \widetilde{H_n})=\{1_G\}$.

Упражнение на практику: если есть $H_i$ с этими тремя свойствами, дающие в произведении
$G$, то $G \cong H_1 \times \dots \times H_n$.
Использование: доказательство КТО: $(\Z/6\Z)^+ \cong (\Z/2\Z)^+ \times (\Z/3\Z)^+$.

\section{} % 29
Пусть $G$ "--- группа, $X$ "--- множество, $G$ действует на $X$ операцией $\cdot$,
если $1\cdot x = x$ и $g_1\cdot(g_2 \cdot x)=(g_1g_2)\cdot x$ (тут $\cdot$ "--- левый
$G$-оператор мн-ва $X$).

Можно зафиксировать $g\in G$: $\phi_g(x) = g \cdot x$.
Тогда $\phi_g \circ \phi_h = \phi_{gh}$ (очевидно) и $\phi_g$ "--- биекция
(так как есть обратное).

Обозначим $Sym(X)$ как мн-во биекций $X \to X$, это группа относительно композиций ($S_{|X|}$).
Тогда $\phi_g \in Sym(X)$.
Получили гомоморфизм $\Phi \colon G \to Sym(X)$ (очевидно).
Если возьмём любой такой гомоморфизм, получим $G$-оператор
(проверить два свойства).
Т.о. действие элемента группы на множестве "--- это некоторая перестановка элементов $X$,
композиция элементов "--- произведение перестановок.

Действие точное, если: $g$ оставляет элементы на месте $\iff$ $g=1$.
Точность $\iff \ker \Phi =\{1\}$.
В частности, для точных (по т. о гомоморфизме): $\Phi(G) \cong G$.
Действие транзитивное, если любой элемент $X$ можно перевести в любой некоторым элементом $G$.

Если $a \in X$, то орбита $a$ есть $Ga$.
Второе определение: вводим $x \sim y \iff (\exists g \colon gx=y)$ (оно
действительно отношение эквивалентности), орбита "--- класс по этому отношению
(элемент $X/G \eqDef X/\sim$); показать эквивалентность определений.
Можно представить $X$ как объединение орбит из $X/G$.
Можно сужать $\cdot$ на орбиту $A$, получим действие (проверить св-ва),
причём транзитивное, что удобно.

Стабилизатор элемента $a$ ($\St a$) "--- мн-во эл-тов $G$, оставляющих $a$ на месте.
$\St a$ "--- подгруппа (есть 1, замкнуто по умножению, есть обратные).

\section{} % 30
Говорим, что $X$ изоморфно $X'$ как $G$-операторное мн-во, если
есть биекция $f$ и $f(gx)=_{X'}gf(x)$.
Теорема: орбита $Ga$ изоморфна мн-ву классов смежности $\{ g\St_a \mid g \in G \}$.
Док-во: переводим $ga \lra g \St_a$, показываем корректность (элемент с каждой стороны можно получить
разными $g$; трюк с $g_2^{-1}g_1 \in \St_a$), инъективность (в лоб, воспользовались биективностью
действия группы), сюръективность (очевидно), гомоморфность (очевидно).

Orbit Stabilier Theorem: если $G$ конечна и действует на $X$, то для любого $a$:
$|Ga| \cdot |\St_a| = |G|$.
Док-во: $|Ga|$ есть число левых смежных классов на $\St_a$, т.е. индекс подгруппы $\St_a$,
применили теорему Лагранжа про произведение индекса и числа элементов.
Для бесконечных $|Ga|=[G:\St_a]$ (без док-ва?).
Следствие: $|\St_a|$ и $|Ga|$ делят $|G|$.

\section{} % 31
$Z(G)$ "--- центр группы; элементы, коммутирующие вообще со всеми из $G$ (например, 1).
$Z(G) \unlhd G$, так как $gag^{-1}=agg^{-1}=a$.
Если конечная группа имеет порядок $p^k$ ($p$ простое, $k\in \N$), то она "--- $p$-группа.
Примеры: $\Z/7\Z$ (циклическая), $\Z/49\Z \times \Z/7\Z$ (не циклическая, но абелева),
верхнетреугольные матрицы над конечным полем $\Z/p\Z$ с единицами на диагонали (не абелева).

Теорема: в $p$-группе $|Z(G)|>1$.
Док-во: вводим действие $G\times G \to G$: $g \cdot x \to gxg^{-1}$.
Условие на принадлежность центру: $|Gx|=1 \iff gxg^{-1} = x \iff gx = xg \iff x \in Z(G)$.
Считаем $|G|$ суммированием по порядкам орбит (aka классы сопряжённости).
Орбита "--- подгруппа, её порядок делит $|G|$, т.е. сумма
порядков всех орбит размера $>1$ делит $|G|$, считаем остаток (равный $|Z(G)|$), он хотя
бы 1 и делится на $p$.

\section{} % 32
Обозначим за $X^g$ мн-во неподвижных точек $X$ относительно элемента группы $g$.
Лемма Бернсайда: $G$ и $X$ конечны, тогда число орбит ($|X/G|$) в $|G|$ раз меньше $\sum_{g \in G} X^g$
Док-во: считаем сначала по определению $X^g$, поменять суммы местами,
получили $\sum_{a\in X} \St_a=\sum \frac{|G|}{|Ga|}$, распилили $X$ на орбиты,
сократили $|Ga|$ и мощность орбиты, получили что надо.

Пример: число ожерелий длины $n=6$ из $k=3$ цветов (можно поворачивать и отражать).
$X$ "--- расстановки бусин ($|X|=k^n$), $G$ "--- группа поворотов/отражений, $X/G$ "--- различные
расстановки.
Считаем мощности мн-ва неподвижных точек для разных элементов группы, лемма Бернсайда,
получаем 92.

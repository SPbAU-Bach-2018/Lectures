\section{} % 2.12
\t{Functor} "--- контейнер, в нём есть \t{fmap} (он же \t{<\$>}), выводятся \t{(<\$)} (заменить всё на константу) и флипнутый
\t{(\$>)}, а также \t{void} (меняет всё на void).
Представители: списки, \t{Either e}, \t{(,) s}, \t{(->) e} (с композицией).
Не меняет структуру (законы: \t{fmap id} и композиция).
\t{Pointed} (в Haskell нет, надо явно писать) "--- добавили \t{pure} (вложили элемент в контейнер), закон: \t{fmap g . pure = pure . g}.

\t{Applicative} "--- от \t{Pointed}, добавился \t{<*>}, применяющий функции из левого аргумента к аргументам из правого.
Пример: \t{Maybe}.
Законы: \t{fmap g xs = pure g <*> xs}; \t{pure id}, композиция ф-ций внутри контейнера и поочерёдное применение;
\t{pure} к функции и к элементу; \t{pure (\$ x) <*> g}.
Списки как аппликативы: либо <<все со всеми>> (по умолчанию), либо делаем \t{ZipList} (у него надо, чтобы \t{pure} возвращал бесконечный список)/
Пара как аппликатив ок, если первый элемент "--- моноид.
В аппликативах еще есть \t{liftA2 :: (a -> b -> c) -> f a -> f b -> f c}, и \t{liftA3}.

\section{} % 2.13
Есть нейтральный \t{mempty}, есть ассоциативная операция \t{mappend} (два св-ва).
Можно выразить \t{mconcat} через свёртку, целые числа "--- моноид и по сложению,
и по умножению (создаём \t{newtype Sum a = Sum \{ getSum :: a\}} и выбираем),
можно взять bool по конъюнкции/дизъюнкции.

\section{} % 2.14
Foldable (минимальное определение либо \t{foldMap}, либо \t{foldr}), есть две свёртки и симметричные
\t{foldMap} и \t{fold}, работающие с результатом-моноидом.
Traversable "--- Functor и Foldable, есть \t{sequenceA :: Applicative f => t (f a) -> f (t a)} и \t{traverse g = sequenceA . fmap g} (друг через друга выражаются).
\t{sequenceA} обходит Traversable и вытаскивает наружу аппликатив (было дерево, а в каждом элементе "--- \t{ZipList} длина 3, стал один \t{ZipList} из трёх деревьев с числами).
Представители: \t{Maybe}, списки, реализация похожа на Functor, только добавились \t{pure}, \t{<\$>} и \t{<*>} между вызовами.

\section{} % 2.15

\section{} % 2.16

\section{} % 2.17

\section{} % 2.18

\section{} % 2.19

\section{} % 2.20

\section{} % 2.21

\section{} % 2.22

\section{} % 2.23

\section{} % 2.01
\t{=} "--- иммутабельное связывание (не присваивание), \t{foo = let z = x + y in print z}
(может задавать и функцию, и локальное связывание),
первый символ идентификатора маленький, в GHCI для глобального связывания "--- \t{let},
есть лямбды \t{\\y -> x + y}, ассоциативность как в лямбдах (иногда нужны скобки),
порядок объявления в файле неважен, Haskell ленив.
Guard'ы: \t{factorial n | n == 1 = 1} (смотрят сверху вниз до первого \t{true}).
Можно функции задавать в несколько строк с pattern matching.
Пример с факториалом: наивный и с хвостовой рекурсией (аккумулятор), можно использовать \t{where} и \t{let}
(первая может быть общая для всех guard'ов внутри одной строчки объявления).

\section{} % 2.02
Булы, символ юникода, \t{Int} фиксированного размера, \t{Integer} произвольного размера,
функции (\t{->}), кортежи, void \t{()}, список с элементами одного типа.
Можно писать \t{42::Integer} (а то тип полиморфен у констант).
Конструктор типа \t{Bool}, конструкторы данных "--- \t{True}, \t{False}, все с большой буквы.

Каррирование всегда есть: \t{mult 2} частично применена.
Параметрический полиморфизм: \t{k :: a -> b -> a} (могут входить не конкретные типы,
а переменные), должно работать для произвольных переменных типа.

Программа состоит из набора модулей, управляют пространством имён, есть списки экспорта/импорта:
\t{module A (foo, bar) where} (экспорт), \t{import B (f,g)} (импорт), конфликты можно разрешать
при помощи \t{import qualified} и пользоваться \t{B.f}.

\section{} % 2.03
Оператор "--- комбинация из нескольких спецсимволов, все бинарные и инфиксные (кроме унарного минуса,
его не переопределить).
\t{a +++ b = a + b} "--- нормальный стиль, \t{(+++) = +} "--- функциональный (префиксный) стиль,
да еще и бесточечный (какие-то последние аргументы опущены, всё через функции высших порядков).
Функции можно использовать/определять в инфиксном: \t{a `plus` b}.
Можно задавать приоритет и ассоциатвиность операторов: \t{infix}, \t{infixl}, \t{infixr}
(например, \t{infixl 6 +++} даст плюс).
Сечение "--- частичное применение оператора \t{(2+) == (+) 2} или \t{(+2)}, скобки обязательны.
Оператор \t{\$} "--- аппликация с наименьшим возможным приоритетом.

\section{} % 2.04

\section{} % 2.05

\section{} % 2.06

\section{} % 2.07

\section{} % 2.08

\section{} % 2.09

\section{} % 2.10

\section{} % 2.11

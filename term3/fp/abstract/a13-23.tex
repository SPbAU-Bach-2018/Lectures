\section{} % 1.13
Предтермы "--- переменные, аппликации, абстрации, в точности термы бестипового исчисления.
Утверждение о типизации: $M \colon \tau$.
Объявление "--- утверждение о типизации термовой переменной.
Контекст (базис, окружение), "--- мн-во объявлений различных переменных, контекст можно расширять,
добавляя новую переменную, еще можно смотреть на него как на частичную ф-цию $V \to \mathcal T$.
Утверждение $M\colon \tau$ выводимо в $\Gamma$ ($\Gamma \vdash M \colon \tau$), если можно вывести
по правилам: взять переменную из контекста, взять функцию и переменную из контекста, взять абстракцию по переменной из контекста.
Если есть $\Gamma$ и $\tau$ такие, что $M$ типизируется, то $M$ "--- допустимый терм.

\section{} % 1.14
Предтермы "--- переменные, аппликации, абстракции с указанием типа переменной.
Остальное такое же.
Только выводимость меняется: надо при взятии абстракции указывать её тип (причём этот тип вообще произвольный).

\section{} % 1.15
Тип "--- чисто синтаксическая конструкция $M \colon \sigma$,
дают частичную спецификацию, ловят простые ошибки.
Система Карри (неявная типизация): термы те же, но каждому можно приписать некоторое множество типов (пустое, конечное, бесконечное).
Система Чёрча (явная типизация): каждый тёрм аннотирован, каждый тип имеет уникальный тип, выводимый из аннотации.

Множество типов: переменные типа $\alpha$, $\beta$, \dots, стрелочные типы $(\sigma \to \tau) \in \mathcal T$
(берём начальные буквы для переменных, буквы из конца "--- для произвольных типов).
Ассоциативность: $\sigma_1 \to (\sigma_2 \to \sigma_3)$.
Типизация: переменные как угодно, аппликацию естественно, абстракцию как стрелочный.
Стиль Карри: не указываем в лямбдах тип, стиль Чёрча "--- указываем ($\lambda x \colon \alpha . x) \colon \alpha \to \alpha$).

Лемма об инверсии (лемма генерации): если в $\Gamma$ выводима переменная, аппликация или абстракция, то естественные <<предшественники>>
тоже выводимы (в абстракции в контекст добавляется тип, у Карри "--- под квантором, у Чёрча "--- берётся из терма).
Лемма о типизируемости подтерма: если $\Gamma \vdash M \colon \sigma$, то любой подтерм тоже типизируется в каком-то контексте.
Лемма разбавления: если $\Gamma \subseteq \Delta$ и $\Gamma \vdash M \colon \tau$, то из $\Delta$ тоже.
Лемма о свободных переменных: они должны лежать в контексте для типизации.
Лемма сужения: если выкинем всё, кроме свободных переменных, всё еще типизируем.
Предтерм $x\,x$ нетипизируем, так как типы "--- конечные строки.

В типах тоже можно делать подстановку.
Лемма подстановки типа: если $\Gamma \vdash M \colon \tau$, то можно одновременно заменить типовую переменную $\alpha$ на тип $\tau$ в контексте и в
утверждении, всё ок (для Чёрча надо еще в самом $M$ заменить).
Лемма подстановки тёрма: если подходящая по типу подстановка терма сохраняет тип.
Теорема о редукции субъекта: тип сохраняется при $\beta$-редукциях/

Тёрм в системе Чёрча имеет единственный тип (в частности, любые $\beta$-эквивалентные имеют одинаковый единственный тип).
А у Карри единственности типа нет (возможно добавление ненужных деталей в тип).

\section{} % 1.16
Из Чёрча легко отобразить в Карри: стираем типы.
Из Карри можно поднимать в Чёрча: надо подобрать какие-то типы для абстракций.
Типы обитаемы в обеих системах одновременно.
Задача проверки типа (соответствует ли терм ему), синтеза типа (построить тип по терму), обитаемости типа
(есть ли терм с типом), для Чёрча и Карри всё разрешимо.
ЗПТ выглядит проще ЗСТ, но в аппликации требует сгенерировать какой-нибудь тип для аргумента, поэтому эквивалентны.

Терм слабо нормализуем (WN), если есть последовательность редукций, приводящая к нормальной ($K\,I\,\Omega$).
Сильно нормализуем, если любая приводит к нормальной ($K\,I\,K$).
Система слабо (сильно) нормализуема если все допустимые термы слабо (сильно) нормализуемы.
Обе системы сильно нормализуемы.

\section{} % 1.17

\section{} % 1.18

\section{} % 1.19

\section{} % 1.20

\section{} % 1.21

\section{} % 1.22

\section{} % 1.23
